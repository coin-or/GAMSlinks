\chapter{\CBC}
\label{cha:coincbc}
\hypertarget{cha:coincbc}{}

\CBC (COIN-OR Branch and Cut) is an open-source mixed integer programming solver working with the COIN-OR LP solver \textsc{CLP} and the COIN-OR Cut generator library \textsc{Cgl}.
The code has been written primarily by John J. Forrest.

For more information we refer to the website of \CBC, \textsc{CGL}, and \textsc{CLP}:
\url{https://projects.coin-or.org/Cbc}, \url{https://projects.coin-or.org/Cgl}, \url{https://projects.coin-or.org/Clp}.
Most of the \CBC documentation in the section was copied from the help in the \CBC standalone version.

The \CBC link in \GAMS supports continuous, binary, integer, semicontinuous, semiinteger variables, special ordered sets of type 1 and 2, and branching priorities (see chapter 17.1 of the \GAMS User's Guide).
% Quadratic objective functions are not supported yet.

\section{Usage}

The following statement can be used inside your \GAMS program to specify using \CBC
\begin{verbatim}
  Option LP = CBC;     { or MIP or RMIP }
\end{verbatim}

The above statement should appear before the Solve statement.
If \CBC was specified as the default solver during \GAMS installation, the above statement is not necessary.

\subsection{Specification of Options}

The \GAMS/\CBC options file consists of one option or comment per line.
An asterisk (*) at the beginning of a line causes the entire line to be ignored.
Otherwise, the line will be interpreted as an option name and value separated by any amount of white space (blanks or tabs).
Following is an example options file \texttt{cbc.opt}.
\begin{verbatim}
  cuts root
  perturbation off
\end{verbatim}
It will cause \CBC to use cut generators only in the root node and turns off the perturbation of the LP relaxation.

\GAMS/\CBC currently does not support the \GAMS Branch-and-Cut-and-Heuristic (BCH) Facility.
If you need to use \GAMS/\CBC with BCH, please consider to use a \GAMS system of version $\leq 23.3$, available at \url{http://www.gams.com/download/download_old.htm}.

% \GAMS/\CBC supports the \GAMS Branch-and-Cut-and-Heuristic (BCH) Facility.
% The \GAMS BCH facility automates all major steps necessary to define, execute, and control the use of user defined routines within the framework of general purpose MIP codes.
% Currently supported are user defined cut generators and heuristics.
% Please see the BCH documentation at \texttt{http://www.gams.com/docs/bch.htm} for further information.

\section{Options}

Among all \CBC options, the following \GAMS parameters are currently supported in \CBC:
\hyperlink{reslim}{reslim}, \hyperlink{iterlim}{iterlim}, \hyperlink{nodlim}{nodlim}, \hyperlink{optca}{optca}, \hyperlink{optcr}{optcr}, \hyperlink{increment}{cheat}, \hyperlink{cutoff}{cutoff}, \hyperlink{threads}{threads} (experimental on Windows systems).

There are many parameters which can affect the performance the \CBC{}s Branch and Cut Algorithm.
First just try with default settings and look carefully at the log file.
Did cuts help? Did they take too long? Look at the output to see which cuts were effective and then do some tuning (see the option \hyperlink{cuts}{cuts}).
If the \hyperlink{preprocess}{preprocessing} reduced the size of the problem or strengthened many coefficients then it is probably wise to leave it on.
Switch off \hyperlink{heuristics}{heuristics} which did not provide solutions.
The other major area to look at is the search.
Hopefully good solutions were obtained fairly early in the search so the important point is to select the best variable to branch on.
See whether strong branching did a good job -- or did it just take a lot of iterations?
Adjust the options \hyperlink{strongbranching}{strongbranching} and \hyperlink{trustpseudocosts}{trustpseudocosts}.

In the following, we summarize all available \CBC options.
\subsection{Summary of CoinCbc Options}

Among all Cbc options, the following GAMS parameters are currently supported in CoinCbc:
\hyperlink{reslim}{reslim}, \hyperlink{iterlim}{iterlim}, \hyperlink{nodlim}{nodlim}, \hyperlink{optca}{optca}, \hyperlink{optcr}{optcr}, \hyperlink{cutoff}{cutoff}.

Currently CoinCbc understands the following options:
\begin{tabbing}
\hspace {1.1in} \= \\
\hyperlink{roundingheuristic}
{roundingheuristic} \> rounding heuristic \\
\hyperlink{localsearch}
{localsearch} \> local search \\
\hyperlink{strongbranching}
{strongbranching} \> strong branching \\
\hyperlink{integerpresolve}
{integerpresolve} \> integer presolve \\
\hyperlink{findsos}
{findsos} \> find sos in integer presolve \\
\hyperlink{cuts}
{cuts} \> switch for cut generation \\
\hyperlink{probing}
{probing} \> probing \\
\hyperlink{gomorycuts}
{gomorycuts} \> Gomory cuts \\
\hyperlink{knapsackcuts}
{knapsackcuts} \> knapsack cover cuts \\
\hyperlink{oddholecuts}
{oddholecuts} \> odd hole cuts \\
\hyperlink{cliquecuts}
{cliquecuts} \> clique cuts \\
\hyperlink{flowcovercuts}
{flowcovercuts} \> flow cover cuts \\
\hyperlink{mircuts}
{mircuts} \> mixed integer rounding cuts \\
\hyperlink{redsplitcuts}
{redsplitcuts} \> reduce and split cuts \\
\hyperlink{cutsonlyatroot}
{cutsonlyatroot} \> whether cuts are only generated at the root node \\
\hyperlink{startalg}
{startalg} \> LP solver for root node \\
\hyperlink{writemps}
{writemps} \> create MPS file for problem \\
\hyperlink{tol_dual}
{tol\_dual} \> dual feasibility tolerance \\
\hyperlink{tol_primal}
{tol\_primal} \> primal feasibility tolerance \\
\hyperlink{tol_integer}
{tol\_integer} \> tolerance for integrality \\
\hyperlink{scaling}
{scaling} \> switch for scaling of LP \\
\hyperlink{presolve}
{scaling} \> switch for initial presolve of LP \\
\hyperlink{printfrequency}
{printfrequency} \> print frequency \\
\hyperlink{nodecompare}
{nodecompare} \> comparision method to determine tree search order \\
\hyperlink{reslim}
{reslim} \> resource limit \\
\hyperlink{iterlim}
{iterlim} \> iteration limit \\
\hyperlink{nodelim}
{nodelim} \> node limit \\
\hyperlink{nodlim}
{nodlim} \> node limit \\
\hyperlink{optca}
{optca} \> absolute stopping tolerance \\
\hyperlink{optcr}
{optcr} \> relative stopping tolerance \\
\hyperlink{cutoff}
{cutoff} \> cutoff for objective function value \\
\end{tabbing}



In the following, we give a detailed description of all available \CBC options.
\subsection{Detailed Descriptions of CoinCbc Options}

\begin{description}

\item[\label{roundingheuristic}\hypertarget{roundingheuristic}
{\textbf{roundingheuristic (\slshape{integer})}}]\hspace{1.0in}

A simple rounding heuristic to find feasible solutions of the MIP.

\textsl{(default = 1)}

\item[\label{localsearch}\hypertarget{localsearch}
{\textbf{localsearch (\slshape{integer})}}]\hspace{1.0in}

A local search heuristic to find feasible solutions of the MIP.

\textsl{(default = 1)}
\begin{itemize}
\item[0] local search heuristic is not used
\item[1] local search heuristic is used
\end{itemize}

\item[\label{strongbranching}\hypertarget{strongbranching}
{\textbf{strongbranching (\slshape{integer})}}]\hspace{1.0in}

If strong branching is switched on, then the maximum number of candidates to be evaluated for strong branching is set to 10 (if the problem has less then 5000 variables) or 5 (if the problem has more then 5000 variables).
If strong branching is switched off, the maximum number is set to 0. (However, Cbc reports still some (but less) strong branching operations.)

\textsl{(default = 1)}
\begin{itemize}
\item[0] strong branching is switched off
\item[1] strong branching is switched on
\end{itemize}

\item[\label{integerpresolve}\hypertarget{integerpresolve}
{\textbf{integerpresolve (\slshape{integer})}}]\hspace{1.0in}

If switched on, a preprocessing on the MIP will be performed.
The integer presolve also does probing, independent of the setting of the 'probing' parameter.\\
From the CglPreprocess description:
This method uses other cut generators to strengthen cuts, establish that some cuts are redundant, fix variables and find relationships such as $x + y = 1$.
While cuts can be added at any time in the tree, some cuts are actually just stronger versions of existing ones.
They can thus replace the existing cuts rather than being added as new cuts. This is awkward in the tree but reasonable at the root node.

\textsl{(default = 1)}
\begin{itemize}
\item[0] don't do integer presolve
\item[1] do integer presolve
\end{itemize}

\item[\label{findsos}\hypertarget{findsos}
{\textbf{findsos (\slshape{integer})}}]\hspace{1.0in}

This option switches the identification for special ordered sets of type 1 in the integer preprocessing on.
Cbc then searches for rows with upper bound 1 and where all nonzero coefficients are 1.

\textsl{(default = 1)}
\begin{itemize}
\item[0] don't try to find special ordered sets
\item[1] try to find special ordered sets
\end{itemize}

\item[\label{cuts}\hypertarget{cuts}
{\textbf{cuts (\slshape{integer})}}]\hspace{1.0in}

Regulates the use of cut generator routines in Cbc.\\
If set to automatic, then the parameters for the single cut generators determine which are added.

\textsl{(default = 0)}
\begin{itemize}
\item[-1] no cuts will be generated
\item[0] automatic
\item[1] cuts from all available cut classes will be generated
\end{itemize}

\item[\label{probing}\hypertarget{probing}
{\textbf{probing (\slshape{integer})}}]\hspace{1.0in}

This option switches probing on as cut generator.\\
From the CglProbing description:
For selected integer variables (e.g. unsatisfied ones) the effect of setting them up or down is investigated.
Setting a variable up may in turn set other variables (continuous as well as integer).

\textsl{(default = 1)}
\begin{itemize}
\item[0] don't do probing
\item[1] do probing
\end{itemize}

\item[\label{gomorycuts}\hypertarget{gomorycuts}
{\textbf{gomorycuts (\slshape{integer})}}]\hspace{1.0in}

This options switches the generation of mixed integer Gomory Cuts on.

\textsl{(default = 1)}
\begin{itemize}
\item[0] don't add gomory cuts
\item[1] add gomory cuts
\end{itemize}

\item[\label{knapsackcuts}\hypertarget{knapsackcuts}
{\textbf{knapsackcuts (\slshape{integer})}}]\hspace{1.0in}

This options switches the generation of knapsack cover cuts on.\\
From the CglKnapsackCover description:
CglKnapsackCover looks for a series of different types of minimal covers.
If a minimal cover is found, it lifts the associated minimal cover inequality and adds the lifted cut to the cut set.

\textsl{(default = 1)}
\begin{itemize}
\item[0] don't add knapsack cover cuts
\item[1] add knapsack cover cuts
\end{itemize}

\item[\label{oddholecuts}\hypertarget{oddholecuts}
{\textbf{oddholecuts (\slshape{integer})}}]\hspace{1.0in}

This options switches the generation of odd hole cuts on.\\
From the CglOddHole description:
This looks at all rows of type $\sum_i x_i \leq 1 \textrm{(or = 1)}$ (with $x$ binary) and sees if there is an odd cycle cut.
This is then lifted by using the corresponding Chvatal cut, i.e. by summing up all rows in the cycle.
The right hand side will be odd and all odd coefficients can be reduced by one.
The constraint is
$\sum_j \textrm{even}(j)\,x_j \leq \textrm{odd}$
which can be replaced by
$\sum_j \frac{1}{2}\textrm{even}(j)\,x_j \leq \frac{1}{2}(\textrm{odd}-1)$.
A similar cut can be generated for $\sum_i x_i \geq 1$.

\textsl{(default = 0)}
\begin{itemize}
\item[0] don't add odd hole cuts
\item[1] add odd hole cuts
\end{itemize}

\item[\label{cliquecuts}\hypertarget{cliquecuts}
{\textbf{cliquecuts (\slshape{integer})}}]\hspace{1.0in}

This options switches the generation of clique cuts on.
Clique cuts are of the form ''sum of a set of variables $\leq$ 1''.

\textsl{(default = 1)}
\begin{itemize}
\item[0] don't add clique cuts
\item[1] add clique cuts
\end{itemize}

\item[\label{flowcovercuts}\hypertarget{flowcovercuts}
{\textbf{flowcovercuts (\slshape{integer})}}]\hspace{1.0in}

This options switches the generation of flow cover cuts on.\\
From the CglFlowCover description:
The Cgl Flow Cover Cut generator generates lifted simple generalized flow cover inequalities.
Since flow cover inequalities are generally not facet-defining, they are lifted to obtain stronger inequalities.
Although flow cover inequalities requires a special problem structure to be generated, they are quite useful for solving general mixed integer linear programs.

\textsl{(default = 1)}
\begin{itemize}
\item[0] don't add flow cover cuts
\item[1] add flow cover cuts
\end{itemize}

\item[\label{mircuts}\hypertarget{mircuts}
{\textbf{mircuts (\slshape{integer})}}]\hspace{1.0in}

This options switches the generation of mixed integer rounding cuts on.

\textsl{(default = 1)}
\begin{itemize}
\item[0] don't add mixed integer rounding cuts
\item[1] add mixed integer rounding cuts
\end{itemize}

\item[\label{redsplitcuts}\hypertarget{redsplitcuts}
{\textbf{redsplitcuts (\slshape{integer})}}]\hspace{1.0in}

This options switches the generation of reduce and split cuts on.\\
From the CglRedSplit description:
Reduce-and-Split cuts are variants of Gomory cuts:
Starting from the current optimal tableau, linear combinations of the rows of the current optimal simplex tableau are used for generating Gomory cuts.
The choice of the linear combinations is driven by the objective of reducing the coefficients of the non basic continuous variables in the resulting row.

\textsl{(default = 0)}
\begin{itemize}
\item[0] don't add reduce and split cuts
\item[1] add reduce and split cuts
\end{itemize}

\item[\label{cutsonlyatroot}\hypertarget{cutsonlyatroot}
{\textbf{cutsonlyatroot (\slshape{integer})}}]\hspace{1.0in}

This option determines whether cuts should be generated only at the root node.

\textsl{(default = 1)}
\begin{itemize}
\item[0] generate cuts always in the branch and bound
\item[1] generate cuts only at root node
\end{itemize}

\item[\label{startalg}\hypertarget{startalg}
{\textbf{startalg (\slshape{string})}}]\hspace{1.0in}

This option determines whether a primal or dual simplex algorithm should be used to solve the root node.

\textsl{(default = dual)}
\begin{itemize}
\item[primal] primal simplex algorithm
\item[dual] dual simplex algorithm
\end{itemize}

\item[\label{writemps}\hypertarget{writemps}
{\textbf{writemps (\slshape{string})}}]\hspace{1.0in}

Write an MPS problem file.
The parameter value is the name of the MPS file.

\item[\label{tol_dual}\hypertarget{tol_dual}
{\textbf{tol\_dual (\slshape{real})}}]\hspace{1.0in}

The maximum amount the dual constraints can be violated and still be considered feasible.

\textsl{(default = 1e-7)}

\item[\label{tol_primal}\hypertarget{tol_primal}
{\textbf{tol\_primal (\slshape{real})}}]\hspace{1.0in}

The maximum amount the primal constraints can be violated and still be considered feasible.

\textsl{(default = 1e-7)}

\item[\label{tol_integer}\hypertarget{tol_integer}
{\textbf{tol\_integer (\slshape{real})}}]\hspace{1.0in}

An integer variable is deemed to be at an integral value if it is no further than the value of this parameter from the next integral value away.

\textsl{(default = 1e-6)}


\item[\label{scaling}\hypertarget{scaling}
{\textbf{scaling (\slshape{integer})}}]\hspace{1.0in}

This option determines whether the linear relaxation should be scaled.

\textsl{(default = 1)}

\item[\label{presolve}\hypertarget{presolve}
{\textbf{presolve (\slshape{integer})}}]\hspace{1.0in}

This option determines whether a presolve should be applied to the linear relaxation before it is solved the first time.

\textsl{(default = 1)}

\item[\label{printfrequency}\hypertarget{printfrequency}
{\textbf{printfrequency (\slshape{integer})}}]\hspace{1.0in}

This option controls the number of nodes evaluated between status prints.

\textsl{(default = 10)}

\item[\label{nodecompare}\hypertarget{nodecompare}
{\textbf{nodecompare (\slshape{string})}}]\hspace{1.0in}

This option determines the comparision method that is used to determine the tree search order.
The following documentation is taken from the Cbc manual.

\textsl{(default = default)}
\begin{itemize}
\item[default] 
This is designed to do a mostly depth-first search until a solution has been found.
It then use estimates that are designed to give a slightly better solution.
If a reasonable number of nodes have been explored (or a reasonable number of solutions found),
then this class will adopt a breadth-first search (i.e., making a comparison based strictly on objective function values) unless the tree is very large, in which case it will revert to depth-first search.
\item[depth] 
This will always choose the node deepest in tree.
It gives minimum tree size but may take a long time to find the best solution.
\item[objective] 
This will always choose the node with the best objective value.
This may give a very large tree.
It is likely that the first solution found will be the best and the search should finish soon after the first solution is found.
\end{itemize}

\item[\label{reslim}\hypertarget{reslim}
{\textbf{reslim (\slshape{real})}}]\hspace{1.0in}

Maximum time in seconds.

\textsl{(default = GAMS reslim)}

\item[\label{iterlim}\hypertarget{iterlim}
{\textbf{iterlim (\slshape{integer})}}]\hspace{1.0in}

Maximum number of iterations.

\textsl{(default = GAMS iterlim)}

\item[\label{nodelim}\hypertarget{nodelim}
{\textbf{nodelim (\slshape{integer})}}]\hspace{1.0in}

Maximum number of nodes in the Branch and Bound.

\textsl{(default = GAMS nodlim)}

\item[\label{nodlim}\hypertarget{nodlim}
{\textbf{nodlim (\slshape{integer})}}]\hspace{1.0in}

Maximum number of nodes in the Branch and Bound.
This option is overwritten by nodelim, if set.

\textsl{(default = GAMS nodlim)}

\item[\label{optca}\hypertarget{optca}
{\textbf{optca (\slshape{real})}}]\hspace{1.0in}

Absolute optimality criterion for a MIP.
Cbc stops if the gap between the best known solution and the best possible solution is less than this value.

\textsl{(default = GAMS optca)}

\item[\label{optcr}\hypertarget{optcr}
{\textbf{optcr (\slshape{real})}}]\hspace{1.0in}

Relative optimality criterion for a MIP.
Cbc stops if the relative gap between the best known solution and the best possible solution is less than this value.

\textsl{(default = GAMS optcr)}

\item[\label{cutoff}\hypertarget{cutoff}
{\textbf{cutoff (\slshape{real})}}]\hspace{1.0in}

Cbc stops if the objective function values exceeds (in case of maximization) or falls below (in case of minimization) this value.

\textsl{(default = GAMS cutoff)}
\end{description}


%\bibliographystyle{plain}
%\bibliography{coinlibd}
%\renewcommand{\bibname}{CBC References}
%\chapter{\CBC}
\label{cha:coincbc}
\hypertarget{cha:coincbc}{}

\CBC (COIN-OR Branch and Cut) is an open-source mixed integer programming solver working with the COIN-OR LP solver \textsc{CLP} and the COIN-OR Cut generator library \textsc{Cgl}.
The code has been written primarily by John J. Forrest.

For more information we refer to the website of \CBC, \textsc{CGL}, and \textsc{CLP}:
\url{https://projects.coin-or.org/Cbc}, \url{https://projects.coin-or.org/Cgl}, \url{https://projects.coin-or.org/Clp}.
Most of the \CBC documentation in the section was copied from the help in the \CBC standalone version.

The \CBC link in \GAMS supports continuous, binary, integer, semicontinuous, semiinteger variables, special ordered sets of type 1 and 2, and branching priorities (see chapter 17.1 of the \GAMS User's Guide).
% Quadratic objective functions are not supported yet.

\section{Usage}

The following statement can be used inside your \GAMS program to specify using \CBC
\begin{verbatim}
  Option LP = CBC;     { or MIP or RMIP }
\end{verbatim}

The above statement should appear before the Solve statement.
If \CBC was specified as the default solver during \GAMS installation, the above statement is not necessary.

\subsection{Specification of Options}

The \GAMS/\CBC options file consists of one option or comment per line.
An asterisk (*) at the beginning of a line causes the entire line to be ignored.
Otherwise, the line will be interpreted as an option name and value separated by any amount of white space (blanks or tabs).
Following is an example options file \texttt{cbc.opt}.
\begin{verbatim}
  cuts root
  perturbation off
\end{verbatim}
It will cause \CBC to use cut generators only in the root node and turns off the perturbation of the LP relaxation.

\GAMS/\CBC currently does not support the \GAMS Branch-and-Cut-and-Heuristic (BCH) Facility.
If you need to use \GAMS/\CBC with BCH, please consider to use a \GAMS system of version $\leq 23.3$, available at \url{http://www.gams.com/download/download_old.htm}.

% \GAMS/\CBC supports the \GAMS Branch-and-Cut-and-Heuristic (BCH) Facility.
% The \GAMS BCH facility automates all major steps necessary to define, execute, and control the use of user defined routines within the framework of general purpose MIP codes.
% Currently supported are user defined cut generators and heuristics.
% Please see the BCH documentation at \texttt{http://www.gams.com/docs/bch.htm} for further information.

\section{Options}

Among all \CBC options, the following \GAMS parameters are currently supported in \CBC:
\hyperlink{reslim}{reslim}, \hyperlink{iterlim}{iterlim}, \hyperlink{nodlim}{nodlim}, \hyperlink{optca}{optca}, \hyperlink{optcr}{optcr}, \hyperlink{increment}{cheat}, \hyperlink{cutoff}{cutoff}, \hyperlink{threads}{threads} (experimental on Windows systems).

There are many parameters which can affect the performance the \CBC{}s Branch and Cut Algorithm.
First just try with default settings and look carefully at the log file.
Did cuts help? Did they take too long? Look at the output to see which cuts were effective and then do some tuning (see the option \hyperlink{cuts}{cuts}).
If the \hyperlink{preprocess}{preprocessing} reduced the size of the problem or strengthened many coefficients then it is probably wise to leave it on.
Switch off \hyperlink{heuristics}{heuristics} which did not provide solutions.
The other major area to look at is the search.
Hopefully good solutions were obtained fairly early in the search so the important point is to select the best variable to branch on.
See whether strong branching did a good job -- or did it just take a lot of iterations?
Adjust the options \hyperlink{strongbranching}{strongbranching} and \hyperlink{trustpseudocosts}{trustpseudocosts}.

In the following, we summarize all available \CBC options.
\subsection{Summary of CoinCbc Options}

Among all Cbc options, the following GAMS parameters are currently supported in CoinCbc:
\hyperlink{reslim}{reslim}, \hyperlink{iterlim}{iterlim}, \hyperlink{nodlim}{nodlim}, \hyperlink{optca}{optca}, \hyperlink{optcr}{optcr}, \hyperlink{cutoff}{cutoff}.

Currently CoinCbc understands the following options:
\begin{tabbing}
\hspace {1.1in} \= \\
\hyperlink{roundingheuristic}
{roundingheuristic} \> rounding heuristic \\
\hyperlink{localsearch}
{localsearch} \> local search \\
\hyperlink{strongbranching}
{strongbranching} \> strong branching \\
\hyperlink{integerpresolve}
{integerpresolve} \> integer presolve \\
\hyperlink{findsos}
{findsos} \> find sos in integer presolve \\
\hyperlink{cuts}
{cuts} \> switch for cut generation \\
\hyperlink{probing}
{probing} \> probing \\
\hyperlink{gomorycuts}
{gomorycuts} \> Gomory cuts \\
\hyperlink{knapsackcuts}
{knapsackcuts} \> knapsack cover cuts \\
\hyperlink{oddholecuts}
{oddholecuts} \> odd hole cuts \\
\hyperlink{cliquecuts}
{cliquecuts} \> clique cuts \\
\hyperlink{flowcovercuts}
{flowcovercuts} \> flow cover cuts \\
\hyperlink{mircuts}
{mircuts} \> mixed integer rounding cuts \\
\hyperlink{redsplitcuts}
{redsplitcuts} \> reduce and split cuts \\
\hyperlink{cutsonlyatroot}
{cutsonlyatroot} \> whether cuts are only generated at the root node \\
\hyperlink{startalg}
{startalg} \> LP solver for root node \\
\hyperlink{writemps}
{writemps} \> create MPS file for problem \\
\hyperlink{tol_dual}
{tol\_dual} \> dual feasibility tolerance \\
\hyperlink{tol_primal}
{tol\_primal} \> primal feasibility tolerance \\
\hyperlink{tol_integer}
{tol\_integer} \> tolerance for integrality \\
\hyperlink{scaling}
{scaling} \> switch for scaling of LP \\
\hyperlink{presolve}
{scaling} \> switch for initial presolve of LP \\
\hyperlink{printfrequency}
{printfrequency} \> print frequency \\
\hyperlink{nodecompare}
{nodecompare} \> comparision method to determine tree search order \\
\hyperlink{reslim}
{reslim} \> resource limit \\
\hyperlink{iterlim}
{iterlim} \> iteration limit \\
\hyperlink{nodelim}
{nodelim} \> node limit \\
\hyperlink{nodlim}
{nodlim} \> node limit \\
\hyperlink{optca}
{optca} \> absolute stopping tolerance \\
\hyperlink{optcr}
{optcr} \> relative stopping tolerance \\
\hyperlink{cutoff}
{cutoff} \> cutoff for objective function value \\
\end{tabbing}



In the following, we give a detailed description of all available \CBC options.
\subsection{Detailed Descriptions of CoinCbc Options}

\begin{description}

\item[\label{roundingheuristic}\hypertarget{roundingheuristic}
{\textbf{roundingheuristic (\slshape{integer})}}]\hspace{1.0in}

A simple rounding heuristic to find feasible solutions of the MIP.

\textsl{(default = 1)}

\item[\label{localsearch}\hypertarget{localsearch}
{\textbf{localsearch (\slshape{integer})}}]\hspace{1.0in}

A local search heuristic to find feasible solutions of the MIP.

\textsl{(default = 1)}
\begin{itemize}
\item[0] local search heuristic is not used
\item[1] local search heuristic is used
\end{itemize}

\item[\label{strongbranching}\hypertarget{strongbranching}
{\textbf{strongbranching (\slshape{integer})}}]\hspace{1.0in}

If strong branching is switched on, then the maximum number of candidates to be evaluated for strong branching is set to 10 (if the problem has less then 5000 variables) or 5 (if the problem has more then 5000 variables).
If strong branching is switched off, the maximum number is set to 0. (However, Cbc reports still some (but less) strong branching operations.)

\textsl{(default = 1)}
\begin{itemize}
\item[0] strong branching is switched off
\item[1] strong branching is switched on
\end{itemize}

\item[\label{integerpresolve}\hypertarget{integerpresolve}
{\textbf{integerpresolve (\slshape{integer})}}]\hspace{1.0in}

If switched on, a preprocessing on the MIP will be performed.
The integer presolve also does probing, independent of the setting of the 'probing' parameter.\\
From the CglPreprocess description:
This method uses other cut generators to strengthen cuts, establish that some cuts are redundant, fix variables and find relationships such as $x + y = 1$.
While cuts can be added at any time in the tree, some cuts are actually just stronger versions of existing ones.
They can thus replace the existing cuts rather than being added as new cuts. This is awkward in the tree but reasonable at the root node.

\textsl{(default = 1)}
\begin{itemize}
\item[0] don't do integer presolve
\item[1] do integer presolve
\end{itemize}

\item[\label{findsos}\hypertarget{findsos}
{\textbf{findsos (\slshape{integer})}}]\hspace{1.0in}

This option switches the identification for special ordered sets of type 1 in the integer preprocessing on.
Cbc then searches for rows with upper bound 1 and where all nonzero coefficients are 1.

\textsl{(default = 1)}
\begin{itemize}
\item[0] don't try to find special ordered sets
\item[1] try to find special ordered sets
\end{itemize}

\item[\label{cuts}\hypertarget{cuts}
{\textbf{cuts (\slshape{integer})}}]\hspace{1.0in}

Regulates the use of cut generator routines in Cbc.\\
If set to automatic, then the parameters for the single cut generators determine which are added.

\textsl{(default = 0)}
\begin{itemize}
\item[-1] no cuts will be generated
\item[0] automatic
\item[1] cuts from all available cut classes will be generated
\end{itemize}

\item[\label{probing}\hypertarget{probing}
{\textbf{probing (\slshape{integer})}}]\hspace{1.0in}

This option switches probing on as cut generator.\\
From the CglProbing description:
For selected integer variables (e.g. unsatisfied ones) the effect of setting them up or down is investigated.
Setting a variable up may in turn set other variables (continuous as well as integer).

\textsl{(default = 1)}
\begin{itemize}
\item[0] don't do probing
\item[1] do probing
\end{itemize}

\item[\label{gomorycuts}\hypertarget{gomorycuts}
{\textbf{gomorycuts (\slshape{integer})}}]\hspace{1.0in}

This options switches the generation of mixed integer Gomory Cuts on.

\textsl{(default = 1)}
\begin{itemize}
\item[0] don't add gomory cuts
\item[1] add gomory cuts
\end{itemize}

\item[\label{knapsackcuts}\hypertarget{knapsackcuts}
{\textbf{knapsackcuts (\slshape{integer})}}]\hspace{1.0in}

This options switches the generation of knapsack cover cuts on.\\
From the CglKnapsackCover description:
CglKnapsackCover looks for a series of different types of minimal covers.
If a minimal cover is found, it lifts the associated minimal cover inequality and adds the lifted cut to the cut set.

\textsl{(default = 1)}
\begin{itemize}
\item[0] don't add knapsack cover cuts
\item[1] add knapsack cover cuts
\end{itemize}

\item[\label{oddholecuts}\hypertarget{oddholecuts}
{\textbf{oddholecuts (\slshape{integer})}}]\hspace{1.0in}

This options switches the generation of odd hole cuts on.\\
From the CglOddHole description:
This looks at all rows of type $\sum_i x_i \leq 1 \textrm{(or = 1)}$ (with $x$ binary) and sees if there is an odd cycle cut.
This is then lifted by using the corresponding Chvatal cut, i.e. by summing up all rows in the cycle.
The right hand side will be odd and all odd coefficients can be reduced by one.
The constraint is
$\sum_j \textrm{even}(j)\,x_j \leq \textrm{odd}$
which can be replaced by
$\sum_j \frac{1}{2}\textrm{even}(j)\,x_j \leq \frac{1}{2}(\textrm{odd}-1)$.
A similar cut can be generated for $\sum_i x_i \geq 1$.

\textsl{(default = 0)}
\begin{itemize}
\item[0] don't add odd hole cuts
\item[1] add odd hole cuts
\end{itemize}

\item[\label{cliquecuts}\hypertarget{cliquecuts}
{\textbf{cliquecuts (\slshape{integer})}}]\hspace{1.0in}

This options switches the generation of clique cuts on.
Clique cuts are of the form ''sum of a set of variables $\leq$ 1''.

\textsl{(default = 1)}
\begin{itemize}
\item[0] don't add clique cuts
\item[1] add clique cuts
\end{itemize}

\item[\label{flowcovercuts}\hypertarget{flowcovercuts}
{\textbf{flowcovercuts (\slshape{integer})}}]\hspace{1.0in}

This options switches the generation of flow cover cuts on.\\
From the CglFlowCover description:
The Cgl Flow Cover Cut generator generates lifted simple generalized flow cover inequalities.
Since flow cover inequalities are generally not facet-defining, they are lifted to obtain stronger inequalities.
Although flow cover inequalities requires a special problem structure to be generated, they are quite useful for solving general mixed integer linear programs.

\textsl{(default = 1)}
\begin{itemize}
\item[0] don't add flow cover cuts
\item[1] add flow cover cuts
\end{itemize}

\item[\label{mircuts}\hypertarget{mircuts}
{\textbf{mircuts (\slshape{integer})}}]\hspace{1.0in}

This options switches the generation of mixed integer rounding cuts on.

\textsl{(default = 1)}
\begin{itemize}
\item[0] don't add mixed integer rounding cuts
\item[1] add mixed integer rounding cuts
\end{itemize}

\item[\label{redsplitcuts}\hypertarget{redsplitcuts}
{\textbf{redsplitcuts (\slshape{integer})}}]\hspace{1.0in}

This options switches the generation of reduce and split cuts on.\\
From the CglRedSplit description:
Reduce-and-Split cuts are variants of Gomory cuts:
Starting from the current optimal tableau, linear combinations of the rows of the current optimal simplex tableau are used for generating Gomory cuts.
The choice of the linear combinations is driven by the objective of reducing the coefficients of the non basic continuous variables in the resulting row.

\textsl{(default = 0)}
\begin{itemize}
\item[0] don't add reduce and split cuts
\item[1] add reduce and split cuts
\end{itemize}

\item[\label{cutsonlyatroot}\hypertarget{cutsonlyatroot}
{\textbf{cutsonlyatroot (\slshape{integer})}}]\hspace{1.0in}

This option determines whether cuts should be generated only at the root node.

\textsl{(default = 1)}
\begin{itemize}
\item[0] generate cuts always in the branch and bound
\item[1] generate cuts only at root node
\end{itemize}

\item[\label{startalg}\hypertarget{startalg}
{\textbf{startalg (\slshape{string})}}]\hspace{1.0in}

This option determines whether a primal or dual simplex algorithm should be used to solve the root node.

\textsl{(default = dual)}
\begin{itemize}
\item[primal] primal simplex algorithm
\item[dual] dual simplex algorithm
\end{itemize}

\item[\label{writemps}\hypertarget{writemps}
{\textbf{writemps (\slshape{string})}}]\hspace{1.0in}

Write an MPS problem file.
The parameter value is the name of the MPS file.

\item[\label{tol_dual}\hypertarget{tol_dual}
{\textbf{tol\_dual (\slshape{real})}}]\hspace{1.0in}

The maximum amount the dual constraints can be violated and still be considered feasible.

\textsl{(default = 1e-7)}

\item[\label{tol_primal}\hypertarget{tol_primal}
{\textbf{tol\_primal (\slshape{real})}}]\hspace{1.0in}

The maximum amount the primal constraints can be violated and still be considered feasible.

\textsl{(default = 1e-7)}

\item[\label{tol_integer}\hypertarget{tol_integer}
{\textbf{tol\_integer (\slshape{real})}}]\hspace{1.0in}

An integer variable is deemed to be at an integral value if it is no further than the value of this parameter from the next integral value away.

\textsl{(default = 1e-6)}


\item[\label{scaling}\hypertarget{scaling}
{\textbf{scaling (\slshape{integer})}}]\hspace{1.0in}

This option determines whether the linear relaxation should be scaled.

\textsl{(default = 1)}

\item[\label{presolve}\hypertarget{presolve}
{\textbf{presolve (\slshape{integer})}}]\hspace{1.0in}

This option determines whether a presolve should be applied to the linear relaxation before it is solved the first time.

\textsl{(default = 1)}

\item[\label{printfrequency}\hypertarget{printfrequency}
{\textbf{printfrequency (\slshape{integer})}}]\hspace{1.0in}

This option controls the number of nodes evaluated between status prints.

\textsl{(default = 10)}

\item[\label{nodecompare}\hypertarget{nodecompare}
{\textbf{nodecompare (\slshape{string})}}]\hspace{1.0in}

This option determines the comparision method that is used to determine the tree search order.
The following documentation is taken from the Cbc manual.

\textsl{(default = default)}
\begin{itemize}
\item[default] 
This is designed to do a mostly depth-first search until a solution has been found.
It then use estimates that are designed to give a slightly better solution.
If a reasonable number of nodes have been explored (or a reasonable number of solutions found),
then this class will adopt a breadth-first search (i.e., making a comparison based strictly on objective function values) unless the tree is very large, in which case it will revert to depth-first search.
\item[depth] 
This will always choose the node deepest in tree.
It gives minimum tree size but may take a long time to find the best solution.
\item[objective] 
This will always choose the node with the best objective value.
This may give a very large tree.
It is likely that the first solution found will be the best and the search should finish soon after the first solution is found.
\end{itemize}

\item[\label{reslim}\hypertarget{reslim}
{\textbf{reslim (\slshape{real})}}]\hspace{1.0in}

Maximum time in seconds.

\textsl{(default = GAMS reslim)}

\item[\label{iterlim}\hypertarget{iterlim}
{\textbf{iterlim (\slshape{integer})}}]\hspace{1.0in}

Maximum number of iterations.

\textsl{(default = GAMS iterlim)}

\item[\label{nodelim}\hypertarget{nodelim}
{\textbf{nodelim (\slshape{integer})}}]\hspace{1.0in}

Maximum number of nodes in the Branch and Bound.

\textsl{(default = GAMS nodlim)}

\item[\label{nodlim}\hypertarget{nodlim}
{\textbf{nodlim (\slshape{integer})}}]\hspace{1.0in}

Maximum number of nodes in the Branch and Bound.
This option is overwritten by nodelim, if set.

\textsl{(default = GAMS nodlim)}

\item[\label{optca}\hypertarget{optca}
{\textbf{optca (\slshape{real})}}]\hspace{1.0in}

Absolute optimality criterion for a MIP.
Cbc stops if the gap between the best known solution and the best possible solution is less than this value.

\textsl{(default = GAMS optca)}

\item[\label{optcr}\hypertarget{optcr}
{\textbf{optcr (\slshape{real})}}]\hspace{1.0in}

Relative optimality criterion for a MIP.
Cbc stops if the relative gap between the best known solution and the best possible solution is less than this value.

\textsl{(default = GAMS optcr)}

\item[\label{cutoff}\hypertarget{cutoff}
{\textbf{cutoff (\slshape{real})}}]\hspace{1.0in}

Cbc stops if the objective function values exceeds (in case of maximization) or falls below (in case of minimization) this value.

\textsl{(default = GAMS cutoff)}
\end{description}


%\bibliographystyle{plain}
%\bibliography{coinlibd}
%\renewcommand{\bibname}{CBC References}
%\chapter{\CBC}
\label{cha:coincbc}
\hypertarget{cha:coincbc}{}

\CBC (COIN-OR Branch and Cut) is an open-source mixed integer programming solver working with the COIN-OR LP solver \textsc{CLP} and the COIN-OR Cut generator library \textsc{Cgl}.
The code has been written primarily by John J. Forrest.

For more information we refer to the website of \CBC, \textsc{CGL}, and \textsc{CLP}:
\url{https://projects.coin-or.org/Cbc}, \url{https://projects.coin-or.org/Cgl}, \url{https://projects.coin-or.org/Clp}.
Most of the \CBC documentation in the section was copied from the help in the \CBC standalone version.

The \CBC link in \GAMS supports continuous, binary, integer, semicontinuous, semiinteger variables, special ordered sets of type 1 and 2, and branching priorities (see chapter 17.1 of the \GAMS User's Guide).
% Quadratic objective functions are not supported yet.

\section{Usage}

The following statement can be used inside your \GAMS program to specify using \CBC
\begin{verbatim}
  Option LP = CBC;     { or MIP or RMIP }
\end{verbatim}

The above statement should appear before the Solve statement.
If \CBC was specified as the default solver during \GAMS installation, the above statement is not necessary.

\subsection{Specification of Options}

The \GAMS/\CBC options file consists of one option or comment per line.
An asterisk (*) at the beginning of a line causes the entire line to be ignored.
Otherwise, the line will be interpreted as an option name and value separated by any amount of white space (blanks or tabs).
Following is an example options file \texttt{cbc.opt}.
\begin{verbatim}
  cuts root
  perturbation off
\end{verbatim}
It will cause \CBC to use cut generators only in the root node and turns off the perturbation of the LP relaxation.

\GAMS/\CBC currently does not support the \GAMS Branch-and-Cut-and-Heuristic (BCH) Facility.
If you need to use \GAMS/\CBC with BCH, please consider to use a \GAMS system of version $\leq 23.3$, available at \url{http://www.gams.com/download/download_old.htm}.

% \GAMS/\CBC supports the \GAMS Branch-and-Cut-and-Heuristic (BCH) Facility.
% The \GAMS BCH facility automates all major steps necessary to define, execute, and control the use of user defined routines within the framework of general purpose MIP codes.
% Currently supported are user defined cut generators and heuristics.
% Please see the BCH documentation at \texttt{http://www.gams.com/docs/bch.htm} for further information.

\section{Options}

Among all \CBC options, the following \GAMS parameters are currently supported in \CBC:
\hyperlink{reslim}{reslim}, \hyperlink{iterlim}{iterlim}, \hyperlink{nodlim}{nodlim}, \hyperlink{optca}{optca}, \hyperlink{optcr}{optcr}, \hyperlink{increment}{cheat}, \hyperlink{cutoff}{cutoff}, \hyperlink{threads}{threads} (experimental on Windows systems).

There are many parameters which can affect the performance the \CBC{}s Branch and Cut Algorithm.
First just try with default settings and look carefully at the log file.
Did cuts help? Did they take too long? Look at the output to see which cuts were effective and then do some tuning (see the option \hyperlink{cuts}{cuts}).
If the \hyperlink{preprocess}{preprocessing} reduced the size of the problem or strengthened many coefficients then it is probably wise to leave it on.
Switch off \hyperlink{heuristics}{heuristics} which did not provide solutions.
The other major area to look at is the search.
Hopefully good solutions were obtained fairly early in the search so the important point is to select the best variable to branch on.
See whether strong branching did a good job -- or did it just take a lot of iterations?
Adjust the options \hyperlink{strongbranching}{strongbranching} and \hyperlink{trustpseudocosts}{trustpseudocosts}.

In the following, we summarize all available \CBC options.
\subsection{Summary of CoinCbc Options}

Among all Cbc options, the following GAMS parameters are currently supported in CoinCbc:
\hyperlink{reslim}{reslim}, \hyperlink{iterlim}{iterlim}, \hyperlink{nodlim}{nodlim}, \hyperlink{optca}{optca}, \hyperlink{optcr}{optcr}, \hyperlink{cutoff}{cutoff}.

Currently CoinCbc understands the following options:
\begin{tabbing}
\hspace {1.1in} \= \\
\hyperlink{roundingheuristic}
{roundingheuristic} \> rounding heuristic \\
\hyperlink{localsearch}
{localsearch} \> local search \\
\hyperlink{strongbranching}
{strongbranching} \> strong branching \\
\hyperlink{integerpresolve}
{integerpresolve} \> integer presolve \\
\hyperlink{findsos}
{findsos} \> find sos in integer presolve \\
\hyperlink{cuts}
{cuts} \> switch for cut generation \\
\hyperlink{probing}
{probing} \> probing \\
\hyperlink{gomorycuts}
{gomorycuts} \> Gomory cuts \\
\hyperlink{knapsackcuts}
{knapsackcuts} \> knapsack cover cuts \\
\hyperlink{oddholecuts}
{oddholecuts} \> odd hole cuts \\
\hyperlink{cliquecuts}
{cliquecuts} \> clique cuts \\
\hyperlink{flowcovercuts}
{flowcovercuts} \> flow cover cuts \\
\hyperlink{mircuts}
{mircuts} \> mixed integer rounding cuts \\
\hyperlink{redsplitcuts}
{redsplitcuts} \> reduce and split cuts \\
\hyperlink{cutsonlyatroot}
{cutsonlyatroot} \> whether cuts are only generated at the root node \\
\hyperlink{startalg}
{startalg} \> LP solver for root node \\
\hyperlink{writemps}
{writemps} \> create MPS file for problem \\
\hyperlink{tol_dual}
{tol\_dual} \> dual feasibility tolerance \\
\hyperlink{tol_primal}
{tol\_primal} \> primal feasibility tolerance \\
\hyperlink{tol_integer}
{tol\_integer} \> tolerance for integrality \\
\hyperlink{scaling}
{scaling} \> switch for scaling of LP \\
\hyperlink{presolve}
{scaling} \> switch for initial presolve of LP \\
\hyperlink{printfrequency}
{printfrequency} \> print frequency \\
\hyperlink{nodecompare}
{nodecompare} \> comparision method to determine tree search order \\
\hyperlink{reslim}
{reslim} \> resource limit \\
\hyperlink{iterlim}
{iterlim} \> iteration limit \\
\hyperlink{nodelim}
{nodelim} \> node limit \\
\hyperlink{nodlim}
{nodlim} \> node limit \\
\hyperlink{optca}
{optca} \> absolute stopping tolerance \\
\hyperlink{optcr}
{optcr} \> relative stopping tolerance \\
\hyperlink{cutoff}
{cutoff} \> cutoff for objective function value \\
\end{tabbing}



In the following, we give a detailed description of all available \CBC options.
\subsection{Detailed Descriptions of CoinCbc Options}

\begin{description}

\item[\label{roundingheuristic}\hypertarget{roundingheuristic}
{\textbf{roundingheuristic (\slshape{integer})}}]\hspace{1.0in}

A simple rounding heuristic to find feasible solutions of the MIP.

\textsl{(default = 1)}

\item[\label{localsearch}\hypertarget{localsearch}
{\textbf{localsearch (\slshape{integer})}}]\hspace{1.0in}

A local search heuristic to find feasible solutions of the MIP.

\textsl{(default = 1)}
\begin{itemize}
\item[0] local search heuristic is not used
\item[1] local search heuristic is used
\end{itemize}

\item[\label{strongbranching}\hypertarget{strongbranching}
{\textbf{strongbranching (\slshape{integer})}}]\hspace{1.0in}

If strong branching is switched on, then the maximum number of candidates to be evaluated for strong branching is set to 10 (if the problem has less then 5000 variables) or 5 (if the problem has more then 5000 variables).
If strong branching is switched off, the maximum number is set to 0. (However, Cbc reports still some (but less) strong branching operations.)

\textsl{(default = 1)}
\begin{itemize}
\item[0] strong branching is switched off
\item[1] strong branching is switched on
\end{itemize}

\item[\label{integerpresolve}\hypertarget{integerpresolve}
{\textbf{integerpresolve (\slshape{integer})}}]\hspace{1.0in}

If switched on, a preprocessing on the MIP will be performed.
The integer presolve also does probing, independent of the setting of the 'probing' parameter.\\
From the CglPreprocess description:
This method uses other cut generators to strengthen cuts, establish that some cuts are redundant, fix variables and find relationships such as $x + y = 1$.
While cuts can be added at any time in the tree, some cuts are actually just stronger versions of existing ones.
They can thus replace the existing cuts rather than being added as new cuts. This is awkward in the tree but reasonable at the root node.

\textsl{(default = 1)}
\begin{itemize}
\item[0] don't do integer presolve
\item[1] do integer presolve
\end{itemize}

\item[\label{findsos}\hypertarget{findsos}
{\textbf{findsos (\slshape{integer})}}]\hspace{1.0in}

This option switches the identification for special ordered sets of type 1 in the integer preprocessing on.
Cbc then searches for rows with upper bound 1 and where all nonzero coefficients are 1.

\textsl{(default = 1)}
\begin{itemize}
\item[0] don't try to find special ordered sets
\item[1] try to find special ordered sets
\end{itemize}

\item[\label{cuts}\hypertarget{cuts}
{\textbf{cuts (\slshape{integer})}}]\hspace{1.0in}

Regulates the use of cut generator routines in Cbc.\\
If set to automatic, then the parameters for the single cut generators determine which are added.

\textsl{(default = 0)}
\begin{itemize}
\item[-1] no cuts will be generated
\item[0] automatic
\item[1] cuts from all available cut classes will be generated
\end{itemize}

\item[\label{probing}\hypertarget{probing}
{\textbf{probing (\slshape{integer})}}]\hspace{1.0in}

This option switches probing on as cut generator.\\
From the CglProbing description:
For selected integer variables (e.g. unsatisfied ones) the effect of setting them up or down is investigated.
Setting a variable up may in turn set other variables (continuous as well as integer).

\textsl{(default = 1)}
\begin{itemize}
\item[0] don't do probing
\item[1] do probing
\end{itemize}

\item[\label{gomorycuts}\hypertarget{gomorycuts}
{\textbf{gomorycuts (\slshape{integer})}}]\hspace{1.0in}

This options switches the generation of mixed integer Gomory Cuts on.

\textsl{(default = 1)}
\begin{itemize}
\item[0] don't add gomory cuts
\item[1] add gomory cuts
\end{itemize}

\item[\label{knapsackcuts}\hypertarget{knapsackcuts}
{\textbf{knapsackcuts (\slshape{integer})}}]\hspace{1.0in}

This options switches the generation of knapsack cover cuts on.\\
From the CglKnapsackCover description:
CglKnapsackCover looks for a series of different types of minimal covers.
If a minimal cover is found, it lifts the associated minimal cover inequality and adds the lifted cut to the cut set.

\textsl{(default = 1)}
\begin{itemize}
\item[0] don't add knapsack cover cuts
\item[1] add knapsack cover cuts
\end{itemize}

\item[\label{oddholecuts}\hypertarget{oddholecuts}
{\textbf{oddholecuts (\slshape{integer})}}]\hspace{1.0in}

This options switches the generation of odd hole cuts on.\\
From the CglOddHole description:
This looks at all rows of type $\sum_i x_i \leq 1 \textrm{(or = 1)}$ (with $x$ binary) and sees if there is an odd cycle cut.
This is then lifted by using the corresponding Chvatal cut, i.e. by summing up all rows in the cycle.
The right hand side will be odd and all odd coefficients can be reduced by one.
The constraint is
$\sum_j \textrm{even}(j)\,x_j \leq \textrm{odd}$
which can be replaced by
$\sum_j \frac{1}{2}\textrm{even}(j)\,x_j \leq \frac{1}{2}(\textrm{odd}-1)$.
A similar cut can be generated for $\sum_i x_i \geq 1$.

\textsl{(default = 0)}
\begin{itemize}
\item[0] don't add odd hole cuts
\item[1] add odd hole cuts
\end{itemize}

\item[\label{cliquecuts}\hypertarget{cliquecuts}
{\textbf{cliquecuts (\slshape{integer})}}]\hspace{1.0in}

This options switches the generation of clique cuts on.
Clique cuts are of the form ''sum of a set of variables $\leq$ 1''.

\textsl{(default = 1)}
\begin{itemize}
\item[0] don't add clique cuts
\item[1] add clique cuts
\end{itemize}

\item[\label{flowcovercuts}\hypertarget{flowcovercuts}
{\textbf{flowcovercuts (\slshape{integer})}}]\hspace{1.0in}

This options switches the generation of flow cover cuts on.\\
From the CglFlowCover description:
The Cgl Flow Cover Cut generator generates lifted simple generalized flow cover inequalities.
Since flow cover inequalities are generally not facet-defining, they are lifted to obtain stronger inequalities.
Although flow cover inequalities requires a special problem structure to be generated, they are quite useful for solving general mixed integer linear programs.

\textsl{(default = 1)}
\begin{itemize}
\item[0] don't add flow cover cuts
\item[1] add flow cover cuts
\end{itemize}

\item[\label{mircuts}\hypertarget{mircuts}
{\textbf{mircuts (\slshape{integer})}}]\hspace{1.0in}

This options switches the generation of mixed integer rounding cuts on.

\textsl{(default = 1)}
\begin{itemize}
\item[0] don't add mixed integer rounding cuts
\item[1] add mixed integer rounding cuts
\end{itemize}

\item[\label{redsplitcuts}\hypertarget{redsplitcuts}
{\textbf{redsplitcuts (\slshape{integer})}}]\hspace{1.0in}

This options switches the generation of reduce and split cuts on.\\
From the CglRedSplit description:
Reduce-and-Split cuts are variants of Gomory cuts:
Starting from the current optimal tableau, linear combinations of the rows of the current optimal simplex tableau are used for generating Gomory cuts.
The choice of the linear combinations is driven by the objective of reducing the coefficients of the non basic continuous variables in the resulting row.

\textsl{(default = 0)}
\begin{itemize}
\item[0] don't add reduce and split cuts
\item[1] add reduce and split cuts
\end{itemize}

\item[\label{cutsonlyatroot}\hypertarget{cutsonlyatroot}
{\textbf{cutsonlyatroot (\slshape{integer})}}]\hspace{1.0in}

This option determines whether cuts should be generated only at the root node.

\textsl{(default = 1)}
\begin{itemize}
\item[0] generate cuts always in the branch and bound
\item[1] generate cuts only at root node
\end{itemize}

\item[\label{startalg}\hypertarget{startalg}
{\textbf{startalg (\slshape{string})}}]\hspace{1.0in}

This option determines whether a primal or dual simplex algorithm should be used to solve the root node.

\textsl{(default = dual)}
\begin{itemize}
\item[primal] primal simplex algorithm
\item[dual] dual simplex algorithm
\end{itemize}

\item[\label{writemps}\hypertarget{writemps}
{\textbf{writemps (\slshape{string})}}]\hspace{1.0in}

Write an MPS problem file.
The parameter value is the name of the MPS file.

\item[\label{tol_dual}\hypertarget{tol_dual}
{\textbf{tol\_dual (\slshape{real})}}]\hspace{1.0in}

The maximum amount the dual constraints can be violated and still be considered feasible.

\textsl{(default = 1e-7)}

\item[\label{tol_primal}\hypertarget{tol_primal}
{\textbf{tol\_primal (\slshape{real})}}]\hspace{1.0in}

The maximum amount the primal constraints can be violated and still be considered feasible.

\textsl{(default = 1e-7)}

\item[\label{tol_integer}\hypertarget{tol_integer}
{\textbf{tol\_integer (\slshape{real})}}]\hspace{1.0in}

An integer variable is deemed to be at an integral value if it is no further than the value of this parameter from the next integral value away.

\textsl{(default = 1e-6)}


\item[\label{scaling}\hypertarget{scaling}
{\textbf{scaling (\slshape{integer})}}]\hspace{1.0in}

This option determines whether the linear relaxation should be scaled.

\textsl{(default = 1)}

\item[\label{presolve}\hypertarget{presolve}
{\textbf{presolve (\slshape{integer})}}]\hspace{1.0in}

This option determines whether a presolve should be applied to the linear relaxation before it is solved the first time.

\textsl{(default = 1)}

\item[\label{printfrequency}\hypertarget{printfrequency}
{\textbf{printfrequency (\slshape{integer})}}]\hspace{1.0in}

This option controls the number of nodes evaluated between status prints.

\textsl{(default = 10)}

\item[\label{nodecompare}\hypertarget{nodecompare}
{\textbf{nodecompare (\slshape{string})}}]\hspace{1.0in}

This option determines the comparision method that is used to determine the tree search order.
The following documentation is taken from the Cbc manual.

\textsl{(default = default)}
\begin{itemize}
\item[default] 
This is designed to do a mostly depth-first search until a solution has been found.
It then use estimates that are designed to give a slightly better solution.
If a reasonable number of nodes have been explored (or a reasonable number of solutions found),
then this class will adopt a breadth-first search (i.e., making a comparison based strictly on objective function values) unless the tree is very large, in which case it will revert to depth-first search.
\item[depth] 
This will always choose the node deepest in tree.
It gives minimum tree size but may take a long time to find the best solution.
\item[objective] 
This will always choose the node with the best objective value.
This may give a very large tree.
It is likely that the first solution found will be the best and the search should finish soon after the first solution is found.
\end{itemize}

\item[\label{reslim}\hypertarget{reslim}
{\textbf{reslim (\slshape{real})}}]\hspace{1.0in}

Maximum time in seconds.

\textsl{(default = GAMS reslim)}

\item[\label{iterlim}\hypertarget{iterlim}
{\textbf{iterlim (\slshape{integer})}}]\hspace{1.0in}

Maximum number of iterations.

\textsl{(default = GAMS iterlim)}

\item[\label{nodelim}\hypertarget{nodelim}
{\textbf{nodelim (\slshape{integer})}}]\hspace{1.0in}

Maximum number of nodes in the Branch and Bound.

\textsl{(default = GAMS nodlim)}

\item[\label{nodlim}\hypertarget{nodlim}
{\textbf{nodlim (\slshape{integer})}}]\hspace{1.0in}

Maximum number of nodes in the Branch and Bound.
This option is overwritten by nodelim, if set.

\textsl{(default = GAMS nodlim)}

\item[\label{optca}\hypertarget{optca}
{\textbf{optca (\slshape{real})}}]\hspace{1.0in}

Absolute optimality criterion for a MIP.
Cbc stops if the gap between the best known solution and the best possible solution is less than this value.

\textsl{(default = GAMS optca)}

\item[\label{optcr}\hypertarget{optcr}
{\textbf{optcr (\slshape{real})}}]\hspace{1.0in}

Relative optimality criterion for a MIP.
Cbc stops if the relative gap between the best known solution and the best possible solution is less than this value.

\textsl{(default = GAMS optcr)}

\item[\label{cutoff}\hypertarget{cutoff}
{\textbf{cutoff (\slshape{real})}}]\hspace{1.0in}

Cbc stops if the objective function values exceeds (in case of maximization) or falls below (in case of minimization) this value.

\textsl{(default = GAMS cutoff)}
\end{description}


%\bibliographystyle{plain}
%\bibliography{coinlibd}
%\renewcommand{\bibname}{CBC References}
%\chapter{\CBC}
\label{cha:coincbc}
\hypertarget{cha:coincbc}{}

\CBC (COIN-OR Branch and Cut) is an open-source mixed integer programming solver working with the COIN-OR LP solver \textsc{CLP} and the COIN-OR Cut generator library \textsc{Cgl}.
The code has been written primarily by John J. Forrest.

For more information we refer to the website of \CBC, \textsc{CGL}, and \textsc{CLP}:
\url{https://projects.coin-or.org/Cbc}, \url{https://projects.coin-or.org/Cgl}, \url{https://projects.coin-or.org/Clp}.
Most of the \CBC documentation in the section was copied from the help in the \CBC standalone version.

The \CBC link in \GAMS supports continuous, binary, integer, semicontinuous, semiinteger variables, special ordered sets of type 1 and 2, and branching priorities (see chapter 17.1 of the \GAMS User's Guide).
% Quadratic objective functions are not supported yet.

\section{Usage}

The following statement can be used inside your \GAMS program to specify using \CBC
\begin{verbatim}
  Option LP = CBC;     { or MIP or RMIP }
\end{verbatim}

The above statement should appear before the Solve statement.
If \CBC was specified as the default solver during \GAMS installation, the above statement is not necessary.

\subsection{Specification of Options}

The \GAMS/\CBC options file consists of one option or comment per line.
An asterisk (*) at the beginning of a line causes the entire line to be ignored.
Otherwise, the line will be interpreted as an option name and value separated by any amount of white space (blanks or tabs).
Following is an example options file \texttt{cbc.opt}.
\begin{verbatim}
  cuts root
  perturbation off
\end{verbatim}
It will cause \CBC to use cut generators only in the root node and turns off the perturbation of the LP relaxation.

\GAMS/\CBC currently does not support the \GAMS Branch-and-Cut-and-Heuristic (BCH) Facility.
If you need to use \GAMS/\CBC with BCH, please consider to use a \GAMS system of version $\leq 23.3$, available at \url{http://www.gams.com/download/download_old.htm}.

% \GAMS/\CBC supports the \GAMS Branch-and-Cut-and-Heuristic (BCH) Facility.
% The \GAMS BCH facility automates all major steps necessary to define, execute, and control the use of user defined routines within the framework of general purpose MIP codes.
% Currently supported are user defined cut generators and heuristics.
% Please see the BCH documentation at \texttt{http://www.gams.com/docs/bch.htm} for further information.

\section{Options}

Among all \CBC options, the following \GAMS parameters are currently supported in \CBC:
\hyperlink{reslim}{reslim}, \hyperlink{iterlim}{iterlim}, \hyperlink{nodlim}{nodlim}, \hyperlink{optca}{optca}, \hyperlink{optcr}{optcr}, \hyperlink{increment}{cheat}, \hyperlink{cutoff}{cutoff}, \hyperlink{threads}{threads} (experimental on Windows systems).

There are many parameters which can affect the performance the \CBC{}s Branch and Cut Algorithm.
First just try with default settings and look carefully at the log file.
Did cuts help? Did they take too long? Look at the output to see which cuts were effective and then do some tuning (see the option \hyperlink{cuts}{cuts}).
If the \hyperlink{preprocess}{preprocessing} reduced the size of the problem or strengthened many coefficients then it is probably wise to leave it on.
Switch off \hyperlink{heuristics}{heuristics} which did not provide solutions.
The other major area to look at is the search.
Hopefully good solutions were obtained fairly early in the search so the important point is to select the best variable to branch on.
See whether strong branching did a good job -- or did it just take a lot of iterations?
Adjust the options \hyperlink{strongbranching}{strongbranching} and \hyperlink{trustpseudocosts}{trustpseudocosts}.

In the following, we summarize all available \CBC options.
\input{optcbc_s}

In the following, we give a detailed description of all available \CBC options.
\input{optcbc_a}

%\bibliographystyle{plain}
%\bibliography{coinlibd}
%\renewcommand{\bibname}{CBC References}
%\input{cbc.bbl}

\chapterend

\chapterend

\chapterend

\chapterend
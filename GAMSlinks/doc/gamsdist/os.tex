\chapter{Optimization Services (OS)}

COIN-OR \OS (\textbf{O}ptimization \textbf{S}ervices) is an initiative to provide a set of standards for representing optimization instances, results, solver options, and communication between clients and solvers in a distributed environment using Web Services.
The code has been written primarily by Horand Gassmann, Jun Ma, and Kipp Martin.
Kipp Martin is the COIN-OR project leader for \OS.

For more information we refer to the web sites \url{http://www.optimizationservices.org} and \url{https://projects.coin-or.org/OS}, the \OS manual~\cite{Ma2005}, and the papers~\cite{FourerMaMartin2009,FourerMaMartin2010,OSManual}.

With the \OS link in \GAMS, you can send your instance to an Optimization Services Server for remote solving.

\OS supports continuous, binary, integer, semicontinuous, and semiinteger variables and linear and nonlinear equations.
Special ordered sets and indicator constraints are currently not supported.
Initial values are currently not supported by the \GAMS/\OS link.

\section{Usage}

The following statement can be used inside your \GAMS program to specify using \OS
\begin{verbatim}
  Option MINLP = OS;     { or LP, RMIP, MIP, DNLP, NLP, RMINLP, QCP, RMIQCP, MIQCP }
\end{verbatim}

The above statement should appear before the Solve statement.

By default, for a given instance of a \GAMS model, nothing happens.
To solve an instance remotely, you have to specify the URL of an Optimization Services Server via the option \texttt{service}.
Usually, the server chooses an appropriate solver for your instance, depending on their availability on the server.
A fully equipped server chooses
\textsc{CLP} for continuous linear models (LP and RMIP),
\IPOPT for continuous nonlinear models (NLP, DNLP, RMINLP, QCP, RMIQCP),
\CBC for mixed-integer linear models (MIP), and
\BONMIN for mixed-integer nonlinear models (MIQCP, MINLP).
An easy way to influence the choice of the solver on the server is the \texttt{solver} option.

Further options can be provided in an OSoL (Optimization Services Options Language) file, which is specified via the the \texttt{readosol} option.
An example OSoL file looks like
\begin{verbatim}
<?xml version="1.0" encoding="UTF-8"?>
<osol xmlns="os.optimizationservices.org" xmlns:xs="http://www.w3.org/2001/XMLSchema"
      xmlns:xsi="http://www.w3.org/2001/XMLSchema-instance"
      xsi:schemaLocation="os.optimizationservices.org
                          http://www.optimizationservices.org/schemas/2.0/OSoL.xsd">
<optimization>
  <solverOptions numberOfSolverOptions="3">
    <solverOption name="cuts" solver="cbc" value="off" />
    <solverOption name="max_active_nodes" solver="symphony"  value="2" />
    <solverOption name="max_iter" solver="ipopt" type="integer" value="2000"/>
  </solverOptions>
</optimization>
</osol>
\end{verbatim}
It specifies that if \CBC is used, then cutting planes are disabled,
if \textsc{SYMPHONY} is used, then at most 2 nodes should be active,
and if \IPOPT is used, then a limit of 2000 iterations is imposed.

By default, the call to the server is a \emph{synchronous} call.
The \GAMS process will wait for the result and then display the result.
This may not be desirable when solving large optimization models.
In order to use the remote solver service in an \emph{asynchronous} fashion, one can make use of the \GAMS Grid Computing Facility, see Appendix I in the \GAMS manual.

\section{Detailed Options Descriptions}

\subsection{Detailed Descriptions of CoinOS Options}

\begin{description}

\item[\label{readosol}\hypertarget{readosol}
{\textbf{readosol (\slshape{string})}}]\hspace{1.0in}

Specifies the name of an option file in OSoL format that is given to the solver or an OS server.


\item[\label{writeosil}\hypertarget{writeosil}
{\textbf{writeosil (\slshape{string})}}]\hspace{1.0in}

Specifies the name of a file in which the GAMS model instance should be writting in OSiL format.


\item[\label{writeosrl}\hypertarget{writeosrl}
{\textbf{writeosrl (\slshape{string})}}]\hspace{1.0in}

Specifies the name of a file in which the result of a solve process (solution, status, ...) should be writting in OSrL format.


\item[\label{service}\hypertarget{service}
{\textbf{service (\slshape{string})}}]\hspace{1.0in}

Specifies the URL of an Optimization Services Server.
If specified, then this server is contacted and the service method specified under service\_method is executed.
Note that by default the server executes CBC to solve an instance.
You can change the solver with the solver option.


\item[\label{service_method}\hypertarget{service_method}
{\textbf{service\_method (\slshape{string})}}]\hspace{1.0in}

Specifies the method to execute on a server.

\textsl{(default = solve)}
\begin{itemize}
\item[solve] 
Specifies that an OS server should solve the provided instance and return the result.
CoinOS will wait until the OS server returns the result.
\item[getJobID] 
Specifies that an available Job identifier should be requested from an OS server.
\item[knock] 
Specifies that the status of a solve process is requested from an OS server.
The knock service method requires an OSpL file as input, see option readospl.
\item[kill] 
Specifies that a solve process should be interrupted on an OS server.
You should specify an OSoL file containing the JobID by using the readosil option.
\item[send] 
Specifies that the model instance is send to an OS server and the server should solve this instance.
CoinOS does not wait until the OS server returns the result but returns when submission of the instance is completed.
\item[retrieve] 
Specifies that an optimization result should be requested from an OS server.
You should specify an OSoL file containing the JobID by using the readosil option.
\end{itemize}

\item[\label{solver}\hypertarget{solver}
{\textbf{solver (\slshape{string})}}]\hspace{1.0in}

Specifies the solver that is used to solve an instance.
Valid values are Clp, Cbc, Glpk, Ipopt, Bonmin, Couenne.


\item[\label{readospl}\hypertarget{readospl}
{\textbf{readospl (\slshape{string})}}]\hspace{1.0in}

Specifies the name of an OSpL file to use for the knock method.


\item[\label{writeospl}\hypertarget{writeospl}
{\textbf{writeospl (\slshape{string})}}]\hspace{1.0in}

Specifies the name of an OSpL file in which the answer from the knock or kill method is written.

\end{description}


\bibliographystyle{plain}
%\bibliography{coinlibd}
%\renewcommand{\bibname}{OS References}
\chapter{Optimization Services (OS)}

COIN-OR \OS (\textbf{O}ptimization \textbf{S}ervices) is an initiative to provide a set of standards for representing optimization instances, results, solver options, and communication between clients and solvers in a distributed environment using Web Services.
The code has been written primarily by Horand Gassmann, Jun Ma, and Kipp Martin.
Kipp Martin is the COIN-OR project leader for \OS.

For more information we refer to the web sites \url{http://www.optimizationservices.org} and \url{https://projects.coin-or.org/OS}, the \OS manual~\cite{Ma2005}, and the papers~\cite{FourerMaMartin2009,FourerMaMartin2010,OSManual}.

With the \OS link in \GAMS, you can send your instance to an Optimization Services Server for remote solving.

\OS supports continuous, binary, integer, semicontinuous, and semiinteger variables and linear and nonlinear equations.
Special ordered sets and indicator constraints are currently not supported.
Initial values are currently not supported by the \GAMS/\OS link.

\section{Usage}

The following statement can be used inside your \GAMS program to specify using \OS
\begin{verbatim}
  Option MINLP = OS;     { or LP, RMIP, MIP, DNLP, NLP, RMINLP, QCP, RMIQCP, MIQCP }
\end{verbatim}

The above statement should appear before the Solve statement.

By default, for a given instance of a \GAMS model, nothing happens.
To solve an instance remotely, you have to specify the URL of an Optimization Services Server via the option \texttt{service}.
Usually, the server chooses an appropriate solver for your instance, depending on their availability on the server.
A fully equipped server chooses
\textsc{CLP} for continuous linear models (LP and RMIP),
\IPOPT for continuous nonlinear models (NLP, DNLP, RMINLP, QCP, RMIQCP),
\CBC for mixed-integer linear models (MIP), and
\BONMIN for mixed-integer nonlinear models (MIQCP, MINLP).
An easy way to influence the choice of the solver on the server is the \texttt{solver} option.

Further options can be provided in an OSoL (Optimization Services Options Language) file, which is specified via the the \texttt{readosol} option.
An example OSoL file looks like
\begin{verbatim}
<?xml version="1.0" encoding="UTF-8"?>
<osol xmlns="os.optimizationservices.org" xmlns:xs="http://www.w3.org/2001/XMLSchema"
      xmlns:xsi="http://www.w3.org/2001/XMLSchema-instance"
      xsi:schemaLocation="os.optimizationservices.org
                          http://www.optimizationservices.org/schemas/2.0/OSoL.xsd">
<optimization>
  <solverOptions numberOfSolverOptions="3">
    <solverOption name="cuts" solver="cbc" value="off" />
    <solverOption name="max_active_nodes" solver="symphony"  value="2" />
    <solverOption name="max_iter" solver="ipopt" type="integer" value="2000"/>
  </solverOptions>
</optimization>
</osol>
\end{verbatim}
It specifies that if \CBC is used, then cutting planes are disabled,
if \textsc{SYMPHONY} is used, then at most 2 nodes should be active,
and if \IPOPT is used, then a limit of 2000 iterations is imposed.

By default, the call to the server is a \emph{synchronous} call.
The \GAMS process will wait for the result and then display the result.
This may not be desirable when solving large optimization models.
In order to use the remote solver service in an \emph{asynchronous} fashion, one can make use of the \GAMS Grid Computing Facility, see Appendix I in the \GAMS manual.

\section{Detailed Options Descriptions}

\subsection{Detailed Descriptions of CoinOS Options}

\begin{description}

\item[\label{readosol}\hypertarget{readosol}
{\textbf{readosol (\slshape{string})}}]\hspace{1.0in}

Specifies the name of an option file in OSoL format that is given to the solver or an OS server.


\item[\label{writeosil}\hypertarget{writeosil}
{\textbf{writeosil (\slshape{string})}}]\hspace{1.0in}

Specifies the name of a file in which the GAMS model instance should be writting in OSiL format.


\item[\label{writeosrl}\hypertarget{writeosrl}
{\textbf{writeosrl (\slshape{string})}}]\hspace{1.0in}

Specifies the name of a file in which the result of a solve process (solution, status, ...) should be writting in OSrL format.


\item[\label{service}\hypertarget{service}
{\textbf{service (\slshape{string})}}]\hspace{1.0in}

Specifies the URL of an Optimization Services Server.
If specified, then this server is contacted and the service method specified under service\_method is executed.
Note that by default the server executes CBC to solve an instance.
You can change the solver with the solver option.


\item[\label{service_method}\hypertarget{service_method}
{\textbf{service\_method (\slshape{string})}}]\hspace{1.0in}

Specifies the method to execute on a server.

\textsl{(default = solve)}
\begin{itemize}
\item[solve] 
Specifies that an OS server should solve the provided instance and return the result.
CoinOS will wait until the OS server returns the result.
\item[getJobID] 
Specifies that an available Job identifier should be requested from an OS server.
\item[knock] 
Specifies that the status of a solve process is requested from an OS server.
The knock service method requires an OSpL file as input, see option readospl.
\item[kill] 
Specifies that a solve process should be interrupted on an OS server.
You should specify an OSoL file containing the JobID by using the readosil option.
\item[send] 
Specifies that the model instance is send to an OS server and the server should solve this instance.
CoinOS does not wait until the OS server returns the result but returns when submission of the instance is completed.
\item[retrieve] 
Specifies that an optimization result should be requested from an OS server.
You should specify an OSoL file containing the JobID by using the readosil option.
\end{itemize}

\item[\label{solver}\hypertarget{solver}
{\textbf{solver (\slshape{string})}}]\hspace{1.0in}

Specifies the solver that is used to solve an instance.
Valid values are Clp, Cbc, Glpk, Ipopt, Bonmin, Couenne.


\item[\label{readospl}\hypertarget{readospl}
{\textbf{readospl (\slshape{string})}}]\hspace{1.0in}

Specifies the name of an OSpL file to use for the knock method.


\item[\label{writeospl}\hypertarget{writeospl}
{\textbf{writeospl (\slshape{string})}}]\hspace{1.0in}

Specifies the name of an OSpL file in which the answer from the knock or kill method is written.

\end{description}


\bibliographystyle{plain}
%\bibliography{coinlibd}
%\renewcommand{\bibname}{OS References}
\chapter{Optimization Services (OS)}

COIN-OR \OS (\textbf{O}ptimization \textbf{S}ervices) is an initiative to provide a set of standards for representing optimization instances, results, solver options, and communication between clients and solvers in a distributed environment using Web Services.
The code has been written primarily by Horand Gassmann, Jun Ma, and Kipp Martin.
Kipp Martin is the COIN-OR project leader for \OS.

For more information we refer to the web sites \url{http://www.optimizationservices.org} and \url{https://projects.coin-or.org/OS}, the \OS manual~\cite{Ma2005}, and the papers~\cite{FourerMaMartin2009,FourerMaMartin2010,OSManual}.

With the \OS link in \GAMS, you can send your instance to an Optimization Services Server for remote solving.

\OS supports continuous, binary, integer, semicontinuous, and semiinteger variables and linear and nonlinear equations.
Special ordered sets and indicator constraints are currently not supported.
Initial values are currently not supported by the \GAMS/\OS link.

\section{Usage}

The following statement can be used inside your \GAMS program to specify using \OS
\begin{verbatim}
  Option MINLP = OS;     { or LP, RMIP, MIP, DNLP, NLP, RMINLP, QCP, RMIQCP, MIQCP }
\end{verbatim}

The above statement should appear before the Solve statement.

By default, for a given instance of a \GAMS model, nothing happens.
To solve an instance remotely, you have to specify the URL of an Optimization Services Server via the option \texttt{service}.
Usually, the server chooses an appropriate solver for your instance, depending on their availability on the server.
A fully equipped server chooses
\textsc{CLP} for continuous linear models (LP and RMIP),
\IPOPT for continuous nonlinear models (NLP, DNLP, RMINLP, QCP, RMIQCP),
\CBC for mixed-integer linear models (MIP), and
\BONMIN for mixed-integer nonlinear models (MIQCP, MINLP).
An easy way to influence the choice of the solver on the server is the \texttt{solver} option.

Further options can be provided in an OSoL (Optimization Services Options Language) file, which is specified via the the \texttt{readosol} option.
An example OSoL file looks like
\begin{verbatim}
<?xml version="1.0" encoding="UTF-8"?>
<osol xmlns="os.optimizationservices.org" xmlns:xs="http://www.w3.org/2001/XMLSchema"
      xmlns:xsi="http://www.w3.org/2001/XMLSchema-instance"
      xsi:schemaLocation="os.optimizationservices.org
                          http://www.optimizationservices.org/schemas/2.0/OSoL.xsd">
<optimization>
  <solverOptions numberOfSolverOptions="3">
    <solverOption name="cuts" solver="cbc" value="off" />
    <solverOption name="max_active_nodes" solver="symphony"  value="2" />
    <solverOption name="max_iter" solver="ipopt" type="integer" value="2000"/>
  </solverOptions>
</optimization>
</osol>
\end{verbatim}
It specifies that if \CBC is used, then cutting planes are disabled,
if \textsc{SYMPHONY} is used, then at most 2 nodes should be active,
and if \IPOPT is used, then a limit of 2000 iterations is imposed.

By default, the call to the server is a \emph{synchronous} call.
The \GAMS process will wait for the result and then display the result.
This may not be desirable when solving large optimization models.
In order to use the remote solver service in an \emph{asynchronous} fashion, one can make use of the \GAMS Grid Computing Facility, see Appendix I in the \GAMS manual.

\section{Detailed Options Descriptions}

\subsection{Detailed Descriptions of CoinOS Options}

\begin{description}

\item[\label{readosol}\hypertarget{readosol}
{\textbf{readosol (\slshape{string})}}]\hspace{1.0in}

Specifies the name of an option file in OSoL format that is given to the solver or an OS server.


\item[\label{writeosil}\hypertarget{writeosil}
{\textbf{writeosil (\slshape{string})}}]\hspace{1.0in}

Specifies the name of a file in which the GAMS model instance should be writting in OSiL format.


\item[\label{writeosrl}\hypertarget{writeosrl}
{\textbf{writeosrl (\slshape{string})}}]\hspace{1.0in}

Specifies the name of a file in which the result of a solve process (solution, status, ...) should be writting in OSrL format.


\item[\label{service}\hypertarget{service}
{\textbf{service (\slshape{string})}}]\hspace{1.0in}

Specifies the URL of an Optimization Services Server.
If specified, then this server is contacted and the service method specified under service\_method is executed.
Note that by default the server executes CBC to solve an instance.
You can change the solver with the solver option.


\item[\label{service_method}\hypertarget{service_method}
{\textbf{service\_method (\slshape{string})}}]\hspace{1.0in}

Specifies the method to execute on a server.

\textsl{(default = solve)}
\begin{itemize}
\item[solve] 
Specifies that an OS server should solve the provided instance and return the result.
CoinOS will wait until the OS server returns the result.
\item[getJobID] 
Specifies that an available Job identifier should be requested from an OS server.
\item[knock] 
Specifies that the status of a solve process is requested from an OS server.
The knock service method requires an OSpL file as input, see option readospl.
\item[kill] 
Specifies that a solve process should be interrupted on an OS server.
You should specify an OSoL file containing the JobID by using the readosil option.
\item[send] 
Specifies that the model instance is send to an OS server and the server should solve this instance.
CoinOS does not wait until the OS server returns the result but returns when submission of the instance is completed.
\item[retrieve] 
Specifies that an optimization result should be requested from an OS server.
You should specify an OSoL file containing the JobID by using the readosil option.
\end{itemize}

\item[\label{solver}\hypertarget{solver}
{\textbf{solver (\slshape{string})}}]\hspace{1.0in}

Specifies the solver that is used to solve an instance.
Valid values are Clp, Cbc, Glpk, Ipopt, Bonmin, Couenne.


\item[\label{readospl}\hypertarget{readospl}
{\textbf{readospl (\slshape{string})}}]\hspace{1.0in}

Specifies the name of an OSpL file to use for the knock method.


\item[\label{writeospl}\hypertarget{writeospl}
{\textbf{writeospl (\slshape{string})}}]\hspace{1.0in}

Specifies the name of an OSpL file in which the answer from the knock or kill method is written.

\end{description}


\bibliographystyle{plain}
%\bibliography{coinlibd}
%\renewcommand{\bibname}{OS References}
\chapter{Optimization Services (OS)}

COIN-OR \OS (\textbf{O}ptimization \textbf{S}ervices) is an initiative to provide a set of standards for representing optimization instances, results, solver options, and communication between clients and solvers in a distributed environment using Web Services.
The code has been written primarily by Horand Gassmann, Jun Ma, and Kipp Martin.
Kipp Martin is the COIN-OR project leader for \OS.

For more information we refer to the web sites \url{http://www.optimizationservices.org} and \url{https://projects.coin-or.org/OS}, the \OS manual~\cite{Ma2005}, and the papers~\cite{FourerMaMartin2009,FourerMaMartin2010,OSManual}.

With the \OS link in \GAMS, you can send your instance to an Optimization Services Server for remote solving.

\OS supports continuous, binary, integer, semicontinuous, and semiinteger variables and linear and nonlinear equations.
Special ordered sets and indicator constraints are currently not supported.
Initial values are currently not supported by the \GAMS/\OS link.

\section{Usage}

The following statement can be used inside your \GAMS program to specify using \OS
\begin{verbatim}
  Option MINLP = OS;     { or LP, RMIP, MIP, DNLP, NLP, RMINLP, QCP, RMIQCP, MIQCP }
\end{verbatim}

The above statement should appear before the Solve statement.

By default, for a given instance of a \GAMS model, nothing happens.
To solve an instance remotely, you have to specify the URL of an Optimization Services Server via the option \texttt{service}.
Usually, the server chooses an appropriate solver for your instance, depending on their availability on the server.
A fully equipped server chooses
\textsc{CLP} for continuous linear models (LP and RMIP),
\IPOPT for continuous nonlinear models (NLP, DNLP, RMINLP, QCP, RMIQCP),
\CBC for mixed-integer linear models (MIP), and
\BONMIN for mixed-integer nonlinear models (MIQCP, MINLP).
An easy way to influence the choice of the solver on the server is the \texttt{solver} option.

Further options can be provided in an OSoL (Optimization Services Options Language) file, which is specified via the the \texttt{readosol} option.
An example OSoL file looks like
\begin{verbatim}
<?xml version="1.0" encoding="UTF-8"?>
<osol xmlns="os.optimizationservices.org" xmlns:xs="http://www.w3.org/2001/XMLSchema"
      xmlns:xsi="http://www.w3.org/2001/XMLSchema-instance"
      xsi:schemaLocation="os.optimizationservices.org
                          http://www.optimizationservices.org/schemas/2.0/OSoL.xsd">
<optimization>
  <solverOptions numberOfSolverOptions="3">
    <solverOption name="cuts" solver="cbc" value="off" />
    <solverOption name="max_active_nodes" solver="symphony"  value="2" />
    <solverOption name="max_iter" solver="ipopt" type="integer" value="2000"/>
  </solverOptions>
</optimization>
</osol>
\end{verbatim}
It specifies that if \CBC is used, then cutting planes are disabled,
if \textsc{SYMPHONY} is used, then at most 2 nodes should be active,
and if \IPOPT is used, then a limit of 2000 iterations is imposed.

By default, the call to the server is a \emph{synchronous} call.
The \GAMS process will wait for the result and then display the result.
This may not be desirable when solving large optimization models.
In order to use the remote solver service in an \emph{asynchronous} fashion, one can make use of the \GAMS Grid Computing Facility, see Appendix I in the \GAMS manual.

\section{Detailed Options Descriptions}

\input{optosd_a}

\bibliographystyle{plain}
%\bibliography{coinlibd}
%\renewcommand{\bibname}{OS References}
\input{os.bbl}

\chapterend


\chapterend


\chapterend


\chapterend

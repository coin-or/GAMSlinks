\begin{description}

\item[\label{glpkstartalg}\hypertarget{glpkstartalg}
{\textbf{startalg (\slshape{string})}}]\hspace{1.0in}

This option determines whether a primal or dual simplex algorithm should be used to solve an LP or the root node of a MIP.

\textsl{(default = primal)}
\begin{itemize}
\item[primal] 
Let GLPK use a primal simplex algorithm.
\item[dual] 
Let GLPK use a dual simplex algorithm.
\end{itemize}

\item[\label{glpkscaling}\hypertarget{glpkscaling}
{\textbf{scaling (\slshape{string})}}]\hspace{1.0in}

This option determines the method how the constraint matrix is scaled.
Note that scaling is only applied when the \hyperlink{presolve}{presolver} is turned off, which is on by default.

\textsl{(default = meanequilibrium)}
\begin{itemize}
\item[off] 
Turn off scaling.
\item[equilibrium] 
Let GLPK use an equilibrium scaling method.
\item[mean] 
Let GLPK use a geometric mean scaling method.
\item[meanequilibrium] 
Let GLPK use first a geometric mean scaling, then an equilibrium scaling.
\end{itemize}

\item[\label{pricing}\hypertarget{pricing}
{\textbf{pricing (\slshape{string})}}]\hspace{1.0in}

Sets the pricing method for both primal and dual simplex.

\textsl{(default = steepestedge)}
\begin{itemize}
\item[textbook] 
Use a textbook pricing rule.
\item[steepestedge] 
Use a steepest edge pricing rule.
\end{itemize}

\item[\label{factorization}\hypertarget{factorization}
{\textbf{factorization (\slshape{string})}}]\hspace{1.0in}

Sets the method for the LP basis factorization.

If you observe poor performce, then you may try setting the factorization method to forresttomlin.

\textsl{(default = givens)}
\begin{itemize}
\item[forresttomlin] 
Does a LU factorization followed by Forrest-Tomlin updates.
This method is fast, but less stable than others.
\item[bartelsgolub] 
Does a LU factorization followed by a Schur complement and Bartels-Golub updates.
This method is slower than Forrest-Tomlin, but more stable.
\item[givens] 
Does a LU factorization followed by a Schur complement and Givens rotation updates.
This method is slower than Forrest-Tomlin, but more stable.
\end{itemize}

% \item[\label{initbasis}\hypertarget{initbasis}
% {\textbf{initbasis (\slshape{string})}}]\hspace{1.0in}
% 
% Sets the method that computes the initial basis.
% Setting this option has only effect if the \hyperlink{presolve}{presolver} is turned off, which is on by default.
% 
% \textsl{(default = advanced)}
% \begin{itemize}
% \item[standard] 
% Uses the standard initial basis of all slacks.
% \item[advanced] 
% Computes an advanced initial basis.
% \item[bixby] 
% Uses Bixby's initial basis.
% \item[user] 
% Uses basis provided by GAMS.
% Note that a feasible basis need to be given, otherwise Glpk will fail.
% \end{itemize}

\item[\label{glpktol_dual}\hypertarget{glpktol_dual}
{\textbf{tol\_dual (\slshape{real})}}]\hspace{1.0in}

Absolute tolerance used to check if the current basis solution is dual feasible.
% (Glpk manual: Do not change this parameter without detailed understanding its purpose.)

\textsl{(default = 1e-7)}

\item[\label{glpktol_primal}\hypertarget{glpktol_primal}
{\textbf{tol\_primal (\slshape{real})}}]\hspace{1.0in}

Relative tolerance used to check if the current basis solution is primal feasible.
% (Glpk manual: Do not change this parameter without detailed understanding its purpose.)

\textsl{(default = 1e-7)}

\item[\label{glpktol_integer}\hypertarget{glpktol_integer}
{\textbf{tol\_integer (\slshape{real})}}]\hspace{1.0in}

Absolute tolerance used to check if the current basis solution is integer feasible.
% (Glpk manual: Do not change this parameter without detailed understanding its purpose.)

\textsl{(default = 1e-5)}

\item[\label{backtracking}\hypertarget{backtracking}
{\textbf{backtracking (\slshape{string})}}]\hspace{1.0in}

Determines which method to use for the backtracking heuristic.

\textsl{(default = bestprojection)}
\begin{itemize}
\item[depthfirst] 
Let GLPK use a depth first search.
\item[breadthfirst] 
Let GLPK use a breadth first search.
\item[bestprojection] 
Let GLPK use a best projection heuristic.
\end{itemize}

\item[\label{glpkpresolve}\hypertarget{glpkpresolve}
{\textbf{presolve (\slshape{integer})}}]\hspace{1.0in}

Determines whether the LP presolver should be used.

\textsl{(default = 1)}
\begin{itemize}
\item[0] 
Turns off the LP presolver.
\item[1] 
Turns on the LP presolver.
\end{itemize}

\item[\label{glpkcuts}\hypertarget{glpkcuts}
{\textbf{cuts (\slshape{integer})}}]\hspace{1.0in}

Determines which cuts generator to use: none, all, or user-defined

\textsl{(default = 0)}
\begin{itemize}
\item[-1] 
Turn off all cut generators
\item[0] 
Turn on or off each cut generators separately
\item[1] 
Turn on all cut generators
\end{itemize}

\item[\label{glpkcovercuts}\hypertarget{glpkcovercuts}
{\textbf{covercuts (\slshape{integer})}}]\hspace{1.0in}

Whether to enable cover cuts.

\textsl{(default = 1)}
\begin{itemize}
\item[0] 
Turn off cover cuts
\item[1] 
Turn on cover cuts
\end{itemize}

\item[\label{glpkcliquecuts}\hypertarget{glpkcliquecuts}
{\textbf{cliquecuts (\slshape{integer})}}]\hspace{1.0in}

Whether to enable clique cuts.

\textsl{(default = 1)}
\begin{itemize}
\item[0] 
Turn off clique cuts
\item[1] 
Turn on clique cuts
\end{itemize}

\item[\label{glpkgomorycuts}\hypertarget{glpkgomorycuts}
{\textbf{gomorycuts (\slshape{integer})}}]\hspace{1.0in}

Whether to enable Gomorys mixed-integer linear cuts.

\textsl{(default = 1)}
\begin{itemize}
\item[0] 
Turn off gomory cuts
\item[1] 
Turn on gomory cuts
\end{itemize}

\item[\label{glpkmircuts}\hypertarget{glpkmircuts}
{\textbf{mircuts (\slshape{integer})}}]\hspace{1.0in}

Whether to enable mixed-integer rounding cuts.

\textsl{(default = 0)}
\begin{itemize}
\item[0] 
Turn off mir cuts
\item[1] 
Turn on mir cuts
\end{itemize}

\item[\label{glpkreslim}\hypertarget{glpkreslim}
{\textbf{reslim (\slshape{real})}}]\hspace{1.0in}

Maximum time in seconds.

\textsl{(default = GAMS reslim)}

\item[\label{glpkiterlim}\hypertarget{glpkiterlim}
{\textbf{iterlim (\slshape{integer})}}]\hspace{1.0in}

Maximum number of simplex iterations.

\textsl{(default = GAMS iterlim)}

\item[\label{glpkoptcr}\hypertarget{glpkoptcr}
{\textbf{optcr (\slshape{real})}}]\hspace{1.0in}

Relative optimality criterion for a MIP.
The search is stoped when the relative gap between the incumbent and the bound given by the LP relaxation is smaller than this value.

\textsl{(default = GAMS optcr)}

\end{description}

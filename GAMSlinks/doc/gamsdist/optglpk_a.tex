\subsection{Detailed Descriptions of CoinGlpk Options}

\begin{description}

\item[\label{glpkwritemps}\hypertarget{glpkwritemps}
{\textbf{writemps (\slshape{string})}}]\hspace{1.0in}

Write an MPS problem file.
The parameter value is the name of the MPS file.


\item[\label{glpkstartalg}\hypertarget{glpkstartalg}
{\textbf{startalg (\slshape{string})}}]\hspace{1.0in}

This option determines whether a primal or dual simplex algorithm should be used to solve the root node.

\textsl{(default = primal)}
\begin{itemize}
\item[primal] use the primal simplex algorithm for the root node
\item[dual] use the dual simplex algorithm for the root node
\end{itemize}

\item[\label{glpkscaling}\hypertarget{glpkscaling}
{\textbf{scaling (\slshape{string})}}]\hspace{1.0in}

This option determines the method how the constraint matrix is scaled.

\textsl{(default = equilibrium)}
\begin{tabbing}
\hspace{1.1in} \= \\
off \> no scaling \\
equilibrium \> equilibrium scaling \\
mean \> geometric mean scaling \\
meanequilibrium \> geometric mean scaling then equilibrium scaling
\end{tabbing}

\item[\label{pricing}\hypertarget{pricing}
{\textbf{pricing (\slshape{string})}}]\hspace{1.0in}

Sets the pricing method for both primal and dual simplex.

\textsl{(default = textbook)}
\begin{tabbing}
\hspace{1in} \= \\
textbook \> textbook pricing \\
steepestedge \> steepest edge pricing
\end{tabbing}

\item[\label{glpktol_dual}\hypertarget{glpktol_dual}
{\textbf{tol\_dual (\slshape{real})}}]\hspace{1.0in}

Absolute tolerance used to check if the current basis solution is dual feasible.
(Glpk manual: Do not change this parameter without detailed understanding its purpose.)

\textsl{(default = 1e-7)}

\item[\label{glpktol_primal}\hypertarget{glpktol_primal}
{\textbf{tol\_primal (\slshape{real})}}]\hspace{1.0in}

Relative tolerance used to check if the current basis solution is primal feasible.
(Glpk manual: Do not change this parameter without detailed understanding its purpose.)

\textsl{(default = 1e-7)}

\item[\label{glpktol_integer}\hypertarget{glpktol_integer}
{\textbf{tol\_integer (\slshape{real})}}]\hspace{1.0in}

Absolute tolerance used to check if the current basis solution is integer feasible.
(Glpk manual: Do not change this parameter without detailed understanding its purpose.)

\textsl{(default = 1e-5)}

\item[\label{backtracking}\hypertarget{backtracking}
{\textbf{backtracking (\slshape{string})}}]\hspace{1.0in}

Determines which method to use for the backtracking heuristic.

\textsl{(default = bestprojection)}
\begin{tabbing}
\hspace{1in} \= \\
depthfirst \> depth first search \\
breadthfirst \> breadth first search \\
bestprojection \> using best projection heuristic
\end{tabbing}

\item[\label{glpkcuts}\hypertarget{glpkcuts}
{\textbf{cuts (\slshape{integer})}}]\hspace{1.0in}

Regulates the use of cut generator routines for the root problem.\\
If set to automatic, then the parameters for the single cut generators determine which are added.

\textsl{(default = 0)}
\begin{itemize}
\item[-1] no cuts will be generated
\item[0] automatic
\item[1] cuts from all available cut classes will be generated
\end{itemize}

\item[\label{glpkgomorycuts}\hypertarget{glpkgomorycuts}
{\textbf{gomorycuts (\slshape{integer})}}]\hspace{1.0in}

This options switches the generation of mixed integer Gomory Cuts on.

\textsl{(default = 1)}
\begin{itemize}
\item[0] don't add gomory cuts
\item[1] add gomory cuts
\end{itemize}


\item[\label{glpkcliquecuts}\hypertarget{glpkcliquecuts}
{\textbf{cliquecuts (\slshape{integer})}}]\hspace{1.0in}

This options switches the generation of clique cuts on.

\textsl{(default = 1)}
\begin{itemize}
\item[0] don't add clique cuts
\item[1] add clique cuts
\end{itemize}


\item[\label{glpkcovercuts}\hypertarget{glpkcovercuts}
{\textbf{covercuts (\slshape{integer})}}]\hspace{1.0in}

This options switches the generation of mixed cover cuts on.

\textsl{(default = 1)}
\begin{itemize}
\item[0] don't add mixed cover cuts
\item[1] add mixed cover cuts
\end{itemize}


\item[\label{reslim_fixedrun}\hypertarget{reslim_fixedrun}
{\textbf{reslim\_fixedrun (\slshape{real})}}]\hspace{1.0in}

Maximum time in seconds for solving the MIP with fixed discrete variables.

\textsl{(default = 1000)}

\item[\label{glpkreslim}\hypertarget{glpkreslim}
{\textbf{reslim (\slshape{real})}}]\hspace{1.0in}

Maximum time in seconds.

\textsl{(default = GAMS reslim)}

\item[\label{glpkiterlim}\hypertarget{glpkiterlim}
{\textbf{iterlim (\slshape{integer})}}]\hspace{1.0in}

Maximum number of iterations.

\textsl{(default = GAMS iterlim)}

\item[\label{glpkoptcr}\hypertarget{glpkoptcr}
{\textbf{optcr (\slshape{real})}}]\hspace{1.0in}

Relative optimality criterion for a MIP.
Glpk uses this parameter as relative tolerance to check if the value of the objective function is not better than in the best known integer feasible solution.

\textsl{(default = GAMS optcr)}
\end{description}

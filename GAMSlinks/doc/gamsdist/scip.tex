
\newcommand{\MYGAMS}{\textsc{GAMS}\xspace}
\newcommand{\SCIP}{\textsc{SCIP}\xspace}

% formating option print
% \printoption{option name}{range of values}{default value}{description}{enum description}
\newcommand{\printoption}[5]%
{\textbf{\ttfamily #1}\; (#2) \hfill \texttt{#3}\\
#4
#5
}
\newcommand{\printoptioncategory}[1]{\smallskip

\textbf{#1}

}

\chapter{SCIP}
\textbf{Stefan Vigerske, Humboldt University Berlin, Germany}
\vspace{1cm}

\minitoc

\section{Introduction}

\SCIP (\textbf{S}olving \textbf{C}onstraint \textbf{I}nteger \textbf{P}rograms) is developed at the Konrad-Zuse-Zentrum f\"ur Informationstechnik Berlin (ZIB).
The \SCIP main developer had been Tobias Achterberg, current developers are Timo Berthold, Gerald Gamrath, Stefan Heinz, Thorsten Koch, Stefan Vigerske, Robert Waniek, Michael Winkler, and Kati Wolter.
Since \SCIP is distributed under the ZIB Academic License, it is only available for users with a \MYGAMS academic license.

\SCIP is a framework for Constraint Integer Programming oriented towards the needs of Mathematical Programming experts who want to have total control of the solution process and access detailed information down to the guts of the solver.
\SCIP can also be used as a pure MIP solver or as a framework for branch-cut-and-price.
Within \MYGAMS, the MIP and MIQCP solving facilities of \SCIP are available.

For more detailed information, we refer to \cite{Achterberg2007,Ach09,AchBeKoWo08,Berthold2006,BertholdHeinzVigerske2009,Wolter2006} and the \SCIP web site \url{http://scip.zib.de}.

\MYGAMS/\SCIP uses the COIN-OR linear solver \textsc{CLP} as LP solver and the COIN-OR Interior Point Optimizer \textsc{Ipopt}~\cite{WaBi06} as nonlinear solver.

\section{Model requirements}

\SCIP supports continuous, binary, and integer variables, special ordered sets, and branching priorities.
Semi-continuous or semi-integer variables (see chapter 17.1 of the \MYGAMS User's Guide) and indicator constraints are not supported yet.

\section{Usage}

The following statement can be used inside your \MYGAMS program to specify using \SCIP
\begin{verbatim}
  Option MIP = SCIP;     { or LP or RMIP or QCP or RMIQCP or MIQCP }
\end{verbatim}

The above statement should appear before the Solve statement.
If \SCIP was specified as the default solver during \MYGAMS installation, the above statement is not necessary.

If a continuous linear program (LP or RMIP) is given to \MYGAMS/\SCIP, the linear programming solver is called directly.
In this case, it is not possible to provide an option file.

% \MYGAMS/\SCIP supports the \MYGAMS Branch-and-Cut-and-Heuristic (BCH) Facility.
% The \MYGAMS BCH facility automates all major steps necessary to define, execute, and control the use of user defined routines within the framework of general purpose MIP codes.
% Currently supported are user defined cut generators and heuristics and the incumbent reporting callback.
% Please see the BCH documentation at \texttt{http://www.gams.com/docs/bch.htm} for further information.
% 
% Information on the use of BCH callback routines is displayed in an extra column in the \SCIP iteration output.
% The first number in this column (below the ``BCH'' in the header) is the number of callbacks to \MYGAMS models that have been made so far (accumulated from cutgeneration, heuristic, and incumbent callbacks).
% The number below ``cut'' or ``cuts'' gives the number of cutting planes that have been generated by the users cutgenerator.
% Finally, the number below ``sol'' or ``sols'' gives the number of primal solutions that have been generated by the users heuristic.
% If \SCIP accepts a heuristic solution as new incumbent solution, it prints a `G' in the first column of the iteration output.

\MYGAMS/\SCIP currently does not support the \MYGAMS Branch-and-Cut-and-Heuristic (BCH) Facility.
If you need to use \MYGAMS/\SCIP with BCH, please consider to use a \MYGAMS system of version $\leq 23.3$, available at \url{http://www.gams.com/download/download_old.htm}.

\subsubsection{Specification of \SCIP Options}

\MYGAMS/\SCIP currently supports the \MYGAMS parameters \texttt{reslim}, \texttt{iterlim}, \texttt{nodlim}, \texttt{optcr}, and \texttt{optca}.

For a MIP, QCP, or MIQCP solve the user can specify options by a \SCIP options file.
A \SCIP options file consists of one option or comment per line.
A pound sign (\#) at the beginning of a line causes the entire line to be ignored.
Otherwise, the line will be interpreted as an option name and value separated by an equal sign (=) and any amount of white space (blanks or tabs).
Further, string values have to be enclosed in quotation marks.

A small example for a scip.opt file is:
\begin{verbatim}
  presolving/probing/maxrounds = 0
  separating/maxrounds         = 0
  separating/maxroundsroot     = 0
\end{verbatim}
%   gams/usercutcall         = "bchcutgen.gms"
It causes \MYGAMS/\SCIP to disable probing during presolve and to turn off all cut generators.
% and to use a user defined cut generator.

\section{Detailed Options Description}

\SCIP supports a large set of options.
Sample option files can be obtained from \\ \url{http://www.gams.com/~svigerske/scip1.2}.

% Further, there is a set of options that are specific to the \MYGAMS/\SCIP interface.
% , most of them for control of the \MYGAMS BCH facility.

In the following we give a detailed list of \SCIP options.

\printoptioncategory{GAMS interface specific options}
\printoption{gams/dumpsolutions}%
{string}%
{}%
{name of solutions index gdx file for writing all solutions}%
{}

\printoption{gams/indicatorfile}%
{string}%
{}%
{name of GAMS options file that contains definitions on indicators}%
{}

\printoption{gams/interactive}%
{string}%
{}%
{command to be issued to the SCIP shell instead of issuing a solve command}%
{}

\printoption{gams/mipstart}%
{boolean}%
{TRUE}%
{whether to try GAMS variable level values as initial primal solution}%
{}

\printoption{gams/printstatistics}%
{boolean}%
{FALSE}%
{whether to print statistics on a MIP solve}%
{}

\printoption{gams/resolvenlp}%
{boolean}%
{TRUE}%
{whether to resolve MINLP with fixed discrete variables if best solution violates some constraints}%
{}

\printoption{gams/solvetrace/file}%
{string}%
{}%
{name of file where to write branch-and-bound trace information too}%
{}

\printoption{gams/solvetrace/nodefreq}%
{$0\leq\textrm{integer}$}%
{$100$}%
{frequency in number of nodes when to write branch-and-bound trace information, 0 to disable}%
{}

\printoption{gams/solvetrace/timefreq}%
{$0\leq\textrm{real}$}%
{$5$}%
{frequency in seconds when to write branch-and-bound trace information, 0.0 to disable}%
{}

\printoptioncategory{Branching}
\printoption{branching/allfullstrong/maxbounddist}%
{$0\leq\textrm{real}\leq1$}%
{$1$}%
{maximal relative distance from current node's dual bound to primal bound compared to best node's dual bound for applying branching rule (0.0: only on current best node, 1.0: on all nodes)}%
{}

\printoption{branching/allfullstrong/maxdepth}%
{$-1\leq\textrm{integer}$}%
{$-1$}%
{maximal depth level, up to which branching rule $<$allfullstrong$>$ should be used (-1 for no limit)}%
{}

\printoption{branching/allfullstrong/priority}%
{$-536870912\leq\textrm{integer}\leq536870911$}%
{$-1000$}%
{priority of branching rule $<$allfullstrong$>$}%
{}

\printoption{branching/clamp}%
{$0\leq\textrm{real}\leq0.5$}%
{$0.2$}%
{minimal relative distance of branching point to bounds when branching on a continuous variable}%
{}

\printoption{branching/delaypscostupdate}%
{boolean}%
{TRUE}%
{should updating pseudo costs for continuous variables be delayed to the time after separation?}%
{}

\printoption{branching/fullstrong/maxbounddist}%
{$0\leq\textrm{real}\leq1$}%
{$1$}%
{maximal relative distance from current node's dual bound to primal bound compared to best node's dual bound for applying branching rule (0.0: only on current best node, 1.0: on all nodes)}%
{}

\printoption{branching/fullstrong/maxdepth}%
{$-1\leq\textrm{integer}$}%
{$-1$}%
{maximal depth level, up to which branching rule $<$fullstrong$>$ should be used (-1 for no limit)}%
{}

\printoption{branching/fullstrong/priority}%
{$-536870912\leq\textrm{integer}\leq536870911$}%
{$0$}%
{priority of branching rule $<$fullstrong$>$}%
{}

\printoption{branching/inference/maxbounddist}%
{$0\leq\textrm{real}\leq1$}%
{$1$}%
{maximal relative distance from current node's dual bound to primal bound compared to best node's dual bound for applying branching rule (0.0: only on current best node, 1.0: on all nodes)}%
{}

\printoption{branching/inference/maxdepth}%
{$-1\leq\textrm{integer}$}%
{$-1$}%
{maximal depth level, up to which branching rule $<$inference$>$ should be used (-1 for no limit)}%
{}

\printoption{branching/inference/priority}%
{$-536870912\leq\textrm{integer}\leq536870911$}%
{$1000$}%
{priority of branching rule $<$inference$>$}%
{}

\printoption{branching/inference/useweightedsum}%
{boolean}%
{TRUE}%
{should a weighted sum of inference, conflict and cutoff weights be used?}%
{}

\printoption{branching/leastinf/maxbounddist}%
{$0\leq\textrm{real}\leq1$}%
{$1$}%
{maximal relative distance from current node's dual bound to primal bound compared to best node's dual bound for applying branching rule (0.0: only on current best node, 1.0: on all nodes)}%
{}

\printoption{branching/leastinf/maxdepth}%
{$-1\leq\textrm{integer}$}%
{$-1$}%
{maximal depth level, up to which branching rule $<$leastinf$>$ should be used (-1 for no limit)}%
{}

\printoption{branching/leastinf/priority}%
{$-536870912\leq\textrm{integer}\leq536870911$}%
{$50$}%
{priority of branching rule $<$leastinf$>$}%
{}

\printoption{branching/lpgainnormalize}%
{character}%
{s}%
{strategy for normalization of LP gain when updating pseudocosts of continuous variables (divide by movement of 'l'p value, reduction in 'd'omain width, or reduction in domain width of 's'ibling)}%
{}

\printoption{branching/mostinf/maxbounddist}%
{$0\leq\textrm{real}\leq1$}%
{$1$}%
{maximal relative distance from current node's dual bound to primal bound compared to best node's dual bound for applying branching rule (0.0: only on current best node, 1.0: on all nodes)}%
{}

\printoption{branching/mostinf/maxdepth}%
{$-1\leq\textrm{integer}$}%
{$-1$}%
{maximal depth level, up to which branching rule $<$mostinf$>$ should be used (-1 for no limit)}%
{}

\printoption{branching/mostinf/priority}%
{$-536870912\leq\textrm{integer}\leq536870911$}%
{$100$}%
{priority of branching rule $<$mostinf$>$}%
{}

\printoption{branching/preferbinary}%
{boolean}%
{FALSE}%
{should branching on binary variables be preferred?}%
{}

\printoption{branching/pscost/maxbounddist}%
{$0\leq\textrm{real}\leq1$}%
{$1$}%
{maximal relative distance from current node's dual bound to primal bound compared to best node's dual bound for applying branching rule (0.0: only on current best node, 1.0: on all nodes)}%
{}

\printoption{branching/pscost/maxdepth}%
{$-1\leq\textrm{integer}$}%
{$-1$}%
{maximal depth level, up to which branching rule $<$pscost$>$ should be used (-1 for no limit)}%
{}

\printoption{branching/pscost/narymaxdepth}%
{$-1\leq\textrm{integer}$}%
{$-1$}%
{maximal depth where to do n-ary branching, -1 to turn off}%
{}

\printoption{branching/pscost/naryminwidth}%
{$0\leq\textrm{real}\leq1$}%
{$0.001$}%
{minimal domain width in children when doing n-ary branching, relative to global bounds}%
{}

\printoption{branching/pscost/narywidthfactor}%
{$1\leq\textrm{real}$}%
{$2$}%
{factor of domain width in n-ary branching when creating nodes with increasing distance from branching value}%
{}

\printoption{branching/pscost/nchildren}%
{$2\leq\textrm{integer}$}%
{$2$}%
{number of children to create in n-ary branching}%
{}

\printoption{branching/pscost/priority}%
{$-536870912\leq\textrm{integer}\leq536870911$}%
{$2000$}%
{priority of branching rule $<$pscost$>$}%
{}

\printoption{branching/pscost/strategy}%
{character}%
{u}%
{strategy for utilizing pseudo-costs of external branching candidates (multiply as in pseudo costs 'u'pdate rule, or by 'd'omain reduction, or by domain reduction of 's'ibling, or by 'v'ariable score)}%
{}

\printoption{branching/random/maxbounddist}%
{$0\leq\textrm{real}\leq1$}%
{$1$}%
{maximal relative distance from current node's dual bound to primal bound compared to best node's dual bound for applying branching rule (0.0: only on current best node, 1.0: on all nodes)}%
{}

\printoption{branching/random/maxdepth}%
{$-1\leq\textrm{integer}$}%
{$-1$}%
{maximal depth level, up to which branching rule $<$random$>$ should be used (-1 for no limit)}%
{}

\printoption{branching/random/priority}%
{$-536870912\leq\textrm{integer}\leq536870911$}%
{$-100000$}%
{priority of branching rule $<$random$>$}%
{}

\printoption{branching/random/seed}%
{$0\leq\textrm{integer}$}%
{$0$}%
{initial random seed value}%
{}

\printoption{branching/relpscost/initcand}%
{$0\leq\textrm{integer}$}%
{$100$}%
{maximal number of candidates initialized with strong branching per node}%
{}

\printoption{branching/relpscost/inititer}%
{$0\leq\textrm{integer}$}%
{$0$}%
{iteration limit for strong branching initializations of pseudo cost entries (0: auto)}%
{}

\printoption{branching/relpscost/maxbounddist}%
{$0\leq\textrm{real}\leq1$}%
{$1$}%
{maximal relative distance from current node's dual bound to primal bound compared to best node's dual bound for applying branching rule (0.0: only on current best node, 1.0: on all nodes)}%
{}

\printoption{branching/relpscost/maxdepth}%
{$-1\leq\textrm{integer}$}%
{$-1$}%
{maximal depth level, up to which branching rule $<$relpscost$>$ should be used (-1 for no limit)}%
{}

\printoption{branching/relpscost/priority}%
{$-536870912\leq\textrm{integer}\leq536870911$}%
{$10000$}%
{priority of branching rule $<$relpscost$>$}%
{}

\printoption{branching/relpscost/sbiterofs}%
{$0\leq\textrm{integer}$}%
{$100000$}%
{additional number of allowed strong branching LP iterations}%
{}

\printoption{branching/relpscost/sbiterquot}%
{$0\leq\textrm{real}$}%
{$0.5$}%
{maximal fraction of strong branching LP iterations compared to node relaxation LP iterations}%
{}

\printoptioncategory{Branching (advanced options)}
\printoption{branching/fullstrong/reevalage}%
{$0\leq\textrm{integer}$}%
{$10$}%
{number of intermediate LPs solved to trigger reevaluation of strong branching value for a variable that was already evaluated at the current node}%
{}

\printoption{branching/inference/conflictweight}%
{$0\leq\textrm{real}$}%
{$1000$}%
{weight in score calculations for conflict score}%
{}

\printoption{branching/inference/cutoffweight}%
{$0\leq\textrm{real}$}%
{$1$}%
{weight in score calculations for cutoff score}%
{}

\printoption{branching/inference/fractionals}%
{boolean}%
{TRUE}%
{should branching on LP solution be restricted to the fractional variables?}%
{}

\printoption{branching/inference/inferenceweight}%
{$\textrm{real}$}%
{$1$}%
{weight in score calculations for inference score}%
{}

\printoption{branching/pscost/maxscoreweight}%
{$\textrm{real}$}%
{$1.3$}%
{weight for maximum of scores of a branching candidate when building weighted sum of min/max/sum of scores}%
{}

\printoption{branching/pscost/minscoreweight}%
{$\textrm{real}$}%
{$0.8$}%
{weight for minimum of scores of a branching candidate when building weighted sum of min/max/sum of scores}%
{}

\printoption{branching/pscost/sumscoreweight}%
{$\textrm{real}$}%
{$0.1$}%
{weight for sum of scores of a branching candidate when building weighted sum of min/max/sum of scores}%
{}

\printoption{branching/relpscost/conflictlengthweight}%
{$\textrm{real}$}%
{$0$}%
{weight in score calculations for conflict length score}%
{}

\printoption{branching/relpscost/conflictweight}%
{$\textrm{real}$}%
{$0.01$}%
{weight in score calculations for conflict score}%
{}

\printoption{branching/relpscost/cutoffweight}%
{$\textrm{real}$}%
{$0.0001$}%
{weight in score calculations for cutoff score}%
{}

\printoption{branching/relpscost/inferenceweight}%
{$\textrm{real}$}%
{$0.0001$}%
{weight in score calculations for inference score}%
{}

\printoption{branching/relpscost/maxbdchgs}%
{$-1\leq\textrm{integer}$}%
{$5$}%
{maximal number of bound tightenings before the node is reevaluated (-1: unlimited)}%
{}

\printoption{branching/relpscost/maxlookahead}%
{$1\leq\textrm{integer}$}%
{$8$}%
{maximal number of further variables evaluated without better score}%
{}

\printoption{branching/relpscost/maxreliable}%
{$0\leq\textrm{real}$}%
{$8$}%
{maximal value for minimum pseudo cost size to regard pseudo cost value as reliable}%
{}

\printoption{branching/relpscost/minreliable}%
{$0\leq\textrm{real}$}%
{$1$}%
{minimal value for minimum pseudo cost size to regard pseudo cost value as reliable}%
{}

\printoption{branching/relpscost/pscostweight}%
{$\textrm{real}$}%
{$1$}%
{weight in score calculations for pseudo cost score}%
{}

\printoption{branching/scorefac}%
{$0\leq\textrm{real}\leq1$}%
{$0.167$}%
{branching score factor to weigh downward and upward gain prediction in sum score function}%
{}

\printoption{branching/scorefunc}%
{character}%
{p}%
{branching score function ('s'um, 'p'roduct)}%
{}

\printoptioncategory{Conflict analysis}
\printoption{conflict/bounddisjunction/continuousfrac}%
{$0\leq\textrm{real}\leq1$}%
{$0.4$}%
{maximal percantage of continuous variables within a conflict}%
{}

\printoption{conflict/enable}%
{boolean}%
{TRUE}%
{should conflict analysis be enabled?}%
{}

\printoption{conflict/preferbinary}%
{boolean}%
{FALSE}%
{should binary conflicts be preferred?}%
{}

\printoption{conflict/restartfac}%
{$0\leq\textrm{real}$}%
{$1.5$}%
{factor to increase restartnum with after each restart}%
{}

\printoption{conflict/restartnum}%
{$0\leq\textrm{integer}$}%
{$0$}%
{number of successful conflict analysis calls that trigger a restart (0: disable conflict restarts)}%
{}

\printoption{conflict/useboundlp}%
{boolean}%
{FALSE}%
{should bound exceeding LP conflict analysis be used?}%
{}

\printoption{conflict/useinflp}%
{boolean}%
{TRUE}%
{should infeasible LP conflict analysis be used?}%
{}

\printoption{conflict/useprop}%
{boolean}%
{TRUE}%
{should propagation conflict analysis be used?}%
{}

\printoption{conflict/usepseudo}%
{boolean}%
{TRUE}%
{should pseudo solution conflict analysis be used?}%
{}

\printoption{conflict/usesb}%
{boolean}%
{FALSE}%
{should infeasible/bound exceeding strong branching conflict analysis be used?}%
{}

\printoptioncategory{Conflict analysis (advanced options)}
\printoption{conflict/allowlocal}%
{boolean}%
{TRUE}%
{should conflict constraints be generated that are only valid locally?}%
{}

\printoption{conflict/bounddisjunction/priority}%
{$\textrm{integer}$}%
{$-3000000$}%
{priority of conflict handler $<$bounddisjunction$>$}%
{}

\printoption{conflict/depthscorefac}%
{$\textrm{real}$}%
{$1$}%
{score factor for depth level in bound relaxation heuristic of LP analysis}%
{}

\printoption{conflict/dynamic}%
{boolean}%
{TRUE}%
{should the conflict constraints be subject to aging?}%
{}

\printoption{conflict/fuiplevels}%
{$-1\leq\textrm{integer}$}%
{$-1$}%
{number of depth levels up to which first UIP's are used in conflict analysis (-1: use All-FirstUIP rule)}%
{}

\printoption{conflict/ignorerelaxedbd}%
{boolean}%
{FALSE}%
{should relaxed bounds be ignored?}%
{}

\printoption{conflict/indicatorconflict/priority}%
{$\textrm{integer}$}%
{$200000$}%
{priority of conflict handler $<$indicatorconflict$>$}%
{}

\printoption{conflict/interconss}%
{$-1\leq\textrm{integer}$}%
{$-1$}%
{maximal number of intermediate conflict constraints generated in conflict graph (-1: use every intermediate constraint)}%
{}

\printoption{conflict/keepreprop}%
{boolean}%
{TRUE}%
{should constraints be kept for repropagation even if they are too long?}%
{}

\printoption{conflict/linear/priority}%
{$\textrm{integer}$}%
{$-1000000$}%
{priority of conflict handler $<$linear$>$}%
{}

\printoption{conflict/logicor/priority}%
{$\textrm{integer}$}%
{$800000$}%
{priority of conflict handler $<$logicor$>$}%
{}

\printoption{conflict/lpiterations}%
{$-1\leq\textrm{integer}$}%
{$10$}%
{maximal number of LP iterations in each LP resolving loop (-1: no limit)}%
{}

\printoption{conflict/maxconss}%
{$-1\leq\textrm{integer}$}%
{$10$}%
{maximal number of conflict constraints accepted at an infeasible node (-1: use all generated conflict constraints)}%
{}

\printoption{conflict/maxlploops}%
{$-1\leq\textrm{integer}$}%
{$2$}%
{maximal number of LP resolving loops during conflict analysis (-1: no limit)}%
{}

\printoption{conflict/maxvarsfac}%
{$0\leq\textrm{real}$}%
{$0.1$}%
{maximal fraction of variables involved in a conflict constraint}%
{}

\printoption{conflict/minmaxvars}%
{$0\leq\textrm{integer}$}%
{$30$}%
{minimal absolute maximum of variables involved in a conflict constraint}%
{}

\printoption{conflict/reconvlevels}%
{$-1\leq\textrm{integer}$}%
{$-1$}%
{number of depth levels up to which UIP reconvergence constraints are generated (-1: generate reconvergence constraints in all depth levels)}%
{}

\printoption{conflict/removable}%
{boolean}%
{TRUE}%
{should the conflict's relaxations be subject to LP aging and cleanup?}%
{}

\printoption{conflict/repropagate}%
{boolean}%
{TRUE}%
{should earlier nodes be repropagated in order to replace branching decisions by deductions?}%
{}

\printoption{conflict/scorefac}%
{$10^{- 6}\leq\textrm{real}\leq1$}%
{$0.98$}%
{factor to decrease importance of variables' earlier conflict scores}%
{}

\printoption{conflict/separate}%
{boolean}%
{TRUE}%
{should the conflict constraints be separated?}%
{}

\printoption{conflict/setppc/priority}%
{$\textrm{integer}$}%
{$700000$}%
{priority of conflict handler $<$setppc$>$}%
{}

\printoption{conflict/settlelocal}%
{boolean}%
{FALSE}%
{should conflict constraints be attached only to the local subtree where they can be useful?}%
{}

\printoptioncategory{Constraints}
\printoption{constraints/abspower/branchminconverror}%
{boolean}%
{FALSE}%
{whether to compute branching point such that the convexification error is minimized (after branching on 0.0)}%
{}

\printoption{constraints/abspower/cutmaxrange}%
{$0\leq\textrm{real}$}%
{$10^{  7}$}%
{maximal coef range of a cut (maximal coefficient divided by minimal coefficient) in order to be added to LP relaxation}%
{}

\printoption{constraints/abspower/dualpresolve}%
{boolean}%
{TRUE}%
{should dual presolve be applied?}%
{}

\printoption{constraints/abspower/linfeasshift}%
{boolean}%
{TRUE}%
{whether to try to make solutions in check function feasible by shifting the linear variable z}%
{}

\printoption{constraints/abspower/minefficacyenfofac}%
{$1\leq\textrm{real}$}%
{$2$}%
{minimal target efficacy of a cut in order to add it to relaxation during enforcement as factor of feasibility tolerance (may be ignored)}%
{}

\printoption{constraints/abspower/minefficacysepa}%
{$0\leq\textrm{real}$}%
{$0.0001$}%
{minimal efficacy for a cut to be added to the LP during separation; overwrites separating/efficacy}%
{}

\printoption{constraints/abspower/preferzerobranch}%
{$0\leq\textrm{integer}\leq3$}%
{$1$}%
{how much to prefer branching on 0.0 when sign of variable is not fixed yet: 0 no preference, 1 prefer if LP solution will be cutoff in both child nodes, 2 prefer always, 3 ensure always}%
{}

\printoption{constraints/abspower/projectrefpoint}%
{boolean}%
{TRUE}%
{whether to project the reference point when linearizing an absolute power constraint in a convex region}%
{}

\printoption{constraints/abspower/propfreq}%
{$-1\leq\textrm{integer}$}%
{$1$}%
{frequency for propagating domains (-1: never, 0: only in root node)}%
{}

\printoption{constraints/abspower/sepafreq}%
{$-1\leq\textrm{integer}$}%
{$1$}%
{frequency for separating cuts (-1: never, 0: only in root node)}%
{}

\printoption{constraints/abspower/sepainboundsonly}%
{boolean}%
{FALSE}%
{whether to separate linearization cuts only in the variable bounds (does not affect enforcement)}%
{}

\printoption{constraints/abspower/sepanlpmincont}%
{$0\leq\textrm{real}\leq2$}%
{$1$}%
{minimal required fraction of continuous variables in problem to use solution of NLP relaxation in root for separation}%
{}

\printoption{constraints/and/propfreq}%
{$-1\leq\textrm{integer}$}%
{$1$}%
{frequency for propagating domains (-1: never, 0: only in root node)}%
{}

\printoption{constraints/and/sepafreq}%
{$-1\leq\textrm{integer}$}%
{$1$}%
{frequency for separating cuts (-1: never, 0: only in root node)}%
{}

\printoption{constraints/bivariate/linfeasshift}%
{boolean}%
{TRUE}%
{whether to try to make solutions in check function feasible by shifting a linear variable (esp. useful if constraint was actually objective function)}%
{}

\printoption{constraints/bivariate/maxproprounds}%
{$0\leq\textrm{integer}$}%
{$1$}%
{limit on number of propagation rounds for a single constraint within one round of SCIP propagation}%
{}

\printoption{constraints/bivariate/minefficacyenfo}%
{$0\leq\textrm{real}$}%
{$2 \cdot 10^{- 6}$}%
{minimal target efficacy of a cut in order to add it to relaxation during enforcement (may be ignored)}%
{}

\printoption{constraints/bivariate/minefficacysepa}%
{$0\leq\textrm{real}$}%
{$0.0001$}%
{minimal efficacy for a cut to be added to the LP during separation; overwrites separating/efficacy}%
{}

\printoption{constraints/bivariate/ninitlprefpoints}%
{$0\leq\textrm{integer}$}%
{$3$}%
{number of reference points in each direction where to compute linear support for envelope in LP initialization}%
{}

\printoption{constraints/bivariate/propfreq}%
{$-1\leq\textrm{integer}$}%
{$1$}%
{frequency for propagating domains (-1: never, 0: only in root node)}%
{}

\printoption{constraints/bivariate/sepafreq}%
{$-1\leq\textrm{integer}$}%
{$1$}%
{frequency for separating cuts (-1: never, 0: only in root node)}%
{}

\printoption{constraints/bounddisjunction/propfreq}%
{$-1\leq\textrm{integer}$}%
{$1$}%
{frequency for propagating domains (-1: never, 0: only in root node)}%
{}

\printoption{constraints/bounddisjunction/sepafreq}%
{$-1\leq\textrm{integer}$}%
{$-1$}%
{frequency for separating cuts (-1: never, 0: only in root node)}%
{}

\printoption{constraints/indicator/maxsepacuts}%
{$0\leq\textrm{integer}$}%
{$100$}%
{maximal number of cuts separated per separation round}%
{}

\printoption{constraints/indicator/maxsepacutsroot}%
{$0\leq\textrm{integer}$}%
{$2000$}%
{maximal number of cuts separated per separation round in the root node}%
{}

\printoption{constraints/indicator/propfreq}%
{$-1\leq\textrm{integer}$}%
{$1$}%
{frequency for propagating domains (-1: never, 0: only in root node)}%
{}

\printoption{constraints/indicator/sepafreq}%
{$-1\leq\textrm{integer}$}%
{$10$}%
{frequency for separating cuts (-1: never, 0: only in root node)}%
{}

\printoption{constraints/integral/propfreq}%
{$-1\leq\textrm{integer}$}%
{$-1$}%
{frequency for propagating domains (-1: never, 0: only in root node)}%
{}

\printoption{constraints/integral/sepafreq}%
{$-1\leq\textrm{integer}$}%
{$-1$}%
{frequency for separating cuts (-1: never, 0: only in root node)}%
{}

\printoption{constraints/knapsack/maxrounds}%
{$-1\leq\textrm{integer}$}%
{$5$}%
{maximal number of separation rounds per node (-1: unlimited)}%
{}

\printoption{constraints/knapsack/maxroundsroot}%
{$-1\leq\textrm{integer}$}%
{$-1$}%
{maximal number of separation rounds per node in the root node (-1: unlimited)}%
{}

\printoption{constraints/knapsack/maxsepacuts}%
{$0\leq\textrm{integer}$}%
{$50$}%
{maximal number of cuts separated per separation round}%
{}

\printoption{constraints/knapsack/maxsepacutsroot}%
{$0\leq\textrm{integer}$}%
{$200$}%
{maximal number of cuts separated per separation round in the root node}%
{}

\printoption{constraints/knapsack/propfreq}%
{$-1\leq\textrm{integer}$}%
{$1$}%
{frequency for propagating domains (-1: never, 0: only in root node)}%
{}

\printoption{constraints/knapsack/sepafreq}%
{$-1\leq\textrm{integer}$}%
{$0$}%
{frequency for separating cuts (-1: never, 0: only in root node)}%
{}

\printoption{constraints/linear/maxrounds}%
{$-1\leq\textrm{integer}$}%
{$5$}%
{maximal number of separation rounds per node (-1: unlimited)}%
{}

\printoption{constraints/linear/maxroundsroot}%
{$-1\leq\textrm{integer}$}%
{$-1$}%
{maximal number of separation rounds per node in the root node (-1: unlimited)}%
{}

\printoption{constraints/linear/maxsepacuts}%
{$0\leq\textrm{integer}$}%
{$50$}%
{maximal number of cuts separated per separation round}%
{}

\printoption{constraints/linear/maxsepacutsroot}%
{$0\leq\textrm{integer}$}%
{$200$}%
{maximal number of cuts separated per separation round in the root node}%
{}

\printoption{constraints/linear/propfreq}%
{$-1\leq\textrm{integer}$}%
{$1$}%
{frequency for propagating domains (-1: never, 0: only in root node)}%
{}

\printoption{constraints/linear/sepafreq}%
{$-1\leq\textrm{integer}$}%
{$0$}%
{frequency for separating cuts (-1: never, 0: only in root node)}%
{}

\printoption{constraints/linear/separateall}%
{boolean}%
{FALSE}%
{should all constraints be subject to cardinality cut generation instead of only the ones with non-zero dual value?}%
{}

\printoption{constraints/linear/upgrade/knapsack}%
{boolean}%
{TRUE}%
{enable linear upgrading for constraint handler $<$knapsack$>$}%
{}

\printoption{constraints/linear/upgrade/logicor}%
{boolean}%
{TRUE}%
{enable linear upgrading for constraint handler $<$logicor$>$}%
{}

\printoption{constraints/linear/upgrade/setppc}%
{boolean}%
{TRUE}%
{enable linear upgrading for constraint handler $<$setppc$>$}%
{}

\printoption{constraints/linear/upgrade/varbound}%
{boolean}%
{TRUE}%
{enable linear upgrading for constraint handler $<$varbound$>$}%
{}

\printoption{constraints/logicor/propfreq}%
{$-1\leq\textrm{integer}$}%
{$1$}%
{frequency for propagating domains (-1: never, 0: only in root node)}%
{}

\printoption{constraints/logicor/sepafreq}%
{$-1\leq\textrm{integer}$}%
{$0$}%
{frequency for separating cuts (-1: never, 0: only in root node)}%
{}

\printoption{constraints/nonlinear/cutmaxrange}%
{$0\leq\textrm{real}$}%
{$10^{  7}$}%
{maximal coef range of a cut (maximal coefficient divided by minimal coefficient) in order to be added to LP relaxation}%
{}

\printoption{constraints/nonlinear/linfeasshift}%
{boolean}%
{TRUE}%
{whether to try to make solutions in check function feasible by shifting a linear variable (esp. useful if constraint was actually objective function)}%
{}

\printoption{constraints/nonlinear/maxproprounds}%
{$0\leq\textrm{integer}$}%
{$1$}%
{limit on number of propagation rounds for a single constraint within one round of SCIP propagation}%
{}

\printoption{constraints/nonlinear/propfreq}%
{$-1\leq\textrm{integer}$}%
{$1$}%
{frequency for propagating domains (-1: never, 0: only in root node)}%
{}

\printoption{constraints/nonlinear/reformulate}%
{boolean}%
{TRUE}%
{whether to reformulate expression graph}%
{}

\printoption{constraints/nonlinear/sepafreq}%
{$-1\leq\textrm{integer}$}%
{$1$}%
{frequency for separating cuts (-1: never, 0: only in root node)}%
{}

\printoption{constraints/nonlinear/sepanlpmincont}%
{$0\leq\textrm{real}\leq2$}%
{$1$}%
{minimal required fraction of continuous variables in problem to use solution of NLP relaxation in root for separation}%
{}

\printoption{constraints/nonlinear/upgrade/abspower}%
{boolean}%
{TRUE}%
{enable nonlinear upgrading for constraint handler $<$abspower$>$}%
{}

\printoption{constraints/nonlinear/upgrade/and}%
{boolean}%
{TRUE}%
{enable nonlinear upgrading for constraint handler $<$and$>$}%
{}

\printoption{constraints/nonlinear/upgrade/bivariate}%
{boolean}%
{FALSE}%
{enable nonlinear upgrading for constraint handler $<$bivariate$>$}%
{}

\printoption{constraints/nonlinear/upgrade/linear}%
{boolean}%
{TRUE}%
{enable nonlinear upgrading for constraint handler $<$linear$>$}%
{}

\printoption{constraints/nonlinear/upgrade/quadratic}%
{boolean}%
{TRUE}%
{enable nonlinear upgrading for constraint handler $<$quadratic$>$}%
{}

\printoption{constraints/quadratic/checkcurvature}%
{boolean}%
{TRUE}%
{whether multivariate quadratic functions should be checked for convexity/concavity}%
{}

\printoption{constraints/quadratic/empathy4and}%
{$0\leq\textrm{integer}\leq2$}%
{$0$}%
{empathy level for using the AND constraint handler: 0 always avoid using AND; 1 use AND sometimes; 2 use AND as often as possible}%
{}

\printoption{constraints/quadratic/propfreq}%
{$-1\leq\textrm{integer}$}%
{$1$}%
{frequency for propagating domains (-1: never, 0: only in root node)}%
{}

\printoption{constraints/quadratic/replacebinaryprod}%
{$0\leq\textrm{integer}$}%
{$\infty$}%
{max. length of linear term which when multiplied with a binary variables is replaced by an auxiliary variable and a linear reformulation (0 to turn off)}%
{}

\printoption{constraints/quadratic/sepafreq}%
{$-1\leq\textrm{integer}$}%
{$1$}%
{frequency for separating cuts (-1: never, 0: only in root node)}%
{}

\printoption{constraints/quadratic/sepanlpmincont}%
{$0\leq\textrm{real}\leq2$}%
{$1$}%
{minimal required fraction of continuous variables in problem to use solution of NLP relaxation in root for separation}%
{}

\printoption{constraints/quadratic/upgrade/abspower}%
{boolean}%
{TRUE}%
{enable quadratic upgrading for constraint handler $<$abspower$>$}%
{}

\printoption{constraints/quadratic/upgrade/bivariate}%
{boolean}%
{FALSE}%
{enable quadratic upgrading for constraint handler $<$bivariate$>$}%
{}

\printoption{constraints/quadratic/upgrade/bounddisjunction}%
{boolean}%
{TRUE}%
{enable quadratic upgrading for constraint handler $<$bounddisjunction$>$}%
{}

\printoption{constraints/quadratic/upgrade/linear}%
{boolean}%
{TRUE}%
{enable quadratic upgrading for constraint handler $<$linear$>$}%
{}

\printoption{constraints/quadratic/upgrade/soc}%
{boolean}%
{TRUE}%
{enable quadratic upgrading for constraint handler $<$soc$>$}%
{}

\printoption{constraints/setppc/propfreq}%
{$-1\leq\textrm{integer}$}%
{$1$}%
{frequency for propagating domains (-1: never, 0: only in root node)}%
{}

\printoption{constraints/setppc/sepafreq}%
{$-1\leq\textrm{integer}$}%
{$0$}%
{frequency for separating cuts (-1: never, 0: only in root node)}%
{}

\printoption{constraints/soc/glineur}%
{boolean}%
{TRUE}%
{whether the Glineur Outer Approximation should be used instead of Ben-Tal Nemirovski}%
{}

\printoption{constraints/soc/linfeasshift}%
{boolean}%
{TRUE}%
{whether to try to make solutions feasible in check by shifting the variable on the right hand side}%
{}

\printoption{constraints/soc/minefficacy}%
{$0\leq\textrm{real}$}%
{$0.0001$}%
{minimal efficacy of a cut to be added to LP in separation}%
{}

\printoption{constraints/soc/nauxvars}%
{$0\leq\textrm{integer}$}%
{$0$}%
{number of auxiliary variables to use when creating a linear outer approx. of a SOC3 constraint; 0 to turn off}%
{}

\printoption{constraints/soc/nlpform}%
{character}%
{a}%
{which formulation to use when adding a SOC constraint to the NLP (a: automatic, q: nonconvex quadratic form, s: convex sqrt form, e: convex exponential-sqrt form, d: convex division form)}%
{}

\printoption{constraints/soc/propfreq}%
{$-1\leq\textrm{integer}$}%
{$1$}%
{frequency for propagating domains (-1: never, 0: only in root node)}%
{}

\printoption{constraints/soc/sepafreq}%
{$-1\leq\textrm{integer}$}%
{$1$}%
{frequency for separating cuts (-1: never, 0: only in root node)}%
{}

\printoption{constraints/soc/sepanlpmincont}%
{$0\leq\textrm{real}\leq2$}%
{$1$}%
{minimal required fraction of continuous variables in problem to use solution of NLP relaxation in root for separation}%
{}

\printoption{constraints/SOS1/branchnonzeros}%
{boolean}%
{FALSE}%
{Branch on SOS constraint with most number of nonzeros?}%
{}

\printoption{constraints/SOS1/branchsos}%
{boolean}%
{TRUE}%
{Use SOS1 branching in enforcing (otherwise leave decision to branching rules)?}%
{}

\printoption{constraints/SOS1/branchweight}%
{boolean}%
{FALSE}%
{Branch on SOS cons. with highest nonzero-variable weight for branching (needs branchnonzeros = false)?}%
{}

\printoption{constraints/SOS1/propfreq}%
{$-1\leq\textrm{integer}$}%
{$1$}%
{frequency for propagating domains (-1: never, 0: only in root node)}%
{}

\printoption{constraints/SOS1/sepafreq}%
{$-1\leq\textrm{integer}$}%
{$0$}%
{frequency for separating cuts (-1: never, 0: only in root node)}%
{}

\printoption{constraints/SOS2/propfreq}%
{$-1\leq\textrm{integer}$}%
{$1$}%
{frequency for propagating domains (-1: never, 0: only in root node)}%
{}

\printoption{constraints/SOS2/sepafreq}%
{$-1\leq\textrm{integer}$}%
{$0$}%
{frequency for separating cuts (-1: never, 0: only in root node)}%
{}

\printoption{constraints/superindicator/propfreq}%
{$-1\leq\textrm{integer}$}%
{$1$}%
{frequency for propagating domains (-1: never, 0: only in root node)}%
{}

\printoption{constraints/superindicator/sepafreq}%
{$-1\leq\textrm{integer}$}%
{$-1$}%
{frequency for separating cuts (-1: never, 0: only in root node)}%
{}

\printoption{constraints/varbound/propfreq}%
{$-1\leq\textrm{integer}$}%
{$1$}%
{frequency for propagating domains (-1: never, 0: only in root node)}%
{}

\printoption{constraints/varbound/sepafreq}%
{$-1\leq\textrm{integer}$}%
{$0$}%
{frequency for separating cuts (-1: never, 0: only in root node)}%
{}

\printoption{constraints/varbound/usebdwidening}%
{boolean}%
{TRUE}%
{should bound widening be used in conflict analysis?}%
{}

\printoptioncategory{Constraints (advanced options)}
\printoption{constraints/abspower/addvarboundcons}%
{boolean}%
{TRUE}%
{should variable bound constraints be added for derived variable bounds?}%
{}

\printoption{constraints/abspower/delaypresol}%
{boolean}%
{FALSE}%
{should presolving method be delayed, if other presolvers found reductions?}%
{}

\printoption{constraints/abspower/delayprop}%
{boolean}%
{FALSE}%
{should propagation method be delayed, if other propagators found reductions?}%
{}

\printoption{constraints/abspower/delaysepa}%
{boolean}%
{FALSE}%
{should separation method be delayed, if other separators found cuts?}%
{}

\printoption{constraints/abspower/eagerfreq}%
{$-1\leq\textrm{integer}$}%
{$100$}%
{frequency for using all instead of only the useful constraints in separation, propagation and enforcement (-1: never, 0: only in first evaluation)}%
{}

\printoption{constraints/abspower/maxprerounds}%
{$-1\leq\textrm{integer}$}%
{$-1$}%
{maximal number of presolving rounds the constraint handler participates in (-1: no limit)}%
{}

\printoption{constraints/abspower/timingmask}%
{$1\leq\textrm{integer}\leq15$}%
{$15$}%
{timing when constraint propagation should be called (1:BEFORELP, 2:DURINGLPLOOP, 4:AFTERLPLOOP, 15:ALWAYS))}%
{}

\printoption{constraints/agelimit}%
{$-1\leq\textrm{integer}$}%
{$0$}%
{maximum age an unnecessary constraint can reach before it is deleted (0: dynamic, -1: keep all constraints)}%
{}

\printoption{constraints/and/aggrlinearization}%
{boolean}%
{FALSE}%
{should an aggregated linearization be used?}%
{}

\printoption{constraints/and/delaypresol}%
{boolean}%
{FALSE}%
{should presolving method be delayed, if other presolvers found reductions?}%
{}

\printoption{constraints/and/delayprop}%
{boolean}%
{FALSE}%
{should propagation method be delayed, if other propagators found reductions?}%
{}

\printoption{constraints/and/delaysepa}%
{boolean}%
{FALSE}%
{should separation method be delayed, if other separators found cuts?}%
{}

\printoption{constraints/and/dualpresolving}%
{boolean}%
{TRUE}%
{should dual presolving be performed?}%
{}

\printoption{constraints/and/eagerfreq}%
{$-1\leq\textrm{integer}$}%
{$100$}%
{frequency for using all instead of only the useful constraints in separation, propagation and enforcement (-1: never, 0: only in first evaluation)}%
{}

\printoption{constraints/and/enforcecuts}%
{boolean}%
{TRUE}%
{should cuts be separated during LP enforcing?}%
{}

\printoption{constraints/and/linearize}%
{boolean}%
{FALSE}%
{should the "and" constraint get linearized and removed (in presolving)?}%
{}

\printoption{constraints/and/maxprerounds}%
{$-1\leq\textrm{integer}$}%
{$-1$}%
{maximal number of presolving rounds the constraint handler participates in (-1: no limit)}%
{}

\printoption{constraints/and/presolpairwise}%
{boolean}%
{TRUE}%
{should pairwise constraint comparison be performed in presolving?}%
{}

\printoption{constraints/and/presolusehashing}%
{boolean}%
{TRUE}%
{should hash table be used for detecting redundant constraints in advance}%
{}

\printoption{constraints/and/timingmask}%
{$1\leq\textrm{integer}\leq15$}%
{$1$}%
{timing when constraint propagation should be called (1:BEFORELP, 2:DURINGLPLOOP, 4:AFTERLPLOOP, 15:ALWAYS))}%
{}

\printoption{constraints/bivariate/cutmaxrange}%
{$0\leq\textrm{real}$}%
{$10^{  7}$}%
{maximal coef range of a cut (maximal coefficient divided by minimal coefficient) in order to be added to LP relaxation}%
{}

\printoption{constraints/bivariate/delaypresol}%
{boolean}%
{FALSE}%
{should presolving method be delayed, if other presolvers found reductions?}%
{}

\printoption{constraints/bivariate/delayprop}%
{boolean}%
{FALSE}%
{should propagation method be delayed, if other propagators found reductions?}%
{}

\printoption{constraints/bivariate/delaysepa}%
{boolean}%
{FALSE}%
{should separation method be delayed, if other separators found cuts?}%
{}

\printoption{constraints/bivariate/eagerfreq}%
{$-1\leq\textrm{integer}$}%
{$100$}%
{frequency for using all instead of only the useful constraints in separation, propagation and enforcement (-1: never, 0: only in first evaluation)}%
{}

\printoption{constraints/bivariate/maxprerounds}%
{$-1\leq\textrm{integer}$}%
{$-1$}%
{maximal number of presolving rounds the constraint handler participates in (-1: no limit)}%
{}

\printoption{constraints/bivariate/timingmask}%
{$1\leq\textrm{integer}\leq15$}%
{$1$}%
{timing when constraint propagation should be called (1:BEFORELP, 2:DURINGLPLOOP, 4:AFTERLPLOOP, 15:ALWAYS))}%
{}

\printoption{constraints/bounddisjunction/delaypresol}%
{boolean}%
{FALSE}%
{should presolving method be delayed, if other presolvers found reductions?}%
{}

\printoption{constraints/bounddisjunction/delayprop}%
{boolean}%
{FALSE}%
{should propagation method be delayed, if other propagators found reductions?}%
{}

\printoption{constraints/bounddisjunction/delaysepa}%
{boolean}%
{FALSE}%
{should separation method be delayed, if other separators found cuts?}%
{}

\printoption{constraints/bounddisjunction/eagerfreq}%
{$-1\leq\textrm{integer}$}%
{$100$}%
{frequency for using all instead of only the useful constraints in separation, propagation and enforcement (-1: never, 0: only in first evaluation)}%
{}

\printoption{constraints/bounddisjunction/maxprerounds}%
{$-1\leq\textrm{integer}$}%
{$-1$}%
{maximal number of presolving rounds the constraint handler participates in (-1: no limit)}%
{}

\printoption{constraints/bounddisjunction/timingmask}%
{$1\leq\textrm{integer}\leq15$}%
{$1$}%
{timing when constraint propagation should be called (1:BEFORELP, 2:DURINGLPLOOP, 4:AFTERLPLOOP, 15:ALWAYS))}%
{}

\printoption{constraints/disableenfops}%
{boolean}%
{FALSE}%
{should enforcement of pseudo solution be disabled?}%
{}

\printoption{constraints/indicator/addcoupling}%
{boolean}%
{TRUE}%
{Add coupling constraints if big-M is small enough?}%
{}

\printoption{constraints/indicator/addcouplingcons}%
{boolean}%
{FALSE}%
{Add initial coupling inequalities as linear constraints, if 'addcoupling' is true?}%
{}

\printoption{constraints/indicator/addopposite}%
{boolean}%
{FALSE}%
{Add opposite inequality in nodes in which the binary variable has been fixed to 0?}%
{}

\printoption{constraints/indicator/branchindicators}%
{boolean}%
{FALSE}%
{Branch on indicator constraints in enforcing?}%
{}

\printoption{constraints/indicator/conflictsupgrade}%
{boolean}%
{FALSE}%
{Try to upgrade bounddisjunction conflicts by replacing slack variables?}%
{}

\printoption{constraints/indicator/delaypresol}%
{boolean}%
{FALSE}%
{should presolving method be delayed, if other presolvers found reductions?}%
{}

\printoption{constraints/indicator/delayprop}%
{boolean}%
{FALSE}%
{should propagation method be delayed, if other propagators found reductions?}%
{}

\printoption{constraints/indicator/delaysepa}%
{boolean}%
{FALSE}%
{should separation method be delayed, if other separators found cuts?}%
{}

\printoption{constraints/indicator/dualreductions}%
{boolean}%
{TRUE}%
{should dual reduction steps be performed?}%
{}

\printoption{constraints/indicator/eagerfreq}%
{$-1\leq\textrm{integer}$}%
{$100$}%
{frequency for using all instead of only the useful constraints in separation, propagation and enforcement (-1: never, 0: only in first evaluation)}%
{}

\printoption{constraints/indicator/enforcecuts}%
{boolean}%
{FALSE}%
{In enforcing try to generate cuts (only if sepaalternativelp is true)?}%
{}

\printoption{constraints/indicator/forcerestart}%
{boolean}%
{FALSE}%
{force restart if we have a max FS instance and gap is 1?}%
{}

\printoption{constraints/indicator/generatebilinear}%
{boolean}%
{FALSE}%
{Do not generate indicator constraint, but a bilinear constraint instead?}%
{}

\printoption{constraints/indicator/genlogicor}%
{boolean}%
{FALSE}%
{Generate logicor constraints instead of cuts?}%
{}

\printoption{constraints/indicator/maxconditionaltlp}%
{$0\leq\textrm{real}$}%
{$0$}%
{maximum estimated condition of the solution basis matrix of the alternative LP to be trustworthy (0.0 to disable check)}%
{}

\printoption{constraints/indicator/maxcouplingvalue}%
{$0\leq\textrm{real}\leq10^{  9}$}%
{$10000$}%
{maximum coefficient for binary variable in coupling constraint}%
{}

\printoption{constraints/indicator/maxprerounds}%
{$-1\leq\textrm{integer}$}%
{$-1$}%
{maximal number of presolving rounds the constraint handler participates in (-1: no limit)}%
{}

\printoption{constraints/indicator/nolinconscont}%
{boolean}%
{FALSE}%
{decompose problem - do not generate linear constraint if all variables are continuous}%
{}

\printoption{constraints/indicator/removeindicators}%
{boolean}%
{FALSE}%
{Remove indicator constraint if corresponding variable bound constraint has been added?}%
{}

\printoption{constraints/indicator/restartfrac}%
{$0\leq\textrm{real}\leq1$}%
{$0.9$}%
{fraction of binary variables that need to be fixed before restart occurs (in forcerestart)}%
{}

\printoption{constraints/indicator/sepaalternativelp}%
{boolean}%
{FALSE}%
{Separate using the alternative LP?}%
{}

\printoption{constraints/indicator/sepacouplingcuts}%
{boolean}%
{FALSE}%
{Should the coupling inequalities be separated dynamically?}%
{}

\printoption{constraints/indicator/sepacouplinglocal}%
{boolean}%
{FALSE}%
{Allow to use local bounds in order to separated coupling inequalities?}%
{}

\printoption{constraints/indicator/sepacouplingvalue}%
{$0\leq\textrm{real}\leq10^{  9}$}%
{$10000$}%
{maximum coefficient for binary variable in separated coupling constraint}%
{}

\printoption{constraints/indicator/timingmask}%
{$1\leq\textrm{integer}\leq15$}%
{$1$}%
{timing when constraint propagation should be called (1:BEFORELP, 2:DURINGLPLOOP, 4:AFTERLPLOOP, 15:ALWAYS))}%
{}

\printoption{constraints/indicator/trysolutions}%
{boolean}%
{TRUE}%
{Try to make solutions feasible by setting indicator variables?}%
{}

\printoption{constraints/indicator/updatebounds}%
{boolean}%
{FALSE}%
{Update bounds of original variables for separation?}%
{}

\printoption{constraints/integral/delaypresol}%
{boolean}%
{FALSE}%
{should presolving method be delayed, if other presolvers found reductions?}%
{}

\printoption{constraints/integral/delayprop}%
{boolean}%
{FALSE}%
{should propagation method be delayed, if other propagators found reductions?}%
{}

\printoption{constraints/integral/delaysepa}%
{boolean}%
{FALSE}%
{should separation method be delayed, if other separators found cuts?}%
{}

\printoption{constraints/integral/eagerfreq}%
{$-1\leq\textrm{integer}$}%
{$-1$}%
{frequency for using all instead of only the useful constraints in separation, propagation and enforcement (-1: never, 0: only in first evaluation)}%
{}

\printoption{constraints/integral/maxprerounds}%
{$-1\leq\textrm{integer}$}%
{$0$}%
{maximal number of presolving rounds the constraint handler participates in (-1: no limit)}%
{}

\printoption{constraints/integral/timingmask}%
{$1\leq\textrm{integer}\leq15$}%
{$1$}%
{timing when constraint propagation should be called (1:BEFORELP, 2:DURINGLPLOOP, 4:AFTERLPLOOP, 15:ALWAYS))}%
{}

\printoption{constraints/knapsack/delaypresol}%
{boolean}%
{FALSE}%
{should presolving method be delayed, if other presolvers found reductions?}%
{}

\printoption{constraints/knapsack/delayprop}%
{boolean}%
{FALSE}%
{should propagation method be delayed, if other propagators found reductions?}%
{}

\printoption{constraints/knapsack/delaysepa}%
{boolean}%
{FALSE}%
{should separation method be delayed, if other separators found cuts?}%
{}

\printoption{constraints/knapsack/disaggregation}%
{boolean}%
{TRUE}%
{should disaggregation of knapsack constraints be allowed in preprocessing?}%
{}

\printoption{constraints/knapsack/dualpresolving}%
{boolean}%
{TRUE}%
{should dual presolving steps be performed?}%
{}

\printoption{constraints/knapsack/eagerfreq}%
{$-1\leq\textrm{integer}$}%
{$100$}%
{frequency for using all instead of only the useful constraints in separation, propagation and enforcement (-1: never, 0: only in first evaluation)}%
{}

\printoption{constraints/knapsack/maxcardbounddist}%
{$0\leq\textrm{real}\leq1$}%
{$0$}%
{maximal relative distance from current node's dual bound to primal bound compared to best node's dual bound for separating knapsack cuts}%
{}

\printoption{constraints/knapsack/maxprerounds}%
{$-1\leq\textrm{integer}$}%
{$-1$}%
{maximal number of presolving rounds the constraint handler participates in (-1: no limit)}%
{}

\printoption{constraints/knapsack/negatedclique}%
{boolean}%
{TRUE}%
{should negated clique information be used in solving process}%
{}

\printoption{constraints/knapsack/presolpairwise}%
{boolean}%
{TRUE}%
{should pairwise constraint comparison be performed in presolving?}%
{}

\printoption{constraints/knapsack/presolusehashing}%
{boolean}%
{TRUE}%
{should hash table be used for detecting redundant constraints in advance}%
{}

\printoption{constraints/knapsack/sepacardfreq}%
{$-1\leq\textrm{integer}$}%
{$1$}%
{multiplier on separation frequency, how often knapsack cuts are separated (-1: never, 0: only at root)}%
{}

\printoption{constraints/knapsack/simplifyinequalities}%
{boolean}%
{TRUE}%
{should presolving try to simplify knapsacks}%
{}

\printoption{constraints/knapsack/timingmask}%
{$1\leq\textrm{integer}\leq15$}%
{$1$}%
{timing when constraint propagation should be called (1:BEFORELP, 2:DURINGLPLOOP, 4:AFTERLPLOOP, 15:ALWAYS))}%
{}

\printoption{constraints/knapsack/usegubs}%
{boolean}%
{FALSE}%
{should GUB information be used for separation?}%
{}

\printoption{constraints/linear/aggregatevariables}%
{boolean}%
{TRUE}%
{should presolving search for aggregations in equations}%
{}

\printoption{constraints/linear/delaypresol}%
{boolean}%
{FALSE}%
{should presolving method be delayed, if other presolvers found reductions?}%
{}

\printoption{constraints/linear/delayprop}%
{boolean}%
{FALSE}%
{should propagation method be delayed, if other propagators found reductions?}%
{}

\printoption{constraints/linear/delaysepa}%
{boolean}%
{FALSE}%
{should separation method be delayed, if other separators found cuts?}%
{}

\printoption{constraints/linear/dualpresolving}%
{boolean}%
{TRUE}%
{should dual presolving steps be performed?}%
{}

\printoption{constraints/linear/eagerfreq}%
{$-1\leq\textrm{integer}$}%
{$100$}%
{frequency for using all instead of only the useful constraints in separation, propagation and enforcement (-1: never, 0: only in first evaluation)}%
{}

\printoption{constraints/linear/maxaggrnormscale}%
{$0\leq\textrm{real}$}%
{$0$}%
{maximal allowed relative gain in maximum norm for constraint aggregation (0.0: disable constraint aggregation)}%
{}

\printoption{constraints/linear/maxcardbounddist}%
{$0\leq\textrm{real}\leq1$}%
{$0$}%
{maximal relative distance from current node's dual bound to primal bound compared to best node's dual bound for separating knapsack cardinality cuts}%
{}

\printoption{constraints/linear/maxprerounds}%
{$-1\leq\textrm{integer}$}%
{$-1$}%
{maximal number of presolving rounds the constraint handler participates in (-1: no limit)}%
{}

\printoption{constraints/linear/mingainpernmincomparisons}%
{$0\leq\textrm{real}$}%
{$10^{- 6}$}%
{minimal gain per minimal pairwise presolve comparisons to repeat pairwise comparison round}%
{}

\printoption{constraints/linear/nmincomparisons}%
{$1\leq\textrm{integer}$}%
{$200000$}%
{number for minimal pairwise presolve comparisons}%
{}

\printoption{constraints/linear/presolpairwise}%
{boolean}%
{TRUE}%
{should pairwise constraint comparison be performed in presolving?}%
{}

\printoption{constraints/linear/presolusehashing}%
{boolean}%
{TRUE}%
{should hash table be used for detecting redundant constraints in advance}%
{}

\printoption{constraints/linear/simplifyinequalities}%
{boolean}%
{TRUE}%
{should presolving try to simplify inequalities}%
{}

\printoption{constraints/linear/sortvars}%
{boolean}%
{TRUE}%
{apply binaries sorting in decr. order of coeff abs value?}%
{}

\printoption{constraints/linear/tightenboundsfreq}%
{$-1\leq\textrm{integer}$}%
{$1$}%
{multiplier on propagation frequency, how often the bounds are tightened (-1: never, 0: only at root)}%
{}

\printoption{constraints/linear/timingmask}%
{$1\leq\textrm{integer}\leq15$}%
{$1$}%
{timing when constraint propagation should be called (1:BEFORELP, 2:DURINGLPLOOP, 4:AFTERLPLOOP, 15:ALWAYS))}%
{}

\printoption{constraints/logicor/delaypresol}%
{boolean}%
{FALSE}%
{should presolving method be delayed, if other presolvers found reductions?}%
{}

\printoption{constraints/logicor/delayprop}%
{boolean}%
{FALSE}%
{should propagation method be delayed, if other propagators found reductions?}%
{}

\printoption{constraints/logicor/delaysepa}%
{boolean}%
{FALSE}%
{should separation method be delayed, if other separators found cuts?}%
{}

\printoption{constraints/logicor/dualpresolving}%
{boolean}%
{TRUE}%
{should dual presolving steps be performed?}%
{}

\printoption{constraints/logicor/eagerfreq}%
{$-1\leq\textrm{integer}$}%
{$100$}%
{frequency for using all instead of only the useful constraints in separation, propagation and enforcement (-1: never, 0: only in first evaluation)}%
{}

\printoption{constraints/logicor/maxprerounds}%
{$-1\leq\textrm{integer}$}%
{$-1$}%
{maximal number of presolving rounds the constraint handler participates in (-1: no limit)}%
{}

\printoption{constraints/logicor/negatedclique}%
{boolean}%
{TRUE}%
{should negated clique information be used in presolving}%
{}

\printoption{constraints/logicor/presolpairwise}%
{boolean}%
{TRUE}%
{should pairwise constraint comparison be performed in presolving?}%
{}

\printoption{constraints/logicor/presolusehashing}%
{boolean}%
{TRUE}%
{should hash table be used for detecting redundant constraints in advance}%
{}

\printoption{constraints/logicor/timingmask}%
{$1\leq\textrm{integer}\leq15$}%
{$1$}%
{timing when constraint propagation should be called (1:BEFORELP, 2:DURINGLPLOOP, 4:AFTERLPLOOP, 15:ALWAYS))}%
{}

\printoption{constraints/nonlinear/assumeconvex}%
{boolean}%
{FALSE}%
{whether to assume that nonlinear functions in inequalities ($<$=) are convex (disables reformulation)}%
{}

\printoption{constraints/nonlinear/delaypresol}%
{boolean}%
{FALSE}%
{should presolving method be delayed, if other presolvers found reductions?}%
{}

\printoption{constraints/nonlinear/delayprop}%
{boolean}%
{FALSE}%
{should propagation method be delayed, if other propagators found reductions?}%
{}

\printoption{constraints/nonlinear/delaysepa}%
{boolean}%
{FALSE}%
{should separation method be delayed, if other separators found cuts?}%
{}

\printoption{constraints/nonlinear/eagerfreq}%
{$-1\leq\textrm{integer}$}%
{$100$}%
{frequency for using all instead of only the useful constraints in separation, propagation and enforcement (-1: never, 0: only in first evaluation)}%
{}

\printoption{constraints/nonlinear/maxexpansionexponent}%
{$1\leq\textrm{integer}$}%
{$2$}%
{maximal exponent where still expanding non-monomial polynomials in expression simplification}%
{}

\printoption{constraints/nonlinear/maxprerounds}%
{$-1\leq\textrm{integer}$}%
{$-1$}%
{maximal number of presolving rounds the constraint handler participates in (-1: no limit)}%
{}

\printoption{constraints/nonlinear/minefficacyenfofac}%
{$1\leq\textrm{real}$}%
{$2$}%
{minimal target efficacy of a cut in order to add it to relaxation during enforcement as a factor of the feasibility tolerance (may be ignored)}%
{}

\printoption{constraints/nonlinear/minefficacysepa}%
{$0\leq\textrm{real}$}%
{$0.0001$}%
{minimal efficacy for a cut to be added to the LP during separation; overwrites separating/efficacy}%
{}

\printoption{constraints/nonlinear/timingmask}%
{$1\leq\textrm{integer}\leq15$}%
{$1$}%
{timing when constraint propagation should be called (1:BEFORELP, 2:DURINGLPLOOP, 4:AFTERLPLOOP, 15:ALWAYS))}%
{}

\printoption{constraints/obsoleteage}%
{$-1\leq\textrm{integer}$}%
{$-1$}%
{age of a constraint after which it is marked obsolete (0: dynamic, -1 do not mark constraints obsolete)}%
{}

\printoption{constraints/quadratic/binreforminitial}%
{boolean}%
{FALSE}%
{whether to make non-varbound linear constraints added due to replacing products with binary variables initial}%
{}

\printoption{constraints/quadratic/checkfactorable}%
{boolean}%
{TRUE}%
{whether constraint functions should be checked to be factorable}%
{}

\printoption{constraints/quadratic/cutmaxrange}%
{$0\leq\textrm{real}$}%
{$10^{  7}$}%
{maximal coef range of a cut (maximal coefficient divided by minimal coefficient) in order to be added to LP relaxation}%
{}

\printoption{constraints/quadratic/delaypresol}%
{boolean}%
{FALSE}%
{should presolving method be delayed, if other presolvers found reductions?}%
{}

\printoption{constraints/quadratic/delayprop}%
{boolean}%
{FALSE}%
{should propagation method be delayed, if other propagators found reductions?}%
{}

\printoption{constraints/quadratic/delaysepa}%
{boolean}%
{FALSE}%
{should separation method be delayed, if other separators found cuts?}%
{}

\printoption{constraints/quadratic/disaggregate}%
{boolean}%
{FALSE}%
{whether to disaggregate quadratic parts that decompose into a sum of non-overlapping quadratic terms}%
{}

\printoption{constraints/quadratic/eagerfreq}%
{$-1\leq\textrm{integer}$}%
{$100$}%
{frequency for using all instead of only the useful constraints in separation, propagation and enforcement (-1: never, 0: only in first evaluation)}%
{}

\printoption{constraints/quadratic/linearizeheursol}%
{boolean}%
{TRUE}%
{whether linearizations of convex quadratic constraints should be added to cutpool in a solution found by some heuristic}%
{}

\printoption{constraints/quadratic/linfeasshift}%
{boolean}%
{TRUE}%
{whether to try to make solutions in check function feasible by shifting a linear variable (esp. useful if constraint was actually objective function)}%
{}

\printoption{constraints/quadratic/maxprerounds}%
{$-1\leq\textrm{integer}$}%
{$-1$}%
{maximal number of presolving rounds the constraint handler participates in (-1: no limit)}%
{}

\printoption{constraints/quadratic/maxproprounds}%
{$0\leq\textrm{integer}$}%
{$1$}%
{limit on number of propagation rounds for a single constraint within one round of SCIP propagation during solve}%
{}

\printoption{constraints/quadratic/maxproproundspresolve}%
{$0\leq\textrm{integer}$}%
{$10$}%
{limit on number of propagation rounds for a single constraint within one round of SCIP presolve}%
{}

\printoption{constraints/quadratic/minefficacyenfofac}%
{$1\leq\textrm{real}$}%
{$2$}%
{minimal target efficacy of a cut in order to add it to relaxation during enforcement as a factor of the feasibility tolerance (may be ignored)}%
{}

\printoption{constraints/quadratic/minefficacysepa}%
{$0\leq\textrm{real}$}%
{$0.0001$}%
{minimal efficacy for a cut to be added to the LP during separation; overwrites separating/efficacy}%
{}

\printoption{constraints/quadratic/scaling}%
{boolean}%
{TRUE}%
{whether a quadratic constraint should be scaled w.r.t. the current gradient norm when checking for feasibility}%
{}

\printoption{constraints/quadratic/timingmask}%
{$1\leq\textrm{integer}\leq15$}%
{$1$}%
{timing when constraint propagation should be called (1:BEFORELP, 2:DURINGLPLOOP, 4:AFTERLPLOOP, 15:ALWAYS))}%
{}

\printoption{constraints/setppc/addvariablesascliques}%
{boolean}%
{FALSE}%
{should we try to generate extra cliques out of all binary variables to maybe fasten redundant constraint detection}%
{}

\printoption{constraints/setppc/cliquelifting}%
{boolean}%
{FALSE}%
{ should we try to lift variables into other clique constraints, fix variables, aggregate them, and also shrink the amount of variables in clique constraints}%
{}

\printoption{constraints/setppc/cliqueshrinking}%
{boolean}%
{TRUE}%
{should we try to shrink the number of variables in a clique constraints, by replacing more than one variable by only one}%
{}

\printoption{constraints/setppc/delaypresol}%
{boolean}%
{FALSE}%
{should presolving method be delayed, if other presolvers found reductions?}%
{}

\printoption{constraints/setppc/delayprop}%
{boolean}%
{FALSE}%
{should propagation method be delayed, if other propagators found reductions?}%
{}

\printoption{constraints/setppc/delaysepa}%
{boolean}%
{FALSE}%
{should separation method be delayed, if other separators found cuts?}%
{}

\printoption{constraints/setppc/dualpresolving}%
{boolean}%
{TRUE}%
{should dual presolving steps be performed?}%
{}

\printoption{constraints/setppc/eagerfreq}%
{$-1\leq\textrm{integer}$}%
{$100$}%
{frequency for using all instead of only the useful constraints in separation, propagation and enforcement (-1: never, 0: only in first evaluation)}%
{}

\printoption{constraints/setppc/maxprerounds}%
{$-1\leq\textrm{integer}$}%
{$-1$}%
{maximal number of presolving rounds the constraint handler participates in (-1: no limit)}%
{}

\printoption{constraints/setppc/npseudobranches}%
{$0\leq\textrm{integer}$}%
{$2$}%
{number of children created in pseudo branching (0: disable pseudo branching)}%
{}

\printoption{constraints/setppc/presolpairwise}%
{boolean}%
{TRUE}%
{should pairwise constraint comparison be performed in presolving?}%
{}

\printoption{constraints/setppc/presolusehashing}%
{boolean}%
{TRUE}%
{should hash table be used for detecting redundant constraints in advance}%
{}

\printoption{constraints/setppc/timingmask}%
{$1\leq\textrm{integer}\leq15$}%
{$1$}%
{timing when constraint propagation should be called (1:BEFORELP, 2:DURINGLPLOOP, 4:AFTERLPLOOP, 15:ALWAYS))}%
{}

\printoption{constraints/soc/delaypresol}%
{boolean}%
{FALSE}%
{should presolving method be delayed, if other presolvers found reductions?}%
{}

\printoption{constraints/soc/delayprop}%
{boolean}%
{FALSE}%
{should propagation method be delayed, if other propagators found reductions?}%
{}

\printoption{constraints/soc/delaysepa}%
{boolean}%
{FALSE}%
{should separation method be delayed, if other separators found cuts?}%
{}

\printoption{constraints/soc/eagerfreq}%
{$-1\leq\textrm{integer}$}%
{$100$}%
{frequency for using all instead of only the useful constraints in separation, propagation and enforcement (-1: never, 0: only in first evaluation)}%
{}

\printoption{constraints/soc/maxprerounds}%
{$-1\leq\textrm{integer}$}%
{$-1$}%
{maximal number of presolving rounds the constraint handler participates in (-1: no limit)}%
{}

\printoption{constraints/soc/projectpoint}%
{boolean}%
{FALSE}%
{whether the reference point of a cut should be projected onto the feasible set of the SOC constraint}%
{}

\printoption{constraints/soc/scaling}%
{boolean}%
{TRUE}%
{whether a constraint should be scaled w.r.t. the current gradient norm when checking for feasibility}%
{}

\printoption{constraints/soc/sparsify}%
{boolean}%
{FALSE}%
{whether to sparsify cuts}%
{}

\printoption{constraints/soc/sparsifymaxloss}%
{$0\leq\textrm{real}\leq1$}%
{$0.2$}%
{maximal loss in cut efficacy by sparsification}%
{}

\printoption{constraints/soc/sparsifynzgrowth}%
{$1\leq\textrm{real}$}%
{$1.3$}%
{growth rate of maximal allowed nonzeros in cuts in sparsification}%
{}

\printoption{constraints/soc/timingmask}%
{$1\leq\textrm{integer}\leq15$}%
{$1$}%
{timing when constraint propagation should be called (1:BEFORELP, 2:DURINGLPLOOP, 4:AFTERLPLOOP, 15:ALWAYS))}%
{}

\printoption{constraints/SOS1/delaypresol}%
{boolean}%
{FALSE}%
{should presolving method be delayed, if other presolvers found reductions?}%
{}

\printoption{constraints/SOS1/delayprop}%
{boolean}%
{FALSE}%
{should propagation method be delayed, if other propagators found reductions?}%
{}

\printoption{constraints/SOS1/delaysepa}%
{boolean}%
{FALSE}%
{should separation method be delayed, if other separators found cuts?}%
{}

\printoption{constraints/SOS1/eagerfreq}%
{$-1\leq\textrm{integer}$}%
{$100$}%
{frequency for using all instead of only the useful constraints in separation, propagation and enforcement (-1: never, 0: only in first evaluation)}%
{}

\printoption{constraints/SOS1/maxprerounds}%
{$-1\leq\textrm{integer}$}%
{$-1$}%
{maximal number of presolving rounds the constraint handler participates in (-1: no limit)}%
{}

\printoption{constraints/SOS1/timingmask}%
{$1\leq\textrm{integer}\leq15$}%
{$1$}%
{timing when constraint propagation should be called (1:BEFORELP, 2:DURINGLPLOOP, 4:AFTERLPLOOP, 15:ALWAYS))}%
{}

\printoption{constraints/SOS2/delaypresol}%
{boolean}%
{FALSE}%
{should presolving method be delayed, if other presolvers found reductions?}%
{}

\printoption{constraints/SOS2/delayprop}%
{boolean}%
{FALSE}%
{should propagation method be delayed, if other propagators found reductions?}%
{}

\printoption{constraints/SOS2/delaysepa}%
{boolean}%
{FALSE}%
{should separation method be delayed, if other separators found cuts?}%
{}

\printoption{constraints/SOS2/eagerfreq}%
{$-1\leq\textrm{integer}$}%
{$100$}%
{frequency for using all instead of only the useful constraints in separation, propagation and enforcement (-1: never, 0: only in first evaluation)}%
{}

\printoption{constraints/SOS2/maxprerounds}%
{$-1\leq\textrm{integer}$}%
{$-1$}%
{maximal number of presolving rounds the constraint handler participates in (-1: no limit)}%
{}

\printoption{constraints/SOS2/timingmask}%
{$1\leq\textrm{integer}\leq15$}%
{$1$}%
{timing when constraint propagation should be called (1:BEFORELP, 2:DURINGLPLOOP, 4:AFTERLPLOOP, 15:ALWAYS))}%
{}

\printoption{constraints/superindicator/checkslacktype}%
{boolean}%
{TRUE}%
{should type of slack constraint be checked when creating superindicator constraint?}%
{}

\printoption{constraints/superindicator/delaypresol}%
{boolean}%
{FALSE}%
{should presolving method be delayed, if other presolvers found reductions?}%
{}

\printoption{constraints/superindicator/delayprop}%
{boolean}%
{FALSE}%
{should propagation method be delayed, if other propagators found reductions?}%
{}

\printoption{constraints/superindicator/delaysepa}%
{boolean}%
{FALSE}%
{should separation method be delayed, if other separators found cuts?}%
{}

\printoption{constraints/superindicator/eagerfreq}%
{$-1\leq\textrm{integer}$}%
{$100$}%
{frequency for using all instead of only the useful constraints in separation, propagation and enforcement (-1: never, 0: only in first evaluation)}%
{}

\printoption{constraints/superindicator/maxprerounds}%
{$-1\leq\textrm{integer}$}%
{$-1$}%
{maximal number of presolving rounds the constraint handler participates in (-1: no limit)}%
{}

\printoption{constraints/superindicator/maxupgdcoeflinear}%
{$0\leq\textrm{real}\leq10^{ 15}$}%
{$10000$}%
{maximum big-M coefficient of binary variable in upgrade to a linear constraint (relative to smallest coefficient)}%
{}

\printoption{constraints/superindicator/timingmask}%
{$1\leq\textrm{integer}\leq15$}%
{$1$}%
{timing when constraint propagation should be called (1:BEFORELP, 2:DURINGLPLOOP, 4:AFTERLPLOOP, 15:ALWAYS))}%
{}

\printoption{constraints/superindicator/upgdprioindicator}%
{$-1\leq\textrm{integer}$}%
{$1$}%
{priority for upgrading to an indicator constraint (-1: never)}%
{}

\printoption{constraints/superindicator/upgdpriolinear}%
{$-1\leq\textrm{integer}$}%
{$2$}%
{priority for upgrading to an indicator constraint (-1: never)}%
{}

\printoption{constraints/varbound/delaypresol}%
{boolean}%
{FALSE}%
{should presolving method be delayed, if other presolvers found reductions?}%
{}

\printoption{constraints/varbound/delayprop}%
{boolean}%
{FALSE}%
{should propagation method be delayed, if other propagators found reductions?}%
{}

\printoption{constraints/varbound/delaysepa}%
{boolean}%
{FALSE}%
{should separation method be delayed, if other separators found cuts?}%
{}

\printoption{constraints/varbound/eagerfreq}%
{$-1\leq\textrm{integer}$}%
{$100$}%
{frequency for using all instead of only the useful constraints in separation, propagation and enforcement (-1: never, 0: only in first evaluation)}%
{}

\printoption{constraints/varbound/maxlpcoef}%
{$0\leq\textrm{real}$}%
{$10^{  6}$}%
{maximum coefficient in varbound constraint to be added as a row into LP}%
{}

\printoption{constraints/varbound/maxprerounds}%
{$-1\leq\textrm{integer}$}%
{$-1$}%
{maximal number of presolving rounds the constraint handler participates in (-1: no limit)}%
{}

\printoption{constraints/varbound/presolpairwise}%
{boolean}%
{TRUE}%
{should pairwise constraint comparison be performed in presolving?}%
{}

\printoption{constraints/varbound/timingmask}%
{$1\leq\textrm{integer}\leq15$}%
{$1$}%
{timing when constraint propagation should be called (1:BEFORELP, 2:DURINGLPLOOP, 4:AFTERLPLOOP, 15:ALWAYS))}%
{}

\printoptioncategory{Output}
\printoption{display/avgdualbound/active}%
{$0\leq\textrm{integer}\leq2$}%
{$1$}%
{display activation status of display column $<$avgdualbound$>$ (0: off, 1: auto, 2:on)}%
{}

\printoption{display/conflicts/active}%
{$0\leq\textrm{integer}\leq2$}%
{$1$}%
{display activation status of display column $<$conflicts$>$ (0: off, 1: auto, 2:on)}%
{}

\printoption{display/conss/active}%
{$0\leq\textrm{integer}\leq2$}%
{$1$}%
{display activation status of display column $<$conss$>$ (0: off, 1: auto, 2:on)}%
{}

\printoption{display/curcols/active}%
{$0\leq\textrm{integer}\leq2$}%
{$1$}%
{display activation status of display column $<$curcols$>$ (0: off, 1: auto, 2:on)}%
{}

\printoption{display/curconss/active}%
{$0\leq\textrm{integer}\leq2$}%
{$1$}%
{display activation status of display column $<$curconss$>$ (0: off, 1: auto, 2:on)}%
{}

\printoption{display/curdualbound/active}%
{$0\leq\textrm{integer}\leq2$}%
{$1$}%
{display activation status of display column $<$curdualbound$>$ (0: off, 1: auto, 2:on)}%
{}

\printoption{display/currows/active}%
{$0\leq\textrm{integer}\leq2$}%
{$1$}%
{display activation status of display column $<$currows$>$ (0: off, 1: auto, 2:on)}%
{}

\printoption{display/cutoffbound/active}%
{$0\leq\textrm{integer}\leq2$}%
{$1$}%
{display activation status of display column $<$cutoffbound$>$ (0: off, 1: auto, 2:on)}%
{}

\printoption{display/cuts/active}%
{$0\leq\textrm{integer}\leq2$}%
{$1$}%
{display activation status of display column $<$cuts$>$ (0: off, 1: auto, 2:on)}%
{}

\printoption{display/depth/active}%
{$0\leq\textrm{integer}\leq2$}%
{$1$}%
{display activation status of display column $<$depth$>$ (0: off, 1: auto, 2:on)}%
{}

\printoption{display/dualbound/active}%
{$0\leq\textrm{integer}\leq2$}%
{$1$}%
{display activation status of display column $<$dualbound$>$ (0: off, 1: auto, 2:on)}%
{}

\printoption{display/estimate/active}%
{$0\leq\textrm{integer}\leq2$}%
{$1$}%
{display activation status of display column $<$estimate$>$ (0: off, 1: auto, 2:on)}%
{}

\printoption{display/feasST/active}%
{$0\leq\textrm{integer}\leq2$}%
{$0$}%
{display activation status of display column $<$feasST$>$ (0: off, 1: auto, 2:on)}%
{}

\printoption{display/freq}%
{$-1\leq\textrm{integer}$}%
{$100$}%
{frequency for displaying node information lines}%
{}

\printoption{display/gap/active}%
{$0\leq\textrm{integer}\leq2$}%
{$1$}%
{display activation status of display column $<$gap$>$ (0: off, 1: auto, 2:on)}%
{}

\printoption{display/headerfreq}%
{$-1\leq\textrm{integer}$}%
{$15$}%
{frequency for displaying header lines (every n'th node information line)}%
{}

\printoption{display/lpavgiterations/active}%
{$0\leq\textrm{integer}\leq2$}%
{$1$}%
{display activation status of display column $<$lpavgiterations$>$ (0: off, 1: auto, 2:on)}%
{}

\printoption{display/lpcond/active}%
{$0\leq\textrm{integer}\leq2$}%
{$1$}%
{display activation status of display column $<$lpcond$>$ (0: off, 1: auto, 2:on)}%
{}

\printoption{display/lpinfo}%
{boolean}%
{FALSE}%
{should the LP solver display status messages?}%
{}

\printoption{display/lpiterations/active}%
{$0\leq\textrm{integer}\leq2$}%
{$1$}%
{display activation status of display column $<$lpiterations$>$ (0: off, 1: auto, 2:on)}%
{}

\printoption{display/lpobj/active}%
{$0\leq\textrm{integer}\leq2$}%
{$1$}%
{display activation status of display column $<$lpobj$>$ (0: off, 1: auto, 2:on)}%
{}

\printoption{display/maxdepth/active}%
{$0\leq\textrm{integer}\leq2$}%
{$1$}%
{display activation status of display column $<$maxdepth$>$ (0: off, 1: auto, 2:on)}%
{}

\printoption{display/memused/active}%
{$0\leq\textrm{integer}\leq2$}%
{$1$}%
{display activation status of display column $<$memused$>$ (0: off, 1: auto, 2:on)}%
{}

\printoption{display/nexternbranchcands/active}%
{$0\leq\textrm{integer}\leq2$}%
{$1$}%
{display activation status of display column $<$nexternbranchcands$>$ (0: off, 1: auto, 2:on)}%
{}

\printoption{display/nfrac/active}%
{$0\leq\textrm{integer}\leq2$}%
{$1$}%
{display activation status of display column $<$nfrac$>$ (0: off, 1: auto, 2:on)}%
{}

\printoption{display/nnodes/active}%
{$0\leq\textrm{integer}\leq2$}%
{$1$}%
{display activation status of display column $<$nnodes$>$ (0: off, 1: auto, 2:on)}%
{}

\printoption{display/nodesleft/active}%
{$0\leq\textrm{integer}\leq2$}%
{$1$}%
{display activation status of display column $<$nodesleft$>$ (0: off, 1: auto, 2:on)}%
{}

\printoption{display/nsols/active}%
{$0\leq\textrm{integer}\leq2$}%
{$1$}%
{display activation status of display column $<$nsols$>$ (0: off, 1: auto, 2:on)}%
{}

\printoption{display/plungedepth/active}%
{$0\leq\textrm{integer}\leq2$}%
{$1$}%
{display activation status of display column $<$plungedepth$>$ (0: off, 1: auto, 2:on)}%
{}

\printoption{display/poolsize/active}%
{$0\leq\textrm{integer}\leq2$}%
{$1$}%
{display activation status of display column $<$poolsize$>$ (0: off, 1: auto, 2:on)}%
{}

\printoption{display/primalbound/active}%
{$0\leq\textrm{integer}\leq2$}%
{$1$}%
{display activation status of display column $<$primalbound$>$ (0: off, 1: auto, 2:on)}%
{}

\printoption{display/primalgap/active}%
{$0\leq\textrm{integer}\leq2$}%
{$0$}%
{display activation status of display column $<$primalgap$>$ (0: off, 1: auto, 2:on)}%
{}

\printoption{display/pseudoobj/active}%
{$0\leq\textrm{integer}\leq2$}%
{$1$}%
{display activation status of display column $<$pseudoobj$>$ (0: off, 1: auto, 2:on)}%
{}

\printoption{display/separounds/active}%
{$0\leq\textrm{integer}\leq2$}%
{$1$}%
{display activation status of display column $<$separounds$>$ (0: off, 1: auto, 2:on)}%
{}

\printoption{display/solfound/active}%
{$0\leq\textrm{integer}\leq2$}%
{$1$}%
{display activation status of display column $<$solfound$>$ (0: off, 1: auto, 2:on)}%
{}

\printoption{display/sols/active}%
{$0\leq\textrm{integer}\leq2$}%
{$0$}%
{display activation status of display column $<$sols$>$ (0: off, 1: auto, 2:on)}%
{}

\printoption{display/strongbranchs/active}%
{$0\leq\textrm{integer}\leq2$}%
{$1$}%
{display activation status of display column $<$strongbranchs$>$ (0: off, 1: auto, 2:on)}%
{}

\printoption{display/time/active}%
{$0\leq\textrm{integer}\leq2$}%
{$1$}%
{display activation status of display column $<$time$>$ (0: off, 1: auto, 2:on)}%
{}

\printoption{display/vars/active}%
{$0\leq\textrm{integer}\leq2$}%
{$1$}%
{display activation status of display column $<$vars$>$ (0: off, 1: auto, 2:on)}%
{}

\printoption{display/verblevel}%
{$0\leq\textrm{integer}\leq5$}%
{$4$}%
{verbosity level of output}%
{}

\printoption{display/width}%
{$0\leq\textrm{integer}$}%
{$139$}%
{maximal number of characters in a node information line}%
{}

\printoptioncategory{Heuristics}
\printoption{heuristics/actconsdiving/backtrack}%
{boolean}%
{TRUE}%
{use one level of backtracking if infeasibility is encountered?}%
{}

\printoption{heuristics/actconsdiving/freq}%
{$-1\leq\textrm{integer}$}%
{$-1$}%
{frequency for calling primal heuristic $<$actconsdiving$>$ (-1: never, 0: only at depth freqofs)}%
{}

\printoption{heuristics/actconsdiving/freqofs}%
{$0\leq\textrm{integer}$}%
{$5$}%
{frequency offset for calling primal heuristic $<$actconsdiving$>$}%
{}

\printoption{heuristics/actconsdiving/maxlpiterofs}%
{$0\leq\textrm{integer}$}%
{$1000$}%
{additional number of allowed LP iterations}%
{}

\printoption{heuristics/actconsdiving/maxlpiterquot}%
{$0\leq\textrm{real}$}%
{$0.05$}%
{maximal fraction of diving LP iterations compared to node LP iterations}%
{}

\printoption{heuristics/clique/freq}%
{$-1\leq\textrm{integer}$}%
{$-1$}%
{frequency for calling primal heuristic $<$clique$>$ (-1: never, 0: only at depth freqofs)}%
{}

\printoption{heuristics/clique/freqofs}%
{$0\leq\textrm{integer}$}%
{$0$}%
{frequency offset for calling primal heuristic $<$clique$>$}%
{}

\printoption{heuristics/clique/minfixingrate}%
{$0\leq\textrm{real}\leq1$}%
{$0.5$}%
{minimum percentage of integer variables that have to be fixable}%
{}

\printoption{heuristics/clique/nodesofs}%
{$0\leq\textrm{integer}$}%
{$500$}%
{number of nodes added to the contingent of the total nodes}%
{}

\printoption{heuristics/clique/nodesquot}%
{$0\leq\textrm{real}\leq1$}%
{$0.1$}%
{contingent of sub problem nodes in relation to the number of nodes of the original problem}%
{}

\printoption{heuristics/coefdiving/backtrack}%
{boolean}%
{TRUE}%
{use one level of backtracking if infeasibility is encountered?}%
{}

\printoption{heuristics/coefdiving/freq}%
{$-1\leq\textrm{integer}$}%
{$10$}%
{frequency for calling primal heuristic $<$coefdiving$>$ (-1: never, 0: only at depth freqofs)}%
{}

\printoption{heuristics/coefdiving/freqofs}%
{$0\leq\textrm{integer}$}%
{$1$}%
{frequency offset for calling primal heuristic $<$coefdiving$>$}%
{}

\printoption{heuristics/coefdiving/maxlpiterofs}%
{$0\leq\textrm{integer}$}%
{$1000$}%
{additional number of allowed LP iterations}%
{}

\printoption{heuristics/coefdiving/maxlpiterquot}%
{$0\leq\textrm{real}$}%
{$0.05$}%
{maximal fraction of diving LP iterations compared to node LP iterations}%
{}

\printoption{heuristics/crossover/freq}%
{$-1\leq\textrm{integer}$}%
{$30$}%
{frequency for calling primal heuristic $<$crossover$>$ (-1: never, 0: only at depth freqofs)}%
{}

\printoption{heuristics/crossover/freqofs}%
{$0\leq\textrm{integer}$}%
{$0$}%
{frequency offset for calling primal heuristic $<$crossover$>$}%
{}

\printoption{heuristics/crossover/minfixingrate}%
{$0\leq\textrm{real}\leq1$}%
{$0.666$}%
{minimum percentage of integer variables that have to be fixed}%
{}

\printoption{heuristics/crossover/nodesofs}%
{$0\leq\textrm{integer}$}%
{$500$}%
{number of nodes added to the contingent of the total nodes}%
{}

\printoption{heuristics/crossover/nodesquot}%
{$0\leq\textrm{real}\leq1$}%
{$0.1$}%
{contingent of sub problem nodes in relation to the number of nodes of the original problem}%
{}

\printoption{heuristics/crossover/nusedsols}%
{$2\leq\textrm{integer}$}%
{$3$}%
{number of solutions to be taken into account}%
{}

\printoption{heuristics/dins/freq}%
{$-1\leq\textrm{integer}$}%
{$-1$}%
{frequency for calling primal heuristic $<$dins$>$ (-1: never, 0: only at depth freqofs)}%
{}

\printoption{heuristics/dins/freqofs}%
{$0\leq\textrm{integer}$}%
{$0$}%
{frequency offset for calling primal heuristic $<$dins$>$}%
{}

\printoption{heuristics/dins/minnodes}%
{$0\leq\textrm{integer}$}%
{$500$}%
{minimum number of nodes required to start the subproblem}%
{}

\printoption{heuristics/dins/neighborhoodsize}%
{$1\leq\textrm{integer}$}%
{$18$}%
{radius (using Manhattan metric) of the incumbent's neighborhood to be searched}%
{}

\printoption{heuristics/dins/nodesofs}%
{$0\leq\textrm{integer}$}%
{$5000$}%
{number of nodes added to the contingent of the total nodes}%
{}

\printoption{heuristics/dins/nodesquot}%
{$0\leq\textrm{real}\leq1$}%
{$0.05$}%
{contingent of sub problem nodes in relation to the number of nodes of the original problem}%
{}

\printoption{heuristics/dins/solnum}%
{$1\leq\textrm{integer}$}%
{$5$}%
{number of pool-solutions to be checked for flag array update (for hard fixing of binary variables)}%
{}

\printoption{heuristics/feaspump/alphadiff}%
{$0\leq\textrm{real}\leq1$}%
{$1$}%
{threshold difference for the convex parameter to perform perturbation}%
{}

\printoption{heuristics/feaspump/beforecuts}%
{boolean}%
{TRUE}%
{should the feasibility pump be called at root node before cut separation?}%
{}

\printoption{heuristics/feaspump/freq}%
{$-1\leq\textrm{integer}$}%
{$20$}%
{frequency for calling primal heuristic $<$feaspump$>$ (-1: never, 0: only at depth freqofs)}%
{}

\printoption{heuristics/feaspump/freqofs}%
{$0\leq\textrm{integer}$}%
{$0$}%
{frequency offset for calling primal heuristic $<$feaspump$>$}%
{}

\printoption{heuristics/feaspump/maxlpiterofs}%
{$0\leq\textrm{integer}$}%
{$1000$}%
{additional number of allowed LP iterations}%
{}

\printoption{heuristics/feaspump/maxlpiterquot}%
{$0\leq\textrm{real}$}%
{$0.01$}%
{maximal fraction of diving LP iterations compared to node LP iterations}%
{}

\printoption{heuristics/feaspump/neighborhoodsize}%
{$1\leq\textrm{integer}$}%
{$18$}%
{radius (using Manhattan metric) of the neighborhood to be searched in stage 3}%
{}

\printoption{heuristics/feaspump/objfactor}%
{$0\leq\textrm{real}\leq1$}%
{$1$}%
{factor by which the regard of the objective is decreased in each round, 1.0 for dynamic}%
{}

\printoption{heuristics/feaspump/pertsolfound}%
{boolean}%
{TRUE}%
{should a random perturbation be performed if a feasible solution was found?}%
{}

\printoption{heuristics/feaspump/stage3}%
{boolean}%
{FALSE}%
{should we solve a local branching sub-MIP if no solution could be found?}%
{}

\printoption{heuristics/feaspump/usefp20}%
{boolean}%
{FALSE}%
{should an iterative round-and-propagate scheme be used to find the integral points?}%
{}

\printoption{heuristics/fixandinfer/freq}%
{$-1\leq\textrm{integer}$}%
{$-1$}%
{frequency for calling primal heuristic $<$fixandinfer$>$ (-1: never, 0: only at depth freqofs)}%
{}

\printoption{heuristics/fixandinfer/freqofs}%
{$0\leq\textrm{integer}$}%
{$0$}%
{frequency offset for calling primal heuristic $<$fixandinfer$>$}%
{}

\printoption{heuristics/fracdiving/backtrack}%
{boolean}%
{TRUE}%
{use one level of backtracking if infeasibility is encountered?}%
{}

\printoption{heuristics/fracdiving/freq}%
{$-1\leq\textrm{integer}$}%
{$10$}%
{frequency for calling primal heuristic $<$fracdiving$>$ (-1: never, 0: only at depth freqofs)}%
{}

\printoption{heuristics/fracdiving/freqofs}%
{$0\leq\textrm{integer}$}%
{$3$}%
{frequency offset for calling primal heuristic $<$fracdiving$>$}%
{}

\printoption{heuristics/fracdiving/maxlpiterofs}%
{$0\leq\textrm{integer}$}%
{$1000$}%
{additional number of allowed LP iterations}%
{}

\printoption{heuristics/fracdiving/maxlpiterquot}%
{$0\leq\textrm{real}$}%
{$0.05$}%
{maximal fraction of diving LP iterations compared to node LP iterations}%
{}

\printoption{heuristics/guideddiving/backtrack}%
{boolean}%
{TRUE}%
{use one level of backtracking if infeasibility is encountered?}%
{}

\printoption{heuristics/guideddiving/freq}%
{$-1\leq\textrm{integer}$}%
{$10$}%
{frequency for calling primal heuristic $<$guideddiving$>$ (-1: never, 0: only at depth freqofs)}%
{}

\printoption{heuristics/guideddiving/freqofs}%
{$0\leq\textrm{integer}$}%
{$7$}%
{frequency offset for calling primal heuristic $<$guideddiving$>$}%
{}

\printoption{heuristics/guideddiving/maxlpiterofs}%
{$0\leq\textrm{integer}$}%
{$1000$}%
{additional number of allowed LP iterations}%
{}

\printoption{heuristics/guideddiving/maxlpiterquot}%
{$0\leq\textrm{real}$}%
{$0.05$}%
{maximal fraction of diving LP iterations compared to node LP iterations}%
{}

\printoption{heuristics/intdiving/backtrack}%
{boolean}%
{TRUE}%
{use one level of backtracking if infeasibility is encountered?}%
{}

\printoption{heuristics/intdiving/freq}%
{$-1\leq\textrm{integer}$}%
{$-1$}%
{frequency for calling primal heuristic $<$intdiving$>$ (-1: never, 0: only at depth freqofs)}%
{}

\printoption{heuristics/intdiving/freqofs}%
{$0\leq\textrm{integer}$}%
{$9$}%
{frequency offset for calling primal heuristic $<$intdiving$>$}%
{}

\printoption{heuristics/intdiving/maxlpiterofs}%
{$0\leq\textrm{integer}$}%
{$1000$}%
{additional number of allowed LP iterations}%
{}

\printoption{heuristics/intdiving/maxlpiterquot}%
{$0\leq\textrm{real}$}%
{$0.05$}%
{maximal fraction of diving LP iterations compared to node LP iterations}%
{}

\printoption{heuristics/intshifting/freq}%
{$-1\leq\textrm{integer}$}%
{$10$}%
{frequency for calling primal heuristic $<$intshifting$>$ (-1: never, 0: only at depth freqofs)}%
{}

\printoption{heuristics/intshifting/freqofs}%
{$0\leq\textrm{integer}$}%
{$0$}%
{frequency offset for calling primal heuristic $<$intshifting$>$}%
{}

\printoption{heuristics/linesearchdiving/backtrack}%
{boolean}%
{TRUE}%
{use one level of backtracking if infeasibility is encountered?}%
{}

\printoption{heuristics/linesearchdiving/freq}%
{$-1\leq\textrm{integer}$}%
{$10$}%
{frequency for calling primal heuristic $<$linesearchdiving$>$ (-1: never, 0: only at depth freqofs)}%
{}

\printoption{heuristics/linesearchdiving/freqofs}%
{$0\leq\textrm{integer}$}%
{$6$}%
{frequency offset for calling primal heuristic $<$linesearchdiving$>$}%
{}

\printoption{heuristics/linesearchdiving/maxlpiterofs}%
{$0\leq\textrm{integer}$}%
{$1000$}%
{additional number of allowed LP iterations}%
{}

\printoption{heuristics/linesearchdiving/maxlpiterquot}%
{$0\leq\textrm{real}$}%
{$0.05$}%
{maximal fraction of diving LP iterations compared to node LP iterations}%
{}

\printoption{heuristics/localbranching/freq}%
{$-1\leq\textrm{integer}$}%
{$-1$}%
{frequency for calling primal heuristic $<$localbranching$>$ (-1: never, 0: only at depth freqofs)}%
{}

\printoption{heuristics/localbranching/freqofs}%
{$0\leq\textrm{integer}$}%
{$0$}%
{frequency offset for calling primal heuristic $<$localbranching$>$}%
{}

\printoption{heuristics/localbranching/neighborhoodsize}%
{$1\leq\textrm{integer}$}%
{$18$}%
{radius (using Manhattan metric) of the incumbent's neighborhood to be searched}%
{}

\printoption{heuristics/localbranching/nodesofs}%
{$0\leq\textrm{integer}$}%
{$1000$}%
{number of nodes added to the contingent of the total nodes}%
{}

\printoption{heuristics/localbranching/nodesquot}%
{$0\leq\textrm{real}\leq1$}%
{$0.05$}%
{contingent of sub problem nodes in relation to the number of nodes of the original problem}%
{}

\printoption{heuristics/mutation/freq}%
{$-1\leq\textrm{integer}$}%
{$-1$}%
{frequency for calling primal heuristic $<$mutation$>$ (-1: never, 0: only at depth freqofs)}%
{}

\printoption{heuristics/mutation/freqofs}%
{$0\leq\textrm{integer}$}%
{$8$}%
{frequency offset for calling primal heuristic $<$mutation$>$}%
{}

\printoption{heuristics/mutation/minfixingrate}%
{$10^{- 6}\leq\textrm{real}\leq0.999999$}%
{$0.8$}%
{percentage of integer variables that have to be fixed}%
{}

\printoption{heuristics/mutation/nodesofs}%
{$0\leq\textrm{integer}$}%
{$500$}%
{number of nodes added to the contingent of the total nodes}%
{}

\printoption{heuristics/mutation/nodesquot}%
{$0\leq\textrm{real}\leq1$}%
{$0.1$}%
{contingent of sub problem nodes in relation to the number of nodes of the original problem}%
{}

\printoption{heuristics/nlpdiving/backtrack}%
{boolean}%
{TRUE}%
{use one level of backtracking if infeasibility is encountered?}%
{}

\printoption{heuristics/nlpdiving/fixquot}%
{$0\leq\textrm{real}\leq1$}%
{$0.2$}%
{percentage of fractional variables that should be fixed before the next NLP solve}%
{}

\printoption{heuristics/nlpdiving/freq}%
{$-1\leq\textrm{integer}$}%
{$10$}%
{frequency for calling primal heuristic $<$nlpdiving$>$ (-1: never, 0: only at depth freqofs)}%
{}

\printoption{heuristics/nlpdiving/freqofs}%
{$0\leq\textrm{integer}$}%
{$3$}%
{frequency offset for calling primal heuristic $<$nlpdiving$>$}%
{}

\printoption{heuristics/nlpdiving/maxfeasnlps}%
{$1\leq\textrm{integer}$}%
{$10$}%
{maximal number of NLPs with feasible solution to solve during one dive}%
{}

\printoption{heuristics/nlpdiving/maxnlpiterabs}%
{$0\leq\textrm{integer}$}%
{$200$}%
{minimial absolute number of allowed NLP iterations}%
{}

\printoption{heuristics/nlpdiving/maxnlpiterrel}%
{$0\leq\textrm{integer}$}%
{$10$}%
{additional allowed number of NLP iterations relative to successfully found solutions}%
{}

\printoption{heuristics/nlpdiving/minsuccquot}%
{$0\leq\textrm{real}\leq1$}%
{$0.1$}%
{heuristic will not run if less then this percentage of calls succeeded (0.0: no limit)}%
{}

\printoption{heuristics/nlpdiving/nlpfastfail}%
{boolean}%
{TRUE}%
{should the NLP solver stop early if it converges slow?}%
{}

\printoption{heuristics/nlpdiving/prefercover}%
{boolean}%
{TRUE}%
{should variables in a minimal cover be preferred?}%
{}

\printoption{heuristics/nlpdiving/solvesubmip}%
{boolean}%
{FALSE}%
{should a sub-MIP be solved if all cover variables are fixed?}%
{}

\printoption{heuristics/nlpdiving/varselrule}%
{character}%
{d}%
{which variable selection should be used? ('f'ractionality, 'c'oefficient, 'p'seudocost, 'g'uided, 'd'ouble, 'v'eclen)}%
{}

\printoption{heuristics/objpscostdiving/freq}%
{$-1\leq\textrm{integer}$}%
{$20$}%
{frequency for calling primal heuristic $<$objpscostdiving$>$ (-1: never, 0: only at depth freqofs)}%
{}

\printoption{heuristics/objpscostdiving/freqofs}%
{$0\leq\textrm{integer}$}%
{$4$}%
{frequency offset for calling primal heuristic $<$objpscostdiving$>$}%
{}

\printoption{heuristics/objpscostdiving/maxlpiterofs}%
{$0\leq\textrm{integer}$}%
{$1000$}%
{additional number of allowed LP iterations}%
{}

\printoption{heuristics/objpscostdiving/maxlpiterquot}%
{$0\leq\textrm{real}\leq1$}%
{$0.01$}%
{maximal fraction of diving LP iterations compared to total iteration number}%
{}

\printoption{heuristics/octane/freq}%
{$-1\leq\textrm{integer}$}%
{$-1$}%
{frequency for calling primal heuristic $<$octane$>$ (-1: never, 0: only at depth freqofs)}%
{}

\printoption{heuristics/octane/freqofs}%
{$0\leq\textrm{integer}$}%
{$0$}%
{frequency offset for calling primal heuristic $<$octane$>$}%
{}

\printoption{heuristics/oneopt/freq}%
{$-1\leq\textrm{integer}$}%
{$1$}%
{frequency for calling primal heuristic $<$oneopt$>$ (-1: never, 0: only at depth freqofs)}%
{}

\printoption{heuristics/oneopt/freqofs}%
{$0\leq\textrm{integer}$}%
{$0$}%
{frequency offset for calling primal heuristic $<$oneopt$>$}%
{}

\printoption{heuristics/pscostdiving/backtrack}%
{boolean}%
{TRUE}%
{use one level of backtracking if infeasibility is encountered?}%
{}

\printoption{heuristics/pscostdiving/freq}%
{$-1\leq\textrm{integer}$}%
{$10$}%
{frequency for calling primal heuristic $<$pscostdiving$>$ (-1: never, 0: only at depth freqofs)}%
{}

\printoption{heuristics/pscostdiving/freqofs}%
{$0\leq\textrm{integer}$}%
{$2$}%
{frequency offset for calling primal heuristic $<$pscostdiving$>$}%
{}

\printoption{heuristics/pscostdiving/maxlpiterofs}%
{$0\leq\textrm{integer}$}%
{$1000$}%
{additional number of allowed LP iterations}%
{}

\printoption{heuristics/pscostdiving/maxlpiterquot}%
{$0\leq\textrm{real}$}%
{$0.05$}%
{maximal fraction of diving LP iterations compared to node LP iterations}%
{}

\printoption{heuristics/rens/freq}%
{$-1\leq\textrm{integer}$}%
{$0$}%
{frequency for calling primal heuristic $<$rens$>$ (-1: never, 0: only at depth freqofs)}%
{}

\printoption{heuristics/rens/freqofs}%
{$0\leq\textrm{integer}$}%
{$0$}%
{frequency offset for calling primal heuristic $<$rens$>$}%
{}

\printoption{heuristics/rens/minfixingrate}%
{$0\leq\textrm{real}\leq1$}%
{$0.5$}%
{minimum percentage of integer variables that have to be fixable}%
{}

\printoption{heuristics/rens/nodesofs}%
{$0\leq\textrm{integer}$}%
{$500$}%
{number of nodes added to the contingent of the total nodes}%
{}

\printoption{heuristics/rens/nodesquot}%
{$0\leq\textrm{real}\leq1$}%
{$0.1$}%
{contingent of sub problem nodes in relation to the number of nodes of the original problem}%
{}

\printoption{heuristics/rens/startsol}%
{character}%
{l}%
{solution that is used for fixing values ('l'p relaxation, 'n'lp relaxation)}%
{}

\printoption{heuristics/rins/freq}%
{$-1\leq\textrm{integer}$}%
{$-1$}%
{frequency for calling primal heuristic $<$rins$>$ (-1: never, 0: only at depth freqofs)}%
{}

\printoption{heuristics/rins/freqofs}%
{$0\leq\textrm{integer}$}%
{$5$}%
{frequency offset for calling primal heuristic $<$rins$>$}%
{}

\printoption{heuristics/rins/minfixingrate}%
{$0\leq\textrm{real}\leq1$}%
{$0$}%
{minimum percentage of integer variables that have to be fixed}%
{}

\printoption{heuristics/rins/nodesofs}%
{$0\leq\textrm{integer}$}%
{$500$}%
{number of nodes added to the contingent of the total nodes}%
{}

\printoption{heuristics/rins/nodesquot}%
{$0\leq\textrm{real}\leq1$}%
{$0.1$}%
{contingent of sub problem nodes in relation to the number of nodes of the original problem}%
{}

\printoption{heuristics/rootsoldiving/freq}%
{$-1\leq\textrm{integer}$}%
{$20$}%
{frequency for calling primal heuristic $<$rootsoldiving$>$ (-1: never, 0: only at depth freqofs)}%
{}

\printoption{heuristics/rootsoldiving/freqofs}%
{$0\leq\textrm{integer}$}%
{$5$}%
{frequency offset for calling primal heuristic $<$rootsoldiving$>$}%
{}

\printoption{heuristics/rootsoldiving/maxlpiterofs}%
{$0\leq\textrm{integer}$}%
{$1000$}%
{additional number of allowed LP iterations}%
{}

\printoption{heuristics/rootsoldiving/maxlpiterquot}%
{$0\leq\textrm{real}$}%
{$0.01$}%
{maximal fraction of diving LP iterations compared to node LP iterations}%
{}

\printoption{heuristics/rounding/freq}%
{$-1\leq\textrm{integer}$}%
{$1$}%
{frequency for calling primal heuristic $<$rounding$>$ (-1: never, 0: only at depth freqofs)}%
{}

\printoption{heuristics/rounding/freqofs}%
{$0\leq\textrm{integer}$}%
{$0$}%
{frequency offset for calling primal heuristic $<$rounding$>$}%
{}

\printoption{heuristics/shiftandpropagate/freq}%
{$-1\leq\textrm{integer}$}%
{$0$}%
{frequency for calling primal heuristic $<$shiftandpropagate$>$ (-1: never, 0: only at depth freqofs)}%
{}

\printoption{heuristics/shiftandpropagate/freqofs}%
{$0\leq\textrm{integer}$}%
{$0$}%
{frequency offset for calling primal heuristic $<$shiftandpropagate$>$}%
{}

\printoption{heuristics/shifting/freq}%
{$-1\leq\textrm{integer}$}%
{$10$}%
{frequency for calling primal heuristic $<$shifting$>$ (-1: never, 0: only at depth freqofs)}%
{}

\printoption{heuristics/shifting/freqofs}%
{$0\leq\textrm{integer}$}%
{$0$}%
{frequency offset for calling primal heuristic $<$shifting$>$}%
{}

\printoption{heuristics/simplerounding/freq}%
{$-1\leq\textrm{integer}$}%
{$1$}%
{frequency for calling primal heuristic $<$simplerounding$>$ (-1: never, 0: only at depth freqofs)}%
{}

\printoption{heuristics/simplerounding/freqofs}%
{$0\leq\textrm{integer}$}%
{$0$}%
{frequency offset for calling primal heuristic $<$simplerounding$>$}%
{}

\printoption{heuristics/subnlp/forbidfixings}%
{boolean}%
{TRUE}%
{whether to add constraints that forbid specific fixings that turned out to be infeasible}%
{}

\printoption{heuristics/subnlp/freq}%
{$-1\leq\textrm{integer}$}%
{$1$}%
{frequency for calling primal heuristic $<$subnlp$>$ (-1: never, 0: only at depth freqofs)}%
{}

\printoption{heuristics/subnlp/freqofs}%
{$0\leq\textrm{integer}$}%
{$0$}%
{frequency offset for calling primal heuristic $<$subnlp$>$}%
{}

\printoption{heuristics/subnlp/itermin}%
{$0\leq\textrm{integer}$}%
{$300$}%
{contingent of NLP iterations in relation to the number of nodes in SCIP}%
{}

\printoption{heuristics/subnlp/iteroffset}%
{$0\leq\textrm{integer}$}%
{$500$}%
{number of iterations added to the contingent of the total number of iterations}%
{}

\printoption{heuristics/subnlp/iterquotient}%
{$0\leq\textrm{real}$}%
{$0.1$}%
{contingent of NLP iterations in relation to the number of nodes in SCIP}%
{}

\printoption{heuristics/subnlp/nlpiterlimit}%
{$0\leq\textrm{integer}$}%
{$0$}%
{iteration limit of NLP solver; 0 to use solver default}%
{}

\printoption{heuristics/subnlp/nlptimelimit}%
{$0\leq\textrm{real}$}%
{$0$}%
{time limit of NLP solver; 0 to use solver default}%
{}

\printoption{heuristics/subnlp/nlpverblevel}%
{$0\leq\textrm{integer}$}%
{$0$}%
{verbosity level of NLP solver}%
{}

\printoption{heuristics/subnlp/runalways}%
{boolean}%
{FALSE}%
{whether to run NLP heuristic always if starting point available (does not use iteroffset,iterquot,itermin)}%
{}

\printoption{heuristics/trivial/freq}%
{$-1\leq\textrm{integer}$}%
{$0$}%
{frequency for calling primal heuristic $<$trivial$>$ (-1: never, 0: only at depth freqofs)}%
{}

\printoption{heuristics/trivial/freqofs}%
{$0\leq\textrm{integer}$}%
{$0$}%
{frequency offset for calling primal heuristic $<$trivial$>$}%
{}

\printoption{heuristics/trysol/freq}%
{$-1\leq\textrm{integer}$}%
{$1$}%
{frequency for calling primal heuristic $<$trysol$>$ (-1: never, 0: only at depth freqofs)}%
{}

\printoption{heuristics/trysol/freqofs}%
{$0\leq\textrm{integer}$}%
{$0$}%
{frequency offset for calling primal heuristic $<$trysol$>$}%
{}

\printoption{heuristics/twoopt/freq}%
{$-1\leq\textrm{integer}$}%
{$-1$}%
{frequency for calling primal heuristic $<$twoopt$>$ (-1: never, 0: only at depth freqofs)}%
{}

\printoption{heuristics/twoopt/freqofs}%
{$0\leq\textrm{integer}$}%
{$0$}%
{frequency offset for calling primal heuristic $<$twoopt$>$}%
{}

\printoption{heuristics/undercover/fixingalts}%
{string}%
{li}%
{prioritized sequence of fixing values used ('l'p relaxation, 'n'lp relaxation, 'i'ncumbent solution)}%
{}

\printoption{heuristics/undercover/freq}%
{$-1\leq\textrm{integer}$}%
{$0$}%
{frequency for calling primal heuristic $<$undercover$>$ (-1: never, 0: only at depth freqofs)}%
{}

\printoption{heuristics/undercover/freqofs}%
{$0\leq\textrm{integer}$}%
{$0$}%
{frequency offset for calling primal heuristic $<$undercover$>$}%
{}

\printoption{heuristics/undercover/nodesofs}%
{$0\leq\textrm{integer}$}%
{$500$}%
{number of nodes added to the contingent of the total nodes}%
{}

\printoption{heuristics/undercover/nodesquot}%
{$0\leq\textrm{real}\leq1$}%
{$0.1$}%
{contingent of sub problem nodes in relation to the number of nodes of the original problem}%
{}

\printoption{heuristics/undercover/onlyconvexify}%
{boolean}%
{FALSE}%
{should we only fix variables in order to obtain a convex problem?}%
{}

\printoption{heuristics/undercover/postnlp}%
{boolean}%
{TRUE}%
{should the NLP heuristic be called to polish a feasible solution?}%
{}

\printoption{heuristics/vbounds/freq}%
{$-1\leq\textrm{integer}$}%
{$-1$}%
{frequency for calling primal heuristic $<$vbounds$>$ (-1: never, 0: only at depth freqofs)}%
{}

\printoption{heuristics/vbounds/freqofs}%
{$0\leq\textrm{integer}$}%
{$0$}%
{frequency offset for calling primal heuristic $<$vbounds$>$}%
{}

\printoption{heuristics/vbounds/minfixingrate}%
{$0\leq\textrm{real}\leq1$}%
{$0.5$}%
{minimum percentage of integer variables that have to be fixable}%
{}

\printoption{heuristics/vbounds/nodesofs}%
{$0\leq\textrm{integer}$}%
{$500$}%
{number of nodes added to the contingent of the total nodes}%
{}

\printoption{heuristics/vbounds/nodesquot}%
{$0\leq\textrm{real}\leq1$}%
{$0.1$}%
{contingent of sub problem nodes in relation to the number of nodes of the original problem}%
{}

\printoption{heuristics/veclendiving/backtrack}%
{boolean}%
{TRUE}%
{use one level of backtracking if infeasibility is encountered?}%
{}

\printoption{heuristics/veclendiving/freq}%
{$-1\leq\textrm{integer}$}%
{$10$}%
{frequency for calling primal heuristic $<$veclendiving$>$ (-1: never, 0: only at depth freqofs)}%
{}

\printoption{heuristics/veclendiving/freqofs}%
{$0\leq\textrm{integer}$}%
{$4$}%
{frequency offset for calling primal heuristic $<$veclendiving$>$}%
{}

\printoption{heuristics/veclendiving/maxlpiterofs}%
{$0\leq\textrm{integer}$}%
{$1000$}%
{additional number of allowed LP iterations}%
{}

\printoption{heuristics/veclendiving/maxlpiterquot}%
{$0\leq\textrm{real}$}%
{$0.05$}%
{maximal fraction of diving LP iterations compared to node LP iterations}%
{}

\printoption{heuristics/zeroobj/freq}%
{$-1\leq\textrm{integer}$}%
{$-1$}%
{frequency for calling primal heuristic $<$zeroobj$>$ (-1: never, 0: only at depth freqofs)}%
{}

\printoption{heuristics/zeroobj/freqofs}%
{$0\leq\textrm{integer}$}%
{$0$}%
{frequency offset for calling primal heuristic $<$zeroobj$>$}%
{}

\printoption{heuristics/zeroobj/nodesofs}%
{$0\leq\textrm{integer}$}%
{$100$}%
{number of nodes added to the contingent of the total nodes}%
{}

\printoption{heuristics/zeroobj/nodesquot}%
{$0\leq\textrm{real}\leq1$}%
{$0.1$}%
{contingent of sub problem nodes in relation to the number of nodes of the original problem}%
{}

\printoption{heuristics/zirounding/freq}%
{$-1\leq\textrm{integer}$}%
{$1$}%
{frequency for calling primal heuristic $<$zirounding$>$ (-1: never, 0: only at depth freqofs)}%
{}

\printoption{heuristics/zirounding/freqofs}%
{$0\leq\textrm{integer}$}%
{$0$}%
{frequency offset for calling primal heuristic $<$zirounding$>$}%
{}

\printoptioncategory{Heuristics (advanced options)}
\printoption{heuristics/actconsdiving/maxdepth}%
{$-1\leq\textrm{integer}$}%
{$-1$}%
{maximal depth level to call primal heuristic $<$actconsdiving$>$ (-1: no limit)}%
{}

\printoption{heuristics/actconsdiving/maxdiveavgquot}%
{$0\leq\textrm{real}$}%
{$0$}%
{maximal quotient (curlowerbound - lowerbound)/(avglowerbound - lowerbound) where diving is performed (0.0: no limit)}%
{}

\printoption{heuristics/actconsdiving/maxdiveavgquotnosol}%
{$0\leq\textrm{real}$}%
{$0$}%
{maximal AVGQUOT when no solution was found yet (0.0: no limit)}%
{}

\printoption{heuristics/actconsdiving/maxdiveubquot}%
{$0\leq\textrm{real}\leq1$}%
{$0.8$}%
{maximal quotient (curlowerbound - lowerbound)/(cutoffbound - lowerbound) where diving is performed (0.0: no limit)}%
{}

\printoption{heuristics/actconsdiving/maxdiveubquotnosol}%
{$0\leq\textrm{real}\leq1$}%
{$0.1$}%
{maximal UBQUOT when no solution was found yet (0.0: no limit)}%
{}

\printoption{heuristics/actconsdiving/maxreldepth}%
{$0\leq\textrm{real}\leq1$}%
{$1$}%
{maximal relative depth to start diving}%
{}

\printoption{heuristics/actconsdiving/minreldepth}%
{$0\leq\textrm{real}\leq1$}%
{$0$}%
{minimal relative depth to start diving}%
{}

\printoption{heuristics/actconsdiving/priority}%
{$-536870912\leq\textrm{integer}\leq536870911$}%
{$-1003700$}%
{priority of heuristic $<$actconsdiving$>$}%
{}

\printoption{heuristics/clique/copycuts}%
{boolean}%
{TRUE}%
{should all active cuts from cutpool be copied to constraints in subproblem?}%
{}

\printoption{heuristics/clique/initseed}%
{$0\leq\textrm{integer}$}%
{$0$}%
{initial random seed value to permutate variables}%
{}

\printoption{heuristics/clique/maxdepth}%
{$-1\leq\textrm{integer}$}%
{$-1$}%
{maximal depth level to call primal heuristic $<$clique$>$ (-1: no limit)}%
{}

\printoption{heuristics/clique/maxnodes}%
{$0\leq\textrm{integer}$}%
{$5000$}%
{maximum number of nodes to regard in the subproblem}%
{}

\printoption{heuristics/clique/maxproprounds}%
{$-1\leq\textrm{integer}\leq536870911$}%
{$2$}%
{maximum number of propagation rounds during probing (-1 infinity)}%
{}

\printoption{heuristics/clique/minimprove}%
{$0\leq\textrm{real}\leq1$}%
{$0.01$}%
{factor by which clique heuristic should at least improve the incumbent}%
{}

\printoption{heuristics/clique/minnodes}%
{$0\leq\textrm{integer}$}%
{$500$}%
{minimum number of nodes required to start the subproblem}%
{}

\printoption{heuristics/clique/multiplier}%
{$0\leq\textrm{real}$}%
{$1.1$}%
{value to increase nodenumber to determine the next run}%
{}

\printoption{heuristics/clique/priority}%
{$-536870912\leq\textrm{integer}\leq536870911$}%
{$-1000500$}%
{priority of heuristic $<$clique$>$}%
{}

\printoption{heuristics/coefdiving/maxdepth}%
{$-1\leq\textrm{integer}$}%
{$-1$}%
{maximal depth level to call primal heuristic $<$coefdiving$>$ (-1: no limit)}%
{}

\printoption{heuristics/coefdiving/maxdiveavgquot}%
{$0\leq\textrm{real}$}%
{$0$}%
{maximal quotient (curlowerbound - lowerbound)/(avglowerbound - lowerbound) where diving is performed (0.0: no limit)}%
{}

\printoption{heuristics/coefdiving/maxdiveavgquotnosol}%
{$0\leq\textrm{real}$}%
{$0$}%
{maximal AVGQUOT when no solution was found yet (0.0: no limit)}%
{}

\printoption{heuristics/coefdiving/maxdiveubquot}%
{$0\leq\textrm{real}\leq1$}%
{$0.8$}%
{maximal quotient (curlowerbound - lowerbound)/(cutoffbound - lowerbound) where diving is performed (0.0: no limit)}%
{}

\printoption{heuristics/coefdiving/maxdiveubquotnosol}%
{$0\leq\textrm{real}\leq1$}%
{$0.1$}%
{maximal UBQUOT when no solution was found yet (0.0: no limit)}%
{}

\printoption{heuristics/coefdiving/maxreldepth}%
{$0\leq\textrm{real}\leq1$}%
{$1$}%
{maximal relative depth to start diving}%
{}

\printoption{heuristics/coefdiving/minreldepth}%
{$0\leq\textrm{real}\leq1$}%
{$0$}%
{minimal relative depth to start diving}%
{}

\printoption{heuristics/coefdiving/priority}%
{$-536870912\leq\textrm{integer}\leq536870911$}%
{$-1001000$}%
{priority of heuristic $<$coefdiving$>$}%
{}

\printoption{heuristics/crossover/copycuts}%
{boolean}%
{TRUE}%
{if uselprows == FALSE, should all active cuts from cutpool be copied to constraints in subproblem?}%
{}

\printoption{heuristics/crossover/dontwaitatroot}%
{boolean}%
{FALSE}%
{should the nwaitingnodes parameter be ignored at the root node?}%
{}

\printoption{heuristics/crossover/maxdepth}%
{$-1\leq\textrm{integer}$}%
{$-1$}%
{maximal depth level to call primal heuristic $<$crossover$>$ (-1: no limit)}%
{}

\printoption{heuristics/crossover/maxnodes}%
{$0\leq\textrm{integer}$}%
{$5000$}%
{maximum number of nodes to regard in the subproblem}%
{}

\printoption{heuristics/crossover/minimprove}%
{$0\leq\textrm{real}\leq1$}%
{$0.01$}%
{factor by which Crossover should at least improve the incumbent}%
{}

\printoption{heuristics/crossover/minnodes}%
{$0\leq\textrm{integer}$}%
{$500$}%
{minimum number of nodes required to start the subproblem}%
{}

\printoption{heuristics/crossover/nwaitingnodes}%
{$0\leq\textrm{integer}$}%
{$200$}%
{number of nodes without incumbent change that heuristic should wait}%
{}

\printoption{heuristics/crossover/permute}%
{boolean}%
{FALSE}%
{should the subproblem be permuted to increase diversification?}%
{}

\printoption{heuristics/crossover/priority}%
{$-536870912\leq\textrm{integer}\leq536870911$}%
{$-1104000$}%
{priority of heuristic $<$crossover$>$}%
{}

\printoption{heuristics/crossover/randomization}%
{boolean}%
{TRUE}%
{should the choice which sols to take be randomized?}%
{}

\printoption{heuristics/crossover/uselprows}%
{boolean}%
{FALSE}%
{should subproblem be created out of the rows in the LP rows?}%
{}

\printoption{heuristics/dins/copycuts}%
{boolean}%
{TRUE}%
{if uselprows == FALSE, should all active cuts from cutpool be copied to constraints in subproblem?}%
{}

\printoption{heuristics/dins/maxdepth}%
{$-1\leq\textrm{integer}$}%
{$-1$}%
{maximal depth level to call primal heuristic $<$dins$>$ (-1: no limit)}%
{}

\printoption{heuristics/dins/maxnodes}%
{$0\leq\textrm{integer}$}%
{$5000$}%
{maximum number of nodes to regard in the subproblem}%
{}

\printoption{heuristics/dins/minimprove}%
{$0\leq\textrm{real}\leq1$}%
{$0.01$}%
{factor by which dins should at least improve the incumbent}%
{}

\printoption{heuristics/dins/nwaitingnodes}%
{$0\leq\textrm{integer}$}%
{$0$}%
{number of nodes without incumbent change that heuristic should wait}%
{}

\printoption{heuristics/dins/priority}%
{$-536870912\leq\textrm{integer}\leq536870911$}%
{$-1105000$}%
{priority of heuristic $<$dins$>$}%
{}

\printoption{heuristics/dins/uselprows}%
{boolean}%
{FALSE}%
{should subproblem be created out of the rows in the LP rows?}%
{}

\printoption{heuristics/feaspump/copycuts}%
{boolean}%
{TRUE}%
{should all active cuts from cutpool be copied to constraints in subproblem?}%
{}

\printoption{heuristics/feaspump/cyclelength}%
{$1\leq\textrm{integer}\leq100$}%
{$3$}%
{maximum length of cycles to be checked explicitly in each round}%
{}

\printoption{heuristics/feaspump/maxdepth}%
{$-1\leq\textrm{integer}$}%
{$-1$}%
{maximal depth level to call primal heuristic $<$feaspump$>$ (-1: no limit)}%
{}

\printoption{heuristics/feaspump/maxloops}%
{$-1\leq\textrm{integer}$}%
{$10000$}%
{maximal number of pumping loops (-1: no limit)}%
{}

\printoption{heuristics/feaspump/maxsols}%
{$-1\leq\textrm{integer}$}%
{$10$}%
{total number of feasible solutions found up to which heuristic is called (-1: no limit)}%
{}

\printoption{heuristics/feaspump/maxstallloops}%
{$-1\leq\textrm{integer}$}%
{$10$}%
{maximal number of pumping rounds without fractionality improvement (-1: no limit)}%
{}

\printoption{heuristics/feaspump/minflips}%
{$1\leq\textrm{integer}$}%
{$10$}%
{minimum number of random variables to flip, if a 1-cycle is encountered}%
{}

\printoption{heuristics/feaspump/perturbfreq}%
{$1\leq\textrm{integer}$}%
{$100$}%
{number of iterations until a random perturbation is forced}%
{}

\printoption{heuristics/feaspump/priority}%
{$-536870912\leq\textrm{integer}\leq536870911$}%
{$-1000000$}%
{priority of heuristic $<$feaspump$>$}%
{}

\printoption{heuristics/fixandinfer/maxdepth}%
{$-1\leq\textrm{integer}$}%
{$-1$}%
{maximal depth level to call primal heuristic $<$fixandinfer$>$ (-1: no limit)}%
{}

\printoption{heuristics/fixandinfer/minfixings}%
{$0\leq\textrm{integer}$}%
{$100$}%
{minimal number of fixings to apply before dive may be aborted}%
{}

\printoption{heuristics/fixandinfer/priority}%
{$-536870912\leq\textrm{integer}\leq536870911$}%
{$-500000$}%
{priority of heuristic $<$fixandinfer$>$}%
{}

\printoption{heuristics/fixandinfer/proprounds}%
{$-1\leq\textrm{integer}$}%
{$0$}%
{maximal number of propagation rounds in probing subproblems (-1: no limit, 0: auto)}%
{}

\printoption{heuristics/fracdiving/maxdepth}%
{$-1\leq\textrm{integer}$}%
{$-1$}%
{maximal depth level to call primal heuristic $<$fracdiving$>$ (-1: no limit)}%
{}

\printoption{heuristics/fracdiving/maxdiveavgquot}%
{$0\leq\textrm{real}$}%
{$0$}%
{maximal quotient (curlowerbound - lowerbound)/(avglowerbound - lowerbound) where diving is performed (0.0: no limit)}%
{}

\printoption{heuristics/fracdiving/maxdiveavgquotnosol}%
{$0\leq\textrm{real}$}%
{$0$}%
{maximal AVGQUOT when no solution was found yet (0.0: no limit)}%
{}

\printoption{heuristics/fracdiving/maxdiveubquot}%
{$0\leq\textrm{real}\leq1$}%
{$0.8$}%
{maximal quotient (curlowerbound - lowerbound)/(cutoffbound - lowerbound) where diving is performed (0.0: no limit)}%
{}

\printoption{heuristics/fracdiving/maxdiveubquotnosol}%
{$0\leq\textrm{real}\leq1$}%
{$0.1$}%
{maximal UBQUOT when no solution was found yet (0.0: no limit)}%
{}

\printoption{heuristics/fracdiving/maxreldepth}%
{$0\leq\textrm{real}\leq1$}%
{$1$}%
{maximal relative depth to start diving}%
{}

\printoption{heuristics/fracdiving/minreldepth}%
{$0\leq\textrm{real}\leq1$}%
{$0$}%
{minimal relative depth to start diving}%
{}

\printoption{heuristics/fracdiving/priority}%
{$-536870912\leq\textrm{integer}\leq536870911$}%
{$-1003000$}%
{priority of heuristic $<$fracdiving$>$}%
{}

\printoption{heuristics/guideddiving/maxdepth}%
{$-1\leq\textrm{integer}$}%
{$-1$}%
{maximal depth level to call primal heuristic $<$guideddiving$>$ (-1: no limit)}%
{}

\printoption{heuristics/guideddiving/maxdiveavgquot}%
{$0\leq\textrm{real}$}%
{$0$}%
{maximal quotient (curlowerbound - lowerbound)/(avglowerbound - lowerbound) where diving is performed (0.0: no limit)}%
{}

\printoption{heuristics/guideddiving/maxdiveubquot}%
{$0\leq\textrm{real}\leq1$}%
{$0.8$}%
{maximal quotient (curlowerbound - lowerbound)/(cutoffbound - lowerbound) where diving is performed (0.0: no limit)}%
{}

\printoption{heuristics/guideddiving/maxreldepth}%
{$0\leq\textrm{real}\leq1$}%
{$1$}%
{maximal relative depth to start diving}%
{}

\printoption{heuristics/guideddiving/minreldepth}%
{$0\leq\textrm{real}\leq1$}%
{$0$}%
{minimal relative depth to start diving}%
{}

\printoption{heuristics/guideddiving/priority}%
{$-536870912\leq\textrm{integer}\leq536870911$}%
{$-1007000$}%
{priority of heuristic $<$guideddiving$>$}%
{}

\printoption{heuristics/intdiving/maxdepth}%
{$-1\leq\textrm{integer}$}%
{$-1$}%
{maximal depth level to call primal heuristic $<$intdiving$>$ (-1: no limit)}%
{}

\printoption{heuristics/intdiving/maxdiveavgquot}%
{$0\leq\textrm{real}$}%
{$0$}%
{maximal quotient (curlowerbound - lowerbound)/(avglowerbound - lowerbound) where diving is performed (0.0: no limit)}%
{}

\printoption{heuristics/intdiving/maxdiveavgquotnosol}%
{$0\leq\textrm{real}$}%
{$0$}%
{maximal AVGQUOT when no solution was found yet (0.0: no limit)}%
{}

\printoption{heuristics/intdiving/maxdiveubquot}%
{$0\leq\textrm{real}\leq1$}%
{$0.8$}%
{maximal quotient (curlowerbound - lowerbound)/(cutoffbound - lowerbound) where diving is performed (0.0: no limit)}%
{}

\printoption{heuristics/intdiving/maxdiveubquotnosol}%
{$0\leq\textrm{real}\leq1$}%
{$0.1$}%
{maximal UBQUOT when no solution was found yet (0.0: no limit)}%
{}

\printoption{heuristics/intdiving/maxreldepth}%
{$0\leq\textrm{real}\leq1$}%
{$1$}%
{maximal relative depth to start diving}%
{}

\printoption{heuristics/intdiving/minreldepth}%
{$0\leq\textrm{real}\leq1$}%
{$0$}%
{minimal relative depth to start diving}%
{}

\printoption{heuristics/intdiving/priority}%
{$-536870912\leq\textrm{integer}\leq536870911$}%
{$-1003500$}%
{priority of heuristic $<$intdiving$>$}%
{}

\printoption{heuristics/intshifting/maxdepth}%
{$-1\leq\textrm{integer}$}%
{$-1$}%
{maximal depth level to call primal heuristic $<$intshifting$>$ (-1: no limit)}%
{}

\printoption{heuristics/intshifting/priority}%
{$-536870912\leq\textrm{integer}\leq536870911$}%
{$-10000$}%
{priority of heuristic $<$intshifting$>$}%
{}

\printoption{heuristics/linesearchdiving/maxdepth}%
{$-1\leq\textrm{integer}$}%
{$-1$}%
{maximal depth level to call primal heuristic $<$linesearchdiving$>$ (-1: no limit)}%
{}

\printoption{heuristics/linesearchdiving/maxdiveavgquot}%
{$0\leq\textrm{real}$}%
{$0$}%
{maximal quotient (curlowerbound - lowerbound)/(avglowerbound - lowerbound) where diving is performed (0.0: no limit)}%
{}

\printoption{heuristics/linesearchdiving/maxdiveavgquotnosol}%
{$0\leq\textrm{real}$}%
{$0$}%
{maximal AVGQUOT when no solution was found yet (0.0: no limit)}%
{}

\printoption{heuristics/linesearchdiving/maxdiveubquot}%
{$0\leq\textrm{real}\leq1$}%
{$0.8$}%
{maximal quotient (curlowerbound - lowerbound)/(cutoffbound - lowerbound) where diving is performed (0.0: no limit)}%
{}

\printoption{heuristics/linesearchdiving/maxdiveubquotnosol}%
{$0\leq\textrm{real}\leq1$}%
{$0.1$}%
{maximal UBQUOT when no solution was found yet (0.0: no limit)}%
{}

\printoption{heuristics/linesearchdiving/maxreldepth}%
{$0\leq\textrm{real}\leq1$}%
{$1$}%
{maximal relative depth to start diving}%
{}

\printoption{heuristics/linesearchdiving/minreldepth}%
{$0\leq\textrm{real}\leq1$}%
{$0$}%
{minimal relative depth to start diving}%
{}

\printoption{heuristics/linesearchdiving/priority}%
{$-536870912\leq\textrm{integer}\leq536870911$}%
{$-1006000$}%
{priority of heuristic $<$linesearchdiving$>$}%
{}

\printoption{heuristics/localbranching/copycuts}%
{boolean}%
{TRUE}%
{if uselprows == FALSE, should all active cuts from cutpool be copied to constraints in subproblem?}%
{}

\printoption{heuristics/localbranching/maxdepth}%
{$-1\leq\textrm{integer}$}%
{$-1$}%
{maximal depth level to call primal heuristic $<$localbranching$>$ (-1: no limit)}%
{}

\printoption{heuristics/localbranching/maxnodes}%
{$0\leq\textrm{integer}$}%
{$10000$}%
{maximum number of nodes to regard in the subproblem}%
{}

\printoption{heuristics/localbranching/minimprove}%
{$0\leq\textrm{real}\leq1$}%
{$0.01$}%
{factor by which localbranching should at least improve the incumbent}%
{}

\printoption{heuristics/localbranching/minnodes}%
{$0\leq\textrm{integer}$}%
{$1000$}%
{minimum number of nodes required to start the subproblem}%
{}

\printoption{heuristics/localbranching/nwaitingnodes}%
{$0\leq\textrm{integer}$}%
{$200$}%
{number of nodes without incumbent change that heuristic should wait}%
{}

\printoption{heuristics/localbranching/priority}%
{$-536870912\leq\textrm{integer}\leq536870911$}%
{$-1102000$}%
{priority of heuristic $<$localbranching$>$}%
{}

\printoption{heuristics/localbranching/uselprows}%
{boolean}%
{FALSE}%
{should subproblem be created out of the rows in the LP rows?}%
{}

\printoption{heuristics/mutation/copycuts}%
{boolean}%
{TRUE}%
{if uselprows == FALSE, should all active cuts from cutpool be copied to constraints in subproblem?}%
{}

\printoption{heuristics/mutation/maxdepth}%
{$-1\leq\textrm{integer}$}%
{$-1$}%
{maximal depth level to call primal heuristic $<$mutation$>$ (-1: no limit)}%
{}

\printoption{heuristics/mutation/maxnodes}%
{$0\leq\textrm{integer}$}%
{$5000$}%
{maximum number of nodes to regard in the subproblem}%
{}

\printoption{heuristics/mutation/minimprove}%
{$0\leq\textrm{real}\leq1$}%
{$0.01$}%
{factor by which mutation should at least improve the incumbent}%
{}

\printoption{heuristics/mutation/minnodes}%
{$0\leq\textrm{integer}$}%
{$500$}%
{minimum number of nodes required to start the subproblem}%
{}

\printoption{heuristics/mutation/nwaitingnodes}%
{$0\leq\textrm{integer}$}%
{$200$}%
{number of nodes without incumbent change that heuristic should wait}%
{}

\printoption{heuristics/mutation/priority}%
{$-536870912\leq\textrm{integer}\leq536870911$}%
{$-1103000$}%
{priority of heuristic $<$mutation$>$}%
{}

\printoption{heuristics/mutation/uselprows}%
{boolean}%
{FALSE}%
{should subproblem be created out of the rows in the LP rows?}%
{}

\printoption{heuristics/nlpdiving/lp}%
{boolean}%
{FALSE}%
{should the LP relaxation be solved before the NLP relaxation?}%
{}

\printoption{heuristics/nlpdiving/maxdepth}%
{$-1\leq\textrm{integer}$}%
{$-1$}%
{maximal depth level to call primal heuristic $<$nlpdiving$>$ (-1: no limit)}%
{}

\printoption{heuristics/nlpdiving/maxdiveavgquot}%
{$0\leq\textrm{real}$}%
{$0$}%
{maximal quotient (curlowerbound - lowerbound)/(avglowerbound - lowerbound) where diving is performed (0.0: no limit)}%
{}

\printoption{heuristics/nlpdiving/maxdiveavgquotnosol}%
{$0\leq\textrm{real}$}%
{$0$}%
{maximal AVGQUOT when no solution was found yet (0.0: no limit)}%
{}

\printoption{heuristics/nlpdiving/maxdiveubquot}%
{$0\leq\textrm{real}\leq1$}%
{$0.8$}%
{maximal quotient (curlowerbound - lowerbound)/(cutoffbound - lowerbound) where diving is performed (0.0: no limit)}%
{}

\printoption{heuristics/nlpdiving/maxdiveubquotnosol}%
{$0\leq\textrm{real}\leq1$}%
{$0.1$}%
{maximal UBQUOT when no solution was found yet (0.0: no limit)}%
{}

\printoption{heuristics/nlpdiving/maxreldepth}%
{$0\leq\textrm{real}\leq1$}%
{$1$}%
{maximal relative depth to start diving}%
{}

\printoption{heuristics/nlpdiving/minreldepth}%
{$0\leq\textrm{real}\leq1$}%
{$0$}%
{minimal relative depth to start diving}%
{}

\printoption{heuristics/nlpdiving/nlpstart}%
{character}%
{s}%
{which point should be used as starting point for the NLP solver? ('n'one, last 'f'easible, from dive's'tart)}%
{}

\printoption{heuristics/nlpdiving/preferlpfracs}%
{boolean}%
{FALSE}%
{prefer variables that are also fractional in LP solution?}%
{}

\printoption{heuristics/nlpdiving/priority}%
{$-536870912\leq\textrm{integer}\leq536870911$}%
{$-1003000$}%
{priority of heuristic $<$nlpdiving$>$}%
{}

\printoption{heuristics/objpscostdiving/depthfac}%
{$0\leq\textrm{real}$}%
{$0.5$}%
{maximal diving depth: number of binary/integer variables times depthfac}%
{}

\printoption{heuristics/objpscostdiving/depthfacnosol}%
{$0\leq\textrm{real}$}%
{$2$}%
{maximal diving depth factor if no feasible solution was found yet}%
{}

\printoption{heuristics/objpscostdiving/maxdepth}%
{$-1\leq\textrm{integer}$}%
{$-1$}%
{maximal depth level to call primal heuristic $<$objpscostdiving$>$ (-1: no limit)}%
{}

\printoption{heuristics/objpscostdiving/maxreldepth}%
{$0\leq\textrm{real}\leq1$}%
{$1$}%
{maximal relative depth to start diving}%
{}

\printoption{heuristics/objpscostdiving/maxsols}%
{$-1\leq\textrm{integer}$}%
{$-1$}%
{total number of feasible solutions found up to which heuristic is called (-1: no limit)}%
{}

\printoption{heuristics/objpscostdiving/minreldepth}%
{$0\leq\textrm{real}\leq1$}%
{$0$}%
{minimal relative depth to start diving}%
{}

\printoption{heuristics/objpscostdiving/priority}%
{$-536870912\leq\textrm{integer}\leq536870911$}%
{$-1004000$}%
{priority of heuristic $<$objpscostdiving$>$}%
{}

\printoption{heuristics/octane/ffirst}%
{$1\leq\textrm{integer}$}%
{$10$}%
{number of 0-1-points to be tested at first whether they violate a common row}%
{}

\printoption{heuristics/octane/fmax}%
{$1\leq\textrm{integer}$}%
{$100$}%
{number of 0-1-points to be tested as possible solutions by OCTANE}%
{}

\printoption{heuristics/octane/maxdepth}%
{$-1\leq\textrm{integer}$}%
{$-1$}%
{maximal depth level to call primal heuristic $<$octane$>$ (-1: no limit)}%
{}

\printoption{heuristics/octane/priority}%
{$-536870912\leq\textrm{integer}\leq536870911$}%
{$-1008000$}%
{priority of heuristic $<$octane$>$}%
{}

\printoption{heuristics/octane/useavgnbray}%
{boolean}%
{TRUE}%
{should the weighted average of the nonbasic cone be used as one ray direction?}%
{}

\printoption{heuristics/octane/useavgray}%
{boolean}%
{TRUE}%
{should the average of the basic cone be used as one ray direction?}%
{}

\printoption{heuristics/octane/useavgwgtray}%
{boolean}%
{TRUE}%
{should the weighted average of the basic cone be used as one ray direction?}%
{}

\printoption{heuristics/octane/usediffray}%
{boolean}%
{FALSE}%
{should the difference between the root solution and the current LP solution be used as one ray direction?}%
{}

\printoption{heuristics/octane/usefracspace}%
{boolean}%
{TRUE}%
{execute OCTANE only in the space of fractional variables (TRUE) or in the full space?}%
{}

\printoption{heuristics/octane/useobjray}%
{boolean}%
{TRUE}%
{should the inner normal of the objective be used as one ray direction?}%
{}

\printoption{heuristics/oneopt/beforepresol}%
{boolean}%
{FALSE}%
{should the heuristic be called before presolving?}%
{}

\printoption{heuristics/oneopt/duringroot}%
{boolean}%
{TRUE}%
{should the heuristic be called before and during the root node?}%
{}

\printoption{heuristics/oneopt/forcelpconstruction}%
{boolean}%
{FALSE}%
{should the construction of the LP be forced even if LP solving is deactivated?}%
{}

\printoption{heuristics/oneopt/maxdepth}%
{$-1\leq\textrm{integer}$}%
{$-1$}%
{maximal depth level to call primal heuristic $<$oneopt$>$ (-1: no limit)}%
{}

\printoption{heuristics/oneopt/priority}%
{$-536870912\leq\textrm{integer}\leq536870911$}%
{$-20000$}%
{priority of heuristic $<$oneopt$>$}%
{}

\printoption{heuristics/oneopt/weightedobj}%
{boolean}%
{TRUE}%
{should the objective be weighted with the potential shifting value when sorting the shifting candidates?}%
{}

\printoption{heuristics/pscostdiving/maxdepth}%
{$-1\leq\textrm{integer}$}%
{$-1$}%
{maximal depth level to call primal heuristic $<$pscostdiving$>$ (-1: no limit)}%
{}

\printoption{heuristics/pscostdiving/maxdiveavgquot}%
{$0\leq\textrm{real}$}%
{$0$}%
{maximal quotient (curlowerbound - lowerbound)/(avglowerbound - lowerbound) where diving is performed (0.0: no limit)}%
{}

\printoption{heuristics/pscostdiving/maxdiveavgquotnosol}%
{$0\leq\textrm{real}$}%
{$0$}%
{maximal AVGQUOT when no solution was found yet (0.0: no limit)}%
{}

\printoption{heuristics/pscostdiving/maxdiveubquot}%
{$0\leq\textrm{real}\leq1$}%
{$0.8$}%
{maximal quotient (curlowerbound - lowerbound)/(cutoffbound - lowerbound) where diving is performed (0.0: no limit)}%
{}

\printoption{heuristics/pscostdiving/maxdiveubquotnosol}%
{$0\leq\textrm{real}\leq1$}%
{$0.1$}%
{maximal UBQUOT when no solution was found yet (0.0: no limit)}%
{}

\printoption{heuristics/pscostdiving/maxreldepth}%
{$0\leq\textrm{real}\leq1$}%
{$1$}%
{maximal relative depth to start diving}%
{}

\printoption{heuristics/pscostdiving/minreldepth}%
{$0\leq\textrm{real}\leq1$}%
{$0$}%
{minimal relative depth to start diving}%
{}

\printoption{heuristics/pscostdiving/priority}%
{$-536870912\leq\textrm{integer}\leq536870911$}%
{$-1002000$}%
{priority of heuristic $<$pscostdiving$>$}%
{}

\printoption{heuristics/rens/addallsols}%
{boolean}%
{FALSE}%
{should all subproblem solutions be added to the original SCIP?}%
{}

\printoption{heuristics/rens/binarybounds}%
{boolean}%
{TRUE}%
{should general integers get binary bounds [floor(.),ceil(.)] ?}%
{}

\printoption{heuristics/rens/copycuts}%
{boolean}%
{TRUE}%
{if uselprows == FALSE, should all active cuts from cutpool be copied to constraints in subproblem?}%
{}

\printoption{heuristics/rens/extratime}%
{boolean}%
{FALSE}%
{should the RENS sub-CIP get its own full time limit? This is only for tesing and not recommended!}%
{}

\printoption{heuristics/rens/fullscale}%
{boolean}%
{FALSE}%
{should the RENS sub-CIP be solved with cuts, conflicts, strong branching,... This is only for tesing and not recommended!}%
{}

\printoption{heuristics/rens/maxdepth}%
{$-1\leq\textrm{integer}$}%
{$-1$}%
{maximal depth level to call primal heuristic $<$rens$>$ (-1: no limit)}%
{}

\printoption{heuristics/rens/maxnodes}%
{$0\leq\textrm{integer}$}%
{$5000$}%
{maximum number of nodes to regard in the subproblem}%
{}

\printoption{heuristics/rens/minimprove}%
{$0\leq\textrm{real}\leq1$}%
{$0.01$}%
{factor by which RENS should at least improve the incumbent}%
{}

\printoption{heuristics/rens/minnodes}%
{$0\leq\textrm{integer}$}%
{$500$}%
{minimum number of nodes required to start the subproblem}%
{}

\printoption{heuristics/rens/priority}%
{$-536870912\leq\textrm{integer}\leq536870911$}%
{$-1100000$}%
{priority of heuristic $<$rens$>$}%
{}

\printoption{heuristics/rens/uselprows}%
{boolean}%
{FALSE}%
{should subproblem be created out of the rows in the LP rows?}%
{}

\printoption{heuristics/rins/copycuts}%
{boolean}%
{TRUE}%
{if uselprows == FALSE, should all active cuts from cutpool be copied to constraints in subproblem?}%
{}

\printoption{heuristics/rins/maxdepth}%
{$-1\leq\textrm{integer}$}%
{$-1$}%
{maximal depth level to call primal heuristic $<$rins$>$ (-1: no limit)}%
{}

\printoption{heuristics/rins/maxnodes}%
{$0\leq\textrm{integer}$}%
{$5000$}%
{maximum number of nodes to regard in the subproblem}%
{}

\printoption{heuristics/rins/minimprove}%
{$0\leq\textrm{real}\leq1$}%
{$0.01$}%
{factor by which rins should at least improve the incumbent}%
{}

\printoption{heuristics/rins/minnodes}%
{$0\leq\textrm{integer}$}%
{$500$}%
{minimum number of nodes required to start the subproblem}%
{}

\printoption{heuristics/rins/nwaitingnodes}%
{$0\leq\textrm{integer}$}%
{$200$}%
{number of nodes without incumbent change that heuristic should wait}%
{}

\printoption{heuristics/rins/priority}%
{$-536870912\leq\textrm{integer}\leq536870911$}%
{$-1101000$}%
{priority of heuristic $<$rins$>$}%
{}

\printoption{heuristics/rins/uselprows}%
{boolean}%
{FALSE}%
{should subproblem be created out of the rows in the LP rows?}%
{}

\printoption{heuristics/rootsoldiving/alpha}%
{$0\leq\textrm{real}\leq1$}%
{$0.9$}%
{soft rounding factor to fade out objective coefficients}%
{}

\printoption{heuristics/rootsoldiving/depthfac}%
{$0\leq\textrm{real}$}%
{$0.5$}%
{maximal diving depth: number of binary/integer variables times depthfac}%
{}

\printoption{heuristics/rootsoldiving/depthfacnosol}%
{$0\leq\textrm{real}$}%
{$2$}%
{maximal diving depth factor if no feasible solution was found yet}%
{}

\printoption{heuristics/rootsoldiving/maxdepth}%
{$-1\leq\textrm{integer}$}%
{$-1$}%
{maximal depth level to call primal heuristic $<$rootsoldiving$>$ (-1: no limit)}%
{}

\printoption{heuristics/rootsoldiving/maxreldepth}%
{$0\leq\textrm{real}\leq1$}%
{$1$}%
{maximal relative depth to start diving}%
{}

\printoption{heuristics/rootsoldiving/maxsols}%
{$-1\leq\textrm{integer}$}%
{$-1$}%
{total number of feasible solutions found up to which heuristic is called (-1: no limit)}%
{}

\printoption{heuristics/rootsoldiving/minreldepth}%
{$0\leq\textrm{real}\leq1$}%
{$0$}%
{minimal relative depth to start diving}%
{}

\printoption{heuristics/rootsoldiving/priority}%
{$-536870912\leq\textrm{integer}\leq536870911$}%
{$-1005000$}%
{priority of heuristic $<$rootsoldiving$>$}%
{}

\printoption{heuristics/rounding/maxdepth}%
{$-1\leq\textrm{integer}$}%
{$-1$}%
{maximal depth level to call primal heuristic $<$rounding$>$ (-1: no limit)}%
{}

\printoption{heuristics/rounding/oncepernode}%
{boolean}%
{FALSE}%
{should the heuristic only be called once per node?}%
{}

\printoption{heuristics/rounding/priority}%
{$-536870912\leq\textrm{integer}\leq536870911$}%
{$-1000$}%
{priority of heuristic $<$rounding$>$}%
{}

\printoption{heuristics/rounding/successfactor}%
{$-1\leq\textrm{integer}$}%
{$100$}%
{number of calls per found solution that are considered as standard success, a higher factor causes the heuristic to be called more often}%
{}

\printoption{heuristics/shiftandpropagate/cutoffbreaker}%
{$-1\leq\textrm{integer}\leq1000000$}%
{$15$}%
{The number of cutoffs before heuristic stops}%
{}

\printoption{heuristics/shiftandpropagate/maxdepth}%
{$-1\leq\textrm{integer}$}%
{$-1$}%
{maximal depth level to call primal heuristic $<$shiftandpropagate$>$ (-1: no limit)}%
{}

\printoption{heuristics/shiftandpropagate/nproprounds}%
{$-1\leq\textrm{integer}\leq1000$}%
{$10$}%
{The number of propagation rounds used for each propagation}%
{}

\printoption{heuristics/shiftandpropagate/onlywithoutsol}%
{boolean}%
{TRUE}%
{Should heuristic only be executed if no primal solution was found, yet?}%
{}

\printoption{heuristics/shiftandpropagate/priority}%
{$-536870912\leq\textrm{integer}\leq536870911$}%
{$1000$}%
{priority of heuristic $<$shiftandpropagate$>$}%
{}

\printoption{heuristics/shiftandpropagate/probing}%
{boolean}%
{TRUE}%
{Should domains be reduced by probing?}%
{}

\printoption{heuristics/shiftandpropagate/relax}%
{boolean}%
{TRUE}%
{Should continuous variables be relaxed?}%
{}

\printoption{heuristics/shiftandpropagate/sortkey}%
{character}%
{u}%
{the key for variable sorting: (n)orms or (r)andom}%
{}

\printoption{heuristics/shiftandpropagate/sortvars}%
{boolean}%
{TRUE}%
{Should variables be sorted for the heuristic?}%
{}

\printoption{heuristics/shifting/maxdepth}%
{$-1\leq\textrm{integer}$}%
{$-1$}%
{maximal depth level to call primal heuristic $<$shifting$>$ (-1: no limit)}%
{}

\printoption{heuristics/shifting/priority}%
{$-536870912\leq\textrm{integer}\leq536870911$}%
{$-5000$}%
{priority of heuristic $<$shifting$>$}%
{}

\printoption{heuristics/simplerounding/maxdepth}%
{$-1\leq\textrm{integer}$}%
{$-1$}%
{maximal depth level to call primal heuristic $<$simplerounding$>$ (-1: no limit)}%
{}

\printoption{heuristics/simplerounding/oncepernode}%
{boolean}%
{FALSE}%
{should the heuristic only be called once per node?}%
{}

\printoption{heuristics/simplerounding/priority}%
{$-536870912\leq\textrm{integer}\leq536870911$}%
{$0$}%
{priority of heuristic $<$simplerounding$>$}%
{}

\printoption{heuristics/subnlp/keepcopy}%
{boolean}%
{TRUE}%
{whether to keep SCIP copy or to create new copy each time heuristic is applied}%
{}

\printoption{heuristics/subnlp/maxdepth}%
{$-1\leq\textrm{integer}$}%
{$-1$}%
{maximal depth level to call primal heuristic $<$subnlp$>$ (-1: no limit)}%
{}

\printoption{heuristics/subnlp/maxpresolverounds}%
{$-1\leq\textrm{integer}$}%
{$-1$}%
{limit on number of presolve rounds in sub-SCIP (-1 for unlimited, 0 for no presolve)}%
{}

\printoption{heuristics/subnlp/minimprove}%
{$0\leq\textrm{real}\leq1$}%
{$0.01$}%
{factor by which NLP heuristic should at least improve the incumbent}%
{}

\printoption{heuristics/subnlp/nlpoptfile}%
{string}%
{}%
{name of an NLP solver specific options file}%
{}

\printoption{heuristics/subnlp/priority}%
{$-536870912\leq\textrm{integer}\leq536870911$}%
{$-2000000$}%
{priority of heuristic $<$subnlp$>$}%
{}

\printoption{heuristics/subnlp/resolvefromscratch}%
{boolean}%
{TRUE}%
{should the NLP resolve be started from the original starting point or the infeasible solution?}%
{}

\printoption{heuristics/subnlp/resolvetolfactor}%
{$0\leq\textrm{real}\leq1$}%
{$0.001$}%
{if SCIP does not accept a NLP feasible solution, resolve NLP with feas. tolerance reduced by this factor (set to 1.0 to turn off resolve)}%
{}

\printoption{heuristics/trivial/maxdepth}%
{$-1\leq\textrm{integer}$}%
{$-1$}%
{maximal depth level to call primal heuristic $<$trivial$>$ (-1: no limit)}%
{}

\printoption{heuristics/trivial/priority}%
{$-536870912\leq\textrm{integer}\leq536870911$}%
{$10000$}%
{priority of heuristic $<$trivial$>$}%
{}

\printoption{heuristics/trysol/maxdepth}%
{$-1\leq\textrm{integer}$}%
{$-1$}%
{maximal depth level to call primal heuristic $<$trysol$>$ (-1: no limit)}%
{}

\printoption{heuristics/trysol/priority}%
{$-536870912\leq\textrm{integer}\leq536870911$}%
{$-3000000$}%
{priority of heuristic $<$trysol$>$}%
{}

\printoption{heuristics/twoopt/intopt}%
{boolean}%
{FALSE}%
{ Should Integer-2-Optimization be applied or not?}%
{}

\printoption{heuristics/twoopt/matchingrate}%
{$0\leq\textrm{real}\leq1$}%
{$0.5$}%
{parameter to determine the percentage of rows two variables have to share before they are considered equal}%
{}

\printoption{heuristics/twoopt/maxdepth}%
{$-1\leq\textrm{integer}$}%
{$-1$}%
{maximal depth level to call primal heuristic $<$twoopt$>$ (-1: no limit)}%
{}

\printoption{heuristics/twoopt/maxnslaves}%
{$-1\leq\textrm{integer}\leq1000000$}%
{$199$}%
{maximum number of slaves for one master variable}%
{}

\printoption{heuristics/twoopt/priority}%
{$-536870912\leq\textrm{integer}\leq536870911$}%
{$-20100$}%
{priority of heuristic $<$twoopt$>$}%
{}

\printoption{heuristics/twoopt/waitingnodes}%
{$0\leq\textrm{integer}\leq10000$}%
{$0$}%
{user parameter to determine number of nodes to wait after last best solution before calling heuristic}%
{}

\printoption{heuristics/undercover/beforecuts}%
{boolean}%
{TRUE}%
{should the heuristic be called at root node before cut separation?}%
{}

\printoption{heuristics/undercover/conflictweight}%
{$\textrm{real}$}%
{$1000$}%
{weight for conflict score in fixing order}%
{}

\printoption{heuristics/undercover/copycuts}%
{boolean}%
{TRUE}%
{should all active cuts from cutpool be copied to constraints in subproblem?}%
{}

\printoption{heuristics/undercover/coverbd}%
{boolean}%
{FALSE}%
{should bounddisjunction constraints be covered (or just copied)?}%
{}

\printoption{heuristics/undercover/coveringobj}%
{character}%
{u}%
{objective function of the covering problem (influenced nonlinear 'c'onstraints/'t'erms, 'd'omain size, 'l'ocks, 'm'in of up/down locks, 'u'nit penalties)}%
{}

\printoption{heuristics/undercover/cutoffweight}%
{$0\leq\textrm{real}$}%
{$1$}%
{weight for cutoff score in fixing order}%
{}

\printoption{heuristics/undercover/fixingorder}%
{character}%
{v}%
{order in which variables should be fixed (increasing 'C'onflict score, decreasing 'c'onflict score, increasing 'V'ariable index, decreasing 'v'ariable index}%
{}

\printoption{heuristics/undercover/fixintfirst}%
{boolean}%
{FALSE}%
{should integer variables in the cover be fixed first?}%
{}

\printoption{heuristics/undercover/inferenceweight}%
{$\textrm{real}$}%
{$1$}%
{weight for inference score in fixing order}%
{}

\printoption{heuristics/undercover/locksrounding}%
{boolean}%
{TRUE}%
{shall LP values for integer vars be rounded according to locks?}%
{}

\printoption{heuristics/undercover/maxbacktracks}%
{$0\leq\textrm{integer}$}%
{$6$}%
{maximum number of backtracks in fix-and-propagate}%
{}

\printoption{heuristics/undercover/maxcoversizeconss}%
{$0\leq\textrm{real}$}%
{$\infty$}%
{maximum coversize maximum coversize (as ratio to the percentage of non-affected constraints)}%
{}

\printoption{heuristics/undercover/maxcoversizevars}%
{$0\leq\textrm{real}\leq1$}%
{$1$}%
{maximum coversize (as fraction of total number of variables)}%
{}

\printoption{heuristics/undercover/maxdepth}%
{$-1\leq\textrm{integer}$}%
{$-1$}%
{maximal depth level to call primal heuristic $<$undercover$>$ (-1: no limit)}%
{}

\printoption{heuristics/undercover/maxnodes}%
{$0\leq\textrm{integer}$}%
{$500$}%
{maximum number of nodes to regard in the subproblem}%
{}

\printoption{heuristics/undercover/maxrecovers}%
{$0\leq\textrm{integer}$}%
{$0$}%
{maximum number of recoverings}%
{}

\printoption{heuristics/undercover/maxreorders}%
{$0\leq\textrm{integer}$}%
{$1$}%
{maximum number of reorderings of the fixing order}%
{}

\printoption{heuristics/undercover/mincoveredabs}%
{$0\leq\textrm{integer}$}%
{$5$}%
{minimum number of nonlinear constraints in the original problem}%
{}

\printoption{heuristics/undercover/mincoveredrel}%
{$0\leq\textrm{real}\leq1$}%
{$0.15$}%
{minimum percentage of nonlinear constraints in the original problem}%
{}

\printoption{heuristics/undercover/minimprove}%
{$-1\leq\textrm{real}\leq1$}%
{$0$}%
{factor by which the heuristic should at least improve the incumbent}%
{}

\printoption{heuristics/undercover/minnodes}%
{$0\leq\textrm{integer}$}%
{$500$}%
{minimum number of nodes required to start the subproblem}%
{}

\printoption{heuristics/undercover/priority}%
{$-536870912\leq\textrm{integer}\leq536870911$}%
{$-1110000$}%
{priority of heuristic $<$undercover$>$}%
{}

\printoption{heuristics/undercover/recoverdiv}%
{$0\leq\textrm{real}\leq1$}%
{$0.9$}%
{fraction of covering variables in the last cover which need to change their value when recovering}%
{}

\printoption{heuristics/undercover/reusecover}%
{boolean}%
{FALSE}%
{shall the cover be reused if a conflict was added after an infeasible subproblem?}%
{}

\printoption{heuristics/vbounds/copycuts}%
{boolean}%
{TRUE}%
{should all active cuts from cutpool be copied to constraints in subproblem?}%
{}

\printoption{heuristics/vbounds/maxdepth}%
{$-1\leq\textrm{integer}$}%
{$-1$}%
{maximal depth level to call primal heuristic $<$vbounds$>$ (-1: no limit)}%
{}

\printoption{heuristics/vbounds/maxnodes}%
{$0\leq\textrm{integer}$}%
{$5000$}%
{maximum number of nodes to regard in the subproblem}%
{}

\printoption{heuristics/vbounds/maxproprounds}%
{$-1\leq\textrm{integer}\leq536870911$}%
{$2$}%
{maximum number of propagation rounds during probing (-1 infinity)}%
{}

\printoption{heuristics/vbounds/minimprove}%
{$0\leq\textrm{real}\leq1$}%
{$0.01$}%
{factor by which vbounds heuristic should at least improve the incumbent}%
{}

\printoption{heuristics/vbounds/minnodes}%
{$0\leq\textrm{integer}$}%
{$500$}%
{minimum number of nodes required to start the subproblem}%
{}

\printoption{heuristics/vbounds/priority}%
{$-536870912\leq\textrm{integer}\leq536870911$}%
{$-1106000$}%
{priority of heuristic $<$vbounds$>$}%
{}

\printoption{heuristics/veclendiving/maxdepth}%
{$-1\leq\textrm{integer}$}%
{$-1$}%
{maximal depth level to call primal heuristic $<$veclendiving$>$ (-1: no limit)}%
{}

\printoption{heuristics/veclendiving/maxdiveavgquot}%
{$0\leq\textrm{real}$}%
{$0$}%
{maximal quotient (curlowerbound - lowerbound)/(avglowerbound - lowerbound) where diving is performed (0.0: no limit)}%
{}

\printoption{heuristics/veclendiving/maxdiveavgquotnosol}%
{$0\leq\textrm{real}$}%
{$0$}%
{maximal AVGQUOT when no solution was found yet (0.0: no limit)}%
{}

\printoption{heuristics/veclendiving/maxdiveubquot}%
{$0\leq\textrm{real}\leq1$}%
{$0.8$}%
{maximal quotient (curlowerbound - lowerbound)/(cutoffbound - lowerbound) where diving is performed (0.0: no limit)}%
{}

\printoption{heuristics/veclendiving/maxdiveubquotnosol}%
{$0\leq\textrm{real}\leq1$}%
{$0.1$}%
{maximal UBQUOT when no solution was found yet (0.0: no limit)}%
{}

\printoption{heuristics/veclendiving/maxreldepth}%
{$0\leq\textrm{real}\leq1$}%
{$1$}%
{maximal relative depth to start diving}%
{}

\printoption{heuristics/veclendiving/minreldepth}%
{$0\leq\textrm{real}\leq1$}%
{$0$}%
{minimal relative depth to start diving}%
{}

\printoption{heuristics/veclendiving/priority}%
{$-536870912\leq\textrm{integer}\leq536870911$}%
{$-1003100$}%
{priority of heuristic $<$veclendiving$>$}%
{}

\printoption{heuristics/zeroobj/addallsols}%
{boolean}%
{FALSE}%
{should all subproblem solutions be added to the original SCIP?}%
{}

\printoption{heuristics/zeroobj/maxdepth}%
{$-1\leq\textrm{integer}$}%
{$0$}%
{maximal depth level to call primal heuristic $<$zeroobj$>$ (-1: no limit)}%
{}

\printoption{heuristics/zeroobj/maxlpiters}%
{$-1\leq\textrm{integer}$}%
{$5000$}%
{maximum number of LP iterations to be performed in the subproblem}%
{}

\printoption{heuristics/zeroobj/maxnodes}%
{$0\leq\textrm{integer}$}%
{$1000$}%
{maximum number of nodes to regard in the subproblem}%
{}

\printoption{heuristics/zeroobj/minimprove}%
{$0\leq\textrm{real}\leq1$}%
{$0.01$}%
{factor by which zeroobj should at least improve the incumbent}%
{}

\printoption{heuristics/zeroobj/minnodes}%
{$0\leq\textrm{integer}$}%
{$100$}%
{minimum number of nodes required to start the subproblem}%
{}

\printoption{heuristics/zeroobj/onlywithoutsol}%
{boolean}%
{TRUE}%
{should heuristic only be executed if no primal solution was found, yet?}%
{}

\printoption{heuristics/zeroobj/priority}%
{$-536870912\leq\textrm{integer}\leq536870911$}%
{$100$}%
{priority of heuristic $<$zeroobj$>$}%
{}

\printoption{heuristics/zirounding/maxdepth}%
{$-1\leq\textrm{integer}$}%
{$-1$}%
{maximal depth level to call primal heuristic $<$zirounding$>$ (-1: no limit)}%
{}

\printoption{heuristics/zirounding/maxroundingloops}%
{$-1\leq\textrm{integer}$}%
{$2$}%
{determines maximum number of rounding loops}%
{}

\printoption{heuristics/zirounding/minstopncalls}%
{$1\leq\textrm{integer}$}%
{$1000$}%
{determines the minimum number of calls before percentage-based deactivation of Zirounding is applied}%
{}

\printoption{heuristics/zirounding/priority}%
{$-536870912\leq\textrm{integer}\leq536870911$}%
{$-500$}%
{priority of heuristic $<$zirounding$>$}%
{}

\printoption{heuristics/zirounding/stoppercentage}%
{$0\leq\textrm{real}\leq1$}%
{$0.02$}%
{if percentage of found solutions falls below this parameter, Zirounding will be deactivated}%
{}

\printoption{heuristics/zirounding/stopziround}%
{boolean}%
{TRUE}%
{flag to determine if Zirounding is deactivated after a certain percentage of unsuccessful calls}%
{}

\printoptioncategory{Limits}
\printoption{limits/absgap}%
{$0\leq\textrm{real}$}%
{$0$}%
{solving stops, if the absolute gap = $|$primalbound - dualbound$|$ is below the given value}%
{}

\printoption{limits/bestsol}%
{$-1\leq\textrm{integer}$}%
{$-1$}%
{solving stops, if the given number of solution improvements were found (-1: no limit)}%
{}

\printoption{limits/gap}%
{$0\leq\textrm{real}$}%
{$0.1$}%
{solving stops, if the relative gap = $|$primal - dual$|$/MIN($|$dual$|$,$|$primal$|$) is below the given value}%
{}

\printoption{limits/maxorigsol}%
{$0\leq\textrm{integer}$}%
{$10$}%
{maximal number of solutions candidates to store in the solution storage of the original problem}%
{}

\printoption{limits/maxsol}%
{$1\leq\textrm{integer}$}%
{$100$}%
{maximal number of solutions to store in the solution storage}%
{}

\printoption{limits/memory}%
{$0\leq\textrm{real}$}%
{$\infty$}%
{maximal memory usage in MB; reported memory usage is lower than real memory usage!}%
{}

\printoption{limits/nodes}%
{$-1\leq\textrm{integer}$}%
{$-1$}%
{maximal number of nodes to process (-1: no limit)}%
{}

\printoption{limits/restarts}%
{$-1\leq\textrm{integer}$}%
{$-1$}%
{solving stops, if the given number of restarts was triggered (-1: no limit)}%
{}

\printoption{limits/solutions}%
{$-1\leq\textrm{integer}$}%
{$-1$}%
{solving stops, if the given number of solutions were found (-1: no limit)}%
{}

\printoption{limits/stallnodes}%
{$-1\leq\textrm{integer}$}%
{$-1$}%
{solving stops, if the given number of nodes was processed since the last improvement of the primal solution value (-1: no limit)}%
{}

\printoption{limits/time}%
{$0\leq\textrm{real}$}%
{$1000$}%
{maximal time in seconds to run}%
{}

\printoption{limits/totalnodes}%
{$-1\leq\textrm{integer}$}%
{$-1$}%
{maximal number of total nodes (incl. restarts) to process (-1: no limit)}%
{}

\printoptioncategory{LP}
\printoption{lp/initalgorithm}%
{character}%
{s}%
{LP algorithm for solving initial LP relaxations (automatic 's'implex, 'p'rimal simplex, 'd'ual simplex, 'b'arrier, barrier with 'c'rossover)}%
{}

\printoption{lp/pricing}%
{character}%
{l}%
{LP pricing strategy ('l'pi default, 'a'uto, 'f'ull pricing, 'p'artial, 's'teepest edge pricing, 'q'uickstart steepest edge pricing, 'd'evex pricing)}%
{}

\printoption{lp/resolvealgorithm}%
{character}%
{s}%
{LP algorithm for resolving LP relaxations if a starting basis exists (automatic 's'implex, 'p'rimal simplex, 'd'ual simplex, 'b'arrier, barrier with 'c'rossover)}%
{}

\printoption{lp/solvedepth}%
{$-1\leq\textrm{integer}$}%
{$-1$}%
{maximal depth for solving LP at the nodes (-1: no depth limit)}%
{}

\printoption{lp/solvefreq}%
{$-1\leq\textrm{integer}$}%
{$1$}%
{frequency for solving LP at the nodes (-1: never; 0: only root LP)}%
{}

\printoptioncategory{LP (advanced options)}
\printoption{lp/checkfeas}%
{boolean}%
{TRUE}%
{should LP solutions be checked, resolving LP when numerical troubles occur?}%
{}

\printoption{lp/checkstability}%
{boolean}%
{TRUE}%
{should LP solver's return status be checked for stability?}%
{}

\printoption{lp/cleanupcols}%
{boolean}%
{FALSE}%
{should new non-basic columns be removed after LP solving?}%
{}

\printoption{lp/cleanupcolsroot}%
{boolean}%
{FALSE}%
{should new non-basic columns be removed after root LP solving?}%
{}

\printoption{lp/cleanuprows}%
{boolean}%
{TRUE}%
{should new basic rows be removed after LP solving?}%
{}

\printoption{lp/cleanuprowsroot}%
{boolean}%
{TRUE}%
{should new basic rows be removed after root LP solving?}%
{}

\printoption{lp/clearinitialprobinglp}%
{boolean}%
{TRUE}%
{should lp state be cleared at the end of probing mode when lp was initially unsolved, e.g., when called right after presolving?}%
{}

\printoption{lp/colagelimit}%
{$-1\leq\textrm{integer}$}%
{$10$}%
{maximum age a dynamic column can reach before it is deleted from the LP (-1: don't delete columns due to aging)}%
{}

\printoption{lp/fastmip}%
{$0\leq\textrm{integer}\leq1$}%
{$1$}%
{which FASTMIP setting of LP solver should be used? 0: off, 1: low}%
{}

\printoption{lp/freesolvalbuffers}%
{boolean}%
{FALSE}%
{should the buffers for storing LP solution values during diving be freed at end of diving?}%
{}

\printoption{lp/iterlim}%
{$-1\leq\textrm{integer}$}%
{$-1$}%
{iteration limit for each single LP solve (-1: no limit)}%
{}

\printoption{lp/lexdualalgo}%
{boolean}%
{FALSE}%
{should the lexicographic dual alogrithm be used?}%
{}

\printoption{lp/lexdualbasic}%
{boolean}%
{FALSE}%
{choose fractional basic variables in lexicographic dual algorithm?}%
{}

\printoption{lp/lexdualmaxrounds}%
{$-1\leq\textrm{integer}$}%
{$2$}%
{maximum number of rounds in the  lexicographic dual algorithm (-1: unbounded)}%
{}

\printoption{lp/lexdualrootonly}%
{boolean}%
{TRUE}%
{should the lexicographic dual algorithm be applied only at the root node}%
{}

\printoption{lp/lexdualstalling}%
{boolean}%
{TRUE}%
{turn on the lex dual algorithm only when stalling?}%
{}

\printoption{lp/presolving}%
{boolean}%
{TRUE}%
{should presolving of LP solver be used?}%
{}

\printoption{lp/resolveiterfac}%
{$-1\leq\textrm{real}$}%
{$-1$}%
{factor of average LP iterations that is used as LP iteration limit for LP resolve (-1: unlimited)}%
{}

\printoption{lp/resolveitermin}%
{$1\leq\textrm{integer}$}%
{$1000$}%
{minimum number of iterations that are allowed for LP resolve}%
{}

\printoption{lp/resolverestore}%
{boolean}%
{FALSE}%
{should the LP be resolved to restore the state at start of diving (if FALSE we buffer the solution values)?}%
{}

\printoption{lp/rootiterlim}%
{$-1\leq\textrm{integer}$}%
{$-1$}%
{iteration limit for initial root LP solve (-1: no limit)}%
{}

\printoption{lp/rowagelimit}%
{$-1\leq\textrm{integer}$}%
{$10$}%
{maximum age a dynamic row can reach before it is deleted from the LP (-1: don't delete rows due to aging)}%
{}

\printoption{lp/rowrepswitch}%
{$-1\leq\textrm{real}$}%
{$-1$}%
{simplex algorithm shall use row representation of the basis if number of rows divided by number of columns exceeds this value (-1.0 to disable row representation)}%
{}

\printoption{lp/scaling}%
{boolean}%
{TRUE}%
{should scaling of LP solver be used?}%
{}

\printoption{lp/threads}%
{$0\leq\textrm{integer}\leq64$}%
{$0$}%
{number of threads used for solving the LP (0: automatic)}%
{}

\printoptioncategory{Memory}
\printoption{memory/savefac}%
{$0\leq\textrm{real}\leq1$}%
{$0.8$}%
{fraction of maximal memory usage resulting in switch to memory saving mode}%
{}

\printoptioncategory{Memory (advanced options)}
\printoption{memory/arraygrowfac}%
{$1\leq\textrm{real}\leq10$}%
{$1.2$}%
{memory growing factor for dynamically allocated arrays}%
{}

\printoption{memory/arraygrowinit}%
{$0\leq\textrm{integer}$}%
{$4$}%
{initial size of dynamically allocated arrays}%
{}

\printoption{memory/pathgrowfac}%
{$1\leq\textrm{real}\leq10$}%
{$2$}%
{memory growing factor for path array}%
{}

\printoption{memory/pathgrowinit}%
{$0\leq\textrm{integer}$}%
{$256$}%
{initial size of path array}%
{}

\printoption{memory/treegrowfac}%
{$1\leq\textrm{real}\leq10$}%
{$2$}%
{memory growing factor for tree array}%
{}

\printoption{memory/treegrowinit}%
{$0\leq\textrm{integer}$}%
{$65536$}%
{initial size of tree array}%
{}

\printoptioncategory{Micellaneous}
\printoption{misc/catchctrlc}%
{boolean}%
{TRUE}%
{should the CTRL-C interrupt be caught by SCIP?}%
{}

\printoption{misc/estimexternmem}%
{boolean}%
{TRUE}%
{should the usage of external memory be estimated?}%
{}

\printoption{misc/improvingsols}%
{boolean}%
{FALSE}%
{should only solutions be checked which improve the primal bound}%
{}

\printoption{misc/permutationseed}%
{$-1\leq\textrm{integer}$}%
{$-1$}%
{seed value for permuting the problem after the problem was transformed (-1: no permutation)}%
{}

\printoption{misc/printreason}%
{boolean}%
{TRUE}%
{should the reason be printed if a given start solution is infeasible}%
{}

\printoption{misc/resetstat}%
{boolean}%
{TRUE}%
{should the statistics be reseted if the transformed problem is freed (in case of a benders decomposition this parameter should be set to FALSE)}%
{}

\printoption{misc/transorigsols}%
{boolean}%
{TRUE}%
{should SCIP try to transfer original solutions to the extended space (after presolving)?}%
{}

\printoption{misc/useconstable}%
{boolean}%
{TRUE}%
{should a hashtable be used to map from constraint names to constraints?}%
{}

\printoption{misc/usesmalltables}%
{boolean}%
{FALSE}%
{should smaller hashtables be used? yields better performance for small problems with about 100 variables}%
{}

\printoption{misc/usevartable}%
{boolean}%
{TRUE}%
{should a hashtable be used to map from variable names to variables?}%
{}

\printoptioncategory{Node Selection}
\printoption{nodeselection/bfs/stdpriority}%
{$-536870912\leq\textrm{integer}\leq536870911$}%
{$100000$}%
{priority of node selection rule $<$bfs$>$ in standard mode}%
{}

\printoption{nodeselection/childsel}%
{character}%
{h}%
{child selection rule ('d'own, 'u'p, 'p'seudo costs, 'i'nference, 'l'p value, 'r'oot LP value difference, 'h'ybrid inference/root LP value difference)}%
{}

\printoption{nodeselection/dfs/stdpriority}%
{$-536870912\leq\textrm{integer}\leq536870911$}%
{$0$}%
{priority of node selection rule $<$dfs$>$ in standard mode}%
{}

\printoption{nodeselection/estimate/bestnodefreq}%
{$0\leq\textrm{integer}$}%
{$10$}%
{frequency at which the best node instead of the best estimate is selected (0: never)}%
{}

\printoption{nodeselection/estimate/breadthfirstdepth}%
{$-1\leq\textrm{integer}$}%
{$-1$}%
{depth until breadth-fisrt search is applied}%
{}

\printoption{nodeselection/estimate/stdpriority}%
{$-536870912\leq\textrm{integer}\leq536870911$}%
{$200000$}%
{priority of node selection rule $<$estimate$>$ in standard mode}%
{}

\printoption{nodeselection/hybridestim/bestnodefreq}%
{$0\leq\textrm{integer}$}%
{$1000$}%
{frequency at which the best node instead of the hybrid best estimate / best bound is selected (0: never)}%
{}

\printoption{nodeselection/hybridestim/stdpriority}%
{$-536870912\leq\textrm{integer}\leq536870911$}%
{$50000$}%
{priority of node selection rule $<$hybridestim$>$ in standard mode}%
{}

\printoption{nodeselection/restartdfs/countonlyleaves}%
{boolean}%
{TRUE}%
{count only leaf nodes (otherwise all nodes)?}%
{}

\printoption{nodeselection/restartdfs/selectbestfreq}%
{$0\leq\textrm{integer}$}%
{$100$}%
{frequency for selecting the best node instead of the deepest one}%
{}

\printoption{nodeselection/restartdfs/stdpriority}%
{$-536870912\leq\textrm{integer}\leq536870911$}%
{$10000$}%
{priority of node selection rule $<$restartdfs$>$ in standard mode}%
{}

\printoptioncategory{Node Selection (advanced options)}
\printoption{nodeselection/bfs/maxplungedepth}%
{$-1\leq\textrm{integer}$}%
{$-1$}%
{maximal plunging depth, before new best node is forced to be selected (-1 for dynamic setting)}%
{}

\printoption{nodeselection/bfs/maxplungequot}%
{$0\leq\textrm{real}$}%
{$0.25$}%
{maximal quotient (curlowerbound - lowerbound)/(cutoffbound - lowerbound) where plunging is performed}%
{}

\printoption{nodeselection/bfs/memsavepriority}%
{$-536870912\leq\textrm{integer}\leq536870911$}%
{$0$}%
{priority of node selection rule $<$bfs$>$ in memory saving mode}%
{}

\printoption{nodeselection/bfs/minplungedepth}%
{$-1\leq\textrm{integer}$}%
{$-1$}%
{minimal plunging depth, before new best node may be selected (-1 for dynamic setting)}%
{}

\printoption{nodeselection/dfs/memsavepriority}%
{$-536870912\leq\textrm{integer}\leq536870911$}%
{$100000$}%
{priority of node selection rule $<$dfs$>$ in memory saving mode}%
{}

\printoption{nodeselection/estimate/maxplungedepth}%
{$-1\leq\textrm{integer}$}%
{$-1$}%
{maximal plunging depth, before new best node is forced to be selected (-1 for dynamic setting)}%
{}

\printoption{nodeselection/estimate/maxplungequot}%
{$0\leq\textrm{real}$}%
{$0.25$}%
{maximal quotient (estimate - lowerbound)/(cutoffbound - lowerbound) where plunging is performed}%
{}

\printoption{nodeselection/estimate/memsavepriority}%
{$-536870912\leq\textrm{integer}\leq536870911$}%
{$100$}%
{priority of node selection rule $<$estimate$>$ in memory saving mode}%
{}

\printoption{nodeselection/estimate/minplungedepth}%
{$-1\leq\textrm{integer}$}%
{$-1$}%
{minimal plunging depth, before new best node may be selected (-1 for dynamic setting)}%
{}

\printoption{nodeselection/hybridestim/estimweight}%
{$0\leq\textrm{real}\leq1$}%
{$0.1$}%
{weight of estimate value in node selection score (0: pure best bound search, 1: pure best estimate search)}%
{}

\printoption{nodeselection/hybridestim/maxplungedepth}%
{$-1\leq\textrm{integer}$}%
{$-1$}%
{maximal plunging depth, before new best node is forced to be selected (-1 for dynamic setting)}%
{}

\printoption{nodeselection/hybridestim/maxplungequot}%
{$0\leq\textrm{real}$}%
{$0.25$}%
{maximal quotient (estimate - lowerbound)/(cutoffbound - lowerbound) where plunging is performed}%
{}

\printoption{nodeselection/hybridestim/memsavepriority}%
{$-536870912\leq\textrm{integer}\leq536870911$}%
{$50$}%
{priority of node selection rule $<$hybridestim$>$ in memory saving mode}%
{}

\printoption{nodeselection/hybridestim/minplungedepth}%
{$-1\leq\textrm{integer}$}%
{$-1$}%
{minimal plunging depth, before new best node may be selected (-1 for dynamic setting)}%
{}

\printoption{nodeselection/restartdfs/memsavepriority}%
{$-536870912\leq\textrm{integer}\leq536870911$}%
{$50000$}%
{priority of node selection rule $<$restartdfs$>$ in memory saving mode}%
{}

\printoptioncategory{Tolerances}
\printoption{numerics/dualfeastol}%
{$10^{-17}\leq\textrm{real}\leq0.001$}%
{$10^{- 6}$}%
{feasibility tolerance for reduced costs in LP solution}%
{}

\printoption{numerics/epsilon}%
{$10^{-20}\leq\textrm{real}\leq0.001$}%
{$10^{- 9}$}%
{absolute values smaller than this are considered zero}%
{}

\printoption{numerics/feastol}%
{$10^{-17}\leq\textrm{real}\leq0.001$}%
{$10^{- 6}$}%
{feasibility tolerance for constraints}%
{}

\printoption{numerics/lpfeastol}%
{$10^{-17}\leq\textrm{real}\leq0.001$}%
{$10^{- 6}$}%
{primal feasibility tolerance of LP solver}%
{}

\printoption{numerics/sumepsilon}%
{$10^{-17}\leq\textrm{real}\leq0.001$}%
{$10^{- 6}$}%
{absolute values of sums smaller than this are considered zero}%
{}

\printoptioncategory{Tolerances (advanced options)}
\printoption{numerics/barrierconvtol}%
{$10^{-17}\leq\textrm{real}\leq0.001$}%
{$10^{-10}$}%
{LP convergence tolerance used in barrier algorithm}%
{}

\printoption{numerics/boundstreps}%
{$10^{-17}\leq\textrm{real}$}%
{$0.05$}%
{minimal relative improve for strengthening bounds}%
{}

\printoption{numerics/hugeval}%
{$0\leq\textrm{real}$}%
{$10^{ 15}$}%
{values larger than this are considered huge and should be handled separately (e.g., in activity computation)}%
{}

\printoption{numerics/pseudocostdelta}%
{$0\leq\textrm{real}$}%
{$0.0001$}%
{minimal objective distance value to use for branching pseudo cost updates}%
{}

\printoption{numerics/pseudocosteps}%
{$10^{-17}\leq\textrm{real}\leq1$}%
{$0.1$}%
{minimal variable distance value to use for branching pseudo cost updates}%
{}

\printoption{numerics/recomputefac}%
{$0\leq\textrm{real}$}%
{$10^{  7}$}%
{minimal decrease factor that causes the recomputation of a value (e.g., pseudo objective) instead of an update}%
{}

\printoptioncategory{Presolving}
\printoption{presolving/boundshift/maxrounds}%
{$-1\leq\textrm{integer}$}%
{$0$}%
{maximal number of presolving rounds the presolver participates in (-1: no limit)}%
{}

\printoption{presolving/components/intfactor}%
{$0\leq\textrm{real}$}%
{$1$}%
{the weight of an integer variable compared to binary variables}%
{}

\printoption{presolving/components/maxintvars}%
{$-1\leq\textrm{integer}$}%
{$500$}%
{maximum number of integer (or binary) variables to solve a subproblem directly (-1: unlimited)}%
{}

\printoption{presolving/components/maxrounds}%
{$-1\leq\textrm{integer}$}%
{$-1$}%
{maximal number of presolving rounds the presolver participates in (-1: no limit)}%
{}

\printoption{presolving/components/nodelimit}%
{$-1\leq\textrm{integer}$}%
{$10000$}%
{maximum number of nodes to be solved in subproblems}%
{}

\printoption{presolving/components/reldecrease}%
{$0\leq\textrm{real}\leq1$}%
{$0.2$}%
{percentage by which the number of variables has to be decreased after the last component solving to allow running again (1.0: do not run again)}%
{}

\printoption{presolving/components/writeproblems}%
{boolean}%
{FALSE}%
{should the single components be written as an .lp-file?}%
{}

\printoption{presolving/convertinttobin/maxrounds}%
{$-1\leq\textrm{integer}$}%
{$0$}%
{maximal number of presolving rounds the presolver participates in (-1: no limit)}%
{}

\printoption{presolving/domcol/maxrounds}%
{$-1\leq\textrm{integer}$}%
{$-1$}%
{maximal number of presolving rounds the presolver participates in (-1: no limit)}%
{}

\printoption{presolving/dualfix/maxrounds}%
{$-1\leq\textrm{integer}$}%
{$-1$}%
{maximal number of presolving rounds the presolver participates in (-1: no limit)}%
{}

\printoption{presolving/gateextraction/maxrounds}%
{$-1\leq\textrm{integer}$}%
{$-1$}%
{maximal number of presolving rounds the presolver participates in (-1: no limit)}%
{}

\printoption{presolving/implics/maxrounds}%
{$-1\leq\textrm{integer}$}%
{$-1$}%
{maximal number of presolving rounds the presolver participates in (-1: no limit)}%
{}

\printoption{presolving/inttobinary/maxrounds}%
{$-1\leq\textrm{integer}$}%
{$-1$}%
{maximal number of presolving rounds the presolver participates in (-1: no limit)}%
{}

\printoption{presolving/maxrestarts}%
{$-1\leq\textrm{integer}$}%
{$-1$}%
{maximal number of restarts (-1: unlimited)}%
{}

\printoption{presolving/maxrounds}%
{$-1\leq\textrm{integer}$}%
{$-1$}%
{maximal number of presolving rounds (-1: unlimited, 0: off)}%
{}

\printoption{presolving/trivial/maxrounds}%
{$-1\leq\textrm{integer}$}%
{$-1$}%
{maximal number of presolving rounds the presolver participates in (-1: no limit)}%
{}

\printoptioncategory{Presolving (advanced options)}
\printoption{presolving/abortfac}%
{$0\leq\textrm{real}\leq1$}%
{$0.0001$}%
{abort presolve, if at most this fraction of the problem was changed in last presolve round}%
{}

\printoption{presolving/boundshift/delay}%
{boolean}%
{FALSE}%
{should presolver be delayed, if other presolvers found reductions?}%
{}

\printoption{presolving/boundshift/flipping}%
{boolean}%
{TRUE}%
{is flipping allowed (multiplying with -1)?}%
{}

\printoption{presolving/boundshift/integer}%
{boolean}%
{TRUE}%
{shift only integer ranges?}%
{}

\printoption{presolving/boundshift/maxshift}%
{$0\leq\textrm{integer}$}%
{$\infty$}%
{absolute value of maximum shift}%
{}

\printoption{presolving/boundshift/priority}%
{$-536870912\leq\textrm{integer}\leq536870911$}%
{$7900000$}%
{priority of presolver $<$boundshift$>$}%
{}

\printoption{presolving/components/delay}%
{boolean}%
{TRUE}%
{should presolver be delayed, if other presolvers found reductions?}%
{}

\printoption{presolving/components/priority}%
{$-536870912\leq\textrm{integer}\leq536870911$}%
{$-9200000$}%
{priority of presolver $<$components$>$}%
{}

\printoption{presolving/convertinttobin/delay}%
{boolean}%
{FALSE}%
{should presolver be delayed, if other presolvers found reductions?}%
{}

\printoption{presolving/convertinttobin/maxdomainsize}%
{$0\leq\textrm{integer}$}%
{$\infty$}%
{absolute value of maximum domain size for converting an integer variable to binaries variables}%
{}

\printoption{presolving/convertinttobin/onlypoweroftwo}%
{boolean}%
{FALSE}%
{should only integer variables with a domain size of 2\^p - 1 be converted(, there we don't need an knapsack-constraint for restricting the sum of the binaries)}%
{}

\printoption{presolving/convertinttobin/priority}%
{$-536870912\leq\textrm{integer}\leq536870911$}%
{$6000000$}%
{priority of presolver $<$convertinttobin$>$}%
{}

\printoption{presolving/convertinttobin/samelocksinbothdirections}%
{boolean}%
{FALSE}%
{should only integer variables with uplocks equals downlocks be converted}%
{}

\printoption{presolving/domcol/delay}%
{boolean}%
{TRUE}%
{should presolver be delayed, if other presolvers found reductions?}%
{}

\printoption{presolving/domcol/priority}%
{$-536870912\leq\textrm{integer}\leq536870911$}%
{$20000000$}%
{priority of presolver $<$domcol$>$}%
{}

\printoption{presolving/donotaggr}%
{boolean}%
{FALSE}%
{should aggregation of variables be forbidden?}%
{}

\printoption{presolving/donotmultaggr}%
{boolean}%
{FALSE}%
{should multi-aggregation of variables be forbidden?}%
{}

\printoption{presolving/dualfix/delay}%
{boolean}%
{FALSE}%
{should presolver be delayed, if other presolvers found reductions?}%
{}

\printoption{presolving/dualfix/priority}%
{$-536870912\leq\textrm{integer}\leq536870911$}%
{$8000000$}%
{priority of presolver $<$dualfix$>$}%
{}

\printoption{presolving/gateextraction/delay}%
{boolean}%
{TRUE}%
{should presolver be delayed, if other presolvers found reductions?}%
{}

\printoption{presolving/gateextraction/onlysetpart}%
{boolean}%
{FALSE}%
{should we only try to extract set-partitioning constraints and no and-constraints}%
{}

\printoption{presolving/gateextraction/priority}%
{$-536870912\leq\textrm{integer}\leq536870911$}%
{$1000000$}%
{priority of presolver $<$gateextraction$>$}%
{}

\printoption{presolving/gateextraction/searchequations}%
{boolean}%
{TRUE}%
{should we try to extract set-partitioning constraint out of one logicor and one corresponding set-packing constraint}%
{}

\printoption{presolving/gateextraction/sorting}%
{$-1\leq\textrm{integer}\leq1$}%
{$1$}%
{order logicor contraints to extract big-gates before smaller ones (-1), do not order them (0) or order them to extract smaller gates at first (1)}%
{}

\printoption{presolving/immrestartfac}%
{$0\leq\textrm{real}\leq1$}%
{$0.2$}%
{fraction of integer variables that were fixed in the root node triggering an immediate restart with preprocessing}%
{}

\printoption{presolving/implics/delay}%
{boolean}%
{FALSE}%
{should presolver be delayed, if other presolvers found reductions?}%
{}

\printoption{presolving/implics/priority}%
{$-536870912\leq\textrm{integer}\leq536870911$}%
{$-10000$}%
{priority of presolver $<$implics$>$}%
{}

\printoption{presolving/inttobinary/delay}%
{boolean}%
{FALSE}%
{should presolver be delayed, if other presolvers found reductions?}%
{}

\printoption{presolving/inttobinary/priority}%
{$-536870912\leq\textrm{integer}\leq536870911$}%
{$7000000$}%
{priority of presolver $<$inttobinary$>$}%
{}

\printoption{presolving/restartfac}%
{$0\leq\textrm{real}\leq1$}%
{$0.05$}%
{fraction of integer variables that were fixed in the root node triggering a restart with preprocessing after root node evaluation}%
{}

\printoption{presolving/restartminred}%
{$0\leq\textrm{real}\leq1$}%
{$0.1$}%
{minimal fraction of integer variables removed after restart to allow for an additional restart}%
{}

\printoption{presolving/subrestartfac}%
{$0\leq\textrm{real}\leq1$}%
{$1$}%
{fraction of integer variables that were globally fixed during the solving process triggering a restart with preprocessing}%
{}

\printoption{presolving/trivial/delay}%
{boolean}%
{FALSE}%
{should presolver be delayed, if other presolvers found reductions?}%
{}

\printoption{presolving/trivial/priority}%
{$-536870912\leq\textrm{integer}\leq536870911$}%
{$9000000$}%
{priority of presolver $<$trivial$>$}%
{}

\printoptioncategory{Domain Propagation}
\printoption{propagating/abortoncutoff}%
{boolean}%
{TRUE}%
{should propagation be aborted immediately? setting this to FALSE could help conflict analysis to produce more conflict constraints}%
{}

\printoption{propagating/genvbounds/freq}%
{$-1\leq\textrm{integer}$}%
{$1$}%
{frequency for calling propagator $<$genvbounds$>$ (-1: never, 0: only in root node)}%
{}

\printoption{propagating/genvbounds/maxprerounds}%
{$-1\leq\textrm{integer}$}%
{$-1$}%
{maximal number of presolving rounds the propagator participates in (-1: no limit)}%
{}

\printoption{propagating/maxrounds}%
{$-1\leq\textrm{integer}$}%
{$100$}%
{maximal number of propagation rounds per node (-1: unlimited)}%
{}

\printoption{propagating/maxroundsroot}%
{$-1\leq\textrm{integer}$}%
{$1000$}%
{maximal number of propagation rounds in the root node (-1: unlimited)}%
{}

\printoption{propagating/obbt/freq}%
{$-1\leq\textrm{integer}$}%
{$0$}%
{frequency for calling propagator $<$obbt$>$ (-1: never, 0: only in root node)}%
{}

\printoption{propagating/obbt/itlimitfactor}%
{$\textrm{real}$}%
{$5$}%
{multiple of root node LP iterations used as total LP iteration limit for obbt ($<$= 0: no limit )}%
{}

\printoption{propagating/obbt/maxlookahead}%
{$-1\leq\textrm{integer}$}%
{$3$}%
{maximal number of bounds evaluated without success per group (-1: no limit)}%
{}

\printoption{propagating/obbt/maxprerounds}%
{$-1\leq\textrm{integer}$}%
{$-1$}%
{maximal number of presolving rounds the propagator participates in (-1: no limit)}%
{}

\printoption{propagating/probing/freq}%
{$-1\leq\textrm{integer}$}%
{$-1$}%
{frequency for calling propagator $<$probing$>$ (-1: never, 0: only in root node)}%
{}

\printoption{propagating/probing/maxprerounds}%
{$-1\leq\textrm{integer}$}%
{$-1$}%
{maximal number of presolving rounds the propagator participates in (-1: no limit)}%
{}

\printoption{propagating/probing/maxruns}%
{$-1\leq\textrm{integer}$}%
{$1$}%
{maximal number of runs, probing participates in (-1: no limit)}%
{}

\printoption{propagating/pseudoobj/freq}%
{$-1\leq\textrm{integer}$}%
{$1$}%
{frequency for calling propagator $<$pseudoobj$>$ (-1: never, 0: only in root node)}%
{}

\printoption{propagating/pseudoobj/maxprerounds}%
{$-1\leq\textrm{integer}$}%
{$-1$}%
{maximal number of presolving rounds the propagator participates in (-1: no limit)}%
{}

\printoption{propagating/redcost/continuous}%
{boolean}%
{FALSE}%
{should reduced cost fixing be also applied to continuous variables?}%
{}

\printoption{propagating/redcost/freq}%
{$-1\leq\textrm{integer}$}%
{$1$}%
{frequency for calling propagator $<$redcost$>$ (-1: never, 0: only in root node)}%
{}

\printoption{propagating/redcost/maxprerounds}%
{$-1\leq\textrm{integer}$}%
{$-1$}%
{maximal number of presolving rounds the propagator participates in (-1: no limit)}%
{}

\printoption{propagating/redcost/useimplics}%
{boolean}%
{TRUE}%
{should implications be used to strength the reduced cost for binary variables?}%
{}

\printoption{propagating/rootredcost/freq}%
{$-1\leq\textrm{integer}$}%
{$1$}%
{frequency for calling propagator $<$rootredcost$>$ (-1: never, 0: only in root node)}%
{}

\printoption{propagating/rootredcost/maxprerounds}%
{$-1\leq\textrm{integer}$}%
{$-1$}%
{maximal number of presolving rounds the propagator participates in (-1: no limit)}%
{}

\printoption{propagating/vbounds/dotoposort}%
{boolean}%
{TRUE}%
{should the bounds be topologically sorted in advance?}%
{}

\printoption{propagating/vbounds/freq}%
{$-1\leq\textrm{integer}$}%
{$1$}%
{frequency for calling propagator $<$vbounds$>$ (-1: never, 0: only in root node)}%
{}

\printoption{propagating/vbounds/maxprerounds}%
{$-1\leq\textrm{integer}$}%
{$-1$}%
{maximal number of presolving rounds the propagator participates in (-1: no limit)}%
{}

\printoption{propagating/vbounds/sortcliques}%
{boolean}%
{FALSE}%
{should cliques be regarded for the topological sort?}%
{}

\printoption{propagating/vbounds/usebdwidening}%
{boolean}%
{TRUE}%
{should bound widening be used to initialize conflict analysis?}%
{}

\printoption{propagating/vbounds/usecliques}%
{boolean}%
{FALSE}%
{should cliques be propagated?}%
{}

\printoption{propagating/vbounds/useimplics}%
{boolean}%
{FALSE}%
{should implications be propagated?}%
{}

\printoption{propagating/vbounds/usevbounds}%
{boolean}%
{TRUE}%
{should vbounds be propagated?}%
{}

\printoptioncategory{Domain Propagation (advanced options)}
\printoption{propagating/genvbounds/delay}%
{boolean}%
{FALSE}%
{should propagator be delayed, if other propagators found reductions?}%
{}

\printoption{propagating/genvbounds/presoldelay}%
{boolean}%
{FALSE}%
{should presolving be delayed, if other presolvers found reductions?}%
{}

\printoption{propagating/genvbounds/presolpriority}%
{$-536870912\leq\textrm{integer}\leq536870911$}%
{$-2000000$}%
{presolving priority of propagator $<$genvbounds$>$}%
{}

\printoption{propagating/genvbounds/priority}%
{$-536870912\leq\textrm{integer}\leq536870911$}%
{$-10$}%
{priority of propagator $<$genvbounds$>$}%
{}

\printoption{propagating/genvbounds/timingmask}%
{$1\leq\textrm{integer}\leq15$}%
{$15$}%
{timing when propagator should be called (1:BEFORELP, 2:DURINGLPLOOP, 4:AFTERLPLOOP, 15:ALWAYS))}%
{}

\printoption{propagating/obbt/creategenvbounds}%
{boolean}%
{TRUE}%
{should obbt try to provide genvbounds if possible?}%
{}

\printoption{propagating/obbt/delay}%
{boolean}%
{TRUE}%
{should propagator be delayed, if other propagators found reductions?}%
{}

\printoption{propagating/obbt/minfilter}%
{$1\leq\textrm{integer}$}%
{$2$}%
{minimal number of filtered bounds to apply another filter round}%
{}

\printoption{propagating/obbt/normalize}%
{boolean}%
{TRUE}%
{should coefficients in filtering be normalized w.r.t. the domains sizes?}%
{}

\printoption{propagating/obbt/presoldelay}%
{boolean}%
{FALSE}%
{should presolving be delayed, if other presolvers found reductions?}%
{}

\printoption{propagating/obbt/presolpriority}%
{$-536870912\leq\textrm{integer}\leq536870911$}%
{$0$}%
{presolving priority of propagator $<$obbt$>$}%
{}

\printoption{propagating/obbt/priority}%
{$-536870912\leq\textrm{integer}\leq536870911$}%
{$-1000000$}%
{priority of propagator $<$obbt$>$}%
{}

\printoption{propagating/obbt/timingmask}%
{$1\leq\textrm{integer}\leq15$}%
{$4$}%
{timing when propagator should be called (1:BEFORELP, 2:DURINGLPLOOP, 4:AFTERLPLOOP, 15:ALWAYS))}%
{}

\printoption{propagating/probing/delay}%
{boolean}%
{TRUE}%
{should propagator be delayed, if other propagators found reductions?}%
{}

\printoption{propagating/probing/maxdepth}%
{$-1\leq\textrm{integer}$}%
{$-1$}%
{maximal depth until propagation is executed(-1: no limit)}%
{}

\printoption{propagating/probing/maxfixings}%
{$0\leq\textrm{integer}$}%
{$25$}%
{maximal number of fixings found, until probing is interrupted (0: don't iterrupt)}%
{}

\printoption{propagating/probing/maxsumuseless}%
{$0\leq\textrm{integer}$}%
{$0$}%
{maximal number of probings without fixings, until probing is aborted (0: don't abort)}%
{}

\printoption{propagating/probing/maxtotaluseless}%
{$0\leq\textrm{integer}$}%
{$50$}%
{maximal number of successive probings without fixings, bound changes, and implications, until probing is aborted (0: don't abort)}%
{}

\printoption{propagating/probing/maxuseless}%
{$0\leq\textrm{integer}$}%
{$1000$}%
{maximal number of successive probings without fixings, until probing is aborted (0: don't abort)}%
{}

\printoption{propagating/probing/presoldelay}%
{boolean}%
{TRUE}%
{should presolving be delayed, if other presolvers found reductions?}%
{}

\printoption{propagating/probing/presolpriority}%
{$-536870912\leq\textrm{integer}\leq536870911$}%
{$-100000$}%
{presolving priority of propagator $<$probing$>$}%
{}

\printoption{propagating/probing/priority}%
{$-536870912\leq\textrm{integer}\leq536870911$}%
{$-100000$}%
{priority of propagator $<$probing$>$}%
{}

\printoption{propagating/probing/proprounds}%
{$-1\leq\textrm{integer}$}%
{$-1$}%
{maximal number of propagation rounds in probing subproblems (-1: no limit, 0: auto)}%
{}

\printoption{propagating/probing/timingmask}%
{$1\leq\textrm{integer}\leq15$}%
{$4$}%
{timing when propagator should be called (1:BEFORELP, 2:DURINGLPLOOP, 4:AFTERLPLOOP, 15:ALWAYS))}%
{}

\printoption{propagating/pseudoobj/delay}%
{boolean}%
{FALSE}%
{should propagator be delayed, if other propagators found reductions?}%
{}

\printoption{propagating/pseudoobj/force}%
{boolean}%
{FALSE}%
{should the propagator be forced even active pricer are present?}%
{}

\printoption{propagating/pseudoobj/maximplvars}%
{$-1\leq\textrm{integer}$}%
{$50000$}%
{maximum number of binary variables the implications are used if turned on (-1: unlimited)?}%
{}

\printoption{propagating/pseudoobj/maxnewvars}%
{$0\leq\textrm{integer}$}%
{$1000$}%
{number of variable added after the propgatore is reinitialized?}%
{}

\printoption{propagating/pseudoobj/maxvarsfrac}%
{$0\leq\textrm{real}\leq1$}%
{$0.1$}%
{maximal fraction of none binary variables with non-zero objective without a bound reduction before aborted}%
{}

\printoption{propagating/pseudoobj/minuseless}%
{$0\leq\textrm{integer}$}%
{$100$}%
{minimal number of successive none binary variable propagator whithout a bound reduction before aborted}%
{}

\printoption{propagating/pseudoobj/presoldelay}%
{boolean}%
{TRUE}%
{should presolving be delayed, if other presolvers found reductions?}%
{}

\printoption{propagating/pseudoobj/presolpriority}%
{$-536870912\leq\textrm{integer}\leq536870911$}%
{$6000000$}%
{presolving priority of propagator $<$pseudoobj$>$}%
{}

\printoption{propagating/pseudoobj/priority}%
{$-536870912\leq\textrm{integer}\leq536870911$}%
{$3000000$}%
{priority of propagator $<$pseudoobj$>$}%
{}

\printoption{propagating/pseudoobj/propcutoffbound}%
{boolean}%
{TRUE}%
{propagate new cutoff bound directly globally}%
{}

\printoption{propagating/pseudoobj/propfullinroot}%
{boolean}%
{TRUE}%
{do we want to propagate all none binary variables if we are propagating the root node}%
{}

\printoption{propagating/pseudoobj/propuseimplics}%
{boolean}%
{TRUE}%
{use implications to strengthen the propagation of binary variable (increasing the objective change)?}%
{}

\printoption{propagating/pseudoobj/respropuseimplics}%
{boolean}%
{TRUE}%
{use implications to strengthen the resolve propagation of binary variable (increasing the objective change)?}%
{}

\printoption{propagating/pseudoobj/timingmask}%
{$1\leq\textrm{integer}\leq15$}%
{$5$}%
{timing when propagator should be called (1:BEFORELP, 2:DURINGLPLOOP, 4:AFTERLPLOOP, 15:ALWAYS))}%
{}

\printoption{propagating/redcost/delay}%
{boolean}%
{FALSE}%
{should propagator be delayed, if other propagators found reductions?}%
{}

\printoption{propagating/redcost/presoldelay}%
{boolean}%
{FALSE}%
{should presolving be delayed, if other presolvers found reductions?}%
{}

\printoption{propagating/redcost/presolpriority}%
{$-536870912\leq\textrm{integer}\leq536870911$}%
{$0$}%
{presolving priority of propagator $<$redcost$>$}%
{}

\printoption{propagating/redcost/priority}%
{$-536870912\leq\textrm{integer}\leq536870911$}%
{$1000000$}%
{priority of propagator $<$redcost$>$}%
{}

\printoption{propagating/redcost/timingmask}%
{$1\leq\textrm{integer}\leq15$}%
{$2$}%
{timing when propagator should be called (1:BEFORELP, 2:DURINGLPLOOP, 4:AFTERLPLOOP, 15:ALWAYS))}%
{}

\printoption{propagating/rootredcost/delay}%
{boolean}%
{FALSE}%
{should propagator be delayed, if other propagators found reductions?}%
{}

\printoption{propagating/rootredcost/presoldelay}%
{boolean}%
{FALSE}%
{should presolving be delayed, if other presolvers found reductions?}%
{}

\printoption{propagating/rootredcost/presolpriority}%
{$-536870912\leq\textrm{integer}\leq536870911$}%
{$0$}%
{presolving priority of propagator $<$rootredcost$>$}%
{}

\printoption{propagating/rootredcost/priority}%
{$-536870912\leq\textrm{integer}\leq536870911$}%
{$10000000$}%
{priority of propagator $<$rootredcost$>$}%
{}

\printoption{propagating/rootredcost/timingmask}%
{$1\leq\textrm{integer}\leq15$}%
{$5$}%
{timing when propagator should be called (1:BEFORELP, 2:DURINGLPLOOP, 4:AFTERLPLOOP, 15:ALWAYS))}%
{}

\printoption{propagating/vbounds/delay}%
{boolean}%
{FALSE}%
{should propagator be delayed, if other propagators found reductions?}%
{}

\printoption{propagating/vbounds/presoldelay}%
{boolean}%
{FALSE}%
{should presolving be delayed, if other presolvers found reductions?}%
{}

\printoption{propagating/vbounds/presolpriority}%
{$-536870912\leq\textrm{integer}\leq536870911$}%
{$0$}%
{presolving priority of propagator $<$vbounds$>$}%
{}

\printoption{propagating/vbounds/priority}%
{$-536870912\leq\textrm{integer}\leq536870911$}%
{$3000000$}%
{priority of propagator $<$vbounds$>$}%
{}

\printoption{propagating/vbounds/timingmask}%
{$1\leq\textrm{integer}\leq15$}%
{$5$}%
{timing when propagator should be called (1:BEFORELP, 2:DURINGLPLOOP, 4:AFTERLPLOOP, 15:ALWAYS))}%
{}

\printoptioncategory{Separation}
\printoption{separating/cgmip/addviolationcons}%
{boolean}%
{FALSE}%
{add constraint to subscip that only allows violated cuts (otherwise add obj. limit)?}%
{}

\printoption{separating/cgmip/addviolconshdlr}%
{boolean}%
{FALSE}%
{add constraint handler to filter out violated cuts?}%
{}

\printoption{separating/cgmip/allowlocal}%
{boolean}%
{FALSE}%
{Allow to generate local cuts?}%
{}

\printoption{separating/cgmip/cmirownbounds}%
{boolean}%
{FALSE}%
{tell CMIR-generator which bounds to used in rounding?}%
{}

\printoption{separating/cgmip/conshdlrusenorm}%
{boolean}%
{TRUE}%
{should the violation constraint handler use the norm of a cut to check for feasibility?}%
{}

\printoption{separating/cgmip/contconvert}%
{boolean}%
{FALSE}%
{Convert some integral variables to be continuous to reduce the size of the sub-MIP?}%
{}

\printoption{separating/cgmip/contconvfrac}%
{$0\leq\textrm{real}\leq1$}%
{$0.1$}%
{fraction of integral variables converted to be continuous (if contconvert)}%
{}

\printoption{separating/cgmip/contconvmin}%
{$-1\leq\textrm{integer}$}%
{$100$}%
{minimum number of integral variables before some are converted to be continuous}%
{}

\printoption{separating/cgmip/decisiontree}%
{boolean}%
{FALSE}%
{Use decision tree to turn separation on/off?}%
{}

\printoption{separating/cgmip/dynamiccuts}%
{boolean}%
{TRUE}%
{should generated cuts be removed from the LP if they are no longer tight?}%
{}

\printoption{separating/cgmip/earlyterm}%
{boolean}%
{TRUE}%
{terminate separation if a violated (but possibly sub-optimal) cut has been found?}%
{}

\printoption{separating/cgmip/freq}%
{$-1\leq\textrm{integer}$}%
{$-1$}%
{frequency for calling separator $<$cgmip$>$ (-1: never, 0: only in root node)}%
{}

\printoption{separating/cgmip/intconvert}%
{boolean}%
{FALSE}%
{Convert some integral variables attaining fractional values to have integral value?}%
{}

\printoption{separating/cgmip/intconvfrac}%
{$0\leq\textrm{real}\leq1$}%
{$0.1$}%
{fraction of frac. integral variables converted to have integral value (if intconvert)}%
{}

\printoption{separating/cgmip/intconvmin}%
{$-1\leq\textrm{integer}$}%
{$100$}%
{minimum number of integral variables before some are converted to have integral value}%
{}

\printoption{separating/cgmip/maxdepth}%
{$-1\leq\textrm{integer}$}%
{$-1$}%
{maximal depth at which the separator is applied (-1: unlimited)}%
{}

\printoption{separating/cgmip/maxnodelimit}%
{$-1\leq\textrm{integer}$}%
{$5000$}%
{maximum number of nodes considered for sub-MIP (-1: unlimited)}%
{}

\printoption{separating/cgmip/maxrounds}%
{$-1\leq\textrm{integer}$}%
{$5$}%
{maximal number of cgmip separation rounds per node (-1: unlimited)}%
{}

\printoption{separating/cgmip/maxroundsroot}%
{$-1\leq\textrm{integer}$}%
{$50$}%
{maximal number of cgmip separation rounds in the root node (-1: unlimited)}%
{}

\printoption{separating/cgmip/maxrowage}%
{$-1\leq\textrm{integer}$}%
{$-1$}%
{maximal age of rows to consider if onlyactiverows is false}%
{}

\printoption{separating/cgmip/minnodelimit}%
{$-1\leq\textrm{integer}$}%
{$500$}%
{minimum number of nodes considered for sub-MIP (-1: unlimited)}%
{}

\printoption{separating/cgmip/objlone}%
{boolean}%
{FALSE}%
{Should the objective of the sub-MIP minimize the l1-norm of the multipliers?}%
{}

\printoption{separating/cgmip/objweighsize}%
{boolean}%
{FALSE}%
{Weigh each row by its size?}%
{}

\printoption{separating/cgmip/onlyactiverows}%
{boolean}%
{FALSE}%
{Use only active rows to generate cuts?}%
{}

\printoption{separating/cgmip/onlyintvars}%
{boolean}%
{FALSE}%
{Generate cuts for problems with only integer variables?}%
{}

\printoption{separating/cgmip/onlyrankone}%
{boolean}%
{FALSE}%
{Separate only rank 1 inequalities?}%
{}

\printoption{separating/cgmip/primalseparation}%
{boolean}%
{TRUE}%
{only separate cuts that are tight for the best feasible solution?}%
{}

\printoption{separating/cgmip/skipmultbounds}%
{boolean}%
{TRUE}%
{Skip the upper bounds on the multipliers in the sub-MIP?}%
{}

\printoption{separating/cgmip/usecmir}%
{boolean}%
{TRUE}%
{use CMIR-generator (otherwise add cut directly)?}%
{}

\printoption{separating/cgmip/usecutpool}%
{boolean}%
{TRUE}%
{use cutpool to store CG-cuts even if the are not efficient?}%
{}

\printoption{separating/cgmip/usestrongcg}%
{boolean}%
{FALSE}%
{use strong CG-function to strengthen cut?}%
{}

\printoption{separating/clique/freq}%
{$-1\leq\textrm{integer}$}%
{$0$}%
{frequency for calling separator $<$clique$>$ (-1: never, 0: only in root node)}%
{}

\printoption{separating/clique/maxsepacuts}%
{$-1\leq\textrm{integer}$}%
{$10$}%
{maximal number of clique cuts separated per separation round (-1: no limit)}%
{}

\printoption{separating/closecuts/freq}%
{$-1\leq\textrm{integer}$}%
{$-1$}%
{frequency for calling separator $<$closecuts$>$ (-1: never, 0: only in root node)}%
{}

\printoption{separating/cmir/dynamiccuts}%
{boolean}%
{TRUE}%
{should generated cuts be removed from the LP if they are no longer tight?}%
{}

\printoption{separating/cmir/freq}%
{$-1\leq\textrm{integer}$}%
{$0$}%
{frequency for calling separator $<$cmir$>$ (-1: never, 0: only in root node)}%
{}

\printoption{separating/cmir/maxrounds}%
{$-1\leq\textrm{integer}$}%
{$3$}%
{maximal number of cmir separation rounds per node (-1: unlimited)}%
{}

\printoption{separating/cmir/maxroundsroot}%
{$-1\leq\textrm{integer}$}%
{$10$}%
{maximal number of cmir separation rounds in the root node (-1: unlimited)}%
{}

\printoption{separating/cmir/maxsepacuts}%
{$0\leq\textrm{integer}$}%
{$100$}%
{maximal number of cmir cuts separated per separation round}%
{}

\printoption{separating/cmir/maxsepacutsroot}%
{$0\leq\textrm{integer}$}%
{$500$}%
{maximal number of cmir cuts separated per separation round in the root node}%
{}

\printoption{separating/flowcover/dynamiccuts}%
{boolean}%
{TRUE}%
{should generated cuts be removed from the LP if they are no longer tight?}%
{}

\printoption{separating/flowcover/freq}%
{$-1\leq\textrm{integer}$}%
{$0$}%
{frequency for calling separator $<$flowcover$>$ (-1: never, 0: only in root node)}%
{}

\printoption{separating/flowcover/maxrounds}%
{$-1\leq\textrm{integer}$}%
{$5$}%
{maximal number of separation rounds per node (-1: unlimited)}%
{}

\printoption{separating/flowcover/maxroundsroot}%
{$-1\leq\textrm{integer}$}%
{$15$}%
{maximal number of separation rounds in the root node (-1: unlimited)}%
{}

\printoption{separating/flowcover/maxsepacuts}%
{$0\leq\textrm{integer}$}%
{$100$}%
{maximal number of flow cover cuts separated per separation round}%
{}

\printoption{separating/flowcover/maxsepacutsroot}%
{$0\leq\textrm{integer}$}%
{$200$}%
{maximal number of flow cover cuts separated per separation round in the root}%
{}

\printoption{separating/gomory/away}%
{$0\leq\textrm{real}\leq0.5$}%
{$0.01$}%
{minimal integrality violation of a basis variable in order to try Gomory cut}%
{}

\printoption{separating/gomory/dynamiccuts}%
{boolean}%
{TRUE}%
{should generated cuts be removed from the LP if they are no longer tight?}%
{}

\printoption{separating/gomory/freq}%
{$-1\leq\textrm{integer}$}%
{$0$}%
{frequency for calling separator $<$gomory$>$ (-1: never, 0: only in root node)}%
{}

\printoption{separating/gomory/maxrounds}%
{$-1\leq\textrm{integer}$}%
{$5$}%
{maximal number of gomory separation rounds per node (-1: unlimited)}%
{}

\printoption{separating/gomory/maxroundsroot}%
{$-1\leq\textrm{integer}$}%
{$10$}%
{maximal number of gomory separation rounds in the root node (-1: unlimited)}%
{}

\printoption{separating/gomory/maxsepacuts}%
{$0\leq\textrm{integer}$}%
{$50$}%
{maximal number of gomory cuts separated per separation round}%
{}

\printoption{separating/gomory/maxsepacutsroot}%
{$0\leq\textrm{integer}$}%
{$200$}%
{maximal number of gomory cuts separated per separation round in the root node}%
{}

\printoption{separating/impliedbounds/freq}%
{$-1\leq\textrm{integer}$}%
{$0$}%
{frequency for calling separator $<$impliedbounds$>$ (-1: never, 0: only in root node)}%
{}

\printoption{separating/intobj/freq}%
{$-1\leq\textrm{integer}$}%
{$-1$}%
{frequency for calling separator $<$intobj$>$ (-1: never, 0: only in root node)}%
{}

\printoption{separating/maxbounddist}%
{$0\leq\textrm{real}\leq1$}%
{$1$}%
{maximal relative distance from current node's dual bound to primal bound compared to best node's dual bound for applying separation (0.0: only on current best node, 1.0: on all nodes)}%
{}

\printoption{separating/maxcuts}%
{$0\leq\textrm{integer}$}%
{$100$}%
{maximal number of cuts separated per separation round (0: disable local separation)}%
{}

\printoption{separating/maxcutsroot}%
{$0\leq\textrm{integer}$}%
{$2000$}%
{maximal number of separated cuts at the root node (0: disable root node separation)}%
{}

\printoption{separating/maxrounds}%
{$-1\leq\textrm{integer}$}%
{$5$}%
{maximal number of separation rounds per node (-1: unlimited)}%
{}

\printoption{separating/maxroundsroot}%
{$-1\leq\textrm{integer}$}%
{$-1$}%
{maximal number of separation rounds in the root node (-1: unlimited)}%
{}

\printoption{separating/maxstallrounds}%
{$-1\leq\textrm{integer}$}%
{$5$}%
{maximal number of consecutive separation rounds without objective or integrality improvement (-1: no additional restriction)}%
{}

\printoption{separating/mcf/dynamiccuts}%
{boolean}%
{TRUE}%
{should generated cuts be removed from the LP if they are no longer tight?}%
{}

\printoption{separating/mcf/freq}%
{$-1\leq\textrm{integer}$}%
{$0$}%
{frequency for calling separator $<$mcf$>$ (-1: never, 0: only in root node)}%
{}

\printoption{separating/mcf/maxsepacuts}%
{$-1\leq\textrm{integer}$}%
{$100$}%
{maximal number of mcf cuts separated per separation round}%
{}

\printoption{separating/mcf/maxsepacutsroot}%
{$-1\leq\textrm{integer}$}%
{$200$}%
{maximal number of mcf cuts separated per separation round in the root node  -- default separation}%
{}

\printoption{separating/minefficacy}%
{$0\leq\textrm{real}$}%
{$0.05$}%
{minimal efficacy for a cut to enter the LP}%
{}

\printoption{separating/minefficacyroot}%
{$0\leq\textrm{real}$}%
{$0.01$}%
{minimal efficacy for a cut to enter the LP in the root node}%
{}

\printoption{separating/minortho}%
{$0\leq\textrm{real}\leq1$}%
{$0.5$}%
{minimal orthogonality for a cut to enter the LP}%
{}

\printoption{separating/minorthoroot}%
{$0\leq\textrm{real}\leq1$}%
{$0.5$}%
{minimal orthogonality for a cut to enter the LP in the root node}%
{}

\printoption{separating/oddcycle/freq}%
{$-1\leq\textrm{integer}$}%
{$-1$}%
{frequency for calling separator $<$oddcycle$>$ (-1: never, 0: only in root node)}%
{}

\printoption{separating/oddcycle/liftoddcycles}%
{boolean}%
{FALSE}%
{should odd cycle cuts be lifted?}%
{}

\printoption{separating/oddcycle/maxrounds}%
{$-1\leq\textrm{integer}$}%
{$10$}%
{maximal number of oddcycle separation rounds per node (-1: unlimited)}%
{}

\printoption{separating/oddcycle/maxroundsroot}%
{$-1\leq\textrm{integer}$}%
{$10$}%
{maximal number of oddcycle separation rounds in the root node (-1: unlimited)}%
{}

\printoption{separating/oddcycle/maxsepacuts}%
{$0\leq\textrm{integer}$}%
{$5000$}%
{maximal number of oddcycle cuts separated per separation round}%
{}

\printoption{separating/oddcycle/maxsepacutsroot}%
{$0\leq\textrm{integer}$}%
{$5000$}%
{maximal number of oddcycle cuts separated per separation round in the root node}%
{}

\printoption{separating/oddcycle/usegls}%
{boolean}%
{TRUE}%
{should the search method by Groetschel, Lovasz, Schrijver be used? Otherwise use levelgraph method by Hoffman, Padberg.}%
{}

\printoption{separating/poolfreq}%
{$-1\leq\textrm{integer}$}%
{$0$}%
{separation frequency for the global cut pool (-1: disable global cut pool, 0: only separate pool at the root)}%
{}

\printoption{separating/rapidlearning/freq}%
{$-1\leq\textrm{integer}$}%
{$-1$}%
{frequency for calling separator $<$rapidlearning$>$ (-1: never, 0: only in root node)}%
{}

\printoption{separating/strongcg/dynamiccuts}%
{boolean}%
{TRUE}%
{should generated cuts be removed from the LP if they are no longer tight?}%
{}

\printoption{separating/strongcg/freq}%
{$-1\leq\textrm{integer}$}%
{$0$}%
{frequency for calling separator $<$strongcg$>$ (-1: never, 0: only in root node)}%
{}

\printoption{separating/strongcg/maxrounds}%
{$-1\leq\textrm{integer}$}%
{$5$}%
{maximal number of strong CG separation rounds per node (-1: unlimited)}%
{}

\printoption{separating/strongcg/maxroundsroot}%
{$-1\leq\textrm{integer}$}%
{$20$}%
{maximal number of strong CG separation rounds in the root node (-1: unlimited)}%
{}

\printoption{separating/strongcg/maxsepacuts}%
{$0\leq\textrm{integer}$}%
{$50$}%
{maximal number of strong CG cuts separated per separation round}%
{}

\printoption{separating/strongcg/maxsepacutsroot}%
{$0\leq\textrm{integer}$}%
{$500$}%
{maximal number of strong CG cuts separated per separation round in the root node}%
{}

\printoption{separating/zerohalf/dynamiccuts}%
{boolean}%
{TRUE}%
{should generated cuts be removed from the LP if they are no longer tight?}%
{}

\printoption{separating/zerohalf/freq}%
{$-1\leq\textrm{integer}$}%
{$-1$}%
{frequency for calling separator $<$zerohalf$>$ (-1: never, 0: only in root node)}%
{}

\printoption{separating/zerohalf/maxrounds}%
{$-1\leq\textrm{integer}$}%
{$5$}%
{maximal number of zerohalf separation rounds per node (-1: unlimited)}%
{}

\printoption{separating/zerohalf/maxroundsroot}%
{$-1\leq\textrm{integer}$}%
{$10$}%
{maximal number of zerohalf separation rounds in the root node (-1: unlimited)}%
{}

\printoption{separating/zerohalf/maxsepacuts}%
{$0\leq\textrm{integer}$}%
{$50$}%
{maximal number of {0,1/2}-cuts separated per separation round}%
{}

\printoption{separating/zerohalf/maxsepacutsroot}%
{$0\leq\textrm{integer}$}%
{$500$}%
{maximal number of {0,1/2}-cuts separated per separation round in the root node}%
{}

\printoption{separating/zerohalf/preprocessing/decomposeproblem}%
{boolean}%
{FALSE}%
{should problem be decomposed into subproblems (if possible) before applying preprocessing?}%
{}

\printoption{separating/zerohalf/preprocessing/delta}%
{$0\leq\textrm{real}\leq1$}%
{$0.5$}%
{value of delta parameter used in preprocessing method 'd'}%
{}

\printoption{separating/zerohalf/preprocessing/ppmethods}%
{string}%
{CXGXIM}%
{preprocessing methods and ordering:\\   \#                      'd' columns with small LP solution,\\   \#                      'G' modified Gaussian elimination,\\   \#                      'i' identical columns,\\   \#                      'I' identical rows,\\   \#                      'L' large slack rows,\\   \#                      'M' large slack rows (minslack),\\   \#                      's' column singletons,\\   \#                      'X' add trivial zerohalf cuts,\\   \#                      'z' zero columns,\\   \#                      'Z' zero rows,\\   \#                      'C' fast {'z','s'},\\   \#                      'R' fast {'Z','L','I'}\\   \#
   \#                      '-' no preprocessing\\   \#}%
{}

\printoption{separating/zerohalf/separating/auxip/objective}%
{character}%
{v}%
{auxiliary IP objective:\\   \#                      'v' maximize cut violation,\\   \#                      'u' minimize number of aggregated rows in cut,\\   \#                      'w' minimize number of aggregated rows in cut\\   \#                          weighted by the number of rows in the aggregation,\\   \#                      'p' maximize cut violation and penalize a high number\\   \#                          of aggregated rows in the cut weighted by the number\\   \#                          of rows in the aggregation and the penalty factor p\\   \#}%
{}

\printoption{separating/zerohalf/separating/auxip/penaltyfactor}%
{$0\leq\textrm{real}\leq1$}%
{$0.001$}%
{penalty factor used with objective function 'p' of auxiliary IP}%
{}

\printoption{separating/zerohalf/separating/auxip/settingsfile}%
{string}%
{-}%
{optional settings file of the auxiliary IP (-: none)}%
{}

\printoption{separating/zerohalf/separating/auxip/sollimit}%
{$-1\leq\textrm{integer}$}%
{$-1$}%
{limits/solutions setting of the auxiliary IP}%
{}

\printoption{separating/zerohalf/separating/auxip/useallsols}%
{boolean}%
{TRUE}%
{should all (proper) solutions of the auxiliary IP be used to generate cuts instead of using only the best?}%
{}

\printoption{separating/zerohalf/separating/forcecutstolp}%
{boolean}%
{FALSE}%
{should the cuts be forced to enter the LP?}%
{}

\printoption{separating/zerohalf/separating/forcecutstosepastore}%
{boolean}%
{FALSE}%
{should the cuts be forced to enter SCIP's sepastore?}%
{}

\printoption{separating/zerohalf/separating/minviolation}%
{$0.001\leq\textrm{real}\leq0.5$}%
{$0.3$}%
{minimal violation of a {0,1/2}-cut to be separated}%
{}

\printoption{separating/zerohalf/separating/sepamethods}%
{string}%
{2g}%
{separating methods and ordering:\\   \#                      '!' stop further processing if a cut was found,\\   \#                      '2' exact polynomial time algorithm (only if matrix has max 2 odd entries per row),\\   \#                      'e' enumeration heuristics (k=1: try all preprocessed rows),\\   \#                      'E' enumeration heuristics (k=2: try all combinations of up to two preprocessed rows),\\   \#                      'g' Extended Gaussian elimination heuristics,\\   \#                      's' auxiliary IP heuristics (i.e. number of solved nodes is limited)\\   \#                      'S' auxiliary IP exact      (i.e. unlimited number of nodes)\\   \#
   \#                      '-' no processing\\   \#}%
{}

\printoptioncategory{Separation (advanced options)}
\printoption{separating/cgmip/cutcoefbnd}%
{$0\leq\textrm{real}$}%
{$1000$}%
{bounds on the values of the coefficients in the CG-cut}%
{}

\printoption{separating/cgmip/delay}%
{boolean}%
{FALSE}%
{should separator be delayed, if other separators found cuts?}%
{}

\printoption{separating/cgmip/maxbounddist}%
{$0\leq\textrm{real}\leq1$}%
{$0$}%
{maximal relative distance from current node's dual bound to primal bound compared to best node's dual bound for applying separator $<$cgmip$>$ (0.0: only on current best node, 1.0: on all nodes)}%
{}

\printoption{separating/cgmip/memorylimit}%
{$0\leq\textrm{real}$}%
{$\infty$}%
{memory limit for sub-MIP}%
{}

\printoption{separating/cgmip/objweight}%
{$0\leq\textrm{real}$}%
{$0.001$}%
{weight used for the row combination coefficient in the sub-MIP objective}%
{}

\printoption{separating/cgmip/priority}%
{$-536870912\leq\textrm{integer}\leq536870911$}%
{$-1000$}%
{priority of separator $<$cgmip$>$}%
{}

\printoption{separating/cgmip/timelimit}%
{$0\leq\textrm{real}$}%
{$\infty$}%
{time limit for sub-MIP}%
{}

\printoption{separating/clique/backtrackfreq}%
{$0\leq\textrm{integer}$}%
{$1000$}%
{frequency for premature backtracking up to tree level 1 (0: no backtracking)}%
{}

\printoption{separating/clique/cliquedensity}%
{$0\leq\textrm{real}\leq1$}%
{$0.05$}%
{minimal density of cliques to use a dense clique table}%
{}

\printoption{separating/clique/cliquetablemem}%
{$0\leq\textrm{real}\leq2.09715 \cdot 10^{  6}$}%
{$20000$}%
{maximal memory size of dense clique table (in kb)}%
{}

\printoption{separating/clique/delay}%
{boolean}%
{FALSE}%
{should separator be delayed, if other separators found cuts?}%
{}

\printoption{separating/clique/maxbounddist}%
{$0\leq\textrm{real}\leq1$}%
{$0$}%
{maximal relative distance from current node's dual bound to primal bound compared to best node's dual bound for applying separator $<$clique$>$ (0.0: only on current best node, 1.0: on all nodes)}%
{}

\printoption{separating/clique/maxtreenodes}%
{$-1\leq\textrm{integer}$}%
{$10000$}%
{maximal number of nodes in branch and bound tree (-1: no limit)}%
{}

\printoption{separating/clique/maxzeroextensions}%
{$-1\leq\textrm{integer}$}%
{$1000$}%
{maximal number of zero-valued variables extending the clique (-1: no limit)}%
{}

\printoption{separating/clique/priority}%
{$-536870912\leq\textrm{integer}\leq536870911$}%
{$-5000$}%
{priority of separator $<$clique$>$}%
{}

\printoption{separating/clique/scaleval}%
{$1\leq\textrm{real}$}%
{$1000$}%
{factor for scaling weights}%
{}

\printoption{separating/closecuts/closethres}%
{$-1\leq\textrm{integer}$}%
{$50$}%
{threshold on number of generated cuts below which the ordinary separation is started}%
{}

\printoption{separating/closecuts/delay}%
{boolean}%
{FALSE}%
{should separator be delayed, if other separators found cuts?}%
{}

\printoption{separating/closecuts/inclobjcutoff}%
{boolean}%
{FALSE}%
{include an objective cutoff when computing the relative interior?}%
{}

\printoption{separating/closecuts/maxbounddist}%
{$0\leq\textrm{real}\leq1$}%
{$1$}%
{maximal relative distance from current node's dual bound to primal bound compared to best node's dual bound for applying separator $<$closecuts$>$ (0.0: only on current best node, 1.0: on all nodes)}%
{}

\printoption{separating/closecuts/maxlpiterfactor}%
{$-1\leq\textrm{real}$}%
{$2$}%
{factor for maximal LP iterations in relative interior computation compared to node LP iterations (negative for no limit)}%
{}

\printoption{separating/closecuts/maxunsuccessful}%
{$-1\leq\textrm{integer}$}%
{$0$}%
{turn off separation in current node after unsuccessful calls (-1 never turn off)}%
{}

\printoption{separating/closecuts/priority}%
{$-536870912\leq\textrm{integer}\leq536870911$}%
{$1000000$}%
{priority of separator $<$closecuts$>$}%
{}

\printoption{separating/closecuts/recomputerelint}%
{boolean}%
{FALSE}%
{recompute relative interior point in each separation call?}%
{}

\printoption{separating/closecuts/relintnormtype}%
{character}%
{o}%
{type of norm to use when computing relative interior: 'o'ne norm, 's'upremum norm}%
{}

\printoption{separating/closecuts/sepacombvalue}%
{$0\leq\textrm{real}\leq1$}%
{$0.3$}%
{convex combination value for close cuts}%
{}

\printoption{separating/closecuts/separelint}%
{boolean}%
{TRUE}%
{generate close cuts w.r.t. relative interior point (best solution otherwise)?}%
{}

\printoption{separating/cmir/aggrtol}%
{$0\leq\textrm{real}$}%
{$0.1$}%
{tolerance for bound distances used to select continuous variable in current aggregated constraint to be eliminated}%
{}

\printoption{separating/cmir/delay}%
{boolean}%
{FALSE}%
{should separator be delayed, if other separators found cuts?}%
{}

\printoption{separating/cmir/densityoffset}%
{$0\leq\textrm{integer}$}%
{$100$}%
{additional number of variables allowed in row on top of density}%
{}

\printoption{separating/cmir/densityscore}%
{$0\leq\textrm{real}$}%
{$0.0001$}%
{weight of row density in the aggregation scoring of the rows}%
{}

\printoption{separating/cmir/fixintegralrhs}%
{boolean}%
{TRUE}%
{should an additional variable be complemented if f0 = 0?}%
{}

\printoption{separating/cmir/maxaggdensity}%
{$0\leq\textrm{real}\leq1$}%
{$0.2$}%
{maximal density of aggregated row}%
{}

\printoption{separating/cmir/maxaggrs}%
{$0\leq\textrm{integer}$}%
{$3$}%
{maximal number of aggregations for each row per separation round}%
{}

\printoption{separating/cmir/maxaggrsroot}%
{$0\leq\textrm{integer}$}%
{$6$}%
{maximal number of aggregations for each row per separation round in the root node}%
{}

\printoption{separating/cmir/maxbounddist}%
{$0\leq\textrm{real}\leq1$}%
{$0$}%
{maximal relative distance from current node's dual bound to primal bound compared to best node's dual bound for applying separator $<$cmir$>$ (0.0: only on current best node, 1.0: on all nodes)}%
{}

\printoption{separating/cmir/maxconts}%
{$0\leq\textrm{integer}$}%
{$10$}%
{maximal number of active continuous variables in aggregated row}%
{}

\printoption{separating/cmir/maxcontsroot}%
{$0\leq\textrm{integer}$}%
{$10$}%
{maximal number of active continuous variables in aggregated row in the root node}%
{}

\printoption{separating/cmir/maxfails}%
{$-1\leq\textrm{integer}$}%
{$20$}%
{maximal number of consecutive unsuccessful aggregation tries (-1: unlimited)}%
{}

\printoption{separating/cmir/maxfailsroot}%
{$-1\leq\textrm{integer}$}%
{$100$}%
{maximal number of consecutive unsuccessful aggregation tries in the root node (-1: unlimited)}%
{}

\printoption{separating/cmir/maxrowdensity}%
{$0\leq\textrm{real}\leq1$}%
{$0.05$}%
{maximal density of row to be used in aggregation}%
{}

\printoption{separating/cmir/maxrowfac}%
{$0\leq\textrm{real}$}%
{$10000$}%
{maximal row aggregation factor}%
{}

\printoption{separating/cmir/maxslack}%
{$0\leq\textrm{real}$}%
{$0$}%
{maximal slack of rows to be used in aggregation}%
{}

\printoption{separating/cmir/maxslackroot}%
{$0\leq\textrm{real}$}%
{$0.1$}%
{maximal slack of rows to be used in aggregation in the root node}%
{}

\printoption{separating/cmir/maxtestdelta}%
{$-1\leq\textrm{integer}$}%
{$-1$}%
{maximal number of different deltas to try (-1: unlimited)}%
{}

\printoption{separating/cmir/maxtries}%
{$-1\leq\textrm{integer}$}%
{$100$}%
{maximal number of rows to start aggregation with per separation round (-1: unlimited)}%
{}

\printoption{separating/cmir/maxtriesroot}%
{$-1\leq\textrm{integer}$}%
{$-1$}%
{maximal number of rows to start aggregation with per separation round in the root node (-1: unlimited)}%
{}

\printoption{separating/cmir/priority}%
{$-536870912\leq\textrm{integer}\leq536870911$}%
{$-3000$}%
{priority of separator $<$cmir$>$}%
{}

\printoption{separating/cmir/slackscore}%
{$0\leq\textrm{real}$}%
{$0.001$}%
{weight of slack in the aggregation scoring of the rows}%
{}

\printoption{separating/cmir/trynegscaling}%
{boolean}%
{TRUE}%
{should negative values also be tested in scaling?}%
{}

\printoption{separating/cutagelimit}%
{$-1\leq\textrm{integer}$}%
{$100$}%
{maximum age a cut can reach before it is deleted from the global cut pool, or -1 to keep all cuts}%
{}

\printoption{separating/efficacynorm}%
{character}%
{e}%
{row norm to use for efficacy calculation ('e'uclidean, 'm'aximum, 's'um, 'd'iscrete)}%
{}

\printoption{separating/flowcover/delay}%
{boolean}%
{FALSE}%
{should separator be delayed, if other separators found cuts?}%
{}

\printoption{separating/flowcover/maxbounddist}%
{$0\leq\textrm{real}\leq1$}%
{$0$}%
{maximal relative distance from current node's dual bound to primal bound compared to best node's dual bound for applying separator $<$flowcover$>$ (0.0: only on current best node, 1.0: on all nodes)}%
{}

\printoption{separating/flowcover/maxfails}%
{$-1\leq\textrm{integer}$}%
{$50$}%
{maximal number of consecutive fails to generate a cut per separation round (-1: unlimited)}%
{}

\printoption{separating/flowcover/maxfailsroot}%
{$-1\leq\textrm{integer}$}%
{$100$}%
{maximal number of consecutive fails to generate a cut per separation round in the root (-1: unlimited)}%
{}

\printoption{separating/flowcover/maxrowdensity}%
{$0\leq\textrm{real}\leq1$}%
{$1$}%
{maximal density of row to separate flow cover cuts for}%
{}

\printoption{separating/flowcover/maxslack}%
{$0\leq\textrm{real}$}%
{$\infty$}%
{maximal slack of rows to separate flow cover cuts for}%
{}

\printoption{separating/flowcover/maxslackroot}%
{$0\leq\textrm{real}$}%
{$\infty$}%
{maximal slack of rows to separate flow cover cuts for in the root}%
{}

\printoption{separating/flowcover/maxtestdelta}%
{$0\leq\textrm{integer}$}%
{$10$}%
{cut generation heuristic: maximal number of different deltas to try}%
{}

\printoption{separating/flowcover/maxtries}%
{$-1\leq\textrm{integer}$}%
{$100$}%
{maximal number of rows to separate flow cover cuts for per separation round (-1: unlimited)}%
{}

\printoption{separating/flowcover/maxtriesroot}%
{$-1\leq\textrm{integer}$}%
{$-1$}%
{maximal number of rows to separate flow cover cuts for per separation round in the root (-1: unlimited)}%
{}

\printoption{separating/flowcover/multbyminusone}%
{boolean}%
{TRUE}%
{should flow cover cuts be separated for 0-1 single node flow set with reversed arcs in addition?}%
{}

\printoption{separating/flowcover/priority}%
{$-536870912\leq\textrm{integer}\leq536870911$}%
{$-4000$}%
{priority of separator $<$flowcover$>$}%
{}

\printoption{separating/flowcover/slackscore}%
{$0\leq\textrm{real}$}%
{$0.001$}%
{weight of slack in the scoring of the rows}%
{}

\printoption{separating/gomory/delay}%
{boolean}%
{FALSE}%
{should separator be delayed, if other separators found cuts?}%
{}

\printoption{separating/gomory/delayedcuts}%
{boolean}%
{TRUE}%
{should cuts be added to the delayed cut pool?}%
{}

\printoption{separating/gomory/forcecuts}%
{boolean}%
{TRUE}%
{if conversion to integral coefficients failed still use the cut}%
{}

\printoption{separating/gomory/makeintegral}%
{boolean}%
{TRUE}%
{try to scale cuts to integral coefficients}%
{}

\printoption{separating/gomory/maxbounddist}%
{$0\leq\textrm{real}\leq1$}%
{$0$}%
{maximal relative distance from current node's dual bound to primal bound compared to best node's dual bound for applying separator $<$gomory$>$ (0.0: only on current best node, 1.0: on all nodes)}%
{}

\printoption{separating/gomory/maxweightrange}%
{$1\leq\textrm{real}$}%
{$10000$}%
{maximal valid range max($|$weights$|$)/min($|$weights$|$) of row weights}%
{}

\printoption{separating/gomory/priority}%
{$-536870912\leq\textrm{integer}\leq536870911$}%
{$-1000$}%
{priority of separator $<$gomory$>$}%
{}

\printoption{separating/gomory/separaterows}%
{boolean}%
{TRUE}%
{separate rows with integral slack}%
{}

\printoption{separating/impliedbounds/delay}%
{boolean}%
{FALSE}%
{should separator be delayed, if other separators found cuts?}%
{}

\printoption{separating/impliedbounds/maxbounddist}%
{$0\leq\textrm{real}\leq1$}%
{$0$}%
{maximal relative distance from current node's dual bound to primal bound compared to best node's dual bound for applying separator $<$impliedbounds$>$ (0.0: only on current best node, 1.0: on all nodes)}%
{}

\printoption{separating/impliedbounds/priority}%
{$-536870912\leq\textrm{integer}\leq536870911$}%
{$-50$}%
{priority of separator $<$impliedbounds$>$}%
{}

\printoption{separating/intobj/delay}%
{boolean}%
{FALSE}%
{should separator be delayed, if other separators found cuts?}%
{}

\printoption{separating/intobj/maxbounddist}%
{$0\leq\textrm{real}\leq1$}%
{$0$}%
{maximal relative distance from current node's dual bound to primal bound compared to best node's dual bound for applying separator $<$intobj$>$ (0.0: only on current best node, 1.0: on all nodes)}%
{}

\printoption{separating/intobj/priority}%
{$-536870912\leq\textrm{integer}\leq536870911$}%
{$-100$}%
{priority of separator $<$intobj$>$}%
{}

\printoption{separating/maxaddrounds}%
{$-1\leq\textrm{integer}$}%
{$1$}%
{maximal additional number of separation rounds in subsequent price-and-cut loops (-1: no additional restriction)}%
{}

\printoption{separating/maxroundsrootsubrun}%
{$-1\leq\textrm{integer}$}%
{$1$}%
{maximal number of separation rounds in the root node of a subsequent run (-1: unlimited)}%
{}

\printoption{separating/maxruns}%
{$-1\leq\textrm{integer}$}%
{$-1$}%
{maximal number of runs for which separation is enabled (-1: unlimited)}%
{}

\printoption{separating/mcf/checkcutshoreconnectivity}%
{boolean}%
{TRUE}%
{should we separate only if the cuts shores are connected?}%
{}

\printoption{separating/mcf/delay}%
{boolean}%
{FALSE}%
{should separator be delayed, if other separators found cuts?}%
{}

\printoption{separating/mcf/fixintegralrhs}%
{boolean}%
{TRUE}%
{should an additional variable be complemented if f0 = 0?}%
{}

\printoption{separating/mcf/maxarcinconsistencyratio}%
{$0\leq\textrm{real}$}%
{$0.5$}%
{maximum inconsistency ratio of arcs not to be deleted}%
{}

\printoption{separating/mcf/maxbounddist}%
{$0\leq\textrm{real}\leq1$}%
{$0$}%
{maximal relative distance from current node's dual bound to primal bound compared to best node's dual bound for applying separator $<$mcf$>$ (0.0: only on current best node, 1.0: on all nodes)}%
{}

\printoption{separating/mcf/maxinconsistencyratio}%
{$0\leq\textrm{real}$}%
{$0.02$}%
{maximum inconsistency ratio for separation at all}%
{}

\printoption{separating/mcf/maxtestdelta}%
{$-1\leq\textrm{integer}$}%
{$20$}%
{maximal number of different deltas to try (-1: unlimited)  -- default separation}%
{}

\printoption{separating/mcf/maxweightrange}%
{$1\leq\textrm{real}$}%
{$10^{  6}$}%
{maximal valid range max($|$weights$|$)/min($|$weights$|$) of row weights}%
{}

\printoption{separating/mcf/modeltype}%
{$0\leq\textrm{integer}\leq2$}%
{$0$}%
{model type of network (0: auto, 1:directed, 2:undirected)}%
{}

\printoption{separating/mcf/nclusters}%
{$2\leq\textrm{integer}\leq32$}%
{$5$}%
{number of clusters to generate in the shrunken network -- default separation}%
{}

\printoption{separating/mcf/priority}%
{$-536870912\leq\textrm{integer}\leq536870911$}%
{$-10000$}%
{priority of separator $<$mcf$>$}%
{}

\printoption{separating/mcf/separateflowcutset}%
{boolean}%
{TRUE}%
{should we separate flowcutset inequalities on the network cuts?}%
{}

\printoption{separating/mcf/separateknapsack}%
{boolean}%
{TRUE}%
{should we separate knapsack cover inequalities on the network cuts?}%
{}

\printoption{separating/mcf/separatesinglenodecuts}%
{boolean}%
{TRUE}%
{should we separate inequalities based on single-node cuts?}%
{}

\printoption{separating/mcf/trynegscaling}%
{boolean}%
{FALSE}%
{should negative values also be tested in scaling?}%
{}

\printoption{separating/objparalfac}%
{$0\leq\textrm{real}$}%
{$0.0001$}%
{factor to scale objective parallelism of cut in separation score calculation}%
{}

\printoption{separating/oddcycle/addselfarcs}%
{boolean}%
{TRUE}%
{add links between a variable and its negated}%
{}

\printoption{separating/oddcycle/allowmultiplecuts}%
{boolean}%
{TRUE}%
{even if a variable is already covered by a cut, still allow another cut to cover it too}%
{}

\printoption{separating/oddcycle/delay}%
{boolean}%
{FALSE}%
{should separator be delayed, if other separators found cuts?}%
{}

\printoption{separating/oddcycle/includetriangles}%
{boolean}%
{TRUE}%
{separate triangles found as 3-cycles or repaired larger cycles}%
{}

\printoption{separating/oddcycle/lpliftcoef}%
{boolean}%
{FALSE}%
{choose lifting candidate by coef*lpvalue or only by coef}%
{}

\printoption{separating/oddcycle/maxbounddist}%
{$0\leq\textrm{real}\leq1$}%
{$1$}%
{maximal relative distance from current node's dual bound to primal bound compared to best node's dual bound for applying separator $<$oddcycle$>$ (0.0: only on current best node, 1.0: on all nodes)}%
{}

\printoption{separating/oddcycle/maxcutslevel}%
{$0\leq\textrm{integer}$}%
{$50$}%
{maximal number of oddcycle cuts generated in every level of the level graph}%
{}

\printoption{separating/oddcycle/maxcutsroot}%
{$0\leq\textrm{integer}$}%
{$1$}%
{maximal number of oddcycle cuts generated per chosen variable as root of the level graph}%
{}

\printoption{separating/oddcycle/maxnlevels}%
{$0\leq\textrm{integer}$}%
{$20$}%
{maximal number of levels in level graph}%
{}

\printoption{separating/oddcycle/maxpernodeslevel}%
{$0\leq\textrm{integer}\leq100$}%
{$100$}%
{percentage of nodes allowed in the same level of the level graph [0-100]}%
{}

\printoption{separating/oddcycle/maxreference}%
{$0\leq\textrm{integer}$}%
{$0$}%
{minimal weight on an edge (in level graph or bipartite graph)}%
{}

\printoption{separating/oddcycle/maxunsucessfull}%
{$0\leq\textrm{integer}$}%
{$3$}%
{number of unsuccessful calls at current node}%
{}

\printoption{separating/oddcycle/multiplecuts}%
{boolean}%
{FALSE}%
{even if a variable is already covered by a cut, still try it as start node for a cycle search}%
{}

\printoption{separating/oddcycle/offsetnodeslevel}%
{$0\leq\textrm{integer}$}%
{$10$}%
{offset of nodes allowed in the same level of the level graph (additional to the percentage of levelnodes)}%
{}

\printoption{separating/oddcycle/offsettestvars}%
{$0\leq\textrm{integer}$}%
{$100$}%
{offset of variables to try the chosen method on (additional to the percentage of testvars)}%
{}

\printoption{separating/oddcycle/percenttestvars}%
{$0\leq\textrm{integer}\leq100$}%
{$0$}%
{percentage of variables to try the chosen method on [0-100]}%
{}

\printoption{separating/oddcycle/priority}%
{$-536870912\leq\textrm{integer}\leq536870911$}%
{$-15000$}%
{priority of separator $<$oddcycle$>$}%
{}

\printoption{separating/oddcycle/recalcliftcoef}%
{boolean}%
{TRUE}%
{calculate lifting coefficient of every candidate in every step (or only if its chosen)}%
{}

\printoption{separating/oddcycle/repaircycles}%
{boolean}%
{TRUE}%
{try to repair violated cycles with double appearance of a variable}%
{}

\printoption{separating/oddcycle/scalingfactor}%
{$1\leq\textrm{integer}$}%
{$1000$}%
{factor for scaling of the arc-weights}%
{}

\printoption{separating/oddcycle/sortrootneighbors}%
{boolean}%
{TRUE}%
{sort level of the root neighbors by fractionality (maxfrac)}%
{}

\printoption{separating/oddcycle/sortswitch}%
{$0\leq\textrm{integer}\leq4$}%
{$3$}%
{use sorted variable array (unsorted(0),maxlp(1),minlp(2),maxfrac(3),minfrac(4))}%
{}

\printoption{separating/orthofac}%
{$0\leq\textrm{real}$}%
{$1$}%
{factor to scale orthogonality of cut in separation score calculation (0.0 to disable orthogonality calculation)}%
{}

\printoption{separating/orthofunc}%
{character}%
{e}%
{function used for calc. scalar prod. in orthogonality test ('e'uclidean, 'd'iscrete)}%
{}

\printoption{separating/rapidlearning/applybdchgs}%
{boolean}%
{TRUE}%
{should the found global bound deductions be applied in the original SCIP?}%
{}

\printoption{separating/rapidlearning/applyconflicts}%
{boolean}%
{TRUE}%
{should the found conflicts be applied in the original SCIP?}%
{}

\printoption{separating/rapidlearning/applyinfervals}%
{boolean}%
{TRUE}%
{should the inference values be used as initialization in the original SCIP?}%
{}

\printoption{separating/rapidlearning/applyprimalsol}%
{boolean}%
{TRUE}%
{should the incumbent solution be copied to the original SCIP?}%
{}

\printoption{separating/rapidlearning/applysolved}%
{boolean}%
{TRUE}%
{should a solved status be copied to the original SCIP?}%
{}

\printoption{separating/rapidlearning/contvars}%
{boolean}%
{FALSE}%
{should rapid learning be applied when there are continuous variables?}%
{}

\printoption{separating/rapidlearning/contvarsquot}%
{$0\leq\textrm{real}\leq1$}%
{$0.3$}%
{maximal portion of continuous variables to apply rapid learning}%
{}

\printoption{separating/rapidlearning/copycuts}%
{boolean}%
{TRUE}%
{should all active cuts from cutpool be copied to constraints in subproblem?}%
{}

\printoption{separating/rapidlearning/delay}%
{boolean}%
{FALSE}%
{should separator be delayed, if other separators found cuts?}%
{}

\printoption{separating/rapidlearning/lpiterquot}%
{$0\leq\textrm{real}$}%
{$0.2$}%
{maximal fraction of LP iterations compared to node LP iterations}%
{}

\printoption{separating/rapidlearning/maxbounddist}%
{$0\leq\textrm{real}\leq1$}%
{$1$}%
{maximal relative distance from current node's dual bound to primal bound compared to best node's dual bound for applying separator $<$rapidlearning$>$ (0.0: only on current best node, 1.0: on all nodes)}%
{}

\printoption{separating/rapidlearning/maxnconss}%
{$0\leq\textrm{integer}$}%
{$10000$}%
{maximum problem size (constraints) for which rapid learning will be called}%
{}

\printoption{separating/rapidlearning/maxnodes}%
{$0\leq\textrm{integer}$}%
{$5000$}%
{maximum number of nodes considered in rapid learning run}%
{}

\printoption{separating/rapidlearning/maxnvars}%
{$0\leq\textrm{integer}$}%
{$10000$}%
{maximum problem size (variables) for which rapid learning will be called}%
{}

\printoption{separating/rapidlearning/minnodes}%
{$0\leq\textrm{integer}$}%
{$500$}%
{minimum number of nodes considered in rapid learning run}%
{}

\printoption{separating/rapidlearning/priority}%
{$-536870912\leq\textrm{integer}\leq536870911$}%
{$-1200000$}%
{priority of separator $<$rapidlearning$>$}%
{}

\printoption{separating/rapidlearning/reducedinfer}%
{boolean}%
{FALSE}%
{should the inference values only be used when rapidlearning found other reductions?}%
{}

\printoption{separating/strongcg/delay}%
{boolean}%
{FALSE}%
{should separator be delayed, if other separators found cuts?}%
{}

\printoption{separating/strongcg/maxbounddist}%
{$0\leq\textrm{real}\leq1$}%
{$0$}%
{maximal relative distance from current node's dual bound to primal bound compared to best node's dual bound for applying separator $<$strongcg$>$ (0.0: only on current best node, 1.0: on all nodes)}%
{}

\printoption{separating/strongcg/maxweightrange}%
{$1\leq\textrm{real}$}%
{$10000$}%
{maximal valid range max($|$weights$|$)/min($|$weights$|$) of row weights}%
{}

\printoption{separating/strongcg/priority}%
{$-536870912\leq\textrm{integer}\leq536870911$}%
{$-2000$}%
{priority of separator $<$strongcg$>$}%
{}

\printoption{separating/zerohalf/delay}%
{boolean}%
{FALSE}%
{should separator be delayed, if other separators found cuts?}%
{}

\printoption{separating/zerohalf/ignoreprevzhcuts}%
{boolean}%
{FALSE}%
{should zerohalf cuts found in previous callbacks ignored?}%
{}

\printoption{separating/zerohalf/maxbounddist}%
{$0\leq\textrm{real}\leq1$}%
{$0$}%
{maximal relative distance from current node's dual bound to primal bound compared to best node's dual bound for applying separator $<$zerohalf$>$ (0.0: only on current best node, 1.0: on all nodes)}%
{}

\printoption{separating/zerohalf/maxcutsfound}%
{$0\leq\textrm{integer}$}%
{$100$}%
{maximal number of {0,1/2}-cuts determined per separation round\\   \#                      (this includes separated but inefficacious cuts)}%
{}

\printoption{separating/zerohalf/maxcutsfoundroot}%
{$0\leq\textrm{integer}$}%
{$1000$}%
{maximal number of {0,1/2}-cuts determined per separation round in the root node\\   \#                      (this includes separated but inefficacious cuts)}%
{}

\printoption{separating/zerohalf/maxdepth}%
{$-1\leq\textrm{integer}$}%
{$-1$}%
{separating cuts only if depth $<$= maxdepth (-1: unlimited)}%
{}

\printoption{separating/zerohalf/maxncalls}%
{$-1\leq\textrm{integer}$}%
{$-1$}%
{maximal number of calls (-1: unlimited)}%
{}

\printoption{separating/zerohalf/maxtestdelta}%
{$-1\leq\textrm{integer}$}%
{$10$}%
{maximal number of different deltas to try for cmir (-1: unlimited, 0: delta=1)}%
{}

\printoption{separating/zerohalf/onlyorigrows}%
{boolean}%
{FALSE}%
{should only original LP rows be considered (i.e. ignore previously added LP rows)?}%
{}

\printoption{separating/zerohalf/priority}%
{$-536870912\leq\textrm{integer}\leq536870911$}%
{$-6000$}%
{priority of separator $<$zerohalf$>$}%
{}

\printoption{separating/zerohalf/relaxcontvars}%
{boolean}%
{FALSE}%
{should continuous variables be relaxed by adding variable bounds?}%
{}

\printoption{separating/zerohalf/scalefraccoeffs}%
{boolean}%
{TRUE}%
{should rows be scaled to make fractional coefficients integer?}%
{}

\printoption{separating/zerohalf/trynegscaling}%
{boolean}%
{TRUE}%
{should negative values also be tested in scaling for cmir?}%
{}

\printoption{separating/zerohalf/usezhcutpool}%
{boolean}%
{TRUE}%
{should zerohalf cuts be filtered using a cutpool?}%
{}

\printoptioncategory{Timing}
\printoption{timing/clocktype}%
{$1\leq\textrm{integer}\leq2$}%
{$1$}%
{default clock type (1: CPU user seconds, 2: wall clock time)}%
{}

\printoption{timing/enabled}%
{boolean}%
{TRUE}%
{is timing enabled?}%
{}

\printoption{timing/reading}%
{boolean}%
{FALSE}%
{belongs reading time to solving time?}%
{}



% \item[\label{scipusercutcall}\hypertarget{scipusercutcall}
% {\textbf{gams/usercutcall (\slshape{string})}}]\hspace{1.0in}
% 
% The \MYGAMS command line (minus the gams executable name) to call the cut generator.
% 
% 
% \item[\label{scipusercutfirst}\hypertarget{scipusercutfirst}
% {\textbf{gams/usercutfirst (\slshape{integer})}}]\hspace{1.0in}
% 
% Calls the cut generator for the first $n$ nodes.
% 
% \textsl{(default = 10)}
% 
% \item[\label{scipusercutfreq}\hypertarget{scipusercutfreq}
% {\textbf{gams/usercutfreq (\slshape{integer})}}]\hspace{1.0in}
% 
% Determines the frequency of the cut generator model calls.
% 
% \textsl{(default = 10)}
% 
% \item[\label{scipusercutinterval}\hypertarget{scipusercutinterval}
% {\textbf{gams/usercutinterval (\slshape{integer})}}]\hspace{1.0in}
% 
% Determines the interval when to apply the multiplier for the frequency of the cut generator model calls.
% See gams/userheurinterval for details.
% 
% \textsl{(default = 100)}
% 
% \item[\label{scipusercutmult}\hypertarget{scipusercutmult}
% {\textbf{gams/usercutmult (\slshape{integer})}}]\hspace{1.0in}
% 
% Determines the multiplier for the frequency of the cut generator model calls.
% 
% \textsl{(default = 2)}
% 
% \item[\label{scipusercutnewint}\hypertarget{scipusercutnewint}
% {\textbf{gams/usercutnewint (\slshape{integer})}}]\hspace{1.0in}
% 
% Calls the cut generator if the solver found a new integer feasible solution.
% 
% \textsl{(default = TRUE)}
% \begin{itemize}
% \item[FALSE] Do not call cut generator because a new integer feasible solution is found.
% \item[TRUE] Let \SCIP call the cut generator if a new integer feasible solution is found.
% \end{itemize}
% 
% \item[\label{scipusergdxin}\hypertarget{scipusergdxin}
% {\textbf{gams/usergdxin (\slshape{string})}}]\hspace{1.0in}
% 
% The name of the GDX file read back into \SCIP.
% 
% \textsl{(default =} \verb=bchin.gdx=)
% 
% \item[\label{scipusergdxname}\hypertarget{scipusergdxname}
% {\textbf{gams/usergdxname (\slshape{string})}}]\hspace{1.0in}
% 
% The name of the GDX file exported from the solver with the solution at the node.
% 
% \textsl{(default =} \verb=bchout.gdx=)
% 
% \item[\label{scipusergdxnameinc}\hypertarget{scipusergdxnameinc}
% {\textbf{gams/usergdxnameinc (\slshape{string})}}]\hspace{1.0in}
% 
% The name of the GDX file exported from the solver with the incumbent solution.
% 
% \textsl{(default =} \verb=bchout_i.gdx=)
% 
% \item[\label{scipusergdxprefix}\hypertarget{scipusergdxprefix}
% {\textbf{gams/usergdxprefix (\slshape{string})}}]\hspace{1.0in}
% 
% Prefixes to use for gams/usergdxin, gams/usergdxname, and gams/usergdxnameinc.
% 
% 
% \item[\label{scipuserheurcall}\hypertarget{scipuserheurcall}
% {\textbf{gams/userheurcall (\slshape{string})}}]\hspace{1.0in}
% 
% The \MYGAMS command line (minus the gams executable name) to call the heuristic.
% 
% 
% \item[\label{scipuserheurfirst}\hypertarget{scipuserheurfirst}
% {\textbf{gams/userheurfirst (\slshape{integer})}}]\hspace{1.0in}
% 
% Calls the heuristic for the first $n$ nodes.
% 
% \textsl{(default = 10)}
% 
% \item[\label{scipuserheurfreq}\hypertarget{scipuserheurfreq}
% {\textbf{gams/userheurfreq (\slshape{integer})}}]\hspace{1.0in}
% 
% Determines the frequency of the heuristic model calls.
% 
% \textsl{(default = 10)}
% 
% \item[\label{scipuserheurinterval}\hypertarget{scipuserheurinterval}
% {\textbf{gams/userheurinterval (\slshape{integer})}}]\hspace{1.0in}
% 
% Determines the interval when to apply the multiplier for the frequency of the heuristic model calls.
% For example, for the first 100 (gams/userheurinterval) nodes, the solver call every 10th (gams/userheurfreq) node the heuristic.
% After 100 nodes, the frequency gets multiplied by 10 (gams/userheurmult), so that for the next 100 node the solver calls the heuristic every 20th node.
% For nodes 200-300, the heuristic get called every 40th node, for nodes 300-400 every 80th node and after node 400 every 100th node.
% 
% \textsl{(default = 100)}
% 
% \item[\label{scipuserheurmult}\hypertarget{scipuserheurmult}
% {\textbf{gams/userheurmult (\slshape{integer})}}]\hspace{1.0in}
% 
% Determines the multiplier for the frequency of the heuristic model calls.
% 
% \textsl{(default = 2)}
% 
% \item[\label{scipuserheurnewint}\hypertarget{scipuserheurnewint}
% {\textbf{gams/userheurnewint (\slshape{integer})}}]\hspace{1.0in}
% 
% Calls the heuristic if the solver found a new integer feasible solution.
% 
% \textsl{(default = TRUE)}
% \begin{itemize}
% \item[FALSE] Do not call heuristic because a new integer feasible solution is found.
% \item[TRUE] Let \SCIP call the heuristic if a new integer feasible solution is found.
% \end{itemize}
% 
% \item[\label{scipuserheurobjfirst}\hypertarget{scipuserheurobjfirst}
% {\textbf{gams/userheurobjfirst (\slshape{integer})}}]\hspace{1.0in}
% 
% Similar to gams/userheurfirst but only calls the heuristic if the relaxed objective value promises a significant improvement of the current incumbent, i.e., the LP value of the node has to be closer to the best bound than the current incumbent.
% 
% \textsl{(default = FALSE)}
% 
% \item[\label{scipuserjobid}\hypertarget{scipuserjobid}
% {\textbf{gams/userjobid (\slshape{string})}}]\hspace{1.0in}
% 
% Postfixes to use for gams/gdxname, gams/gdxnameinc, and gams/gdxin.
% 
% 
% \item[\label{scipuserkeep}\hypertarget{scipuserkeep}
% {\textbf{gams/userkeep (\slshape{integer})}}]\hspace{1.0in}
% 
% Calls gamskeep instead of gams
% 
% \textsl{(default = FALSE)}

\bibliographystyle{plain}
\bibliography{coin_scip}

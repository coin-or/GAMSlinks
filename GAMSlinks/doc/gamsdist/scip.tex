
% \renewcommand{\GAMS}{\textsc{GAMS}\xspace}
\newcommand{\SCIP}{\textsc{SCIP}\xspace}

\chapter{SCIP}
\textbf{Stefan Vigerske, Humboldt University Berlin, Germany}
\vspace{1cm}

\minitoc

\section{Introduction}

\SCIP (\textbf{S}olving \textbf{C}onstraint \textbf{I}nteger \textbf{P}rograms) is developed at the Konrad-Zuse-Zentrum f\"ur Informationstechnik Berlin (ZIB).
The \SCIP main developer is Tobias Achterberg, current developers are Timo Berthold, Gerald Gamrath, Stefan Heinz, Thorsten Koch, Stefan Vigerske, Robert Waniek, Michael Winkler, and Kati Wolter.
Since \SCIP is distributed under the ZIB Academic License, it is only available for users with a \GAMS academic license.

\SCIP is a framework for Constraint Integer Programming oriented towards the needs of Mathematical Programming experts who want to have total control of the solution process and access detailed information down to the guts of the solver.
\SCIP can also be used as a pure MIP solver or as a framework for branch-cut-and-price.
Within \GAMS, the MIP and MIQCP solving facilities of \SCIP are available.

For more detailed information, we refer to \cite{Achterberg2007,Ach09,AchBeKoWo08,Berthold2006,BertholdHeinzVigerske2009,Wolter2006} and the \SCIP web site \url{http://scip.zib.de}.

\GAMS/\SCIP uses the COIN-OR linear solver \textsc{CLP} as LP solver and the COIN-OR Interior Point Optimizer \IPOPT~\cite{WaBi06} as nonlinear solver.

\section{Model requirements}

\SCIP supports continuous, binary, and integer variables, special ordered sets, and branching priorities.
Semi-continuous or semi-integer variables (see chapter 17.1 of the \GAMS User's Guide) and indicator constraints are not supported yet.

\section{Usage}

The following statement can be used inside your \GAMS program to specify using \SCIP
\begin{verbatim}
  Option MIP = SCIP;     { or LP or RMIP or QCP or RMIQCP or MIQCP }
\end{verbatim}

The above statement should appear before the Solve statement.
If \SCIP was specified as the default solver during \GAMS installation, the above statement is not necessary.

If a continuous linear program (LP or RMIP) is given to \GAMS/\SCIP, the linear programming solver is called directly.
In this case, it is not possible to provide an option file.

% \GAMS/\SCIP supports the \GAMS Branch-and-Cut-and-Heuristic (BCH) Facility.
% The \GAMS BCH facility automates all major steps necessary to define, execute, and control the use of user defined routines within the framework of general purpose MIP codes.
% Currently supported are user defined cut generators and heuristics and the incumbent reporting callback.
% Please see the BCH documentation at \texttt{http://www.gams.com/docs/bch.htm} for further information.
% 
% Information on the use of BCH callback routines is displayed in an extra column in the \SCIP iteration output.
% The first number in this column (below the ``BCH'' in the header) is the number of callbacks to \GAMS models that have been made so far (accumulated from cutgeneration, heuristic, and incumbent callbacks).
% The number below ``cut'' or ``cuts'' gives the number of cutting planes that have been generated by the users cutgenerator.
% Finally, the number below ``sol'' or ``sols'' gives the number of primal solutions that have been generated by the users heuristic.
% If \SCIP accepts a heuristic solution as new incumbent solution, it prints a `G' in the first column of the iteration output.

\GAMS/\SCIP currently does not support the \GAMS Branch-and-Cut-and-Heuristic (BCH) Facility.
If you need to use \GAMS/\SCIP with BCH, please consider to use a \GAMS system of version $\leq 23.3$, available at \url{http://www.gams.com/download/download_old.htm}.

\subsubsection{Specification of \SCIP Options}

\GAMS/\SCIP currently supports the \GAMS parameters \texttt{reslim}, \texttt{iterlim}, \texttt{nodlim}, \texttt{optcr}, and \texttt{optca}.

For a MIP, QCP, or MIQCP solve the user can specify options by a \SCIP options file.
A \SCIP options file consists of one option or comment per line.
A pound sign (\#) at the beginning of a line causes the entire line to be ignored.
Otherwise, the line will be interpreted as an option name and value separated by an equal sign (=) and any amount of white space (blanks or tabs).
Further, string values have to be enclosed in quotation marks.

A small example for a scip.opt file is:
\begin{verbatim}
  presolving/probing/maxrounds = 0
  separating/maxrounds         = 0
  separating/maxroundsroot     = 0
\end{verbatim}
%   gams/usercutcall         = "bchcutgen.gms"
It causes \GAMS/\SCIP to disable probing during presolve and to turn off all cut generators.
% and to use a user defined cut generator.

\section{Detailed Options Description}

%TODO put all non-advanced options here

\SCIP supports a large set of options.
Sample option files can be obtained from
\begin{verbatim}
     http://www.gams.com/~svigerske/scip1.2
\end{verbatim}

Further, there is a set of options that are specific to the \GAMS/\SCIP interface.
% , most of them for control of the \GAMS BCH facility.


\begin{description}
\item[\label{scipinteractive}\hypertarget{scipinteractive}
{\textbf{gams/interactive (\slshape{integer})}}]\hspace{1.0in}

whether a SCIP shell should be opened instead of starting the solution process.
This option is not available in demo mode.

\textsl{(default = FALSE)}
\begin{itemize}
\item[FALSE] Do not open the SCIP shell.
\item[TRUE] Open the SCIP shell.
\end{itemize}


\item[\label{scipmipstart}\hypertarget{scipmipstart}
{\textbf{gams/mipstart (\slshape{integer})}}]\hspace{1.0in}

This option controls the use of advanced starting values for mixed integer programs.
A setting of TRUE indicates that the variable level values should be checked to see if they provide an integer feasible solution before starting optimization.

\textsl{(default = TRUE)}
\begin{itemize}
\item[FALSE] Do not use the initial variable levels.
\item[TRUE] Try to use the initial variable levels as a MIP starting solution.
\end{itemize}


\item[\label{scipnames}\hypertarget{scipnames}
{\textbf{gams/names (\slshape{integer})}}]\hspace{1.0in}

This option causes \GAMS names for the variables and equations to be loaded into \SCIP.
These names will then be used for error messages, log entries, and so forth.
Turning names off may help if memory is very tight.

\textsl{(default = FALSE)}
\begin{itemize}
\item[FALSE] Do not load variable and equation names.
\item[TRUE] Load variable and equation names.
\end{itemize}


\item[\label{scipprintstat}\hypertarget{scipprintstat}
{\textbf{gams/print\_statistics (\slshape{integer})}}]\hspace{1.0in}

This option controls the printing of solve statistics after a MIP solve.
Turning on this option indicates that statistics like the number of
generated cuts of each type or the calls of heuristics are printed after the
MIP solve.

\textsl{(default = FALSE)}
\begin{itemize}
\item[FALSE] Do not print statistics.
\item[TRUE] Print statistics.
\end{itemize}


% \item[\label{scipusercutcall}\hypertarget{scipusercutcall}
% {\textbf{gams/usercutcall (\slshape{string})}}]\hspace{1.0in}
% 
% The \GAMS command line (minus the gams executable name) to call the cut generator.
% 
% 
% \item[\label{scipusercutfirst}\hypertarget{scipusercutfirst}
% {\textbf{gams/usercutfirst (\slshape{integer})}}]\hspace{1.0in}
% 
% Calls the cut generator for the first $n$ nodes.
% 
% \textsl{(default = 10)}
% 
% \item[\label{scipusercutfreq}\hypertarget{scipusercutfreq}
% {\textbf{gams/usercutfreq (\slshape{integer})}}]\hspace{1.0in}
% 
% Determines the frequency of the cut generator model calls.
% 
% \textsl{(default = 10)}
% 
% \item[\label{scipusercutinterval}\hypertarget{scipusercutinterval}
% {\textbf{gams/usercutinterval (\slshape{integer})}}]\hspace{1.0in}
% 
% Determines the interval when to apply the multiplier for the frequency of the cut generator model calls.
% See gams/userheurinterval for details.
% 
% \textsl{(default = 100)}
% 
% \item[\label{scipusercutmult}\hypertarget{scipusercutmult}
% {\textbf{gams/usercutmult (\slshape{integer})}}]\hspace{1.0in}
% 
% Determines the multiplier for the frequency of the cut generator model calls.
% 
% \textsl{(default = 2)}
% 
% \item[\label{scipusercutnewint}\hypertarget{scipusercutnewint}
% {\textbf{gams/usercutnewint (\slshape{integer})}}]\hspace{1.0in}
% 
% Calls the cut generator if the solver found a new integer feasible solution.
% 
% \textsl{(default = TRUE)}
% \begin{itemize}
% \item[FALSE] Do not call cut generator because a new integer feasible solution is found.
% \item[TRUE] Let \SCIP call the cut generator if a new integer feasible solution is found.
% \end{itemize}
% 
% \item[\label{scipusergdxin}\hypertarget{scipusergdxin}
% {\textbf{gams/usergdxin (\slshape{string})}}]\hspace{1.0in}
% 
% The name of the GDX file read back into \SCIP.
% 
% \textsl{(default =} \verb=bchin.gdx=)
% 
% \item[\label{scipusergdxname}\hypertarget{scipusergdxname}
% {\textbf{gams/usergdxname (\slshape{string})}}]\hspace{1.0in}
% 
% The name of the GDX file exported from the solver with the solution at the node.
% 
% \textsl{(default =} \verb=bchout.gdx=)
% 
% \item[\label{scipusergdxnameinc}\hypertarget{scipusergdxnameinc}
% {\textbf{gams/usergdxnameinc (\slshape{string})}}]\hspace{1.0in}
% 
% The name of the GDX file exported from the solver with the incumbent solution.
% 
% \textsl{(default =} \verb=bchout_i.gdx=)
% 
% \item[\label{scipusergdxprefix}\hypertarget{scipusergdxprefix}
% {\textbf{gams/usergdxprefix (\slshape{string})}}]\hspace{1.0in}
% 
% Prefixes to use for gams/usergdxin, gams/usergdxname, and gams/usergdxnameinc.
% 
% 
% \item[\label{scipuserheurcall}\hypertarget{scipuserheurcall}
% {\textbf{gams/userheurcall (\slshape{string})}}]\hspace{1.0in}
% 
% The \GAMS command line (minus the gams executable name) to call the heuristic.
% 
% 
% \item[\label{scipuserheurfirst}\hypertarget{scipuserheurfirst}
% {\textbf{gams/userheurfirst (\slshape{integer})}}]\hspace{1.0in}
% 
% Calls the heuristic for the first $n$ nodes.
% 
% \textsl{(default = 10)}
% 
% \item[\label{scipuserheurfreq}\hypertarget{scipuserheurfreq}
% {\textbf{gams/userheurfreq (\slshape{integer})}}]\hspace{1.0in}
% 
% Determines the frequency of the heuristic model calls.
% 
% \textsl{(default = 10)}
% 
% \item[\label{scipuserheurinterval}\hypertarget{scipuserheurinterval}
% {\textbf{gams/userheurinterval (\slshape{integer})}}]\hspace{1.0in}
% 
% Determines the interval when to apply the multiplier for the frequency of the heuristic model calls.
% For example, for the first 100 (gams/userheurinterval) nodes, the solver call every 10th (gams/userheurfreq) node the heuristic.
% After 100 nodes, the frequency gets multiplied by 10 (gams/userheurmult), so that for the next 100 node the solver calls the heuristic every 20th node.
% For nodes 200-300, the heuristic get called every 40th node, for nodes 300-400 every 80th node and after node 400 every 100th node.
% 
% \textsl{(default = 100)}
% 
% \item[\label{scipuserheurmult}\hypertarget{scipuserheurmult}
% {\textbf{gams/userheurmult (\slshape{integer})}}]\hspace{1.0in}
% 
% Determines the multiplier for the frequency of the heuristic model calls.
% 
% \textsl{(default = 2)}
% 
% \item[\label{scipuserheurnewint}\hypertarget{scipuserheurnewint}
% {\textbf{gams/userheurnewint (\slshape{integer})}}]\hspace{1.0in}
% 
% Calls the heuristic if the solver found a new integer feasible solution.
% 
% \textsl{(default = TRUE)}
% \begin{itemize}
% \item[FALSE] Do not call heuristic because a new integer feasible solution is found.
% \item[TRUE] Let \SCIP call the heuristic if a new integer feasible solution is found.
% \end{itemize}
% 
% \item[\label{scipuserheurobjfirst}\hypertarget{scipuserheurobjfirst}
% {\textbf{gams/userheurobjfirst (\slshape{integer})}}]\hspace{1.0in}
% 
% Similar to gams/userheurfirst but only calls the heuristic if the relaxed objective value promises a significant improvement of the current incumbent, i.e., the LP value of the node has to be closer to the best bound than the current incumbent.
% 
% \textsl{(default = FALSE)}
% 
% \item[\label{scipuserjobid}\hypertarget{scipuserjobid}
% {\textbf{gams/userjobid (\slshape{string})}}]\hspace{1.0in}
% 
% Postfixes to use for gams/gdxname, gams/gdxnameinc, and gams/gdxin.
% 
% 
% \item[\label{scipuserkeep}\hypertarget{scipuserkeep}
% {\textbf{gams/userkeep (\slshape{integer})}}]\hspace{1.0in}
% 
% Calls gamskeep instead of gams
% 
% \textsl{(default = FALSE)}

\bibliographystyle{plain}
\bibliography{coin_scip}

\end{description}

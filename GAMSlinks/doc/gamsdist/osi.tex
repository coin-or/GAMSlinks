\chapter{\OSICPLEX, \OSIGUROBI, \OSIMOSEK, \OSISOPLEX, \OSIXPRESS}

The ``bare bone'' solver links \GAMS/\OSICPLEX, \GAMS/\OSIGUROBI, \GAMS/\OSIMOSEK, \GAMS/\OSISOPLEX, and \GAMS/\OSIXPRESS
allow users to solve their \GAMS models with \SOPLEX or a standalone license of \CPLEX, \GUROBI, \MOSEK, or \XPRESS.
Since \SOPLEX is distributed under the ZIB Academic License, it is only available for users with a \GAMS academic license.
The links use the COIN-OR Open Solver Interface (\OSI) to communicate with these solvers.
The \OSICPLEX link has been written primarily by Tobias Achterberg,
the \OSIGUROBI link has been written primarily by Stefan Vigerske,
the \OSIMOSEK link has been written primarily by Bo Jensen,
the \OSISOPLEX link has been written primarily by Tobias Achterberg, Ambros M.\ Gleixner, and Wei Huang, and
the \OSIXPRESS link has been written primarily by John Doe.
Matthew Saltzman is the COIN-OR project leader for \OSI.

For more information we refer to the \OSI web site \texttt{https://projects.coin-or.org/Osi}.

The \OSI links support linear equations and continuous, binary, and integer variables.
Semicontinuous and Semiinteger variables, special ordered sets, branching priorities, and indicator constraints are not supported by \OSI.
\OSISOPLEX solves only LPs, no MIPs.

\section{Usage}

The following statement can be used inside your \GAMS program to specify using \OSIGUROBI
\begin{verbatim}
  Option MIP = OSIGUROBI;     { or LP or RMIP }
\end{verbatim}

The above statement should appear before the Solve statement.

The links support the general \GAMS options \texttt{reslim}, \texttt{optca}, \texttt{optcr}, \texttt{nodlim}, \texttt{iterlim}, and \texttt{threads} (except for \OSISOPLEX).
For \OSICPLEX, \OSIGUROBI, \OSIMOSEK, and \OSIXPRESS an option file in the format required by the solver can be provided via the \GAMS \texttt{optfile} option.
See Section~\ref{sub:osioptions} for details.

If a MIP is solved via one of the \OSI links, only primal solution values are reported by default.
To receive also the dual values for the LP that is obtained from the MIP by fixing all discrete variables, the \GAMS option \texttt{integer1} must be set to a nonzero value. Note that this may lead to solving another LP after the MIP solve has finished.

Setting the \GAMS option \texttt{integer2} to a nonzero value makes variable and equation names available to the solver.
This option may be useful for debugging purposes.

Setting the \GAMS option \texttt{integer3} to a nonzero value leads to writing the model instance to a file in LP or MPS format before starting the solution process (\texttt{integer3=1} writes an MPS file, \texttt{integer3=2} writes an LP files, \texttt{integer3=4} writes a native MPS file; sum these values to write several files).
The name of the MPS file is chosen to be the name of the \GAMS model file with the extension \texttt{.gms} replaced by \texttt{.mps}.
This option may be useful for debugging purposes.

For \OSICPLEX, \OSIGUROBI, and \OSIXPRESS, setting the \GAMS option \texttt{integer4} to a nonzero value leads to passing the variable level values (\texttt{.l} suffix) to the MIP solver as initial solution. This is analog to the \texttt{mipstart} option of the full \CPLEX and \GUROBI links and the \texttt{loadmipsol} option of the full \XPRESS link.

\subsection{Option files}
\label{sub:osioptions}

\subsubsection{\OSICPLEX Options}
In an \OSICPLEX option file, each line lists one option setting, where the option name and value are separated by space.\\
Example:
\begin{verbatim}
CPX_PARAM_MIPEMPHASIS            2
CPX_PARAM_HEURFREQ               42
CPX_PARAM_MIPDISPLAY             4
\end{verbatim}

\subsubsection{\OSIGUROBI Options}
In an \OSIGUROBI option file, each line lists one option setting, where the option name and value are separated by space.

Example:
\begin{verbatim}
Cuts 2
Heuristics 0.1
\end{verbatim}

\subsubsection{\OSIMOSEK Options}
An \OSIMOSEK option file begins with the line \texttt{BEGIN MOSEK} and terminates with \texttt{END MOSEK}.
Comments are introduced with an '\%', empty lines are ignored.
Each other line starts with a MOSEK parameter value, followed by space, and a value for that parameter.

Example:
\begin{verbatim}
BEGIN MOSEK
% disable probing and solve the root node by the interior point solver
MSK_IPAR_MIO_PRESOLVE_PROBING MSK_OFF
MSK_IPAR_MIO_ROOT_OPTIMIZER   MSK_OPTIMIZER_INTPNT
END MOSEK
\end{verbatim}

\subsubsection{\OSIXPRESS Options}
In an \OSIXPRESS option file, each line lists one option setting, where the option name and value are separated by an equal sign.

Example:
\begin{verbatim}
MIPLOG = 3
HEURFREQ = 2
\end{verbatim}

%\bibliographystyle{plain}
%\bibliography{coinlibd}
%\renewcommand{\bibname}{Osi References}
%\chapter{\OSICPLEX, \OSIGUROBI, \OSIMOSEK, \OSISOPLEX, \OSIXPRESS}

The ``bare bone'' solver links \GAMS/\OSICPLEX, \GAMS/\OSIGUROBI, \GAMS/\OSIMOSEK, \GAMS/\OSISOPLEX, and \GAMS/\OSIXPRESS
allow users to solve their \GAMS models with \SOPLEX or a standalone license of \CPLEX, \GUROBI, \MOSEK, or \XPRESS.
Since \SOPLEX is distributed under the ZIB Academic License, it is only available for users with a \GAMS academic license.
The links use the COIN-OR Open Solver Interface (\OSI) to communicate with these solvers.
The \OSICPLEX link has been written primarily by Tobias Achterberg,
the \OSIGUROBI link has been written primarily by Stefan Vigerske,
the \OSIMOSEK link has been written primarily by Bo Jensen,
the \OSISOPLEX link has been written primarily by Tobias Achterberg, Ambros M.\ Gleixner, and Wei Huang, and
the \OSIXPRESS link has been written primarily by John Doe.
Matthew Saltzman is the COIN-OR project leader for \OSI.

For more information we refer to the \OSI web site \texttt{https://projects.coin-or.org/Osi}.

The \OSI links support linear equations and continuous, binary, and integer variables.
Semicontinuous and Semiinteger variables, special ordered sets, branching priorities, and indicator constraints are not supported by \OSI.
\OSISOPLEX solves only LPs, no MIPs.

\section{Usage}

The following statement can be used inside your \GAMS program to specify using \OSIGUROBI
\begin{verbatim}
  Option MIP = OSIGUROBI;     { or LP or RMIP }
\end{verbatim}

The above statement should appear before the Solve statement.

The links support the general \GAMS options \texttt{reslim}, \texttt{optca}, \texttt{optcr}, \texttt{nodlim}, \texttt{iterlim}, and \texttt{threads} (except for \OSISOPLEX).
For \OSICPLEX, \OSIGUROBI, \OSIMOSEK, and \OSIXPRESS an option file in the format required by the solver can be provided via the \GAMS \texttt{optfile} option.
See Section~\ref{sub:osioptions} for details.

If a MIP is solved via one of the \OSI links, only primal solution values are reported by default.
To receive also the dual values for the LP that is obtained from the MIP by fixing all discrete variables, the \GAMS option \texttt{integer1} must be set to a nonzero value. Note that this may lead to solving another LP after the MIP solve has finished.

Setting the \GAMS option \texttt{integer2} to a nonzero value makes variable and equation names available to the solver.
This option may be useful for debugging purposes.

Setting the \GAMS option \texttt{integer3} to a nonzero value leads to writing the model instance to a file in LP or MPS format before starting the solution process (\texttt{integer3=1} writes an MPS file, \texttt{integer3=2} writes an LP files, \texttt{integer3=4} writes a native MPS file; sum these values to write several files).
The name of the MPS file is chosen to be the name of the \GAMS model file with the extension \texttt{.gms} replaced by \texttt{.mps}.
This option may be useful for debugging purposes.

For \OSICPLEX, \OSIGUROBI, and \OSIXPRESS, setting the \GAMS option \texttt{integer4} to a nonzero value leads to passing the variable level values (\texttt{.l} suffix) to the MIP solver as initial solution. This is analog to the \texttt{mipstart} option of the full \CPLEX and \GUROBI links and the \texttt{loadmipsol} option of the full \XPRESS link.

\subsection{Option files}
\label{sub:osioptions}

\subsubsection{\OSICPLEX Options}
In an \OSICPLEX option file, each line lists one option setting, where the option name and value are separated by space.\\
Example:
\begin{verbatim}
CPX_PARAM_MIPEMPHASIS            2
CPX_PARAM_HEURFREQ               42
CPX_PARAM_MIPDISPLAY             4
\end{verbatim}

\subsubsection{\OSIGUROBI Options}
In an \OSIGUROBI option file, each line lists one option setting, where the option name and value are separated by space.

Example:
\begin{verbatim}
Cuts 2
Heuristics 0.1
\end{verbatim}

\subsubsection{\OSIMOSEK Options}
An \OSIMOSEK option file begins with the line \texttt{BEGIN MOSEK} and terminates with \texttt{END MOSEK}.
Comments are introduced with an '\%', empty lines are ignored.
Each other line starts with a MOSEK parameter value, followed by space, and a value for that parameter.

Example:
\begin{verbatim}
BEGIN MOSEK
% disable probing and solve the root node by the interior point solver
MSK_IPAR_MIO_PRESOLVE_PROBING MSK_OFF
MSK_IPAR_MIO_ROOT_OPTIMIZER   MSK_OPTIMIZER_INTPNT
END MOSEK
\end{verbatim}

\subsubsection{\OSIXPRESS Options}
In an \OSIXPRESS option file, each line lists one option setting, where the option name and value are separated by an equal sign.

Example:
\begin{verbatim}
MIPLOG = 3
HEURFREQ = 2
\end{verbatim}

%\bibliographystyle{plain}
%\bibliography{coinlibd}
%\renewcommand{\bibname}{Osi References}
%\chapter{\OSICPLEX, \OSIGUROBI, \OSIMOSEK, \OSISOPLEX, \OSIXPRESS}

The ``bare bone'' solver links \GAMS/\OSICPLEX, \GAMS/\OSIGUROBI, \GAMS/\OSIMOSEK, \GAMS/\OSISOPLEX, and \GAMS/\OSIXPRESS
allow users to solve their \GAMS models with \SOPLEX or a standalone license of \CPLEX, \GUROBI, \MOSEK, or \XPRESS.
Since \SOPLEX is distributed under the ZIB Academic License, it is only available for users with a \GAMS academic license.
The links use the COIN-OR Open Solver Interface (\OSI) to communicate with these solvers.
The \OSICPLEX link has been written primarily by Tobias Achterberg,
the \OSIGUROBI link has been written primarily by Stefan Vigerske,
the \OSIMOSEK link has been written primarily by Bo Jensen,
the \OSISOPLEX link has been written primarily by Tobias Achterberg, Ambros M.\ Gleixner, and Wei Huang, and
the \OSIXPRESS link has been written primarily by John Doe.
Matthew Saltzman is the COIN-OR project leader for \OSI.

For more information we refer to the \OSI web site \texttt{https://projects.coin-or.org/Osi}.

The \OSI links support linear equations and continuous, binary, and integer variables.
Semicontinuous and Semiinteger variables, special ordered sets, branching priorities, and indicator constraints are not supported by \OSI.
\OSISOPLEX solves only LPs, no MIPs.

\section{Usage}

The following statement can be used inside your \GAMS program to specify using \OSIGUROBI
\begin{verbatim}
  Option MIP = OSIGUROBI;     { or LP or RMIP }
\end{verbatim}

The above statement should appear before the Solve statement.

The links support the general \GAMS options \texttt{reslim}, \texttt{optca}, \texttt{optcr}, \texttt{nodlim}, \texttt{iterlim}, and \texttt{threads} (except for \OSISOPLEX).
For \OSICPLEX, \OSIGUROBI, \OSIMOSEK, and \OSIXPRESS an option file in the format required by the solver can be provided via the \GAMS \texttt{optfile} option.
See Section~\ref{sub:osioptions} for details.

If a MIP is solved via one of the \OSI links, only primal solution values are reported by default.
To receive also the dual values for the LP that is obtained from the MIP by fixing all discrete variables, the \GAMS option \texttt{integer1} must be set to a nonzero value. Note that this may lead to solving another LP after the MIP solve has finished.

Setting the \GAMS option \texttt{integer2} to a nonzero value makes variable and equation names available to the solver.
This option may be useful for debugging purposes.

Setting the \GAMS option \texttt{integer3} to a nonzero value leads to writing the model instance to a file in LP or MPS format before starting the solution process (\texttt{integer3=1} writes an MPS file, \texttt{integer3=2} writes an LP files, \texttt{integer3=4} writes a native MPS file; sum these values to write several files).
The name of the MPS file is chosen to be the name of the \GAMS model file with the extension \texttt{.gms} replaced by \texttt{.mps}.
This option may be useful for debugging purposes.

For \OSICPLEX, \OSIGUROBI, and \OSIXPRESS, setting the \GAMS option \texttt{integer4} to a nonzero value leads to passing the variable level values (\texttt{.l} suffix) to the MIP solver as initial solution. This is analog to the \texttt{mipstart} option of the full \CPLEX and \GUROBI links and the \texttt{loadmipsol} option of the full \XPRESS link.

\subsection{Option files}
\label{sub:osioptions}

\subsubsection{\OSICPLEX Options}
In an \OSICPLEX option file, each line lists one option setting, where the option name and value are separated by space.\\
Example:
\begin{verbatim}
CPX_PARAM_MIPEMPHASIS            2
CPX_PARAM_HEURFREQ               42
CPX_PARAM_MIPDISPLAY             4
\end{verbatim}

\subsubsection{\OSIGUROBI Options}
In an \OSIGUROBI option file, each line lists one option setting, where the option name and value are separated by space.

Example:
\begin{verbatim}
Cuts 2
Heuristics 0.1
\end{verbatim}

\subsubsection{\OSIMOSEK Options}
An \OSIMOSEK option file begins with the line \texttt{BEGIN MOSEK} and terminates with \texttt{END MOSEK}.
Comments are introduced with an '\%', empty lines are ignored.
Each other line starts with a MOSEK parameter value, followed by space, and a value for that parameter.

Example:
\begin{verbatim}
BEGIN MOSEK
% disable probing and solve the root node by the interior point solver
MSK_IPAR_MIO_PRESOLVE_PROBING MSK_OFF
MSK_IPAR_MIO_ROOT_OPTIMIZER   MSK_OPTIMIZER_INTPNT
END MOSEK
\end{verbatim}

\subsubsection{\OSIXPRESS Options}
In an \OSIXPRESS option file, each line lists one option setting, where the option name and value are separated by an equal sign.

Example:
\begin{verbatim}
MIPLOG = 3
HEURFREQ = 2
\end{verbatim}

%\bibliographystyle{plain}
%\bibliography{coinlibd}
%\renewcommand{\bibname}{Osi References}
%\chapter{\OSICPLEX, \OSIGUROBI, \OSIMOSEK, \OSISOPLEX, \OSIXPRESS}

The ``bare bone'' solver links \GAMS/\OSICPLEX, \GAMS/\OSIGUROBI, \GAMS/\OSIMOSEK, \GAMS/\OSISOPLEX, and \GAMS/\OSIXPRESS
allow users to solve their \GAMS models with \SOPLEX or a standalone license of \CPLEX, \GUROBI, \MOSEK, or \XPRESS.
Since \SOPLEX is distributed under the ZIB Academic License, it is only available for users with a \GAMS academic license.
The links use the COIN-OR Open Solver Interface (\OSI) to communicate with these solvers.
The \OSICPLEX link has been written primarily by Tobias Achterberg,
the \OSIGUROBI link has been written primarily by Stefan Vigerske,
the \OSIMOSEK link has been written primarily by Bo Jensen,
the \OSISOPLEX link has been written primarily by Tobias Achterberg, Ambros M.\ Gleixner, and Wei Huang, and
the \OSIXPRESS link has been written primarily by John Doe.
Matthew Saltzman is the COIN-OR project leader for \OSI.

For more information we refer to the \OSI web site \texttt{https://projects.coin-or.org/Osi}.

The \OSI links support linear equations and continuous, binary, and integer variables.
Semicontinuous and Semiinteger variables, special ordered sets, branching priorities, and indicator constraints are not supported by \OSI.
\OSISOPLEX solves only LPs, no MIPs.

\section{Usage}

The following statement can be used inside your \GAMS program to specify using \OSIGUROBI
\begin{verbatim}
  Option MIP = OSIGUROBI;     { or LP or RMIP }
\end{verbatim}

The above statement should appear before the Solve statement.

The links support the general \GAMS options \texttt{reslim}, \texttt{optca}, \texttt{optcr}, \texttt{nodlim}, \texttt{iterlim}, and \texttt{threads} (except for \OSISOPLEX).
For \OSICPLEX, \OSIGUROBI, \OSIMOSEK, and \OSIXPRESS an option file in the format required by the solver can be provided via the \GAMS \texttt{optfile} option.
See Section~\ref{sub:osioptions} for details.

If a MIP is solved via one of the \OSI links, only primal solution values are reported by default.
To receive also the dual values for the LP that is obtained from the MIP by fixing all discrete variables, the \GAMS option \texttt{integer1} must be set to a nonzero value. Note that this may lead to solving another LP after the MIP solve has finished.

Setting the \GAMS option \texttt{integer2} to a nonzero value makes variable and equation names available to the solver.
This option may be useful for debugging purposes.

Setting the \GAMS option \texttt{integer3} to a nonzero value leads to writing the model instance to a file in LP or MPS format before starting the solution process (\texttt{integer3=1} writes an MPS file, \texttt{integer3=2} writes an LP files, \texttt{integer3=4} writes a native MPS file; sum these values to write several files).
The name of the MPS file is chosen to be the name of the \GAMS model file with the extension \texttt{.gms} replaced by \texttt{.mps}.
This option may be useful for debugging purposes.

For \OSICPLEX, \OSIGUROBI, and \OSIXPRESS, setting the \GAMS option \texttt{integer4} to a nonzero value leads to passing the variable level values (\texttt{.l} suffix) to the MIP solver as initial solution. This is analog to the \texttt{mipstart} option of the full \CPLEX and \GUROBI links and the \texttt{loadmipsol} option of the full \XPRESS link.

\subsection{Option files}
\label{sub:osioptions}

\subsubsection{\OSICPLEX Options}
In an \OSICPLEX option file, each line lists one option setting, where the option name and value are separated by space.\\
Example:
\begin{verbatim}
CPX_PARAM_MIPEMPHASIS            2
CPX_PARAM_HEURFREQ               42
CPX_PARAM_MIPDISPLAY             4
\end{verbatim}

\subsubsection{\OSIGUROBI Options}
In an \OSIGUROBI option file, each line lists one option setting, where the option name and value are separated by space.

Example:
\begin{verbatim}
Cuts 2
Heuristics 0.1
\end{verbatim}

\subsubsection{\OSIMOSEK Options}
An \OSIMOSEK option file begins with the line \texttt{BEGIN MOSEK} and terminates with \texttt{END MOSEK}.
Comments are introduced with an '\%', empty lines are ignored.
Each other line starts with a MOSEK parameter value, followed by space, and a value for that parameter.

Example:
\begin{verbatim}
BEGIN MOSEK
% disable probing and solve the root node by the interior point solver
MSK_IPAR_MIO_PRESOLVE_PROBING MSK_OFF
MSK_IPAR_MIO_ROOT_OPTIMIZER   MSK_OPTIMIZER_INTPNT
END MOSEK
\end{verbatim}

\subsubsection{\OSIXPRESS Options}
In an \OSIXPRESS option file, each line lists one option setting, where the option name and value are separated by an equal sign.

Example:
\begin{verbatim}
MIPLOG = 3
HEURFREQ = 2
\end{verbatim}

%\bibliographystyle{plain}
%\bibliography{coinlibd}
%\renewcommand{\bibname}{Osi References}
%\input{osi.bbl}

\chapterend


\chapterend


\chapterend


\chapterend

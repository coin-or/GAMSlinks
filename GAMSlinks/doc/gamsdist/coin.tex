%%%
%%% tabular with the option-table header
%%%
\renewenvironment{option_tabular}%
{\begin{tabular}{p{.16\textwidth}p{.65\textwidth}p{.11\textwidth}}
\hline
\textbf{Option}      &
\textbf{Description} &
\textbf{Default}     \\
\hline}
{\end{tabular}}

%%%
%%% list to use inside a tabular cell
%%%
\renewenvironment{tab_list}[1]%
{\begin{list}{}{\vspace*{-1.5ex}\renewcommand{\makelabel}{\desclabel}\parsep-0.15cm\labelwidth#1\leftmargin#1\setlength{\labelsep}{\itemindent}\topsep0cm\parskip0cm\partopsep0cm}}%
{\end{list}}


\chapter{COIN-OR}
\textbf{Stefan Vigerske, Humboldt University Berlin, Germany}
\vspace{1cm}

\minitoc


\section{Introduction}

COIN-OR (\textbf{CO}mputational \textbf{IN}frastructure for \textbf{O}perations \textbf{R}esearch, \texttt{http://www.coin-or.org}) is an initiative to spur the development of open-source software for the operations research community.
One of the projects hosted at COIN-OR is the GAMSlinks project (\texttt{https://projects.coin-or.org/GAMSlinks}).
It is dedicated to the development of interfaces between GAMS and open source solvers.
Some of these links and solvers have also found their way into the regular GAMS distribution.
They are currently available for Linux (32 and 64 bit), Windows (32 and 64 bit), Sun Solaris (Intel 64 bit), and Darwin (Intel 32 bit) systems.
With the availability of source code for the GAMSlinks the user is not limited to the out of the box solvers that come with a regular GAMS distribution, but can extend and build these interfaces by themselves.

As of the upcoming GAMS Distribution, available solvers include:
\begin{itemize}
\item CoinBonmin 0.99: Basic Open-source Nonlinear Mixed Integer programming\\
(model types: LP, RMIP, MIP, DNLP, NLP, RMINLP, MINLP, QCP, RMIQCP, MIQCP)
\item CoinCbc 2.2: COIN-OR Branch and Cut\\
(model types: LP, MIP, RMIP)
\item CoinGlpk 4.31: Gnu Linear Programming Kit\\
(model types: LP, MIP, RMIP)
\item CoinIpopt 3.5: Interior Point Optimizer\\
(model types: LP, RMIP, DNLP, NLP, RMINLP, QCP, RMIQCP)
\item CoinScip 1.00: Solving Constrained Integer Programs\\
(model types: LP, MIP, RMIP)
\end{itemize}

% The open source concept opens new possibilities for our advanced users:
% 
% \begin{itemize}
% \item Make modifications to the existing links
% \item Port the links to another platform supported by GAMS
% \item Connect new solvers to GAMS/COIN-OR
% \item Extend the documentation
% \end{itemize}

For more information see the COIN-OR/GAMSlinks web site at
\texttt{https://projects.coin-or.org/GAMSlinks}.

\section{CoinBonmin}

GAMS/CoinBonmin brings the open source MINLP solver Bonmin from the COIN-OR foundation to the broad audience of GAMS users.

Bonmin (\textbf{B}asic \textbf{O}pen-source \textbf{N}onlinear \textbf{M}ixed \textbf{In}teger programming) 0.99 is an open-source solver for mixed-integer nonlinear programming (MINLPs), whereof some parts are still experimental.
The code is developed in a joined project of IBM and the Carnegie Mellon University.
The COIN-OR project leader for Bonmin is Pierre Bonami.

Bonmin implements five different algorithms:
\begin{itemize}
\item {B-BB}: a simple branch-and-bound algorithm based on solving a continuous nonlinear program at each node of the search tree and branching on variables (default)
\item {B-OA}: an outer-approximation based decomposition algorithm
\item {B-QG}: an outer-approximation based branch-and-cut algorithm (by Queseda and Grossmann)
\item {B-Hyb}: a hybrid outer-approximation/nonlinear programming based branch-and-cut algorithm
\item {B-Ecp}: an ECP cuts based branch-and-cut algorithm a la FilMINT
\end{itemize}
The algorithms are exact when the problem is convex, otherwise they are heuristics.

For convex MINLPs, experiments on a reasonably large test set of problems have shown that B-Hyp is the algorithm of choice (it solved most of the problems in 3 hours of computing time).
Nevertheless, there are cases where B-OA is much faster than B-Hyb and others where B-BB is interesting.
B-QG corresponds mainly to a specific parameter setting of B-Hyb where some features are disabled.
B-Ecp has just been added to Bonmin and is still experimental.
For nonconvex MINLPs, it is strongly recommend to use B-BB, even though it is still a heuristic for such problems too.
% Therefore, it is the default algorithm in Bonmin.

For more information we refer to
\begin{itemize}
\item the Bonmin web site \texttt{https://projects.coin-or.org/Bonmin} and
\item the paper P. Bonami, L.T. Biegler, A.R. Conn, G. Cornuejols, I.E. Grossmann, C.D. Laird, J. Lee, A. Lodi, F. Margot, N.Sawaya and A. W\"achter, An Algorithmic Framework for Convex Mixed Integer Nonlinear Programs, \emph{IBM Research Report} RC23771, Oct. 2005.
\end{itemize}
Most of the Bonmin documentation in the section is taken from the Bonmin manual available on the Bonmin web site.

\subsection{Model requirements}

Bonmin can handle mixed-integer nonlinear programming models which functions can be nonconvex, but should be twice continuously differentiable.
The Bonmin link in GAMS supports continuous, binary, and integer variables, special ordered sets, branching priorities, but no semi-continuous or semi-integer variables (see chapter 17.1 of the GAMS User's Guide).

If GAMS/CoinBonmin is called for a model with only continuous variables, the interface directly calls Ipopt.

\subsection{Usage of CoinBonmin}

The following statement can be used inside your GAMS program to specify using CoinBonmin
\begin{verbatim}
  Option MINLP = CoinBonmin;     { or LP, RMIP, MIP, DNLP, NLP, RMINLP, QCP, RMIQCP, MIQCP }
\end{verbatim}

The above statement should appear before the Solve statement.
If CoinBonmin was specified as the default solver during GAMS installation, the above statement is not necessary.

GAMS/CoinBonmin now support the GAMS Branch-and-Cut-and-Heuristic (BCH) Facility.
The GAMS BCH facility automates all major steps necessary to define, execute, and control the use of user defined routines within the framework of general purpose MIP and MINLP codes.
Currently supported are user defined cut generators and heuristics, where cut generator cannot be used in Bonmins pure B\&B algorithm (B-BB).
Please see the BCH documentation at \texttt{http://www.gams.com/docs/bch.htm} for further information.

\subsection{Specification of CoinBonmin Options}
\label{sub:bonminoptionspec}

A Bonmin option file contains both Ipopt and Bonmin options, for clarity all Bonmin options should be preceded with the prefix ``\texttt{bonmin.}'', except those corresponding to the BCH facility.
%Bonmin has many options that can be adjusted for the algorithm (see Section \ref{sub:bonminoptions}).
The scheme to name option files is the same as for all other GAMS solvers.
Specifying \texttt{optfile=1} let Gams/CoinBonmin read \texttt{coinbonmin.opt}, \texttt{optfile=2} corresponds to \texttt{coinbonmin.op2}, and so on.
The format of the option file is the same as for Ipopt (see Section \ref{sub:ipoptoptionspec}).

The most important option in Bonmin is the choice of the solution algorithm.
This can be set by using the option named \texttt{bonmin.algorithm} which can be set to \texttt{B-BB}, \texttt{B-OA}, \texttt{B-QG}, \texttt{B-Hyb}, or \texttt{B-Ecp} (its default value is \texttt{B-BB}).
Depending on the value of this option, certain other options may be available or not.

In the context of outer approximation decomposition, several options are available for configuring the MIP subsolver CBC.
By setting the option \texttt{bonmin.milp\_subsolver} to \texttt{Cbc\_Par}, a version of CBC is chosen that can be parameterized by the user.
The options that can be set are the node-selection strategy, the number of strong-branching candidates, the number of branches before pseudo costs are to be trusted, and the frequency of some cut generators.
Their name coincide with Bonmins options, except that the ``\texttt{bonmin.}'' prefix is replaced with  ``\texttt{milp\_sub.}''.

An example of a \texttt{bonmin.opt} file is the following:
\begin{verbatim}
   bonmin.algorithm       B-Hyb
   bonmin.oa_log_level    4
   bonmin.milp_subsolver  Cbc_Par
   milp_sub.cover_cuts    0
   print_level            6
   userheurcall           "bchheur.gms reslim 10"
\end{verbatim}
This sets the algorithm to be used to the hybrid algorithm, the level of outer approximation related output to $4$, the MIP subsolver for outer approximation to a parameterized version of CBC, switches off cover cutting planes for the MIP subsolver, sets the print level for \texttt{Ipopt} to $6$, and let Bonmin call a user defined heuristic specified in the model \texttt{bchheur.gms} with a timelimit of 10 seconds.

GAMS/CoinBonmin understands currently the following GAMS parameters: reslim (time limit), iterlim (iteration limit), nodlim (node limit), cutoff, optca (absolute gap tolerance), and optcr (relative gap tolerance).
You can set them either on the command line, e.g. \verb+nodlim=1000+, or inside your GAMS program, e.g. \verb+Option nodlim=1000;+.

\subsection{Description of CoinBonmin Options}
\label{sub:bonminoptions}

The following tables gives the list of options together with their types, default values and availability in each of the four main algorithms.
The column labeled `type' indicates the type of the parameter (`F' stands for float, `I' for integer, and `S' for
string).
The column labeled `default' indicates the global default value.
Then for each of the four algorithm \texttt{B-BB}, \texttt{B-OA}, \texttt{B-QG}, and \texttt{B-Hyb}, `$+$' indicates that the option is available for that particular algorithm while `$-$' indicates that it is not.
Options that are marked with $^*$ are those that can be used to configure the MIP subsolver.

\begin{center}
\begin{tabular}{|l|r|r|r|r|r|r|}\hline
Option & type &  default & {\tt B-BB} & {\tt B-OA} & {\tt B-QG} & {\tt B-Hyb} \\
\hline
\hline
\multicolumn{1}{|c}{} & \multicolumn{6}{l|}{Algorithm choice}\\
\hline
algorithm& S& B-BB& +& +& +& +\\
\hline
\multicolumn{1}{|c}{} & \multicolumn{6}{l|}{Output}\\
\hline
bb\_log\_interval& I& 100& +&--& +& +\\
bb\_log\_level& I& 1& +&--& +& +\\
lp\_log\_level& I& 0&--&--& +& +\\
milp\_log\_level& I& 0&--& +&--& +\\
nlp\_log\_level& I& 1& +& +& +& +\\
oa\_cuts\_log\_level& I& 0&--& +& +& +\\
oa\_log\_frequency& F& 100& +& +& +& +\\
oa\_log\_level& I& 1& +& +& +& +\\
\hline
\multicolumn{1}{|c}{} & \multicolumn{6}{l|}{Branch-and-bound options}\\
\hline
allowable\_fraction\_gap& F& optcr ($0.1$)& +& +& +& +\\
allowable\_gap& F& optca ($0$)& +& +& +& +\\
cutoff& F& cutoff or $10^{100}$& +& +& +& +\\
cutoff\_decr& F& $10^{-5}$& +& +& +& +\\
integer\_tolerance& F& $10^{-6}$& +& +& +& +\\
iteration\_limit& I& $\infty$& +& +& +& +\\
nlp\_failure\_behavior& S& stop& +& +& +& +\\
node\_comparison& S& dynamic& +& +& +& +\\
node\_limit& I& nodlim or iterlim ($10000$)& +& +& +& +\\
num\_cut\_passes& I& 1&--& +& +& +\\
num\_cut\_passes\_at\_root& I& 20&--& +& +& +\\
number\_before\_trust& I& 8& +& +& +& +\\
number\_strong\_branch& I& 20& +& +& +& +\\
solution\_limit& I& $\infty$& +& +& +& +\\
time\_limit& F& reslim ($1000$)& +& +& +& +\\
tree\_search\_strategy& S& topnode& +& +& +& +\\
variable\_selection& S& strongbranching& +& +& +& +\\
\hline
\multicolumn{1}{|c}{} & \multicolumn{6}{l|}{Diving options}\\
\hline
max\_backtracks\_in\_dive& I& 5& +& +& +& +\\
max\_dive\_depth& I& $\infty$& +& +& +& +\\
stop\_diving\_on\_cutoff& S& no& +& +& +& +\\
\hline
\end{tabular}

\begin{tabular}{|l|r|r|r|r|r|r|}\hline
Option & type &  default & {\tt B-BB} & {\tt B-OA} & {\tt B-QG} & {\tt B-Hyb} \\
\hline
\hline
\multicolumn{1}{|c}{} & \multicolumn{6}{l|}{ECP based strong branching}\\
\hline
ecp\_abs\_tol\_strong& F& $10^{-6}$& +& +& +& +\\
ecp\_max\_rounds\_strong& I& 0& +& +& +& +\\
ecp\_rel\_tol\_strong& F& 0.1& +& +& +& +\\
lp\_strong\_warmstart\_method& S& Basis& +& +& +& +\\
\hline
\multicolumn{1}{|c}{} & \multicolumn{6}{l|}{ECP cuts generation}\\
\hline
ecp\_abs\_tol& F& $10^{-6}$&--&--&--& +\\
ecp\_max\_rounds& I& 5&--&--&--& +\\
ecp\_propability\_factor& F& 1000&--&--&--& +\\
ecp\_rel\_tol& F& 0&--&--&--& +\\
filmint\_ecp\_cuts& I& 0&--&--&--& +\\
\hline
\multicolumn{1}{|c}{} & \multicolumn{6}{l|}{GAMS Branch Cut and Heuristic Facility}\\
\hline
usercutcall& S& none& --& +& +& +\\
usercutfirst& I& 10& --& +& +& +\\
usercutfreq& I& 10& --& +& +& +\\
usercutinterval& I& 100& --& +& +& +\\
usercutmult& I& 2& --& +& +& +\\
usercutnewint& I& 0& --& +& +& +\\
usergdxin& S& bchin.gdx& +& +& +& +\\
usergdxname& S& bchout.gdx& +& +& +& +\\
usergdxnameinc& S& bchout\_i.gdx& +& +& +& +\\
usergdxprefix& S&none & +& +& +& +\\
userheurcall& S& none& +& +& +& +\\
userheurfirst& I& 10& +& +& +& +\\
userheurfreq& I& 10& +& +& +& +\\
userheurinterval& I& 100& +& +& +& +\\
userheurmult& I& 2& +& +& +& +\\
userheurnewint& I& 0& +& +& +& +\\
userheurobjfirst& I& 0& +& +& +& +\\
userjobid& S& none& +& +& +& +\\
userkeep& I& 0& +& +& +& +\\
\hline
\multicolumn{1}{|c}{} & \multicolumn{6}{l|}{MILP cutting planes in hybrid algorithm}\\
\hline
2mir\_cuts& I& 0&--& +&--& +\\
Gomory\_cuts& I& \-5&--& +&--& +\\
clique\_cuts& I& \-5&--& +&--& +\\
cover\_cuts& I& \-5&--& +&--& +\\
flow\_covers\_cuts& I& \-5&--& +&--& +\\
lift\_and\_project\_cuts& I& 0&--& +&--& +\\
mir\_cuts& I& \-5&--& +&--& +\\
probing\_cuts& I& \-5&--& +&--& +\\
reduce\_and\_split\_cuts& I& 0&--& +&--& +\\
\hline
\multicolumn{1}{|c}{} & \multicolumn{6}{l|}{NLP interface}\\
\hline
warm\_start& S& none& +&--&--&--\\
\hline
\multicolumn{1}{|c}{} & \multicolumn{6}{l|}{NLP solution robustness}\\
\hline
max\_consecutive\_failures& I& 10& +& +& +& +\\
max\_random\_point\_radius& F& 100000& +&--&--&--\\
num\_iterations\_suspect& I& \-1& +& +& +& +\\
num\_retry\_unsolved\_random\_point& I& 0& +& +& +& +\\
random\_point\_perturbation\_interval& F& 1& +&--&--&--\\
random\_point\_type& S& Jon& +&--&--&--\\
\hline
\end{tabular}

\begin{tabular}{|l|r|r|r|r|r|r|}\hline
Option & type &  default & {\tt B-BB} & {\tt B-OA} & {\tt B-QG} & {\tt B-Hyb} \\
\hline
\hline
\multicolumn{1}{|c}{} & \multicolumn{6}{l|}{NLP solves in hybrid algorithm}\\
\hline
nlp\_solve\_frequency& I& 10&--&--&--& +\\
nlp\_solve\_max\_depth& I& 10&--&--&--& +\\
nlp\_solves\_per\_depth& F& $10^{30}$&--&--&--& +\\
\hline
\multicolumn{1}{|c}{} & \multicolumn{6}{l|}{Nonconvex problems}\\
\hline
max\_consecutive\_infeasible& I& 0& +& +& +& +\\
num\_resolve\_at\_infeasibles& I& 0& +&--&--&--\\
num\_resolve\_at\_node& I& 0& +&--&--&--\\
num\_resolve\_at\_root& I& 0& +&--&--&--\\
\hline
\multicolumn{1}{|c}{} & \multicolumn{6}{l|}{Outer Approximation decomposition}\\
\hline
milp\_subsolver& S& Cbc\_D&--& +&--& +\\
oa\_dec\_time\_limit& F& 30& +& +& +& +\\
\hline
\multicolumn{1}{|c}{} & \multicolumn{6}{l|}{Outer Approximation cuts generation}\\
\hline
add\_only\_violated\_oa& S& no&--& +& +& +\\
cut\_strengthening\_type& S& none&--& +& +& +\\
disjunctive\_cut\_type& S& none&--& +& +& +\\
oa\_cuts\_scope& S& global&--& +& +& +\\
tiny\_element& F& $10^{-8}$&--& +& +& +\\
very\_tiny\_element& F& $10^{-17}$&--& +& +& +\\
\hline
\multicolumn{1}{|c}{} & \multicolumn{6}{l|}{Strong branching setup}\\
\hline
candidate\_sort\_criterion& S& best\-ps\-cost& +& +& +& +\\
maxmin\_crit\_have\_sol& F& 0.1& +& +& +& +\\
maxmin\_crit\_no\_sol& F& 0.7& +& +& +& +\\
min\_number\_strong\_branch& I& 0& +& +& +& +\\
number\_before\_trust\_list& I& 0& +& +& +& +\\
number\_look\_ahead& I& 0& +& +& +& +\\
number\_strong\_branch\_root& I& $\infty$& +& +& +& +\\
setup\_pseudo\_frac& F& 0.5& +& +& +& +\\
trust\_strong\_branching\_for\_pseudo\_cost& S& yes& +& +& +& +\\
\hline
\end{tabular}
\end{center}

\printoptioncategory{Algorithm choice}
\printoption{algorithm}%
{\ttfamily B-BB, B-OA, B-QG, B-Hyb, B-Ecp, B-iFP}%
{B-BB}%
{Choice of the algorithm.\\
This will preset some of the options of bonmin depending on the algorithm choice.}%
{\begin{list}{}{
\setlength{\parsep}{0em}
\setlength{\leftmargin}{5ex}
\setlength{\labelwidth}{2ex}
\setlength{\itemindent}{0ex}
\setlength{\topsep}{0pt}}
\item[\texttt{B-BB}] simple branch-and-bound algorithm,
\item[\texttt{B-OA}] OA Decomposition algorithm,
\item[\texttt{B-QG}] Quesada and Grossmann branch-and-cut algorithm,
\item[\texttt{B-Hyb}] hybrid outer approximation based branch-and-cut,
\item[\texttt{B-Ecp}] ecp cuts based branch-and-cut a la FilMINT.
\item[\texttt{B-iFP}] Iterated Feasibility Pump for MINLP.
\end{list}
}

\printoptioncategory{Branch-and-bound options}
\printoption{allowable\_fraction\_gap}%
{$\textrm{real}$}%
{$0.1$}%
{Specify the value of relative gap under which the algorithm stops.\\
Stop the tree search when the gap between the objective value of the best known solution and the best bound on the objective of any solution is less than this fraction of the absolute value of the best known solution value.}%
{}

\printoption{allowable\_gap}%
{$\textrm{real}$}%
{$0$}%
{Specify the value of absolute gap under which the algorithm stops.\\
Stop the tree search when the gap between the objective value of the best known solution and the best bound on the objective of any solution is less than this.}%
{}

\printoption{cutoff}%
{$-10^{ 100}\leq\textrm{real}\leq10^{ 100}$}%
{$10^{ 100}$}%
{Specify cutoff value.\\
cutoff should be the value of a feasible solution known by the user (if any). The algorithm will only look for solutions better than cutoff.}%
{}

\printoption{cutoff\_decr}%
{$-10^{ 10}\leq\textrm{real}\leq10^{ 10}$}%
{$10^{- 5}$}%
{Specify cutoff decrement.\\
Specify the amount by which cutoff is decremented below a new best upper-bound (usually a small positive value but in non-convex problems it may be a negative value).}%
{}

\printoption{enable\_dynamic\_nlp}%
{\ttfamily no, yes}%
{no}%
{Enable dynamic linear and quadratic rows addition in nlp}%
{}

\printoption{integer\_tolerance}%
{$0<\textrm{real}$}%
{$10^{- 6}$}%
{Set integer tolerance.\\
Any number within that value of an integer is considered integer.}%
{}

\printoption{iteration\_limit}%
{$0\leq\textrm{integer}$}%
{$\infty$}%
{Set the cumulated maximum number of iteration in the algorithm used to process nodes continuous relaxations in the branch-and-bound.\\
value 0 deactivates option.}%
{}

\printoption{nlp\_failure\_behavior}%
{\ttfamily stop, fathom}%
{stop}%
{Set the behavior when an NLP or a series of NLP are unsolved by Ipopt (we call unsolved an NLP for which Ipopt is not able to guarantee optimality within the specified tolerances).\\
If set to "fathom", the algorithm will fathom the node when Ipopt fails to find a solution to the nlp at that node whithin the specified tolerances. The algorithm then becomes a heuristic, and the user will be warned that the solution might not be optimal.}%
{\begin{list}{}{
\setlength{\parsep}{0em}
\setlength{\leftmargin}{5ex}
\setlength{\labelwidth}{2ex}
\setlength{\itemindent}{0ex}
\setlength{\topsep}{0pt}}
\item[\texttt{stop}] Stop when failure happens.
\item[\texttt{fathom}] Continue when failure happens.
\end{list}
}

\printoption{node\_comparison}%
{\ttfamily best-bound, depth-first, breadth-first, dynamic, best-guess}%
{best-bound}%
{Choose the node selection strategy.\\
Choose the strategy for selecting the next node to be processed.}%
{\begin{list}{}{
\setlength{\parsep}{0em}
\setlength{\leftmargin}{5ex}
\setlength{\labelwidth}{2ex}
\setlength{\itemindent}{0ex}
\setlength{\topsep}{0pt}}
\item[\texttt{best-bound}] choose node with the smallest bound,
\item[\texttt{depth-first}] Perform depth first search,
\item[\texttt{breadth-first}] Perform breadth first search,
\item[\texttt{dynamic}] Cbc dynamic strategy (starts with a depth first search and turn to best bound after 3 integer feasible solutions have been found).
\item[\texttt{best-guess}] choose node with smallest guessed integer solution
\end{list}
}

\printoption{node\_limit}%
{$0\leq\textrm{integer}$}%
{$\infty$}%
{Set the maximum number of nodes explored in the branch-and-bound search.}%
{}

\printoption{num\_cut\_passes}%
{$0\leq\textrm{integer}$}%
{$1$}%
{Set the maximum number of cut passes at regular nodes of the branch-and-cut.}%
{}

\printoption{num\_cut\_passes\_at\_root}%
{$0\leq\textrm{integer}$}%
{$20$}%
{Set the maximum number of cut passes at regular nodes of the branch-and-cut.}%
{}

\printoption{number\_before\_trust}%
{$0\leq\textrm{integer}$}%
{$8$}%
{Set the number of branches on a variable before its pseudo costs are to be believed in dynamic strong branching.\\
A value of 0 disables pseudo costs.}%
{}

\printoption{number\_strong\_branch}%
{$0\leq\textrm{integer}$}%
{$20$}%
{Choose the maximum number of variables considered for strong branching.\\
Set the number of variables on which to do strong branching.}%
{}

\printoption{solution\_limit}%
{$0\leq\textrm{integer}$}%
{$\infty$}%
{Abort after that much integer feasible solution have been found by algorithm\\
value 0 deactivates option}%
{}

\printoption{time\_limit}%
{$0\leq\textrm{real}$}%
{$1000$}%
{Set the global maximum computation time (in secs) for the algorithm.}%
{}

\printoption{tree\_search\_strategy}%
{\ttfamily top-node, dive, probed-dive, dfs-dive, dfs-dive-dynamic}%
{probed-dive}%
{Pick a strategy for traversing the tree\\
All strategies can be used in conjunction with any of the node comparison functions. Options which affect dfs-dive are max-backtracks-in-dive and max-dive-depth. The dfs-dive won't work in a non-convex problem where objective does not decrease down branches.}%
{\begin{list}{}{
\setlength{\parsep}{0em}
\setlength{\leftmargin}{5ex}
\setlength{\labelwidth}{2ex}
\setlength{\itemindent}{0ex}
\setlength{\topsep}{0pt}}
\item[\texttt{top-node}]  Always pick the top node as sorted by the node comparison function
\item[\texttt{dive}] Dive in the tree if possible, otherwise pick top node as sorted by the tree comparison function.
\item[\texttt{probed-dive}] Dive in the tree exploring two childs before continuing the dive at each level.
\item[\texttt{dfs-dive}] Dive in the tree if possible doing a depth first search. Backtrack on leaves or when a prescribed depth is attained or when estimate of best possible integer feasible solution in subtree is worst than cutoff. Once a prescribed limit of backtracks is attained pick top node as sorted by the tree comparison function
\item[\texttt{dfs-dive-dynamic}] Same as dfs-dive but once enough solution are found switch to best-bound and if too many nodes switch to depth-first.
\end{list}
}

\printoption{variable\_selection}%
{\ttfamily most-fractional, strong-branching, reliability-branching, curvature-estimator, qp-strong-branching, lp-strong-branching, nlp-strong-branching, osi-simple, osi-strong, random}%
{strong-branching}%
{Chooses variable selection strategy}%
{\begin{list}{}{
\setlength{\parsep}{0em}
\setlength{\leftmargin}{5ex}
\setlength{\labelwidth}{2ex}
\setlength{\itemindent}{0ex}
\setlength{\topsep}{0pt}}
\item[\texttt{most-fractional}] Choose most fractional variable
\item[\texttt{strong-branching}] Perform strong branching
\item[\texttt{reliability-branching}] Use reliability branching
\item[\texttt{curvature-estimator}] Use curvature estimation to select branching variable
\item[\texttt{qp-strong-branching}] Perform strong branching with QP approximation
\item[\texttt{lp-strong-branching}] Perform strong branching with LP approximation
\item[\texttt{nlp-strong-branching}] Perform strong branching with NLP approximation
\item[\texttt{osi-simple}] Osi method to do simple branching
\item[\texttt{osi-strong}] Osi method to do strong branching
\item[\texttt{random}] Method to choose branching variable randomly
\end{list}
}

\printoptioncategory{ECP cuts generation}
\printoption{ecp\_abs\_tol}%
{$0\leq\textrm{real}$}%
{$10^{- 6}$}%
{Set the absolute termination tolerance for ECP rounds.}%
{}

\printoption{ecp\_max\_rounds}%
{$0\leq\textrm{integer}$}%
{$5$}%
{Set the maximal number of rounds of ECP cuts.}%
{}

\printoption{ecp\_probability\_factor}%
{$\textrm{real}$}%
{$10$}%
{Factor appearing in formula for skipping ECP cuts.\\
Choosing -1 disables the skipping.}%
{}

\printoption{ecp\_rel\_tol}%
{$0\leq\textrm{real}$}%
{$0$}%
{Set the relative termination tolerance for ECP rounds.}%
{}

\printoption{filmint\_ecp\_cuts}%
{$0\leq\textrm{integer}$}%
{$0$}%
{Specify the frequency (in terms of nodes) at which some a la filmint ecp cuts are generated.\\
A frequency of 0 amounts to to never solve the NLP relaxation.}%
{}

\printoptioncategory{Feasibility checker using OA cuts}
\printoption{feas\_check\_cut\_types}%
{\ttfamily outer-approx, Benders}%
{outer-approx}%
{Choose the type of cuts generated when an integer feasible solution is found\\
If it seems too much memory is used should try Benders to use less}%
{\begin{list}{}{
\setlength{\parsep}{0em}
\setlength{\leftmargin}{5ex}
\setlength{\labelwidth}{2ex}
\setlength{\itemindent}{0ex}
\setlength{\topsep}{0pt}}
\item[\texttt{outer-approx}] Generate a set of Outer Approximations cuts.
\item[\texttt{Benders}] Generate a single Benders cut.
\end{list}
}

\printoption{feas\_check\_discard\_policy}%
{\ttfamily detect-cycles, keep-all, treated-as-normal}%
{detect-cycles}%
{How cuts from feasibility checker are discarded\\
Normally to avoid cycle cuts from feasibility checker should not be discarded in the node where they are generated. However Cbc sometimes does it if no care is taken which can lead to an infinite loop in Bonmin (usualy on simple problems). To avoid this one can instruct Cbc to never discard a cut but if we do that for all cuts it can lead to memory problems. The default policy here is to detect cycles and only then impose to Cbc to keep the cut. The two other alternative are to instruct Cbc to keep all cuts or to just ignore the problem and hope for the best}%
{\begin{list}{}{
\setlength{\parsep}{0em}
\setlength{\leftmargin}{5ex}
\setlength{\labelwidth}{2ex}
\setlength{\itemindent}{0ex}
\setlength{\topsep}{0pt}}
\item[\texttt{detect-cycles}] Detect if a cycle occurs and only in this case force not to discard.
\item[\texttt{keep-all}] Force cuts from feasibility checker not to be discarded (memory hungry but sometimes better).
\item[\texttt{treated-as-normal}] Cuts from memory checker can be discarded as any other cuts (code may cycle then)
\end{list}
}

\printoption{generate\_benders\_after\_so\_many\_oa}%
{$0\leq\textrm{integer}$}%
{$5000$}%
{Specify that after so many oa cuts have been generated Benders cuts should be generated instead.\\
It seems that sometimes generating too many oa cuts slows down the optimization compared to Benders due to the size of the LP. With this option we specify that after so many OA cuts have been generated we should switch to Benders cuts.}%
{}

\printoptioncategory{MILP Solver}
\printoption{cpx\_parallel\_strategy}%
{$-1\leq\textrm{integer}\leq1$}%
{$0$}%
{Strategy of parallel search mode in CPLEX.\\
-1 = opportunistic, 0 = automatic, 1 = deterministic (refer to CPLEX documentation)}%
{}

\printoption{milp\_solver}%
{Cbc\_D, Cbc\_Par, Cplex}%
{Cbc\_D}%
{Choose the subsolver to solve MILP sub-problems in OA decompositions.\\
 To use Cplex, a valid license is required.}%
{\begin{list}{}{
\setlength{\parsep}{0em}
\setlength{\leftmargin}{5ex}
\setlength{\labelwidth}{2ex}
\setlength{\itemindent}{0ex}
\setlength{\topsep}{0pt}}
\item[\texttt{Cbc\_D}] Coin Branch and Cut with its default
\item[\texttt{Cbc\_Par}] Coin Branch and Cut with passed parameters
\item[\texttt{Cplex}] IBM CPLEX
\end{list}
}

\printoption{milp\_strategy}%
{\ttfamily find\_good\_sol, solve\_to\_optimality}%
{find\_good\_sol}%
{Choose a strategy for MILPs.}%
{\begin{list}{}{
\setlength{\parsep}{0em}
\setlength{\leftmargin}{5ex}
\setlength{\labelwidth}{2ex}
\setlength{\itemindent}{0ex}
\setlength{\topsep}{0pt}}
\item[\texttt{find\_good\_sol}] Stop sub milps when a solution improving the incumbent is found
\item[\texttt{solve\_to\_optimality}] Solve MILPs to optimality
\end{list}
}

\printoption{number\_cpx\_threads}%
{$0\leq\textrm{integer}$}%
{$0$}%
{Set number of threads to use with cplex.\\
(refer to CPLEX documentation)}%
{}

\printoptioncategory{MILP cutting planes in hybrid algorithm (B-Hyb)}
\printoption{2mir\_cuts}%
{$-100\leq\textrm{integer}$}%
{$0$}%
{Frequency (in terms of nodes) for generating 2-MIR cuts in branch-and-cut\\
If k $>$ 0, cuts are generated every k nodes, if -99 $<$ k $<$ 0 cuts are generated every -k nodes but Cbc may decide to stop generating cuts, if not enough are generated at the root node, if k=-99 generate cuts only at the root node, if k=0 or 100 do not generate cuts.}%
{}

\printoption{Gomory\_cuts}%
{$-100\leq\textrm{integer}$}%
{$-5$}%
{Frequency k (in terms of nodes) for generating Gomory cuts in branch-and-cut.\\
See option \texttt{2mir\_cuts} for the meaning of k.}%
{}

\printoption{clique\_cuts}%
{$-100\leq\textrm{integer}$}%
{$-5$}%
{Frequency (in terms of nodes) for generating clique cuts in branch-and-cut\\
See option \texttt{2mir\_cuts} for the meaning of k.}%
{}

\printoption{cover\_cuts}%
{$-100\leq\textrm{integer}$}%
{$0$}%
{Frequency (in terms of nodes) for generating cover cuts in branch-and-cut\\
See option \texttt{2mir\_cuts} for the meaning of k.}%
{}

\printoption{flow\_cover\_cuts}%
{$-100\leq\textrm{integer}$}%
{$-5$}%
{Frequency (in terms of nodes) for generating flow cover cuts in branch-and-cut\\
See option \texttt{2mir\_cuts} for the meaning of k.}%
{}

\printoption{lift\_and\_project\_cuts}%
{$-100\leq\textrm{integer}$}%
{$0$}%
{Frequency (in terms of nodes) for generating lift-and-project cuts in branch-and-cut\\
See option \texttt{2mir\_cuts} for the meaning of k.}%
{}

\printoption{mir\_cuts}%
{$-100\leq\textrm{integer}$}%
{$-5$}%
{Frequency (in terms of nodes) for generating MIR cuts in branch-and-cut\\
See option \texttt{2mir\_cuts} for the meaning of k.}%
{}

\printoption{reduce\_and\_split\_cuts}%
{$-100\leq\textrm{integer}$}%
{$0$}%
{Frequency (in terms of nodes) for generating reduce-and-split cuts in branch-and-cut\\
See option \texttt{2mir\_cuts} for the meaning of k.}%
{}

\printoptioncategory{MINLP Heuristics}
\printoption{feasibility\_pump\_objective\_norm}%
{$1\leq\textrm{integer}\leq2$}%
{$1$}%
{Norm of feasibility pump objective function}%
{}

\printoption{fp\_pass\_infeasible}%
{\ttfamily no, yes}%
{no}%
{Say whether feasibility pump should claim to converge or not}%
{\begin{list}{}{
\setlength{\parsep}{0em}
\setlength{\leftmargin}{5ex}
\setlength{\labelwidth}{2ex}
\setlength{\itemindent}{0ex}
\setlength{\topsep}{0pt}}
\item[\texttt{no}] When master MILP is infeasible just bail out (don't stop all algorithm). This is the option for using in B-Hyb.
\item[\texttt{yes}] Claim convergence, numerically dangerous.
\end{list}
}

\printoption{heuristic\_RINS}%
{\ttfamily no, yes}%
{no}%
{if yes runs the RINS heuristic}%
{
}

\printoption{heuristic\_dive\_MIP\_fractional}%
{\ttfamily no, yes}%
{no}%
{if yes runs the Dive MIP Fractional heuristic}%
{
}

\printoption{heuristic\_dive\_MIP\_vectorLength}%
{\ttfamily no, yes}%
{no}%
{if yes runs the Dive MIP VectorLength heuristic}%
{
}

\printoption{heuristic\_dive\_fractional}%
{\ttfamily no, yes}%
{no}%
{if yes runs the Dive Fractional heuristic}%
{
}

\printoption{heuristic\_dive\_vectorLength}%
{\ttfamily no, yes}%
{no}%
{if yes runs the Dive VectorLength heuristic}%
{
}

\printoption{heuristic\_feasibility\_pump}%
{\ttfamily no, yes}%
{no}%
{whether the heuristic feasibility pump should be used}%
{
}

\printoption{pump\_for\_minlp}%
{\ttfamily no, yes}%
{no}%
{if yes runs FP for MINLP}%
{
}

\printoptioncategory{NLP interface}
\printoption{warm\_start}%
{\ttfamily none, optimum, interior\_point}%
{none}%
{Select the warm start method\\
This will affect the function getWarmStart(), and as a consequence the warm starting in the various algorithms.}%
{\begin{list}{}{
\setlength{\parsep}{0em}
\setlength{\leftmargin}{5ex}
\setlength{\labelwidth}{2ex}
\setlength{\itemindent}{0ex}
\setlength{\topsep}{0pt}}
\item[\texttt{none}] No warm start
\item[\texttt{optimum}] Warm start with direct parent optimum
\item[\texttt{interior\_point}] Warm start with an interior point of direct parent
\end{list}
}

\printoptioncategory{NLP solution robustness}
\printoption{max\_consecutive\_failures}%
{$0\leq\textrm{integer}$}%
{$10$}%
{(temporarily removed) Number $n$ of consecutive unsolved problems before aborting a branch of the tree.\\
When $n > 0$, continue exploring a branch of the tree until $n$ consecutive problems in the branch are unsolved (we call unsolved a problem for which Ipopt can not guarantee optimality within the specified tolerances).}%
{}

\printoption{max\_random\_point\_radius}%
{$0<\textrm{real}$}%
{$100000$}%
{Set max value r for coordinate of a random point.\\
When picking a random point, coordinate i will be in the interval [min(max(l,-r),u-r), max(min(u,r),l+r)] (where l is the lower bound for the variable and u is its upper bound)}%
{}

\printoption{num\_iterations\_suspect}%
{$-1\leq\textrm{integer}$}%
{$-1$}%
{Number of iterations over which a node is considered "suspect" (for debugging purposes only, see detailed documentation).\\
When the number of iterations to solve a node is above this number, the subproblem at this node is considered to be suspect and it will be outputed in a file (set to -1 to deactivate this).}%
{}

\printoption{num\_retry\_unsolved\_random\_point}%
{$0\leq\textrm{integer}$}%
{$0$}%
{Number $k$ of times that the algorithm will try to resolve an unsolved NLP with a random starting point (we call unsolved an NLP for which Ipopt is not able to guarantee optimality within the specified tolerances).\\
When Ipopt fails to solve a continuous NLP sub-problem, if $k > 0$, the algorithm will try again to solve the failed NLP with $k$ new randomly chosen starting points  or until the problem is solved with success.}%
{}

\printoption{random\_point\_perturbation\_interval}%
{$0<\textrm{real}$}%
{$1$}%
{Amount by which starting point is perturbed when choosing to pick random point by perturbating starting point}%
{}

\printoption{random\_point\_type}%
{\ttfamily Jon, Andreas, Claudia}%
{Jon}%
{method to choose a random starting point}%
{\begin{list}{}{
\setlength{\parsep}{0em}
\setlength{\leftmargin}{5ex}
\setlength{\labelwidth}{2ex}
\setlength{\itemindent}{0ex}
\setlength{\topsep}{0pt}}
\item[\texttt{Jon}] Choose random point uniformly between the bounds
\item[\texttt{Andreas}] perturb the starting point of the problem within a prescribed interval
\item[\texttt{Claudia}] perturb the starting point using the perturbation radius suffix information
\end{list}
}

\printoptioncategory{NLP solves in hybrid algorithm (B-Hyb)}
\printoption{nlp\_solve\_frequency}%
{$0\leq\textrm{integer}$}%
{$10$}%
{Specify the frequency (in terms of nodes) at which NLP relaxations are solved in B-Hyb.\\
A frequency of 0 amounts to to never solve the NLP relaxation.}%
{}

\printoption{nlp\_solve\_max\_depth}%
{$0\leq\textrm{integer}$}%
{$10$}%
{Set maximum depth in the tree at which NLP relaxations are solved in B-Hyb.\\
A depth of 0 amounts to to never solve the NLP relaxation.}%
{}

\printoption{nlp\_solves\_per\_depth}%
{$0\leq\textrm{real}$}%
{$10^{ 100}$}%
{Set average number of nodes in the tree at which NLP relaxations are solved in B-Hyb for each depth.}%
{}

\printoptioncategory{Nonconvex problems}
\printoption{coeff\_var\_threshold}%
{$0\leq\textrm{real}$}%
{$0.1$}%
{Coefficient of variation threshold (for dynamic definition of cutoff\_decr).}%
{}

\printoption{dynamic\_def\_cutoff\_decr}%
{\ttfamily no, yes}%
{no}%
{Do you want to define the parameter cutoff\_decr dynamically?}%
{
}

\printoption{first\_perc\_for\_cutoff\_decr}%
{$\textrm{real}$}%
{$-0.02$}%
{The percentage used when, the coeff of variance is smaller than the threshold, to compute the cutoff\_decr dynamically.}%
{}

\printoption{max\_consecutive\_infeasible}%
{$0\leq\textrm{integer}$}%
{$0$}%
{Number of consecutive infeasible subproblems before aborting a branch.\\
Will continue exploring a branch of the tree until "max\_consecutive\_infeasible"consecutive problems are infeasibles by the NLP sub-solver.}%
{}

\printoption{num\_resolve\_at\_infeasibles}%
{$0\leq\textrm{integer}$}%
{$0$}%
{Number $k$ of tries to resolve an infeasible node (other than the root) of the tree with different starting point.\\
The algorithm will solve all the infeasible nodes with $k$ different random starting points and will keep the best local optimum found.}%
{}

\printoption{num\_resolve\_at\_node}%
{$0\leq\textrm{integer}$}%
{$0$}%
{Number $k$ of tries to resolve a node (other than the root) of the tree with different starting point.\\
The algorithm will solve all the nodes with $k$ different random starting points and will keep the best local optimum found.}%
{}

\printoption{num\_resolve\_at\_root}%
{$0\leq\textrm{integer}$}%
{$0$}%
{Number $k$ of tries to resolve the root node with different starting points.\\
The algorithm will solve the root node with $k$ random starting points and will keep the best local optimum found.}%
{}

\printoption{second\_perc\_for\_cutoff\_decr}%
{$\textrm{real}$}%
{$-0.05$}%
{The percentage used when, the coeff of variance is greater than the threshold, to compute the cutoff\_decr dynamically.}%
{}

\printoptioncategory{Outer Approximation Decomposition (B-OA)}
\printoption{oa\_decomposition}%
{\ttfamily no, yes}%
{no}%
{If yes do initial OA decomposition}%
{}

\printoptioncategory{Outer Approximation cuts generation}
\printoption{add\_only\_violated\_oa}%
{\ttfamily no, yes}%
{no}%
{Do we add all OA cuts or only the ones violated by current point?}%
{\begin{list}{}{
\setlength{\parsep}{0em}
\setlength{\leftmargin}{5ex}
\setlength{\labelwidth}{2ex}
\setlength{\itemindent}{0ex}
\setlength{\topsep}{0pt}}
\item[\texttt{no}] Add all cuts
\item[\texttt{yes}] Add only violated Cuts
\end{list}
}

\printoption{oa\_cuts\_scope}%
{\ttfamily local, global}%
{global}%
{Specify if OA cuts added are to be set globally or locally valid}%
{\begin{list}{}{
\setlength{\parsep}{0em}
\setlength{\leftmargin}{5ex}
\setlength{\labelwidth}{2ex}
\setlength{\itemindent}{0ex}
\setlength{\topsep}{0pt}}
\item[\texttt{local}] Cuts are treated as locally valid
\item[\texttt{global}] Cuts are treated as globally valid
\end{list}
}

\printoption{tiny\_element}%
{$-0\leq\textrm{real}$}%
{$10^{- 8}$}%
{Value for tiny element in OA cut\\
We will remove "cleanly" (by relaxing cut) an element lower than this.}%
{}

\printoption{very\_tiny\_element}%
{$-0\leq\textrm{real}$}%
{$10^{-17}$}%
{Value for very tiny element in OA cut\\
Algorithm will take the risk of neglecting an element lower than this.}%
{}

\printoptioncategory{Output}
\printoption{bb\_log\_interval}%
{$0\leq\textrm{integer}$}%
{$100$}%
{Interval at which node level output is printed.\\
Set the interval (in terms of number of nodes) at which a log on node resolutions (consisting of lower and upper bounds) is given.}%
{}

\printoption{bb\_log\_level}%
{$0\leq\textrm{integer}\leq5$}%
{$1$}%
{specify main branch-and-bound log level.\\
Set the level of output of the branch-and-bound : 0 - none, 1 - minimal, 2 - normal low, 3 - normal high}%
{}

\printoption{fp\_log\_frequency}%
{$0<\textrm{real}$}%
{$100$}%
{display an update on lower and upper bounds in FP every n seconds}%
{}

\printoption{fp\_log\_level}%
{$0\leq\textrm{integer}\leq2$}%
{$1$}%
{specify FP iterations log level.\\
Set the level of output of OA decomposition solver : 0 - none, 1 - normal, 2 - verbose}%
{}

\printoption{lp\_log\_level}%
{$0\leq\textrm{integer}\leq4$}%
{$0$}%
{specify LP log level.\\
Set the level of output of the linear programming sub-solver in B-Hyb or B-QG : 0 - none, 1 - minimal, 2 - normal low, 3 - normal high, 4 - verbose}%
{}

\printoption{milp\_log\_level}%
{$0\leq\textrm{integer}\leq4$}%
{$0$}%
{specify MILP solver log level.\\
Set the level of output of the MILP subsolver in OA : 0 - none, 1 - minimal, 2 - normal low, 3 - normal high}%
{}

\printoption{nlp\_log\_at\_root}%
{$0\leq\textrm{integer}\leq12$}%
{$5$}%
{ Specify a different log level for root relaxtion.}%
{}

\printoption{nlp\_log\_level}%
{$0\leq\textrm{integer}\leq2$}%
{$1$}%
{specify NLP solver interface log level (independent from ipopt print\_level).\\
Set the level of output of the OsiTMINLPInterface : 0 - none, 1 - normal, 2 - verbose}%
{}

\printoption{oa\_cuts\_log\_level}%
{$0\leq\textrm{integer}$}%
{$0$}%
{level of log when generating OA cuts.\\
0: outputs nothing,\\1: when a cut is generated, its violation and index of row from which it originates,\\2: always output violation of the cut.\\3: output generated cuts incidence vectors.}%
{}

\printoption{oa\_log\_frequency}%
{$0<\textrm{real}$}%
{$100$}%
{display an update on lower and upper bounds in OA every n seconds}%
{}

\printoption{oa\_log\_level}%
{$0\leq\textrm{integer}\leq2$}%
{$1$}%
{specify OA iterations log level.\\
Set the level of output of OA decomposition solver : 0 - none, 1 - normal, 2 - verbose}%
{}

\printoptioncategory{Strong branching setup}
\printoption{candidate\_sort\_criterion}%
{\ttfamily best-ps-cost, worst-ps-cost, most-fractional, least-fractional}%
{best-ps-cost}%
{Choice of the criterion to choose candidates in strong-branching}%
{\begin{list}{}{
\setlength{\parsep}{0em}
\setlength{\leftmargin}{5ex}
\setlength{\labelwidth}{2ex}
\setlength{\itemindent}{0ex}
\setlength{\topsep}{0pt}}
\item[\texttt{best-ps-cost}] Sort by decreasing pseudo-cost
\item[\texttt{worst-ps-cost}] Sort by increasing pseudo-cost
\item[\texttt{most-fractional}] Sort by decreasing integer infeasibility
\item[\texttt{least-fractional}] Sort by increasing integer infeasibility
\end{list}
}

\printoption{maxmin\_crit\_have\_sol}%
{$0\leq\textrm{real}\leq1$}%
{$0.1$}%
{Weight towards minimum in of lower and upper branching estimates when a solution has been found.}%
{}

\printoption{maxmin\_crit\_no\_sol}%
{$0\leq\textrm{real}\leq1$}%
{$0.7$}%
{Weight towards minimum in of lower and upper branching estimates when no solution has been found yet.}%
{}

\printoption{min\_number\_strong\_branch}%
{$0\leq\textrm{integer}$}%
{$0$}%
{Sets minimum number of variables for strong branching (overriding trust)}%
{}

\printoption{number\_before\_trust\_list}%
{$-1\leq\textrm{integer}$}%
{$0$}%
{Set the number of branches on a variable before its pseudo costs are to be believed during setup of strong branching candidate list.\\
The default value is that of "number\_before\_trust"}%
{}

\printoption{number\_look\_ahead}%
{$0\leq\textrm{integer}$}%
{$0$}%
{Sets limit of look-ahead strong-branching trials}%
{}

\printoption{number\_strong\_branch\_root}%
{$0\leq\textrm{integer}$}%
{$\infty$}%
{Maximum number of variables considered for strong branching in root node.}%
{}

\printoption{setup\_pseudo\_frac}%
{$0\leq\textrm{real}\leq1$}%
{$0.5$}%
{Proportion of strong branching list that has to be taken from most-integer-infeasible list.}%
{}

\printoption{trust\_strong\_branching\_for\_pseudo\_cost}%
{\ttfamily no, yes}%
{yes}%
{Whether or not to trust strong branching results for updating pseudo costs.}%
{}



\subsubsection{Ipopt}
All Ipopt options are available in CoinBonmin, please refer to section \ref{sub:ipoptoptions} for a detailed description.
The default value of the following Ipopt parameters are changed in Gams/CoinBonmin:
\begin{itemize}
\item \texttt{mu\_strategy} and \texttt{mu\_oracle} are set, respectively, to {\tt adaptive} and {\tt probing} by default.
\item \texttt{gamma\_phi} and \texttt{gamma\_theta} are set to $10^{-8}$ and $10^{-4}$ respectively. This has the effect of reducing the size of the filter in the line search performed by Ipopt.
\item \texttt{required\_infeasibility\_reduction} is set to $0.1$.
This increases the required infeasibility reduction when Ipopt enters the restoration phase and should thus help
detect infeasible problems faster.
\item \texttt{expect\_infeasible\_problem} is set to {\tt yes} which enables some heuristics to detect infeasible problems faster.
\item \texttt{warm\_start\_init\_point} is set to {\tt yes} when a full primal/dual starting point is available (generally for all the optimizations after the continuous relaxation has been solved).
\item \texttt{print\_level} is set to $0$ by default to turn off Ipopt output.
\item \texttt{bound\_relax\_factor} is set to $0.0$.
\end{itemize}

\section{CoinCbc}
\label{sec:coincbc}
\hypertarget{sec:coincbc}{}

GAMS/CoinCBC brings the open source LP/MIP solver CBC from the COIN-OR foundation to the broad audience of GAMS users.

CBC (COIN-OR Branch and Cut) is an open-source mixed integer programming solver working with the COIN-OR LP solver CLP and the COIN-OR Cut generator library Cgl.
The code has been written primarily by John J. Forrest, who is the COIN-OR project leader for Cbc.

For more information we refer to
\begin{itemize}
\item the CBC web site \texttt{https://projects.coin-or.org/Cbc},
\item the Cgl web site \texttt{https://projects.coin-or.org/Cgl}, and
\item the CLP web site \texttt{https://projects.coin-or.org/Clp}.
\end{itemize}
Most of the CBC documentation in the section was copied from the help in the CBC standalone version.

\subsection{Model requirements}

The CBC link in GAMS supports continuous, binary, integer, semicontinuous, semiinteger variables, special ordered sets of type 1 and 2, and branching priorities (see chapter 17.1 of the GAMS User's Guide).
% Quadratic objective functions are not supported yet.

\subsection{Usage of CoinCbc}

The following statement can be used inside your GAMS program to specify using CoinCBC
\begin{verbatim}
  Option LP = CoinCbc;     { or MIP or RMIP }
\end{verbatim}

The above statement should appear before the Solve statement.
If CoinCBC was specified as the default solver during GAMS installation, the above statement is not necessary.

There are many parameters which can affect the performance the CBCs Branch and Cut Algorithm.
First just try with default settings and look carefully at the log file.
Did cuts help? Did they take too long? Look at the output to see which cuts were effective and then do some tuning (see the option \hyperlink{cuts}{cuts}).
If the \hyperlink{preprocess}{preprocessing} reduced the size of the problem or strengthened many coefficients then it is probably wise to leave it on.
Switch off \hyperlink{heuristics}{heuristics} which did not provide solutions.
The other major area to look at is the search. Hopefully good solutions were obtained fairly early in the search so the important point is to select the best variable to branch on.
See whether strong branching did a good job - or did it just take a lot of iterations.
Adjust the options \hyperlink{strongbranching}{strongbranching} and \hyperlink{trustpseudocosts}{trustpseudocosts}.

The GAMS/CoinCBC options file consists of one option or comment per line.
An asterisk (*) at the beginning of a line causes the entire line to be ignored.
Otherwise, the line will be interpreted as an option name and value separated by any amount of white space (blanks or tabs).
Following is an example options file coincbc.opt.
\begin{verbatim}
  cuts root
  perturbation off
\end{verbatim}
It will cause CoinCBC to use cut generators only in the root node and turns off the perturbation of the LP relaxation.

GAMS/CoinCBC now support the GAMS Branch-and-Cut-and-Heuristic (BCH) Facility.
The GAMS BCH facility automates all major steps necessary to define, execute, and control the use of user defined routines within the framework of general purpose MIP codes.
Currently supported are user defined cut generators and heuristics.
Please see the BCH documentation at \texttt{http://www.gams.com/docs/bch.htm} for further information.

% \subsection{CoinCbc output}
% \subsection{Some CoinCbc features}


\subsubsection{General Options}
\begin{tabbing}
\hspace {1.3in} \= \\
\hyperlink{reslim}
{reslim} \> resource limit \\
\hyperlink{special}
{special} \> options passed unseen to CBC \\
\hyperlink{writemps}
{writemps} \> create MPS file for problem
\end{tabbing}

\subsubsection{LP Options}
\begin{tabbing}
\hspace {1.3in} \= \\
\hyperlink{iterlim}
{iterlim} \> iteration limit \\
\hyperlink{idiotcrash}
{idiotcrash} \> idiot crash \\
\hyperlink{sprintcrash}
{sprintcrash} \> sprint crash \\
\hyperlink{sifting}
{sifting} \> synonym for sprint crash \\
\hyperlink{crash}
{crash} \> use crash method to get dual feasible \\
\hyperlink{maxfactor}
{maxfactor} \> maximum number of iterations between refactorizations \\
\hyperlink{crossover}
{crossover} \> crossover to simplex algorithm after barrier \\
\hyperlink{dualpivot}
{dualpivot} \> dual pivot choice algorithm \\
\hyperlink{primalpivot}
{primalpivot} \> primal pivot choice algorithm \\
\hyperlink{perturbation}
{perturbation} \> perturbation of problem \\
\hyperlink{scaling}
{scaling} \> scaling method \\
\hyperlink{presolve}
{presolve} \> switch for initial presolve of LP \\
\hyperlink{passpresolve}
{passpresolve} \> how many passes to do in presolve \\
\hyperlink{randomseedclp}
{randomseedclp} \> random seed for CLP \\
\hyperlink{tol_dual}
{tol\_dual} \> dual feasibility tolerance \\
\hyperlink{tol_primal}
{tol\_primal} \> primal feasibility tolerance \\
\hyperlink{tol_presolve}
{tol\_presolve} \> tolerance used in presolve \\
\hyperlink{startalg}
{startalg} \> LP solver for root node
\end{tabbing}


\subsubsection{MIP Options}
\begin{tabbing}
\hspace {1.3in} \= \\
\hyperlink{threads}
{threads} \> number of threads to use (available on Unix variants only) \\
\hyperlink{strategy}
{strategy} \> switches on groups of features \\
\hyperlink{mipstart}
{mipstart} \> whether it should be tried to use the initial variable levels as initial MIP solution \\
\hyperlink{tol_integer}
{tol\_integer} \> tolerance for integrality \\
\hyperlink{sollim}
{sollim} \> limit on number of solutions \\
\hyperlink{dumpsolutions}
{dumpsolutions} \> name of solutions index gdx file for writing alternate solutions \\
\hyperlink{maxsol}
{maxsol} \> maximal number of solutions to store during search \\
\hyperlink{strongbranching}
{strongbranching} \> strong branching \\
\hyperlink{trustpseudocosts}
{trustpseudocosts} \> after howmany nodes we trust the pseudo costs \\
\hyperlink{coststrategy}
{coststrategy} \> how to use costs as priorities \\
\hyperlink{extravariables}
{extravariables} \> group together variables with same cost \\
\hyperlink{multiplerootpasses}
{multiplerootpasses} \> runs multiple copies of the solver at the root node \\
\hyperlink{nodestrategy}
{nodestrategy} \> how to select nodes \\
\hyperlink{preprocess}
{preprocess} \> integer presolve \\
\hyperlink{printfrequency}
{printfrequency} \> frequency of status prints \\
\hyperlink{randomseedcbc}
{randomseedcbc} \> random seed for CBC \\
\hyperlink{loglevel}
{loglevel} \> CBC loglevel \\
\hyperlink{increment}
{increment} \> increment of cutoff when new incumbent \\
\hyperlink{solvefinal}
{solvefinal} \> final solve of MIP with fixed discrete variables \\
\hyperlink{solvetrace}
{solvetrace} \> name of trace file for solving information \\
\hyperlink{solvetracenodefreq}
{solvetracenodefreq} \> frequency in number of nodes for writing to solve trace file \\
\hyperlink{solvetracetimefreq}
{solvetracetimefreq} \> frequency in seconds for writing to solve trace file \\
\hyperlink{nodelim}
{nodelim} \> node limit \\
\hyperlink{nodlim}
{nodlim} \> node limit \\
\hyperlink{optca}
{optca} \> absolute stopping tolerance \\
\hyperlink{optcr}
{optcr} \> relative stopping tolerance \\
\hyperlink{cutoff}
{cutoff} \> cutoff for objective function value \\
\hyperlink{cutoffconstraint}
{cutoffconstraint} \> whether to add a constraint from the objective function
\end{tabbing}


\subsubsection{MIP Options for Cutting Plane Generators}
\begin{tabbing}
\hspace {1.3in} \= \\
\hyperlink{cutdepth}
{cutdepth} \> depth in tree at which cuts are applied \\
\hyperlink{cut_passes_root}
{cut\_passes\_root} \> number of cut passes at root node \\
\hyperlink{cut_passes_tree}
{cut\_passes\_tree} \> number of cut passes at nodes in the tree \\
\hyperlink{cut_passes_slow}
{cut\_passes\_slow} \> number of cut passes for slow cut generators \\
\hyperlink{cuts}
{cuts} \> global switch for cutgenerators \\
\hyperlink{cliquecuts}
{cliquecuts} \> Clique Cuts \\
\hyperlink{flowcovercuts}
{flowcovercuts} \> Flow Cover Cuts \\
\hyperlink{gomorycuts}
{gomorycuts} \> Gomory Cuts \\
\hyperlink{gomorycuts2}
{gomorycuts2} \> Gomory Cuts 2nd implementation \\
\hyperlink{knapsackcuts}
{knapsackcuts} \> Knapsack Cover Cuts \\
\hyperlink{liftandprojectcuts}
{liftandprojectcuts} \> Lift and Project Cuts \\
\hyperlink{mircuts}
{mircuts} \> Mixed Integer Rounding Cuts \\
\hyperlink{twomircuts}
{twomircuts} \> Two Phase Mixed Integer Rounding Cuts \\
\hyperlink{probingcuts}
{probingcuts} \> Probing Cuts \\
\hyperlink{reduceandsplitcuts}
{reduceandsplitcuts} \> Reduce and Split Cuts \\
\hyperlink{reduceandsplitcuts2}
{reduceandsplitcuts2} \> Reduce and Split Cuts 2nd implementation \\
\hyperlink{residualcapacitycuts}
{residualcapacitycuts} \> Residual Capacity Cuts \\
\hyperlink{zerohalfcuts}
{zerohalfcuts} \> Zero-Half Cuts \\
\end{tabbing}


\subsubsection{MIP Options for Heuristics}
\begin{tabbing}
\hspace {1.3in} \= \\
\hyperlink{heuristics}
{heuristics} \> global switch for heuristics \\
\hyperlink{combinesolutions}
{combinesolutions} \> combine solutions heuristic \\
\hyperlink{dins}
{dins} \> distance induced neighborhood search \\
\hyperlink{divingrandom}
{divingrandom} \> turns on random diving heuristic \\
\hyperlink{divingcoefficient}
{divingcoefficient} \> coefficient diving heuristic \\
\hyperlink{divingfractional}
{divingfractional} \> fractional diving heuristic \\
\hyperlink{divingguided}
{divingguided} \> guided diving heuristic \\
\hyperlink{divinglinesearch}
{divinglinesearch} \> line search diving heuristic \\
\hyperlink{divingpseudocost}
{divingpseudocost} \> pseudo cost diving heuristic \\
\hyperlink{divingvectorlength}
{divingvectorlength} \> vector length diving heuristic \\
\hyperlink{feaspump}
{feaspump} \> feasibility pump \\
\hyperlink{feaspump_passes}
{feaspump\_passes} \> number of feasibility passes \\
\hyperlink{greedyheuristic}
{greedyheuristic} \> greedy heuristic \\
\hyperlink{localtreesearch}
{localtreesearch} \> local tree search heuristic \\
\hyperlink{naiveheuristics}
{naiveheuristics} \> naive heuristics \\
\hyperlink{pivotandfix}
{pivotandfix} \> pivot and fix heuristic \\
\hyperlink{randomizedrounding}
{randomizedrounding} \> randomized rounding heuristis \\
\hyperlink{rens}
{rens} \> relaxation enforced neighborhood search \\
\hyperlink{rins}
{rins} \> relaxed induced neighborhood search \\
\hyperlink{roundingheuristic}
{roundingheuristic} \> rounding heuristic \\
\hyperlink{vubheuristic}
{vubheuristic} \> VUB heuristic \\
\hyperlink{proximitysearch}
{proximitysearch} \> proximity search heuristic
\end{tabbing}


\subsubsection{General Options}

\begin{description}

\item[\label{iterlim}\hypertarget{iterlim}
{\textbf{iterlim (\slshape{integer})}}]\hspace{1.0in}

For an LP, this is the maximum number of iterations to solve the LP.
For a MIP, this option is ignored.

\textsl{(default = GAMS iterlim)}

\item[\label{names}\hypertarget{names}
{\textbf{names (\slshape{integer})}}]\hspace{1.0in}

This option causes GAMS names for the variables and equations to be loaded into Cbc.
These names will then be used for error messages, log entries, and so forth.
Turning names off may help if memory is very tight.

\textsl{(default = 0)}
\begin{itemize}
\item[0] Do not load variable and equation names.
\item[1] Load variable and equation names.
\end{itemize}

\item[\label{reslim}\hypertarget{reslim}
{\textbf{reslim (\slshape{real})}}]\hspace{1.0in}

Maximum CPU time in seconds.

\textsl{(default = GAMS reslim)}

\item[\label{writemps}\hypertarget{writemps}
{\textbf{writemps (\slshape{string})}}]\hspace{1.0in}

Write the problem formulation in MPS format.
The parameter value is the name of the MPS file.

\item[\label{special}\hypertarget{special}
{\textbf{special (\slshape{string})}}]\hspace{1.0in}

This parameter let you specify CBC options which are not supported by the GAMS/CoinCBC interface.

The string value given to this parameter is split up into parts at each space and added to the array of parameters given to CBC (in front of the -solve command).
Hence, you can use it like the command line parameters for the CBC standalone version.

\end{description}

\subsubsection{LP Options}

\begin{description}

\item[\label{idiotcrash}\hypertarget{idiotcrash}
{\textbf{idiotcrash (\slshape{integer})}}]\hspace{1.0in}

This is a type of `crash' which works well on some homogeneous problems.
It works best on problems with unit elements and right hand sides but will do something to any model.
It should only be used before the primal simplex algorithm.

A positive number determines the number of passes that idiotcrash is called.

\textsl{(default = -1)}
\begin{itemize}
\item[-1] 
Let CLP decide by itself whether to use it.
\item[0] 
Switch this method off.
\end{itemize}

\item[\label{sprintcrash}\hypertarget{sprintcrash}
{\textbf{sprintcrash (\slshape{integer})}}]\hspace{1.0in}

For long and thin problems this method may solve a series of small problems created by taking a subset of the columns.
Cplex calls it `sifting'.

A positive number determines the number of passes that sprintcrash is called.

\textsl{(default = -1)}
\begin{itemize}
\item[-1] 
Let CLP decide by itself whether to use it.
\item[0] 
Switch this method off.
\end{itemize}

\item[\label{sifting}\hypertarget{sifting}
{\textbf{sifting (\slshape{integer})}}]\hspace{1.0in}

Synonym for \hyperlink{sprintcrash}{sprintcrash}.

\textsl{(default = -1)}

\item[\label{crash}\hypertarget{crash}
{\textbf{crash (\slshape{string})}}]\hspace{1.0in}

Determines whether CLP should use a crash algorithm to find a dual feasible basis.

\textsl{(default = off)}
\begin{itemize}
\item[off] 
Switch off the creation of dual feasible basis by the crash method.
\item[on] 
Switch on the creation of dual feasible basis by the crash method.
\item[solow\_halim] 
Switch on a crash variant due to Solow and Halim.
\item[halim\_solow] 
Switch on a crash variant due to Solow and Halim with modifications of John J. Forrest.
\end{itemize}

\item[\label{maxfactor}\hypertarget{maxfactor}
{\textbf{maxfactor (\slshape{integer})}}]\hspace{1.0in}

Maximum number of iterations between refactorizations in CLP.

If this is left at the default value of 200 then CLP will guess at a value to use.
CLP may decide to refactorize earlier for accuracy.

\textsl{(default = 200)}

\item[\label{crossover}\hypertarget{crossover}
{\textbf{crossover (\slshape{integer})}}]\hspace{1.0in}

Determines whether CLP should crossover to the simplex algorithm after the barrier algorithm finished.

Interior point algorithms do not obtain a basic solution.
This option will crossover to a basic solution suitable for ranging or branch and cut.

\textsl{(default = 1)}
\begin{itemize}
\item[0] 
Turn off crossover to simplex algorithm after barrier algorithm finished.
\item[1] 
Turn on crossover to simplex algorithm after barrier algorithm finished.
\end{itemize}

\item[\label{dualpivot}\hypertarget{dualpivot}
{\textbf{dualpivot (\slshape{string})}}]\hspace{1.0in}

Choice of the pivoting strategy in the dual simplex algorithm.

\textsl{(default = auto)}
\begin{itemize}
\item[auto] 
Let CLP use a variant of the steepest choice method which starts like partial, i.e., scans only a subset of the primal infeasibilities,
and later changes to full pricing when the factorization becomes denser.
\item[dantzig] 
Let CLP use the pivoting strategy due to Dantzig.
\item[steepest] 
Let CLP use the steepest choice method.
\item[partial] 
Let CLP use a variant of the steepest choice method which scans only a subset of the primal infeasibilities to select the pivot step.
\end{itemize}

\item[\label{primalpivot}\hypertarget{primalpivot}
{\textbf{primalpivot (\slshape{string})}}]\hspace{1.0in}

Choice of the pivoting strategy in the primal simplex algorithm.

\textsl{(default = auto)}
\begin{itemize}
\item[auto] 
Let CLP use a variant of the exact devex method.
\item[dantzig] 
Let CLP use the pivoting strategy due to Dantzig.
\item[steepest] 
Let CLP use the steepest choice method.
\item[partial] 
Let CLP use a variant of the exact devex method which scans only a subset of the primal infeasibilities to select the pivot step.
\item[exact] 
Let CLP use the exact devex method.
\item[change] 
Let CLP initially use Dantzig pivot method until the factorization becomes denser.
\end{itemize}

\item[\label{perturbation}\hypertarget{perturbation}
{\textbf{perturbation (\slshape{integer})}}]\hspace{1.0in}

Determines whether CLP should perturb the problem before starting.
Perturbation helps to stop cycling, but CLP uses other measures for this.
However, large problems and especially ones with unit elements and unit right hand sides or costs benefit from perturbation.
Normally CLP tries to be intelligent, but you can switch this off.

\textsl{(default = 1)}
\begin{itemize}
\item[0] 
Turns off perturbation of LP.
\item[1] 
Turns on perturbation of LP.
\end{itemize}

\item[\label{scaling}\hypertarget{scaling}
{\textbf{scaling (\slshape{string})}}]\hspace{1.0in}

Scaling can help in solving problems which might otherwise fail because of lack of accuracy.
It can also reduce the number of iterations.
It is not applied if the range of elements is small.
Both methods do several passes alternating between rows and columns using current scale factors from one and applying them to the other.

\textsl{(default = auto)}
\begin{itemize}
\item[off] 
Turns off scaling.
\item[auto] 
Let CLP choose the scaling method automatically.
It decides for one of these methods depending on which gives the better ratio of the largest element to the smallest one.
\item[equilibrium] 
Let CLP use an equilibrium based scaling method which uses the largest scaled element.
\item[geometric] 
Let CLP use a geometric based scaling method which uses the squareroot of the product of largest and smallest element.
\end{itemize}

\item[\label{presolve}\hypertarget{presolve}
{\textbf{presolve (\slshape{integer})}}]\hspace{1.0in}

Presolve analyzes the model to find such things as redundant constraints, constraints which fix some variables, constraints which can be transformed into bounds, etc.
For the initial solve of any problem this is worth doing unless you know that it will have no effect.

\textsl{(default = 1)}
\begin{itemize}
\item[0] 
Turns off the initial presolve.
\item[1] 
Turns on the initial presolve.
\end{itemize}

\item[\label{tol_dual}\hypertarget{tol_dual}
{\textbf{tol\_dual (\slshape{real})}}]\hspace{1.0in}

The maximum amount the dual constraints can be violated and still be considered feasible.

\textsl{(default = 1e-7)}

\item[\label{tol_primal}\hypertarget{tol_primal}
{\textbf{tol\_primal (\slshape{real})}}]\hspace{1.0in}

The maximum amount the primal constraints can be violated and still be considered feasible.

\textsl{(default = 1e-7)}

\item[\label{tol_presolve}\hypertarget{tol_presolve}
{\textbf{tol\_presolve (\slshape{real})}}]\hspace{1.0in}

The tolerance used in presolve.

\textsl{(default = 1e-8)}

\item[\label{passpresolve}\hypertarget{passpresolve}
{\textbf{passpresolve (\slshape{integer})}}]\hspace{1.0in}

Normally Presolve does 5 passes but you may want to do less to make
it more lightweight or do more if improvements are still being made.
As Presolve will return if nothing is being taken out, you should
not normally need to use this fine tuning.

\textsl{(default = 5)}

\item[\label{startalg}\hypertarget{startalg}
{\textbf{startalg (\slshape{string})}}]\hspace{1.0in}

Determines the algorithm to use for an LP or the initial LP relaxation if the problem is a MIP.

\textsl{(default = dual)}
\begin{itemize}
\item[primal] 
Let CLP use the primal simplex algorithm.
\item[dual] 
Let CLP use the dual simplex algorithm.
\item[barrier] 
Let CLP use a primal dual predictor corrector algorithm.
\end{itemize}

\end{description}

\subsubsection{MIP Options}

\begin{description}

\item[\label{mipstart}\hypertarget{mipstart}
{\textbf{mipstart (\slshape{integer})}}]\hspace{1.0in}

This option controls the use of advanced starting values for mixed integer programs.
A setting of 1 indicates that the variable level values should be checked to see if they provide an integer feasible solution before starting optimization.

\textsl{(default = 0)}
\begin{itemize}
\item[0] 
Do not use the initial variable levels.
\item[1] 
Try to use the initial variable levels as a MIP starting solution.
\end{itemize}

\item[\label{strategy}\hypertarget{strategy}
{\textbf{strategy (\slshape{integer})}}]\hspace{1.0in}

Setting strategy to 1 (the default) uses Gomory cuts using tolerance of 0.01 at root,
does a possible restart after 100 nodes if Cbc can fix many variables and activates
a diving and RINS heuristic and makes feasibility pump more aggressive.

\textsl{(default = 1)}
\begin{itemize}
\item[0] 
Use this setting for easy problems.
\item[1] 
This is the default setting.
\item[2]
Use this setting for difficult problems.
\end{itemize}

\item[\label{tol_integer}\hypertarget{tol_integer}
{\textbf{tol\_integer (\slshape{real})}}]\hspace{1.0in}

For an optimal solution, no integer variable may be farther than this from an integer value.

\textsl{(default = 1e-6)}

\item[\label{sollim}\hypertarget{sollim}
{\textbf{sollim (\slshape{integer})}}]\hspace{1.0in}

A limit on number of feasible solutions that CBC should find for a MIP.

\textsl{(default = -1)}
\begin{itemize}
\item[-1] 
No limit on the number of feasible solutions.
\end{itemize}

\item[\label{strongbranching}\hypertarget{strongbranching}
{\textbf{strongbranching (\slshape{integer})}}]\hspace{1.0in}

Determines the number of variables to look at in strong branching.

In order to decide which variable to branch on, the code will choose up to this number of unsatisfied variables and try minimal up and down branches.
The most effective one is chosen.
If a variable is branched on many times then the previous average up and down costs may be used - see the option \hyperlink{trustpseudocosts}{trustpseudocosts}.

\textsl{(default = 5)}

\item[\label{trustpseudocosts}\hypertarget{trustpseudocosts}
{\textbf{trustpseudocosts (\slshape{integer})}}]\hspace{1.0in}

Using strong branching computes pseudo-costs.
This parameter determines after how many branches for a variable we just trust the pseudo costs and do not do any more strong branching.

\textsl{(default = 5)}

\item[\label{coststrategy}\hypertarget{coststrategy}
{\textbf{coststrategy (\slshape{string})}}]\hspace{1.0in}

This parameter influence the branching variable selection.

If turned on, then the variables are sorted in order of their absolute costs, and branching is done first on variables with largest cost.
This primitive strategy can be surprisingly effective.

\textsl{(default = off)}
\begin{itemize}
\item[off] 
Turns off a specific cost strategy.
\item[priorities] 
Assigns highest priority to variables with largest absolute cost.
\item[columnorder] 
Assigns the priorities 1, 2, 3,.. with respect to the column ordering.
\item[binaryfirst] 
Handles two sets of priorities such that binary variables get high priority.
\item[binarylast] 
Handles two sets of priorities such that binary variables get low priority.
\item[length] 
Assigns high priority to variables that are at most nonzero.
\end{itemize}

\item[\label{nodestrategy}\hypertarget{nodestrategy}
{\textbf{nodestrategy (\slshape{string})}}]\hspace{1.0in}

This determines the strategy used to select the next node from the branch and cut tree.

\textsl{(default = fewest)}
\begin{itemize}
\item[hybrid] 
Let CBC do first a breath search on nodes with a small depth in the tree and then switch to choose nodes with fewest infeasibilities.
\item[fewest] 
This will let CBC choose the node with the fewest infeasibilities.
\item[depth] 
This will let CBC always choose the node deepest in tree.
It gives minimum tree size but may take a long time to find the best solution.
\item[upfewest] 
This will let CBC choose the node with the fewest infeasibilities and do up branches first.
\item[downfewest] 
This will let CBC choose the node with the fewest infeasibilities and do down branches first.
\item[updepth] 
This will let CBC choose the node deepest in tree and do up branches first.
\item[downdepth] 
This will let CBC choose the node deepest in tree and do down branches first.
\end{itemize}

\item[\label{preprocess}\hypertarget{preprocess}
{\textbf{preprocess (\slshape{string})}}]\hspace{1.0in}

This option controls the MIP specific presolve routines.
They try to reduce the size of the model in a similar way to presolve and also try to strengthen the model.
This can be very useful and is worth trying.

\textsl{(default = on)}
\begin{itemize}
\item[off] 
Turns off the presolve routines.
\item[on] 
Turns on the presolve routines.
\item[equal] 
Turns on the presolve routines and let CBC turn inequalities with more than 5 elements into equalities (cliques) by adding slack variables.
\item[equalall] 
Turns on the presolve routines and let CBC turn all inequalities into equalities by adding slack variables.
\item[sos] 
This option let CBC search for rows with upper bound 1 and where all nonzero coefficients are 1 and creates special ordered sets if the sets are not overlapping and all integer variables (except for at most one) are in the sets.
\item[trysos] 
This option is similar to sos, but allows any number integer variables to be outside of the sets.
\end{itemize}

\item[\label{threads}\hypertarget{threads}
{\textbf{threads (\slshape{integer})}}]\hspace{1.0in}

This option controls the multithreading feature of CBC, which is currently available only on Unix variants.
A number between 1 and 100 sets the number of threads used for parallel branch and bound.
A number $100+n$ with $n$ between 1 and 100 says that $n$ threads are used to parallelize the branch and bound, but also heuristics such as RINS which do branch and bound on a reduced model also use threads.
A number $200+n$ with $n$ between 1 and 100 says that $n$ threads are used to parallelize the branch and bound, but also the cut generators at the root node (i.e., before threads are useful) are run in parallel.
A number $300+n$ with $n$ between 1 and 100 combines the $100+n$ and $200+n$ options.
A number $400+n$ with $n$ between 1 and 100 says that $n$ threads are used in sub-trees.
Thus, $n$ threads are used to parallelize the branch and bound, but also heuristics use threads and the cut generators at the root node are run in parallel.
The $100+n$, $200+n$, and $300+n$ options are experimental.


\item[\label{printfrequency}\hypertarget{printfrequency}
{\textbf{printfrequency (\slshape{integer})}}]\hspace{1.0in}

Controls the number of nodes that are evaluated between status prints.

\textsl{(default = 0)}
\begin{itemize}
\item[0] 
Automatic choice, which is 100 for large problems and 1000 for small problems.
\end{itemize}

\item[\label{increment}\hypertarget{increment}
{\textbf{increment (\slshape{real})}}]\hspace{1.0in}

A valid solution must be at least this much better than last integer solution.

If this option is not set then it CBC will try and work one out.
E.g., if all objective coefficients are multiples of 0.01 and only integer variables have entries in objective then this can be set to 0.01.

\textsl{(default = GAMS cheat)}

\item[\label{nodelim}\hypertarget{nodelim}
{\textbf{nodelim (\slshape{integer})}}]\hspace{1.0in}

Maximum number of nodes that are considered in the Branch and Bound.

\textsl{(default = GAMS nodlim)}

\item[\label{nodlim}\hypertarget{nodlim}
{\textbf{nodlim (\slshape{integer})}}]\hspace{1.0in}

Maximum number of nodes that are considered in the Branch and Bound.
This option is overwritten by nodelim, if set.

\textsl{(default = GAMS nodlim)}

\item[\label{optca}\hypertarget{optca}
{\textbf{optca (\slshape{real})}}]\hspace{1.0in}

Absolute optimality criterion for a MIP.
CBC stops if the gap between the best known solution and the best possible solution is less than this value.

\textsl{(default = GAMS optca)}

\item[\label{optcr}\hypertarget{optcr}
{\textbf{optcr (\slshape{real})}}]\hspace{1.0in}

Relative optimality criterion for a MIP.
CBC stops if the relative gap between the best known solution and the best possible solution is less than this value.

\textsl{(default = GAMS optcr)}

\item[\label{cutoff}\hypertarget{cutoff}
{\textbf{cutoff (\slshape{real})}}]\hspace{1.0in}

CBC stops if the objective function values exceeds (in case of maximization) or falls below (in case of minimization) this value.

\textsl{(default = GAMS cutoff)}

\end{description}

\subsubsection{MIP Options for Cutting Plane Generators}

\begin{description}


\item[\label{cutdepth}\hypertarget{cutdepth}
{\textbf{cutdepth (\slshape{integer})}}]\hspace{1.0in}

If the depth in the tree is a multiple of cutdepth, then cut generators are applied.

Cut generators may be off, on only at the root, on if they look useful, or on at some interval.
Setting this option to a positive value K let CBC call a cutgenerator on a node whenever the depth in the tree is a multiple of K.

\textsl{(default = -1)}
\begin{itemize}
\item[-1]
Does not turn on cut generators because the depth of the tree is a multiple of a value.
\end{itemize}

\item[\label{cut_passes_root}\hypertarget{cut_passes_root}
{\textbf{cut\_passes\_root (\slshape{integer})}}]\hspace{1.0in}

Determines the number of rounds that the cut generators are applied in the root node.

A negative value $-n$ means that $n$ passes are also applied if the objective does not drop.

\textsl{(default = 100 passes if the MIP has less than 500 columns, 100 passes (but stop if the drop in the objective function value is small) if it has less than 5000 columns, and 20 passes otherwise)}


\item[\label{cut_passes_tree}\hypertarget{cut_passes_tree}
{\textbf{cut\_passes\_tree (\slshape{integer})}}]\hspace{1.0in}

Determines the number of rounds that the cut generators are applied in the nodes of the tree other than the root node.

A negative value $-n$ means that $n$ passes are also applied if the objective does not drop.

\textsl{(default = 1)}

\item[\label{cuts}\hypertarget{cuts}
{\textbf{cuts (\slshape{string})}}]\hspace{1.0in}

A global switch to turn on or off the cutgenerators.

This can be used to switch on or off all default cut generators.
Then you can set individual ones off or on using the specific options.

\textsl{(default = on)}
\begin{itemize}
\item[off] 
Turns off all cut generators.
\item[on] 
Turns on all default cut generators and CBC will try them in the branch and cut tree (see the option \hyperlink{cutdepth}{cutdepth} on how to fine tune the behaviour).
\item[root] 
Let CBC generate cuts only at the root node.
\item[ifmove] 
Let CBC use cut generators in the tree if they look as if they are doing some good and moving the objective value.
\item[forceon] 
Turns on all default cut generators and force CBC to use the cut generator at every node.
\end{itemize}

\item[\label{cliquecuts}\hypertarget{cliquecuts}
{\textbf{cliquecuts (\slshape{string})}}]\hspace{1.0in}

Determines whether and when CBC should try to generate clique cuts.
See the option \hyperlink{cuts}{cuts} for an explanation on the different values.

Clique cuts are of the form ``sum of a set of variables $<=$ 1''.

Reference: M. Eso, Parallel branch and cut for set partitioning, Cornell University, 1999.

\textsl{(default = ifmove)}

\item[\label{flowcovercuts}\hypertarget{flowcovercuts}
{\textbf{flowcovercuts (\slshape{string})}}]\hspace{1.0in}

Determines whether and when CBC should try to generate flow cover cuts.

See the option \hyperlink{cuts}{cuts} for an explanation on the different values.

The flow cover cut generator generates lifted simple generalized flow cover inequalities.
Since flow cover inequalities are generally not facet-defining, they are lifted to obtain stronger inequalities.
Although flow cover inequalities requires a special problem structure to be generated, they are quite useful for solving general mixed integer linear programs.

Reference: Z. Gu, G.L. Nemhauser, M.W.P. Savelsbergh, Lifted flow cover inequalities for mixed 0-1 integer programs, Math. Programming A 85 (1999) 439-467.

\textsl{(default = ifmove)}

\item[\label{gomorycuts}\hypertarget{gomorycuts}
{\textbf{gomorycuts (\slshape{string})}}]\hspace{1.0in}

Determines whether and when CBC should try to generate mixed-integer Gomory cuts.

See the option \hyperlink{cuts}{cuts} for an explanation on the different values.

Reference: Laurence A. Wolsey, Integer Programming, Wiley, John \& Sons, (1998) 124-132.

\textsl{(default = ifmove)}

\item[\label{knapsackcuts}\hypertarget{knapsackcuts}
{\textbf{knapsackcuts (\slshape{string})}}]\hspace{1.0in}

Determines whether and when CBC should try to generate knapsack cover cuts.

See the option \hyperlink{cuts}{cuts} for an explanation on the different values.

The knapsack cover cut generator looks for a series of different types of minimal covers.
If a minimal cover is found, it lifts the associated minimal cover inequality and adds the lifted cut to the cut set.

Reference: S. Martello, and P. Toth, Knapsack Problems, Wiley, 1990, p30.

\textsl{(default = ifmove)}

\item[\label{liftandprojectcuts}\hypertarget{liftandprojectcuts}
{\textbf{liftandprojectcuts (\slshape{string})}}]\hspace{1.0in}

Determines whether and when CBC should try to generate lift and project cuts.
They might be expensive to compute, thus they are switched off by default.

See the option \hyperlink{cuts}{cuts} for an explanation on the different values.

Reference: E. Balas and M. Perregaard, A precise correspondence between lift-and-project cuts, simple disjunctive cuts, and mixed integer Gomory cuts for 0-1 programming. Math. Program., 94(203,Ser. B):221-245,2003.

\textsl{(default = off)}

\item[\label{mircuts}\hypertarget{mircuts}
{\textbf{mircuts (\slshape{string})}}]\hspace{1.0in}

Determines whether and when CBC should try to generate mixed integer rounding cuts.

See the option \hyperlink{cuts}{cuts} for an explanation on the different values.

Reference: H. Marchand and L. A. Wolsey, Aggregation and Mixed Integer Rounding to Solve MIPs, Operations Research, 49(3), (2001).

\textsl{(default = ifmove)}

\item[\label{twomircuts}\hypertarget{twomircuts}
{\textbf{twomircuts (\slshape{string})}}]\hspace{1.0in}

Determines whether and when CBC should try to generate two phase mixed integer rounding cuts.

See the option \hyperlink{cuts}{cuts} for an explanation on the different values.

Reference: S. Dash, and O. Guenluek, Valid Inequalities Based on Simple Mixed-integer Sets, to appear in Math. Programming.

\textsl{(default = root)}

\item[\label{probingcuts}\hypertarget{probingcuts}
{\textbf{probingcuts (\slshape{string})}}]\hspace{1.0in}

Determines whether and when CBC should try to generate cuts based on probing.

Additional to the values for the option \hyperlink{cuts}{cuts} three more values are possible here.

Reference: M. Savelsbergh, Preprocessing and Probing Techniques for Mixed Integer Programming Problems, ORSA Journal on Computing 6 (1994), 445.

\textsl{(default = ifmove)}
\begin{itemize}
\item[off] 
Turns off Probing.
\item[on] 
Turns on Probing and CBC will try it in the branch and cut tree (see the option \hyperlink{cutdepth}{cutdepth} how to fine tune this behaviour).
\item[root] 
Let CBC do Probing only at the root node.
\item[ifmove] 
Let CBC do Probing in the tree if it looks as if it is doing some good and moves the objective value.
\item[forceon] 
Turns on Probing and forces CBC to do Probing at every node.
\item[forceonbut] 
Turns on Probing and forces CBC to call the cut generator at every node, but does only probing, not strengthening etc.
\item[forceonstrong] 
If CBC is forced to turn Probing on at every node (by setting this option to force), but this generator produces no cuts, then it is actually turned on only weakly (i.e., just every now and then).
Setting forceonstrong forces CBC strongly to do probing at every node.
\item[forceonbutstrong] 
This is like forceonstrong, but does only probing (column fixing) and turns off row strengthening, so the matrix will not change inside the branch and bound.
\end{itemize}

\item[\label{reduceandsplitcuts}\hypertarget{reduceandsplitcuts}
{\textbf{reduceandsplitcuts (\slshape{string})}}]\hspace{1.0in}

Determines whether and when CBC should try to generate reduced and split cuts.

See the option \hyperlink{cuts}{cuts} for an explanation on the different values.

Reduce and split cuts are variants of Gomory cuts.
Starting from the current optimal tableau, linear combinations of the rows of the current optimal simplex tableau are used for generating Gomory cuts.
The choice of the linear combinations is driven by the objective of reducing the coefficients of the non basic continuous variables in the resulting row.

Reference: K. Anderson, G. Cornuejols, and Yanjun Li, Reduce-and-Split Cuts: Improving the Performance of Mixed Integer Gomory Cuts, Management Science 51 (2005).

\textsl{(default = off)}

\item[\label{residualcapacitycuts}\hypertarget{residualcapacitycuts}
{\textbf{residualcapacitycuts (\slshape{string})}}]\hspace{1.0in}

Determines whether and when CBC should try to generate residual capacity cuts.

See the option \hyperlink{cuts}{cuts} for an explanation on the different values.

These inequalities are particularly useful for Network Design and Capacity Planning models.

References:\\
T.L. Magnanti, P. Mirchandani, and R. Vachani, The convex hull of two core capacitated network design problems, Math. Programming, 60 (1993), pp. 233-250.\\
A. Atamturk and D. Rajan, On splittable and unsplittable flow capacitated network design arc-set polyhedra, Math. Programming, 92 (2002), pp. 315-333.

\textsl{(default = off)}

\end{description}

\subsubsection{MIP Options for Heuristics}

\begin{description}

\item[\label{heuristics}\hypertarget{heuristics}
{\textbf{heuristics (\slshape{integer})}}]\hspace{1.0in}

This parameter can be used to switch on or off all heuristics, except for the local tree search as it dramatically alters the search.
Then you can set individual ones off or on.

\textsl{(default = 1)}
\begin{itemize}
\item[0] 
Turns all MIP heuristics off.
\item[1] 
Turns all MIP heuristics on (except \hyperlink{localtreesearch}{local tree search}).
\end{itemize}

\item[\label{combinesolutions}\hypertarget{combinesolutions}
{\textbf{combinesolutions (\slshape{integer})}}]\hspace{1.0in}

This parameter control the use of a heuristic which does branch and cut on the given problem by just using variables which have appeared in one or more solutions.
It is obviously only tried after two or more solutions.

\textsl{(default = 1)}
\begin{itemize}
\item[0] 
Turns the combine solutions heuristic off.
\item[1] 
Turns the combine solutions heuristic on.
\end{itemize}

\item[\label{dins}\hypertarget{dins}
{\textbf{dins (\slshape{integer})}}]\hspace{1.0in}

This parameter control the use of the distance induced neighborhood search heuristic.

\textsl{(default = 0)}
\begin{itemize}
\item[0] 
Turns the distance induced neighborhood search off.
\item[1] 
Turns the distance induced neighborhood search on.
\end{itemize}

\item[\label{divingrandom}\hypertarget{divingrandom}
{\textbf{divingrandom (\slshape{integer})}}]\hspace{1.0in}

This switches on a random diving heuristic at various times.

\textsl{(default = 0)}
\begin{itemize}
\item[0] .
Turns the random diving heuristics off.
\item[1] .
Turns the random diving heuristics on.
\end{itemize}

\item[\label{divingcoefficient}\hypertarget{divingcoefficient}
{\textbf{divingcoefficient (\slshape{integer})}}]\hspace{1.0in}

This switches on the coefficient diving heuristic.

\textsl{(default = 1)}
\begin{itemize}
\item[0] .
Turns the coefficient diving heuristics off.
\item[1] .
Turns the coefficient diving heuristics on.
\end{itemize}

\item[\label{divingfractional}\hypertarget{divingfractional}
{\textbf{divingfractional (\slshape{integer})}}]\hspace{1.0in}

This switches on the fractional diving heuristic.

\textsl{(default = 0)}
\begin{itemize}
\item[0] .
Turns the fractional diving heuristics off.
\item[1] .
Turns the fractional diving heuristics on.
\end{itemize}

\item[\label{divingguided}\hypertarget{divingguided}
{\textbf{divingguided (\slshape{integer})}}]\hspace{1.0in}

This switches on the guided diving heuristic.

\textsl{(default = 0)}
\begin{itemize}
\item[0] .
Turns the guided diving heuristics off.
\item[1] .
Turns the guided diving heuristics on.
\end{itemize}

\item[\label{divinglinesearch}\hypertarget{divinglinesearch}
{\textbf{divinglinesearch (\slshape{integer})}}]\hspace{1.0in}

This switches on the line search diving heuristic.

\textsl{(default = 0)}
\begin{itemize}
\item[0] .
Turns the line search diving heuristics off.
\item[1] .
Turns the linesearch diving heuristics on.
\end{itemize}

\item[\label{divingpseudocost}\hypertarget{divingpseudocost}
{\textbf{divingpseudocost (\slshape{integer})}}]\hspace{1.0in}

This switches on the pseudo costs diving heuristic.

\textsl{(default = 0)}
\begin{itemize}
\item[0] .
Turns the pseudo costs diving heuristics off.
\item[1] .
Turns the pseudo costs diving heuristics on.
\end{itemize}

\item[\label{divingvectorlength}\hypertarget{divingvectorlength}
{\textbf{divingvectorlength (\slshape{integer})}}]\hspace{1.0in}

This switches on the vector length diving heuristic.

\textsl{(default = 0)}
\begin{itemize}
\item[0] .
Turns the vector length diving heuristics off.
\item[1] .
Turns the vector length diving heuristics on.
\end{itemize}

\item[\label{feaspump}\hypertarget{feaspump}
{\textbf{feaspump (\slshape{integer})}}]\hspace{1.0in}

This parameter control the use of the feasibility pump heuristic at the root.

This is due to Fischetti and Lodi and uses a sequence of LPs to try and get an integer feasible solution.
Some fine tuning is available by the option \hyperlink{feaspump_passes}{feaspump\_passes}.
Reference: M. Fischetti, F. Glover, and A. Lodi, The feasibility pump, Math. Programming, 104 (2005), pp. 91-104.

\textsl{(default = 1)}
\begin{itemize}
\item[0] 
Turns the feasibility pump off.
\item[1] 
Turns the feasibility pump on.
\end{itemize}

\item[\label{feaspump_passes}\hypertarget{feaspump_passes}
{\textbf{feaspump\_passes (\slshape{integer})}}]\hspace{1.0in}

This fine tunes the feasibility pump heuristic by setting the number of passes.

\textsl{(default = 20)}

\item[\label{greedyheuristic}\hypertarget{greedyheuristic}
{\textbf{greedyheuristic (\slshape{string})}}]\hspace{1.0in}

This parameter control the use of a pair of greedy heuristic which will try to obtain a solution.
It may just fix a percentage of variables and then try a small branch and cut run.

\textsl{(default = on)}
\begin{itemize}
\item[off] 
Turns off the greedy heuristic.
\item[on] 
Turns on the greedy heuristic.
\item[root] 
Turns on the greedy heuristic only for the root node.
\end{itemize}

\item[\label{localtreesearch}\hypertarget{localtreesearch}
{\textbf{localtreesearch (\slshape{integer})}}]\hspace{1.0in}

This parameter control the use of a local search algorithm when a solution is found.

It is from Fischetti and Lodi and is not really a heuristic although it can be used as one (with limited functionality).
This heuristic is not controlled by the option \hyperlink{heuristics}{heuristics}.

Reference: M. Fischetti and A. Lodi, Local Branching, Math. Programming B, 98 (2003), pp. 23-47.

\textsl{(default = 0)}
\begin{itemize}
\item[0] 
Turns the local tree search off.
\item[1] 
Turns the local tree search on.
\end{itemize}

\item[\label{naiveheuristics}\hypertarget{naiveheuristics}
{\textbf{naiveheuristics (\slshape{integer})}}]\hspace{1.0in}

This parameter controls the use of some naive heuristics, e.g., fixing of all integers with costs to zero.<BR$>$

\textsl{(default = 0)}
\begin{itemize}
\item[0] .
Turns the naive heuristics off.
\item[1] .
Turns the naive heuristics on.
\end{itemize}

\item[\label{randomizedrounding}\hypertarget{randomizedrounding}
{\textbf{randomizedrounding (\slshape{integer})}}]\hspace{1.0in}

This parameter controls the use of the randomized rounding heuristic.

\textsl{(default = 0)}
\begin{itemize}
\item[0] .
Turns the randomized rounding heuristic off.
\item[1] .
Turns the randomized rounding heuristic on.
\end{itemize}

\item[\label{rens}\hypertarget{rens}
{\textbf{rens (\slshape{integer})}}]\hspace{1.0in}

This parameter controls the use of the relaxation enforced neighborhood search heuristic.<BR$>$

\textsl{(default = 0)}
\begin{itemize}
\item[0] .
Turns the relaxation enforced neighborhood search off.
\item[1] .
Turns the relaxation enforced neighborhood search on.
\end{itemize}

\item[\label{pivotandfix}\hypertarget{pivotandfix}
{\textbf{pivotandfix (\slshape{integer})}}]\hspace{1.0in}

This parameter controls the use of the pivot and fix heuristic.

\textsl{(default = 0)}
\begin{itemize}
\item[0] .
Turns the naive pivot and fix heuristic off.
\item[1] .
Turns the naive pivot and fix heuristic on.
\end{itemize}

\item[\label{rins}\hypertarget{rins}
{\textbf{rins (\slshape{integer})}}]\hspace{1.0in}

This parameter control the use of the relaxed induced neighborhood search heuristic.

This heuristic compares the current solution with the best incumbent, fixes all discrete variables with the same value, presolves the problem, and does a branch and bound for 200 nodes.

Reference: E. Danna, E. Rothberg, and C. Le Pape, Exploring relaxation induced neighborhoods to improve MIP solutions, Math. Programming, 102 (1) (2005), pp. 71-91.

\textsl{(default = 0)}
\begin{itemize}
\item[0] 
Turns the relaxed induced neighborhood search off.
\item[1] 
Turns the relaxed induced neighborhood search on.
\end{itemize}

\item[\label{roundingheuristic}\hypertarget{roundingheuristic}
{\textbf{roundingheuristic (\slshape{integer})}}]\hspace{1.0in}

This parameter control the use of a simple (but effective) rounding heuristic at each node of tree.

\textsl{(default = 1)}
\begin{itemize}
\item[0] 
Turns the rounding heuristic off.
\item[1] 
Turns the rounding heuristic on.
\end{itemize}

\item[\label{vubheuristic}\hypertarget{vubheuristic}
{\textbf{vubheuristic (\slshape{integer})}}]\hspace{1.0in}

This parameter control the use of the VUB heuristic.
If it is set (between -2 and 20), Cbc will try  and fix some integer variables

\textsl{(default = -1)}

\end{description}

% \subsubsection{MIP Options for the GAMS Branch Cut and Heuristic Facility}
% 
% \begin{description}
% \item[\label{usercutcall}\hypertarget{usercutcall}
% {\textbf{usercutcall (\slshape{string})}}]\hspace{1.0in}
% 
% The GAMS command line (minus the gams executable name) to call the cut generator.
% 
% 
% \item[\label{usercutfirst}\hypertarget{usercutfirst}
% {\textbf{usercutfirst (\slshape{integer})}}]\hspace{1.0in}
% 
% Calls the cut generator for the first $n$ nodes.
% 
% \textsl{(default = 10)}
% 
% \item[\label{usercutfreq}\hypertarget{usercutfreq}
% {\textbf{usercutfreq (\slshape{integer})}}]\hspace{1.0in}
% 
% Determines the frequency of the cut generator model calls.
% 
% \textsl{(default = 10)}
% 
% \item[\label{usercutinterval}\hypertarget{usercutinterval}
% {\textbf{usercutinterval (\slshape{integer})}}]\hspace{1.0in}
% 
% Determines the interval when to apply the multiplier for the frequency of the cut generator model calls.
% See userheurinterval for details.
% 
% \textsl{(default = 100)}
% 
% \item[\label{usercutmult}\hypertarget{usercutmult}
% {\textbf{usercutmult (\slshape{integer})}}]\hspace{1.0in}
% 
% Determines the multiplier for the frequency of the cut generator model calls.
% 
% \textsl{(default = 2)}
% 
% \item[\label{usercutnewint}\hypertarget{usercutnewint}
% {\textbf{usercutnewint (\slshape{integer})}}]\hspace{1.0in}
% 
% Calls the cut generator if the solver found a new integer feasible solution.
% 
% \textsl{(default = 1)}
% \begin{itemize}
% \item[0] .
% Do not call cut generator because a new integer feasible solution is found.
% \item[1] .
% Let CBC call the cut generator if a new integer feasible solution is found.
% \end{itemize}
% 
% \item[\label{usergdxin}\hypertarget{usergdxin}
% {\textbf{usergdxin (\slshape{string})}}]\hspace{1.0in}
% 
% The name of the GDX file read back into CBC.
% 
% \textsl{(default =} \verb=bchin.gdx=)
% 
% \item[\label{usergdxname}\hypertarget{usergdxname}
% {\textbf{usergdxname (\slshape{string})}}]\hspace{1.0in}
% 
% The name of the GDX file exported from the solver with the solution at the node.
% 
% \textsl{(default =} \verb=bchout.gdx=)
% 
% \item[\label{usergdxnameinc}\hypertarget{usergdxnameinc}
% {\textbf{usergdxnameinc (\slshape{string})}}]\hspace{1.0in}
% 
% The name of the GDX file exported from the solver with the incumbent solution.
% 
% \textsl{(default =} \verb=bchout_i.gdx=)
% 
% \item[\label{usergdxprefix}\hypertarget{usergdxprefix}
% {\textbf{usergdxprefix (\slshape{string})}}]\hspace{1.0in}
% 
% Prefixes to use for usergdxin, usergdxname, and usergdxnameinc.
% 
% 
% \item[\label{userheurcall}\hypertarget{userheurcall}
% {\textbf{userheurcall (\slshape{string})}}]\hspace{1.0in}
% 
% The GAMS command line (minus the gams executable name) to call the heuristic.
% 
% 
% \item[\label{userheurfirst}\hypertarget{userheurfirst}
% {\textbf{userheurfirst (\slshape{integer})}}]\hspace{1.0in}
% 
% Calls the heuristic for the first n nodes.
% 
% \textsl{(default = 10)}
% 
% \item[\label{userheurfreq}\hypertarget{userheurfreq}
% {\textbf{userheurfreq (\slshape{integer})}}]\hspace{1.0in}
% 
% Determines the frequency of the heuristic model calls.
% 
% \textsl{(default = 10)}
% 
% \item[\label{userheurinterval}\hypertarget{userheurinterval}
% {\textbf{userheurinterval (\slshape{integer})}}]\hspace{1.0in}
% 
% Determines the interval when to apply the multiplier for the frequency of the heuristic model calls.
% For example, for the first 100 (userheurinterval) nodes, the solver call every 10th (userheurfreq) node
% the heuristic.
% After 100 nodes, the frequency gets multiplied by 10 (userheurmult), so that for the next 100 node the solver calls the heuristic every 20th node.
% For nodes 200-300, the heuristic get called every 40th node, for nodes 300-400 every 80th node and after node 400 every 100th node.
% 
% \textsl{(default = 100)}
% 
% \item[\label{userheurmult}\hypertarget{userheurmult}
% {\textbf{userheurmult (\slshape{integer})}}]\hspace{1.0in}
% 
% Determines the multiplier for the frequency of the heuristic model calls.
% 
% \textsl{(default = 2)}
% 
% \item[\label{userheurnewint}\hypertarget{userheurnewint}
% {\textbf{userheurnewint (\slshape{integer})}}]\hspace{1.0in}
% 
% Calls the heuristic if the solver found a new integer feasible solution.
% 
% \textsl{(default = 1)}
% \begin{itemize}
% \item[0] .
% Do not call heuristic because a new integer feasible solution is found.
% \item[1] .
% Let CBC call the heuristic if a new integer feasible solution is found.
% \end{itemize}
% 
% \item[\label{userheurobjfirst}\hypertarget{userheurobjfirst}
% {\textbf{userheurobjfirst (\slshape{integer})}}]\hspace{1.0in}
% 
% Similar to userheurfirst but only calls the heuristic if the relaxed objective value promises a significant improvement of the current incumbent, i.e., the LP value of the node has to be closer to the best bound than the current incumbent.
% 
% \textsl{(default = 0)}
% 
% \item[\label{userjobid}\hypertarget{userjobid}
% {\textbf{userjobid (\slshape{string})}}]\hspace{1.0in}
% 
% Postfixes to use for gdxname, gdxnameinc, and gdxin.
% 
% 
% \item[\label{userkeep}\hypertarget{userkeep}
% {\textbf{userkeep (\slshape{integer})}}]\hspace{1.0in}
% 
% Calls gamskeep instead of gams
% 
% \textsl{(default = 0)}
% 
% \end{description}




\section{CoinGlpk}

GAMS/CoinGlpk brings the open source LP/MIP solver Glpk from the GNU Open Software foundation to the broad audience of GAMS users.

The code has been written primarily by A. Makhorin.
The interface uses the OSI Glpk interface written by Vivian De Smedt, Braden Hunsaker, and Lou Hafer.

For more information we refer to
\begin{itemize}
\item the Glpk web site \texttt{http://www.gnu.org/software/glpk/glpk.html} and
\item the Osi web site \texttt{https://projects.coin-or.org/Osi}.
\end{itemize}
Most of the Glpk documentation in the section was taken from the Glpk manual.

\subsection{Model requirements}

Glpk supports continuous, binary, and integer variables, but no special ordered sets, semi-continuous or semi-integer variables (see chapter 17.1 of the GAMS User's Guide).
Also branching priorities are not supported.

\subsection{Usage of CoinGlpk}

The following statement can be used inside your GAMS program to specify using CoinGlpk
\begin{verbatim}
  Option LP = CoinGlpk;     { or MIP or RMIP }
\end{verbatim}

The above statement should appear before the Solve statement.
If CoinGlpk was specified as the default solver during GAMS installation, the above statement is not necessary.

The GAMS/CoinGlpk options file consists of one option or comment per line.
An asterisk (*) at the beginning of a line causes the entire line to be ignored.
Otherwise, the line will be interpreted as an option name and value separated by any amount of white space (blanks or tabs).
Following is an example options file coincbc.opt.
\begin{verbatim}
  factorization givens
  initbasis standard
\end{verbatim}
It will cause CoinGlpk to use Givens rotation updates for the factorization and to use a standard initial basis. (This option setting might help to avoid numerical difficulties in some cases.)

\subsection{Summary of CoinGlpk Options}

Among all Glpk options, the following GAMS parameters are currently supported in CoinGlpk:
\hyperlink{glpkreslim}{reslim}, \hyperlink{glpkiterlim}{iterlim}, \hyperlink{glpkoptcr}{optcr}.

Currently CoinGlpk understands the following options:
\begin{tabbing}
\hspace {1.0in} \= \\
\hyperlink{glpkwritemps}
{writemps} \> create MPS file for problem \\
\hyperlink{glpkstartalg}
{startalg} \> LP solver for root node \\
\hyperlink{scaling}
{scaling} \> scaling method \\
\hyperlink{pricing}
{pricing} \> pricing method \\
\hyperlink{tol_dual}
{tol\_dual} \> dual feasibility tolerance \\
\hyperlink{tol_primal}
{tol\_primal} \> primal feasibility tolerance \\
\hyperlink{tol_integer}
{tol\_integer} \> integer feasibility tolerance \\
\hyperlink{backtracking}
{backtracking} \> backtracking heuristic \\
\hyperlink{glpkcuts}
{cuts} \> generation of cuts for root problem \\
\hyperlink{reslim_fixedrun}
{reslim_fixedrun} \> resource limit for solve with fixed discrete variables \\
\hyperlink{glpkreslim}
{reslim} \> resource limit \\
\hyperlink{glpkiterlim}
{iterlim} \> iteration limit \\
\hyperlink{glpkoptcr}
{optcr} \> relative stopping tolerance \\
\end{tabbing}


\subsection{Detailed Descriptions of CoinGlpk Options}

\begin{description}

\item[\label{glpkwritemps}\hypertarget{glpkwritemps}
{\textbf{writemps (\slshape{string})}}]\hspace{1.0in}

Write an MPS problem file.
The parameter value is the name of the MPS file.


\item[\label{glpkstartalg}\hypertarget{glpkstartalg}
{\textbf{startalg (\slshape{string})}}]\hspace{1.0in}

This option determines whether a primal or dual simplex algorithm should be used to solve the root node.

\textsl{(default = primal)}
\begin{itemize}
\item[primal] use the primal simplex algorithm for the root node
\item[dual] use the dual simplex algorithm for the root node
\end{itemize}

\item[\label{glpkscaling}\hypertarget{glpkscaling}
{\textbf{scaling (\slshape{string})}}]\hspace{1.0in}

This option determines the method how the constraint matrix is scaled.

\textsl{(default = equilibrium)}
\begin{tabbing}
\hspace{1.1in} \= \\
off \> no scaling \\
equilibrium \> equilibrium scaling \\
mean \> geometric mean scaling \\
meanequilibrium \> geometric mean scaling then equilibrium scaling
\end{tabbing}

\item[\label{pricing}\hypertarget{pricing}
{\textbf{pricing (\slshape{string})}}]\hspace{1.0in}

Sets the pricing method for both primal and dual simplex.

\textsl{(default = textbook)}
\begin{tabbing}
\hspace{1in} \= \\
textbook \> textbook pricing \\
steepestedge \> steepest edge pricing
\end{tabbing}

\item[\label{glpktol_dual}\hypertarget{glpktol_dual}
{\textbf{tol\_dual (\slshape{real})}}]\hspace{1.0in}

Absolute tolerance used to check if the current basis solution is dual feasible.
(Glpk manual: Do not change this parameter without detailed understanding its purpose.)

\textsl{(default = 1e-7)}

\item[\label{glpktol_primal}\hypertarget{glpktol_primal}
{\textbf{tol\_primal (\slshape{real})}}]\hspace{1.0in}

Relative tolerance used to check if the current basis solution is primal feasible.
(Glpk manual: Do not change this parameter without detailed understanding its purpose.)

\textsl{(default = 1e-7)}

\item[\label{glpktol_integer}\hypertarget{glpktol_integer}
{\textbf{tol\_integer (\slshape{real})}}]\hspace{1.0in}

Absolute tolerance used to check if the current basis solution is integer feasible.
(Glpk manual: Do not change this parameter without detailed understanding its purpose.)

\textsl{(default = 1e-5)}

\item[\label{backtracking}\hypertarget{backtracking}
{\textbf{backtracking (\slshape{string})}}]\hspace{1.0in}

Determines which method to use for the backtracking heuristic.

\textsl{(default = bestprojection)}
\begin{tabbing}
\hspace{1in} \= \\
depthfirst \> depth first search \\
breadthfirst \> breadth first search \\
bestprojection \> using best projection heuristic
\end{tabbing}

\item[\label{glpkcuts}\hypertarget{glpkcuts}
{\textbf{cuts (\slshape{integer})}}]\hspace{1.0in}

Regulates the use of cut generator routines for the root problem.\\
If set to automatic, then the parameters for the single cut generators determine which are added.

\textsl{(default = 0)}
\begin{itemize}
\item[-1] no cuts will be generated
\item[0] automatic
\item[1] cuts from all available cut classes will be generated
\end{itemize}

\item[\label{glpkgomorycuts}\hypertarget{glpkgomorycuts}
{\textbf{gomorycuts (\slshape{integer})}}]\hspace{1.0in}

This options switches the generation of mixed integer Gomory Cuts on.

\textsl{(default = 1)}
\begin{itemize}
\item[0] don't add gomory cuts
\item[1] add gomory cuts
\end{itemize}


\item[\label{glpkcliquecuts}\hypertarget{glpkcliquecuts}
{\textbf{cliquecuts (\slshape{integer})}}]\hspace{1.0in}

This options switches the generation of clique cuts on.

\textsl{(default = 1)}
\begin{itemize}
\item[0] don't add clique cuts
\item[1] add clique cuts
\end{itemize}


\item[\label{glpkcovercuts}\hypertarget{glpkcovercuts}
{\textbf{covercuts (\slshape{integer})}}]\hspace{1.0in}

This options switches the generation of mixed cover cuts on.

\textsl{(default = 1)}
\begin{itemize}
\item[0] don't add mixed cover cuts
\item[1] add mixed cover cuts
\end{itemize}


\item[\label{reslim_fixedrun}\hypertarget{reslim_fixedrun}
{\textbf{reslim\_fixedrun (\slshape{real})}}]\hspace{1.0in}

Maximum time in seconds for solving the MIP with fixed discrete variables.

\textsl{(default = 1000)}

\item[\label{glpkreslim}\hypertarget{glpkreslim}
{\textbf{reslim (\slshape{real})}}]\hspace{1.0in}

Maximum time in seconds.

\textsl{(default = GAMS reslim)}

\item[\label{glpkiterlim}\hypertarget{glpkiterlim}
{\textbf{iterlim (\slshape{integer})}}]\hspace{1.0in}

Maximum number of iterations.

\textsl{(default = GAMS iterlim)}

\item[\label{glpkoptcr}\hypertarget{glpkoptcr}
{\textbf{optcr (\slshape{real})}}]\hspace{1.0in}

Relative optimality criterion for a MIP.
Glpk uses this parameter as relative tolerance to check if the value of the objective function is not better than in the best known integer feasible solution.

\textsl{(default = GAMS optcr)}
\end{description}




\section{CoinIpopt}

GAMS/CoinIpopt brings the open source NLP solver Ipopt from the COIN-OR foundation to the broad audience of GAMS users.

Ipopt (\textbf{I}nterior \textbf{P}oint \textbf{Opt}imizer) is an open-source solver for large-scale nonlinear programming.
The code has been written primarily by Andreas W\"achter, who is the COIN-OR project leader for Ipopt.

For more information we refer to
\begin{itemize}
\item the Ipopt web site \texttt{https://projects.coin-or.org/Ipopt} and
\item the \emph{implementation paper} A. W\"achter and L. T. Biegler, On the Implementation of a Primal-Dual Interior Point Filter Line Search Algorithm for Large-Scale Nonlinear Programming, \emph{Mathematical Programming} 106(1), pp. 25-57, 2006.
\end{itemize}
Most of the Ipopt documentation in the section was taken from the Ipopt manual available on the Ipopt web site.

\subsection{Model requirements}

Ipopt can handle nonlinear programming models which functions can be nonconvex, but should be twice continuously differentiable.
GAMS/CoinIpopt now supports user-defined scalings of variables and equations using the .scale suffix and scaleopt option (see chapter 17.2 of the GAMS User's Guide).

\subsection{Usage of CoinIpopt}

The following statement can be used inside your GAMS program to specify using CoinIpopt
\begin{verbatim}
  Option NLP = CoinIpopt;     { or LP, RMIP, DNLP, RMINLP, QCP, RMIQCP }
\end{verbatim}

The above statement should appear before the Solve statement.
If CoinIpopt was specified as the default solver during GAMS installation, the above statement is not necessary.

\subsection{The linear solver in Ipopt}
\label{ipoptlinearsolver}
\hypertarget{ipoptlinearsolver}{}

The performance and robustness of Ipopt on larger models heavily relies on the used solver for sparse symmetric indefinite linear systems.
GAMS/CoinIpopt includes the multifrontal massively parallel sparse direct solver MUMPS 4.8.1 (\texttt{http://graal.ens-lyon.fr/MUMPS}).
The user can provide the Parallel Sparse Direct Solver PARDISO or routines from the Harwell Subroutine Library (HSL) as shared (or dynamic) libraries to replace MUMPS.

\subsubsection{Using Harwell Subroutine Library routines with GAMS/CoinIpopt}

If you have routines from the HSL available and want Gams/CoinIpopt to use them, you can provide them in a shared library.
GAMS/CoinIpopt can use MA27, MA28, MA57, and MC19.
By telling Ipopt to use one of these routines (see options linear\_solver, linear\_system\_scaling, nlp\_scaling\_method, dependency\_detector) GAMS/CoinIpopt attempts to load the required routines from the library libhsl.so (Unix-Systems), libhsl.dylib (MacOS X), or libhsl.dll (Windows).
You can also specify the path and name for this library with the option hsl\_library.

For example,
\begin{verbatim}
 linear_solver ma27
 hsl_library /my/path/to/the/hsllib/myhsllib.so
\end{verbatim}
tells Ipopt to use the linear solver MA27 from the HSL library \verb=myhsllib.so= under the specified path.

The HSL routines MA27, MA28, and MC19 are available at \texttt{http://www.cse.clrc.ac.uk/nag/hsl}.
Note that it is your responsibility to ensure that you are entitled to download and use these routines!
You can build a shared library using the ThirdParty/HSL project at COIN-OR.

\subsubsection{Using PARDISO routines with GAMS/CoinIpopt}

If you have the linear solver PARDISO available, then you can tell Gams/CoinIpopt to use by setting the linear\_solver option to pardiso.
GAMS/CoinIpopt then attempts to load the library libpardiso.so (Unix-Systems), libpardiso.dylib (MacOS X), or libpardiso.dll (Windows).
You can also specify the path and name for this library with the option pardiso\_library.

For example,
\begin{verbatim}
 linear_solver pardiso
 pardiso_library /my/path/to/the/pardisolib/mypardisolib.so
\end{verbatim}
tells Ipopt to use the linear solver PARDISO from the library \verb=mypardisolib.so= under the specified path.

PARDISO is available as compiled shared library for several platforms at \texttt{http://www.pardiso-project.org}.
Note that it is your responsibility to ensure that you are entitled to download and use this package!

\subsection{Output}

This section describes the standard Ipopt console output.
The output is designed to provide a quick summary of each iteration as Ipopt solves the problem.

Before Ipopt starts to solve the problem, it displays the problem statistics (number of nonzero-elements in the matrices, number of variables, etc.).
Note that if you have fixed variables (both upper and lower bounds are equal), Ipopt may remove these variables from the problem internally and not include them in the problem statistics.

Following the problem statistics, Ipopt will begin to solve the problem and you will see output resembling the following,
\begin{verbatim}
iter    objective    inf_pr   inf_du lg(mu)  ||d||  lg(rg) alpha_du alpha_pr  ls
   0  1.6109693e+01 1.12e+01 5.28e-01   0.0 0.00e+00    -  0.00e+00 0.00e+00   0
   1  1.8029749e+01 9.90e-01 6.62e+01   0.1 2.05e+00    -  2.14e-01 1.00e+00f  1
   2  1.8719906e+01 1.25e-02 9.04e+00  -2.2 5.94e-02   2.0 8.04e-01 1.00e+00h  1
\end{verbatim}
and the columns of output are defined as
\begin{description}
\item[iter]
The current iteration count.
This includes regular iterations and iterations while in restoration phase.
If the algorithm is in the restoration phase, the letter r' will be appended to the iteration number.
\item[objective]
The unscaled objective value at the current point.
During the restoration phase, this value remains the unscaled objective value for the original problem.
\item[inf\_pr]
The scaled primal infeasibility at the current point.
During the restoration phase, this value is the primal infeasibility of the original problem at the current point.
\item[inf\_du]
The scaled dual infeasibility at the current point.
During the restoration phase, this is the value of the dual infeasibility for the restoration phase problem.
\item[lg(mu)]
$\log_{10}$ of the value of the barrier parameter mu.
\item[$\Vert d\Vert$]
The infinity norm (max) of the primal step (for the original variables $x$ and the internal slack variables $s$).
During the restoration phase, this value includes the values of additional variables, $p$ and $n$.
\item[lg(rg)]
$\log_{10}$ of the value of the regularization term for the Hessian of the Lagrangian in the augmented system.
\item[alpha\_du]
The stepsize for the dual variables.
\item[alpha\_pr]
The stepsize for the primal variables.
\item[ls]
The number of backtracking line search steps.
\end{description}

When the algorithm terminates, IPOPT will output a message to the screen based on the return status of the call to Optimize.
The following is a list of the possible output messages to the console, and a brief description.

\begin{description}
\item[Optimal Solution Found.] ~

    This message indicates that IPOPT found a (locally) optimal point within the desired tolerances.

\item[Solved To Acceptable Level.] ~

    This indicates that the algorithm did not converge to the ``desired'' tolerances, but that it was able to obtain a point satisfying the ``acceptable'' tolerance level as specified by acceptable-* options.
    This may happen if the desired tolerances are too small for the current problem.

\item[Converged to a point of local infeasibility. Problem may be infeasible.] ~

    The restoration phase converged to a point that is a minimizer for the constraint violation (in the $\ell_1$-norm), but is not feasible for the original problem.
    This indicates that the problem may be infeasible (or at least that the algorithm is stuck at a locally infeasible point).
    The returned point (the minimizer of the constraint violation) might help you to find which constraint is causing the problem.
    If you believe that the NLP is feasible, it might help to start the optimization from a different point.

\item[Search Direction is becoming Too Small.] ~

    This indicates that Ipopt is calculating very small step sizes and making very little progress.
    This could happen if the problem has been solved to the best numerical accuracy possible given the current scaling.

\item[Iterates divering; problem might be unbounded.] ~

    This message is printed if the max-norm of the iterates becomes larger than the value of the option diverging\_iterates\_tol.
    This can happen if the problem is unbounded below and the iterates are diverging.

\item[Stopping optimization at current point as requested by user.] ~

    This message is printed if either the time limit or the domain violation limit is reached.

\item[Maximum Number of Iterations Exceeded.] ~

    This indicates that Ipopt has exceeded the maximum number of iterations as specified by the option max\_iter.

\item[Restoration Failed!] ~

    This indicates that the restoration phase failed to find a feasible point that was acceptable to the filter line search for the original problem.
    This could happen if the problem is highly degenerate or does not satisfy the constraint qualification, or if an external function in GAMS provides incorrect derivative information.

\item[Error in step computation (regularization becomes too large?)!] ~

    This messages is printed if Ipopt is unable to compute a search direction, despite several attempts to modify the iteration matrix.
    Usually, the value of the regularization parameter then becomes too large.

\item[Problem has too few degrees of freedom.] ~

    This indicates that your problem, as specified, has too few degrees of freedom.
    This can happen if you have too many equality constraints, or if you fix too many variables (Ipopt removes fixed variables).

\item[Not enough memory.] ~

    An error occurred while trying to allocate memory.
    The problem may be too large for your current memory and swap configuration.

\item[INTERNAL ERROR: Unknown SolverReturn value - Notify IPOPT Authors.] ~

    An unknown internal error has occurred. Please notify the authors of the GAMS/CoinIpopt link or IPOPT (refer to \url{https://projects.coin-or.org/GAMSlinks} or \url{https://projects.coin-or.org/Ipopt}).
\end{description}


\subsection{Specification of CoinIpopt Options}
\label{sub:ipoptoptionspec}

Ipopt has many options that can be adjusted for the algorithm (see Section \ref{sub:ipoptoptions}).
Options are all identified by a string name, and their values can be of one of three types: Number (real), Integer, or String.
Number options are used for things like tolerances, integer options are used for things like maximum number of iterations, and string options are used for setting algorithm details, like the NLP scaling method.
Options can be set by creating a \texttt{ipopt.opt} file in the directory you are executing Ipopt.

The \texttt{ipopt.opt} file is read line by line and each line should contain the option name, followed by whitespace, and then the value.
Comments can be included with the \# symbol. Don't forget to ensure you have a newline at the end of the file. For example,
\begin{verbatim}
# This is a comment

# Turn off the NLP scaling
nlp_scaling_method none

# Change the initial barrier parameter
mu_init 1e-2

# Set the max number of iterations
max_iter 500
\end{verbatim}
is a valid \texttt{ipopt.opt} file.

% You can print the documentation for all Ipopt options by using the option
% \begin{verbatim}
% print_options_documentation yes
% \end{verbatim}
% and running IPOPT.
% This will output all of the options documentation to the console.

Note, that GAMS/CoinIpopt overwrites the Ipopt default setting for the parameters bound\_relax\_factor (set to 0.0) and mu\_strategy (set to adaptive).
You can change these values by specifying these options in your Ipopt options file.

GAMS/CoinIpopt understand currently the following GAMS parameters: reslim (time limit), iterlim (iteration limit), domlim (domain violation limit).
You can set them either on the command line, e.g. \verb+iterlim=500+, or inside your GAMS program, e.g. \verb+Option iterlim=500;+.

\subsection{Detailed Description of CoinIpopt Options}
\label{sub:ipoptoptions}
\printoptioncategory{Barrier Parameter Update}
\printoption{adaptive\_mu\_globalization}%
{\ttfamily kkt-error, obj-constr-filter, never-monotone-mode}%
{obj-constr-filter}%
{Globalization strategy for the adaptive mu selection mode.\\
To achieve global convergence of the adaptive version, the algorithm has to switch to the monotone mode (Fiacco-McCormick approach) when convergence does not seem to appear.  This option sets the criterion used to decide when to do this switch. (Only used if option "mu\_strategy" is chosen as "adaptive".)}%
{\begin{list}{}{
\setlength{\parsep}{0em}
\setlength{\leftmargin}{5ex}
\setlength{\labelwidth}{2ex}
\setlength{\itemindent}{0ex}
\setlength{\topsep}{0pt}}
\item[\texttt{kkt-error}] nonmonotone decrease of kkt-error
\item[\texttt{obj-constr-filter}] 2-dim filter for objective and constraint violation
\item[\texttt{never-monotone-mode}] disables globalization
\end{list}
}

\printoption{adaptive\_mu\_kkt\_norm\_type}%
{\ttfamily 1-norm, 2-norm-squared, max-norm, 2-norm}%
{2-norm-squared}%
{Norm used for the KKT error in the adaptive mu globalization strategies.\\
When computing the KKT error for the globalization strategies, the norm to be used is specified with this option. Note, this options is also used in the QualityFunctionMuOracle.}%
{\begin{list}{}{
\setlength{\parsep}{0em}
\setlength{\leftmargin}{5ex}
\setlength{\labelwidth}{2ex}
\setlength{\itemindent}{0ex}
\setlength{\topsep}{0pt}}
\item[\texttt{1-norm}] use the 1-norm (abs sum)
\item[\texttt{2-norm-squared}] use the 2-norm squared (sum of squares)
\item[\texttt{max-norm}] use the infinity norm (max)
\item[\texttt{2-norm}] use 2-norm
\end{list}
}

\printoption{adaptive\_mu\_kkterror\_red\_fact}%
{$0<\textrm{real}<1$}%
{$0.9999$}%
{Sufficient decrease factor for "kkt-error" globalization strategy.\\
For the "kkt-error" based globalization strategy, the error must decrease by this factor to be deemed sufficient decrease.}%
{}

\printoption{adaptive\_mu\_kkterror\_red\_iters}%
{$0\leq\textrm{integer}$}%
{$4$}%
{Maximum number of iterations requiring sufficient progress.\\
For the "kkt-error" based globalization strategy, sufficient progress must be made for "adaptive\_mu\_kkterror\_red\_iters" iterations. If this number of iterations is exceeded, the globalization strategy switches to the monotone mode.}%
{}

\printoption{adaptive\_mu\_monotone\_init\_factor}%
{$0<\textrm{real}$}%
{$0.8$}%
{Determines the initial value of the barrier parameter when switching to the monotone mode.\\
When the globalization strategy for the adaptive barrier algorithm switches to the monotone mode and fixed\_mu\_oracle is chosen as "average\_compl", the barrier parameter is set to the current average complementarity times the value of "adaptive\_mu\_monotone\_init\_factor".}%
{}

\printoption{adaptive\_mu\_restore\_previous\_iterate}%
{\ttfamily no, yes}%
{no}%
{Indicates if the previous iterate should be restored if the monotone mode is entered.\\
When the globalization strategy for the adaptive barrier algorithm switches to the monotone mode, it can either start from the most recent iterate (no), or from the last iterate that was accepted (yes).}%
{\begin{list}{}{
\setlength{\parsep}{0em}
\setlength{\leftmargin}{5ex}
\setlength{\labelwidth}{2ex}
\setlength{\itemindent}{0ex}
\setlength{\topsep}{0pt}}
\item[\texttt{no}] don't restore accepted iterate
\item[\texttt{yes}] restore accepted iterate
\end{list}
}

\printoption{barrier\_tol\_factor}%
{$0<\textrm{real}$}%
{$10$}%
{Factor for mu in barrier stop test.\\
The convergence tolerance for each barrier problem in the monotone mode is the value of the barrier parameter times "barrier\_tol\_factor". This option is also used in the adaptive mu strategy during the monotone mode. (This is kappa\_epsilon in implementation paper).}%
{}

\printoption{filter\_margin\_fact}%
{$0<\textrm{real}<1$}%
{$10^{- 5}$}%
{Factor determining width of margin for obj-constr-filter adaptive globalization strategy.\\
When using the adaptive globalization strategy, "obj-constr-filter", sufficient progress for a filter entry is defined as follows: (new obj) $<$ (filter obj) - filter\_margin\_fact*(new constr-viol) OR (new constr-viol) $<$ (filter constr-viol) - filter\_margin\_fact*(new constr-viol).  For the description of the "kkt-error-filter" option see "filter\_max\_margin".}%
{}

\printoption{filter\_max\_margin}%
{$0<\textrm{real}$}%
{$1$}%
{Maximum width of margin in obj-constr-filter adaptive globalization strategy.}%
{}

\printoption{fixed\_mu\_oracle}%
{\ttfamily probing, loqo, quality-function, average\_compl}%
{average\_compl}%
{Oracle for the barrier parameter when switching to fixed mode.\\
Determines how the first value of the barrier parameter should be computed when switching to the "monotone mode" in the adaptive strategy. (Only considered if "adaptive" is selected for option "mu\_strategy".)}%
{\begin{list}{}{
\setlength{\parsep}{0em}
\setlength{\leftmargin}{5ex}
\setlength{\labelwidth}{2ex}
\setlength{\itemindent}{0ex}
\setlength{\topsep}{0pt}}
\item[\texttt{probing}] Mehrotra's probing heuristic
\item[\texttt{loqo}] LOQO's centrality rule
\item[\texttt{quality-function}] minimize a quality function
\item[\texttt{average\_compl}] base on current average complementarity
\end{list}
}

\printoption{mu\_allow\_fast\_monotone\_decrease}%
{\ttfamily no, yes}%
{yes}%
{Allow skipping of barrier problem if barrier test is already met.\\
If set to "no", the algorithm enforces at least one iteration per barrier problem, even if the barrier test is already met for the updated barrier parameter.}%
{\begin{list}{}{
\setlength{\parsep}{0em}
\setlength{\leftmargin}{5ex}
\setlength{\labelwidth}{2ex}
\setlength{\itemindent}{0ex}
\setlength{\topsep}{0pt}}
\item[\texttt{no}] Take at least one iteration per barrier problem
\item[\texttt{yes}] Allow fast decrease of mu if barrier test it met
\end{list}
}

\printoption{mu\_init}%
{$0<\textrm{real}$}%
{$0.1$}%
{Initial value for the barrier parameter.\\
This option determines the initial value for the barrier parameter (mu).  It is only relevant in the monotone, Fiacco-McCormick version of the algorithm. (i.e., if "mu\_strategy" is chosen as "monotone")}%
{}

\printoption{mu\_linear\_decrease\_factor}%
{$0<\textrm{real}<1$}%
{$0.2$}%
{Determines linear decrease rate of barrier parameter.\\
For the Fiacco-McCormick update procedure the new barrier parameter mu is obtained by taking the minimum of mu*"mu\_linear\_decrease\_factor" and mu\^"superlinear\_decrease\_power".  (This is kappa\_mu in implementation paper.) This option is also used in the adaptive mu strategy during the monotone mode.}%
{}

\printoption{mu\_max}%
{$0<\textrm{real}$}%
{$100000$}%
{Maximum value for barrier parameter.\\
This option specifies an upper bound on the barrier parameter in the adaptive mu selection mode.  If this option is set, it overwrites the effect of mu\_max\_fact. (Only used if option "mu\_strategy" is chosen as "adaptive".)}%
{}

\printoption{mu\_max\_fact}%
{$0<\textrm{real}$}%
{$1000$}%
{Factor for initialization of maximum value for barrier parameter.\\
This option determines the upper bound on the barrier parameter.  This upper bound is computed as the average complementarity at the initial point times the value of this option. (Only used if option "mu\_strategy" is chosen as "adaptive".)}%
{}

\printoption{mu\_min}%
{$0<\textrm{real}$}%
{$10^{-11}$}%
{Minimum value for barrier parameter.\\
This option specifies the lower bound on the barrier parameter in the adaptive mu selection mode. By default, it is set to the minimum of 1e-11 and min("tol","compl\_inf\_tol")/("barrier\_tol\_factor"+1), which should be a reasonable value. (Only used if option "mu\_strategy" is chosen as "adaptive".)}%
{}

\printoption{mu\_oracle}%
{\ttfamily probing, loqo, quality-function}%
{quality-function}%
{Oracle for a new barrier parameter in the adaptive strategy.\\
Determines how a new barrier parameter is computed in each "free-mode" iteration of the adaptive barrier parameter strategy. (Only considered if "adaptive" is selected for option "mu\_strategy").}%
{\begin{list}{}{
\setlength{\parsep}{0em}
\setlength{\leftmargin}{5ex}
\setlength{\labelwidth}{2ex}
\setlength{\itemindent}{0ex}
\setlength{\topsep}{0pt}}
\item[\texttt{probing}] Mehrotra's probing heuristic
\item[\texttt{loqo}] LOQO's centrality rule
\item[\texttt{quality-function}] minimize a quality function
\end{list}
}

\printoption{mu\_strategy}%
{\ttfamily monotone, adaptive}%
{adaptive}%
{Update strategy for barrier parameter.\\
Determines which barrier parameter update strategy is to be used.}%
{\begin{list}{}{
\setlength{\parsep}{0em}
\setlength{\leftmargin}{5ex}
\setlength{\labelwidth}{2ex}
\setlength{\itemindent}{0ex}
\setlength{\topsep}{0pt}}
\item[\texttt{monotone}] use the monotone (Fiacco-McCormick) strategy
\item[\texttt{adaptive}] use the adaptive update strategy
\end{list}
}

\printoption{mu\_superlinear\_decrease\_power}%
{$1<\textrm{real}<2$}%
{$1.5$}%
{Determines superlinear decrease rate of barrier parameter.\\
For the Fiacco-McCormick update procedure the new barrier parameter mu is obtained by taking the minimum of mu*"mu\_linear\_decrease\_factor" and mu\^"superlinear\_decrease\_power".  (This is theta\_mu in implementation paper.) This option is also used in the adaptive mu strategy during the monotone mode.}%
{}

\printoption{quality\_function\_balancing\_term}%
{\ttfamily none, cubic}%
{none}%
{The balancing term included in the quality function for centrality.\\
This determines whether a term is added to the quality function that penalizes situations where the complementarity is much smaller than dual and primal infeasibilities. (Only used if option "mu\_oracle" is set to "quality-function".)}%
{\begin{list}{}{
\setlength{\parsep}{0em}
\setlength{\leftmargin}{5ex}
\setlength{\labelwidth}{2ex}
\setlength{\itemindent}{0ex}
\setlength{\topsep}{0pt}}
\item[\texttt{none}] no balancing term is added
\item[\texttt{cubic}] Max(0,Max(dual\_inf,primal\_inf)-compl)\^3
\end{list}
}

\printoption{quality\_function\_centrality}%
{\ttfamily none, log, reciprocal, cubed-reciprocal}%
{none}%
{The penalty term for centrality that is included in quality function.\\
This determines whether a term is added to the quality function to penalize deviation from centrality with respect to complementarity.  The complementarity measure here is the xi in the Loqo update rule. (Only used if option "mu\_oracle" is set to "quality-function".)}%
{\begin{list}{}{
\setlength{\parsep}{0em}
\setlength{\leftmargin}{5ex}
\setlength{\labelwidth}{2ex}
\setlength{\itemindent}{0ex}
\setlength{\topsep}{0pt}}
\item[\texttt{none}] no penalty term is added
\item[\texttt{log}] complementarity * the log of the centrality measure
\item[\texttt{reciprocal}] complementarity * the reciprocal of the centrality measure
\item[\texttt{cubed-reciprocal}] complementarity * the reciprocal of the centrality measure cubed
\end{list}
}

\printoption{quality\_function\_max\_section\_steps}%
{$0\leq\textrm{integer}$}%
{$8$}%
{Maximum number of search steps during direct search procedure determining the optimal centering parameter.\\
The golden section search is performed for the quality function based mu oracle. (Only used if option "mu\_oracle" is set to "quality-function".)}%
{}

\printoption{quality\_function\_norm\_type}%
{\ttfamily 1-norm, 2-norm-squared, max-norm, 2-norm}%
{2-norm-squared}%
{Norm used for components of the quality function.\\
(Only used if option "mu\_oracle" is set to "quality-function".)}%
{\begin{list}{}{
\setlength{\parsep}{0em}
\setlength{\leftmargin}{5ex}
\setlength{\labelwidth}{2ex}
\setlength{\itemindent}{0ex}
\setlength{\topsep}{0pt}}
\item[\texttt{1-norm}] use the 1-norm (abs sum)
\item[\texttt{2-norm-squared}] use the 2-norm squared (sum of squares)
\item[\texttt{max-norm}] use the infinity norm (max)
\item[\texttt{2-norm}] use 2-norm
\end{list}
}

\printoption{quality\_function\_section\_qf\_tol}%
{$0\leq\textrm{real}<1$}%
{$0$}%
{Tolerance for the golden section search procedure determining the optimal centering parameter (in the function value space).\\
The golden section search is performed for the quality function based mu oracle. (Only used if option "mu\_oracle" is set to "quality-function".)}%
{}

\printoption{quality\_function\_section\_sigma\_tol}%
{$0\leq\textrm{real}<1$}%
{$0.01$}%
{Tolerance for the section search procedure determining the optimal centering parameter (in sigma space).\\
The golden section search is performed for the quality function based mu oracle. (Only used if option "mu\_oracle" is set to "quality-function".)}%
{}

\printoption{sigma\_max}%
{$0<\textrm{real}$}%
{$100$}%
{Maximum value of the centering parameter.\\
This is the upper bound for the centering parameter chosen by the quality function based barrier parameter update. (Only used if option "mu\_oracle" is set to "quality-function".)}%
{}

\printoption{sigma\_min}%
{$0\leq\textrm{real}$}%
{$10^{- 6}$}%
{Minimum value of the centering parameter.\\
This is the lower bound for the centering parameter chosen by the quality function based barrier parameter update. (Only used if option "mu\_oracle" is set to "quality-function".)}%
{}

\printoption{tau\_min}%
{$0<\textrm{real}<1$}%
{$0.99$}%
{Lower bound on fraction-to-the-boundary parameter tau.\\
(This is tau\_min in the implementation paper.)  This option is also used in the adaptive mu strategy during the monotone mode.}%
{}

\printoptioncategory{Convergence}
\printoption{acceptable\_compl\_inf\_tol}%
{$0<\textrm{real}$}%
{$0.01$}%
{"Acceptance" threshold for the complementarity conditions.\\
Absolute tolerance on the complementarity. "Acceptable" termination requires that the max-norm of the (unscaled) complementarity is less than this threshold; see also acceptable\_tol.}%
{}

\printoption{acceptable\_constr\_viol\_tol}%
{$0<\textrm{real}$}%
{$0.01$}%
{"Acceptance" threshold for the constraint violation.\\
Absolute tolerance on the constraint violation. "Acceptable" termination requires that the max-norm of the (unscaled) constraint violation is less than this threshold; see also acceptable\_tol.}%
{}

\printoption{acceptable\_dual\_inf\_tol}%
{$0<\textrm{real}$}%
{$10^{ 10}$}%
{"Acceptance" threshold for the dual infeasibility.\\
Absolute tolerance on the dual infeasibility. "Acceptable" termination requires that the (max-norm of the unscaled) dual infeasibility is less than this threshold; see also acceptable\_tol.}%
{}

\printoption{acceptable\_iter}%
{$0\leq\textrm{integer}$}%
{$15$}%
{Number of "acceptable" iterates before triggering termination.\\
If the algorithm encounters this many successive "acceptable" iterates (see "acceptable\_tol"), it terminates, assuming that the problem has been solved to best possible accuracy given round-off.  If it is set to zero, this heuristic is disabled.}%
{}

\printoption{acceptable\_obj\_change\_tol}%
{$0\leq\textrm{real}$}%
{$10^{ 20}$}%
{"Acceptance" stopping criterion based on objective function change.\\
If the relative change of the objective function (scaled by Max(1,|f(x)|)) is less than this value, this part of the acceptable tolerance termination is satisfied; see also acceptable\_tol.  This is useful for the quasi-Newton option, which has trouble to bring down the dual infeasibility.}%
{}

\printoption{acceptable\_tol}%
{$0<\textrm{real}$}%
{$10^{- 6}$}%
{"Acceptable" convergence tolerance (relative).\\
Determines which (scaled) overall optimality error is considered to be "acceptable." There are two levels of termination criteria.  If the usual "desired" tolerances (see tol, dual\_inf\_tol etc) are satisfied at an iteration, the algorithm immediately terminates with a success message.  On the other hand, if the algorithm encounters "acceptable\_iter" many iterations in a row that are considered "acceptable", it will terminate before the desired convergence tolerance is met. This is useful in cases where the algorithm might not be able to achieve the "desired" level of accuracy.}%
{}

\printoption{compl\_inf\_tol}%
{$0<\textrm{real}$}%
{$0.0001$}%
{Desired threshold for the complementarity conditions.\\
Absolute tolerance on the complementarity. Successful termination requires that the max-norm of the (unscaled) complementarity is less than this threshold.}%
{}

\printoption{constr\_viol\_tol}%
{$0<\textrm{real}$}%
{$0.0001$}%
{Desired threshold for the constraint violation.\\
Absolute tolerance on the constraint violation. Successful termination requires that the max-norm of the (unscaled) constraint violation is less than this threshold.}%
{}

\printoption{diverging\_iterates\_tol}%
{$0<\textrm{real}$}%
{$10^{ 20}$}%
{Threshold for maximal value of primal iterates.\\
If any component of the primal iterates exceeded this value (in absolute terms), the optimization is aborted with the exit message that the iterates seem to be diverging.}%
{}

\printoption{dual\_inf\_tol}%
{$0<\textrm{real}$}%
{$1$}%
{Desired threshold for the dual infeasibility.\\
Absolute tolerance on the dual infeasibility. Successful termination requires that the max-norm of the (unscaled) dual infeasibility is less than this threshold.}%
{}

\printoption{max\_cpu\_time}%
{$0<\textrm{real}$}%
{$1000$}%
{Maximum number of CPU seconds.\\
A limit on CPU seconds that Ipopt can use to solve one problem.  If during the convergence check this limit is exceeded, Ipopt will terminate with a corresponding error message.}%
{}

\printoption{max\_iter}%
{$0\leq\textrm{integer}$}%
{$\infty$}%
{Maximum number of iterations.\\
The algorithm terminates with an error message if the number of iterations exceeded this number.}%
{}

\printoption{mu\_target}%
{$0\leq\textrm{real}$}%
{$0$}%
{Desired value of complementarity.\\
Usually, the barrier parameter is driven to zero and the termination test for complementarity is measured with respect to zero complementarity.  However, in some cases it might be desired to have Ipopt solve barrier problem for strictly positive value of the barrier parameter.  In this case, the value of "mu\_target" specifies the final value of the barrier parameter, and the termination tests are then defined with respect to the barrier problem for this value of the barrier parameter.}%
{}

\printoption{s\_max}%
{$0<\textrm{real}$}%
{$100$}%
{Scaling threshold for the NLP error.\\
(See paragraph after Eqn. (6) in the implementation paper.)}%
{}

\printoption{tol}%
{$0<\textrm{real}$}%
{$10^{- 8}$}%
{Desired convergence tolerance (relative).\\
Determines the convergence tolerance for the algorithm.  The algorithm terminates successfully, if the (scaled) NLP error becomes smaller than this value, and if the (absolute) criteria according to "dual\_inf\_tol", "primal\_inf\_tol", and "compl\_inf\_tol" are met.  (This is epsilon\_tol in Eqn. (6) in implementation paper).  See also "acceptable\_tol" as a second termination criterion.  Note, some other algorithmic features also use this quantity to determine thresholds etc.}%
{}

\printoptioncategory{Hessian Approximation}
\printoption{hessian\_approximation}%
{\ttfamily exact, limited-memory}%
{exact}%
{Indicates what Hessian information is to be used.\\
This determines which kind of information for the Hessian of the Lagrangian function is used by the algorithm.}%
{\begin{list}{}{
\setlength{\parsep}{0em}
\setlength{\leftmargin}{5ex}
\setlength{\labelwidth}{2ex}
\setlength{\itemindent}{0ex}
\setlength{\topsep}{0pt}}
\item[\texttt{exact}] Use second derivatives provided by the NLP.
\item[\texttt{limited-memory}] Perform a limited-memory quasi-Newton approximation
\end{list}
}

\printoption{hessian\_approximation\_space}%
{\ttfamily nonlinear-variables, all-variables}%
{nonlinear-variables}%
{Indicates in which subspace the Hessian information is to be approximated.}%
{\begin{list}{}{
\setlength{\parsep}{0em}
\setlength{\leftmargin}{5ex}
\setlength{\labelwidth}{2ex}
\setlength{\itemindent}{0ex}
\setlength{\topsep}{0pt}}
\item[\texttt{nonlinear-variables}] only in space of nonlinear variables.
\item[\texttt{all-variables}] in space of all variables (without slacks)
\end{list}
}

\printoption{limited\_memory\_aug\_solver}%
{\ttfamily sherman-morrison, extended}%
{sherman-morrison}%
{Strategy for solving the augmented system for low-rank Hessian.}%
{\begin{list}{}{
\setlength{\parsep}{0em}
\setlength{\leftmargin}{5ex}
\setlength{\labelwidth}{2ex}
\setlength{\itemindent}{0ex}
\setlength{\topsep}{0pt}}
\item[\texttt{sherman-morrison}] use Sherman-Morrison formula
\item[\texttt{extended}] use an extended augmented system
\end{list}
}

\printoption{limited\_memory\_init\_val}%
{$0<\textrm{real}$}%
{$1$}%
{Value for B0 in low-rank update.\\
The starting matrix in the low rank update, B0, is chosen to be this multiple of the identity in the first iteration (when no updates have been performed yet), and is constantly chosen as this value, if "limited\_memory\_initialization" is "constant".}%
{}

\printoption{limited\_memory\_init\_val\_max}%
{$0<\textrm{real}$}%
{$10^{  8}$}%
{Upper bound on value for B0 in low-rank update.\\
The starting matrix in the low rank update, B0, is chosen to be this multiple of the identity in the first iteration (when no updates have been performed yet), and is constantly chosen as this value, if "limited\_memory\_initialization" is "constant".}%
{}

\printoption{limited\_memory\_init\_val\_min}%
{$0<\textrm{real}$}%
{$10^{- 8}$}%
{Lower bound on value for B0 in low-rank update.\\
The starting matrix in the low rank update, B0, is chosen to be this multiple of the identity in the first iteration (when no updates have been performed yet), and is constantly chosen as this value, if "limited\_memory\_initialization" is "constant".}%
{}

\printoption{limited\_memory\_initialization}%
{\ttfamily scalar1, scalar2, scalar3, scalar4, constant}%
{scalar1}%
{Initialization strategy for the limited memory quasi-Newton approximation.\\
Determines how the diagonal Matrix $B_0$ as the first term in the limited memory approximation should be computed.}%
{\begin{list}{}{
\setlength{\parsep}{0em}
\setlength{\leftmargin}{5ex}
\setlength{\labelwidth}{2ex}
\setlength{\itemindent}{0ex}
\setlength{\topsep}{0pt}}
\item[\texttt{scalar1}] sigma = $s^Ty/s^Ts$
\item[\texttt{scalar2}] sigma = $y^Ty/s^Ty$
\item[\texttt{scalar3}] arithmetic average of scalar1 and scalar2
\item[\texttt{scalar4}] geometric average of scalar1 and scalar2
\item[\texttt{constant}] sigma = limited\_memory\_init\_val
\end{list}
}

\printoption{limited\_memory\_max\_history}%
{$0\leq\textrm{integer}$}%
{$6$}%
{Maximum size of the history for the limited quasi-Newton Hessian approximation.\\
This option determines the number of most recent iterations that are taken into account for the limited-memory quasi-Newton approximation.}%
{}

\printoption{limited\_memory\_max\_skipping}%
{$1\leq\textrm{integer}$}%
{$2$}%
{Threshold for successive iterations where update is skipped.\\
If the update is skipped more than this number of successive iterations, we quasi-Newton approximation is reset.}%
{}

\printoption{limited\_memory\_special\_for\_resto}%
{\ttfamily no, yes}%
{no}%
{Determines if the quasi-Newton updates should be special during the restoration phase.\\
Until Nov 2010, Ipopt used a special update during the restoration phase, but it turned out that this does not work well.  The new default uses the regular update procedure and it improves results.  If for some reason you want to get back to the original update, set this option to "yes".}%
{\begin{list}{}{
\setlength{\parsep}{0em}
\setlength{\leftmargin}{5ex}
\setlength{\labelwidth}{2ex}
\setlength{\itemindent}{0ex}
\setlength{\topsep}{0pt}}
\item[\texttt{no}] use the same update as in regular iterations
\item[\texttt{yes}] use the a special update during restoration phase
\end{list}
}

\printoption{limited\_memory\_update\_type}%
{\ttfamily bfgs, sr1}%
{bfgs}%
{Quasi-Newton update formula for the limited memory approximation.\\
Determines which update formula is to be used for the limited-memory quasi-Newton approximation.}%
{\begin{list}{}{
\setlength{\parsep}{0em}
\setlength{\leftmargin}{5ex}
\setlength{\labelwidth}{2ex}
\setlength{\itemindent}{0ex}
\setlength{\topsep}{0pt}}
\item[\texttt{bfgs}] BFGS update (with skipping)
\item[\texttt{sr1}] SR1 (not working well)
\end{list}
}

\printoptioncategory{Initialization}
\printoption{bound\_frac}%
{$0<\textrm{real}\leq0.5$}%
{$0.01$}%
{Desired minimum relative distance from the initial point to bound.\\
Determines how much the initial point might have to be modified in order to be sufficiently inside the bounds (together with "bound\_push").  (This is kappa\_2 in Section 3.6 of implementation paper.)}%
{}

\printoption{bound\_mult\_init\_method}%
{\ttfamily constant, mu-based}%
{constant}%
{Initialization method for bound multipliers\\
This option defines how the iterates for the bound multipliers are initialized.  If "constant" is chosen, then all bound multipliers are initialized to the value of "bound\_mult\_init\_val".  If "mu-based" is chosen, the each value is initialized to the the value of "mu\_init" divided by the corresponding slack variable.  This latter option might be useful if the starting point is close to the optimal solution.}%
{\begin{list}{}{
\setlength{\parsep}{0em}
\setlength{\leftmargin}{5ex}
\setlength{\labelwidth}{2ex}
\setlength{\itemindent}{0ex}
\setlength{\topsep}{0pt}}
\item[\texttt{constant}] set all bound multipliers to the value of bound\_mult\_init\_val
\item[\texttt{mu-based}] initialize to mu\_init/x\_slack
\end{list}
}

\printoption{bound\_mult\_init\_val}%
{$0<\textrm{real}$}%
{$1$}%
{Initial value for the bound multipliers.\\
All dual variables corresponding to bound constraints are initialized to this value.}%
{}

\printoption{bound\_push}%
{$0<\textrm{real}$}%
{$0.01$}%
{Desired minimum absolute distance from the initial point to bound.\\
Determines how much the initial point might have to be modified in order to be sufficiently inside the bounds (together with "bound\_frac").  (This is kappa\_1 in Section 3.6 of implementation paper.)}%
{}

\printoption{constr\_mult\_init\_max}%
{$0\leq\textrm{real}$}%
{$1000$}%
{Maximum allowed least-square guess of constraint multipliers.\\
Determines how large the initial least-square guesses of the constraint multipliers are allowed to be (in max-norm). If the guess is larger than this value, it is discarded and all constraint multipliers are set to zero.  This options is also used when initializing the restoration phase. By default, "resto.constr\_mult\_init\_max" (the one used in RestoIterateInitializer) is set to zero.}%
{}

\printoption{least\_square\_init\_duals}%
{\ttfamily no, yes}%
{no}%
{Least square initialization of all dual variables\\
If set to yes, Ipopt tries to compute least-square multipliers (considering ALL dual variables).  If successful, the bound multipliers are possibly corrected to be at least bound\_mult\_init\_val. This might be useful if the user doesn't know anything about the starting point, or for solving an LP or QP.  This overwrites option "bound\_mult\_init\_method".}%
{\begin{list}{}{
\setlength{\parsep}{0em}
\setlength{\leftmargin}{5ex}
\setlength{\labelwidth}{2ex}
\setlength{\itemindent}{0ex}
\setlength{\topsep}{0pt}}
\item[\texttt{no}] use bound\_mult\_init\_val and least-square equality constraint multipliers
\item[\texttt{yes}] overwrite user-provided point with least-square estimates
\end{list}
}

\printoption{least\_square\_init\_primal}%
{\ttfamily no, yes}%
{no}%
{Least square initialization of the primal variables\\
If set to yes, Ipopt ignores the user provided point and solves a least square problem for the primal variables (x and s), to fit the linearized equality and inequality constraints.  This might be useful if the user doesn't know anything about the starting point, or for solving an LP or QP.}%
{\begin{list}{}{
\setlength{\parsep}{0em}
\setlength{\leftmargin}{5ex}
\setlength{\labelwidth}{2ex}
\setlength{\itemindent}{0ex}
\setlength{\topsep}{0pt}}
\item[\texttt{no}] take user-provided point
\item[\texttt{yes}] overwrite user-provided point with least-square estimates
\end{list}
}

\printoption{slack\_bound\_frac}%
{$0<\textrm{real}\leq0.5$}%
{$0.01$}%
{Desired minimum relative distance from the initial slack to bound.\\
Determines how much the initial slack variables might have to be modified in order to be sufficiently inside the inequality bounds (together with "slack\_bound\_push").  (This is kappa\_2 in Section 3.6 of implementation paper.)}%
{}

\printoption{slack\_bound\_push}%
{$0<\textrm{real}$}%
{$0.01$}%
{Desired minimum absolute distance from the initial slack to bound.\\
Determines how much the initial slack variables might have to be modified in order to be sufficiently inside the inequality bounds (together with "slack\_bound\_frac").  (This is kappa\_1 in Section 3.6 of implementation paper.)}%
{}

\printoptioncategory{Line Search}
\printoption{accept\_after\_max\_steps}%
{$-1\leq\textrm{integer}$}%
{$-1$}%
{Accept a trial point after maximal this number of steps.\\
Even if it does not satisfy line search conditions.}%
{}

\printoption{accept\_every\_trial\_step}%
{\ttfamily no, yes}%
{no}%
{Always accept the first trial step.\\
Setting this option to "yes" essentially disables the line search and makes the algorithm take aggressive steps, without global convergence guarantees.}%
{\begin{list}{}{
\setlength{\parsep}{0em}
\setlength{\leftmargin}{5ex}
\setlength{\labelwidth}{2ex}
\setlength{\itemindent}{0ex}
\setlength{\topsep}{0pt}}
\item[\texttt{no}] don't arbitrarily accept the full step
\item[\texttt{yes}] always accept the full step
\end{list}
}

\printoption{alpha\_for\_y}%
{\ttfamily primal, bound-mult, min, max, full, min-dual-infeas, safer-min-dual-infeas, primal-and-full, dual-and-full, acceptor}%
{primal}%
{Method to determine the step size for constraint multipliers.\\
This option determines how the step size (alpha\_y) will be calculated when updating the constraint multipliers.}%
{\begin{list}{}{
\setlength{\parsep}{0em}
\setlength{\leftmargin}{5ex}
\setlength{\labelwidth}{2ex}
\setlength{\itemindent}{0ex}
\setlength{\topsep}{0pt}}
\item[\texttt{primal}] use primal step size
\item[\texttt{bound-mult}] use step size for the bound multipliers (good for LPs)
\item[\texttt{min}] use the min of primal and bound multipliers
\item[\texttt{max}] use the max of primal and bound multipliers
\item[\texttt{full}] take a full step of size one
\item[\texttt{min-dual-infeas}] choose step size minimizing new dual infeasibility
\item[\texttt{safer-min-dual-infeas}] like "min\_dual\_infeas", but safeguarded by "min" and "max"
\item[\texttt{primal-and-full}] use the primal step size, and full step if delta\_x $<$= alpha\_for\_y\_tol
\item[\texttt{dual-and-full}] use the dual step size, and full step if delta\_x $<$= alpha\_for\_y\_tol
\item[\texttt{acceptor}] Call LSAcceptor to get step size for y
\end{list}
}

\printoption{alpha\_for\_y\_tol}%
{$0\leq\textrm{real}$}%
{$10$}%
{Tolerance for switching to full equality multiplier steps.\\
This is only relevant if "alpha\_for\_y" is chosen "primal-and-full" or "dual-and-full".  The step size for the equality constraint multipliers is taken to be one if the max-norm of the primal step is less than this tolerance.}%
{}

\printoption{alpha\_min\_frac}%
{$0<\textrm{real}<1$}%
{$0.05$}%
{Safety factor for the minimal step size (before switching to restoration phase).\\
(This is gamma\_alpha in Eqn. (20) in the implementation paper.)}%
{}

\printoption{alpha\_red\_factor}%
{$0<\textrm{real}<1$}%
{$0.5$}%
{Fractional reduction of the trial step size in the backtracking line search.\\
At every step of the backtracking line search, the trial step size is reduced by this factor.}%
{}

\printoption{constraint\_violation\_norm\_type}%
{\ttfamily 1-norm, 2-norm, max-norm}%
{1-norm}%
{Norm to be used for the constraint violation in the line search.\\
Determines which norm should be used when the algorithm computes the constraint violation in the line search.}%
{\begin{list}{}{
\setlength{\parsep}{0em}
\setlength{\leftmargin}{5ex}
\setlength{\labelwidth}{2ex}
\setlength{\itemindent}{0ex}
\setlength{\topsep}{0pt}}
\item[\texttt{1-norm}] use the 1-norm
\item[\texttt{2-norm}] use the 2-norm
\item[\texttt{max-norm}] use the infinity norm
\end{list}
}

\printoption{corrector\_compl\_avrg\_red\_fact}%
{$0<\textrm{real}$}%
{$1$}%
{Complementarity tolerance factor for accepting corrector step (unsupported!).\\
This option determines the factor by which complementarity is allowed to increase for a corrector step to be accepted.}%
{}

\printoption{corrector\_type}%
{\ttfamily none, affine, primal-dual}%
{none}%
{The type of corrector steps that should be taken (unsupported!).\\
If "mu\_strategy" is "adaptive", this option determines what kind of corrector steps should be tried.}%
{\begin{list}{}{
\setlength{\parsep}{0em}
\setlength{\leftmargin}{5ex}
\setlength{\labelwidth}{2ex}
\setlength{\itemindent}{0ex}
\setlength{\topsep}{0pt}}
\item[\texttt{none}] no corrector
\item[\texttt{affine}] corrector step towards mu=0
\item[\texttt{primal-dual}] corrector step towards current mu
\end{list}
}

\printoption{delta}%
{$0<\textrm{real}$}%
{$1$}%
{Multiplier for constraint violation in the switching rule.\\
(See Eqn. (19) in the implementation paper.)}%
{}

\printoption{eta\_phi}%
{$0<\textrm{real}<0.5$}%
{$10^{- 8}$}%
{Relaxation factor in the Armijo condition.\\
(See Eqn. (20) in the implementation paper)}%
{}

\printoption{filter\_reset\_trigger}%
{$1\leq\textrm{integer}$}%
{$5$}%
{Number of iterations that trigger the filter reset.\\
If the filter reset heuristic is active and the number of successive iterations in which the last rejected trial step size was rejected because of the filter, the filter is reset.}%
{}

\printoption{gamma\_phi}%
{$0<\textrm{real}<1$}%
{$10^{- 8}$}%
{Relaxation factor in the filter margin for the barrier function.\\
(See Eqn. (18a) in the implementation paper.)}%
{}

\printoption{gamma\_theta}%
{$0<\textrm{real}<1$}%
{$10^{- 5}$}%
{Relaxation factor in the filter margin for the constraint violation.\\
(See Eqn. (18b) in the implementation paper.)}%
{}

\printoption{kappa\_sigma}%
{$0<\textrm{real}$}%
{$10^{ 10}$}%
{Factor limiting the deviation of dual variables from primal estimates.\\
If the dual variables deviate from their primal estimates, a correction is performed. (See Eqn. (16) in the implementation paper.) Setting the value to less than 1 disables the correction.}%
{}

\printoption{kappa\_soc}%
{$0<\textrm{real}$}%
{$0.99$}%
{Factor in the sufficient reduction rule for second order correction.\\
This option determines how much a second order correction step must reduce the constraint violation so that further correction steps are attempted.  (See Step A-5.9 of Algorithm A in the implementation paper.)}%
{}

\printoption{line\_search\_method}%
{\ttfamily filter, cg-penalty, penalty}%
{filter}%
{Globalization method used in backtracking line search\\
Only the "filter" choice is officially supported.  But sometimes, good results might be obtained with the other choices.}%
{\begin{list}{}{
\setlength{\parsep}{0em}
\setlength{\leftmargin}{5ex}
\setlength{\labelwidth}{2ex}
\setlength{\itemindent}{0ex}
\setlength{\topsep}{0pt}}
\item[\texttt{filter}] Filter method
\item[\texttt{cg-penalty}] Chen-Goldfarb penalty function
\item[\texttt{penalty}] Standard penalty function
\end{list}
}

\printoption{max\_filter\_resets}%
{$0\leq\textrm{integer}$}%
{$5$}%
{Maximal allowed number of filter resets\\
A positive number enables a heuristic that resets the filter, whenever in more than "filter\_reset\_trigger" successive iterations the last rejected trial steps size was rejected because of the filter.  This option determine the maximal number of resets that are allowed to take place.}%
{}

\printoption{max\_soc}%
{$0\leq\textrm{integer}$}%
{$4$}%
{Maximum number of second order correction trial steps at each iteration.\\
Choosing 0 disables the second order corrections. (This is p\^{max} of Step A-5.9 of Algorithm A in the implementation paper.)}%
{}

\printoption{nu\_inc}%
{$0<\textrm{real}$}%
{$0.0001$}%
{Increment of the penalty parameter.}%
{}

\printoption{nu\_init}%
{$0<\textrm{real}$}%
{$10^{- 6}$}%
{Initial value of the penalty parameter.}%
{}

\printoption{obj\_max\_inc}%
{$1<\textrm{real}$}%
{$5$}%
{Determines the upper bound on the acceptable increase of barrier objective function.\\
Trial points are rejected if they lead to an increase in the barrier objective function by more than obj\_max\_inc orders of magnitude.}%
{}

\printoption{recalc\_y}%
{\ttfamily no, yes}%
{no}%
{Tells the algorithm to recalculate the equality and inequality multipliers as least square estimates.\\
This asks the algorithm to recompute the multipliers, whenever the current infeasibility is less than recalc\_y\_feas\_tol. Choosing yes might be helpful in the quasi-Newton option.  However, each recalculation requires an extra factorization of the linear system.  If a limited memory quasi-Newton option is chosen, this is used by default.}%
{\begin{list}{}{
\setlength{\parsep}{0em}
\setlength{\leftmargin}{5ex}
\setlength{\labelwidth}{2ex}
\setlength{\itemindent}{0ex}
\setlength{\topsep}{0pt}}
\item[\texttt{no}] use the Newton step to update the multipliers
\item[\texttt{yes}] use least-square multiplier estimates
\end{list}
}

\printoption{recalc\_y\_feas\_tol}%
{$0<\textrm{real}$}%
{$10^{- 6}$}%
{Feasibility threshold for recomputation of multipliers.\\
If recalc\_y is chosen and the current infeasibility is less than this value, then the multipliers are recomputed.}%
{}

\printoption{rho}%
{$0<\textrm{real}<1$}%
{$0.1$}%
{Value in penalty parameter update formula.}%
{}

\printoption{s\_phi}%
{$1<\textrm{real}$}%
{$2.3$}%
{Exponent for linear barrier function model in the switching rule.\\
(See Eqn. (19) in the implementation paper.)}%
{}

\printoption{s\_theta}%
{$1<\textrm{real}$}%
{$1.1$}%
{Exponent for current constraint violation in the switching rule.\\
(See Eqn. (19) in the implementation paper.)}%
{}

\printoption{skip\_corr\_if\_neg\_curv}%
{\ttfamily no, yes}%
{yes}%
{Skip the corrector step in negative curvature iteration (unsupported!).\\
The corrector step is not tried if negative curvature has been encountered during the computation of the search direction in the current iteration. This option is only used if "mu\_strategy" is "adaptive".}%
{\begin{list}{}{
\setlength{\parsep}{0em}
\setlength{\leftmargin}{5ex}
\setlength{\labelwidth}{2ex}
\setlength{\itemindent}{0ex}
\setlength{\topsep}{0pt}}
\item[\texttt{no}] don't skip
\item[\texttt{yes}] skip
\end{list}
}

\printoption{skip\_corr\_in\_monotone\_mode}%
{\ttfamily no, yes}%
{yes}%
{Skip the corrector step during monotone barrier parameter mode (unsupported!).\\
The corrector step is not tried if the algorithm is currently in the monotone mode (see also option "barrier\_strategy").This option is only used if "mu\_strategy" is "adaptive".}%
{\begin{list}{}{
\setlength{\parsep}{0em}
\setlength{\leftmargin}{5ex}
\setlength{\labelwidth}{2ex}
\setlength{\itemindent}{0ex}
\setlength{\topsep}{0pt}}
\item[\texttt{no}] don't skip
\item[\texttt{yes}] skip
\end{list}
}

\printoption{slack\_move}%
{$0\leq\textrm{real}$}%
{$1.81899 \cdot 10^{-12}$}%
{Correction size for very small slacks.\\
Due to numerical issues or the lack of an interior, the slack variables might become very small.  If a slack becomes very small compared to machine precision, the corresponding bound is moved slightly.  This parameter determines how large the move should be.  Its default value is mach\_eps\^{3/4}.  (See also end of Section 3.5 in implementation paper - but actual implementation might be somewhat different.)}%
{}

\printoption{theta\_max\_fact}%
{$0<\textrm{real}$}%
{$10000$}%
{Determines upper bound for constraint violation in the filter.\\
The algorithmic parameter theta\_max is determined as theta\_max\_fact times the maximum of 1 and the constraint violation at initial point.  Any point with a constraint violation larger than theta\_max is unacceptable to the filter (see Eqn. (21) in the implementation paper).}%
{}

\printoption{theta\_min\_fact}%
{$0<\textrm{real}$}%
{$0.0001$}%
{Determines constraint violation threshold in the switching rule.\\
The algorithmic parameter theta\_min is determined as theta\_min\_fact times the maximum of 1 and the constraint violation at initial point.  The switching rules treats an iteration as an h-type iteration whenever the current constraint violation is larger than theta\_min (see paragraph before Eqn. (19) in the implementation paper).}%
{}

\printoption{tiny\_step\_tol}%
{$0\leq\textrm{real}$}%
{$2.22045 \cdot 10^{-15}$}%
{Tolerance for detecting numerically insignificant steps.\\
If the search direction in the primal variables (x and s) is, in relative terms for each component, less than this value, the algorithm accepts the full step without line search.  If this happens repeatedly, the algorithm will terminate with a corresponding exit message. The default value is 10 times machine precision.}%
{}

\printoption{tiny\_step\_y\_tol}%
{$0\leq\textrm{real}$}%
{$0.01$}%
{Tolerance for quitting because of numerically insignificant steps.\\
If the search direction in the primal variables (x and s) is, in relative terms for each component, repeatedly less than tiny\_step\_tol, and the step in the y variables is smaller than this threshold, the algorithm will terminate.}%
{}

\printoption{watchdog\_shortened\_iter\_trigger}%
{$0\leq\textrm{integer}$}%
{$10$}%
{Number of shortened iterations that trigger the watchdog.\\
If the number of successive iterations in which the backtracking line search did not accept the first trial point exceeds this number, the watchdog procedure is activated.  Choosing "0" here disables the watchdog procedure.}%
{}

\printoption{watchdog\_trial\_iter\_max}%
{$1\leq\textrm{integer}$}%
{$3$}%
{Maximum number of watchdog iterations.\\
This option determines the number of trial iterations allowed before the watchdog procedure is aborted and the algorithm returns to the stored point.}%
{}

\printoptioncategory{Linear Solver}
\printoption{linear\_scaling\_on\_demand}%
{\ttfamily no, yes}%
{yes}%
{Flag indicating that linear scaling is only done if it seems required.\\
This option is only important if a linear scaling method (e.g., mc19) is used.  If you choose "no", then the scaling factors are computed for every linear system from the start.  This can be quite expensive. Choosing "yes" means that the algorithm will start the scaling method only when the solutions to the linear system seem not good, and then use it until the end.}%
{\begin{list}{}{
\setlength{\parsep}{0em}
\setlength{\leftmargin}{5ex}
\setlength{\labelwidth}{2ex}
\setlength{\itemindent}{0ex}
\setlength{\topsep}{0pt}}
\item[\texttt{no}] Always scale the linear system.
\item[\texttt{yes}] Start using linear system scaling if solutions seem not good.
\end{list}
}

\printoption{linear\_solver}%
{\ttfamily ma27, ma57, ma77, ma86, ma97, pardiso, mumps}%
{ma27}%
{Linear solver used for step computations.\\
Determines which linear algebra package is to be used for the solution of the augmented linear system (for obtaining the search directions). Note, that MA27, MA57, MA86, and MA97 are only available with a commercially supported GAMS/IpoptH license, or when the user provides a library with HSL code separately. If no GAMS/IpoptH license is available, the default linear solver is MUMPS. Pardiso is only available on Linux and Windows systems. For using Pardiso on non-Linux/Windows systems or MA77, a Pardiso or HSL library need to be provided.}%
{\begin{list}{}{
\setlength{\parsep}{0em}
\setlength{\leftmargin}{5ex}
\setlength{\labelwidth}{2ex}
\setlength{\itemindent}{0ex}
\setlength{\topsep}{0pt}}
\item[\texttt{ma27}] use the Harwell routine MA27
\item[\texttt{ma57}] use the Harwell routine MA57
\item[\texttt{ma77}] use the Harwell routine HSL\_MA77
\item[\texttt{ma86}] use the Harwell routine HSL\_MA86
\item[\texttt{ma97}] use the Harwell routine HSL\_MA97
\item[\texttt{pardiso}] use the Pardiso package
\item[\texttt{mumps}] use MUMPS package
\end{list}
}

\printoption{linear\_system\_scaling}%
{\ttfamily none, mc19, slack-based}%
{mc19}%
{Method for scaling the linear system.\\
Determines the method used to compute symmetric scaling factors for the augmented system (see also the "linear\_scaling\_on\_demand" option).  This scaling is independent of the NLP problem scaling.  By default, MC19 is only used if MA27 or MA57 are selected as linear solvers. Note, that MC19 is only available with a commercially supported GAMS/IpoptH license, or when the user provides a library with HSL code separately. If no commerical GAMS/IpoptH license is available, the default scaling method is slack-based.}%
{\begin{list}{}{
\setlength{\parsep}{0em}
\setlength{\leftmargin}{5ex}
\setlength{\labelwidth}{2ex}
\setlength{\itemindent}{0ex}
\setlength{\topsep}{0pt}}
\item[\texttt{none}] no scaling will be performed
\item[\texttt{mc19}] use the Harwell routine MC19
\item[\texttt{slack-based}] use the slack values
\end{list}
}

\printoptioncategory{MA27 Linear Solver}
\printoption{ma27\_ignore\_singularity}%
{\ttfamily no, yes}%
{no}%
{Enables MA27's ability to solve a linear system even if the matrix is singular.\\
Setting this option to "yes" means that Ipopt will call MA27 to compute solutions for right hand sides, even if MA27 has detected that the matrix is singular (but is still able to solve the linear system). In some cases this might be better than using Ipopt's heuristic of small perturbation of the lower diagonal of the KKT matrix.}%
{\begin{list}{}{
\setlength{\parsep}{0em}
\setlength{\leftmargin}{5ex}
\setlength{\labelwidth}{2ex}
\setlength{\itemindent}{0ex}
\setlength{\topsep}{0pt}}
\item[\texttt{no}] Don't have MA27 solve singular systems
\item[\texttt{yes}] Have MA27 solve singular systems
\end{list}
}

\printoption{ma27\_la\_init\_factor}%
{$1\leq\textrm{real}$}%
{$5$}%
{Real workspace memory for MA27.\\
The initial real workspace memory = la\_init\_factor * memory required by unfactored system. Ipopt will increase the workspace size by meminc\_factor if required.  This option is only available if  Ipopt has been compiled with MA27.}%
{}

\printoption{ma27\_liw\_init\_factor}%
{$1\leq\textrm{real}$}%
{$5$}%
{Integer workspace memory for MA27.\\
The initial integer workspace memory = liw\_init\_factor * memory required by unfactored system. Ipopt will increase the workspace size by meminc\_factor if required.  This option is only available if Ipopt has been compiled with MA27.}%
{}

\printoption{ma27\_meminc\_factor}%
{$1\leq\textrm{real}$}%
{$2$}%
{Increment factor for workspace size for MA27.\\
If the integer or real workspace is not large enough, Ipopt will increase its size by this factor.  This option is only available if Ipopt has been compiled with MA27.}%
{}

\printoption{ma27\_pivtol}%
{$0<\textrm{real}<1$}%
{$10^{- 8}$}%
{Pivot tolerance for the linear solver MA27.\\
A smaller number pivots for sparsity, a larger number pivots for stability.  This option is only available if Ipopt has been compiled with MA27.}%
{}

\printoption{ma27\_pivtolmax}%
{$0<\textrm{real}<1$}%
{$0.0001$}%
{Maximum pivot tolerance for the linear solver MA27.\\
Ipopt may increase pivtol as high as pivtolmax to get a more accurate solution to the linear system.  This option is only available if Ipopt has been compiled with MA27.}%
{}

\printoption{ma27\_skip\_inertia\_check}%
{\ttfamily no, yes}%
{no}%
{Always pretend inertia is correct.\\
Setting this option to "yes" essentially disables inertia check. This option makes the algorithm non-robust and easily fail, but it might give some insight into the necessity of inertia control.}%
{\begin{list}{}{
\setlength{\parsep}{0em}
\setlength{\leftmargin}{5ex}
\setlength{\labelwidth}{2ex}
\setlength{\itemindent}{0ex}
\setlength{\topsep}{0pt}}
\item[\texttt{no}] check inertia
\item[\texttt{yes}] skip inertia check
\end{list}
}

\printoptioncategory{MA28 Linear Solver}
\printoption{ma28\_pivtol}%
{$0<\textrm{real}\leq1$}%
{$0.01$}%
{Pivot tolerance for linear solver MA28.\\
This is used when MA28 tries to find the dependent constraints.}%
{}

\printoptioncategory{MA57 Linear Solver}
\printoption{ma57\_automatic\_scaling}%
{\ttfamily no, yes}%
{no}%
{Controls MA57 automatic scaling\\
This option controls the internal scaling option of MA57. For higher reliability of the MA57 solver, you may want to set this option to yes. This is ICNTL(15) in MA57.}%
{\begin{list}{}{
\setlength{\parsep}{0em}
\setlength{\leftmargin}{5ex}
\setlength{\labelwidth}{2ex}
\setlength{\itemindent}{0ex}
\setlength{\topsep}{0pt}}
\item[\texttt{no}] Do not scale the linear system matrix
\item[\texttt{yes}] Scale the linear system matrix
\end{list}
}

\printoption{ma57\_block\_size}%
{$1\leq\textrm{integer}$}%
{$16$}%
{Controls block size used by Level 3 BLAS in MA57BD\\
This is ICNTL(11) in MA57.}%
{}

\printoption{ma57\_node\_amalgamation}%
{$1\leq\textrm{integer}$}%
{$16$}%
{Node amalgamation parameter\\
This is ICNTL(12) in MA57.}%
{}

\printoption{ma57\_pivot\_order}%
{$0\leq\textrm{integer}\leq5$}%
{$5$}%
{Controls pivot order in MA57\\
This is ICNTL(6) in MA57.}%
{}

\printoption{ma57\_pivtol}%
{$0<\textrm{real}<1$}%
{$10^{- 8}$}%
{Pivot tolerance for the linear solver MA57.\\
A smaller number pivots for sparsity, a larger number pivots for stability. This option is only available if Ipopt has been compiled with MA57.}%
{}

\printoption{ma57\_pivtolmax}%
{$0<\textrm{real}<1$}%
{$0.0001$}%
{Maximum pivot tolerance for the linear solver MA57.\\
Ipopt may increase pivtol as high as ma57\_pivtolmax to get a more accurate solution to the linear system.  This option is only available if Ipopt has been compiled with MA57.}%
{}

\printoption{ma57\_pre\_alloc}%
{$1\leq\textrm{real}$}%
{$1.05$}%
{Safety factor for work space memory allocation for the linear solver MA57.\\
If 1 is chosen, the suggested amount of work space is used.  However, choosing a larger number might avoid reallocation if the suggest values do not suffice.  This option is only available if Ipopt has been compiled with MA57.}%
{}

\printoption{ma57\_small\_pivot\_flag}%
{$0\leq\textrm{integer}\leq1$}%
{$0$}%
{If set to 1, then when small entries defined by CNTL(2) are detected they are removed and the corresponding pivots placed at the end of the factorization.  This can be particularly efficient if the matrix is highly rank deficient.\\
This is ICNTL(16) in MA57.}%
{}

\printoptioncategory{MA77 Linear Solver}
\printoption{ma77\_buffer\_lpage}%
{$1\leq\textrm{integer}$}%
{$4096$}%
{Number of scalars per MA77 buffer page\\
Number of scalars per an in-core buffer in the out-of-core solver MA77. Must be at most ma77\_file\_size.}%
{}

\printoption{ma77\_buffer\_npage}%
{$1\leq\textrm{integer}$}%
{$1600$}%
{Number of pages that make up MA77 buffer\\
Number of pages of size buffer\_lpage that exist in-core for the out-of-core solver MA77.}%
{}

\printoption{ma77\_file\_size}%
{$1\leq\textrm{integer}$}%
{$2097152$}%
{Target size of each temporary file for MA77, scalars per type\\
MA77 uses many temporary files, this option controls the size of each one. It is measured in the number of entries (int or double), NOT bytes.}%
{}

\printoption{ma77\_maxstore}%
{$0\leq\textrm{integer}$}%
{$0$}%
{Maximum storage size for MA77 in-core mode\\
If greater than zero, the maximum size of factors stored in core before out-of-core mode is invoked.}%
{}

\printoption{ma77\_nemin}%
{$1\leq\textrm{integer}$}%
{$8$}%
{Node Amalgamation parameter\\
Two nodes in elimination tree are merged if result has fewer than ma77\_nemin variables.}%
{}

\printoption{ma77\_order}%
{\ttfamily amd, metis}%
{metis}%
{Controls type of ordering used by HSL\_MA77\\
This option controls ordering for the solver HSL\_MA77.}%
{\begin{list}{}{
\setlength{\parsep}{0em}
\setlength{\leftmargin}{5ex}
\setlength{\labelwidth}{2ex}
\setlength{\itemindent}{0ex}
\setlength{\topsep}{0pt}}
\item[\texttt{amd}] Use the HSL\_MC68 approximate minimum degree algorithm
\item[\texttt{metis}] Use the MeTiS nested dissection algorithm (if available)
\end{list}
}

\printoption{ma77\_print\_level}%
{$\textrm{integer}$}%
{$-1$}%
{Debug printing level for the linear solver MA77}%
{}

\printoption{ma77\_small}%
{$0\leq\textrm{real}$}%
{$10^{-20}$}%
{Zero Pivot Threshold\\
Any pivot less than ma77\_small is treated as zero.}%
{}

\printoption{ma77\_static}%
{$0\leq\textrm{real}$}%
{$0$}%
{Static Pivoting Threshold\\
See MA77 documentation. Either ma77\_static=0.0 or ma77\_static$>$ma77\_small. ma77\_static=0.0 disables static pivoting.}%
{}

\printoption{ma77\_u}%
{$0\leq\textrm{real}\leq0.5$}%
{$10^{- 8}$}%
{Pivoting Threshold\\
See MA77 documentation.}%
{}

\printoption{ma77\_umax}%
{$0\leq\textrm{real}\leq0.5$}%
{$0.0001$}%
{Maximum Pivoting Threshold\\
Maximum value to which u will be increased to improve quality.}%
{}

\printoptioncategory{MA86 Linear Solver}
\printoption{ma86\_nemin}%
{$1\leq\textrm{integer}$}%
{$32$}%
{Node Amalgamation parameter\\
Two nodes in elimination tree are merged if result has fewer than ma86\_nemin variables.}%
{}

\printoption{ma86\_order}%
{\ttfamily auto, amd, metis}%
{auto}%
{Controls type of ordering used by HSL\_MA86\\
This option controls ordering for the solver HSL\_MA86.}%
{\begin{list}{}{
\setlength{\parsep}{0em}
\setlength{\leftmargin}{5ex}
\setlength{\labelwidth}{2ex}
\setlength{\itemindent}{0ex}
\setlength{\topsep}{0pt}}
\item[\texttt{auto}] Try both AMD and MeTiS, pick best
\item[\texttt{amd}] Use the HSL\_MC68 approximate minimum degree algorithm
\item[\texttt{metis}] Use the MeTiS nested dissection algorithm (if available)
\end{list}
}

\printoption{ma86\_print\_level}%
{$\textrm{integer}$}%
{$-1$}%
{Debug printing level for the linear solver MA86}%
{}

\printoption{ma86\_scaling}%
{\ttfamily none, mc64, mc77}%
{mc64}%
{Controls scaling of matrix\\
This option controls scaling for the solver HSL\_MA86.}%
{\begin{list}{}{
\setlength{\parsep}{0em}
\setlength{\leftmargin}{5ex}
\setlength{\labelwidth}{2ex}
\setlength{\itemindent}{0ex}
\setlength{\topsep}{0pt}}
\item[\texttt{none}] Do not scale the linear system matrix
\item[\texttt{mc64}] Scale linear system matrix using MC64
\item[\texttt{mc77}] Scale linear system matrix using MC77 [1,3,0]
\end{list}
}

\printoption{ma86\_small}%
{$0\leq\textrm{real}$}%
{$10^{-20}$}%
{Zero Pivot Threshold\\
Any pivot less than ma86\_small is treated as zero.}%
{}

\printoption{ma86\_static}%
{$0\leq\textrm{real}$}%
{$0$}%
{Static Pivoting Threshold\\
See MA86 documentation. Either ma86\_static=0.0 or ma86\_static$>$ma86\_small. ma86\_static=0.0 disables static pivoting.}%
{}

\printoption{ma86\_u}%
{$0\leq\textrm{real}\leq0.5$}%
{$10^{- 8}$}%
{Pivoting Threshold\\
See MA86 documentation.}%
{}

\printoption{ma86\_umax}%
{$0\leq\textrm{real}\leq0.5$}%
{$0.0001$}%
{Maximum Pivoting Threshold\\
Maximum value to which u will be increased to improve quality.}%
{}

\printoptioncategory{MA97 Linear Solver}
\printoption{ma97\_nemin}%
{$1\leq\textrm{integer}$}%
{$8$}%
{Node Amalgamation parameter\\
Two nodes in elimination tree are merged if result has fewer than ma97\_nemin variables.}%
{}

\printoption{ma97\_order}%
{\ttfamily auto, best, amd, metis, matched-auto, matched-metis, matched-amd}%
{auto}%
{Controls type of ordering used by HSL\_MA97}%
{\begin{list}{}{
\setlength{\parsep}{0em}
\setlength{\leftmargin}{5ex}
\setlength{\labelwidth}{2ex}
\setlength{\itemindent}{0ex}
\setlength{\topsep}{0pt}}
\item[\texttt{auto}] Use HSL\_MA97 heuristic to guess best of AMD and METIS
\item[\texttt{best}] Try both AMD and MeTiS, pick best
\item[\texttt{amd}] Use the HSL\_MC68 approximate minimum degree algorithm
\item[\texttt{metis}] Use the MeTiS nested dissection algorithm
\item[\texttt{matched-auto}] Use the HSL\_MC80 matching with heuristic choice of AMD or METIS
\item[\texttt{matched-metis}] Use the HSL\_MC80 matching based ordering with METIS
\item[\texttt{matched-amd}] Use the HSL\_MC80 matching based ordering with AMD
\end{list}
}

\printoption{ma97\_print\_level}%
{$\textrm{integer}$}%
{$0$}%
{Debug printing level for the linear solver MA97}%
{}

\printoption{ma97\_scaling}%
{\ttfamily none, mc30, mc64, mc77, dynamic}%
{dynamic}%
{Specifies strategy for scaling in HSL\_MA97 linear solver}%
{\begin{list}{}{
\setlength{\parsep}{0em}
\setlength{\leftmargin}{5ex}
\setlength{\labelwidth}{2ex}
\setlength{\itemindent}{0ex}
\setlength{\topsep}{0pt}}
\item[\texttt{none}] Do not scale the linear system matrix
\item[\texttt{mc30}] Scale all linear system matrices using MC30
\item[\texttt{mc64}] Scale all linear system matrices using MC64
\item[\texttt{mc77}] Scale all linear system matrices using MC77 [1,3,0]
\item[\texttt{dynamic}] Dynamically select scaling according to rules specified by ma97\_scalingX and ma97\_switchX options.
\end{list}
}

\printoption{ma97\_scaling1}%
{\ttfamily none, mc30, mc64, mc77}%
{mc64}%
{First scaling.\\
If ma97\_scaling=dynamic, this scaling is used according to the trigger ma97\_switch1. If ma97\_switch2 is triggered it is disabled.}%
{\begin{list}{}{
\setlength{\parsep}{0em}
\setlength{\leftmargin}{5ex}
\setlength{\labelwidth}{2ex}
\setlength{\itemindent}{0ex}
\setlength{\topsep}{0pt}}
\item[\texttt{none}] No scaling
\item[\texttt{mc30}] Scale linear system matrix using MC30
\item[\texttt{mc64}] Scale linear system matrix using MC64
\item[\texttt{mc77}] Scale linear system matrix using MC77 [1,3,0]
\end{list}
}

\printoption{ma97\_scaling2}%
{\ttfamily none, mc30, mc64, mc77}%
{mc64}%
{Second scaling.\\
If ma97\_scaling=dynamic, this scaling is used according to the trigger ma97\_switch2. If ma97\_switch3 is triggered it is disabled.}%
{\begin{list}{}{
\setlength{\parsep}{0em}
\setlength{\leftmargin}{5ex}
\setlength{\labelwidth}{2ex}
\setlength{\itemindent}{0ex}
\setlength{\topsep}{0pt}}
\item[\texttt{none}] No scaling
\item[\texttt{mc30}] Scale linear system matrix using MC30
\item[\texttt{mc64}] Scale linear system matrix using MC64
\item[\texttt{mc77}] Scale linear system matrix using MC77 [1,3,0]
\end{list}
}

\printoption{ma97\_scaling3}%
{\ttfamily none, mc30, mc64, mc77}%
{mc64}%
{Third scaling.\\
If ma97\_scaling=dynamic, this scaling is used according to the trigger ma97\_switch3.}%
{\begin{list}{}{
\setlength{\parsep}{0em}
\setlength{\leftmargin}{5ex}
\setlength{\labelwidth}{2ex}
\setlength{\itemindent}{0ex}
\setlength{\topsep}{0pt}}
\item[\texttt{none}] No scaling
\item[\texttt{mc30}] Scale linear system matrix using MC30
\item[\texttt{mc64}] Scale linear system matrix using MC64
\item[\texttt{mc77}] Scale linear system matrix using MC77 [1,3,0]
\end{list}
}

\printoption{ma97\_small}%
{$0\leq\textrm{real}$}%
{$10^{-20}$}%
{Zero Pivot Threshold\\
Any pivot less than ma97\_small is treated as zero.}%
{}

\printoption{ma97\_solve\_blas3}%
{\ttfamily no, yes}%
{no}%
{Controls if blas2 or blas3 routines are used for solve}%
{\begin{list}{}{
\setlength{\parsep}{0em}
\setlength{\leftmargin}{5ex}
\setlength{\labelwidth}{2ex}
\setlength{\itemindent}{0ex}
\setlength{\topsep}{0pt}}
\item[\texttt{no}] Use BLAS2 (faster, some implementations bit incompatible)
\item[\texttt{yes}] Use BLAS3 (slower)
\end{list}
}

\printoption{ma97\_switch1}%
{\ttfamily never, at\_start, at\_start\_reuse, on\_demand, on\_demand\_reuse, high\_delay, high\_delay\_reuse, od\_hd, od\_hd\_reuse}%
{od\_hd\_reuse}%
{First switch, determine when ma97\_scaling1 is enabled.\\
If ma97\_scaling=dynamic, ma97\_scaling1 is enabled according to this condition. If ma97\_switch2 occurs this option is henceforth ignored.}%
{\begin{list}{}{
\setlength{\parsep}{0em}
\setlength{\leftmargin}{5ex}
\setlength{\labelwidth}{2ex}
\setlength{\itemindent}{0ex}
\setlength{\topsep}{0pt}}
\item[\texttt{never}] Scaling is never enabled.
\item[\texttt{at\_start}] Scaling to be used from the very start.
\item[\texttt{at\_start\_reuse}] Scaling to be used on first iteration, then reused thereafter.
\item[\texttt{on\_demand}] Scaling to be used after Ipopt request improved solution (i.e. iterative refinement has failed).
\item[\texttt{on\_demand\_reuse}] As on\_demand, but reuse scaling from previous itr
\item[\texttt{high\_delay}] Scaling to be used after more than 0.05*n delays are present
\item[\texttt{high\_delay\_reuse}] Scaling to be used only when previous itr created more that 0.05*n additional delays, otherwise reuse scaling from previous itr
\item[\texttt{od\_hd}] Combination of on\_demand and high\_delay
\item[\texttt{od\_hd\_reuse}] Combination of on\_demand\_reuse and high\_delay\_reuse
\end{list}
}

\printoption{ma97\_switch2}%
{\ttfamily never, at\_start, at\_start\_reuse, on\_demand, on\_demand\_reuse, high\_delay, high\_delay\_reuse, od\_hd, od\_hd\_reuse}%
{never}%
{Second switch, determine when ma97\_scaling2 is enabled.\\
If ma97\_scaling=dynamic, ma97\_scaling2 is enabled according to this condition. If ma97\_switch3 occurs this option is henceforth ignored.}%
{\begin{list}{}{
\setlength{\parsep}{0em}
\setlength{\leftmargin}{5ex}
\setlength{\labelwidth}{2ex}
\setlength{\itemindent}{0ex}
\setlength{\topsep}{0pt}}
\item[\texttt{never}] Scaling is never enabled.
\item[\texttt{at\_start}] Scaling to be used from the very start.
\item[\texttt{at\_start\_reuse}] Scaling to be used on first iteration, then reused thereafter.
\item[\texttt{on\_demand}] Scaling to be used after Ipopt request improved solution (i.e. iterative refinement has failed).
\item[\texttt{on\_demand\_reuse}] As on\_demand, but reuse scaling from previous itr
\item[\texttt{high\_delay}] Scaling to be used after more than 0.05*n delays are present
\item[\texttt{high\_delay\_reuse}] Scaling to be used only when previous itr created more that 0.05*n additional delays, otherwise reuse scaling from previous itr
\item[\texttt{od\_hd}] Combination of on\_demand and high\_delay
\item[\texttt{od\_hd\_reuse}] Combination of on\_demand\_reuse and high\_delay\_reuse
\end{list}
}

\printoption{ma97\_switch3}%
{\ttfamily never, at\_start, at\_start\_reuse, on\_demand, on\_demand\_reuse, high\_delay, high\_delay\_reuse, od\_hd, od\_hd\_reuse}%
{never}%
{Third switch, determine when ma97\_scaling3 is enabled.\\
If ma97\_scaling=dynamic, ma97\_scaling3 is enabled according to this condition.}%
{\begin{list}{}{
\setlength{\parsep}{0em}
\setlength{\leftmargin}{5ex}
\setlength{\labelwidth}{2ex}
\setlength{\itemindent}{0ex}
\setlength{\topsep}{0pt}}
\item[\texttt{never}] Scaling is never enabled.
\item[\texttt{at\_start}] Scaling to be used from the very start.
\item[\texttt{at\_start\_reuse}] Scaling to be used on first iteration, then reused thereafter.
\item[\texttt{on\_demand}] Scaling to be used after Ipopt request improved solution (i.e. iterative refinement has failed).
\item[\texttt{on\_demand\_reuse}] As on\_demand, but reuse scaling from previous itr
\item[\texttt{high\_delay}] Scaling to be used after more than 0.05*n delays are present
\item[\texttt{high\_delay\_reuse}] Scaling to be used only when previous itr created more that 0.05*n additional delays, otherwise reuse scaling from previous itr
\item[\texttt{od\_hd}] Combination of on\_demand and high\_delay
\item[\texttt{od\_hd\_reuse}] Combination of on\_demand\_reuse and high\_delay\_reuse
\end{list}
}

\printoption{ma97\_u}%
{$0\leq\textrm{real}\leq0.5$}%
{$10^{- 8}$}%
{Pivoting Threshold\\
See MA97 documentation.}%
{}

\printoption{ma97\_umax}%
{$0\leq\textrm{real}\leq0.5$}%
{$0.0001$}%
{Maximum Pivoting Threshold\\
See MA97 documentation.}%
{}

\printoptioncategory{Mumps Linear Solver}
\printoption{mumps\_dep\_tol}%
{$\textrm{real}$}%
{$0$}%
{Pivot threshold for detection of linearly dependent constraints in MUMPS.\\
When MUMPS is used to determine linearly dependent constraints, this is determines the threshold for a pivot to be considered zero.  This is CNTL(3) in MUMPS.}%
{}

\printoption{mumps\_mem\_percent}%
{$0\leq\textrm{integer}$}%
{$1000$}%
{Percentage increase in the estimated working space for MUMPS.\\
In MUMPS when significant extra fill-in is caused by numerical pivoting, larger values of mumps\_mem\_percent may help use the workspace more efficiently.  On the other hand, if memory requirement are too large at the very beginning of the optimization, choosing a much smaller value for this option, such as 5, might reduce memory requirements.}%
{}

\printoption{mumps\_permuting\_scaling}%
{$0\leq\textrm{integer}\leq7$}%
{$7$}%
{Controls permuting and scaling in MUMPS\\
This is ICNTL(6) in MUMPS.}%
{}

\printoption{mumps\_pivot\_order}%
{$0\leq\textrm{integer}\leq7$}%
{$7$}%
{Controls pivot order in MUMPS\\
This is ICNTL(7) in MUMPS.}%
{}

\printoption{mumps\_pivtol}%
{$0\leq\textrm{real}\leq1$}%
{$10^{- 6}$}%
{Pivot tolerance for the linear solver MUMPS.\\
A smaller number pivots for sparsity, a larger number pivots for stability.  This option is only available if Ipopt has been compiled with MUMPS.}%
{}

\printoption{mumps\_pivtolmax}%
{$0\leq\textrm{real}\leq1$}%
{$0.1$}%
{Maximum pivot tolerance for the linear solver MUMPS.\\
Ipopt may increase pivtol as high as pivtolmax to get a more accurate solution to the linear system.  This option is only available if Ipopt has been compiled with MUMPS.}%
{}

\printoption{mumps\_scaling}%
{$-2\leq\textrm{integer}\leq77$}%
{$77$}%
{Controls scaling in MUMPS\\
This is ICNTL(8) in MUMPS.}%
{}

\printoptioncategory{NLP}
\printoption{bound\_relax\_factor}%
{$0\leq\textrm{real}$}%
{$10^{-10}$}%
{Factor for initial relaxation of the bounds.\\
Before start of the optimization, the bounds given by the user are relaxed.  This option sets the factor for this relaxation.  If it is set to zero, then then bounds relaxation is disabled. (See Eqn.(35) in implementation paper.)}%
{}

\printoption{check\_derivatives\_for\_naninf}%
{\ttfamily no, yes}%
{no}%
{Indicates whether it is desired to check for Nan/Inf in derivative matrices\\
Activating this option will cause an error if an invalid number is detected in the constraint Jacobians or the Lagrangian Hessian.  If this is not activated, the test is skipped, and the algorithm might proceed with invalid numbers and fail.  If test is activated and an invalid number is detected, the matrix is written to output with print\_level corresponding to J\_MORE\_DETAILED; so beware of large output!}%
{\begin{list}{}{
\setlength{\parsep}{0em}
\setlength{\leftmargin}{5ex}
\setlength{\labelwidth}{2ex}
\setlength{\itemindent}{0ex}
\setlength{\topsep}{0pt}}
\item[\texttt{no}] Don't check (faster).
\item[\texttt{yes}] Check Jacobians and Hessian for Nan and Inf.
\end{list}
}

\printoption{dependency\_detection\_with\_rhs}%
{\ttfamily no, yes}%
{no}%
{Indicates if the right hand sides of the constraints should be considered during dependency detection}%
{\begin{list}{}{
\setlength{\parsep}{0em}
\setlength{\leftmargin}{5ex}
\setlength{\labelwidth}{2ex}
\setlength{\itemindent}{0ex}
\setlength{\topsep}{0pt}}
\item[\texttt{no}] only look at gradients
\item[\texttt{yes}] also consider right hand side
\end{list}
}

\printoption{dependency\_detector}%
{\ttfamily none, mumps, ma28}%
{none}%
{Indicates which linear solver should be used to detect linearly dependent equality constraints.\\
The default and available choices depend on how Ipopt has been compiled.  This is experimental and does not work well.}%
{\begin{list}{}{
\setlength{\parsep}{0em}
\setlength{\leftmargin}{5ex}
\setlength{\labelwidth}{2ex}
\setlength{\itemindent}{0ex}
\setlength{\topsep}{0pt}}
\item[\texttt{none}] don't check; no extra work at beginning
\item[\texttt{mumps}] use MUMPS
\item[\texttt{ma28}] use MA28
\end{list}
}

\printoption{fixed\_variable\_treatment}%
{\ttfamily make\_parameter, make\_constraint, relax\_bounds}%
{make\_parameter}%
{Determines how fixed variables should be handled.\\
The main difference between those options is that the starting point in the "make\_constraint" case still has the fixed variables at their given values, whereas in the case "make\_parameter" the functions are always evaluated with the fixed values for those variables.  Also, for "relax\_bounds", the fixing bound constraints are relaxed (according to" bound\_relax\_factor"). For both "make\_constraints" and "relax\_bounds", bound multipliers are computed for the fixed variables.}%
{\begin{list}{}{
\setlength{\parsep}{0em}
\setlength{\leftmargin}{5ex}
\setlength{\labelwidth}{2ex}
\setlength{\itemindent}{0ex}
\setlength{\topsep}{0pt}}
\item[\texttt{make\_parameter}] Remove fixed variable from optimization variables
\item[\texttt{make\_constraint}] Add equality constraints fixing variables
\item[\texttt{relax\_bounds}] Relax fixing bound constraints
\end{list}
}

\printoption{honor\_original\_bounds}%
{\ttfamily no, yes}%
{yes}%
{Indicates whether final points should be projected into original bounds.\\
Ipopt might relax the bounds during the optimization (see, e.g., option "bound\_relax\_factor").  This option determines whether the final point should be projected back into the user-provide original bounds after the optimization.}%
{\begin{list}{}{
\setlength{\parsep}{0em}
\setlength{\leftmargin}{5ex}
\setlength{\labelwidth}{2ex}
\setlength{\itemindent}{0ex}
\setlength{\topsep}{0pt}}
\item[\texttt{no}] Leave final point unchanged
\item[\texttt{yes}] Project final point back into original bounds
\end{list}
}

\printoption{jac\_c\_constant}%
{\ttfamily no, yes}%
{no}%
{Indicates whether all equality constraints are linear\\
Activating this option will cause Ipopt to ask for the Jacobian of the equality constraints only once from the NLP and reuse this information later.}%
{\begin{list}{}{
\setlength{\parsep}{0em}
\setlength{\leftmargin}{5ex}
\setlength{\labelwidth}{2ex}
\setlength{\itemindent}{0ex}
\setlength{\topsep}{0pt}}
\item[\texttt{no}] Don't assume that all equality constraints are linear
\item[\texttt{yes}] Assume that equality constraints Jacobian are constant
\end{list}
}

\printoption{jac\_d\_constant}%
{\ttfamily no, yes}%
{no}%
{Indicates whether all inequality constraints are linear\\
Activating this option will cause Ipopt to ask for the Jacobian of the inequality constraints only once from the NLP and reuse this information later.}%
{\begin{list}{}{
\setlength{\parsep}{0em}
\setlength{\leftmargin}{5ex}
\setlength{\labelwidth}{2ex}
\setlength{\itemindent}{0ex}
\setlength{\topsep}{0pt}}
\item[\texttt{no}] Don't assume that all inequality constraints are linear
\item[\texttt{yes}] Assume that equality constraints Jacobian are constant
\end{list}
}

\printoption{kappa\_d}%
{$0\leq\textrm{real}$}%
{$10^{- 5}$}%
{Weight for linear damping term (to handle one-sided bounds).\\
(see Section 3.7 in implementation paper.)}%
{}

\printoption{num\_linear\_variables}%
{$0\leq\textrm{integer}$}%
{$0$}%
{Number of linear variables\\
When the Hessian is approximated, it is assumed that the first num\_linear\_variables variables are linear.  The Hessian is then not approximated in this space.  If the get\_number\_of\_nonlinear\_variables method in the TNLP is implemented, this option is ignored.}%
{}

\printoptioncategory{NLP Scaling}
\printoption{nlp\_scaling\_constr\_target\_gradient}%
{$0\leq\textrm{real}$}%
{$0$}%
{Target value for constraint function gradient size.\\
If a positive number is chosen, the scaling factor the constraint functions is computed so that the gradient has the max norm of the given size at the starting point.  This overrides nlp\_scaling\_max\_gradient for the constraint functions.}%
{}

\printoption{nlp\_scaling\_max\_gradient}%
{$0<\textrm{real}$}%
{$100$}%
{Maximum gradient after NLP scaling.\\
This is the gradient scaling cut-off. If the maximum gradient is above this value, then gradient based scaling will be performed. Scaling parameters are calculated to scale the maximum gradient back to this value. (This is g\_max in Section 3.8 of the implementation paper.) Note: This option is only used if "nlp\_scaling\_method" is chosen as "gradient-based".}%
{}

\printoption{nlp\_scaling\_method}%
{\ttfamily none, gradient-based, equilibration-based}%
{gradient-based}%
{Select the technique used for scaling the NLP.\\
Selects the technique used for scaling the problem internally before it is solved. For user-scaling, the parameters come from the NLP. If you are using AMPL, they can be specified through suffixes ("scaling\_factor")}%
{\begin{list}{}{
\setlength{\parsep}{0em}
\setlength{\leftmargin}{5ex}
\setlength{\labelwidth}{2ex}
\setlength{\itemindent}{0ex}
\setlength{\topsep}{0pt}}
\item[\texttt{none}] no problem scaling will be performed
\item[\texttt{gradient-based}] scale the problem so the maximum gradient at the starting point is scaling\_max\_gradient
\item[\texttt{equilibration-based}] scale the problem so that first derivatives are of order 1 at random points (only available with MC19)
\end{list}
}

\printoption{nlp\_scaling\_min\_value}%
{$0\leq\textrm{real}$}%
{$10^{- 8}$}%
{Minimum value of gradient-based scaling values.\\
This is the lower bound for the scaling factors computed by gradient-based scaling method.  If some derivatives of some functions are huge, the scaling factors will otherwise become very small, and the (unscaled) final constraint violation, for example, might then be significant.  Note: This option is only used if "nlp\_scaling\_method" is chosen as "gradient-based".}%
{}

\printoption{nlp\_scaling\_obj\_target\_gradient}%
{$0\leq\textrm{real}$}%
{$0$}%
{Target value for objective function gradient size.\\
If a positive number is chosen, the scaling factor the objective function is computed so that the gradient has the max norm of the given size at the starting point.  This overrides nlp\_scaling\_max\_gradient for the objective function.}%
{}

\printoptioncategory{Output}
\printoption{inf\_pr\_output}%
{\ttfamily internal, original}%
{original}%
{Determines what value is printed in the "inf\_pr" output column.\\
Ipopt works with a reformulation of the original problem, where slacks are introduced and the problem might have been scaled.  The choice "internal" prints out the constraint violation of this formulation. With "original" the true constraint violation in the original NLP is printed.}%
{\begin{list}{}{
\setlength{\parsep}{0em}
\setlength{\leftmargin}{5ex}
\setlength{\labelwidth}{2ex}
\setlength{\itemindent}{0ex}
\setlength{\topsep}{0pt}}
\item[\texttt{internal}] max-norm of violation of internal equality constraints
\item[\texttt{original}] maximal constraint violation in original NLP
\end{list}
}

\printoption{print\_eval\_error}%
{\ttfamily no, yes}%
{yes}%
{Switch to enable printing information about function evaluation errors into the GAMS listing file.}%
{}

\printoption{print\_frequency\_iter}%
{$1\leq\textrm{integer}$}%
{$1$}%
{Determines at which iteration frequency the summarizing iteration output line should be printed.\\
Summarizing iteration output is printed every print\_frequency\_iter iterations, if at least print\_frequency\_time seconds have passed since last output.}%
{}

\printoption{print\_frequency\_time}%
{$0\leq\textrm{real}$}%
{$0$}%
{Determines at which time frequency the summarizing iteration output line should be printed.\\
Summarizing iteration output is printed if at least print\_frequency\_time seconds have passed since last output and the iteration number is a multiple of print\_frequency\_iter.}%
{}

\printoption{print\_info\_string}%
{\ttfamily no, yes}%
{no}%
{Enables printing of additional info string at end of iteration output.\\
This string contains some insider information about the current iteration.  For details, look for "Diagnostic Tags" in the Ipopt documentation.}%
{\begin{list}{}{
\setlength{\parsep}{0em}
\setlength{\leftmargin}{5ex}
\setlength{\labelwidth}{2ex}
\setlength{\itemindent}{0ex}
\setlength{\topsep}{0pt}}
\item[\texttt{no}] don't print string
\item[\texttt{yes}] print string at end of each iteration output
\end{list}
}

\printoption{print\_level}%
{$0\leq\textrm{integer}\leq12$}%
{$5$}%
{Output verbosity level.\\
Sets the default verbosity level for console output. The larger this value the more detailed is the output.}%
{}

\printoption{print\_timing\_statistics}%
{\ttfamily no, yes}%
{no}%
{Switch to print timing statistics.\\
If selected, the program will print the CPU usage (user time) for selected tasks.}%
{\begin{list}{}{
\setlength{\parsep}{0em}
\setlength{\leftmargin}{5ex}
\setlength{\labelwidth}{2ex}
\setlength{\itemindent}{0ex}
\setlength{\topsep}{0pt}}
\item[\texttt{no}] don't print statistics
\item[\texttt{yes}] print all timing statistics
\end{list}
}

\printoption{replace\_bounds}%
{\ttfamily no, yes}%
{no}%
{Indicates if all variable bounds should be replaced by inequality constraints\\
This option must be set for the inexact algorithm}%
{\begin{list}{}{
\setlength{\parsep}{0em}
\setlength{\leftmargin}{5ex}
\setlength{\labelwidth}{2ex}
\setlength{\itemindent}{0ex}
\setlength{\topsep}{0pt}}
\item[\texttt{no}] leave bounds on variables
\item[\texttt{yes}] replace variable bounds by inequality constraints
\end{list}
}

\printoption{report\_mininfeas\_solution}%
{\ttfamily no, yes}%
{no}%
{Switch to report intermediate solution with minimal constraint violation to GAMS if the final solution is not feasible.\\
This option allows to obtain the most feasible solution found by Ipopt during the iteration process, if it stops at a (locally) infeasible solution, due to a limit (time, iterations, ...), or with a failure in the restoration phase.}%
{}

\printoptioncategory{Pardiso Linear Solver}
\printoption{pardiso\_matching\_strategy}%
{\ttfamily complete, complete+2x2, constraints}%
{complete+2x2}%
{Matching strategy to be used by Pardiso\\
This is IPAR(13) in Pardiso manual.}%
{\begin{list}{}{
\setlength{\parsep}{0em}
\setlength{\leftmargin}{5ex}
\setlength{\labelwidth}{2ex}
\setlength{\itemindent}{0ex}
\setlength{\topsep}{0pt}}
\item[\texttt{complete}] Match complete (IPAR(13)=1)
\item[\texttt{complete+2x2}] Match complete+2x2 (IPAR(13)=2)
\item[\texttt{constraints}] Match constraints (IPAR(13)=3)
\end{list}
}

\printoption{pardiso\_max\_iterative\_refinement\_steps}%
{$\textrm{integer}$}%
{$1$}%
{Limit on number of iterative refinement steps.\\
The solver does not perform more than the absolute value of this value steps of iterative refinement and stops the process if a satisfactory level of accuracy of the solution in terms of backward error is achieved. If negative, the accumulation of the residue uses extended precision real and complex data types. Perturbed pivots result in iterative refinement. The solver automatically performs two steps of iterative refinements when perturbed pivots are obtained during the numerical factorization and this option is set to 0.}%
{}

\printoption{pardiso\_msglvl}%
{$0\leq\textrm{integer}$}%
{$0$}%
{Pardiso message level\\
This determines the amount of analysis output from the Pardiso solver. This is MSGLVL in the Pardiso manual.}%
{}

\printoption{pardiso\_order}%
{\ttfamily amd, one, metis, pmetis}%
{metis}%
{Controls the fill-in reduction ordering algorithm for the input matrix.}%
{\begin{list}{}{
\setlength{\parsep}{0em}
\setlength{\leftmargin}{5ex}
\setlength{\labelwidth}{2ex}
\setlength{\itemindent}{0ex}
\setlength{\topsep}{0pt}}
\item[\texttt{amd}] minimum degree algorithm
\item[\texttt{one}] undocumented
\item[\texttt{metis}] MeTiS nested dissection algorithm
\item[\texttt{pmetis}] parallel (OpenMP) version of MeTiS nested dissection algorithm
\end{list}
}

\printoption{pardiso\_redo\_symbolic\_fact\_only\_if\_inertia\_wrong}%
{\ttfamily no, yes}%
{no}%
{Toggle for handling case when elements were perturbed by Pardiso.}%
{\begin{list}{}{
\setlength{\parsep}{0em}
\setlength{\leftmargin}{5ex}
\setlength{\labelwidth}{2ex}
\setlength{\itemindent}{0ex}
\setlength{\topsep}{0pt}}
\item[\texttt{no}] Always redo symbolic factorization when elements were perturbed
\item[\texttt{yes}] Only redo symbolic factorization when elements were perturbed if also the inertia was wrong
\end{list}
}

\printoption{pardiso\_repeated\_perturbation\_means\_singular}%
{\ttfamily no, yes}%
{no}%
{Interpretation of perturbed elements.}%
{\begin{list}{}{
\setlength{\parsep}{0em}
\setlength{\leftmargin}{5ex}
\setlength{\labelwidth}{2ex}
\setlength{\itemindent}{0ex}
\setlength{\topsep}{0pt}}
\item[\texttt{no}] Don't assume that matrix is singular if elements were perturbed after recent symbolic factorization
\item[\texttt{yes}] Assume that matrix is singular if elements were perturbed after recent symbolic factorization
\end{list}
}

\printoption{pardiso\_skip\_inertia\_check}%
{\ttfamily no, yes}%
{no}%
{Always pretend inertia is correct.\\
Setting this option to "yes" essentially disables inertia check. This option makes the algorithm non-robust and easily fail, but it might give some insight into the necessity of inertia control.}%
{\begin{list}{}{
\setlength{\parsep}{0em}
\setlength{\leftmargin}{5ex}
\setlength{\labelwidth}{2ex}
\setlength{\itemindent}{0ex}
\setlength{\topsep}{0pt}}
\item[\texttt{no}] check inertia
\item[\texttt{yes}] skip inertia check
\end{list}
}

\printoptioncategory{Restoration Phase}
\printoption{bound\_mult\_reset\_threshold}%
{$0\leq\textrm{real}$}%
{$1000$}%
{Threshold for resetting bound multipliers after the restoration phase.\\
After returning from the restoration phase, the bound multipliers are updated with a Newton step for complementarity.  Here, the change in the primal variables during the entire restoration phase is taken to be the corresponding primal Newton step. However, if after the update the largest bound multiplier exceeds the threshold specified by this option, the multipliers are all reset to 1.}%
{}

\printoption{constr\_mult\_reset\_threshold}%
{$0\leq\textrm{real}$}%
{$0$}%
{Threshold for resetting equality and inequality multipliers after restoration phase.\\
After returning from the restoration phase, the constraint multipliers are recomputed by a least square estimate.  This option triggers when those least-square estimates should be ignored.}%
{}

\printoption{evaluate\_orig\_obj\_at\_resto\_trial}%
{\ttfamily no, yes}%
{yes}%
{Determines if the original objective function should be evaluated at restoration phase trial points.\\
Setting this option to "yes" makes the restoration phase algorithm evaluate the objective function of the original problem at every trial point encountered during the restoration phase, even if this value is not required.  In this way, it is guaranteed that the original objective function can be evaluated without error at all accepted iterates; otherwise the algorithm might fail at a point where the restoration phase accepts an iterate that is good for the restoration phase problem, but not the original problem.  On the other hand, if the evaluation of the original objective is expensive, this might be costly.}%
{\begin{list}{}{
\setlength{\parsep}{0em}
\setlength{\leftmargin}{5ex}
\setlength{\labelwidth}{2ex}
\setlength{\itemindent}{0ex}
\setlength{\topsep}{0pt}}
\item[\texttt{no}] skip evaluation
\item[\texttt{yes}] evaluate at every trial point
\end{list}
}

\printoption{expect\_infeasible\_problem}%
{\ttfamily no, yes}%
{no}%
{Enable heuristics to quickly detect an infeasible problem.\\
This options is meant to activate heuristics that may speed up the infeasibility determination if you expect that there is a good chance for the problem to be infeasible.  In the filter line search procedure, the restoration phase is called more quickly than usually, and more reduction in the constraint violation is enforced before the restoration phase is left. If the problem is square, this option is enabled automatically.}%
{\begin{list}{}{
\setlength{\parsep}{0em}
\setlength{\leftmargin}{5ex}
\setlength{\labelwidth}{2ex}
\setlength{\itemindent}{0ex}
\setlength{\topsep}{0pt}}
\item[\texttt{no}] the problem probably be feasible
\item[\texttt{yes}] the problem has a good chance to be infeasible
\end{list}
}

\printoption{expect\_infeasible\_problem\_ctol}%
{$0\leq\textrm{real}$}%
{$0.001$}%
{Threshold for disabling "expect\_infeasible\_problem" option.\\
If the constraint violation becomes smaller than this threshold, the "expect\_infeasible\_problem" heuristics in the filter line search are disabled. If the problem is square, this options is set to 0.}%
{}

\printoption{expect\_infeasible\_problem\_ytol}%
{$0<\textrm{real}$}%
{$10^{  8}$}%
{Multiplier threshold for activating "expect\_infeasible\_problem" option.\\
If the max norm of the constraint multipliers becomes larger than this value and "expect\_infeasible\_problem" is chosen, then the restoration phase is entered.}%
{}

\printoption{max\_resto\_iter}%
{$0\leq\textrm{integer}$}%
{$3000000$}%
{Maximum number of successive iterations in restoration phase.\\
The algorithm terminates with an error message if the number of iterations successively taken in the restoration phase exceeds this number.}%
{}

\printoption{max\_soft\_resto\_iters}%
{$0\leq\textrm{integer}$}%
{$10$}%
{Maximum number of iterations performed successively in soft restoration phase.\\
If the soft restoration phase is performed for more than so many iterations in a row, the regular restoration phase is called.}%
{}

\printoption{required\_infeasibility\_reduction}%
{$0\leq\textrm{real}<1$}%
{$0.9$}%
{Required reduction of infeasibility before leaving restoration phase.\\
The restoration phase algorithm is performed, until a point is found that is acceptable to the filter and the infeasibility has been reduced by at least the fraction given by this option.}%
{}

\printoption{resto\_failure\_feasibility\_threshold}%
{$0\leq\textrm{real}$}%
{$0$}%
{Threshold for primal infeasibility to declare failure of restoration phase.\\
If the restoration phase is terminated because of the "acceptable" termination criteria and the primal infeasibility is smaller than this value, the restoration phase is declared to have failed.  The default value is 1e2*tol, where tol is the general termination tolerance.}%
{}

\printoption{resto\_penalty\_parameter}%
{$0<\textrm{real}$}%
{$1000$}%
{Penalty parameter in the restoration phase objective function.\\
This is the parameter rho in equation (31a) in the Ipopt implementation paper.}%
{}

\printoption{resto\_proximity\_weight}%
{$0\leq\textrm{real}$}%
{$1$}%
{Weighting factor for the proximity term in restoration phase objective.\\
This determines how the parameter zera in equation (29a) in the implementation paper is computed.  zeta here is resto\_proximity\_weight*sqrt(mu), where mu is the current barrier parameter.}%
{}

\printoption{soft\_resto\_pderror\_reduction\_factor}%
{$0\leq\textrm{real}$}%
{$0.9999$}%
{Required reduction in primal-dual error in the soft restoration phase.\\
The soft restoration phase attempts to reduce the primal-dual error with regular steps. If the damped primal-dual step (damped only to satisfy the fraction-to-the-boundary rule) is not decreasing the primal-dual error by at least this factor, then the regular restoration phase is called. Choosing "0" here disables the soft restoration phase.}%
{}

\printoption{start\_with\_resto}%
{\ttfamily no, yes}%
{no}%
{Tells algorithm to switch to restoration phase in first iteration.\\
Setting this option to "yes" forces the algorithm to switch to the feasibility restoration phase in the first iteration. If the initial point is feasible, the algorithm will abort with a failure.}%
{\begin{list}{}{
\setlength{\parsep}{0em}
\setlength{\leftmargin}{5ex}
\setlength{\labelwidth}{2ex}
\setlength{\itemindent}{0ex}
\setlength{\topsep}{0pt}}
\item[\texttt{no}] don't force start in restoration phase
\item[\texttt{yes}] force start in restoration phase
\end{list}
}

\printoptioncategory{Step Calculation}
\printoption{fast\_step\_computation}%
{\ttfamily no, yes}%
{no}%
{Indicates if the linear system should be solved quickly.\\
If set to yes, the algorithm assumes that the linear system that is solved to obtain the search direction, is solved sufficiently well. In that case, no residuals are computed, and the computation of the search direction is a little faster.}%
{\begin{list}{}{
\setlength{\parsep}{0em}
\setlength{\leftmargin}{5ex}
\setlength{\labelwidth}{2ex}
\setlength{\itemindent}{0ex}
\setlength{\topsep}{0pt}}
\item[\texttt{no}] Verify solution of linear system by computing residuals.
\item[\texttt{yes}] Trust that linear systems are solved well.
\end{list}
}

\printoption{first\_hessian\_perturbation}%
{$0<\textrm{real}$}%
{$0.0001$}%
{Size of first x-s perturbation tried.\\
The first value tried for the x-s perturbation in the inertia correction scheme.(This is delta\_0 in the implementation paper.)}%
{}

\printoption{jacobian\_regularization\_exponent}%
{$0\leq\textrm{real}$}%
{$0.25$}%
{Exponent for mu in the regularization for rank-deficient constraint Jacobians.\\
(This is kappa\_c in the implementation paper.)}%
{}

\printoption{jacobian\_regularization\_value}%
{$0\leq\textrm{real}$}%
{$10^{- 8}$}%
{Size of the regularization for rank-deficient constraint Jacobians.\\
(This is bar delta\_c in the implementation paper.)}%
{}

\printoption{max\_hessian\_perturbation}%
{$0<\textrm{real}$}%
{$10^{ 20}$}%
{Maximum value of regularization parameter for handling negative curvature.\\
In order to guarantee that the search directions are indeed proper descent directions, Ipopt requires that the inertia of the (augmented) linear system for the step computation has the correct number of negative and positive eigenvalues. The idea is that this guides the algorithm away from maximizers and makes Ipopt more likely converge to first order optimal points that are minimizers. If the inertia is not correct, a multiple of the identity matrix is added to the Hessian of the Lagrangian in the augmented system. This parameter gives the maximum value of the regularization parameter. If a regularization of that size is not enough, the algorithm skips this iteration and goes to the restoration phase. (This is delta\_w\^max in the implementation paper.)}%
{}

\printoption{max\_refinement\_steps}%
{$0\leq\textrm{integer}$}%
{$10$}%
{Maximum number of iterative refinement steps per linear system solve.\\
Iterative refinement (on the full unsymmetric system) is performed for each right hand side.  This option determines the maximum number of iterative refinement steps.}%
{}

\printoption{mehrotra\_algorithm}%
{\ttfamily no, yes}%
{no}%
{Indicates if we want to do Mehrotra's algorithm.\\
If set to yes, Ipopt runs as Mehrotra's predictor-corrector algorithm. This works usually very well for LPs and convex QPs.  This automatically disables the line search, and chooses the (unglobalized) adaptive mu strategy with the "probing" oracle, and uses "corrector\_type=affine" without any safeguards; you should not set any of those options explicitly in addition.  Also, unless otherwise specified, the values of "bound\_push", "bound\_frac", and "bound\_mult\_init\_val" are set more aggressive, and sets "alpha\_for\_y=bound\_mult".}%
{\begin{list}{}{
\setlength{\parsep}{0em}
\setlength{\leftmargin}{5ex}
\setlength{\labelwidth}{2ex}
\setlength{\itemindent}{0ex}
\setlength{\topsep}{0pt}}
\item[\texttt{no}] Do the usual Ipopt algorithm.
\item[\texttt{yes}] Do Mehrotra's predictor-corrector algorithm.
\end{list}
}

\printoption{min\_hessian\_perturbation}%
{$0\leq\textrm{real}$}%
{$10^{-20}$}%
{Smallest perturbation of the Hessian block.\\
The size of the perturbation of the Hessian block is never selected smaller than this value, unless no perturbation is necessary. (This is delta\_w\^min in implementation paper.)}%
{}

\printoption{min\_refinement\_steps}%
{$0\leq\textrm{integer}$}%
{$1$}%
{Minimum number of iterative refinement steps per linear system solve.\\
Iterative refinement (on the full unsymmetric system) is performed for each right hand side.  This option determines the minimum number of iterative refinements (i.e. at least "min\_refinement\_steps" iterative refinement steps are enforced per right hand side.)}%
{}

\printoption{neg\_curv\_test\_tol}%
{$0<\textrm{real}$}%
{$0$}%
{Tolerance for heuristic to ignore wrong inertia.\\
If positive, incorrect inertia in the augmented system is ignored, and we test if the direction is a direction of positive curvature.  This tolerance determines when the direction is considered to be sufficiently positive.}%
{}

\printoption{perturb\_always\_cd}%
{\ttfamily no, yes}%
{no}%
{Active permanent perturbation of constraint linearization.\\
This options makes the delta\_c and delta\_d perturbation be used for the computation of every search direction.  Usually, it is only used when the iteration matrix is singular.}%
{\begin{list}{}{
\setlength{\parsep}{0em}
\setlength{\leftmargin}{5ex}
\setlength{\labelwidth}{2ex}
\setlength{\itemindent}{0ex}
\setlength{\topsep}{0pt}}
\item[\texttt{no}] perturbation only used when required
\item[\texttt{yes}] always use perturbation
\end{list}
}

\printoption{perturb\_dec\_fact}%
{$0<\textrm{real}<1$}%
{$0.333333$}%
{Decrease factor for x-s perturbation.\\
The factor by which the perturbation is decreased when a trial value is deduced from the size of the most recent successful perturbation. (This is kappa\_w\^- in the implementation paper.)}%
{}

\printoption{perturb\_inc\_fact}%
{$1<\textrm{real}$}%
{$8$}%
{Increase factor for x-s perturbation.\\
The factor by which the perturbation is increased when a trial value was not sufficient - this value is used for the computation of all perturbations except for the first. (This is kappa\_w\^+ in the implementation paper.)}%
{}

\printoption{perturb\_inc\_fact\_first}%
{$1<\textrm{real}$}%
{$100$}%
{Increase factor for x-s perturbation for very first perturbation.\\
The factor by which the perturbation is increased when a trial value was not sufficient - this value is used for the computation of the very first perturbation and allows a different value for for the first perturbation than that used for the remaining perturbations. (This is bar\_kappa\_w\^+ in the implementation paper.)}%
{}

\printoption{residual\_improvement\_factor}%
{$0<\textrm{real}$}%
{$1$}%
{Minimal required reduction of residual test ratio in iterative refinement.\\
If the improvement of the residual test ratio made by one iterative refinement step is not better than this factor, iterative refinement is aborted.}%
{}

\printoption{residual\_ratio\_max}%
{$0<\textrm{real}$}%
{$10^{-10}$}%
{Iterative refinement tolerance\\
Iterative refinement is performed until the residual test ratio is less than this tolerance (or until "max\_refinement\_steps" refinement steps are performed).}%
{}

\printoption{residual\_ratio\_singular}%
{$0<\textrm{real}$}%
{$10^{- 5}$}%
{Threshold for declaring linear system singular after failed iterative refinement.\\
If the residual test ratio is larger than this value after failed iterative refinement, the algorithm pretends that the linear system is singular.}%
{}

\printoptioncategory{Warm Start}
\printoption{warm\_start\_bound\_frac}%
{$0<\textrm{real}\leq0.5$}%
{$0.001$}%
{same as bound\_frac for the regular initializer.}%
{}

\printoption{warm\_start\_bound\_push}%
{$0<\textrm{real}$}%
{$0.001$}%
{same as bound\_push for the regular initializer.}%
{}

\printoption{warm\_start\_init\_point}%
{\ttfamily no, yes}%
{no}%
{Warm-start for initial point\\
Indicates whether this optimization should use a warm start initialization, where values of primal and dual variables are given (e.g., from a previous optimization of a related problem.)}%
{\begin{list}{}{
\setlength{\parsep}{0em}
\setlength{\leftmargin}{5ex}
\setlength{\labelwidth}{2ex}
\setlength{\itemindent}{0ex}
\setlength{\topsep}{0pt}}
\item[\texttt{no}] do not use the warm start initialization
\item[\texttt{yes}] use the warm start initialization
\end{list}
}

\printoption{warm\_start\_mult\_bound\_push}%
{$0<\textrm{real}$}%
{$0.001$}%
{same as mult\_bound\_push for the regular initializer.}%
{}

\printoption{warm\_start\_mult\_init\_max}%
{$\textrm{real}$}%
{$10^{  6}$}%
{Maximum initial value for the equality multipliers.}%
{}

\printoption{warm\_start\_slack\_bound\_frac}%
{$0<\textrm{real}\leq0.5$}%
{$0.001$}%
{same as slack\_bound\_frac for the regular initializer.}%
{}

\printoption{warm\_start\_slack\_bound\_push}%
{$0<\textrm{real}$}%
{$0.001$}%
{same as slack\_bound\_push for the regular initializer.}%
{}




\section{CoinScip}

GAMS/CoinScip brings the MIP solver from the Constrained Integer Programming framework SCIP to the broad audience of academic GAMS users.

The code is developed at the Konrad-Zuse-Zentrum f\"ur Informationstechnik Berlin (ZIB) and has been written primarily by T. Achterberg.
It is distributed under the ZIB Academic License.

For more information we refer to
\begin{itemize}
\item the SCIP web site \texttt{http://scip.zib.de} and
\item the Ph.D. thesis ``Constraint Integer Programming'' by Tobias Achterberg, Berlin 2007.
\end{itemize}

GAMS/CoinScip uses the COIN-OR linear solver CLP from J.J. Forrest as LP solver, see \hyperlink{sec:coincbc}{Section \ref{sec:coincbc}}.

\subsection{Model requirements}

SCIP supports continuous, binary, and integer variables, but no special ordered sets, semi-continuous or semi-integer variables (see chapter 17.1 of the GAMS User's Guide).
Branching priorities are supported.

\subsection{Usage of CoinScip}

The following statement can be used inside your GAMS program to specify using CoinScip
\begin{verbatim}
  Option MIP = CoinScip;     { or LP or RMIP }
\end{verbatim}

The above statement should appear before the Solve statement.
If CoinScip was specified as the default solver during GAMS installation, the above statement is not necessary.

GAMS/CoinScip now support the GAMS Branch-and-Cut-and-Heuristic (BCH) Facility.
The GAMS BCH facility automates all major steps necessary to define, execute, and control the use of user defined routines within the framework of general purpose MIP codes.
Currently supported are user defined cut generators and heuristics and the incumbent reporting callback.
Please see the BCH documentation at \texttt{http://www.gams.com/docs/bch.htm} for further information.

Information on the use of BCH callback routines is displayed in an extra column in the SCIP iteration output.
The first number in this column (below the ``BCH'' in the header) is the number of callbacks to GAMS models that have been made so far (accumulated from cutgeneration, heuristic, and incumbent callbacks).
The number below ``cut'' or ``cuts'' gives the number of cutting planes that have been generated by the users cutgenerator.
Finally, the number below ``sol'' or ``sols'' gives the number of primal solutions that have been generated by the users heuristic.
If SCIP accepts a heuristic solution as new incumbent solution, it prints a `G' in the first column of the iteration output.

\subsection{Specification of CoinScip Options}

GAMS/CoinScip currently supports the GAMS parameters reslim, iterlim, nodlim, optcr, and optca.

Further, for a MIP solve the user can specify options by a SCIP options file.
A SCIP options file consists of one option or comment per line.
A pound sign (\#) at the beginning of a line causes the entire line to be ignored.
Otherwise, the line will be interpreted as an option name and value separated by an equal sign (=) and any amount of white space (blanks or tabs).
Further, string values have to be enclosed in quotation marks.

A small example for a coinscip.opt file is:
\begin{verbatim}
  separating/maxrounds     = 0
  separating/maxroundsroot = 0
  gams/solvefinal          = FALSE
  gams/usercutcall         = "bchcutgen.gms"
\end{verbatim}
It causes GAMS/CoinScip to turn off all cut generators, to skip the final solve of the MIP with fixed discrete variables, and to use a user defined cut generator.

\subsection{Description of CoinScip options}

SCIP supports a large set of options.
Sample option files can be obtained from
\begin{verbatim}
     http://www.gams.com/~svigerske/scip
\end{verbatim}

Further, there is a set of options that are specific to the GAMS/CoinScip interface, most of them for control of the GAMS BCH facility.

\begin{description}
\item[\label{scipnames}\hypertarget{scipnames}
{\textbf{gams/names (\slshape{integer})}}]\hspace{1.0in}

This option causes GAMS names for the variables and equations to be loaded into SCIP.
These names will then be used for error messages, log entries, and so forth.
Turning names off may help if memory is very tight.

\textsl{(default = FALSE)}
\begin{itemize}
\item[FALSE] Do not load variable and equation names.
\item[TRUE] Load variable and equation names.
\end{itemize}


\item[\label{scipsolvefinal}\hypertarget{scipsolvefinal}
{\textbf{gams/solvefinal (\slshape{integer})}}]\hspace{1.0in}

Sometimes the solution process after the branch-and-cut that solves the problem with fixed discrete variables takes a long time and the user is interested in the primal values of the solution only.
In these cases, this option can be used to turn this final solve off.
Without the final solve no proper marginal values are available and only zeros are returned to GAMS.

\textsl{(default = TRUE)}
\begin{itemize}
\item[FALSE] Do not solve the fixed problem.
\item[TRUE] Solve the fixed problem and return duals.
\end{itemize}


\item[\label{scipmipstart}\hypertarget{scipmipstart}
{\textbf{gams/mipstart (\slshape{integer})}}]\hspace{1.0in}

This option controls the use of advanced starting values for mixed integer programs.
A setting of TRUE indicates that the variable level values should be checked to see if they provide an integer feasible solution before starting optimization.

\textsl{(default = TRUE)}
\begin{itemize}
\item[FALSE] Do not use the initial variable levels.
\item[TRUE] Try to use the initial variable levels as a MIP starting solution.
\end{itemize}


\item[\label{scipprintstat}\hypertarget{scipprintstat}
{\textbf{gams/print\_statistics (\slshape{integer})}}]\hspace{1.0in}

This option controls the printing of solve statistics after a MIP solve.
Turning on this option indicates that statistics like the number of
generated cuts of each type or the calls of heuristics are printed after the
MIP solve.

\textsl{(default = FALSE)}
\begin{itemize}
\item[FALSE] Do not print statistics.
\item[TRUE] Print statistics.
\end{itemize}


\item[\label{scipusercutcall}\hypertarget{scipusercutcall}
{\textbf{gams/usercutcall (\slshape{string})}}]\hspace{1.0in}

The GAMS command line (minus the gams executable name) to call the cut generator.


\item[\label{scipusercutfirst}\hypertarget{scipusercutfirst}
{\textbf{gams/usercutfirst (\slshape{integer})}}]\hspace{1.0in}

Calls the cut generator for the first $n$ nodes.

\textsl{(default = 10)}

\item[\label{scipusercutfreq}\hypertarget{scipusercutfreq}
{\textbf{gams/usercutfreq (\slshape{integer})}}]\hspace{1.0in}

Determines the frequency of the cut generator model calls.

\textsl{(default = 10)}

\item[\label{scipusercutinterval}\hypertarget{scipusercutinterval}
{\textbf{gams/usercutinterval (\slshape{integer})}}]\hspace{1.0in}

Determines the interval when to apply the multiplier for the frequency of the cut generator model calls.
See gams/userheurinterval for details.

\textsl{(default = 100)}

\item[\label{scipusercutmult}\hypertarget{scipusercutmult}
{\textbf{gams/usercutmult (\slshape{integer})}}]\hspace{1.0in}

Determines the multiplier for the frequency of the cut generator model calls.

\textsl{(default = 2)}

\item[\label{scipusercutnewint}\hypertarget{scipusercutnewint}
{\textbf{gams/usercutnewint (\slshape{integer})}}]\hspace{1.0in}

Calls the cut generator if the solver found a new integer feasible solution.

\textsl{(default = TRUE)}
\begin{itemize}
\item[FALSE] Do not call cut generator because a new integer feasible solution is found.
\item[TRUE] Let SCIP call the cut generator if a new integer feasible solution is found.
\end{itemize}

\item[\label{scipusergdxin}\hypertarget{scipusergdxin}
{\textbf{gams/usergdxin (\slshape{string})}}]\hspace{1.0in}

The name of the GDX file read back into SCIP.

\textsl{(default =} \verb=bchin.gdx=)

\item[\label{scipusergdxname}\hypertarget{scipusergdxname}
{\textbf{gams/usergdxname (\slshape{string})}}]\hspace{1.0in}

The name of the GDX file exported from the solver with the solution at the node.

\textsl{(default =} \verb=bchout.gdx=)

\item[\label{scipusergdxnameinc}\hypertarget{scipusergdxnameinc}
{\textbf{gams/usergdxnameinc (\slshape{string})}}]\hspace{1.0in}

The name of the GDX file exported from the solver with the incumbent solution.

\textsl{(default =} \verb=bchout_i.gdx=)

\item[\label{scipusergdxprefix}\hypertarget{scipusergdxprefix}
{\textbf{gams/usergdxprefix (\slshape{string})}}]\hspace{1.0in}

Prefixes to use for gams/usergdxin, gams/usergdxname, and gams/usergdxnameinc.


\item[\label{scipuserheurcall}\hypertarget{scipuserheurcall}
{\textbf{gams/userheurcall (\slshape{string})}}]\hspace{1.0in}

The GAMS command line (minus the gams executable name) to call the heuristic.


\item[\label{scipuserheurfirst}\hypertarget{scipuserheurfirst}
{\textbf{gams/userheurfirst (\slshape{integer})}}]\hspace{1.0in}

Calls the heuristic for the first $n$ nodes.

\textsl{(default = 10)}

\item[\label{scipuserheurfreq}\hypertarget{scipuserheurfreq}
{\textbf{gams/userheurfreq (\slshape{integer})}}]\hspace{1.0in}

Determines the frequency of the heuristic model calls.

\textsl{(default = 10)}

\item[\label{scipuserheurinterval}\hypertarget{scipuserheurinterval}
{\textbf{gams/userheurinterval (\slshape{integer})}}]\hspace{1.0in}

Determines the interval when to apply the multiplier for the frequency of the heuristic model calls.
For example, for the first 100 (gams/userheurinterval) nodes, the solver call every 10th (gams/userheurfreq) node the heuristic.
After 100 nodes, the frequency gets multiplied by 10 (gams/userheurmult), so that for the next 100 node the solver calls the heuristic every 20th node.
For nodes 200-300, the heuristic get called every 40th node, for nodes 300-400 every 80th node and after node 400 every 100th node.

\textsl{(default = 100)}

\item[\label{scipuserheurmult}\hypertarget{scipuserheurmult}
{\textbf{gams/userheurmult (\slshape{integer})}}]\hspace{1.0in}

Determines the multiplier for the frequency of the heuristic model calls.

\textsl{(default = 2)}

\item[\label{scipuserheurnewint}\hypertarget{scipuserheurnewint}
{\textbf{gams/userheurnewint (\slshape{integer})}}]\hspace{1.0in}

Calls the heuristic if the solver found a new integer feasible solution.

\textsl{(default = TRUE)}
\begin{itemize}
\item[FALSE] Do not call heuristic because a new integer feasible solution is found.
\item[TRUE] Let SCIP call the heuristic if a new integer feasible solution is found.
\end{itemize}

\item[\label{scipuserheurobjfirst}\hypertarget{scipuserheurobjfirst}
{\textbf{gams/userheurobjfirst (\slshape{integer})}}]\hspace{1.0in}

Similar to gams/userheurfirst but only calls the heuristic if the relaxed objective value promises a significant improvement of the current incumbent, i.e., the LP value of the node has to be closer to the best bound than the current incumbent.

\textsl{(default = FALSE)}

\item[\label{scipuserjobid}\hypertarget{scipuserjobid}
{\textbf{gams/userjobid (\slshape{string})}}]\hspace{1.0in}

Postfixes to use for gams/gdxname, gams/gdxnameinc, and gams/gdxin.


\item[\label{scipuserkeep}\hypertarget{scipuserkeep}
{\textbf{gams/userkeep (\slshape{integer})}}]\hspace{1.0in}

Calls gamskeep instead of gams

\textsl{(default = FALSE)}

\end{description}

\chapterend

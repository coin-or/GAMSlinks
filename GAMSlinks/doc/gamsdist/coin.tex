%%%
%%% tabular with the option-table header
%%%
\renewenvironment{option_tabular}%
{\begin{tabular}{p{.16\textwidth}p{.65\textwidth}p{.11\textwidth}}
\hline
\textbf{Option}      &
\textbf{Description} &
\textbf{Default}     \\
\hline}
{\end{tabular}}

%%%
%%% list to use inside a tabular cell
%%%
\renewenvironment{tab_list}[1]%
{\begin{list}{}{\vspace*{-1.5ex}\renewcommand{\makelabel}{\desclabel}\parsep-0.15cm\labelwidth#1\leftmargin#1\setlength{\labelsep}{\itemindent}\topsep0cm\parskip0cm\partopsep0cm}}%
{\end{list}}


\chapter{COIN-OR}
\textbf{Stefan Vigerske, Humboldt University Berlin, Germany}
\vspace{1cm}

\minitoc


\section{Introduction}

COIN-OR (\textbf{CO}mputational \textbf{IN}frastructure for \textbf{O}perations \textbf{R}esearch, \texttt{http://www.coin-or.org}) is an initiative to spur the development of open-source software for the operations research community.
One of the projects hosted at COIN-OR is the GAMSlinks project (\texttt{https://projects.coin-or.org/GAMSlinks}).
It is dedicated to the development of interfaces between GAMS and open source solvers.
Some of these links and solvers have also found their way into the regular GAMS distribution.
They are currently available for Linux (32 and 64 bit), Windows (32 and 64 bit), Sun Solaris (Intel 64 bit), and Darwin (Intel 32 and 64 bit) systems.
With the availability of source code for the GAMSlinks the user is not limited to the out of the box solvers that come with a regular GAMS distribution, but can extend and build these interfaces by themselves.

Available solvers and tools include:
\begin{itemize}
\item CoinBonmin 1.1: Basic Open-source Nonlinear Mixed Integer programming\\
(model types: LP, RMIP, MIP, DNLP, NLP, RMINLP, MINLP, QCP, RMIQCP, MIQCP)
\item CoinCbc 2.3: COIN-OR Branch and Cut\\
(model types: LP, MIP, RMIP)
\item CoinCouenne 0.2: Convex Over and Under Envelopes for Nonlinear Estimation\\
(model types: LP, RMIP, MIP, DNLP, NLP, RMINLP, MINLP, QCP, RMIQCP, MIQCP)
\item CoinGlpk 4.39: Gnu Linear Programming Kit\\
(model types: LP, MIP, RMIP)
\item CoinIpopt 3.7: Interior Point Optimizer\\
(model types: LP, RMIP, DNLP, NLP, RMINLP, QCP, RMIQCP)
\item CoinOS 2.0: Optimization Services\\
(model types: LP, RMIP, MIP, DNLP, NLP, RMINLP, MINLP, QCP, RMIQCP, MIQCP)
\item CoinScip 1.2.0: Solving Constrained Integer Programs\\
(model types: LP, MIP, RMIP)
\end{itemize}

% The open source concept opens new possibilities for our advanced users:
% 
% \begin{itemize}
% \item Make modifications to the existing links
% \item Port the links to another platform supported by GAMS
% \item Connect new solvers to GAMS/COIN-OR
% \item Extend the documentation
% \end{itemize}

For more information see the COIN-OR/GAMSlinks web site at
\texttt{https://projects.coin-or.org/GAMSlinks}.

\section{CoinBonmin and CoinBonminD}

GAMS/CoinBonmin brings the open source MINLP solver Bonmin to the broad audience of GAMS users.

Bonmin (\textbf{B}asic \textbf{O}pen-source \textbf{N}onlinear \textbf{M}ixed \textbf{In}teger programming) is an open-source solver for convex mixed-integer nonlinear programming (MINLPs).
The code is developed in a joined project of IBM and the Carnegie Mellon University.
The COIN-OR project leader for Bonmin is Pierre Bonami.

Bonmin implements five different algorithms:
\begin{itemize}
\item {B-BB} (\textbf{default}): a simple branch-and-bound algorithm based on solving a continuous nonlinear program at each node of the search tree and branching on integer variables; this algorithm is similar to the one implemented in the solver SBB
\item {B-OA}: an outer-approximation based decomposition algorithm based on iterating solving and improving of a MIP relaxation and solving NLP subproblems; this algorithm is similar to the one implemented in the solvers DICOPT and AlphaECP
\item {B-QG}: an outer-approximation based branch-and-cut algorithm (by Queseda and Grossmann) based on solving a continuous linear program at each node of the search tree, improving the linear program by outer approximation, and branching on integer variables
\item {B-Hyb}: a branch-and-bound algorithm which is a hybrid of B-BB and B-QG and is based on solving either a continuous nonlinear or a continuous linear program at each node of the search tree, improving the linear program by outer approximation, and branching on integer variables
\item {B-Ecp}: an ECP cuts based branch-and-cut algorithm a la FilMINT
\end{itemize}
The algorithms are exact when the problem is \textbf{convex}, otherwise they are heuristics.

For convex MINLPs, experiments on a reasonably large test set of problems have shown that B-Hyb is the algorithm of choice (it solved most of the problems in 3 hours of computing time).
Nevertheless, there are cases where B-OA is much faster than B-Hyb and others where B-BB is interesting.
B-QG corresponds mainly to a specific parameter setting of B-Hyb where some features are disabled.
B-Ecp has recently been added to Bonmin and is still experimental.
For \textbf{nonconvex} MINLPs, it is strongly recommend to use B-BB, even though it is a heuristic for such problems too.
% Therefore, it is the default algorithm in Bonmin.

For more information we refer to
\begin{itemize}
\item the Bonmin web site \texttt{https://projects.coin-or.org/Bonmin} and
\item the paper P. Bonami, L.T. Biegler, A.R. Conn, G. Cornuejols, I.E. Grossmann, C.D. Laird, J. Lee, A. Lodi, F. Margot, N.Sawaya and A. W\"achter, An Algorithmic Framework for Convex Mixed Integer Nonlinear Programs, \emph{Discrete Optimization} 5, 186--204, 2008, and \emph{IBM Research Report} RC23771, Oct. 2005.
\end{itemize}
Most of the Bonmin documentation in the section is taken from the Bonmin manual available on the Bonmin web site.

Gams/BonminD is an experimental in-core communication link.
It offers in-core communication between GAMS and the solver, making potentially large model scratch files unnecessary.
This links supports all features of the traditional link except for the GAMS branch-cut-heuristic facility (BCH).

\subsection{Model requirements}

Bonmin can handle mixed-integer nonlinear programming models which functions can be nonconvex, but should be twice continuously differentiable.
The Bonmin link in GAMS supports continuous, binary, and integer variables, special ordered sets, branching priorities, but no semi-continuous or semi-integer variables (see chapter 17.1 of the GAMS User's Guide).

If GAMS/CoinBonmin is called for a model with only continuous variables, the interface directly calls Ipopt.

\subsection{Usage of CoinBonmin}

The following statement can be used inside your GAMS program to specify using CoinBonmin
\begin{verbatim}
  Option MINLP = CoinBonmin;     { or LP, RMIP, MIP, DNLP, NLP, RMINLP, QCP, RMIQCP, MIQCP }
\end{verbatim}

The above statement should appear before the Solve statement.
If CoinBonmin was specified as the default solver during GAMS installation, the above statement is not necessary.

GAMS/CoinBonmin supports the GAMS Branch-and-Cut-and-Heuristic (BCH) Facility.
The GAMS BCH facility automates all major steps necessary to define, execute, and control the use of user defined routines within the framework of general purpose MIP and MINLP codes.
Currently supported are user defined cut generators and heuristics, where cut generator cannot be used in Bonmins pure B\&B algorithm (B-BB).
Please see the BCH documentation at \texttt{http://www.gams.com/docs/bch.htm} for further information.

\subsection{Specification of CoinBonmin Options}
\label{sub:bonminoptionspec}

A Bonmin option file contains both Ipopt and Bonmin options, for clarity all Bonmin options should be preceded with the prefix ``\texttt{bonmin.}'', except those corresponding to the BCH facility.
%Bonmin has many options that can be adjusted for the algorithm (see Section \ref{sub:bonminoptions}).
The scheme to name option files is the same as for all other GAMS solvers.
Specifying \texttt{optfile=1} let Gams/CoinBonmin read \texttt{coinbonmin.opt}, \texttt{optfile=2} corresponds to \texttt{coinbonmin.op2}, and so on.
The format of the option file is the same as for Ipopt (see Section \ref{sub:ipoptoptionspec}).

The most important option in Bonmin is the choice of the solution algorithm.
This can be set by using the option named \texttt{bonmin.algorithm} which can be set to \texttt{B-BB}, \texttt{B-OA}, \texttt{B-QG}, \texttt{B-Hyb}, or \texttt{B-Ecp} (its default value is \texttt{B-BB}).
Depending on the value of this option, certain other options may be available or not.

In the context of outer approximation decomposition (B-OA), several options are available for configuring the MIP subsolver CBC.
By setting the option \texttt{bonmin.milp\_subsolver} to \texttt{Cbc\_Par}, a version of CBC is chosen that can be parameterized by the user.
The options that can be set are the node-selection strategy, the number of strong-branching candidates, the number of branches before pseudo costs are to be trusted, and the frequency of some cut generators.
Their name coincide with Bonmins options, except that the ``\texttt{bonmin.}'' prefix is replaced with  ``\texttt{milp\_sub.}''.

An example of a \texttt{bonmin.opt} file is the following:
\begin{verbatim}
   bonmin.algorithm       B-Hyb
   bonmin.oa_log_level    4
   bonmin.milp_subsolver  Cbc_Par
   milp_sub.cover_cuts    0
   print_level            6
   userheurcall           "bchheur.gms reslim 10"
\end{verbatim}
This sets the algorithm to be used to the hybrid algorithm, the level of outer approximation related output to $4$, the MIP subsolver for outer approximation to a parameterized version of CBC, switches off cover cutting planes for the MIP subsolver, sets the print level for \texttt{Ipopt} to $6$, and let Bonmin call a user defined heuristic specified in the model \texttt{bchheur.gms} with a timelimit of 10 seconds.

GAMS/CoinBonmin understands currently the following GAMS parameters: reslim (time limit), iterlim (iteration limit), nodlim (node limit), cutoff, optca (absolute gap tolerance), and optcr (relative gap tolerance).
You can set them either on the command line, e.g. \verb+nodlim=1000+, or inside your GAMS program, e.g. \verb+Option nodlim=1000;+.

\subsection{Description of CoinBonmin Options}
\label{sub:bonminoptions}

The following tables gives the list of options together with their types, default values and availability in each of the four main algorithms.
The column labeled `type' indicates the type of the parameter (`F' stands for float, `I' for integer, and `S' for
string).
The column labeled `default' indicates the global default value.
Then for each of the four algorithm \texttt{B-BB}, \texttt{B-OA}, \texttt{B-QG}, and \texttt{B-Hyb}, `$+$' indicates that the option is available for that particular algorithm while `$-$' indicates that it is not.
Options that are marked with $^*$ are those that can be used to configure the MIP subsolver.

\begin{center}
\begin{tabular}{|l|r|r|r|r|r|r|}\hline
Option & type &  default & {\tt B-BB} & {\tt B-OA} & {\tt B-QG} & {\tt B-Hyb} \\
\hline
\hline
\multicolumn{1}{|c}{} & \multicolumn{6}{l|}{Algorithm choice}\\
\hline
algorithm& S& B-BB& +& +& +& +\\
\hline
\multicolumn{1}{|c}{} & \multicolumn{6}{l|}{Output}\\
\hline
bb\_log\_interval& I& 100& +&--& +& +\\
bb\_log\_level& I& 1& +&--& +& +\\
lp\_log\_level& I& 0&--&--& +& +\\
milp\_log\_level& I& 0&--& +&--& +\\
nlp\_log\_level& I& 1& +& +& +& +\\
oa\_cuts\_log\_level& I& 0&--& +& +& +\\
oa\_log\_frequency& F& 100& +& +& +& +\\
oa\_log\_level& I& 1& +& +& +& +\\
\hline
\end{tabular}

\begin{tabular}{|l|r|r|r|r|r|r|}\hline
Option & type &  default & {\tt B-BB} & {\tt B-OA} & {\tt B-QG} & {\tt B-Hyb} \\
\hline
\hline
\multicolumn{1}{|c}{} & \multicolumn{6}{l|}{Branch-and-bound options}\\
\hline
allowable\_fraction\_gap& F& optcr ($0.1$)& +& +& +& +\\
allowable\_gap& F& optca ($0$)& +& +& +& +\\
cutoff& F& cutoff or $10^{100}$& +& +& +& +\\
cutoff\_decr& F& $10^{-5}$& +& +& +& +\\
integer\_tolerance& F& $10^{-6}$& +& +& +& +\\
iteration\_limit& I& $\infty$& +& +& +& +\\
nlp\_failure\_behavior& S& stop& +& +& +& +\\
node\_comparison& S& best-bound& +& +& +& +\\
node\_limit& I& nodlim or iterlim ($\infty$)& +& +& +& +\\
num\_cut\_passes& I& 1&--& +& +& +\\
num\_cut\_passes\_at\_root& I& 20&--& +& +& +\\
number\_before\_trust& I& 8& +& +& +& +\\
number\_strong\_branch& I& 20& +& +& +& +\\
solution\_limit& I& $\infty$& +& +& +& +\\
time\_limit& F& reslim ($1000$)& +& +& +& +\\
tree\_search\_strategy& S& probed-dive& +& +& +& +\\
variable\_selection& S& strong-branching& +& +& +& +\\
\hline
\multicolumn{1}{|c}{} & \multicolumn{6}{l|}{Diving options}\\
\hline
max\_backtracks\_in\_dive& I& 5& +& +& +& +\\
max\_dive\_depth& I& $\infty$& +& +& +& +\\
stop\_diving\_on\_cutoff& S& no& +& +& +& +\\
\hline
\multicolumn{1}{|c}{} & \multicolumn{6}{l|}{ECP based strong branching}\\
\hline
ecp\_abs\_tol\_strong& F& $10^{-6}$& +& +& +& +\\
ecp\_max\_rounds\_strong& I& 0& +& +& +& +\\
ecp\_rel\_tol\_strong& F& 0.1& +& +& +& +\\
lp\_strong\_warmstart\_method& S& Basis& +& +& +& +\\
\hline
\multicolumn{1}{|c}{} & \multicolumn{6}{l|}{ECP cuts generation}\\
\hline
ecp\_abs\_tol& F& $10^{-6}$&--&--&--& +\\
ecp\_max\_rounds& I& 5&--&--&--& +\\
ecp\_propability\_factor& F& 1000&--&--&--& +\\
ecp\_rel\_tol& F& 0&--&--&--& +\\
filmint\_ecp\_cuts& I& 0&--&--&--& +\\
\hline
\multicolumn{1}{|c}{} & \multicolumn{6}{l|}{GAMS Branch Cut and Heuristic Facility}\\
\hline
usercutcall& S& none& --& +& +& +\\
usercutfirst& I& 10& --& +& +& +\\
usercutfreq& I& 10& --& +& +& +\\
usercutinterval& I& 100& --& +& +& +\\
usercutmult& I& 2& --& +& +& +\\
usercutnewint& I& 0& --& +& +& +\\
usergdxin& S& bchin.gdx& +& +& +& +\\
usergdxname& S& bchout.gdx& +& +& +& +\\
usergdxnameinc& S& bchout\_i.gdx& +& +& +& +\\
usergdxprefix& S&none & +& +& +& +\\
userheurcall& S& none& +& +& +& +\\
userheurfirst& I& 10& +& +& +& +\\
userheurfreq& I& 10& +& +& +& +\\
userheurinterval& I& 100& +& +& +& +\\
userheurmult& I& 2& +& +& +& +\\
userheurnewint& I& 0& +& +& +& +\\
userheurobjfirst& I& 0& +& +& +& +\\
userjobid& S& none& +& +& +& +\\
userkeep& I& 0& +& +& +& +\\
\hline
\end{tabular}

\begin{tabular}{|l|r|r|r|r|r|r|}\hline
Option & type &  default & {\tt B-BB} & {\tt B-OA} & {\tt B-QG} & {\tt B-Hyb} \\
\hline
\hline
\multicolumn{1}{|c}{} & \multicolumn{6}{l|}{MILP cutting planes in hybrid algorithm}\\
\hline
2mir\_cuts& I& 0&--& +&--& +\\
Gomory\_cuts& I& -5&--& +&--& +\\
clique\_cuts& I& -5&--& +&--& +\\
cover\_cuts& I& -5&--& +&--& +\\
flow\_covers\_cuts& I& -5&--& +&--& +\\
lift\_and\_project\_cuts& I& 0&--& +&--& +\\
mir\_cuts& I& -5&--& +&--& +\\
% probing\_cuts& I& -5&--& +&--& +\\
reduce\_and\_split\_cuts& I& 0&--& +&--& +\\
\hline
\multicolumn{1}{|c}{} & \multicolumn{6}{l|}{NLP interface}\\
\hline
warm\_start& S& none& +&--&--&--\\
\hline
\multicolumn{1}{|c}{} & \multicolumn{6}{l|}{NLP solution robustness}\\
\hline
max\_consecutive\_failures& I& 10& +& +& +& +\\
max\_random\_point\_radius& F& 100000& +&--&--&--\\
num\_iterations\_suspect& I& -1& +& +& +& +\\
num\_retry\_unsolved\_random\_point& I& 0& +& +& +& +\\
random\_point\_perturbation\_interval& F& 1& +&--&--&--\\
random\_point\_type& S& Jon& +&--&--&--\\
\hline
\multicolumn{1}{|c}{} & \multicolumn{6}{l|}{NLP solves in hybrid algorithm}\\
\hline
nlp\_solve\_frequency& I& 10&--&--&--& +\\
nlp\_solve\_max\_depth& I& 10&--&--&--& +\\
nlp\_solves\_per\_depth& F& $10^{30}$&--&--&--& +\\
\hline
\multicolumn{1}{|c}{} & \multicolumn{6}{l|}{Nonconvex problems}\\
\hline
max\_consecutive\_infeasible& I& 0& +& +& +& +\\
num\_resolve\_at\_infeasibles& I& 0& +&--&--&--\\
num\_resolve\_at\_node& I& 0& +&--&--&--\\
num\_resolve\_at\_root& I& 0& +&--&--&--\\
\hline
\multicolumn{1}{|c}{} & \multicolumn{6}{l|}{Outer Approximation decomposition}\\
\hline
milp\_subsolver& S& Cbc\_D&--& +&--& +\\
oa\_dec\_time\_limit& F& 30& +& +& +& +\\
\hline
\multicolumn{1}{|c}{} & \multicolumn{6}{l|}{Outer Approximation cuts generation}\\
\hline
add\_only\_violated\_oa& S& no&--& +& +& +\\
cut\_strengthening\_type& S& none&--& +& +& +\\
disjunctive\_cut\_type& S& none&--& +& +& +\\
oa\_cuts\_scope& S& global&--& +& +& +\\
tiny\_element& F& $10^{-8}$&--& +& +& +\\
very\_tiny\_element& F& $10^{-17}$&--& +& +& +\\
\hline
\multicolumn{1}{|c}{} & \multicolumn{6}{l|}{Strong branching setup}\\
\hline
candidate\_sort\_criterion& S& best-ps-cost& +& +& +& +\\
maxmin\_crit\_have\_sol& F& 0.1& +& +& +& +\\
maxmin\_crit\_no\_sol& F& 0.7& +& +& +& +\\
min\_number\_strong\_branch& I& 0& +& +& +& +\\
number\_before\_trust\_list& I& 0& +& +& +& +\\
number\_look\_ahead& I& 0& +& +& +& +\\
number\_strong\_branch\_root& I& $\infty$& +& +& +& +\\
setup\_pseudo\_frac& F& 0.5& +& +& +& +\\
trust\_strong\_branching\_for\_pseudo\_cost& S& yes& +& +& +& +\\
\hline
\end{tabular}
\end{center}

\subsubsection{Algorithm choice}
\label{sec:Bonmin_algorithm_choice}
\paragraph{algorithm:} Choice of the algorithm. $\;$ \\
 This will preset some of the options of bonmin
depending on the algorithm choice.
The default value for this string option is "B-BB".
\\ 
Possible values:
\begin{itemize}
   \item B-BB: simple branch-and-bound algorithm,
   \item B-OA: OA Decomposition algorithm,
   \item B-QG: Quesada and Grossmann branch-and-cut algorithm,
   \item B-Hyb: hybrid outer approximation based branch-and-cut,
   \item B-Ecp: ecp cuts based branch-and-cut a la FilMINT.
\end{itemize}

\subsubsection{Branch-and-bound}
\label{sec:bonmin_branch-and-bound_options}
\paragraph{allowable\_fraction\_gap:} Specify the value of relative gap under which the algorithm stops. $\;$ \\
 Stop the tree search when the gap between the
objective value of the best known solution and
the best bound on the objective of any solution
is less than this fraction of the absolute value
of the best known solution value. The valid range for this real option is 
$-1 \cdot 10^{+20} \le {\tt allowable\_fraction\_gap } \le 1 \cdot 10^{+20}$
and its default value is the value of the GAMS parameter optcr, which default value is $0.1$.


\paragraph{allowable\_gap:} Specify the value of absolute gap under which the algorithm stops. $\;$ \\
 Stop the tree search when the gap between the
objective value of the best known solution and
the best bound on the objective of any solution
is less than this. The valid range for this real option is 
$-1 \cdot 10^{+20} \le {\tt allowable\_gap } \le 1 \cdot 10^{+20}$
and its default value is the value of the GAMS parameter optca, which default value is $0$.


\paragraph{cutoff:} Specify cutoff value. $\;$ \\
 cutoff should be the value of a feasible solution
known by the user (if any). The algorithm will
only look for solutions better than cutoof. The valid range for this real option is 
$-1 \cdot 10^{+100} \le {\tt cutoff } \le 1 \cdot 10^{+100}$
and its default value is the value of the GAMS parameter cutoff, if set, otherwise it is $10^{100}$.


\paragraph{cutoff\_decr:} Specify cutoff decrement. $\;$ \\
 Specify the amount by which cutoff is decremented
below a new best upper-bound (usually a small
positive value but in non-convex problems it may
be a negative value). The valid range for this real option is 
$-1 \cdot 10^{+10} \le {\tt cutoff\_decr } \le 1 \cdot 10^{+10}$
and its default value is $1 \cdot 10^{-05}$.


\paragraph{integer\_tolerance:} Set integer tolerance. $\;$ \\
 Any number within that value of an integer is
considered integer. The valid range for this real option is 
$0 <  {\tt integer\_tolerance } <  0.5$
and its default value is $1 \cdot 10^{-06}$.


\paragraph{iteration\_limit:} Set the cumulated maximum number of iteration in the algorithm used to process nodes continuous relaxations in the branch-and-bound. $\;$ \\
 value 0 deactivates option. The valid range for this integer option is
$0 \le {\tt iteration\_limit } <  {\tt +inf}$
and its default value is $2147483647$.


\paragraph{nlp\_failure\_behavior:} Set the behavior when an NLP or a series of NLP are unsolved by Ipopt (we call unsolved an NLP for which Ipopt is not able to guarantee optimality within the specified tolerances). $\;$ \\
 If set to "fathom", the algorithm will fathom the
node when Ipopt fails to find a solution to the
nlp at that node whithin the specified
tolerances. The algorithm then becomes a
heuristic, and the user will be warned that the
solution might not be optimal.
The default value for this string option is "stop".
\\ 
Possible values:
\begin{itemize}
   \item stop: Stop when failure happens.
   \item fathom: Continue when failure happens.
\end{itemize}

\paragraph{node\_comparison:} Choose the node selection strategy. $\;$ \\
 Choose the strategy for selecting the next node
to be processed.
The default value for this string option is "dynamic".
\\ 
Possible values:
\begin{itemize}
   \item best-bound: choose node with the smallest bound,
   \item depth-first: Perform depth first search,
   \item breadth-first: Perform breadth first search,
   \item dynamic: Cbc dynamic strategy (starts with a depth first
search and turn to best bound after 3 integer
feasible solutions have been found).
   \item best-guess: choose node with smallest guessed integer
solution
\end{itemize}

\paragraph{node\_limit:} Set the maximum number of nodes explored in the branch-and-bound search. $\;$ \\
 The valid range for this integer option is
$0 \le {\tt node\_limit } <  {\tt +inf}$
and its default value is the value of the GAMS parameter nodlim, if set, otherwise it is the value of the GAMS parameter iterlim, which default value is 10000.


\paragraph{num\_cut\_passes:} Set the maximum number of cut passes at regular nodes of the branch-and-cut. $\;$ \\
 The valid range for this integer option is
$0 \le {\tt num\_cut\_passes } <  {\tt +inf}$
and its default value is $1$.


\paragraph{num\_cut\_passes\_at\_root:} Set the maximum number of cut passes at regular nodes of the branch-and-cut. $\;$ \\
 The valid range for this integer option is
$0 \le {\tt num\_cut\_passes\_at\_root } <  {\tt +inf}$
and its default value is $20$.


\paragraph{number\_before\_trust:} Set the number of branches on a variable before its pseudo costs are to be believed in dynamic strong branching. $\;$ \\
 A value of -1 disables pseudo costs. The valid range for this integer option is
$-1 \le {\tt number\_before\_trust } <  {\tt +inf}$
and its default value is $8$.


\paragraph{number\_strong\_branch:} Choose the maximum number of variables considered for strong branching. $\;$ \\
 Set the number of variables on which to do strong
branching. The valid range for this integer option is
$0 \le {\tt number\_strong\_branch } <  {\tt +inf}$
and its default value is $20$.


\paragraph{solution\_limit:} Abort after that much integer feasible solution have been found by algorithm $\;$ \\
 value 0 deactivates option The valid range for this integer option is
$0 \le {\tt solution\_limit } <  {\tt +inf}$
and its default value is $2147483647$.


% \paragraph{sos\_constraints:} Wether or not to activate SOS constraints. $\;$ \\
% The default value for this string option is "enable".
% \\ 
% Possible values:
% \begin{itemize}
%    \item enable: 
%    \item disable: 
% \end{itemize}

\paragraph{time\_limit:} Set the global maximum computation time (in seconds) for the algorithm. $\;$ \\
 The valid range for this real option is 
$0 <  {\tt time\_limit } <  {\tt +inf}$
and its default value is the value of the GAMS parameter reslim, which default value is $1000$.


\paragraph{tree\_search\_strategy:} Pick a strategy for traversing the tree $\;$ \\
 All strategies can be used in conjunction with
any of the node comparison functions.Options
which affect dfs-dive are max-backtracks-in-dive
and max-dive-depth. The dfs-dive won't work in a
non-convex problem where objective does not
decrease down branches.
The default value for this string option is "top-node".
\\ 
Possible values:
\begin{itemize}
   \item top-node:  Always pick the top node as sorted by the node
comparison function
   \item dive: Dive in the tree if possible, otherwise pick
top node as sorted by the tree comparison
function
   \item dfs-dive: Dive in the tree if possible doing a depth
first search.Backtrack on leaves or when a
prescribed depth is attained or when estimate
of best possible integer feasible solution in
subtree is worst than cutoff. Once a prescribed
limit of backtracks is attained pick top node
as sorted by the tree comparison function
   \item dfs-dive-dynamic: Same as dfs-dive but once enough solution are
found switch to best-bound and if too many
nodes switch to depth-first.
\end{itemize}

\paragraph{varselect\_stra:} Chooses variable selection strategy\\
The default value for this string option is "strong-branching".
\\ 
Possible values:
\begin{itemize}
   \item most-fractional: Choose most fractional variable
   \item strong-branching: Perform strong branching
   \item reliability-branching: Use reliability branching
   \item curvature-estimator: Use curvature estimation to select branching
variable
   \item qp-strong-branching: Perform strong branching with QP approximation
   \item lp-strong-branching: Perform strong branching with LP approximation
   \item nlp-strong-branching: Perform strong branching with NLP approximation
   \item osi-simple: Osi method to do simple branching
   \item osi-strong: Osi method to do strong branching
\end{itemize}

\subsubsection{Outer Approximation decomposition}
\label{sec:bonmin_options_:_Options_for_MILP_subsolver_in_OA_decomposition}
\label{sec:bonmin_options_:_Options_for_OA_decomposition}

\paragraph{milp\_subsolver:} Choose the subsolver to solve MIP sub-problems in OA decompositions. $\;$ \\
The default value for this string option is "Cbc\_D".
\\ 
Possible values:
\begin{itemize}
   \item Cbc\_D: Coin Branch and Cut with its default
   \item Cbc\_Par: Coin Branch and Cut with passed parameters
\end{itemize}

\paragraph{oa\_dec\_time\_limit:} Specify the maximum number of seconds spent overall in OA decomposition iterations. $\;$ \\
 The valid range for this real option is 
$0 \le {\tt oa\_dec\_time\_limit } <  {\tt +inf}$
and its default value is $30$.


\subsubsection{Outer Approximation cuts}
\label{sec:bonmin_options_:_Outer_Approximation_cuts}
\paragraph{add\_only\_violated\_oa:} Do we add all OA cuts or only the ones violated by current point? $\;$ \\
The default value for this string option is "no".
\\ 
Possible values:
\begin{itemize}
   \item no: Add all cuts
   \item yes: Add only violated Cuts
\end{itemize}

\paragraph{cut\_strengthening\_type:} Determines if and what kind of cut strengthening should be performed. $\;$ \\
The default value for this string option is "none".
\\ 
Possible values:
\begin{itemize}
   \item none: No strengthening of cuts.
   \item sglobal: Strengthen global cuts.
   \item uglobal-slocal: Unstrengthened global and strengthened local
cuts
   \item sglobal-slocal: Strengthened global and strengthened local cuts
\end{itemize}

\paragraph{disjunctive\_cut\_type:} Determine if and what kind of disjunctive cuts should be computed. $\;$ \\
The default value for this string option is "none".
\\ 
Possible values:
\begin{itemize}
   \item none: No disjunctive cuts.
   \item most-fractional: If discrete variables present, compute
disjunction for most-fractional variable
\end{itemize}

\paragraph{oa\_cuts\_scope:} Specify if OA cuts added are to be set globally or locally valid $\;$ \\
The default value for this string option is "global".
\\ 
Possible values:
\begin{itemize}
   \item local: Cuts are treated as globally valid
   \item global: Cuts are treated as locally valid
\end{itemize}

\paragraph{tiny\_element:} Value for tiny element in OA cut $\;$ \\
 We will remove "cleanly" (by relaxing cut) an
element lower than this. The valid range for this real option is 
$-0 \le {\tt tiny\_element } <  {\tt +inf}$
and its default value is $1 \cdot 10^{-08}$.


\paragraph{very\_tiny\_element:} Value for very tiny element in OA cut $\;$ \\
 Algorithm will take the risk of neglecting an
element lower than this. The valid range for this real option is 
$-0 \le {\tt very\_tiny\_element } <  {\tt +inf}$
and its default value is $1 \cdot 10^{-17}$.



\subsubsection{Ecp based strong branching}
\label{sec:Bonmin_ecp_based_strong_branching}
\label{sec:bonmin_options_:_Options_for_ecp_cuts_generation}

\paragraph{ecp\_abs\_tol:} Set the absolute termination tolerance for ECP rounds. $\;$ \\
 The valid range for this real option is 
$0 \le {\tt ecp\_abs\_tol } <  {\tt +inf}$
and its default value is $1 \cdot 10^{-06}$.

\paragraph{ecp\_abs\_tol\_strong:} Set the absolute termination tolerance for ECP rounds in strong branching. $\;$ \\
 The valid range for this real option is 
$0 \le {\tt ecp\_abs\_tol\_strong } <  {\tt +inf}$
and its default value is $1 \cdot 10^{-06}$.


\paragraph{ecp\_max\_rounds:} Set the maximal number of rounds of ECP cuts. $\;$ \\
 The valid range for this integer option is
$0 \le {\tt ecp\_max\_rounds } <  {\tt +inf}$
and its default value is $5$.

\paragraph{ecp\_max\_rounds\_strong:} Set the maximal number of rounds of ECP cuts in strong branching. $\;$ \\
 The valid range for this integer option is
$0 \le {\tt ecp\_max\_rounds\_strong } <  {\tt +inf}$
and its default value is $5$.

\paragraph{ecp\_rel\_tol:} Set the relative termination tolerance for ECP rounds. $\;$ \\
 The valid range for this real option is 
$0 \le {\tt ecp\_rel\_tol } <  {\tt +inf}$
and its default value is $0$.

\paragraph{ecp\_rel\_tol\_strong:} Set the relative termination tolerance for ECP rounds in strong branching. $\;$ \\
 The valid range for this real option is 
$0 \le {\tt ecp\_rel\_tol\_strong } <  {\tt +inf}$
and its default value is $0.1$.

\paragraph{ecp\_propability\_factor:} Factor appearing in formula for skipping ECP cuts. $\;$ \\
 Choosing -1 disables the skipping. The valid range for this real option is 
${\tt -inf} <  {\tt ecp\_propability\_factor } <  {\tt +inf}$
and its default value is $1000$.

\paragraph{filmint\_ecp\_cuts:} Specify the frequency (in terms of nodes) at which some a la filmint ecp cuts are generated. $\;$ \\
 A frequency of 0 amounts to to never solve the
NLP relaxation. The valid range for this integer option is
$0 \le {\tt filmint\_ecp\_cuts } <  {\tt +inf}$
and its default value is $0$.


\subsubsection{Output}
\label{sec:bonmin_output_options}
\paragraph{bb\_log\_interval:} Interval at which node level output is printed. $\;$ \\
 Set the interval (in terms of number of nodes) at
which a log on node resolutions (consisting of
lower and upper bounds) is given. The valid range for this integer option is
$0 \le {\tt bb\_log\_interval } <  {\tt +inf}$
and its default value is $100$.


\paragraph{bb\_log\_level:} specify main branch-and-bound log level. $\;$ \\
 Set the level of output of the branch-and-bound :
0 - none, 1 - minimal, 2 - normal low, 3 - normal
high The valid range for this integer option is
$0 \le {\tt bb\_log\_level } \le 5$
and its default value is $1$.


\paragraph{lp\_log\_level:} specify LP log level. $\;$ \\
 Set the level of output of the linear programming
sub-solver in B-Hyb or B-QG : 0 - none, 1 -
minimal, 2 - normal low, 3 - normal high, 4 -
verbose The valid range for this integer option is
$0 \le {\tt lp\_log\_level } \le 4$
and its default value is $0$.

\paragraph{milp\_log\_level:} specify MILP subsolver log level. $\;$ \\
 Set the level of output of the MILP subsolver in
OA : 0 - none, 1 - minimal, 2 - normal low, 3 -
normal high The valid range for this integer option is
$0 \le {\tt milp\_log\_level } \le 3$
and its default value is $0$.

\paragraph{nlp\_log\_level:} specify NLP solver interface log level (independent from ipopt print\_level). $\;$ \\
 Set the level of output of the OsiTMINLPInterface
: 0 - none, 1 - normal, 2 - verbose The valid range for this integer option is
$0 \le {\tt nlp\_log\_level } \le 2$
and its default value is $1$.

\paragraph{oa\_cuts\_log\_level:} level of log when generating OA cuts. $\;$ \\
 0: outputs nothings,
1: when a cut is generated,
its violation and index of row from which it
originates,
2: always output violation of the
cut.
3: output generated cuts incidence vectors. The valid range for this integer option is
$0 \le {\tt oa\_cuts\_log\_level } <  {\tt +inf}$
and its default value is $0$.

\paragraph{oa\_log\_frequency:} display an update on lower and upper bounds in OA every n seconds $\;$ \\
 The valid range for this real option is 
$0 <  {\tt oa\_log\_frequency } <  {\tt +inf}$
and its default value is $100$.


\paragraph{oa\_log\_level:} specify OA iterations log level. $\;$ \\
 Set the level of output of OA decomposition
solver : 0 - none, 1 - normal, 2 - verbose The valid range for this integer option is
$0 \le {\tt oa\_log\_level } \le 2$
and its default value is $1$.


\subsubsection{NLP interface}
\label{sec:bonmin_nlp_interface_option}
\label{sec:bonmin_options_:_Nlp_solve_options}

\paragraph{nlp\_solve\_frequency:} Specify the frequency (in terms of nodes) at which NLP relaxations are solved in B-Hyb. $\;$ \\
 A frequency of 0 amounts to to never solve the
NLP relaxation. The valid range for this integer option is
$0 \le {\tt nlp\_solve\_frequency } <  {\tt +inf}$
and its default value is $10$.


\paragraph{nlp\_solve\_max\_depth:} Set maximum depth in the tree at which NLP relaxations are solved in B-Hyb. $\;$ \\
 A depth of 0 amounts to to never solve the NLP
relaxation. The valid range for this integer option is
$0 \le {\tt nlp\_solve\_max\_depth } <  {\tt +inf}$
and its default value is $10$.

\paragraph{warm\_start:} Select the warm start method $\;$ \\
 This will affect the function getWarmStart(), and
as a consequence the warm starting in the various
algorithms.
The default value for this string option is "none".
\\ 
Possible values:
\begin{itemize}
   \item none: No warm start
   \item optimum: Warm start with direct parent optimum
   \item interior\_point: Warm start with an interior point of direct
parent
\end{itemize}


\subsubsection{MILP cutting planes}
\label{sec:bonmin_options_for_MILP_cutting_planes}

A value $k$ for the following cut generator frequency parameters is interpreted as followed by CBC:
If $k > 0$, cuts are generated every $k$ nodes, if
$-99 < k < 0$ cuts are generated every $-k$ nodes but
Cbc may decide to stop generating cuts, if not
enough are generated at the root node, if $k=-99$
generate cuts only at the root node, if $k=0$ or
$100$ do not generate cuts.
The valid range for these integer options is $[-100, \infty)$.

\paragraph{2mir\_cuts:} Frequency (in terms of nodes) for generating 2-MIR cuts in branch-and-cut $\;$ \\
The default value is $0$.

\paragraph{Gomory\_cuts:} Frequency (in terms of nodes) for generating Gomory cuts in branch-and-cut. $\;$ \\
The default value is $-5$.

\paragraph{clique\_cuts:} Frequency (in terms of nodes) for generating clique cuts in branch-and-cut $\;$ \\
The default value is $-5$.

\paragraph{cover\_cuts:} Frequency (in terms of nodes) for generating cover cuts in branch-and-cut $\;$ \\
The default value is $-5$.

\paragraph{flow\_covers\_cuts:} Frequency (in terms of nodes) for generating flow cover cuts in branch-and-cut $\;$ \\
The default value is $-5$.

\paragraph{lift\_and\_project\_cuts:} Frequency (in terms of nodes) for generating lift-and-project cuts in branch-and-cut $\;$ \\
The default value is $0$.

\paragraph{mir\_cuts:} Frequency (in terms of nodes) for generating MIR cuts in branch-and-cut $\;$ \\
The default value is $-5$.

\paragraph{probing\_cuts:} Frequency (in terms of nodes) for generating probing cuts in branch-and-cut $\;$ \\
The default value is $-5$.

\paragraph{reduce\_and\_split\_cuts:} Frequency (in terms of nodes) for generating reduce-and-split cuts in branch-and-cut $\;$ \\
The default value is $0$.

\subsubsection{Options for non-convex problems}
\label{sec:bonmin_options_for_non-convex_problems}
\paragraph{max\_consecutive\_infeasible:} Number of consecutive infeasible subproblems before aborting a branch. $\;$ \\
 Will continue exploring a branch of the tree
until "max\_consecutive\_infeasible"consecutive
problems are infeasibles by the NLP sub-solver. The valid range for this integer option is
$0 \le {\tt max\_consecutive\_infeasible } <  {\tt +inf}$
and its default value is $0$.


\paragraph{num\_resolve\_at\_infeasibles:} Number $k$ of tries to resolve an infeasible node (other than the root) of the tree with different starting point. $\;$ \\
 The algorithm will solve all the infeasible nodes
with $k$ different random starting points and
will keep the best local optimum found. The valid range for this integer option is
$0 \le {\tt num\_resolve\_at\_infeasibles } <  {\tt +inf}$
and its default value is $0$.


\paragraph{num\_resolve\_at\_node:} Number $k$ of tries to resolve a node (other than the root) of the tree with different starting point. $\;$ \\
 The algorithm will solve all the nodes with $k$
different random starting points and will keep
the best local optimum found. The valid range for this integer option is
$0 \le {\tt num\_resolve\_at\_node } <  {\tt +inf}$
and its default value is $0$.


\paragraph{num\_resolve\_at\_root:} Number $k$ of tries to resolve the root node with different starting points. $\;$ \\
 The algorithm will solve the root node with $k$
random starting points and will keep the best
local optimum found. The valid range for this integer option is
$0 \le {\tt num\_resolve\_at\_root } <  {\tt +inf}$
and its default value is $0$.


\subsubsection{Robustness}
\label{sec:bonmin_options_for_robustness}
\paragraph{max\_consecutive\_failures:} (temporarily removed) Number $n$ of consecutive unsolved problems before aborting a branch of the tree. $\;$ \\
 When $n > 0$, continue exploring a branch of the
tree until $n$ consecutive problems in the branch
are unsolved (we call unsolved a problem for
which Ipopt can not guarantee optimality within
the specified tolerances). The valid range for this integer option is
$0 \le {\tt max\_consecutive\_failures } <  {\tt +inf}$
and its default value is $10$.


\paragraph{max\_random\_point\_radius:} Set max value r for coordinate of a random point. $\;$ \\
 When picking a random point coordinate i will be
in the interval [min(max(l,-r),u-r),
max(min(u,r),l+r)] (where l is the lower bound
for the variable and u is its upper bound) The valid range for this real option is 
$0 <  {\tt max\_random\_point\_radius } <  {\tt +inf}$
and its default value is $100000$.


\paragraph{num\_iterations\_suspect:} Number of iterations over which a node is considered "suspect" (for debugging purposes only, see detailed documentation). $\;$ \\
 When the number of iterations to solve a node is
above this number, the subproblem at this node is
considered to be suspect and it will be outputed
in a file (set to -1 to deactivate this). The valid range for this integer option is
$-1 \le {\tt num\_iterations\_suspect } <  {\tt +inf}$
and its default value is $-1$.


\paragraph{num\_retry\_unsolved\_random\_point:} Number $k$ of times that the algorithm will try to resolve an unsolved NLP with a random starting point (we call unsolved an NLP for which Ipopt is not able to guarantee optimality within the specified tolerances). $\;$ \\
 When Ipopt fails to solve a continuous NLP
sub-problem, if $k > 0$, the algorithm will try
again to solve the failed NLP with $k$ new
randomly chosen starting points  or until the
problem is solved with success. The valid range for this integer option is
$0 \le {\tt num\_retry\_unsolved\_random\_point } <  {\tt +inf}$
and its default value is $0$.


\paragraph{random\_point\_perturbation\_interval:} Amount by which starting point is perturbed when choosing to pick random point by perturbating starting point $\;$ \\
 The valid range for this real option is 
$0 <  {\tt random\_point\_perturbation\_interval } <  {\tt +inf}$
and its default value is $1$.


\paragraph{random\_point\_type:} method to choose a random starting point $\;$ \\
The default value for this string option is "Jon".
\\ 
Possible values:
\begin{itemize}
   \item Jon: Choose random point uniformly between the bounds
   \item Andreas: perturb the starting point of the problem
within a prescribed interval
   \item Claudia: perturb the starting point using the
perturbation radius suffix information
\end{itemize}


\subsubsection{Diving}
\label{sec:Diving_options}
\paragraph{max\_backtracks\_in\_dive:} Set the number of backtracks in a dive when using dfs-dive tree searchstrategy. $\;$ \\
 The valid range for this integer option is
$0 \le {\tt max\_backtracks\_in\_dive } <  {\tt +inf}$
and its default value is $5$.


\paragraph{max\_dive\_depth:} When using dfs-dive search. Maximum depth to go to from the diving board (node where the diving started. $\;$ \\
 The valid range for this integer option is
$0 \le {\tt max\_dive\_depth } <  {\tt +inf}$
and its default value is $2147483647$.


\paragraph{stop\_diving\_on\_cutoff:} Flag indicating whether we stop diving based on guessed feasible objective and the current cutoff $\;$ \\
The default value for this string option is "no". Possible values are "yes" and "no".


\subsubsection{Experimental options}
\label{sec:bonmin_experimental_options}
\paragraph{maxmin\_crit\_have\_sol:} Weight towards minimum in of lower and upper branching estimates when a solution has been found. $\;$ \\
 The valid range for this real option is 
$0 \le {\tt maxmin\_crit\_have\_sol } \le 1$
and its default value is $0.1$.


\paragraph{maxmin\_crit\_no\_sol:} Weight towards minimum in of lower and upper branching estimates when no solution has been found yet. $\;$ \\
 The valid range for this real option is 
$0 \le {\tt maxmin\_crit\_no\_sol } \le 1$
and its default value is $0.7$.


\paragraph{number\_before\_trust\_list:} Set the number of branches on a variable before its pseudo costs are to be believed during setup of strong branching candidate list. $\;$ \\
 The valid range for this integer option is
$-1 \le {\tt number\_before\_trust\_list } <  {\tt +inf}$ and its default is that of "number\_before\_trust".



\paragraph{number\_strong\_branch\_root:} Maximum number of variables considered for strong branching in root node. $\;$ \\
 The valid range for this integer option is
$0 \le {\tt number\_strong\_branch\_root } <  {\tt +inf}$
and its default value is $2147483647$.


\paragraph{setup\_pseudo\_frac:} Proportion of strong branching list that has to be taken from most-integer-infeasible list. $\;$ \\
 The valid range for this real option is 
$0 \le {\tt setup\_pseudo\_frac } \le 1$
and its default value is $0.5$.


\subsubsection{Ipopt}
All Ipopt options are available in CoinBonmin, please refer to section \ref{sub:ipoptoptions} for a detailed description.
The default value of the following Ipopt parameters are changed in Gams/CoinBonmin:
\begin{itemize}
\item \texttt{mu\_strategy} and \texttt{mu\_oracle} are set, respectively, to {\tt adaptive} and {\tt probing} by default.
\item \texttt{gamma\_phi} and \texttt{gamma\_theta} are set to $10^{-8}$ and $10^{-4}$ respectively. This has the effect of reducing the size of the filter in the line search performed by Ipopt.
\item \texttt{required\_infeasibility\_reduction} is set to $0.1$.
This increases the required infeasibility reduction when Ipopt enters the restoration phase and should thus help
detect infeasible problems faster.
\item \texttt{expect\_infeasible\_problem} is set to {\tt yes} which enables some heuristics to detect infeasible problems faster.
\item \texttt{warm\_start\_init\_point} is set to {\tt yes} when a full primal/dual starting point is available (generally for all the optimizations after the continuous relaxation has been solved).
\item \texttt{print\_level} is set to $0$ by default to turn off Ipopt output.
\item \texttt{bound\_relax\_factor} is set to $10^{-10}$.
\end{itemize}

\section{CoinCbc and CoinCbcD}
\label{sec:coincbc}
\hypertarget{sec:coincbc}{}

GAMS/CoinCBC brings the open source LP/MIP solver CBC to the broad audience of GAMS users.

CBC (COIN-OR Branch and Cut) is an open-source mixed integer programming solver working with the COIN-OR LP solver CLP and the COIN-OR Cut generator library Cgl.
The code has been written primarily by John J. Forrest, who is the COIN-OR project leader for Cbc.

For more information we refer to
\begin{itemize}
\item the CBC web site \texttt{https://projects.coin-or.org/Cbc},
\item the Cgl web site \texttt{https://projects.coin-or.org/Cgl}, and
\item the CLP web site \texttt{https://projects.coin-or.org/Clp}.
\end{itemize}
Most of the CBC documentation in the section was copied from the help in the CBC standalone version.

Gams/CbcD is an experimental in-core communication link.
It offers in-core communication between GAMS and the solver, making potentially large model scratch files unnecessary.
This links supports all features of the traditional link except for the GAMS branch-cut-heuristic facility (BCH).

\subsection{Model requirements}

The CBC link in GAMS supports continuous, binary, integer, semicontinuous, semiinteger variables, special ordered sets of type 1 and 2, and branching priorities (see chapter 17.1 of the GAMS User's Guide).
% Quadratic objective functions are not supported yet.

\subsection{Usage of CoinCbc}

The following statement can be used inside your GAMS program to specify using CoinCBC
\begin{verbatim}
  Option LP = CoinCbc;     { or MIP or RMIP }
\end{verbatim}

The above statement should appear before the Solve statement.
If CoinCBC was specified as the default solver during GAMS installation, the above statement is not necessary.

There are many parameters which can affect the performance the CBCs Branch and Cut Algorithm.
First just try with default settings and look carefully at the log file.
Did cuts help? Did they take too long? Look at the output to see which cuts were effective and then do some tuning (see the option \hyperlink{cuts}{cuts}).
If the \hyperlink{preprocess}{preprocessing} reduced the size of the problem or strengthened many coefficients then it is probably wise to leave it on.
Switch off \hyperlink{heuristics}{heuristics} which did not provide solutions.
The other major area to look at is the search. Hopefully good solutions were obtained fairly early in the search so the important point is to select the best variable to branch on.
See whether strong branching did a good job - or did it just take a lot of iterations.
Adjust the options \hyperlink{strongbranching}{strongbranching} and \hyperlink{trustpseudocosts}{trustpseudocosts}.

The GAMS/CoinCBC options file consists of one option or comment per line.
An asterisk (*) at the beginning of a line causes the entire line to be ignored.
Otherwise, the line will be interpreted as an option name and value separated by any amount of white space (blanks or tabs).
Following is an example options file coincbc.opt.
\begin{verbatim}
  cuts root
  perturbation off
\end{verbatim}
It will cause CoinCBC to use cut generators only in the root node and turns off the perturbation of the LP relaxation.

GAMS/CoinCBC supports the GAMS Branch-and-Cut-and-Heuristic (BCH) Facility.
The GAMS BCH facility automates all major steps necessary to define, execute, and control the use of user defined routines within the framework of general purpose MIP codes.
Currently supported are user defined cut generators and heuristics.
Please see the BCH documentation at \texttt{http://www.gams.com/docs/bch.htm} for further information.

% \subsection{CoinCbc output}
% \subsection{Some CoinCbc features}

\subsection{Summary of CoinCbc Options}

Among all Cbc options, the following GAMS parameters are currently supported in CoinCbc:
\hyperlink{reslim}{reslim}, \hyperlink{iterlim}{iterlim}, \hyperlink{nodlim}{nodlim}, \hyperlink{optca}{optca}, \hyperlink{optcr}{optcr}, \hyperlink{cutoff}{cutoff}.

Currently CoinCbc understands the following options:
\begin{tabbing}
\hspace {1.1in} \= \\
\hyperlink{roundingheuristic}
{roundingheuristic} \> rounding heuristic \\
\hyperlink{localsearch}
{localsearch} \> local search \\
\hyperlink{strongbranching}
{strongbranching} \> strong branching \\
\hyperlink{integerpresolve}
{integerpresolve} \> integer presolve \\
\hyperlink{findsos}
{findsos} \> find sos in integer presolve \\
\hyperlink{cuts}
{cuts} \> switch for cut generation \\
\hyperlink{probing}
{probing} \> probing \\
\hyperlink{gomorycuts}
{gomorycuts} \> Gomory cuts \\
\hyperlink{knapsackcuts}
{knapsackcuts} \> knapsack cover cuts \\
\hyperlink{oddholecuts}
{oddholecuts} \> odd hole cuts \\
\hyperlink{cliquecuts}
{cliquecuts} \> clique cuts \\
\hyperlink{flowcovercuts}
{flowcovercuts} \> flow cover cuts \\
\hyperlink{mircuts}
{mircuts} \> mixed integer rounding cuts \\
\hyperlink{redsplitcuts}
{redsplitcuts} \> reduce and split cuts \\
\hyperlink{cutsonlyatroot}
{cutsonlyatroot} \> whether cuts are only generated at the root node \\
\hyperlink{startalg}
{startalg} \> LP solver for root node \\
\hyperlink{writemps}
{writemps} \> create MPS file for problem \\
\hyperlink{tol_dual}
{tol\_dual} \> dual feasibility tolerance \\
\hyperlink{tol_primal}
{tol\_primal} \> primal feasibility tolerance \\
\hyperlink{tol_integer}
{tol\_integer} \> tolerance for integrality \\
\hyperlink{scaling}
{scaling} \> switch for scaling of LP \\
\hyperlink{presolve}
{scaling} \> switch for initial presolve of LP \\
\hyperlink{printfrequency}
{printfrequency} \> print frequency \\
\hyperlink{nodecompare}
{nodecompare} \> comparision method to determine tree search order \\
\hyperlink{reslim}
{reslim} \> resource limit \\
\hyperlink{iterlim}
{iterlim} \> iteration limit \\
\hyperlink{nodelim}
{nodelim} \> node limit \\
\hyperlink{nodlim}
{nodlim} \> node limit \\
\hyperlink{optca}
{optca} \> absolute stopping tolerance \\
\hyperlink{optcr}
{optcr} \> relative stopping tolerance \\
\hyperlink{cutoff}
{cutoff} \> cutoff for objective function value \\
\end{tabbing}


\subsection{Detailed Descriptions of CoinCbc Options}

\begin{description}

\item[\label{roundingheuristic}\hypertarget{roundingheuristic}
{\textbf{roundingheuristic (\slshape{integer})}}]\hspace{1.0in}

A simple rounding heuristic to find feasible solutions of the MIP.

\textsl{(default = 1)}

\item[\label{localsearch}\hypertarget{localsearch}
{\textbf{localsearch (\slshape{integer})}}]\hspace{1.0in}

A local search heuristic to find feasible solutions of the MIP.

\textsl{(default = 1)}
\begin{itemize}
\item[0] local search heuristic is not used
\item[1] local search heuristic is used
\end{itemize}

\item[\label{strongbranching}\hypertarget{strongbranching}
{\textbf{strongbranching (\slshape{integer})}}]\hspace{1.0in}

If strong branching is switched on, then the maximum number of candidates to be evaluated for strong branching is set to 10 (if the problem has less then 5000 variables) or 5 (if the problem has more then 5000 variables).
If strong branching is switched off, the maximum number is set to 0. (However, Cbc reports still some (but less) strong branching operations.)

\textsl{(default = 1)}
\begin{itemize}
\item[0] strong branching is switched off
\item[1] strong branching is switched on
\end{itemize}

\item[\label{integerpresolve}\hypertarget{integerpresolve}
{\textbf{integerpresolve (\slshape{integer})}}]\hspace{1.0in}

If switched on, a preprocessing on the MIP will be performed.
The integer presolve also does probing, independent of the setting of the 'probing' parameter.\\
From the CglPreprocess description:
This method uses other cut generators to strengthen cuts, establish that some cuts are redundant, fix variables and find relationships such as $x + y = 1$.
While cuts can be added at any time in the tree, some cuts are actually just stronger versions of existing ones.
They can thus replace the existing cuts rather than being added as new cuts. This is awkward in the tree but reasonable at the root node.

\textsl{(default = 1)}
\begin{itemize}
\item[0] don't do integer presolve
\item[1] do integer presolve
\end{itemize}

\item[\label{findsos}\hypertarget{findsos}
{\textbf{findsos (\slshape{integer})}}]\hspace{1.0in}

This option switches the identification for special ordered sets of type 1 in the integer preprocessing on.
Cbc then searches for rows with upper bound 1 and where all nonzero coefficients are 1.

\textsl{(default = 1)}
\begin{itemize}
\item[0] don't try to find special ordered sets
\item[1] try to find special ordered sets
\end{itemize}

\item[\label{cuts}\hypertarget{cuts}
{\textbf{cuts (\slshape{integer})}}]\hspace{1.0in}

Regulates the use of cut generator routines in Cbc.\\
If set to automatic, then the parameters for the single cut generators determine which are added.

\textsl{(default = 0)}
\begin{itemize}
\item[-1] no cuts will be generated
\item[0] automatic
\item[1] cuts from all available cut classes will be generated
\end{itemize}

\item[\label{probing}\hypertarget{probing}
{\textbf{probing (\slshape{integer})}}]\hspace{1.0in}

This option switches probing on as cut generator.\\
From the CglProbing description:
For selected integer variables (e.g. unsatisfied ones) the effect of setting them up or down is investigated.
Setting a variable up may in turn set other variables (continuous as well as integer).

\textsl{(default = 1)}
\begin{itemize}
\item[0] don't do probing
\item[1] do probing
\end{itemize}

\item[\label{gomorycuts}\hypertarget{gomorycuts}
{\textbf{gomorycuts (\slshape{integer})}}]\hspace{1.0in}

This options switches the generation of mixed integer Gomory Cuts on.

\textsl{(default = 1)}
\begin{itemize}
\item[0] don't add gomory cuts
\item[1] add gomory cuts
\end{itemize}

\item[\label{knapsackcuts}\hypertarget{knapsackcuts}
{\textbf{knapsackcuts (\slshape{integer})}}]\hspace{1.0in}

This options switches the generation of knapsack cover cuts on.\\
From the CglKnapsackCover description:
CglKnapsackCover looks for a series of different types of minimal covers.
If a minimal cover is found, it lifts the associated minimal cover inequality and adds the lifted cut to the cut set.

\textsl{(default = 1)}
\begin{itemize}
\item[0] don't add knapsack cover cuts
\item[1] add knapsack cover cuts
\end{itemize}

\item[\label{oddholecuts}\hypertarget{oddholecuts}
{\textbf{oddholecuts (\slshape{integer})}}]\hspace{1.0in}

This options switches the generation of odd hole cuts on.\\
From the CglOddHole description:
This looks at all rows of type $\sum_i x_i \leq 1 \textrm{(or = 1)}$ (with $x$ binary) and sees if there is an odd cycle cut.
This is then lifted by using the corresponding Chvatal cut, i.e. by summing up all rows in the cycle.
The right hand side will be odd and all odd coefficients can be reduced by one.
The constraint is
$\sum_j \textrm{even}(j)\,x_j \leq \textrm{odd}$
which can be replaced by
$\sum_j \frac{1}{2}\textrm{even}(j)\,x_j \leq \frac{1}{2}(\textrm{odd}-1)$.
A similar cut can be generated for $\sum_i x_i \geq 1$.

\textsl{(default = 0)}
\begin{itemize}
\item[0] don't add odd hole cuts
\item[1] add odd hole cuts
\end{itemize}

\item[\label{cliquecuts}\hypertarget{cliquecuts}
{\textbf{cliquecuts (\slshape{integer})}}]\hspace{1.0in}

This options switches the generation of clique cuts on.
Clique cuts are of the form ''sum of a set of variables $\leq$ 1''.

\textsl{(default = 1)}
\begin{itemize}
\item[0] don't add clique cuts
\item[1] add clique cuts
\end{itemize}

\item[\label{flowcovercuts}\hypertarget{flowcovercuts}
{\textbf{flowcovercuts (\slshape{integer})}}]\hspace{1.0in}

This options switches the generation of flow cover cuts on.\\
From the CglFlowCover description:
The Cgl Flow Cover Cut generator generates lifted simple generalized flow cover inequalities.
Since flow cover inequalities are generally not facet-defining, they are lifted to obtain stronger inequalities.
Although flow cover inequalities requires a special problem structure to be generated, they are quite useful for solving general mixed integer linear programs.

\textsl{(default = 1)}
\begin{itemize}
\item[0] don't add flow cover cuts
\item[1] add flow cover cuts
\end{itemize}

\item[\label{mircuts}\hypertarget{mircuts}
{\textbf{mircuts (\slshape{integer})}}]\hspace{1.0in}

This options switches the generation of mixed integer rounding cuts on.

\textsl{(default = 1)}
\begin{itemize}
\item[0] don't add mixed integer rounding cuts
\item[1] add mixed integer rounding cuts
\end{itemize}

\item[\label{redsplitcuts}\hypertarget{redsplitcuts}
{\textbf{redsplitcuts (\slshape{integer})}}]\hspace{1.0in}

This options switches the generation of reduce and split cuts on.\\
From the CglRedSplit description:
Reduce-and-Split cuts are variants of Gomory cuts:
Starting from the current optimal tableau, linear combinations of the rows of the current optimal simplex tableau are used for generating Gomory cuts.
The choice of the linear combinations is driven by the objective of reducing the coefficients of the non basic continuous variables in the resulting row.

\textsl{(default = 0)}
\begin{itemize}
\item[0] don't add reduce and split cuts
\item[1] add reduce and split cuts
\end{itemize}

\item[\label{cutsonlyatroot}\hypertarget{cutsonlyatroot}
{\textbf{cutsonlyatroot (\slshape{integer})}}]\hspace{1.0in}

This option determines whether cuts should be generated only at the root node.

\textsl{(default = 1)}
\begin{itemize}
\item[0] generate cuts always in the branch and bound
\item[1] generate cuts only at root node
\end{itemize}

\item[\label{startalg}\hypertarget{startalg}
{\textbf{startalg (\slshape{string})}}]\hspace{1.0in}

This option determines whether a primal or dual simplex algorithm should be used to solve the root node.

\textsl{(default = dual)}
\begin{itemize}
\item[primal] primal simplex algorithm
\item[dual] dual simplex algorithm
\end{itemize}

\item[\label{writemps}\hypertarget{writemps}
{\textbf{writemps (\slshape{string})}}]\hspace{1.0in}

Write an MPS problem file.
The parameter value is the name of the MPS file.

\item[\label{tol_dual}\hypertarget{tol_dual}
{\textbf{tol\_dual (\slshape{real})}}]\hspace{1.0in}

The maximum amount the dual constraints can be violated and still be considered feasible.

\textsl{(default = 1e-7)}

\item[\label{tol_primal}\hypertarget{tol_primal}
{\textbf{tol\_primal (\slshape{real})}}]\hspace{1.0in}

The maximum amount the primal constraints can be violated and still be considered feasible.

\textsl{(default = 1e-7)}

\item[\label{tol_integer}\hypertarget{tol_integer}
{\textbf{tol\_integer (\slshape{real})}}]\hspace{1.0in}

An integer variable is deemed to be at an integral value if it is no further than the value of this parameter from the next integral value away.

\textsl{(default = 1e-6)}


\item[\label{scaling}\hypertarget{scaling}
{\textbf{scaling (\slshape{integer})}}]\hspace{1.0in}

This option determines whether the linear relaxation should be scaled.

\textsl{(default = 1)}

\item[\label{presolve}\hypertarget{presolve}
{\textbf{presolve (\slshape{integer})}}]\hspace{1.0in}

This option determines whether a presolve should be applied to the linear relaxation before it is solved the first time.

\textsl{(default = 1)}

\item[\label{printfrequency}\hypertarget{printfrequency}
{\textbf{printfrequency (\slshape{integer})}}]\hspace{1.0in}

This option controls the number of nodes evaluated between status prints.

\textsl{(default = 10)}

\item[\label{nodecompare}\hypertarget{nodecompare}
{\textbf{nodecompare (\slshape{string})}}]\hspace{1.0in}

This option determines the comparision method that is used to determine the tree search order.
The following documentation is taken from the Cbc manual.

\textsl{(default = default)}
\begin{itemize}
\item[default] 
This is designed to do a mostly depth-first search until a solution has been found.
It then use estimates that are designed to give a slightly better solution.
If a reasonable number of nodes have been explored (or a reasonable number of solutions found),
then this class will adopt a breadth-first search (i.e., making a comparison based strictly on objective function values) unless the tree is very large, in which case it will revert to depth-first search.
\item[depth] 
This will always choose the node deepest in tree.
It gives minimum tree size but may take a long time to find the best solution.
\item[objective] 
This will always choose the node with the best objective value.
This may give a very large tree.
It is likely that the first solution found will be the best and the search should finish soon after the first solution is found.
\end{itemize}

\item[\label{reslim}\hypertarget{reslim}
{\textbf{reslim (\slshape{real})}}]\hspace{1.0in}

Maximum time in seconds.

\textsl{(default = GAMS reslim)}

\item[\label{iterlim}\hypertarget{iterlim}
{\textbf{iterlim (\slshape{integer})}}]\hspace{1.0in}

Maximum number of iterations.

\textsl{(default = GAMS iterlim)}

\item[\label{nodelim}\hypertarget{nodelim}
{\textbf{nodelim (\slshape{integer})}}]\hspace{1.0in}

Maximum number of nodes in the Branch and Bound.

\textsl{(default = GAMS nodlim)}

\item[\label{nodlim}\hypertarget{nodlim}
{\textbf{nodlim (\slshape{integer})}}]\hspace{1.0in}

Maximum number of nodes in the Branch and Bound.
This option is overwritten by nodelim, if set.

\textsl{(default = GAMS nodlim)}

\item[\label{optca}\hypertarget{optca}
{\textbf{optca (\slshape{real})}}]\hspace{1.0in}

Absolute optimality criterion for a MIP.
Cbc stops if the gap between the best known solution and the best possible solution is less than this value.

\textsl{(default = GAMS optca)}

\item[\label{optcr}\hypertarget{optcr}
{\textbf{optcr (\slshape{real})}}]\hspace{1.0in}

Relative optimality criterion for a MIP.
Cbc stops if the relative gap between the best known solution and the best possible solution is less than this value.

\textsl{(default = GAMS optcr)}

\item[\label{cutoff}\hypertarget{cutoff}
{\textbf{cutoff (\slshape{real})}}]\hspace{1.0in}

Cbc stops if the objective function values exceeds (in case of maximization) or falls below (in case of minimization) this value.

\textsl{(default = GAMS cutoff)}
\end{description}


\section{CoinCouenne}

GAMS/CoinCouenne brings the open source MINLP solver Couenne to the broad audience of GAMS users.

Couenne (\textbf{C}onvex \textbf{O}ver and \textbf{U}nder \textbf{En}velopes for \textbf{N}onlinear \textbf{E}stimation) is an open-source solver for nonconvex mixed-integer nonlinear programming (MINLPs).
The code is developed in a joined project of IBM, Carnegie Mellon University, and Lehigh University.
The COIN-OR project leader for Couenne is Pietro Belotti.

Couenne solves convex and nonconvex MINLPs by an LP based spatial branch-and-bound algorithm that is similar to BARON.
The implementation extends Bonmin by routines to compute valid linear outer approximations for nonconvex problems and methods for bound tightening and branching on nonlinear continuous variables.

For more information we refer to
\begin{itemize}
\item the Couenne web site \texttt{https://projects.coin-or.org/Couenne} and
\item the paper P. Belotti, J. Lee, L. Liberti, F. Margot and A. Waechter, Branching and bounds tightening techniques for non-convex MINLP, \emph{IBM Research Report} RC24620, 2008.
\end{itemize}
Most of the Couenne documentation in the section is taken from the Couenne manual available on the Couenne web site.

\subsection{Model requirements}

Couenne can handle mixed-integer nonlinear programming models which functions can be nonconvex, but should be twice continuously differentiable.
The Couenne link in GAMS supports continuous, binary, and integer variables, but no special ordered sets, semi-continuous or semi-integer variables (see chapter 17.1 of the GAMS User's Guide).

If GAMS/CoinCouenne is called for a linear model, the interface directly calls CBC.

\subsection{Usage of CoinCouenne}

The following statement can be used inside your GAMS program to specify using CoinCouenne
\begin{verbatim}
  Option MINLP = CoinCouenne;     { or LP, RMIP, MIP, DNLP, NLP, RMINLP, QCP, RMIQCP, MIQCP }
\end{verbatim}

The above statement should appear before the Solve statement.
If CoinCouenne was specified as the default solver during GAMS installation, the above statement is not necessary.

\subsection{Specification of CoinCouenne Options}
\label{sub:couenneoptionspec}

A Couenne option file contains Ipopt, Bonmin, and Couenne options, for clarity all Bonmin options should be preceded with the prefix ``\texttt{bonmin.}'' and all Couenne options should be preceded with the prefix ``\texttt{couenne.}''.
All Ipopt and many Bonmin options are available in CoinCouenne, please refer to sections \ref{sub:ipoptoptions} and \ref{sub:bonminoptions} for a detailed description.
The scheme to name option files is the same as for all other GAMS solvers.
Specifying \texttt{optfile=1} let Gams/CoinCouenne read \texttt{coincouenne.opt}, \texttt{optfile=2} corresponds to \texttt{coincouenne.op2}, and so on.
The format of the option file is the same as for Ipopt (see Section \ref{sub:ipoptoptionspec}).

GAMS/CoinCouenne understands currently the following GAMS parameters: reslim (time limit), nodlim (node limit), cutoff, optca (absolute gap tolerance), and optcr (relative gap tolerance).
You can set them either on the command line, e.g. \verb+nodlim=1000+, or inside your GAMS program, e.g. \verb+Option nodlim=1000;+.

\subsection{Description of CoinCouenne Options}
\label{sub:couenneoptions}

The following tables gives the list of options together with their types, default values and availability in each of the four main algorithms.
The column labeled `type' indicates the type of the parameter (`F' stands for float, `I' for integer, and `S' for
string).
The column labeled `default' indicates the global default value.
% Then for each of the four Bonmin algorithms \texttt{B-BB}, \texttt{B-OA}, \texttt{B-QG}, and \texttt{B-Hyb}, `$+$' indicates that the option is available for that particular algorithm while `$-$' indicates that it is not.

\begin{center}
\begin{tabular}{|l|r|r|}\hline
Option & type &  default \\
\hline
\hline
\multicolumn{3}{|c|}{Output options}\\
\hline
boundtightening\_print\_level& I& 0\\
branching\_print\_level& I& 0\\
convexifying\_print\_level& I& 0\\
disjcuts\_print\_level& I& 4\\
nlpheur\_print\_level& I& 4\\
problem\_print\_level& I& 2\\
reformulate\_print\_level& I& 4\\\hline
\multicolumn{3}{|c|}{Reformulation and Linearization options}\\
\hline
convexification\_cuts& I& 1\\
convexification\_type& S& current-point-only\\
convexification\_points& I& 4\\
violated\_cuts\_only& S& yes\\
delete\_redundant& S& yes\\
use\_quadratic& S& no\\\hline
\end{tabular}

\begin{tabular}{|l|r|r|}\hline
Option & type &  default \\
\hline
\hline
\multicolumn{3}{|c|}{Branching options}\\
\hline
cont\_var\_priority& I& 2000\\
branch\_conv\_cuts& S& yes\\
branch\_fbbt& S& yes\\
branch\_pt\_select& S& mid-point\\
branch\_pt\_select\_cube& S& branch\_pt\_select\\
branch\_pt\_select\_div& S& branch\_pt\_select\\
branch\_pt\_select\_exp& S& branch\_pt\_select\\
branch\_pt\_select\_log& S& branch\_pt\_select\\
branch\_pt\_select\_negpow& S& branch\_pt\_select\\
branch\_pt\_select\_pow& S& branch\_pt\_select\\
branch\_pt\_select\_prod& S& branch\_pt\_select\\
branch\_pt\_select\_sqr& S& branch\_pt\_select\\
branch\_pt\_select\_trig& S& branch\_pt\_select\\
branch\_lp\_clamp& F& 0.2\\
branch\_lp\_clamp\_cube& F& 0.2\\
branch\_lp\_clamp\_div& F& 0.2\\
branch\_lp\_clamp\_exp& F& 0.2\\
branch\_lp\_clamp\_log& F& 0.2\\
branch\_lp\_clamp\_negpow& F& 0.2\\
branch\_lp\_clamp\_pow& F& 0.2\\
branch\_lp\_clamp\_prod& F& 0.2\\
branch\_lp\_clamp\_sqr& F& 0.2\\
branch\_lp\_clamp\_trig& F& 0.2\\
branch\_midpoint\_alpha& F& 0.25\\
branching\_object& S& var\_obj\\
red\_cost\_branching& S& no\\
pseudocost\_mult& S& interval\_br\_rev\\
pseudocost\_mult\_lp& S& no\\\hline
\multicolumn{3}{|c|}{Bound tightening options}\\
\hline
feasibility\_bt& S& yes\\
aggressive\_fbbt& S& yes\\
optimality\_bt& S& yes\\
redcost\_bt& S& yes\\
% enable\_lp\_implied\_bounds& S& no\\
log\_num\_abt\_per\_level& I& 2\\
log\_num\_obbt\_per\_level& I& 1\\\hline
\multicolumn{3}{|c|}{Disjunctive cut options}\\
\hline
minlp\_disj\_cuts& I& 0\\
disj\_depth\_level& I& 5\\
disj\_depth\_stop& I& 20\\
disj\_active\_cols& S& no\\
disj\_active\_rows& S& no\\
disj\_cumulative& S& no\\
disj\_init\_number& I& 10\\
disj\_init\_perc& F& 0.5\\\hline
\multicolumn{3}{|c|}{Nonlinear solver options}\\
\hline
local\_optimization\_heuristic& S& yes\\
log\_num\_local\_optimization\_per\_level& I& 2\\\hline
\multicolumn{3}{|c|}{Tolerance options}\\
\hline
feas\_tolerance& F& $10^{-5}$\\\hline
\end{tabular}

\begin{tabular}{|l|r|r|}\hline
Option & type &  default \\
\hline
\hline
\multicolumn{3}{|c|}{MIP cut generators options}\\
\hline
Gomory\_cuts& I& 0\\
clique\_cuts& I& 0\\
cover\_cuts& I& 0\\
flow\_covers\_cuts& I& 0\\
lift\_and\_project\_cuts& I& 0\\
mir\_cuts& I& 0\\
2mir\_cuts& I& 0\\
probing\_cuts& I& 0\\
reduce\_split\_cuts& I& 0\\
% art\_cutoff& F& DBL\_MAX\\
% art\_lower& F& -DBL\_MAX\\
% check\_lp& S& no\\
% couenne\_check& F& DBL\_MAX\\
% display\_stats& S& no\\
% enable\_sos& S& no\\
% opt\_window& F& DBL\_MAX\\
% test\_mode& S& no\\
\hline
\end{tabular}
\end{center}

\printoption{2mir\_cuts}%
{$-100\leq\textrm{integer}$}%
{$0$}%
{Frequency k (in terms of nodes) for generating 2mir\_cuts cuts in branch-and-cut.\\
If k $>$ 0, cuts are generated every k nodes, if -99 $<$ k $<$ 0 cuts are generated every -k nodes but Cbc may decide to stop generating cuts, if not enough are generated at the root node, if k=-99 generate cuts only at the root node, if k=0 or 100 do not generate cuts.}%
{}

\printoption{Gomory\_cuts}%
{$-100\leq\textrm{integer}$}%
{$0$}%
{Frequency k (in terms of nodes) for generating Gomory\_cuts cuts in branch-and-cut.\\
See option \texttt{2mir\_cuts} for the meaning of k.}%
{}

\printoption{aggressive\_fbbt}%
{\ttfamily no, yes}%
{yes}%
{Aggressive feasibility-based bound tightening (to use with NLP points)\\
Aggressive FBBT is a version of probing that also allows to reduce the solution set, although it is not as quick as FBBT. It can be applied up to a certain depth of the B\&B tree -- see ``log\_num\_abt\_per\_level''. In general, this option is useful but can be switched off if a problem is too large and seems not to benefit from it.}%
{}

\printoption{art\_cutoff}%
{$\textrm{real}$}%
{$\infty$}%
{Artificial cutoff\\
Default value is infinity.}%
{}

\printoption{art\_lower}%
{$\textrm{real}$}%
{$-\infty$}%
{Artificial lower bound\\
Default value is -COIN\_DBL\_MAX.}%
{}

\printoption{boundtightening\_print\_level}%
{$-2\leq\textrm{integer}\leq12$}%
{$0$}%
{Output level for bound tightening code in Couenne}%
{}

\printoption{branch\_conv\_cuts}%
{\ttfamily no, yes}%
{yes}%
{Apply convexification cuts before branching (for now only within strong branching)\\
After applying a branching rule and before resolving the subproblem, generate a round of linearization cuts with the new bounds enforced by the rule.}%
{}

\printoption{branch\_fbbt}%
{\ttfamily no, yes}%
{yes}%
{Apply bound tightening before branching\\
After applying a branching rule and before re-solving the subproblem, apply Bound Tightening.}%
{}

\printoption{branch\_lp\_clamp}%
{$0\leq\textrm{real}\leq1$}%
{$0.2$}%
{Defines safe interval percentage for using LP point as a branching point.}%
{}

\printoption{branch\_lp\_clamp\_cube}%
{$0\leq\textrm{real}\leq0.5$}%
{$0.2$}%
{Defines safe interval percentage [0,0.5] for using LP point as a branching point.}%
{}

\printoption{branch\_lp\_clamp\_div}%
{$0\leq\textrm{real}\leq0.5$}%
{$0.2$}%
{Defines safe interval percentage [0,0.5] for using LP point as a branching point.}%
{}

\printoption{branch\_lp\_clamp\_exp}%
{$0\leq\textrm{real}\leq0.5$}%
{$0.2$}%
{Defines safe interval percentage [0,0.5] for using LP point as a branching point.}%
{}

\printoption{branch\_lp\_clamp\_log}%
{$0\leq\textrm{real}\leq0.5$}%
{$0.2$}%
{Defines safe interval percentage [0,0.5] for using LP point as a branching point.}%
{}

\printoption{branch\_lp\_clamp\_negpow}%
{$0\leq\textrm{real}\leq0.5$}%
{$0.2$}%
{Defines safe interval percentage [0,0.5] for using LP point as a branching point.}%
{}

\printoption{branch\_lp\_clamp\_pow}%
{$0\leq\textrm{real}\leq0.5$}%
{$0.2$}%
{Defines safe interval percentage [0,0.5] for using LP point as a branching point.}%
{}

\printoption{branch\_lp\_clamp\_prod}%
{$0\leq\textrm{real}\leq0.5$}%
{$0.2$}%
{Defines safe interval percentage [0,0.5] for using LP point as a branching point.}%
{}

\printoption{branch\_lp\_clamp\_sqr}%
{$0\leq\textrm{real}\leq0.5$}%
{$0.2$}%
{Defines safe interval percentage [0,0.5] for using LP point as a branching point.}%
{}

\printoption{branch\_lp\_clamp\_trig}%
{$0\leq\textrm{real}\leq0.5$}%
{$0.2$}%
{Defines safe interval percentage [0,0.5] for using LP point as a branching point.}%
{}

\printoption{branch\_midpoint\_alpha}%
{$0\leq\textrm{real}\leq1$}%
{$0.25$}%
{Defines convex combination of mid point and current LP point: b = alpha x\_lp + (1-alpha) (lb+ub)/2.}%
{}

\printoption{branch\_pt\_select}%
{\ttfamily lp-clamped, lp-central, balanced, min-area, mid-point, no-branch}%
{mid-point}%
{Chooses branching point selection strategy}%
{\begin{list}{}{
\setlength{\parsep}{0em}
\setlength{\leftmargin}{5ex}
\setlength{\labelwidth}{2ex}
\setlength{\itemindent}{0ex}
\setlength{\topsep}{0pt}}
\item[\texttt{lp-clamped}] LP point clamped in [k,1-k] of the bound intervals (k defined by lp\_clamp)
\item[\texttt{lp-central}] LP point if within [k,1-k] of the bound intervals, middle point otherwise(k defined by branch\_lp\_clamp)
\item[\texttt{balanced}] minimizes max distance from curve to convexification
\item[\texttt{min-area}] minimizes total area of the two convexifications
\item[\texttt{mid-point}] convex combination of current point and mid point
\item[\texttt{no-branch}] do not branch, return null infeasibility; for testing purposes only
\end{list}
}

\printoption{branch\_pt\_select\_cube}%
{\ttfamily common, lp-clamped, lp-central, balanced, min-area, mid-point, no-branch}%
{common}%
{Chooses branching point selection strategy for operator cube.\\
Default is to use the value of \texttt{branch\_pt\_select} (value \texttt{common}).}%
{}

\printoption{branch\_pt\_select\_div}%
{\ttfamily common, lp-clamped, lp-central, balanced, min-area, mid-point, no-branch}%
{common}%
{Chooses branching point selection strategy for operator div.\\
Default is to use the value of \texttt{branch\_pt\_select} (value \texttt{common}).}%
{}

\printoption{branch\_pt\_select\_exp}%
{\ttfamily common, lp-clamped, lp-central, balanced, min-area, mid-point, no-branch}%
{common}%
{Chooses branching point selection strategy for operator exp.\\
Default is to use the value of \texttt{branch\_pt\_select} (value \texttt{common}).}%
{}

\printoption{branch\_pt\_select\_log}%
{\ttfamily common, lp-clamped, lp-central, balanced, min-area, mid-point, no-branch}%
{common}%
{Chooses branching point selection strategy for operator log.\\
Default is to use the value of \texttt{branch\_pt\_select} (value \texttt{common}).}%
{}

\printoption{branch\_pt\_select\_negpow}%
{\ttfamily common, lp-clamped, lp-central, balanced, min-area, mid-point, no-branch}%
{common}%
{Chooses branching point selection strategy for operator negpow.\\
Default is to use the value of \texttt{branch\_pt\_select} (value \texttt{common}).}%
{}

\printoption{branch\_pt\_select\_pow}%
{\ttfamily common, lp-clamped, lp-central, balanced, min-area, mid-point, no-branch}%
{common}%
{Chooses branching point selection strategy for operator pow.\\
Default is to use the value of \texttt{branch\_pt\_select} (value \texttt{common}).}%
{}

\printoption{branch\_pt\_select\_prod}%
{\ttfamily common, lp-clamped, lp-central, balanced, min-area, mid-point, no-branch}%
{common}%
{Chooses branching point selection strategy for operator prod.\\
Default is to use the value of \texttt{branch\_pt\_select} (value \texttt{common}).}%
{}

\printoption{branch\_pt\_select\_sqr}%
{\ttfamily common, lp-clamped, lp-central, balanced, min-area, mid-point, no-branch}%
{common}%
{Chooses branching point selection strategy for operator sqr.\\
Default is to use the value of \texttt{branch\_pt\_select} (value \texttt{common}).}%
{}

\printoption{branch\_pt\_select\_trig}%
{\ttfamily common, lp-clamped, lp-central, balanced, min-area, mid-point, no-branch}%
{common}%
{Chooses branching point selection strategy for operator trig.\\
Default is to use the value of \texttt{branch\_pt\_select} (value \texttt{common}).}%
{}

\printoption{branching\_object}%
{\ttfamily vt\_obj, var\_obj, expr\_obj}%
{var\_obj}%
{type of branching object for variable selection}%
{\begin{list}{}{
\setlength{\parsep}{0em}
\setlength{\leftmargin}{5ex}
\setlength{\labelwidth}{2ex}
\setlength{\itemindent}{0ex}
\setlength{\topsep}{0pt}}
\item[\texttt{vt\_obj}] use Violation Transfer from Tawarmalani and Sahinidis
\item[\texttt{var\_obj}] use one object for each variable
\item[\texttt{expr\_obj}] use one object for each nonlinear expression
\end{list}
}

\printoption{branching\_print\_level}%
{$-2\leq\textrm{integer}\leq12$}%
{$0$}%
{Output level for braching code in Couenne}%
{}

\printoption{check\_lp}%
{\ttfamily no, yes}%
{no}%
{Check all LPs through an independent call to OsiClpSolverInterface::initialSolve()}%
{}

\printoption{clique\_cuts}%
{$-100\leq\textrm{integer}$}%
{$0$}%
{Frequency k (in terms of nodes) for generating clique\_cuts cuts in branch-and-cut.\\
See option \texttt{2mir\_cuts} for the meaning of k.}%
{}

\printoption{cont\_var\_priority}%
{$1\leq\textrm{integer}$}%
{$2000$}%
{Priority of continuous variable branching\\
When branching, this is compared to the priority of integer variables, whose priority is given by int\_var\_priority, and SOS, whose priority is 10. Higher values mean smaller priority.}%
{}

\printoption{convexification\_cuts}%
{$-99\leq\textrm{integer}$}%
{$1$}%
{Specify the frequency (in terms of nodes) at which couenne ecp cuts are generated.\\
A frequency of 0 amounts to never solve the NLP relaxation.}%
{}

\printoption{convexification\_points}%
{$0\leq\textrm{integer}$}%
{$4$}%
{Specify the number of points at which to convexify when convexification type is uniform-grid or around-current-point.}%
{}

\printoption{convexification\_type}%
{\ttfamily current-point-only, uniform-grid, around-current-point}%
{current-point-only}%
{Determines in which point the linear over/under-estimator are generated\\
For the lower envelopes of convex functions, this is the number of points where a supporting hyperplane is generated. This only holds for the initial linearization, as all other linearizations only add at most one cut per expression.}%
{\begin{list}{}{
\setlength{\parsep}{0em}
\setlength{\leftmargin}{5ex}
\setlength{\labelwidth}{2ex}
\setlength{\itemindent}{0ex}
\setlength{\topsep}{0pt}}
\item[\texttt{current-point-only}] Only at current optimum of relaxation
\item[\texttt{uniform-grid}] Points chosen in a uniform grid between the bounds of the problem
\item[\texttt{around-current-point}] At points around current optimum of relaxation
\end{list}
}

\printoption{convexifying\_print\_level}%
{$-2\leq\textrm{integer}\leq12$}%
{$0$}%
{Output level for convexifying code in Couenne}%
{}

\printoption{cover\_cuts}%
{$-100\leq\textrm{integer}$}%
{$0$}%
{Frequency k (in terms of nodes) for generating cover\_cuts cuts in branch-and-cut.\\
See option \texttt{2mir\_cuts} for the meaning of k.}%
{}

\printoption{delete\_redundant}%
{\ttfamily no, yes}%
{yes}%
{Eliminate redundant variables, which appear in the problem as x\_k = x\_h}%
{\begin{list}{}{
\setlength{\parsep}{0em}
\setlength{\leftmargin}{5ex}
\setlength{\labelwidth}{2ex}
\setlength{\itemindent}{0ex}
\setlength{\topsep}{0pt}}
\item[\texttt{no}] Keep redundant variables, making the problem a bit larger
\item[\texttt{yes}] Eliminate redundant variables (the problem will be equivalent, only smaller)
\end{list}
}

\printoption{disj\_active\_cols}%
{\ttfamily yes, no}%
{no}%
{Only include violated variable bounds in the Cut Generating LP (CGLP).\\
This reduces the size of the CGLP, but may produce less efficient cuts.}%
{}

\printoption{disj\_active\_rows}%
{\ttfamily yes, no}%
{no}%
{Only include violated linear inequalities in the CGLP.\\
This reduces the size of the CGLP, but may produce less efficient cuts.}%
{}

\printoption{disj\_cumulative}%
{\ttfamily yes, no}%
{no}%
{Add previous disjunctive cut to current CGLP.\\
When generating disjunctive cuts on a set of disjunctions 1, 2, ..., k, introduce the cut relative to the previous disjunction i-1 in the CGLP used for disjunction i. Notice that, although this makes the cut generated more efficient, it increases the rank of the disjunctive cut generated.}%
{}

\printoption{disj\_depth\_level}%
{$-1\leq\textrm{integer}$}%
{$5$}%
{Depth of the B\&B tree when to start decreasing the number of objects that generate disjunctions.\\
This has a similar behavior as log\_num\_obbt\_per\_level. A value of -1 means that generation can be done at all nodes.}%
{}

\printoption{disj\_depth\_stop}%
{$-1\leq\textrm{integer}$}%
{$20$}%
{Depth of the B\&B tree where separation of disjunctive cuts is stopped.\\
A value of -1 means that generation can be done at all nodes}%
{}

\printoption{disj\_init\_number}%
{$-1\leq\textrm{integer}$}%
{$10$}%
{Maximum number of disjunction to consider at each iteration.\\
-1 means no limit.}%
{}

\printoption{disj\_init\_perc}%
{$0\leq\textrm{real}\leq1$}%
{$0.5$}%
{The maximum fraction of all disjunctions currently violated by the problem to consider for generating disjunctions.}%
{}

\printoption{disjcuts\_print\_level}%
{$-2\leq\textrm{integer}\leq12$}%
{$0$}%
{Output level for disjunctive cuts in Couenne}%
{}

\printoption{display\_stats}%
{\ttfamily yes, no}%
{no}%
{display statistics at the end of the run}%
{}

\printoption{enable\_lp\_implied\_bounds}%
{\ttfamily no, yes}%
{no}%
{Enable OsiSolverInterface::tightenBounds () -- warning: it has caused some trouble to Couenne}%
{}

\printoption{enable\_sos}%
{\ttfamily no, yes}%
{no}%
{Use Special Ordered Sets (SOS) as indicated in the MINLP model}%
{}

\printoption{estimate\_select}%
{\ttfamily normal, product}%
{normal}%
{How the min/max estimates of the subproblems' bounds are used in strong branching}%
{\begin{list}{}{
\setlength{\parsep}{0em}
\setlength{\leftmargin}{5ex}
\setlength{\labelwidth}{2ex}
\setlength{\itemindent}{0ex}
\setlength{\topsep}{0pt}}
\item[\texttt{normal}] as usual in literature
\item[\texttt{product}] use their product
\end{list}
}

\printoption{feas\_pump\_convcuts}%
{\ttfamily integrated, external, postcut, none}%
{none}%
{Separate MILP-feasible, MINLP-infeasible solution during or after MILP solver.}%
{\begin{list}{}{
\setlength{\parsep}{0em}
\setlength{\leftmargin}{5ex}
\setlength{\labelwidth}{2ex}
\setlength{\itemindent}{0ex}
\setlength{\topsep}{0pt}}
\item[\texttt{integrated}] Done within the MILP solver in a branch-and-cut fashion
\item[\texttt{external}] Done after the MILP solver, in a Benders-like fashion
\item[\texttt{postcut}] Do one round of cuts and proceed with NLP
\item[\texttt{none}] Just proceed to the NLP
\end{list}
}

\printoption{feas\_pump\_heuristic}%
{\ttfamily no, yes}%
{no}%
{Apply the nonconvex Feasibility Pump\\
An implementation of the Feasibility Pump for nonconvex MINLPs}%
{}

\printoption{feas\_pump\_iter}%
{$-1\leq\textrm{integer}$}%
{$10$}%
{Number of iterations in the main Feasibility Pump loop\\
-1 means no limit}%
{}

\printoption{feas\_pump\_level}%
{$-1\leq\textrm{integer}$}%
{$3$}%
{Specify the logarithm of the number of feasibility pumps to perform on average for each level of given depth of the tree.\\
Solve as many nlp's at the nodes for each level of the tree. Nodes are randomly selected. If for a given level there are less nodes than this number nlp are solved for every nodes. For example if parameter is 8, nlp's are solved for all node until level 8, then for half the node at level 9, 1/4 at level 10.... Set to -1 to perform at all nodes.}%
{}

\printoption{feas\_pump\_milpmethod}%
{$-1\leq\textrm{integer}\leq4$}%
{$-1$}%
{How should the integral solution be constructed?\\
0: automatic, 1: aggressive heuristics, large node limit, 2: default, node limit, 3: RENS, 4: Objective Feasibility Pump,  -1: solve MILP completely}%
{}

\printoption{feas\_pump\_mult\_dist\_milp}%
{$0\leq\textrm{real}\leq1$}%
{$0$}%
{Weight of the distance in the distance function of the milp problem\\
0: no weight, 1: full weight}%
{}

\printoption{feas\_pump\_mult\_dist\_nlp}%
{$0\leq\textrm{real}\leq1$}%
{$0$}%
{Weight of the distance in the distance function of the nlp problem\\
0: no weight, 1: full weight}%
{}

\printoption{feas\_pump\_mult\_hess\_milp}%
{$0\leq\textrm{real}\leq1$}%
{$0$}%
{Weight of the Hessian in the distance function of the milp problem\\
0: no weight, 1: full weight}%
{}

\printoption{feas\_pump\_mult\_hess\_nlp}%
{$0\leq\textrm{real}\leq1$}%
{$0$}%
{Weight of the Hessian in the distance function of the nlp problem\\
0: no weight, 1: full weight}%
{}

\printoption{feas\_pump\_mult\_objf\_milp}%
{$0\leq\textrm{real}\leq1$}%
{$0$}%
{Weight of the original objective function in the distance function of the milp problem\\
0: no weight, 1: full weight}%
{}

\printoption{feas\_pump\_mult\_objf\_nlp}%
{$0\leq\textrm{real}\leq1$}%
{$0$}%
{Weight of the original objective function in the distance function of the nlp problem\\
0: no weight, 1: full weight}%
{}

\printoption{feas\_pump\_nseprounds}%
{$1\leq\textrm{integer}\leq100000$}%
{$4$}%
{Number of rounds that separate convexification cuts. Must be at least 1}%
{}

\printoption{feas\_pump\_poolcomp}%
{$0\leq\textrm{integer}\leq2$}%
{$0$}%
{Priority field to compare solutions in FP pool\\
0: total number of infeasible objects (integer and nonlinear), 1: maximum infeasibility (integer or nonlinear), 2: objective value.}%
{}

\printoption{feas\_pump\_tabumgt}%
{\ttfamily pool, perturb, cut, none}%
{pool}%
{Retrieval of MILP solutions when the one returned is unsatisfactory}%
{\begin{list}{}{
\setlength{\parsep}{0em}
\setlength{\leftmargin}{5ex}
\setlength{\labelwidth}{2ex}
\setlength{\itemindent}{0ex}
\setlength{\topsep}{0pt}}
\item[\texttt{pool}] Use a solution pool and replace unsatisfactory solution with Euclidean-closest in pool
\item[\texttt{perturb}] Randomly perturb unsatisfactory solution
\item[\texttt{cut}] Separate convexification cuts
\item[\texttt{none}] Bail out of feasibility pump
\end{list}
}

\printoption{feas\_pump\_usescip}%
{\ttfamily no, yes}%
{yes}%
{Should SCIP be used to solve the MILPs?\\
Note, that SCIP is only available for GAMS users with an academic GAMS license.}%
{\begin{list}{}{
\setlength{\parsep}{0em}
\setlength{\leftmargin}{5ex}
\setlength{\labelwidth}{2ex}
\setlength{\itemindent}{0ex}
\setlength{\topsep}{0pt}}
\item[\texttt{no}] Use Cbc's branch-and-cut to solve the MILP
\item[\texttt{yes}] Use SCIP's branch-and-cut or heuristics (see feas\_pump\_milpmethod option) to solve the MILP
\end{list}
}

\printoption{feas\_pump\_vardist}%
{\ttfamily integer, all, int-postprocess}%
{integer}%
{Distance computed on integer-only or on both types of variables, in different flavors.}%
{\begin{list}{}{
\setlength{\parsep}{0em}
\setlength{\leftmargin}{5ex}
\setlength{\labelwidth}{2ex}
\setlength{\itemindent}{0ex}
\setlength{\topsep}{0pt}}
\item[\texttt{integer}] Only compute the distance based on integer coordinates (use post-processing if numerical errors occur)
\item[\texttt{all}] Compute the distance using continuous and integer variables
\item[\texttt{int-postprocess}] Use a post-processing fixed-IP LP to determine a closest-point solution
\end{list}
}

\printoption{feas\_tolerance}%
{$\textrm{real}$}%
{$10^{- 5}$}%
{Tolerance for constraints/auxiliary variables\\
Default value is 1e-5.}%
{}

\printoption{feasibility\_bt}%
{\ttfamily no, yes}%
{yes}%
{Feasibility-based (cheap) bound tightening (FBBT)\\
A pre-processing technique to reduce the bounding box, before the generation of linearization cuts. This is a quick and effective way to reduce the solution set, and it is highly recommended to keep it active.}%
{}

\printoption{fixpoint\_bt}%
{$-99\leq\textrm{integer}$}%
{$0$}%
{The frequency (in terms of nodes) at which Fix Point Bound Tightening is performed.\\
A frequency of 0 (default) means these cuts are never generated. Any positive number n instructs Couenne to generate them at every n nodes of the B\&B tree. A negative number -n means that generation should be attempted at the root node, and if successful it can be repeated at every n nodes, otherwise it is stopped altogether.}%
{}

\printoption{fixpoint\_bt\_model}%
{\ttfamily extended, compact}%
{compact}%
{Choose whether to add an extended fixpoint LP model or a more compact one.}%
{}

\printoption{flow\_covers\_cuts}%
{$-100\leq\textrm{integer}$}%
{$0$}%
{Frequency k (in terms of nodes) for generating flow\_covers\_cuts cuts in branch-and-cut.\\
See option \texttt{2mir\_cuts} for the meaning of k.}%
{}

\printoption{int\_var\_priority}%
{$1\leq\textrm{integer}$}%
{$1000$}%
{Priority of integer variable branching\\
When branching, this is compared to the priority of continuous variables, whose priority is given by cont\_var\_priority, and SOS, whose priority is 10. Higher values mean smaller priority.}%
{}

\printoption{iterative\_rounding\_aggressiveness}%
{$0\leq\textrm{integer}\leq2$}%
{$1$}%
{Aggressiveness of the Iterative Rounding heuristic\\
Set the aggressiveness of the heuristic; i.e., how many iterations should be run, and with which parameters. The maximum time can be overridden by setting the \_time and \_time\_firstcall options. 0 = non aggressive, 1 = standard (default), 2 = aggressive.}%
{}

\printoption{iterative\_rounding\_base\_lbrhs}%
{$0\leq\textrm{integer}$}%
{$15$}%
{Base rhs of the local branching constraint for Iterative Rounding\\
Base rhs for the local branching constraint that defines a neighbourhood of the local incumbent. The base rhs is modified by the algorithm according to variable bounds. This corresponds to k' in the paper. Default 15.}%
{}

\printoption{iterative\_rounding\_heuristic}%
{\ttfamily no, yes}%
{no}%
{Do we use the Iterative Rounding heuristic\\
If enabled, a heuristic based on Iterative Rounding is used to find feasible solutions for the problem. The heuristic may take some time, but usually finds good solutions. Recommended if you want good upper bounds and have Cplex. Not recommended if you do not have Cplex}%
{}

\printoption{iterative\_rounding\_num\_fir\_points}%
{$1\leq\textrm{integer}$}%
{$5$}%
{Max number of points rounded at the beginning of Iterative Rounding\\
Number of different points (obtained solving a log-barrier problem) that the heuristic will try to round at most, during its execution at the root node (i.e. the F-IR heuristic). Default 5.}%
{}

\printoption{iterative\_rounding\_omega}%
{$0<\textrm{real}<1$}%
{$0.2$}%
{Omega parameter of the Iterative Rounding heuristic\\
Set the omega parameter of the heuristic, which represents a multiplicative factor for the minimum log-barrier parameter of the NLP which is solved to obtain feasible points. This corresponds to $\omega'$ in the paper. Default 0.2.}%
{}

\printoption{iterative\_rounding\_time}%
{$\textrm{real}$}%
{$-1$}%
{Specify the maximum time allowed for the Iterative Rounding heuristic\\
Maximum CPU time employed by the Iterative Rounding heuristic; if no solution found in this time, failure is reported. This overrides the CPU time set by Aggressiveness if positive.}%
{}

\printoption{iterative\_rounding\_time\_firstcall}%
{$\textrm{real}$}%
{$-1$}%
{Specify the maximum time allowed for the Iterative Rounding heuristic when no feasible solution is known\\
Maximum CPU time employed by the Iterative Rounding heuristic when no solution is known; if no solution found in this time, failure is reported.This overrides the CPU time set by Aggressiveness if  posive.}%
{}

\printoption{lift\_and\_project\_cuts}%
{$-100\leq\textrm{integer}$}%
{$0$}%
{Frequency k (in terms of nodes) for generating lift\_and\_project\_cuts cuts in branch-and-cut.\\
See option \texttt{2mir\_cuts} for the meaning of k.}%
{}

\printoption{local\_branching\_heuristic}%
{\ttfamily no, yes}%
{no}%
{Apply local branching heuristic\\
A local-branching heuristic based is used to find feasible solutions.}%
{}

\printoption{local\_optimization\_heuristic}%
{\ttfamily no, yes}%
{yes}%
{Search for local solutions of MINLPs\\
If enabled, a heuristic based on Ipopt is used to find feasible solutions for the problem. It is highly recommended that this option is left enabled, as it would be difficult to find feasible solutions otherwise.}%
{}

\printoption{log\_num\_abt\_per\_level}%
{$-1\leq\textrm{integer}$}%
{$2$}%
{Specify the frequency (in terms of nodes) for aggressive bound tightening.\\
If -1, apply at every node (expensive!). If 0, apply at root node only. If k$>$=0, apply with probability 2\^(k - level), level being the current depth of the B\&B tree.}%
{}

\printoption{log\_num\_local\_optimization\_per\_level}%
{$-1\leq\textrm{integer}$}%
{$2$}%
{Specify the logarithm of the number of local optimizations to perform on average for each level of given depth of the tree.\\
Solve as many nlp's at the nodes for each level of the tree. Nodes are randomly selected. If for a given level there are less nodes than this number nlp are solved for every nodes. For example if parameter is 8, nlp's are solved for all node until level 8, then for half the node at level 9, 1/4 at level 10.... Value -1 specify to perform at all nodes.}%
{}

\printoption{log\_num\_obbt\_per\_level}%
{$-1\leq\textrm{integer}$}%
{$1$}%
{Specify the frequency (in terms of nodes) for optimality-based bound tightening.\\
If -1, apply at every node (expensive!). If 0, apply at root node only. If k$>$=0, apply with probability 2\^(k - level), level being the current depth of the B\&B tree.}%
{}

\printoption{lp\_solver}%
{\ttfamily clp, cplex, gurobi, soplex, xpress-mp}%
{clp}%
{Linear Programming solver for the linearization}%
{\begin{list}{}{
\setlength{\parsep}{0em}
\setlength{\leftmargin}{5ex}
\setlength{\labelwidth}{2ex}
\setlength{\itemindent}{0ex}
\setlength{\topsep}{0pt}}
\item[\texttt{clp}] Use the COIN-OR Open Source solver CLP
\item[\texttt{cplex}] Use the commercial solver Cplex (license is needed)
\item[\texttt{gurobi}] Use the commercial solver Gurobi (license is needed)
\item[\texttt{soplex}] Use the freely available Soplex
\item[\texttt{xpress-mp}] Use the commercial solver Xpress MP (license is needed)
\end{list}
}

\printoption{max\_fbbt\_iter}%
{$-1\leq\textrm{integer}$}%
{$3$}%
{Number of FBBT iterations before stopping even with tightened bounds.\\
Set to -1 to impose no upper limit}%
{}

\printoption{minlp\_disj\_cuts}%
{$-99\leq\textrm{integer}$}%
{$0$}%
{The frequency (in terms of nodes) at which Couenne disjunctive cuts are generated.\\
A frequency of 0 (default) means these cuts are never generated. Any positive number n instructs Couenne to generate them at every n nodes of the B\&B tree. A negative number -n means that generation should be attempted at the root node, and if successful it can be repeated at every n nodes, otherwise it is stopped altogether.}%
{}

\printoption{miptrace}%
{string}%
{}%
{Name of file for writing branch-and-bound progress information.}%
{}

\printoption{miptracenodefreq}%
{$0\leq\textrm{integer}$}%
{$100$}%
{Frequency in number of nodes for writing branch-and-bound progress information.\\
giving 0 disables writing of N-lines to trace file}%
{}

\printoption{miptracetimefreq}%
{$0\leq\textrm{real}$}%
{$5$}%
{Frequency in seconds for writing branch-and-bound progress information.\\
giving 0.0 disables writing of T-lines to trace file}%
{}

\printoption{mir\_cuts}%
{$-100\leq\textrm{integer}$}%
{$0$}%
{Frequency k (in terms of nodes) for generating mir\_cuts cuts in branch-and-cut.\\
See option \texttt{2mir\_cuts} for the meaning of k.}%
{}

\printoption{multilinear\_separation}%
{\ttfamily none, simple, tight}%
{tight}%
{Separation for multilinear terms\\
Type of separation for multilinear terms where the dependent variable is also bounded}%
{\begin{list}{}{
\setlength{\parsep}{0em}
\setlength{\leftmargin}{5ex}
\setlength{\labelwidth}{2ex}
\setlength{\itemindent}{0ex}
\setlength{\topsep}{0pt}}
\item[\texttt{none}] No separation -- just use the four McCormick inequalities
\item[\texttt{simple}] Use one considering lower curve only
\item[\texttt{tight}] Use one considering both curves pi(x) = l\_{k+1} and pi(x) = u\_{k+1}
\end{list}
}

\printoption{nlpheur\_print\_level}%
{$-2\leq\textrm{integer}\leq12$}%
{$0$}%
{Output level for NLP heuristic in Couenne}%
{}

\printoption{optimality\_bt}%
{\ttfamily no, yes}%
{yes}%
{Optimality-based (expensive) bound tightening (OBBT)\\
This is another bound reduction technique aiming at reducing the solution set by looking at the initial LP relaxation. This technique is computationally expensive, and should be used only when necessary.}%
{}

\printoption{orbital\_branching}%
{\ttfamily yes, no}%
{no}%
{detect symmetries and apply orbital branching}%
{}

\printoption{output\_level}%
{$-2\leq\textrm{integer}\leq12$}%
{$0$}%
{Output level}%
{}

\printoption{probing\_cuts}%
{$-100\leq\textrm{integer}$}%
{$0$}%
{Frequency k (in terms of nodes) for generating probing\_cuts cuts in branch-and-cut.\\
See option \texttt{2mir\_cuts} for the meaning of k.}%
{}

\printoption{problem\_print\_level}%
{$-2\leq\textrm{integer}\leq12$}%
{$0$}%
{Output level for problem manipulation code in Couenne}%
{}

\printoption{pseudocost\_mult}%
{\ttfamily infeasibility, projectDist, interval\_lp, interval\_lp\_rev, interval\_br, interval\_br\_rev}%
{interval\_br\_rev}%
{Multipliers of pseudocosts for estimating and update estimation of bound}%
{\begin{list}{}{
\setlength{\parsep}{0em}
\setlength{\leftmargin}{5ex}
\setlength{\labelwidth}{2ex}
\setlength{\itemindent}{0ex}
\setlength{\topsep}{0pt}}
\item[\texttt{infeasibility}] infeasibility returned by object
\item[\texttt{projectDist}] distance between current LP point and resulting branches' LP points
\item[\texttt{interval\_lp}] width of the interval between bound and current lp point
\item[\texttt{interval\_lp\_rev}] similar to interval\_lp, reversed
\item[\texttt{interval\_br}] width of the interval between bound and branching point
\item[\texttt{interval\_br\_rev}] similar to interval\_br, reversed
\end{list}
}

\printoption{pseudocost\_mult\_lp}%
{\ttfamily yes, no}%
{no}%
{Use distance between LP points to update multipliers of pseudocosts after simulating branching}%
{}

\printoption{quadrilinear\_decomp}%
{\ttfamily rAI, tri+bi, bi+tri, hier-bi}%
{rAI}%
{type of decomposition for quadrilinear terms (see work by Cafieri, Lee, Liberti)}%
{\begin{list}{}{
\setlength{\parsep}{0em}
\setlength{\leftmargin}{5ex}
\setlength{\labelwidth}{2ex}
\setlength{\itemindent}{0ex}
\setlength{\topsep}{0pt}}
\item[\texttt{rAI}] Recursive decomposition in bilinear terms (as in Ryoo and Sahinidis): x5 = ((x1 x2) x3) x4)
\item[\texttt{tri+bi}] Trilinear and bilinear term: x5 = (x1 (x2 x3 x4))
\item[\texttt{bi+tri}] Bilinear, THEN trilinear term: x5 = ((x1 x2) x3 x4))
\item[\texttt{hier-bi}] Hierarchical decomposition: x5 = ((x1 x2) (x3 x4))
\end{list}
}

\printoption{red\_cost\_branching}%
{\ttfamily no, yes}%
{no}%
{Apply Reduced Cost Branching (instead of the Violation Transfer) -- MUST have vt\_obj enabled}%
{\begin{list}{}{
\setlength{\parsep}{0em}
\setlength{\leftmargin}{5ex}
\setlength{\labelwidth}{2ex}
\setlength{\itemindent}{0ex}
\setlength{\topsep}{0pt}}
\item[\texttt{no}] Use Violation Transfer with $\sum |\pi\_i a\_{ij}|$
\item[\texttt{yes}] Use Reduced cost branching with $|\sum \pi\_i a\_{ij}|$
\end{list}
}

\printoption{redcost\_bt}%
{\ttfamily no, yes}%
{yes}%
{Reduced cost bound tightening\\
This bound reduction technique uses the reduced costs of the LP in order to infer better variable bounds.}%
{}

\printoption{reduce\_split\_cuts}%
{$-100\leq\textrm{integer}$}%
{$0$}%
{Frequency k (in terms of nodes) for generating reduce\_split\_cuts cuts in branch-and-cut.\\
See option \texttt{2mir\_cuts} for the meaning of k.}%
{}

\printoption{reformulate\_print\_level}%
{$-2\leq\textrm{integer}\leq12$}%
{$0$}%
{Output level for reformulating problems in Couenne}%
{}

\printoption{trust\_strong}%
{\ttfamily yes, no}%
{yes}%
{Fathom strong branching LPs when their bound is above the cutoff}%
{}

\printoption{two\_implied\_bt}%
{$-99\leq\textrm{integer}$}%
{$0$}%
{The frequency (in terms of nodes) at which Couenne two-implied bounds are tightened.\\
A frequency of 0 (default) means these cuts are never generated. Any positive number n instructs Couenne to generate them at every n nodes of the B\&B tree. A negative number -n means that generation should be attempted at the root node, and if successful it can be repeated at every n nodes, otherwise it is stopped altogether.}%
{}

\printoption{two\_implied\_max\_trials}%
{$1\leq\textrm{integer}$}%
{$2$}%
{The number of iteration at each call to the cut generator.}%
{}

\printoption{twoimpl\_depth\_level}%
{$-1\leq\textrm{integer}$}%
{$5$}%
{Depth of the B\&B tree when to start decreasing the chance of running this algorithm.\\
This has a similar behavior as log\_num\_obbt\_per\_level. A value of -1 means that generation can be done at all nodes.}%
{}

\printoption{twoimpl\_depth\_stop}%
{$-1\leq\textrm{integer}$}%
{$20$}%
{Depth of the B\&B tree where separation is stopped.\\
A value of -1 means that generation can be done at all nodes}%
{}

\printoption{use\_auxcons}%
{\ttfamily no, yes}%
{yes}%
{Use constraints-defined auxiliaries, i.e. auxiliaries w = f(x) defined by original constraints f(x) - w = 0}%
{}

\printoption{use\_quadratic}%
{\ttfamily no, yes}%
{no}%
{Use quadratic expressions and related exprQuad class\\
If enabled, then quadratic forms are not reformulated and therefore decomposed as a sum of auxiliary variables, each associated with a bilinear term, but rather taken as a whole expression. Envelopes for these expressions are generated through alpha-convexification.}%
{\begin{list}{}{
\setlength{\parsep}{0em}
\setlength{\leftmargin}{5ex}
\setlength{\labelwidth}{2ex}
\setlength{\itemindent}{0ex}
\setlength{\topsep}{0pt}}
\item[\texttt{no}] Use an auxiliary for each bilinear term
\item[\texttt{yes}] Create only one auxiliary for a quadratic expression
\end{list}
}

\printoption{use\_semiaux}%
{\ttfamily no, yes}%
{yes}%
{Use semiauxiliaries, i.e. auxiliaries defined as w $>$= f(x) rather than w := f(x))}%
{\begin{list}{}{
\setlength{\parsep}{0em}
\setlength{\leftmargin}{5ex}
\setlength{\labelwidth}{2ex}
\setlength{\itemindent}{0ex}
\setlength{\topsep}{0pt}}
\item[\texttt{no}] Only use auxiliaries assigned with "=" 
\item[\texttt{yes}] Use auxiliaries defined by w $<$= f(x), w $>$= f(x), and w = f(x)
\end{list}
}

\printoption{violated\_cuts\_only}%
{\ttfamily no, yes}%
{yes}%
{Yes if only violated convexification cuts should be added}%
{}



\section{CoinGlpk}

GAMS/CoinGlpk brings the open source LP/MIP solver Glpk from the GNU Open Software foundation to the broad audience of GAMS users.

The code has been written primarily by A. Makhorin.
The interface uses the OSI Glpk interface written by Vivian De Smedt, Braden Hunsaker, and Lou Hafer.

For more information we refer to
\begin{itemize}
\item the Glpk web site \texttt{http://www.gnu.org/software/glpk/glpk.html} and
\item the Osi web site \texttt{https://projects.coin-or.org/Osi}.
\end{itemize}
Most of the Glpk documentation in the section was taken from the Glpk manual.

\subsection{Model requirements}

Glpk supports continuous, binary, and integer variables, but no special ordered sets, semi-continuous or semi-integer variables (see chapter 17.1 of the GAMS User's Guide).
Also branching priorities are not supported.

\subsection{Usage of CoinGlpk}

The following statement can be used inside your GAMS program to specify using CoinGlpk
\begin{verbatim}
  Option LP = CoinGlpk;     { or MIP or RMIP }
\end{verbatim}

The above statement should appear before the Solve statement.
If CoinGlpk was specified as the default solver during GAMS installation, the above statement is not necessary.

The GAMS/CoinGlpk options file consists of one option or comment per line.
An asterisk (*) at the beginning of a line causes the entire line to be ignored.
Otherwise, the line will be interpreted as an option name and value separated by any amount of white space (blanks or tabs).
Following is an example options file coincbc.opt.
\begin{verbatim}
  factorization givens
\end{verbatim}
It will cause CoinGlpk to use Givens rotation updates for the factorization. (This option setting might help to avoid numerical difficulties in some cases.)

\subsection{Summary of CoinGlpk Options}

Among the following Glpk options, only the GAMS parameters \hyperlink{glpkreslim}{reslim} and \hyperlink{glpkiterlim}{iterlim} are currently supported in CoinGlpk.
Note, that the options optcr and optca to set the MIP gap tolerances are currently not supported.

\begin{tabbing}
\hspace {1.0in} \= \\
\hyperlink{glpkstartalg}
{startalg} \> LP solver for root node \\
\hyperlink{glpkscaling}
{scaling} \> scaling method \\
\hyperlink{pricing}
{pricing} \> pricing method \\
\hyperlink{factorization}
{factorization} \> basis factorization method \\
\hyperlink{initbasis}
{initbasis} \> method for initial basis \\
\hyperlink{glpktol_dual}
{tol\_dual} \> dual feasibility tolerance \\
\hyperlink{glpktol_primal}
{tol\_primal} \> primal feasibility tolerance \\
\hyperlink{glpktol_integer}
{tol\_integer} \> integer feasibility tolerance \\
\hyperlink{backtracking}
{backtracking} \> backtracking heuristic \\
\hyperlink{glpkpresolve}
{presolve} \> LP presolver \\
\hyperlink{reslim_fixedrun}
{reslim\_fixedrun} \> resource limit for solve with fixed discrete variables \\
\hyperlink{noiterlim}
{noiterlim} \> turn off iteration limit \\
\hyperlink{glpkreslim}
{reslim} \> resource limit \\
\hyperlink{glpkiterlim}
{iterlim} \> iteration limit \\
\hyperlink{glpkoptcr}
{optcr} \> relative stopping tolerance on MIP gap \\
\hyperlink{glpkwritemps}
{writemps} \> create MPS file for problem \\
\end{tabbing}


\subsection{Detailed Descriptions of CoinGlpk Options}

\begin{description}

\item[\label{glpkstartalg}\hypertarget{glpkstartalg}
{\textbf{startalg (\slshape{string})}}]\hspace{1.0in}

This option determines whether a primal or dual simplex algorithm should be used to solve an LP or the root node of a MIP.

\textsl{(default = primal)}
\begin{itemize}
\item[primal] 
Let GLPK use a primal simplex algorithm.
\item[dual] 
Let GLPK use a dual simplex algorithm.
\end{itemize}

\item[\label{glpkscaling}\hypertarget{glpkscaling}
{\textbf{scaling (\slshape{string})}}]\hspace{1.0in}

This option determines the method how the constraint matrix is scaled.
Note that scaling is only applied when the \hyperlink{presolve}{presolver} is turned off, which is on by default.

\textsl{(default = meanequilibrium)}
\begin{itemize}
\item[off] 
Turn off scaling.
\item[equilibrium] 
Let GLPK use an equilibrium scaling method.
\item[mean] 
Let GLPK use a geometric mean scaling method.
\item[meanequilibrium] 
Let GLPK use first a geometric mean scaling, then an equilibrium scaling.
\end{itemize}

\item[\label{pricing}\hypertarget{pricing}
{\textbf{pricing (\slshape{string})}}]\hspace{1.0in}

Sets the pricing method for both primal and dual simplex.

\textsl{(default = steepestedge)}
\begin{itemize}
\item[textbook] 
Use a textbook pricing rule.
\item[steepestedge] 
Use a steepest edge pricing rule.
\end{itemize}

\item[\label{factorization}\hypertarget{factorization}
{\textbf{factorization (\slshape{string})}}]\hspace{1.0in}

Sets the method for the LP basis factorization.

If you observe that GLPK reports numerical instabilities than you could try to use a more stable factorization method.

\textsl{(default = forresttomlin)}
\begin{itemize}
\item[forresttomlin] 
Does a LU factorization followed by Forrest-Tomlin updates.
This method is fast, but less stable than others.
\item[bartelsgolub] 
Does a LU factorization followed by a Schur complement and Bartels-Golub updates.
This method is slower than Forrest-Tomlin, but more stable.
\item[givens] 
Does a LU factorization followed by a Schur complement and Givens rotation updates.
This method is slower than Forrest-Tomlin, but more stable.
\end{itemize}

\item[\label{initbasis}\hypertarget{initbasis}
{\textbf{initbasis (\slshape{string})}}]\hspace{1.0in}

Sets the method that computes the initial basis.
Setting this option has only effect if the \hyperlink{presolve}{presolver} is turned off, which is on by default.

\textsl{(default = advanced)}
\begin{itemize}
\item[standard] 
Uses the standard initial basis of all slacks.
\item[advanced] 
Computes an advanced initial basis.
\item[bixby] 
Uses Bixby's initial basis.
\end{itemize}

\item[\label{glpktol_dual}\hypertarget{glpktol_dual}
{\textbf{tol\_dual (\slshape{real})}}]\hspace{1.0in}

Absolute tolerance used to check if the current basis solution is dual feasible.
% (Glpk manual: Do not change this parameter without detailed understanding its purpose.)

\textsl{(default = 1e-7)}

\item[\label{glpktol_primal}\hypertarget{glpktol_primal}
{\textbf{tol\_primal (\slshape{real})}}]\hspace{1.0in}

Relative tolerance used to check if the current basis solution is primal feasible.
% (Glpk manual: Do not change this parameter without detailed understanding its purpose.)

\textsl{(default = 1e-7)}

\item[\label{glpktol_integer}\hypertarget{glpktol_integer}
{\textbf{tol\_integer (\slshape{real})}}]\hspace{1.0in}

Absolute tolerance used to check if the current basis solution is integer feasible.
% (Glpk manual: Do not change this parameter without detailed understanding its purpose.)

\textsl{(default = 1e-5)}

\item[\label{backtracking}\hypertarget{backtracking}
{\textbf{backtracking (\slshape{string})}}]\hspace{1.0in}

Determines which method to use for the backtracking heuristic.

\textsl{(default = bestprojection)}
\begin{itemize}
\item[depthfirst] 
Let GLPK use a depth first search.
\item[breadthfirst] 
Let GLPK use a breadth first search.
\item[bestprojection] 
Let GLPK use a best projection heuristic.
\end{itemize}

\item[\label{glpkpresolve}\hypertarget{glpkpresolve}
{\textbf{presolve (\slshape{integer})}}]\hspace{1.0in}

Determines whether the LP presolver should be used.

\textsl{(default = 1)}
\begin{itemize}
\item[0] 
Turns off the LP presolver.
\item[1] 
Turns on the LP presolver.
\end{itemize}

\item[\label{reslim_fixedrun}\hypertarget{reslim_fixedrun}
{\textbf{reslim\_fixedrun (\slshape{real})}}]\hspace{1.0in}

Maximum time in seconds for solving the MIP with fixed discrete variables.

\textsl{(default = 1000)}

\item[\label{noiterlim}\hypertarget{noiterlim}
{\textbf{noiterlim (\slshape{integer})}}]\hspace{1.0in}

Allows to switch off the \hyperlink{iterlim}{simplex iteration limit}.
You can remove the limit on the simplex iterations by setting the \hyperlink{noiterlim}{noiterlim} option.

\textsl{(default = 0)}
\begin{itemize}
\item[0] 
Keeps simplex iteration limit.
\item[1] 
Turns off simplex iteration limit.
\end{itemize}

\item[\label{glpkreslim}\hypertarget{glpkreslim}
{\textbf{reslim (\slshape{real})}}]\hspace{1.0in}

Maximum time in seconds.

\textsl{(default = GAMS reslim)}

\item[\label{glpkiterlim}\hypertarget{glpkiterlim}
{\textbf{iterlim (\slshape{integer})}}]\hspace{1.0in}

Maximum number of simplex iterations.

\textsl{(default = GAMS iterlim)}

\item[\label{glpkoptcr}\hypertarget{glpkoptcr}
{\textbf{optcr (\slshape{real})}}]\hspace{1.0in}

Relative optimality criterion for a MIP.
The search is stoped when the relative gap between the incumbent and the bound given by the LP relaxation is smaller than this value.

\textsl{(default = GAMS optcr)}

\item[\label{glpkwritemps}\hypertarget{glpkwritemps}
{\textbf{writemps (\slshape{string})}}]\hspace{1.0in}

Write an MPS problem file.
The parameter value is the name of the MPS file.

\end{description}




\section{CoinIpopt and CoinIpoptD}

GAMS/CoinIpopt brings the open source NLP solver Ipopt to the broad audience of GAMS users.

Ipopt (\textbf{I}nterior \textbf{P}oint \textbf{Opt}imizer) is an open-source solver for large-scale nonlinear programming.
The code has been written primarily by Andreas W\"achter, who is the COIN-OR project leader for Ipopt.

For more information we refer to
\begin{itemize}
\item the Ipopt web site \texttt{https://projects.coin-or.org/Ipopt} and
\item the \emph{implementation paper} A. W\"achter and L. T. Biegler, On the Implementation of a Primal-Dual Interior Point Filter Line Search Algorithm for Large-Scale Nonlinear Programming, \emph{Mathematical Programming} 106(1), pp. 25-57, 2006.
\end{itemize}
Most of the Ipopt documentation in the section was taken from the Ipopt manual available on the Ipopt web site.

Gams/IpoptD is an experimental in-core communication link.
It offers in-core communication between GAMS and the solver, making potentially large model scratch files unnecessary.
This links supports all features of the traditional link except.

\subsection{Model requirements}

Ipopt can handle nonlinear programming models which functions can be nonconvex, but should be twice continuously differentiable.

\subsection{Usage of CoinIpopt}

The following statement can be used inside your GAMS program to specify using CoinIpopt
\begin{verbatim}
  Option NLP = CoinIpopt;     { or LP, RMIP, DNLP, RMINLP, QCP, RMIQCP }
\end{verbatim}

The above statement should appear before the Solve statement.
If CoinIpopt was specified as the default solver during GAMS installation, the above statement is not necessary.

\subsection{The linear solver in Ipopt}
\label{ipoptlinearsolver}
\hypertarget{ipoptlinearsolver}{}

The performance and robustness of Ipopt on larger models heavily relies on the used solver for sparse symmetric indefinite linear systems.
GAMS/CoinIpopt includes the sparse solver MUMPS 4.8.3 (\url{http://graal.ens-lyon.fr/MUMPS}).
The user can provide the Parallel Sparse Direct Solver PARDISO or routines from the Harwell Subroutine Library (HSL) as shared (or dynamic) libraries to replace MUMPS.

\subsubsection{Using Harwell Subroutine Library routines with GAMS/CoinIpopt}

If you have routines from the HSL available and want Gams/CoinIpopt to use them, you can provide them in a shared library.
GAMS/CoinIpopt can use MA27, MA28, MA57, and MC19.
By telling Ipopt to use one of these routines (see options linear\_solver, linear\_system\_scaling, nlp\_scaling\_method, dependency\_detector) GAMS/CoinIpopt attempts to load the required routines from the library libhsl.so (Unix-Systems), libhsl.dylib (MacOS X), or libhsl.dll (Windows).
You can also specify the path and name for this library with the option hsl\_library.

For example,
\begin{verbatim}
 linear_solver ma27
 hsl_library /my/path/to/the/hsllib/myhsllib.so
\end{verbatim}
tells Ipopt to use the linear solver MA27 from the HSL library \verb=myhsllib.so= under the specified path.

The HSL routines MA27, MA28, and MC19 are available at \texttt{http://www.cse.clrc.ac.uk/nag/hsl}.
Note that it is your responsibility to ensure that you are entitled to download and use these routines!
You can build a shared library using the ThirdParty/HSL project at COIN-OR.

\subsubsection{Using PARDISO routines with GAMS/CoinIpopt}

If you have the linear solver PARDISO available, then you can tell Gams/CoinIpopt to use by setting the linear\_solver option to pardiso.
GAMS/CoinIpopt then attempts to load the library libpardiso.so (Unix-Systems), libpardiso.dylib (MacOS X), or libpardiso.dll (Windows).
You can also specify the path and name for this library with the option pardiso\_library.

For example,
\begin{verbatim}
 linear_solver pardiso
 pardiso_library /my/path/to/the/pardisolib/mypardisolib.so
\end{verbatim}
tells Ipopt to use the linear solver PARDISO from the library \verb=mypardisolib.so= under the specified path.

PARDISO is available as compiled shared library for several platforms at \texttt{http://www.pardiso-project.org}.
Note that it is your responsibility to ensure that you are entitled to download and use this package!

\subsection{Output}

This section describes the standard Ipopt console output.
The output is designed to provide a quick summary of each iteration as Ipopt solves the problem.

Before Ipopt starts to solve the problem, it displays the problem statistics (number of nonzero-elements in the matrices, number of variables, etc.).
Note that if you have fixed variables (both upper and lower bounds are equal), Ipopt may remove these variables from the problem internally and not include them in the problem statistics.

Following the problem statistics, Ipopt will begin to solve the problem and you will see output resembling the following,
\begin{verbatim}
iter    objective    inf_pr   inf_du lg(mu)  ||d||  lg(rg) alpha_du alpha_pr  ls
   0  1.6109693e+01 1.12e+01 5.28e-01   0.0 0.00e+00    -  0.00e+00 0.00e+00   0
   1  1.8029749e+01 9.90e-01 6.62e+01   0.1 2.05e+00    -  2.14e-01 1.00e+00f  1
   2  1.8719906e+01 1.25e-02 9.04e+00  -2.2 5.94e-02   2.0 8.04e-01 1.00e+00h  1
\end{verbatim}
and the columns of output are defined as
\begin{description}
\item[iter]
The current iteration count.
This includes regular iterations and iterations while in restoration phase.
If the algorithm is in the restoration phase, the letter r' will be appended to the iteration number.
\item[objective]
The unscaled objective value at the current point.
During the restoration phase, this value remains the unscaled objective value for the original problem.
\item[inf\_pr]
The scaled primal infeasibility at the current point.
During the restoration phase, this value is the primal infeasibility of the original problem at the current point.
\item[inf\_du]
The scaled dual infeasibility at the current point.
During the restoration phase, this is the value of the dual infeasibility for the restoration phase problem.
\item[lg(mu)]
$\log_{10}$ of the value of the barrier parameter mu.
\item[$\Vert d\Vert$]
The infinity norm (max) of the primal step (for the original variables $x$ and the internal slack variables $s$).
During the restoration phase, this value includes the values of additional variables, $p$ and $n$.
\item[lg(rg)]
$\log_{10}$ of the value of the regularization term for the Hessian of the Lagrangian in the augmented system.
\item[alpha\_du]
The stepsize for the dual variables.
\item[alpha\_pr]
The stepsize for the primal variables.
\item[ls]
The number of backtracking line search steps.
\end{description}

When the algorithm terminates, IPOPT will output a message to the screen based on the return status of the call to Optimize.
The following is a list of the possible output messages to the console, and a brief description.

\begin{description}
\item[Optimal Solution Found.] ~

    This message indicates that IPOPT found a (locally) optimal point within the desired tolerances.

\item[Solved To Acceptable Level.] ~

    This indicates that the algorithm did not converge to the ``desired'' tolerances, but that it was able to obtain a point satisfying the ``acceptable'' tolerance level as specified by acceptable-* options.
    This may happen if the desired tolerances are too small for the current problem.

\item[Converged to a point of local infeasibility. Problem may be infeasible.] ~

    The restoration phase converged to a point that is a minimizer for the constraint violation (in the $\ell_1$-norm), but is not feasible for the original problem.
    This indicates that the problem may be infeasible (or at least that the algorithm is stuck at a locally infeasible point).
    The returned point (the minimizer of the constraint violation) might help you to find which constraint is causing the problem.
    If you believe that the NLP is feasible, it might help to start the optimization from a different point.

\item[Search Direction is becoming Too Small.] ~

    This indicates that Ipopt is calculating very small step sizes and making very little progress.
    This could happen if the problem has been solved to the best numerical accuracy possible given the current scaling.

\item[Iterates divering; problem might be unbounded.] ~

    This message is printed if the max-norm of the iterates becomes larger than the value of the option diverging\_iterates\_tol.
    This can happen if the problem is unbounded below and the iterates are diverging.

\item[Stopping optimization at current point as requested by user.] ~

    This message is printed if either the time limit or the domain violation limit is reached.

\item[Maximum Number of Iterations Exceeded.] ~

    This indicates that Ipopt has exceeded the maximum number of iterations as specified by the option max\_iter.

\item[Restoration Failed!] ~

    This indicates that the restoration phase failed to find a feasible point that was acceptable to the filter line search for the original problem.
    This could happen if the problem is highly degenerate or does not satisfy the constraint qualification, or if an external function in GAMS provides incorrect derivative information.

\item[Error in step computation (regularization becomes too large?)!] ~

    This messages is printed if Ipopt is unable to compute a search direction, despite several attempts to modify the iteration matrix.
    Usually, the value of the regularization parameter then becomes too large.

\item[Problem has too few degrees of freedom.] ~

    This indicates that your problem, as specified, has too few degrees of freedom.
    This can happen if you have too many equality constraints, or if you fix too many variables (Ipopt removes fixed variables).

\item[Not enough memory.] ~

    An error occurred while trying to allocate memory.
    The problem may be too large for your current memory and swap configuration.

\item[INTERNAL ERROR: Unknown SolverReturn value - Notify IPOPT Authors.] ~

    An unknown internal error has occurred. Please notify the authors of the GAMS/CoinIpopt link or IPOPT (refer to \url{https://projects.coin-or.org/GAMSlinks} or \url{https://projects.coin-or.org/Ipopt}).
\end{description}


\subsection{Specification of CoinIpopt Options}
\label{sub:ipoptoptionspec}

Ipopt has many options that can be adjusted for the algorithm (see Section \ref{sub:ipoptoptions}).
Options are all identified by a string name, and their values can be of one of three types: Number (real), Integer, or String.
Number options are used for things like tolerances, integer options are used for things like maximum number of iterations, and string options are used for setting algorithm details, like the NLP scaling method.
Options can be set by creating a \texttt{ipopt.opt} file in the directory you are executing Ipopt.

The \texttt{ipopt.opt} file is read line by line and each line should contain the option name, followed by whitespace, and then the value.
Comments can be included with the \# symbol. Don't forget to ensure you have a newline at the end of the file. For example,
\begin{verbatim}
# This is a comment

# Turn off the NLP scaling
nlp_scaling_method none

# Change the initial barrier parameter
mu_init 1e-2

# Set the max number of iterations
max_iter 500
\end{verbatim}
is a valid \texttt{ipopt.opt} file.

% You can print the documentation for all Ipopt options by using the option
% \begin{verbatim}
% print_options_documentation yes
% \end{verbatim}
% and running IPOPT.
% This will output all of the options documentation to the console.

Note, that GAMS/CoinIpopt overwrites the Ipopt default setting for the parameters bound\_relax\_factor (set to 0.0) and mu\_strategy (set to adaptive).
You can change these values by specifying these options in your Ipopt options file.

GAMS/CoinIpopt understand currently the following GAMS parameters: reslim (time limit), iterlim (iteration limit), domlim (domain violation limit).
You can set them either on the command line, e.g. \verb+iterlim=500+, or inside your GAMS program, e.g. \verb+Option iterlim=500;+.

\subsection{Detailed Description of CoinIpopt Options}
\label{sub:ipoptoptions}
\subsubsection{Output}

\paragraph{print\_level:} Output verbosity level. $\;$ \\
 Sets the default verbosity level for console
output. The larger this value the more detailed
is the output. The valid range for this integer option is
$0 \le {\tt print\_level } \le 11$
and its default value is $4$.


\paragraph{print\_user\_options:} Print all options set by the user. $\;$ \\
 If selected, the algorithm will print the list of
all options set by the user including their
values and whether they have been used.
The default value for this string option is ``no''.
\\ 
Possible values:
\begin{itemize}
   \item no: don't print options
   \item yes: print options
\end{itemize}

\paragraph{print\_options\_documentation:} Switch to print all algorithmic options. $\;$ \\
 If selected, the algorithm will print the list of
all available algorithmic options with some
documentation before solving the optimization
problem.
The default value for this string option is ``no''.
\\ 
Possible values:
\begin{itemize}
   \item no: don't print list
   \item yes: print list
\end{itemize}

\paragraph{output\_file:} File name of desired output file (leave unset for no file output). $\;$ \\
An output file with this
name will be written (leave unset for no file
output).  The verbosity level is by default set
to ``print\_level'', but can be overridden with
``file\_print\_level''.  The file name is changed
to use only small letters.
With the default settings no output file is generated.
\\ 
Possible values:
\begin{itemize}
   \item *: Any acceptable standard file name
\end{itemize}

\paragraph{file\_print\_level:} Verbosity level for output file. $\;$ \\
 NOTE: This option only works when read from the
ipopt.opt options file! Determines the verbosity
level for the file specified by ``output\_file''.
By default it is the same as ``print\_level''. The valid range for this integer option is
$0 \le {\tt file\_print\_level } \le 11$
and its default value is $4$.

\subsubsection{Termination}

\paragraph{tol:} Desired convergence tolerance (relative). $\;$ \\
 Determines the convergence tolerance for the
algorithm.  The algorithm terminates
successfully, if the (scaled) NLP error becomes
smaller than this value, and if the (absolute)
criteria according to ``dual\_inf\_tol'',
``primal\_inf\_tol'', and ``cmpl\_inf\_tol'' are met.
 (This is $\varepsilon_\mathrm{tol}$ in Eqn. (6) in the
implementation paper).  See also
``acceptable\_tol'' as a second termination
criterion.  Note, some other algorithmic features
also use this quantity to determine thresholds
etc. The valid range for this real option is 
$0 <  {\tt tol } <  {\tt +inf}$
and its default value is $1 \cdot 10^{-08}$.

\paragraph{s\_max:} Scaling threshold for the NLP error. $\;$ \\
The valid range for this integer option is
$0 \le {\tt s\_max } <  {\tt +inf}$
and its default value is $100$.

\paragraph{max\_iter:} Maximum number of iterations. $\;$ \\
 The algorithm terminates with an error message if
the number of iterations exceeded this number. The valid range for this integer option is
$0 \le {\tt max\_iter } <  {\tt +inf}$
and its default value is the value of the GAMS parameter iterlim, which default value is $10000$.


\paragraph{compl\_inf\_tol:} Desired threshold for the complementarity conditions. $\;$ \\
 Absolute tolerance on the complementarity.
Successful termination requires that the max-norm
of the (unscaled) complementarity is less than
this threshold. The valid range for this real option is 
$0 <  {\tt compl\_inf\_tol } <  {\tt +inf}$
and its default value is $0.0001$.


\paragraph{constr\_viol\_tol:} Desired threshold for the constraint violation. $\;$ \\
 Absolute tolerance on the constraint violation.
Successful termination requires that the max-norm
of the (unscaled) constraint violation is less
than this threshold. The default value for this real option is $0.0001$ and its
valid range is $0<\texttt{constr\_viol\_tol}<{\tt +inf}$.


\paragraph{dual\_inf\_tol:} Desired threshold for the dual infeasibility. $\;$ \\
 Absolute tolerance on the dual infeasibility.
Successful termination requires that the max-norm
of the (unscaled) dual infeasibility is less than
this threshold. The valid range for this real option is 
$0 <  {\tt dual\_inf\_tol } <  {\tt +inf}$
and its default value is $0.0001$.


\paragraph{acceptable\_tol:} ``Acceptable'' convergence tolerance (relative). $\;$ \\
 Determines which (scaled) overall optimality
error is considered to be ``acceptable''. There are
two levels of termination criteria.  If the usual
``desired'' tolerances (see tol, dual\_inf\_tol
etc) are satisfied at an iteration, the algorithm
immediately terminates with a success message. 
On the other hand, if the algorithm encounters
``acceptable\_iter'' many iterations in a row that
are considered ``acceptable'', it will terminate
before the desired convergence tolerance is met.
This is useful in cases where the algorithm might
not be able to achieve the ``desired'' level of
accuracy. The valid range for this real option is 
$0 <  {\tt acceptable\_tol } <  {\tt +inf}$
and its default value is $1 \cdot 10^{-06}$.

\paragraph{acceptable\_iter:} Number of ``acceptable'' iterates before triggering termination. $\;$ \\
If the algorithm encounters this many successive ``acceptable'' iterates (see ``acceptable\_tol''), it terminates, assuming that the problem has been solved to best possible accuracy given round-off.
If it is set to zero, this heuristic is disabled.
The valid range for this integer option is
$0 \le {\tt acceptable\_iter } <  {\tt +inf}$
and its default value is $15$.

\paragraph{acceptable\_compl\_inf\_tol:} ``Acceptance'' threshold for the complementarity conditions. $\;$ \\
 Absolute tolerance on the complementarity.
``Acceptable'' termination requires that the
max-norm of the (unscaled) complementarity is
less than this threshold; see also
acceptable\_tol. The valid range for this real option is 
$0 <  {\tt acceptable\_compl\_inf\_tol } <  {\tt +inf}$
and its default value is $0.01$.


\paragraph{acceptable\_constr\_viol\_tol:} ``Acceptance'' threshold for the constraint violation. $\;$ \\
 Absolute tolerance on the constraint violation.
``Acceptable'' termination requires that the
max-norm of the (unscaled) constraint violation
is less than this threshold; see also
acceptable\_tol. The valid range for this real option is 
$0 <  {\tt acceptable\_constr\_viol\_tol } <  {\tt +inf}$
and its default value is $0.01$.


\paragraph{acceptable\_dual\_inf\_tol:} ``Acceptance'' threshold for the dual infeasibility. $\;$ \\
 Absolute tolerance on the dual infeasibility.
``Acceptable'' termination requires that the
(max-norm of the unscaled) dual infeasibility is
less than this threshold; see also
acceptable\_tol. The valid range for this real option is 
$0 <  {\tt acceptable\_dual\_inf\_tol } <  {\tt +inf}$
and its default value is $0.01$.


\paragraph{diverging\_iterates\_tol:} Threshold for maximal value of primal iterates. $\;$ \\
 If any component of the primal iterates exceeded
this value (in absolute terms), the optimization
is aborted with the exit message that the
iterates seem to be diverging. The valid range for this real option is 
$0 <  {\tt diverging\_iterates\_tol } <  {\tt +inf}$
and its default value is $1 \cdot 10^{+20}$.

\subsubsection{NLP Scaling}

\paragraph{obj\_scaling\_factor:} Scaling factor for the objective function. $\;$ \\
 This option sets a scaling factor for the
objective function. The scaling is seen
internally by Ipopt but the unscaled objective is
reported in the console output. If additional
scaling parameters are computed (e.g.
user-scaling or gradient-based), both factors are
multiplied. The valid range for this real option is 
${\tt -inf} <  {\tt obj\_scaling\_factor } <  {\tt +inf}$
and its default value is $1$.


\paragraph{nlp\_scaling\_method:} Select the technique used for scaling the NLP. $\;$ \\
 Selects the technique used for scaling the
problem internally before it is solved. For
user-scaling, the parameters come from the values of the .scale suffix in GAMS.
The default value for this string option is ``gradient-based'' if scaleopt is 0 (default).
If the user provides variable or equation scaling values in GAMS and sets $<$model$>$.scaleopt to 1, then the default for this parameter is ``user-scaling''.
\\ 
Possible values:
\begin{itemize}
   \item none: no problem scaling will be performed
   \item user-scaling: scaling parameters will come from the user
   \item gradient-based: scale the problem so the maximum gradient at
the starting point is scaling\_max\_gradient
   \item equilibration-based: scale the problem so that first derivatives are
of order 1 at random points (only available with MC19)
\end{itemize}

\paragraph{nlp\_scaling\_max\_gradient:} Maximum gradient after NLP scaling. $\;$ \\
 This is the gradient scaling cut-off. If the
maximum gradient is above this value, then
gradient based scaling will be performed. Scaling
parameters are calculated to scale the maximum
gradient back to this value. (This is $g_{\max}$ in
Section 3.8 of the implementation paper.) Note:
This option is only used if
``nlp\_scaling\_method'' is chosen as
``gradient-based''. The valid range for this real option is 
$0 <  {\tt nlp\_scaling\_max\_gradient } <  {\tt +inf}$
and its default value is $100$.

\paragraph{nlp\_scaling\_obj\_target\_gradient:} Target value for objective function gradient size. $\;$ \\
     If a positive number is chosen, the scaling factor the objective function
     is computed so that the gradient as the max norm of the given size at the
     starting point.  This overrides nlp\_scaling\_max\_gradient for the
     objective function.
The valid range for this real option is 
$0 <  {\tt nlp\_scaling\_obj\_target\_gradient } <  {\tt +inf}$
and its default value is $0$.

\subsubsection{NLP corrections}

\paragraph{dependency\_detector:} Indicates which linear solver should be used to detect linearly dependent equality constraints. $\;$ \\
The default value for this string option is ``none''.
\\ 
Possible values:
\begin{itemize}
\item none:                    don't check; no extra work at beginning
\item mumps:                   use MUMPS
\item ma28:                     use MA28
\end{itemize}

\paragraph{dependency\_detection\_with\_rhs:} Indicates if the right hand sides of the constraints should be considered during dependency detection. $\;$ \\
The default value for this string option is ``no''.\\
Possible values:
\begin{itemize}
\item no:                      only look at gradients
\item yes:                     also consider right hand side
\end{itemize}

\paragraph{point\_perturbation\_radius:} Maximal perturbation of an evaluation point. $\;$ \\
     If a random perturbation of a points is required, this number indicates
     the maximal perturbation.  Currently, this is only used when we perturb
     the initial point in order to get a random Jacobian for the linear
     dependency detection of equality constraints.
The valid range for this real option is 
$0 \le {\tt point\_perturbation\_radius } <  {\tt +inf}$
and its default value is $10$.

\paragraph{kappa\_d:} Weight for linear damping term (to handle one-sided bounds). $\;$ \\
The valid range for this real option is 
$0 \le {\tt kappa\_d } <  {\tt +inf}$
and its default value is $10^{-5}$.

\paragraph{bound\_relax\_factor:} Factor for initial relaxation of the bounds. $\;$ \\
 Before start of the optimization, the bounds
given by the user are relaxed.  This option sets
the factor for this relaxation.  If it is set to
zero, then then bounds relaxation is disabled.
(See Eqn.(35) in the implementation paper.)
The valid range for this real option is 
$0 \le {\tt bound\_relax\_factor } <  {\tt +inf}$
and its default value is $0$.


\paragraph{honor\_original\_bounds:} Indicates whether final points should be projected into original bounds. $\;$ \\
 Ipopt might relax the bounds during the
optimization (see, e.g., option
``bound\_relax\_factor'').  This option determines
whether the final point should be projected back
into the user-provide original bounds after the
optimization.
The default value for this string option is ``yes''.
\\ 
Possible values:
\begin{itemize}
   \item no: Leave final point unchanged
   \item yes: Project final point back into original bounds
\end{itemize}

% \paragraph{check\_derivatives\_for\_naninf:} Indicates whether it is desired to check for Nan/Inf in derivative matrices $\;$ \\
%  Activating this option will cause an error if an
% invalid number is detected in the constraint
% Jacobians or the Lagrangian Hessian.  If this is
% not activated, the test is skipped, and the
% algorithm might proceed with invalid numbers and
% fail.
% The default value for this string option is ``no''.
% \\ 
% Possible values:
% \begin{itemize}
%    \item no: Don't check (faster).
%    \item yes: Check Jacobians and Hessian for Nan and Inf.
% \end{itemize}

\paragraph{fixed\_variable\_treatment:} Determines how fixed variables should be handled. $\;$ \\
The main difference between those options is that the starting point in the ``make\_constraint'' case still has the fixed variables at their given values, whereas in the case ``make\_parameter''
the functions are always evaluated with the fixed values for those variables. 
Also, for ``relax\_bounds'', the fixing bound constraints are relaxed (according to ``bound\_relax\_factor'').
For both ``make\_constraints'' and ``relax\_bounds'', bound multipliers are computed for the fixed variables.
The default value for this string option is ``make\_parameter''.
\\ 
Possible values:
\begin{itemize}
\item make\_parameter:    Remove fixed variable from optimization variables.
\item make\_constraint:   Add equality constraints fixing variables.
\item relax\_bounds:      Relax fixing bound constraints.
\end{itemize}


\subsubsection{Initialization}

\paragraph{bound\_frac:} Desired minimum relative distance from the initial point to bound. $\;$ \\
 Determines how much the initial point might have
to be modified in order to be sufficiently inside
the bounds (together with ``bound\_push'').
(This is $\kappa_2$ in Section 3.6 of the implementation paper.)
The valid range for this real option is 
$0 <  {\tt bound\_frac } \le 0.5$
and its default value is $0.01$.


\paragraph{bound\_push:} Desired minimum absolute distance from the initial point to bound. $\;$ \\
 Determines how much the initial point might have
to be modified in order to be sufficiently inside
the bounds (together with ``bound\_frac'').
(This is $\kappa_1$ in Section 3.6 of the implementation paper.)
The valid range for this real option is 
$0 <  {\tt bound\_push } <  {\tt +inf}$
and its default value is $0.01$.


\paragraph{slack\_bound\_push:} Desired minimum absolute distance from the initial slack to bound. $\;$ \\       Determines how much the initial slack variables might have to be modified
in order to be sufficiently inside the inequality bounds (together with ``slack\_bound\_frac'').
(This is $\kappa_1$ in Section 3.6 of the implementation paper.)
The valid range for this real option is 
$0 <  {\tt slack\_bound\_push } <  {\tt +inf}$
and its default value is $0.01$.

\paragraph{slack\_bound\_frac:} Desired minimum relative distance from the initial slack to bound.
     Determines how much the initial slack variables might have to be modified
     in order to be sufficiently inside the inequality bounds (together with
     ``slack\_bound\_push'').
(This is $\kappa_2$ in Section 3.6 of the implementation paper.)
The valid range for this real option is 
$0 <  {\tt slack\_bound\_frac } \le  0.5$
and its default value is $0.01$.


\paragraph{bound\_mult\_init\_val:} Initial value for the bound multipliers. $\;$ \\
 All dual variables corresponding to bound
constraints are initialized to this value. The valid range for this real option is 
$0 <  {\tt bound\_mult\_init\_val } <  {\tt +inf}$
and its default value is $1$.


\paragraph{constr\_mult\_init\_max:} Maximum allowed least-square guess of constraint multipliers. $\;$ \\
 Determines how large the initial least-square
guesses of the constraint multipliers are allowed
to be (in max-norm). If the guess is larger than
this value, it is discarded and all constraint
multipliers are set to zero.  This options is
also used when initializing the restoration
phase. By default,
``resto.constr\_mult\_init\_max'' (the one used in
RestoIterateInitializer) is set to zero. The valid range for this real option is 
$0 \le {\tt constr\_mult\_init\_max } <  {\tt +inf}$
and its default value is $1000$.


\paragraph{bound\_mult\_init\_val:} Initial value for the bound multipliers. $\;$ \\
 All dual variables corresponding to bound
constraints are initialized to this value. The valid range for this real option is 
$0 <  {\tt bound\_mult\_init\_val } <  {\tt +inf}$
and its default value is $1$.

\paragraph{least\_square\_init\_primal:} Least square initialization of the primal variables. $\;$ \\
     If set to yes, Ipopt ignores the user provided point and solves a least
     square problem for the primal variables (x and s), to fit the linearize
     equality and inequality constraints.  This might be useful if the user
     doesn't know anything about the starting point, or for solving an LP or
     QP.
The default value for this string option is ``no''.\\
   Possible values:
\begin{itemize}
    \item no:                      take user-provided point
    \item yes:                     overwrite user-provided point with least-square estimates
\end{itemize}

\paragraph{least\_square\_init\_duals:} Least square initialization of all dual variables. $\;$ \\
     If set to yes, Ipopt tries to compute least-square multipliers
     (considering ALL dual variables).  If successful, the bound multipliers
     are possibly corrected to be at least bound\_mult\_init\_val. This might be
     useful if the user doesn't know anything about the starting point, or for
     solving an LP or QP.
The default value for this string option is ``no''.\\
   Possible values:
\begin{itemize}
    \item no:                      use bound\_mult\_init\_val and least-square equality constraint multipliers
    \item yes:                     overwrite user-provided point with least-square estimates
\end{itemize}

\subsubsection{Barrier parameter update}

\paragraph{mehrotra\_algorithm:} Indicates if we want to do Mehrotra's algorithm. $\;$ \\
 If set to yes, Ipopt runs as Mehrotra's
predictor-corrector algorithm. This works usually
very well for LPs and convex QPs.  This
automatically disables the line search, and
chooses the (unglobalized) adaptive mu strategy
with the ``probing'' oracle, and uses
``corrector\_type=affine'' without any safeguards;
you should not set any of those options
explicitly in addition.  Also, unlessotherwise
specified, the values of ``bound\_push'',
``bound\_frac'', and ``bound\_mult\_init\_val'' are
set more aggressive, and sets
``alpha\_for\_y=bound\_mult''.
The default value for this string option is ``no''.
\\ 
Possible values:
\begin{itemize}
   \item no: Do the usual Ipopt algorithm.
   \item yes: Do Mehrotra's predictor-corrector algorithm.
\end{itemize}

\paragraph{mu\_strategy:} Update strategy for barrier parameter. $\;$ \\
 Determines which barrier parameter update
strategy is to be used.
The default value for this string option is ``adaptive''.
\\ 
Possible values:
\begin{itemize}
   \item monotone: use the monotone (Fiacco-McCormick) strategy
   \item adaptive: use the adaptive update strategy
\end{itemize}

\paragraph{mu\_oracle:} Oracle for a new barrier parameter in the adaptive strategy. $\;$ \\
 Determines how a new barrier parameter is
computed in each ``free-mode'' iteration of the
adaptive barrier parameter strategy. (Only
considered if ``adaptive'' is selected for option
``mu\_strategy'').
The default value for this string option is ``quality-function''.
\\ 
Possible values:
\begin{itemize}
   \item probing: Mehrotra's probing heuristic
   \item loqo: LOQO's centrality rule
   \item quality-function: minimize a quality function
\end{itemize}

\paragraph{quality\_function\_max\_section\_steps:} Maximum number of search steps during direct search procedure determining the optimal centering parameter. $\;$ \\
 The golden section search is performed for the
quality function based mu oracle. The valid range for this integer option is
$0 \le {\tt quality\_function\_max\_section\_steps } <  {\tt +inf}$
and its default value is $8$.
This option is only used if the option ``mu\_oracle'' is set to ``quality-function''.


\paragraph{fixed\_mu\_oracle:} Oracle for the barrier parameter when switching to fixed mode. $\;$ \\
 Determines how the first value of the barrier
parameter should be computed when switching to
the ``monotone mode'' in the adaptive strategy.
(Only considered if ``adaptive'' is selected for
option ``mu\_strategy''.)
The default value for this string option is ``average\_compl''.
\\ 
Possible values:
\begin{itemize}
   \item probing: Mehrotra's probing heuristic
   \item loqo: LOQO's centrality rule
   \item quality-function: minimize a quality function
   \item average\_compl: base on current average complementarity
\end{itemize}

\paragraph{mu\_init:} Initial value for the barrier parameter. $\;$ \\
 This option determines the initial value for the
barrier parameter (mu).  It is only relevant in
the monotone, Fiacco-McCormick version of the
algorithm. (i.e., if ``mu\_strategy'' is chosen as
``monotone'') The valid range for this real option is 
$0 <  {\tt mu\_init } <  {\tt +inf}$
and its default value is $0.1$.

\paragraph{mu\_max\_fact:} Factor for initialization of maximum value for barrier parameter. $\;$ \\
 This option determines the upper bound on the
barrier parameter.  This upper bound is computed
as the average complementarity at the initial
point times the value of this option. (Only used
if option ``mu\_strategy'' is chosen as ``adaptive''.) The valid range for this real option is 
$0 <  {\tt mu\_max\_fact } <  {\tt +inf}$
and its default value is $1000$.


\paragraph{mu\_max:} Maximum value for barrier parameter. $\;$ \\
 This option specifies an upper bound on the
barrier parameter in the adaptive mu selection
mode.  If this option is set, it overwrites the
effect of mu\_max\_fact. (Only used if option
``mu\_strategy'' is chosen as ``adaptive''.) The valid range for this real option is 
$0 <  {\tt mu\_max } <  {\tt +inf}$
and its default value is $100000$.


\paragraph{mu\_min:} Minimum value for barrier parameter. $\;$ \\
 This option specifies the lower bound on the
barrier parameter in the adaptive mu selection
mode. By default, it is set to
min(``tol'', ``compl\_inf\_tol'')/(``barrier\_tol\_fact-
or''+1), which should be a reasonable value. (Only
used if option ``mu\_strategy'' is chosen as
``adaptive''.) The valid range for this real option is 
$0 <  {\tt mu\_min } <  {\tt +inf}$
and its default value is $1 \cdot 10^{-09}$.

\paragraph{barrier\_tol\_factor:} Factor for mu in barrier stop test. $\;$ \\
 The convergence tolerance for each barrier
problem in the monotone mode is the value of the
barrier parameter times ``barrier\_tol\_factor''.
This option is also used in the adaptive mu
strategy during the monotone mode. (This is
$\kappa_\varepsilon$ in the implementation paper). The valid range for this real option is 
$0 <  {\tt barrier\_tol\_factor } <  {\tt +inf}$
and its default value is $10$.

\paragraph{mu\_linear\_decrease\_factor:} Determines linear decrease rate of barrier parameter. $\;$ \\
 For the Fiacco-McCormick update procedure the new
barrier parameter mu is obtained by taking the
minimum of mu$\cdot$``mu\_linear\_decrease\_factor'' and
mu$^\textrm{``superlinear\_decrease\_power''}$.  (This is
$\kappa_\mu$ in the implementation paper.) This option
is also used in the adaptive mu strategy during
the monotone mode. The valid range for this real option is 
$0 <  {\tt mu\_linear\_decrease\_factor } <  1$
and its default value is $0.2$.


\paragraph{mu\_superlinear\_decrease\_power:} Determines superlinear decrease rate of barrier parameter. $\;$ \\
 For the Fiacco-McCormick update procedure the new
barrier parameter mu is obtained by taking the
minimum of mu$\cdot$``mu\_linear\_decrease\_factor'' and
mu$^\textrm{``superlinear\_decrease\_power''}$.  (This is
$\theta_\mu$ in the implementation paper.) This option
is also used in the adaptive mu strategy during
the monotone mode. The valid range for this real option is 
$1 <  {\tt mu\_superlinear\_decrease\_power } <  2$
and its default value is $1.5$.

\subsubsection{Barrier parameter update (expert options)}

\paragraph{mu\_allow\_fast\_monotone\_decrease:} Allow skipping of barrier problem if barrier test is already met. $\;$ \\
 If set to ``no'', the algorithm enforces at least
one iteration per barrier problem, even if the
barrier test is already met for the updated
barrier parameter.
The default value for this string option is ``yes''.
\\ 
Possible values:
\begin{itemize}
   \item no: Take at least one iteration per barrier problem
   \item yes: Allow fast decrease of mu if barrier test it met
\end{itemize}

\paragraph{adaptive\_mu\_globalization:} Globalization strategy for the adaptive mu selection mode. $\;$ \\
 To achieve global convergence of the adaptive
version, the algorithm has to switch to the
monotone mode (Fiacco-McCormick approach) when
convergence does not seem to appear.  This option
sets the criterion used to decide when to do this
switch. (Only used if option ``mu\_strategy'' is
chosen as ``adaptive''.)
The default value for this string option is ``obj-constr-filter''.
\\ 
Possible values:
\begin{itemize}
   \item kkt-error: nonmonotone decrease of kkt-error
   \item obj-constr-filter: 2-dim filter for objective and constraint
violation
   \item never-monotone-mode: disables globalization
\end{itemize}

\paragraph{adaptive\_mu\_kkterror\_red\_iters:} Maximum number of iterations requiring sufficient progress. $\;$ \\
 For the ``kkt-error'' based globalization strategy,
the progress made in at most ``adaptive\_mu\_kkterror\_red\_iters'' iterations must be sufficient.
If this number of iterations is exceeded, the
globalization strategy switches to the monotone
mode. The valid range for this integer option is
$0 \le {\tt adaptive\_mu\_kkterror\_red\_iters } <  {\tt +inf}$
and its default value is $4$.


\paragraph{adaptive\_mu\_kkterror\_red\_fact:} Sufficient decrease factor for ``kkt-error'' globalization strategy. $\;$ \\
 For the ``kkt-error'' based globalization strategy,
the error must decrease by this factor to be
deemed sufficient decrease. The valid range for this real option is 
$0 <  {\tt adaptive\_mu\_kkterror\_red\_fact } <  1$
and its default value is $0.9999$.


\paragraph{filter\_margin\_fact:} Factor determining width of margin for obj-constr-filter adaptive globalization strategy. $\;$ \\
 When using the adaptive globalization strategy,
``obj-constr-filter'', sufficient progress for a
filter entry is defined as follows: (new obj) <
(filter obj) - filter\_margin\_fact$\cdot$ (new
constr-viol) OR (new constr-viol) $<$ (filter
constr-viol) - filter\_margin\_fact$\cdot$ (new
constr-viol).  For the description of the
``kkt-error-filter'' option see
``filter\_max\_margin''. The valid range for this real option is 
$0 <  {\tt filter\_margin\_fact } <  1$
and its default value is $1 \cdot 10^{-05}$.


\paragraph{filter\_max\_margin:} Maximum width of margin in obj-constr-filter adaptive globalization strategy. $\;$ \\
 The valid range for this real option is 
$0 <  {\tt filter\_max\_margin } <  {\tt +inf}$
and its default value is $1$.


\paragraph{adaptive\_mu\_restore\_previous\_iterate:} Indicates if the previous iterate should be restored if the monotone mode is entered. $\;$ \\
 When the globalization strategy for the adaptive
barrier algorithm switches to the monotone mode,
it can either start from the most recent iterate
(no), or from the last iterate that was accepted
(yes).
The default value for this string option is ``no''.
\\ 
Possible values:
\begin{itemize}
   \item no: don't restore accepted iterate
   \item yes: restore accepted iterate
\end{itemize}

\paragraph{adaptive\_mu\_monotone\_init\_factor:} Determines the initial value of the barrier parameter when switching to the monotone mode. $\;$ \\
 When the globalization strategy for the adaptive
barrier algorithm switches to the monotone mode
and the option fixed\_mu\_oracle is chosen as
``average\_compl'', the barrier parameter is set to
the current average complementarity times the
value of ``adaptive\_mu\_monotone\_init\_factor''. The default value for this option is $0.8$ and its valid range is $0 <  {\tt adaptive\_mu\_monotone\_init\_factor } <  {\tt +inf}$.


\paragraph{adaptive\_mu\_kkt\_norm\_type:} Norm used for the KKT error in the adaptive mu globalization strategies. $\;$ \\
 When computing the KKT error for the
globalization strategies, the norm to be used is
specified with this option. Note, this options is
also used in the QualityFunctionMuOracle.
The default value for this string option is ``2-norm-squared''.
\\ 
Possible values:
\begin{itemize}
   \item 1-norm: use the 1-norm (abs sum)
   \item 2-norm-squared: use the 2-norm squared (sum of squares)
   \item max-norm: use the infinity norm (max)
   \item 2-norm: use 2-norm
\end{itemize}

\paragraph{tau\_min:} Lower bound on fraction-to-the-boundary parameter tau. $\;$ \\
 (This is $\tau_{\min}$ in the implementation paper.)  This
option is also used in the adaptive mu strategy
during the monotone mode. The valid range for this real option is 
$0 <  {\tt tau\_min } <  1$
and its default value is $0.99$.


\paragraph{sigma\_max:} Maximum value of the centering parameter. $\;$ \\
 This is the upper bound for the centering
parameter chosen by the quality function based
barrier parameter update. (Only used if option
``mu\_oracle'' is set to ``quality-function''.) The valid range for this real option is 
$0 <  {\tt sigma\_max } <  {\tt +inf}$
and its default value is $100$.


\paragraph{sigma\_min:} Minimum value of the centering parameter. $\;$ \\
 This is the lower bound for the centering
parameter chosen by the quality function based
barrier parameter update. (Only used if option
``mu\_oracle'' is set to ``quality-function''.) The valid range for this real option is 
$0 \le {\tt sigma\_min } <  {\tt +inf}$
and its default value is $1 \cdot 10^{-06}$.


\paragraph{quality\_function\_norm\_type:} Norm used for components of the quality function. $\;$ \\
 (Only used if option ``mu\_oracle'' is set to
``quality-function''.)
The default value for this string option is ``2-norm-squared''.
\\ 
Possible values:
\begin{itemize}
   \item 1-norm: use the 1-norm (abs sum)
   \item 2-norm-squared: use the 2-norm squared (sum of squares)
   \item max-norm: use the infinity norm (max)
   \item 2-norm: use 2-norm
\end{itemize}

\paragraph{quality\_function\_centrality:} The penalty term for centrality that is included in quality function. $\;$ \\
 This determines whether a term is added to the
quality function to penalize deviation from
centrality with respect to complementarity.  The
complementarity measure here is the xi in the
Loqo update rule. (Only used if option
``mu\_oracle'' is set to ``quality-function''.)
The default value for this string option is ``none''.
\\ 
Possible values:
\begin{itemize}
   \item none: no penalty term is added
   \item log: complementarity $\cdot$ the log of the centrality
measure
   \item reciprocal: complementarity $\cdot$ the reciprocal of the
centrality measure
   \item cubed-reciprocal: complementarity $\cdot$ the reciprocal of the
centrality measure cubed
\end{itemize}

\paragraph{quality\_function\_balancing\_term:} The balancing term included in the quality function for centrality. $\;$ \\
 This determines whether a term is added to the
quality function that penalizes situations where
the complementarity is much smaller than dual and
primal infeasibilities. (Only used if option
``mu\_oracle'' is set to ``quality-function''.)
The default value for this string option is ``none''.
\\ 
Possible values:
\begin{itemize}
   \item none: no balancing term is added
   \item cubic: $\max(0,\max(\textrm{dual\_inf},\textrm{primal\_inf})-\textrm{compl})^3$
\end{itemize}

\paragraph{quality\_function\_max\_section\_steps:} Maximum number of search steps during direct search procedure determining the optimal centering parameter. $\;$ \\
 The golden section search is performed for the
quality function based mu oracle. The valid range for this integer option is
$0 \le {\tt quality\_function\_max\_section\_steps } <  {\tt +inf}$
and its default value is $8$.
(Only used if
option ``mu\_oracle'' is set to ``quality-function''.)


\paragraph{quality\_function\_section\_sigma\_tol:} Tolerance for the section search procedure determining the optimal centering parameter (in sigma space). $\;$ \\
 The golden section search is performed for the
quality function based mu oracle. (Only used if
option ``mu\_oracle'' is set to ``quality-function''.) The valid range for this real option is 
$0 \le {\tt quality\_function\_section\_sigma\_tol } <  1$
and its default value is $0.01$.


\paragraph{quality\_function\_section\_qf\_tol:} Tolerance for the golden section search procedure determining the optimal centering parameter (in the function value space). $\;$ \\
 The golden section search is performed for the
quality function based mu oracle. (Only used if
option ``mu\_oracle'' is set to ``quality-function''.) The valid range for this real option is 
$0 \le {\tt quality\_function\_section\_qf\_tol } <  1$
and its default value is $0$.


\subsubsection{Warm start}

\paragraph{warm\_start\_init\_point:} Warm-start for initial point $\;$ \\
 Indicates whether this optimization should use a warm start initialization, where values of dual variables are given by GAMS (You can set marginal values for variables and equations in your GAMS model to set the starting point for the dual variables.)
For the primal values, Ipopt uses the starting point that is given by GAMS (You can set level values for variables (and equations) in your GAMS model to set the starting point for the primal variables.)
The default value for this string option is ``no''.
\\ 
Possible values:
\begin{itemize}
   \item no: do not use the warm start initialization
   \item yes: use the warm start initialization
\end{itemize}

\paragraph{warm\_start\_bound\_push:} same as bound\_push for the regular initializer. $\;$ \\
 The valid range for this real option is 
$0 <  {\tt warm\_start\_bound\_push } <  {\tt +inf}$
and its default value is $0.001$.


\paragraph{warm\_start\_bound\_frac:} same as bound\_frac for the regular initializer. $\;$ \\
 The valid range for this real option is 
$0 <  {\tt warm\_start\_bound\_frac } \le 0.5$
and its default value is $0.001$.


\paragraph{warm\_start\_slack\_bound\_frac:} same as slack\_bound\_frac for the regular initializer. $\;$ \\
 The valid range for this real option is 
$0 <  {\tt warm\_start\_slack\_bound\_frac } \le 0.5$
and its default value is $0.001$.


\paragraph{warm\_start\_slack\_bound\_push:} same as slack\_bound\_push for the regular initializer. $\;$ \\
 The valid range for this real option is 
$0 <  {\tt warm\_start\_slack\_bound\_push } <  {\tt +inf}$
and its default value is $0.001$.


\paragraph{warm\_start\_mult\_bound\_push:} same as mult\_bound\_push for the regular initializer. $\;$ \\
 The valid range for this real option is 
$0 <  {\tt warm\_start\_mult\_bound\_push } <  {\tt +inf}$
and its default value is $0.001$.


\paragraph{warm\_start\_mult\_init\_max:} Maximum initial value for the equality multipliers. $\;$ \\
 The valid range for this real option is 
${\tt -inf} <  {\tt warm\_start\_mult\_init\_max } <  {\tt +inf}$
and its default value is $1 \cdot 10^{+06}$.

\subsubsection{Multiplier updates}

\paragraph{alpha\_for\_y:} Method to determine the step size for constraint multipliers. $\;$ \\
 This option determines how the step size
(alpha\_y) will be calculated when updating the
constraint multipliers.
The default value for this string option is ``primal''.
\\ 
Possible values:
\begin{itemize}
   \item primal: use primal step size
   \item bound\_mult: use step size for the bound multipliers (good
for LPs)
   \item min: use the min of primal and bound multipliers
   \item max: use the max of primal and bound multipliers
   \item full: take a full step of size one
   \item min\_dual\_infeas: choose step size minimizing new dual
infeasibility
   \item safe\_min\_dual\_infeas: like ``min\_dual\_infeas'', but safeguarded by
``min'' and ``max''
\end{itemize}

\paragraph{alpha\_for\_y\_tol:} Tolerance for switching to full equality multiplier steps. $\;$ \\
 This is only relevant if ``alpha\_for\_y'' is
chosen ``primal-and-full'' or ``dual-and-full''.  The
step size for the equality constraint multipliers
is taken to be one if the max-norm of the primal
step is less than this tolerance. The valid range for this real option is 
$0 \le {\tt alpha\_for\_y\_tol } <  {\tt +inf}$
and its default value is $10$.

\paragraph{recalc\_y:} Tells the algorithm to recalculate the equality and inequality multipliers as least square estimates. $\;$ \\
 This asks Ipopt to recompute the
multipliers, whenever the current infeasibility
is less than recalc\_y\_feas\_tol. Choosing yes
might be helpful in the quasi-Newton option. 
However, each recalculation requires an extra
factorization of the linear system.  If a limited
memory quasi-Newton option is chosen, this is
used by default.
The default value for this string option is ``no''.
\\ 
Possible values:
\begin{itemize}
   \item no: use the Newton step to update the multipliers
   \item yes: use least-square multiplier estimates
\end{itemize}

\paragraph{recalc\_y\_feas\_tol:} Feasibility threshold for recomputation of multipliers. $\;$ \\
 If recalc\_y is chosen and the current
infeasibility is less than this value, then the
multipliers are recomputed. The valid range for this real option is 
$0 <  {\tt recalc\_y\_feas\_tol } <  {\tt +inf}$
and its default value is $1 \cdot 10^{-06}$.

\subsubsection{Line search}

\paragraph{max\_soc:} Maximum number of second order correction trial steps at each iteration. $\;$ \\
 Choosing 0 disables the second order corrections.
(This is $p^{\max}$ of Step A-5.9 of Algorithm A in
the implementation paper.) The valid range for this integer option is
$0 \le {\tt max\_soc } <  {\tt +inf}$
and its default value is $4$.


\paragraph{watchdog\_shortened\_iter\_trigger:} Number of shortened iterations that trigger the watchdog. $\;$ \\
 If the number of successive iterations in which
the backtracking line search did not accept the
first trial point exceeds this number, the
watchdog procedure is activated.  Choosing 0
here disables the watchdog procedure. The valid range for this integer option is
$0 \le {\tt watchdog\_shortened\_iter\_trigger } <  {\tt +inf}$
and its default value is $10$.


\paragraph{watchdog\_trial\_iter\_max:} Maximum number of watchdog iterations. $\;$ \\
 This option determines the number of trial
iterations allowed before the watchdog procedure
is aborted and the algorithm returns to the
stored point. The valid range for this integer option is
$1 \le {\tt watchdog\_trial\_iter\_max } <  {\tt +inf}$
and its default value is $3$.

\paragraph{corrector\_type:} The type of corrector steps that should be taken (experimental!). $\;$ \\
 If ``mu\_strategy'' is ``adaptive'', this option
determines what kind of corrector steps should be
tried.
The default value for this string option is ``none''.
\\ 
Possible values:
\begin{itemize}
   \item none: no corrector
   \item affine: corrector step towards mu=0
   \item primal-dual: corrector step towards current mu
\end{itemize}

\subsubsection{Line search (expert options)}

\paragraph{alpha\_red\_factor:} Fractional reduction of the trial step size in the backtracking line search. $\;$ \\
 At every step of the backtracking line search,
the trial step size is reduced by this factor. The valid range for this real option is 
$0 <  {\tt alpha\_red\_factor } <  1$
and its default value is $0.5$.


\paragraph{accept\_every\_trial\_step:} Always accept the first trial step. $\;$ \\
 Setting this option to ``yes'' essentially disables
the line search and makes the algorithm take
aggressive steps, without global convergence
guarantees.
The default value for this string option is ``no''.
\\ 
Possible values:
\begin{itemize}
   \item no: don't arbitrarily accept the full step
   \item yes: always accept the full step
\end{itemize}

\paragraph{tiny\_step\_tol:} Tolerance for detecting numerically insignificant steps. $\;$ \\
 If the search direction in the primal variables
(x and s) is, in relative terms for each
component, less than this value, the algorithm
accepts the full step without line search.  If
this happens repeatedly, the algorithm will
terminate with a corresponding exit message. The
default value is 10 times machine precision. The valid range for this real option is 
$0 \le {\tt tiny\_step\_tol } <  {\tt +inf}$
and its default value is $2.22045 \cdot 10^{-15}$.


\paragraph{tiny\_step\_y\_tol:} Tolerance for quitting because of numerically insignificant steps. $\;$ \\
 If the search direction in the primal variables
(x and s) is, in relative terms for each
component, repeatedly less than tiny\_step\_tol,
and the step in the y variables is smaller than
this threshold, the algorithm will terminate. The valid range for this real option is 
$0 \le {\tt tiny\_step\_y\_tol } <  {\tt +inf}$
and its default value is $0.01$.


\paragraph{theta\_max\_fact:} Determines upper bound for constraint violation in the filter. $\;$ \\
 The algorithmic parameter theta\_max is
determined as theta\_max\_fact times the maximum
of 1 and the constraint violation at initial
point.  Any point with a constraint violation
larger than theta\_max is unacceptable to the
filter (see Eqn. (21) in the implementation paper). The valid range for this real option is 
$0 <  {\tt theta\_max\_fact } <  {\tt +inf}$
and its default value is $10000$.


\paragraph{theta\_min\_fact:} Determines constraint violation threshold in the switching rule. $\;$ \\
 The algorithmic parameter theta\_min is
determined as theta\_min\_fact times the maximum
of 1 and the constraint violation at initial
point.  The switching rules treats an iteration
as an h-type iteration whenever the current
constraint violation is larger than theta\_min
(see paragraph before Eqn. (19) in the implementation
paper). The valid range for this real option is 
$0 <  {\tt theta\_min\_fact } <  {\tt +inf}$
and its default value is $0.0001$.


\paragraph{eta\_phi:} Relaxation factor in the Armijo condition. $\;$ \\
 (See Eqn. (20) in the implementation paper) The valid range for this real option is 
$0 <  {\tt eta\_phi } <  0.5$
and its default value is $1 \cdot 10^{-08}$.


\paragraph{delta:} Multiplier for constraint violation in the switching rule. $\;$ \\
 (See Eqn. (19) in the implementation paper.) The valid range for this real option is 
$0 <  {\tt delta } <  {\tt +inf}$
and its default value is $1$.


\paragraph{s\_phi:} Exponent for linear barrier function model in the switching rule. $\;$ \\
 (See Eqn. (19) in the implementation paper.) The valid range for this real option is 
$1 <  {\tt s\_phi } <  {\tt +inf}$
and its default value is $2.3$.


\paragraph{s\_theta:} Exponent for current constraint violation in the switching rule. $\;$ \\
 (See Eqn. (19) in the implementation paper.) The valid range for this real option is 
$1 <  {\tt s\_theta } <  {\tt +inf}$
and its default value is $1.1$.


\paragraph{gamma\_phi:} Relaxation factor in the filter margin for the barrier function. $\;$ \\
 (See Eqn. (18a) in the implementation paper.) The valid range for this real option is 
$0 <  {\tt gamma\_phi } <  1$
and its default value is $1 \cdot 10^{-08}$.


\paragraph{gamma\_theta:} Relaxation factor in the filter margin for the constraint violation. $\;$ \\
 (See Eqn. (18b) in the implementation paper.) The valid range for this real option is 
$0 <  {\tt gamma\_theta } <  1$
and its default value is $1 \cdot 10^{-05}$.


\paragraph{alpha\_min\_frac:} Safety factor for the minimal step size (before switching to restoration phase). $\;$ \\
 (This is $\gamma_\alpha$ in Eqn. (20) in the
implementation paper.) The default value of this real option is $0.05$ and its
valid range is $0 <  {\tt alpha\_min\_frac } <  1$.


\paragraph{kappa\_soc:} Factor in the sufficient reduction rule for second order correction. $\;$ \\
 This option determines how much a second order
correction step must reduce the constraint
violation so that further correction steps are
attempted.  (See Step A-5.9 of Algorithm A in
the implementation paper.) The valid range for this real option is 
$0 <  {\tt kappa\_soc } <  {\tt +inf}$
and its default value is $0.99$.


\paragraph{obj\_max\_inc:} Determines the upper bound on the acceptable increase of barrier objective function. $\;$ \\
 Trial points are rejected if they lead to an
increase in the barrier objective function by
more than obj\_max\_inc orders of magnitude. The valid range for this real option is 
$1 <  {\tt obj\_max\_inc } <  {\tt +inf}$
and its default value is $5$.


\paragraph{max\_filter\_resets:} Maximal allowed number of filter resets $\;$ \\
 A positive number enables a heuristic that resets
the filter, whenever in more than
``filter\_reset\_trigger'' successive iterations
the last rejected trial steps size was rejected
because of the filter.  This option determine the
maximal number of resets that are allowed to take
place. The valid range for this integer option is
$0 \le {\tt max\_filter\_resets } <  {\tt +inf}$
and its default value is $5$.


\paragraph{filter\_reset\_trigger:} Number of iterations that trigger the filter reset. $\;$ \\
 If the filter reset heuristic is active and the
number of successive iterations in which the last
rejected trial step size was rejected because of
the filter, the filter is reset. The valid range for this integer option is
$1 \le {\tt filter\_reset\_trigger } <  {\tt +inf}$
and its default value is $5$.


% \paragraph{skip\_corr\_if\_neg\_curv:} Skip the corrector step in negative curvature iteration (unsupported!). $\;$ \\
%  The corrector step is not tried if negative
% curvature has been encountered during the
% computation of the search direction in the
% current iteration. This option is only used if
% ``mu\_strategy'' is ``adaptive''.
% The default value for this string option is ``yes''.
% \\ 
% Possible values:
% \begin{itemize}
%    \item no: don't skip
%    \item yes: skip
% \end{itemize}

% \paragraph{skip\_corr\_in\_monotone\_mode:} Skip the corrector step during monotone barrier parameter mode (unsupported!). $\;$ \\
%  The corrector step is not tried if the algorithm
% is currently in the monotone mode (see also
% option ``barrier\_strategy'').This option is only
% used if ``mu\_strategy'' is ``adaptive''.
% The default value for this string option is ``yes''.
% \\ 
% Possible values:
% \begin{itemize}
%    \item no: don't skip
%    \item yes: skip
% \end{itemize}

% \paragraph{corrector\_compl\_avrg\_red\_fact:} Complementarity tolerance factor for accepting corrector step (unsupported!). $\;$ \\
%  This option determines the factor by which
% complementarity is allowed to increase for a
% corrector step to be accepted. The valid range for this real option is 
% $0 <  {\tt corrector\_compl\_avrg\_red\_fact } <  {\tt +inf}$
% and its default value is $1$.


\paragraph{kappa\_sigma:} Factor limiting the deviation of dual variables from primal estimates. $\;$ \\
 If the dual variables deviate from their primal
estimates, a correction is performed. (See Eqn.
(16) in the the implementation paper.) Setting the
value to less than 1 disables the correction. The valid range for this real option is 
$0 <  {\tt kappa\_sigma } <  {\tt +inf}$
and its default value is $1 \cdot 10^{+10}$.


\paragraph{slack\_move:} Correction size for very small slacks. $\;$ \\
 Due to numerical issues or the lack of an
interior, the slack variables might become very
small.  If a slack becomes very small compared to
machine precision, the corresponding bound is
moved slightly.  This parameter determines how
large the move should be.  Its default value is
mach\_eps$^{3/4}$.  (See also end of Section 3.5
in the implementation paper - but actual
the implementation might be somewhat different.) The valid range for this real option is 
$0 \le {\tt slack\_move } <  {\tt +inf}$
and its default value is $1.81899 \cdot 10^{-12}$.

\subsubsection{Restoration phase}

\paragraph{expect\_infeasible\_problem:} Enable heuristics to quickly detect an infeasible problem. $\;$ \\
 This options is meant to activate heuristics that
may speed up the infeasibility determination if
you expect that there is a good chance for the
problem to be infeasible.  In the filter line
search procedure, the restoration phase is called
more quickly than usually, and more reduction in
the constraint violation is enforced before the
restoration phase is left. If the problem is
square, this option is enabled automatically.
The default value for this string option is ``no''.
\\ 
Possible values:
\begin{itemize}
   \item no: the problem probably be feasible
   \item yes: the problem has a good chance to be infeasible
\end{itemize}

\paragraph{expect\_infeasible\_problem\_ctol:} Threshold for disabling ``expect\_infeasible\_problem'' option. $\;$ \\
 If the constraint violation becomes smaller than
this threshold, the ``expect\_infeasible\_problem''
heuristics in the filter line search are
disabled. If the problem is square, this options
is set to 0. The valid range for this real option is 
$0 \le {\tt expect\_infeasible\_problem\_ctol } <  {\tt +inf}$
and its default value is $0.001$.


\paragraph{start\_with\_resto:} Tells algorithm to switch to restoration phase in first iteration. $\;$ \\
 Setting this option to ``yes'' forces the algorithm
to switch to the feasibility restoration phase in
the first iteration. If the initial point is
feasible, the algorithm will abort with a failure.
The default value for this string option is ``no''.
\\ 
Possible values:
\begin{itemize}
   \item no: don't force start in restoration phase
   \item yes: force start in restoration phase
\end{itemize}

\paragraph{soft\_resto\_pderror\_reduction\_factor:} Required reduction in primal-dual error in the soft restoration phase. $\;$ \\
 The soft restoration phase attempts to reduce the
primal-dual error with regular steps. If the
damped primal-dual step (damped only to satisfy
the fraction-to-the-boundary rule) is not
decreasing the primal-dual error by at least this
factor, then the regular restoration phase is
called. Choosing 0 here disables the soft
restoration phase. The valid range for this real option is 
$0 \le {\tt soft\_resto\_pderror\_reduction\_factor } <  {\tt +inf}$
and its default value is $0.9999$.


\paragraph{required\_infeasibility\_reduction:} Required reduction of infeasibility before leaving restoration phase. $\;$ \\
 The restoration phase algorithm is performed,
until a point is found that is acceptable to the
filter and the infeasibility has been reduced by
at least the fraction given by this option. The valid range for this real option is 
$0 \le {\tt required\_infeasibility\_reduction } <  1$
and its default value is $0.9$.


\paragraph{max\_soft\_resto\_iters:} Maximum number of iterations performed successively in soft restoration phase. $\;$ \\
 If the soft restoration phase is performed for
more than so many iterations in a row, the regular
restoration phase is called. The valid range for this integer option is
$0 \le {\tt max\_soft\_resto\_iters } <  {\tt +inf}$
and its default value is $10$.


\paragraph{max\_resto\_iter:} Maximum number of successive iterations in restoration phase. $\;$ \\
 The algorithm terminates with an error message if
the number of iterations successively taken in
the restoration phase exceeds this number. The valid range for this integer option is
$0 \le {\tt max\_resto\_iter } <  {\tt +inf}$
and its default value is $3000000$.


\paragraph{bound\_mult\_reset\_threshold:} Threshold for resetting bound multipliers after the restoration phase. $\;$ \\
 After returning from the restoration phase, the
bound multipliers are updated with a Newton step
for complementarity.  Here, the change in the
primal variables during the entire restoration
phase is taken to be the corresponding primal
Newton step. However, if after the update the
largest bound multiplier exceeds the threshold
specified by this option, the multipliers are all
reset to 1. The valid range for this real option is 
$0 \le {\tt bound\_mult\_reset\_threshold } <  {\tt +inf}$
and its default value is $1000$.


\paragraph{constr\_mult\_reset\_threshold:} Threshold for resetting equality and inequality multipliers after restoration phase. $\;$ \\
 After returning from the restoration phase, the
constraint multipliers are recomputed by a least
square estimate.  This option triggers when those
least-square estimates should be ignored. The valid range for this real option is 
$0 \le {\tt constr\_mult\_reset\_threshold } <  {\tt +inf}$
and its default value is $0$.


\paragraph{evaluate\_orig\_obj\_at\_resto\_trial:} Determines if the original objective function should be evaluated at restoration phase trial points. $\;$ \\
 Setting this option to ``yes'' makes the
restoration phase algorithm evaluate the
objective function of the original problem at
every trial point encountered during the
restoration phase, even if this value is not
required.  In this way, it is guaranteed that the
original objective function can be evaluated
without error at all accepted iterates; otherwise
the algorithm might fail at a point where the
restoration phase accepts an iterate that is good
for the restoration phase problem, but not the
original problem.  On the other hand, if the
evaluation of the original objective is
expensive, this might be costly.
The default value for this string option is ``yes''.
\\ 
Possible values:
\begin{itemize}
   \item no: skip evaluation
   \item yes: evaluate at every trial point
\end{itemize}

\subsubsection{Linear solver}

\paragraph{linear\_solver:} Linear solver used for step computations. $\;$ \\
 Determines which linear algebra package is to be used for the solution of the augmented linear system (for obtaining the search directions).
Note, you need to provide an extra shared library to use MA27, MA57, or PARDISO, see \hyperlink{ipoptlinearsolver}{Section \ref{ipoptlinearsolver}}.
The default value for this string option is ``mumps''.
\\
Possible values:
\begin{itemize}
   \item ma27: use the Harwell routine MA27
   \item ma57: use the Harwell routine MA57
   \item pardiso: use the Pardiso package
%    \item wsmp: use WSMP package
%    \item taucs: use TAUCS package (not yet working)
   \item mumps: use MUMPS package
%    \item custom: use custom linear solver
\end{itemize}

\paragraph{hsl\_library:} Path and filename of HSL library for dynamic load. $\;$ \\
Specify the path to a library that contains HSL routines and can be load via dynamic linking, see also \hyperlink{ipoptlinearsolver}{Section \ref{ipoptlinearsolver}}.

\paragraph{pardiso\_library:} Path and filename of PARDISO library for dynamic load. $\;$ \\
Specify the path to a PARDISO library that and can be load via dynamic linking, see also \hyperlink{ipoptlinearsolver}{Section \ref{ipoptlinearsolver}}.

\paragraph{linear\_system\_scaling:} Method for scaling the linear system. $\;$ \\
 Determines the method used to compute symmetric scaling factors for the augmented system (see also the ``linear\_scaling\_on\_demand'' option).
This scaling is independent of the NLP problem scaling.
By default, MC19 is only used if MA27 or MA57 are selected as linear solvers.
% This option is only available if Ipopt has been compiled with MC19.
The default value for this string option is ``mc19''.
\\
Possible values:
\begin{itemize}
   \item none: no scaling will be performed
   \item mc19: use the Harwell routine MC19
\end{itemize}

\paragraph{linear\_scaling\_on\_demand:} Flag indicating that linear scaling is only done if it seems required. $\;$ \\
 This option is only important if a linear scaling method (e.g., mc19) is used.
If you choose ``no'', then the scaling factors are computed for every linear system from the start.
This can be quite expensive.
Choosing ``yes'' means that the algorithm will start the scaling method only when the solutions to the linear system seem not good, and then use it until the end.
The default value for this string option is ``yes''.
\\
Possible values:
\begin{itemize}
   \item no: Always scale the linear system.
   \item yes: Start using linear system scaling if solutions
seem not good.
\end{itemize}

\paragraph{fast\_step\_computation:} Indicates if the linear system should be solved quickly. $\;$ \\
 If set to yes, the algorithm assumes that the
linear system that is solved to obtain the search
direction, is solved sufficiently well. In that
case, no residuals are computed, and the
computation of the search direction is a little
faster.
The default value for this string option is ``no''.
\\ 
Possible values:
\begin{itemize}
   \item no: Verify solution of linear system by computing
residuals.
   \item yes: Trust that linear systems are solved well.
\end{itemize}

\paragraph{max\_refinement\_steps:} Maximum number of iterative refinement steps per linear system solve. $\;$ \\
 Iterative refinement (on the full unsymmetric
system) is performed for each right hand side. 
This option determines the maximum number of
iterative refinement steps. The valid range for this integer option is
$0 \le {\tt max\_refinement\_steps } <  {\tt +inf}$
and its default value is $10$.


\paragraph{min\_refinement\_steps:} Minimum number of iterative refinement steps per linear system solve. $\;$ \\
 Iterative refinement (on the full unsymmetric
system) is performed for each right hand side. 
This option determines the minimum number of
iterative refinements (i.e. at least
``min\_refinement\_steps'' iterative refinement
steps are enforced per right hand side.) The valid range for this integer option is
$0 \le {\tt min\_refinement\_steps } <  {\tt +inf}$
and its default value is $1$.

\paragraph{residual\_ratio\_max:} Iterative refinement tolerance $\;$ \\
 Iterative refinement is performed until the
residual test ratio is less than this tolerance
(or until the limit ``max\_refinement\_steps'' is hit). The valid range for this real option is 
$0 <  {\tt residual\_ratio\_max } <  {\tt +inf}$
and its default value is $1 \cdot 10^{-10}$.


\paragraph{residual\_ratio\_singular:} Threshold for declaring linear system singular after failed iterative refinement. $\;$ \\
 If the residual test ratio is larger than this
value after failed iterative refinement, the
algorithm pretends that the linear system is
singular. The valid range for this real option is 
$0 <  {\tt residual\_ratio\_singular } <  {\tt +inf}$
and its default value is $1 \cdot 10^{-05}$.


\paragraph{residual\_improvement\_factor:} Minimal required reduction of residual test ratio in iterative refinement. $\;$ \\
 If the improvement of the residual test ratio
made by one iterative refinement step is not
better than this factor, iterative refinement is
aborted. The valid range for this real option is 
$0 <  {\tt residual\_improvement\_factor } <  {\tt +inf}$
and its default value is $1$.


\subsubsection{MUMPS Linear Solver}

\paragraph{mumps\_pivtol:} Pivot tolerance for the linear solver MUMPS. \\
A smaller number pivots for sparsity, a larger number pivots for stability.
The valid range for this real option is
$0 \le {\tt mumps\_pivtol } < {\tt 1}$
and its default value is $1e-6$.

\paragraph{mumps\_pivtolmax:} Maximum pivot tolerance for the linear solver MUMPS. \\
Ipopt may increase pivtol as high as pivtolmax to get a more accurate solution to the linear system.
The valid range for this real option is
$0 \le {\tt mumps\_pivtolmax } < {\tt 1}$
and its default value is $0.1$.

\paragraph{mumps\_mem\_percent:} Percentage increase in the estimated working space for MUMPS. \\
In MUMPS when significant extra fill-in is caused by numerical pivoting, larger values of mumps\_mem\_percent may help use the workspace more efficiently.
The valid range for this integer option is
$0 \le {\tt mumps\_mem\_percent } < {\tt +inf}$
and its default value is $1000$.

\paragraph{mumps\_permuting\_scaling:} Controls permuting and scaling in MUMPS $\;$ \\
 This is ICTL(6) in MUMPS. The valid range for this integer option is
$0 \le {\tt mumps\_permuting\_scaling } \le 7$
and its default value is $7$.


\paragraph{mumps\_pivot\_order:} Controls pivot order in MUMPS $\;$ \\
 This is ICTL(7) in MUMPS. The valid range for this integer option is
$0 \le {\tt mumps\_pivot\_order } \le 7$
and its default value is $7$.


\paragraph{mumps\_scaling:} Controls scaling in MUMPS $\;$ \\
 This is ICTL(8) in MUMPS. The valid range for this integer option is
$-2 \le {\tt mumps\_scaling } \le 7$
and its default value is $7$.


\paragraph{mumps\_dep\_tol:} Pivot threshold for detection of linearly dependent constraints in MUMPS. $\;$ \\
 When MUMPS is used to determine linearly
dependent constraints, this is determines the
threshold for a pivot to be considered zero. 
This is CNTL(3) in MUMPS. The valid range for this real option is 
${\tt -inf} <  {\tt mumps\_dep\_tol } <  {\tt +inf}$
and its default value is $-1$.

\subsubsection{PARDISO Linear Solver}

\paragraph{pardiso\_matching\_strategy:} Matching strategy to be used by Pardiso $\;$ \\
This is IPAR(13) in Pardiso manual.
The default value for this string option is ``complete+2x2''.
\\
Possible values:
\begin{itemize}
   \item complete: Match complete (IPAR(13)=1)
   \item complete+2x2: Match complete+2x2 (IPAR(13)=2)
   \item constraints: Match constraints (IPAR(13)=3)
\end{itemize}

\paragraph{pardiso\_out\_of\_core\_power:} Enables out-of-core variant of Pardiso $\;$ \\
Setting this option to a positive integer $k$ makes Pardiso work in the out-of-core variant where the factor is split in $2^k$ subdomains.
This is IPARM(50) in the Pardiso manual.
The valid range for this integer option is $0 \le {\tt pardiso\_out\_of\_core\_power } <  {\tt +inf}$ and its default value is $0$.

\subsubsection{MA27 Linear Solver}

\paragraph{ma27\_pivtol:} Pivot tolerance for the linear solver MA27. $\;$ \\
A smaller number pivots for sparsity, a larger number pivots for stability.
The valid range for this real option is $0 <  {\tt ma27\_pivtol } <  1$ and its default value is $10^{-8}$.


\paragraph{ma27\_pivtolmax:} Maximum pivot tolerance for the linear solver MA27. $\;$ \\
Ipopt may increase pivtol as high as pivtolmax to get a more accurate solution to the linear
system.
The valid range for this real option is $0 <  {\tt ma27\_pivtolmax } <  1$ and its default value is $0.0001$.


\paragraph{ma27\_liw\_init\_factor:} Integer workspace memory for MA27. $\;$ \\
The initial integer workspace memory = liw\_init\_factor $*$ memory required by unfactored system.
Ipopt will increase the workspace size by meminc\_factor if required.
The default value for this real option is $5$ and its valid range is $1 \le {\tt ma27\_liw\_init\_factor } <  {\tt +inf}$.


\paragraph{ma27\_la\_init\_factor:} Real workspace memory for MA27. $\;$ \\
The initial real workspace memory = la\_init\_factor $*$ memory required by unfactored system. 
Ipopt will increase the workspace size by meminc\_factor if required.
The valid range for this real option is $1 \le {\tt ma27\_la\_init\_factor } <  {\tt +inf}$ and its default value is $5$.


\paragraph{ma27\_meminc\_factor:} Increment factor for workspace size for MA27. $\;$ \\
If the integer or real workspace is not large enough, Ipopt will increase its size by this factor.
The valid range for this real option is $1 \le {\tt ma27\_meminc\_factor } <  {\tt +inf}$ and its default value is $10$.

\subsubsection{MA57 Linear Solver}

\paragraph{ma57\_pivtol:} Pivot tolerance for the linear solver MA57. $\;$ \\
A smaller number pivots for sparsity, a larger number pivots for stability.
The valid range for this real option is $0 <  {\tt ma57\_pivtol } <  1$ and its default value is $10^{-8}$.


\paragraph{ma57\_pivtolmax:} Maximum pivot tolerance for the linear solver MA57. $\;$ \\
Ipopt may increase pivtol as high as ma57\_pivtolmax to get a more accurate solution to the linear system.
The valid range for this real option is $0 <  {\tt ma57\_pivtolmax } <  1$ and its default value is $0.0001$.


\paragraph{ma57\_pre\_alloc:} Safety factor for work space memory allocation for the linear solver MA57. $\;$ \\
If 1 is chosen, the suggested amount of work space is used.
However, choosing a larger number might avoid reallocation if the suggest values do not suffice.
The valid range for this real option is $1 \le {\tt ma57\_pre\_alloc } <  {\tt +inf}$ and its default value is $3$.



\subsubsection{Hessian perturbation}

\paragraph{max\_hessian\_perturbation:} Maximum value of regularization parameter for handling negative curvature. $\;$ \\
 In order to guarantee that the search directions
are indeed proper descent directions, Ipopt
requires that the inertia of the (augmented)
linear system for the step computation has the
correct number of negative and positive
eigenvalues. The idea is that this guides the
algorithm away from maximizers and makes Ipopt
more likely converge to first order optimal
points that are minimizers. If the inertia is not
correct, a multiple of the identity matrix is
added to the Hessian of the Lagrangian in the
augmented system. This parameter gives the
maximum value of the regularization parameter. If
a regularization of that size is not enough, the
algorithm skips this iteration and goes to the
restoration phase. (This is $\delta_w^{\max}$ in the
implementation paper.) The valid range for this real option is 
$0 <  {\tt max\_hessian\_perturbation } <  {\tt +inf}$
and its default value is $1 \cdot 10^{+20}$.


\paragraph{min\_hessian\_perturbation:} Smallest perturbation of the Hessian block. $\;$ \\
 The size of the perturbation of the Hessian block
is never selected smaller than this value, unless
no perturbation is necessary. (This is
$\delta_w^{\min}$ in the implementation paper.) The valid range for this real option is 
$0 \le {\tt min\_hessian\_perturbation } <  {\tt +inf}$
and its default value is $1 \cdot 10^{-20}$.


\paragraph{first\_hessian\_perturbation:} Size of first x-s perturbation tried. $\;$ \\
 The first value tried for the x-s perturbation in
the inertia correction scheme.(This is $\delta_0$
in the implementation paper.) The valid range for this real option is 
$0 <  {\tt first\_hessian\_perturbation } <  {\tt +inf}$
and its default value is $0.0001$.


\paragraph{perturb\_inc\_fact\_first:} Increase factor for x-s perturbation for very first perturbation. $\;$ \\
 The factor by which the perturbation is increased
when a trial value was not sufficient - this
value is used for the computation of the very
first perturbation and allows a different value
for for the first perturbation than that used for
the remaining perturbations. (This is
$\bar\kappa_w^+$ in the implementation paper.) The valid range for this real option is 
$1 <  {\tt perturb\_inc\_fact\_first } <  {\tt +inf}$
and its default value is $100$.


\paragraph{perturb\_inc\_fact:} Increase factor for x-s perturbation. $\;$ \\
 The factor by which the perturbation is increased
when a trial value was not sufficient - this
value is used for the computation of all
perturbations except for the first. (This is
$\kappa_w^+$ in the implementation paper.) The valid range for this real option is 
$1 <  {\tt perturb\_inc\_fact } <  {\tt +inf}$
and its default value is $8$.


\paragraph{perturb\_dec\_fact:} Decrease factor for x-s perturbation. $\;$ \\
 The factor by which the perturbation is decreased
when a trial value is deduced from the size of
the most recent successful perturbation. (This is
$\kappa_w^-$ in the implementation paper.) The valid range for this real option is 
$0 <  {\tt perturb\_dec\_fact } <  1$
and its default value is $0.333333$.


\paragraph{jacobian\_regularization\_value:} Size of the regularization for rank-deficient constraint Jacobians. $\;$ \\
 (This is $\bar\delta_c$ in the implementation
paper.) The valid range for this real option is\\ 
$0 \le {\tt jacobian\_regularization\_value } <  {\tt +inf}$
and its default value is $1 \cdot 10^{-08}$.


\paragraph{jacobian\_regularization\_exponent:} Exponent for mu in the regularization for rank-deficient constraint Jacobians. $\;$ \\
 (This is $\kappa_c$ in the implementation paper.) The default value for this real option is $0.25$
and its valid range is $0 \le {\tt jacobian\_regularization\_exponent } <  {\tt +inf}$.


\paragraph{perturb\_always\_cd:} Active permanent perturbation of constraint linearization. $\;$ \\
 This options makes the delta\_c and delta\_d
perturbation be used for the computation of every
search direction.  Usually, it is only used when
the iteration matrix is singular.
The default value for this string option is ``no''.
\\ 
Possible values:
\begin{itemize}
   \item no: perturbation only used when required
   \item yes: always use perturbation
\end{itemize}

\subsubsection{Hessian approximation}

\paragraph{hessian\_approximation:} Indicates what Hessian information is to be used. $\;$ \\
 This determines which kind of information for the
Hessian of the Lagrangian function is used by the
algorithm.
The default value for this string option is to use ``exact'' if the GAMS system is able to provide a hessian, and ``limited-memory'' otherwise (a warning is issued in this case).
\\ 
Possible values:
\begin{itemize}
   \item exact: Use second derivatives provided by the NLP.
   \item limited-memory: Perform a limited-memory quasi-Newton
approximation
\end{itemize}

\paragraph{hessian\_approximation\_space:} Indicates in which subspace the Hessian information is to be approximated. \\
The default value for this string option is ``nonlinear-variables''.
\\ 
Possible values:
\begin{itemize}
   \item nonlinear-variables: only in space of nonlinear variables.
   \item all-variables: in space of all variables (without slacks)
\end{itemize}

\paragraph{limited\_memory\_max\_history:} Maximum size of the history for the limited quasi-Newton Hessian approximation. $\;$ \\
 This option determines the number of most recent
iterations that are taken into account for the
limited-memory quasi-Newton approximation. The valid range for this integer option is
$0 \le {\tt limited\_memory\_max\_history } <  {\tt +inf}$
and its default value is $6$.


\paragraph{limited\_memory\_update\_type:} Quasi-Newton update formula for the limited memory approximation. $\;$ \\
 Determines which update formula is to be used for
the limited-memory quasi-Newton approximation.
The default value for this string option is ``bfgs''.
\\ 
Possible values:
\begin{itemize}
   \item bfgs: BFGS update (with skipping)
   \item sr1: SR1 (not working well)
\end{itemize}

\paragraph{limited\_memory\_initialization:} Initialization strategy for the limited memory quasi-Newton approximation. $\;$ \\
 Determines how the diagonal Matrix B\_0 as the
first term in the limited memory approximation
should be computed.
The default value for this string option is ``scalar1''.
\\ 
Possible values:
\begin{itemize}
   \item scalar1: sigma = $s^Ty/s^Ts$
   \item scalar2: sigma = $y^Ty/s^Ty$
   \item constant: sigma = limited\_memory\_init\_val
\end{itemize}

\paragraph{limited\_memory\_init\_val:} Value for B0 in low-rank update. $\;$ \\
 The starting matrix in the low rank update, B0,
is chosen to be this multiple of the identity in
the first iteration (when no updates have been
performed yet), and remains constant at this
value, if ``limited\_memory\_initialization'' is
``constant''. The valid range for this real option is 
$0 <  {\tt limited\_memory\_init\_val } <  {\tt +inf}$
and its default value is $1$.


\paragraph{limited\_memory\_max\_skipping:} Threshold for successive iterations where update is skipped. $\;$ \\
 If the update is skipped more than this number of
successive iterations, we quasi-Newton
approximation is reset. The valid range for this integer option is
$1 \le {\tt limited\_memory\_max\_skipping } <  {\tt +inf}$
and its default value is $2$.


% \paragraph{derivative\_test:} Enable derivative checker $\;$ \\
%  If this option is enabled, a (slow) derivative
% test will be performed before the optimization. 
% The test is performed at the user provided
% starting point and marks derivative values that
% seem suspicious.
% The default value for this string option is ``none''.
% \\ 
% Possible values:
% \begin{itemize}
%    \item none: do not perform derivative test
%    \item first-order: perform test of first derivatives at starting
% point
%    \item second-order: perform test of first and second derivatives at
% starting point
% \end{itemize}
% 
% \paragraph{derivative\_test\_perturbation:} Size of the finite difference perturbation in derivative test. $\;$ \\
%  This determines the relative perturbation of the
% variable entries. The valid range for this real option is 
% $0 <  {\tt derivative\_test\_perturbation } <  {\tt +inf}$
% and its default value is $1 \cdot 10^{-08}$.
% 
% 
% \paragraph{derivative\_test\_tol:} Threshold for indicating wrong derivative. $\;$ \\
%  If the relative deviation of the estimated
% derivative from the given one is larger than this
% value, the corresponding derivative is marked as
% wrong. The valid range for this real option is 
% $0 <  {\tt derivative\_test\_tol } <  {\tt +inf}$
% and its default value is $0.0001$.
% 
% 
% \paragraph{derivative\_test\_print\_all:} Indicates whether information for all estimated derivatives should be printed. $\;$ \\
%  Determines verbosity of derivative checker.
% The default value for this string option is ``no''.
% \\ 
% Possible values:
% \begin{itemize}
%    \item no: Print only suspect derivatives
%    \item yes: Print all derivatives
% \end{itemize}
% 


\section{CoinOS}

GAMS/CoinOS brings the Optimization Services project to the broad audience of GAMS users.

OS (\textbf{O}ptimization \textbf{S}ervices) is an initiative to provide a set of standards for representing optimization instances, results, solver options, and communication between clients and solvers in a distributed environment using Web Services.
The code has been written primarily by Horand Gassmann, Jun Ma, and Kipp Martin.
Kipp Martin is the COIN-OR project leader for OS.

For more information we refer to
\begin{itemize}
\item the official OS web site \texttt{http://www.optimizationservices.org},
\item the OS developer web site \texttt{https://projects.coin-or.org/OS}, and
\item the OS manual available on the OS websites.
\end{itemize}

The OS link in GAMS allows you to convert instances of GAMS models into the OS instance language (OSiL) format, let an Optimization Services Server solve your instances remotely, or solve your models locally via the OS solver interfaces (currently Bonmin, Cbc, Clp, Couenne, Glpk, and Ipopt).

\subsection{Model requirements}

OS supports continuous, binary, and integer variables, linear and nonlinear equations.

\subsection{Usage of CoinOS}

The following statement can be used inside your GAMS program to specify using CoinOS
\begin{verbatim}
  Option MINLP = CoinOS;     { or LP, RMIP, MIP, DNLP, NLP, RMINLP, QCP, RMIQCP, MIQCP }
\end{verbatim}

The above statement should appear before the Solve statement.

By default, for a given instance of a GAMS model, CoinOS selects a suitable solver, let the solver solve this instance (locally) via the corresponding OS solver interface and returns the solution to GAMS.
For continuous linear models (LP and RMIP), CoinOS chooses Clp.
For continuous nonlinear models (NLP, DNLP, RMINLP, QCP, RMIQCP), CoinOS chooses Ipopt.
For mixed-integer linear models (MIP), CoinOS chooses Cbc.
For mixed-integer nonlinear models (MIQCP, MINLP), CoinOS chooses Bonmin.
You can change the choice of the solver by setting the ``solver'' option.

For a remove solve, you have to specify the URL of an Optimization Services Server via the option ``service''.
Remove solves are experimental in CoinOS.

Solver options can be specified with an OSoL (Optimization Services Options Language) file.
The osol file is specified with the ``readosol'' option.


\begin{description}

\item[\label{readosol}\hypertarget{readosol}
{\textbf{readosol (\slshape{string})}}]\hspace{1.0in}

Specifies the name of an option file in OSoL format that is given to the OS server.
This way it is possible to pass options directly to the solvers interfaced by OS.


\item[\label{writeosil}\hypertarget{writeosil}
{\textbf{writeosil (\slshape{string})}}]\hspace{1.0in}

Specifies the name of a file in which the GAMS model instance should be writting in OSiL format.


\item[\label{writeosrl}\hypertarget{writeosrl}
{\textbf{writeosrl (\slshape{string})}}]\hspace{1.0in}

Specifies the name of a file in which the result of a solve process (solution, status, ...) should be writting in OSrL format.


\item[\label{service}\hypertarget{service}
{\textbf{service (\slshape{string})}}]\hspace{1.0in}

Specifies the URL of an Optimization Services Server.
The GAMS model is converted into OSiL format, send to the server, and the result translated back into GAMS format.
Note that by default the server chooses a solver that is appropriate to the model type.
You can change the solver with the solver option.


\item[\label{solver}\hypertarget{solver}
{\textbf{solver (\slshape{string})}}]\hspace{1.0in}

Specifies the solver that is used to solve an instance on the OS server.


\end{description}


\section{CoinScip}

GAMS/CoinScip brings the MIP solver from the Constrained Integer Programming framework SCIP to the broad audience of academic GAMS users.

The code is developed at the Konrad-Zuse-Zentrum f\"ur Informationstechnik Berlin (ZIB) and has been written primarily by T. Achterberg.
It is distributed under the ZIB Academic License.

For more information we refer to
\begin{itemize}
\item the SCIP web site \texttt{http://scip.zib.de} and
\item the Ph.D. thesis ``Constraint Integer Programming'' by Tobias Achterberg, Berlin 2007.
\end{itemize}

GAMS/CoinScip uses the COIN-OR linear solver CLP from J.J. Forrest as LP solver, see \hyperlink{sec:coincbc}{Section \ref{sec:coincbc}}.

\subsection{Model requirements}

SCIP supports continuous, binary, and integer variables, special ordered sets, and branching priorities.
Semi-continuous or semi-integer variables (see chapter 17.1 of the GAMS User's Guide) and indicator constraints are not supported by the interface yet.

\subsection{Usage of CoinScip}

The following statement can be used inside your GAMS program to specify using CoinScip
\begin{verbatim}
  Option MIP = CoinScip;     { or LP or RMIP }
\end{verbatim}

The above statement should appear before the Solve statement.
If CoinScip was specified as the default solver during GAMS installation, the above statement is not necessary.

GAMS/CoinScip supports the GAMS Branch-and-Cut-and-Heuristic (BCH) Facility.
The GAMS BCH facility automates all major steps necessary to define, execute, and control the use of user defined routines within the framework of general purpose MIP codes.
Currently supported are user defined cut generators and heuristics and the incumbent reporting callback.
Please see the BCH documentation at \texttt{http://www.gams.com/docs/bch.htm} for further information.

Information on the use of BCH callback routines is displayed in an extra column in the SCIP iteration output.
The first number in this column (below the ``BCH'' in the header) is the number of callbacks to GAMS models that have been made so far (accumulated from cutgeneration, heuristic, and incumbent callbacks).
The number below ``cut'' or ``cuts'' gives the number of cutting planes that have been generated by the users cutgenerator.
Finally, the number below ``sol'' or ``sols'' gives the number of primal solutions that have been generated by the users heuristic.
If SCIP accepts a heuristic solution as new incumbent solution, it prints a `G' in the first column of the iteration output.

\subsection{Specification of CoinScip Options}

GAMS/CoinScip currently supports the GAMS parameters reslim, iterlim, nodlim, optcr, and optca.

Further, for a MIP solve the user can specify options by a SCIP options file.
A SCIP options file consists of one option or comment per line.
A pound sign (\#) at the beginning of a line causes the entire line to be ignored.
Otherwise, the line will be interpreted as an option name and value separated by an equal sign (=) and any amount of white space (blanks or tabs).
Further, string values have to be enclosed in quotation marks.

A small example for a coinscip.opt file is:
\begin{verbatim}
  separating/maxrounds     = 0
  separating/maxroundsroot = 0
  gams/solvefinal          = FALSE
  gams/usercutcall         = "bchcutgen.gms"
\end{verbatim}
It causes GAMS/CoinScip to turn off all cut generators, to skip the final solve of the MIP with fixed discrete variables, and to use a user defined cut generator.

\subsection{Description of CoinScip options}

SCIP supports a large set of options.
Sample option files can be obtained from
\begin{verbatim}
     http://www.gams.com/~svigerske/scip1.2
\end{verbatim}

Further, there is a set of options that are specific to the GAMS/CoinScip interface, most of them for control of the GAMS BCH facility.

\begin{description}
\item[\label{scipnames}\hypertarget{scipnames}
{\textbf{gams/names (\slshape{integer})}}]\hspace{1.0in}

This option causes GAMS names for the variables and equations to be loaded into SCIP.
These names will then be used for error messages, log entries, and so forth.
Turning names off may help if memory is very tight.

\textsl{(default = FALSE)}
\begin{itemize}
\item[FALSE] Do not load variable and equation names.
\item[TRUE] Load variable and equation names.
\end{itemize}


\item[\label{scipsolvefinal}\hypertarget{scipsolvefinal}
{\textbf{gams/solvefinal (\slshape{integer})}}]\hspace{1.0in}

Sometimes the solution process after the branch-and-cut that solves the problem with fixed discrete variables takes a long time and the user is interested in the primal values of the solution only.
In these cases, this option can be used to turn this final solve off.
Without the final solve no proper marginal values are available and only zeros are returned to GAMS.

\textsl{(default = TRUE)}
\begin{itemize}
\item[FALSE] Do not solve the fixed problem.
\item[TRUE] Solve the fixed problem and return duals.
\end{itemize}


\item[\label{scipmipstart}\hypertarget{scipmipstart}
{\textbf{gams/mipstart (\slshape{integer})}}]\hspace{1.0in}

This option controls the use of advanced starting values for mixed integer programs.
A setting of TRUE indicates that the variable level values should be checked to see if they provide an integer feasible solution before starting optimization.

\textsl{(default = TRUE)}
\begin{itemize}
\item[FALSE] Do not use the initial variable levels.
\item[TRUE] Try to use the initial variable levels as a MIP starting solution.
\end{itemize}


\item[\label{scipprintstat}\hypertarget{scipprintstat}
{\textbf{gams/print\_statistics (\slshape{integer})}}]\hspace{1.0in}

This option controls the printing of solve statistics after a MIP solve.
Turning on this option indicates that statistics like the number of
generated cuts of each type or the calls of heuristics are printed after the
MIP solve.

\textsl{(default = FALSE)}
\begin{itemize}
\item[FALSE] Do not print statistics.
\item[TRUE] Print statistics.
\end{itemize}


\item[\label{scipusercutcall}\hypertarget{scipusercutcall}
{\textbf{gams/usercutcall (\slshape{string})}}]\hspace{1.0in}

The GAMS command line (minus the gams executable name) to call the cut generator.


\item[\label{scipusercutfirst}\hypertarget{scipusercutfirst}
{\textbf{gams/usercutfirst (\slshape{integer})}}]\hspace{1.0in}

Calls the cut generator for the first $n$ nodes.

\textsl{(default = 10)}

\item[\label{scipusercutfreq}\hypertarget{scipusercutfreq}
{\textbf{gams/usercutfreq (\slshape{integer})}}]\hspace{1.0in}

Determines the frequency of the cut generator model calls.

\textsl{(default = 10)}

\item[\label{scipusercutinterval}\hypertarget{scipusercutinterval}
{\textbf{gams/usercutinterval (\slshape{integer})}}]\hspace{1.0in}

Determines the interval when to apply the multiplier for the frequency of the cut generator model calls.
See gams/userheurinterval for details.

\textsl{(default = 100)}

\item[\label{scipusercutmult}\hypertarget{scipusercutmult}
{\textbf{gams/usercutmult (\slshape{integer})}}]\hspace{1.0in}

Determines the multiplier for the frequency of the cut generator model calls.

\textsl{(default = 2)}

\item[\label{scipusercutnewint}\hypertarget{scipusercutnewint}
{\textbf{gams/usercutnewint (\slshape{integer})}}]\hspace{1.0in}

Calls the cut generator if the solver found a new integer feasible solution.

\textsl{(default = TRUE)}
\begin{itemize}
\item[FALSE] Do not call cut generator because a new integer feasible solution is found.
\item[TRUE] Let SCIP call the cut generator if a new integer feasible solution is found.
\end{itemize}

\item[\label{scipusergdxin}\hypertarget{scipusergdxin}
{\textbf{gams/usergdxin (\slshape{string})}}]\hspace{1.0in}

The name of the GDX file read back into SCIP.

\textsl{(default =} \verb=bchin.gdx=)

\item[\label{scipusergdxname}\hypertarget{scipusergdxname}
{\textbf{gams/usergdxname (\slshape{string})}}]\hspace{1.0in}

The name of the GDX file exported from the solver with the solution at the node.

\textsl{(default =} \verb=bchout.gdx=)

\item[\label{scipusergdxnameinc}\hypertarget{scipusergdxnameinc}
{\textbf{gams/usergdxnameinc (\slshape{string})}}]\hspace{1.0in}

The name of the GDX file exported from the solver with the incumbent solution.

\textsl{(default =} \verb=bchout_i.gdx=)

\item[\label{scipusergdxprefix}\hypertarget{scipusergdxprefix}
{\textbf{gams/usergdxprefix (\slshape{string})}}]\hspace{1.0in}

Prefixes to use for gams/usergdxin, gams/usergdxname, and gams/usergdxnameinc.


\item[\label{scipuserheurcall}\hypertarget{scipuserheurcall}
{\textbf{gams/userheurcall (\slshape{string})}}]\hspace{1.0in}

The GAMS command line (minus the gams executable name) to call the heuristic.


\item[\label{scipuserheurfirst}\hypertarget{scipuserheurfirst}
{\textbf{gams/userheurfirst (\slshape{integer})}}]\hspace{1.0in}

Calls the heuristic for the first $n$ nodes.

\textsl{(default = 10)}

\item[\label{scipuserheurfreq}\hypertarget{scipuserheurfreq}
{\textbf{gams/userheurfreq (\slshape{integer})}}]\hspace{1.0in}

Determines the frequency of the heuristic model calls.

\textsl{(default = 10)}

\item[\label{scipuserheurinterval}\hypertarget{scipuserheurinterval}
{\textbf{gams/userheurinterval (\slshape{integer})}}]\hspace{1.0in}

Determines the interval when to apply the multiplier for the frequency of the heuristic model calls.
For example, for the first 100 (gams/userheurinterval) nodes, the solver call every 10th (gams/userheurfreq) node the heuristic.
After 100 nodes, the frequency gets multiplied by 10 (gams/userheurmult), so that for the next 100 node the solver calls the heuristic every 20th node.
For nodes 200-300, the heuristic get called every 40th node, for nodes 300-400 every 80th node and after node 400 every 100th node.

\textsl{(default = 100)}

\item[\label{scipuserheurmult}\hypertarget{scipuserheurmult}
{\textbf{gams/userheurmult (\slshape{integer})}}]\hspace{1.0in}

Determines the multiplier for the frequency of the heuristic model calls.

\textsl{(default = 2)}

\item[\label{scipuserheurnewint}\hypertarget{scipuserheurnewint}
{\textbf{gams/userheurnewint (\slshape{integer})}}]\hspace{1.0in}

Calls the heuristic if the solver found a new integer feasible solution.

\textsl{(default = TRUE)}
\begin{itemize}
\item[FALSE] Do not call heuristic because a new integer feasible solution is found.
\item[TRUE] Let SCIP call the heuristic if a new integer feasible solution is found.
\end{itemize}

\item[\label{scipuserheurobjfirst}\hypertarget{scipuserheurobjfirst}
{\textbf{gams/userheurobjfirst (\slshape{integer})}}]\hspace{1.0in}

Similar to gams/userheurfirst but only calls the heuristic if the relaxed objective value promises a significant improvement of the current incumbent, i.e., the LP value of the node has to be closer to the best bound than the current incumbent.

\textsl{(default = FALSE)}

\item[\label{scipuserjobid}\hypertarget{scipuserjobid}
{\textbf{gams/userjobid (\slshape{string})}}]\hspace{1.0in}

Postfixes to use for gams/gdxname, gams/gdxnameinc, and gams/gdxin.


\item[\label{scipuserkeep}\hypertarget{scipuserkeep}
{\textbf{gams/userkeep (\slshape{integer})}}]\hspace{1.0in}

Calls gamskeep instead of gams

\textsl{(default = FALSE)}

\end{description}

\section{CoinCplex, CoinGurobi, CoinMosek, CoinXpress}

GAMS/CoinCplex, GAMS/CoinGurobi, GAMS/CoinMosek, GAMS/CoinXpress bring the open source Open Solver Interface (OSI) to the broad audience of GAMS users.

These ``bare bone'' solver links allow users to solve their GAMS models with a standalone license of Cplex, Gurobi, Mosek, or Xpress.
The links use the COIN-OR Open Solver Interface (OSI) to communicate with these solvers.
The Osi/Cplex link has been written primarily by Tobias Achterberg,
the Osi/Gurobi link has been written primarily by Stefan Vigerske,
the Osi/Mosek link has been written primarily by Bo Jensen, and
the Osi/Xpress link has been written primarily by John Doe.
Matthew Saltzman is the COIN-OR project leader for OSI.

For more information we refer to the OSI web site \texttt{https://projects.coin-or.org/Osi}.

\subsection{Model requirements}

The OSI links support linear equations and continuous, binary, and integer variables.
Semicontinuous and Semiinteger variables, special ordered sets, branching priorities, and indicator constraints are not supported by OSI.

\subsection{Usage of these links}

The following statement can be used inside your GAMS program to specify using CoinGurobi
\begin{verbatim}
  Option MIP = CoinGurobi;     { or LP or RMIP }
\end{verbatim}

The above statement should appear before the Solve statement.

The links only support the general GAMS options reslim, optca (except for Gurobi), optcr, nodlim, and iterlim.
In addition an option file in the format required by the solver can be provided via the GAMS optfile option.

\chapterend

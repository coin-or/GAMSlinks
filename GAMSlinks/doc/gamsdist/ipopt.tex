\chapter{\IPOPT and \IPOPTH}
\label{cha:ipopt}

%\minitoc

COIN-OR \IPOPT (\textbf{I}nterior \textbf{P}oint \textbf{Opt}imizer) is an open-source solver for large-scale nonlinear programming.
The code has been written primarily by Andreas W\"achter, who is the COIN-OR project leader for \IPOPT.

\IPOPT implements an interior point line search filter method for nonlinear programming models which functions can be nonconvex, but should be twice continuously differentiable.
For more information on the algorithm we refer to~\cite{NoWaWa08,Waechter2002,WaBi05b,WaBi05a,WaBi2006} and the \IPOPT web site \url{https://projects.coin-or.org/Ipopt}.
Most of the \IPOPT documentation in the section was taken from the \IPOPT manual~\cite{IpoptManual}.



\section{The linear solver in \IPOPT}
\label{sec:ipoptlinearsolver}
\hypertarget{ipoptlinearsolver}{}

The performance and robustness of \IPOPT on larger models heavily relies on the used solver for sparse symmetric indefinite linear systems.

\GAMS/\IPOPT includes the sparse solver \textsc{MUMPS}~\cite{AmestoyDuffKosterLExcellent2001,AmestoyGuermoucheLExcellentPralet2006} (currently the default), cf.~\url{http://graal.ens-lyon.fr/MUMPS} and \textsc{MKL PARDISO}~\cite{SchGa04,SchGa06} (only Linux and Windows).
In the commerically licensed \GAMS/\IPOPTH version, also the Harwell Subroutine Library (HSL) solvers \textsc{MA27}, \textsc{MA57}, \textsc{HSL\_MA86}, and \textsc{HSL\_MA97} are available and MA27 is used by default.

\textsc{MUMPS}, \textsc{MA57}, \textsc{HSL\_MA86}, and \textsc{HSL\_MA97} use \textsc{METIS} for matrix ordering \cite{KaKu99}, cf.~\url{http://glaros.dtc.umn.edu/gkhome/views/metis/index.html} and \url{http://glaros.dtc.umn.edu/gkhome/fetch/sw/metis/manual.pdf}.
\textsc{METIS} is copyrighted by the regents of the University of Minnesota.

\IPOPT and \IPOPTH can exploit parallelization of the linear solver or the linear algebra routines (Blas and Lapack).
The following table summarizes which options are available on which platform.

\begin{tabular}{l|c|cc|cccc}
& \multicolumn{3}{c|}{\IPOPT and \IPOPTH} & \multicolumn{4}{c}{\IPOPTH only} \\
        & Linear Algebra & MUMPS & MKL PARDISO & MA27 & MA57 & HSL MA86 & HSL MA97 \\ \hline
Linux   & parallel & serial & parallel      & serial & serial & parallel & parallel \\
MacOS X & parallel & serial & not available & serial & serial & parallel & parallel  \\
Solaris & serial   & serial & not available & serial & serial & parallel & parallel  \\
Windows & parallel & serial & parallel      & serial & serial & parallel & parallel  \\
\end{tabular}

The linear solver is chosen by the \texttt{linear\_solver} option.
Benchmarks have shown that \textsc{MA57} and \textsc{HSL\_MA97} are often able to outperform \textsc{MA27} on larger instances. Further, \textsc{PARDISO} often allows for performance that is better than \textsc{MUMPS} and similar to the HSL solvers. If \IPOPT fails to solve an instance with \textsc{PARDISO}, it's worth to try changing the options \texttt{pardiso\_order} and \texttt{pardiso\_max\_iterative\_refinement\_steps}.

\section{Usage}

The following statement can be used inside your \GAMS program to specify using \IPOPT
\begin{verbatim}
  Option NLP = IPOPT;     { or LP, RMIP, DNLP, RMINLP, QCP, RMIQCP }
\end{verbatim}

The above statement should appear before the Solve statement.
If \IPOPT was specified as the default solver during \GAMS installation, the above statement is not necessary.

To use \IPOPTH, the statement should be
\begin{verbatim}
  Option NLP = IPOPTH;    { or LP, RMIP, DNLP, RMINLP, QCP, RMIQCP }
\end{verbatim}



\paragraph{Using Harwell Subroutine Library routines with \GAMS/\IPOPT.}

\GAMS/\IPOPT can use the HSL routines \texttt{MA27}, \texttt{MA28}, \texttt{MA57}, \textsc{HSL\_MA77}, \textsc{HSL\_MA86}, \textsc{HSL\_MA97}, \texttt{MC19}, and \textsc{HSL\_MC68} when provided as shared library.
By telling \IPOPT to use one of these routines (see options \texttt{linear\_solver}, \texttt{linear\_system\_scaling}, \texttt{nlp\_scaling\_method}, \texttt{dependency\_detector}), \GAMS/\IPOPT attempts to load the required routines from the library \texttt{libhsl.so} (Unix-Systems), \texttt{libhsl.dylib} (MacOS X), or \texttt{libhsl.dll} (Windows), respectively.

The HSL routines are available at \url{http://www.hsl.rl.ac.uk/ipopt}.
Note that it is your responsibility to ensure that you are entitled to download and use these routines!
% You can build a shared library using the ThirdParty/HSL project at COIN-OR.

\paragraph{Using PARDISO with \GAMS/\IPOPT or \GAMS/\IPOPTH.}
On Mac OS X and Solaris, setting the option \texttt{linear\_solver} to \texttt{pardiso} lets \GAMS/\IPOPT or \GAMS/\IPOPTH try to load the linear solver PARDISO from the library \texttt{libpardiso.so} (Unix) or \texttt{libpardiso.dylib} (MacOS X), respectively.

PARDISO is available as compiled shared library for several platforms at \texttt{http://www.pardiso-project.org}.
Note that it is your responsibility to ensure that you are entitled to download and use this package!

\subsection{Specification of Options}
\label{sub:ipoptoptionspec}

\IPOPT has many options that can be adjusted for the algorithm (see Section \ref{sub:ipoptoptions}).
Options are all identified by a string name, and their values can be of one of three types: Number (real), Integer, or String.
Number options are used for things like tolerances, integer options are used for things like maximum number of iterations, and string options are used for setting algorithm details, like the NLP scaling method.
Options can be set by creating a \texttt{ipopt.opt} file in the directory you are executing \IPOPT.

The \texttt{ipopt.opt} file is read line by line and each line should contain the option name, followed by whitespace, and then the value.
Comments can be included with the \# symbol. Don't forget to ensure you have a newline at the end of the file. For example,
\begin{verbatim}
# This is a comment

# Turn off the NLP scaling
nlp_scaling_method none

# Change the initial barrier parameter
mu_init 1e-2

# Set the max number of iterations
max_iter 500
\end{verbatim}
is a valid \texttt{ipopt.opt} file.

% You can print the documentation for all \IPOPT options by using the option
% \begin{verbatim}
% print_options_documentation yes
% \end{verbatim}
% and running \IPOPT.
% This will output all of the options documentation to the console.

\GAMS/\IPOPT understand currently the following \GAMS parameters: \texttt{reslim} (time limit), \texttt{iterlim} (iteration limit), \texttt{domlim} (domain violation limit).
You can set them either on the command line, e.g. \verb+iterlim=500+, or inside your \GAMS program, e.g. \verb+Option iterlim=500;+.
Further the option \texttt{threads} can be used to control the number of threads used in the linear algebra routines and the linear solver, see also Section~\ref{sec:ipoptlinearsolver}.

\subsection{Warmstarting Ipopt}

As an interior point solver, it is difficult to warm start \IPOPT.
By default, only the level values of the variables are passed as starting point to \IPOPT.
Setting the \IPOPT option \texttt{warm\_start\_init\_point} to \texttt{yes} enables that also dual values for variables and constraints are passed to \IPOPT.

However, the expected behavior that \IPOPT finishes within one iteration if optimal primal and dual values are passed is not reached this way, yet. This is, because \IPOPT by default moves any initial value that is close to a bound into the interior. The amount on how much the initial point is moved can be controlled by various \texttt{bound\_push} and \texttt{bound\_frac} options.
To make \IPOPT accept an optimal primal/dual solution within one iteration, it should be sufficient to set the following options:
\begin{verbatim}
  warm_start_init_point       yes
  warm_start_bound_push       1e-9
  warm_start_bound_frac       1e-9
  warm_start_slack_bound_frac 1e-9
  warm_start_slack_bound_push 1e-9
  warm_start_mult_bound_push  1e-9
\end{verbatim}

\section{Output}

This section describes the standard \IPOPT console output.
The output is designed to provide a quick summary of each iteration as \IPOPT solves the problem.

Before \IPOPT starts to solve the problem, it displays the problem statistics (number of nonzero-elements in the matrices, number of variables, etc.).
Note that if you have fixed variables (both upper and lower bounds are equal), \IPOPT may remove these variables from the problem internally and not include them in the problem statistics.

Following the problem statistics, \IPOPT will begin to solve the problem and you will see output resembling the following,
\begin{verbatim}
iter    objective    inf_pr   inf_du lg(mu)  ||d||  lg(rg) alpha_du alpha_pr  ls
   0  1.6109693e+01 1.12e+01 5.28e-01   0.0 0.00e+00    -  0.00e+00 0.00e+00   0
   1  1.8029749e+01 9.90e-01 6.62e+01   0.1 2.05e+00    -  2.14e-01 1.00e+00f  1
   2  1.8719906e+01 1.25e-02 9.04e+00  -2.2 5.94e-02   2.0 8.04e-01 1.00e+00h  1
\end{verbatim}
and the columns of output are defined as
\begin{description}
\item[iter]
The current iteration count.
This includes regular iterations and iterations while in restoration phase.
If the algorithm is in the restoration phase, the letter \texttt{r} will be appended to the iteration number.
\item[objective]
The unscaled objective value at the current point.
During the restoration phase, this value remains the unscaled objective value for the original problem.
\item[inf\_pr]
The unscaled constraint violation at the current point.
This quantity is the infinity-norm (max) of the (unscaled) constraint violation.
During the restoration phase, this value remains the constraint violation of the original problem at the current point.
The option ``\texttt{inf\_pr\_output}'' can be used to switch to the printing of a different quantity.
During the restoration phase, this value is the primal infeasibility of the original problem at the current point.
\item[inf\_du]
The scaled dual infeasibility at the current point.
This quantity measure the infinity-norm (max) of the internal dual infeasibility \cite[Eq.~(4a)]{WaBi2006}, including inequality constraints reformulated using slack variables and problem scaling.
During the restoration phase, this is the value of the dual infeasibility for the restoration phase problem.
\item[lg(mu)]
$\log_{10}$ of the value of the barrier parameter $\mu$.
\item[$\Vert d\Vert$]
The infinity norm (max) of the primal step (for the original variables $x$ and the internal slack variables $s$).
During the restoration phase, this value includes the values of additional variables, $p$ and $n$ \cite[Eq.~(10)]{WaBi2006}.
\item[lg(rg)]
$\log_{10}$ of the value of the regularization term for the Hessian of the Lagrangian in the augmented system ($\delta_w$ in \cite[Eq.~(26)]{WaBi2006}).
A dash (``\texttt{-}'') indicates that no regularization was done.
\item[alpha\_du]
The stepsize for the dual variables ($\alpha^z_k$ in \cite[Eq.~(14c)]{WaBi2006})..
\item[alpha\_pr]
The stepsize for the primal variables ($\alpha_k$ in \cite[Eq.~(14a)]{WaBi2006}).
The number is usually followed by a character for additional diagnostic information regarding the step acceptance criterion:
 \begin{list}{blub}{\itemsep0pt}
    \item[\texttt{f}] f-type iteration in the filter method w/o second order correction
    \item[\texttt{F}] f-type iteration in the filter method w/ second order correction
    \item[\texttt{h}] h-type iteration in the filter method w/o second order correction
    \item[\texttt{H}] h-type iteration in the filter method w/ second order correction
    \item[\texttt{k}] penalty value unchanged in merit function method w/o second order correction
    \item[\texttt{K}] penalty value unchanged in merit function method w/ second order correction
    \item[\texttt{n}] penalty value updated in merit function method w/o second order correction
    \item[\texttt{N}] penalty value updated in merit function method w/ second order correction
    \item[\texttt{R}] Restoration phase just started
    \item[\texttt{w}] in watchdog procedure
    \item[\texttt{s}] step accepted in soft restoration phase
    \item[\texttt{t}/\texttt{T}] tiny step accepted without line search
    \item[\texttt{r}] some previous iterate restored
 \end{list}
\item[ls]
The number of backtracking line search steps (does not include second-order correction steps).
\end{description}

Note that the step acceptance mechanisms in \IPOPT consider the
barrier objective function \cite[Eq.~(3a)]{WaBi2006} which is
usually different from the value reported in the \texttt{objective}
column.  Similarly, for the purposes of the step acceptance, the
constraint violation is measured for the internal problem formulation,
which includes slack variables for inequality constraints and
potentially scaling of the constraint functions.  This value, too, is
usually different from the value reported in \texttt{inf\_pr}.  As a
consequence, a new iterate might have worse values both for the
objective function and the constraint violation as reported in the
iteration output, seemingly contradicting globalization procedure.


When the algorithm terminates, \IPOPT will output a message to the screen.
The following is a list of the possible output messages and a brief description.

\begin{description}
\item[Optimal Solution Found.] ~

    This message indicates that \IPOPT found a (locally) optimal point within the desired tolerances.

\item[Solved To Acceptable Level.] ~

    This indicates that the algorithm did not converge to the ``desired'' tolerances, but that it was able to obtain a point satisfying the ``acceptable'' tolerance level as specified by \texttt{acceptable-*} options.
    This may happen if the desired tolerances are too small for the current problem.

\item[Feasible point for square problem found.] ~

    This message is printed if the problem is ``square'' (i.e., it has as many equality constraints as free variables) and \IPOPT found a feasible point.

\item[Converged to a point of local infeasibility. Problem may be infeasible.] ~

    The restoration phase converged to a point that is a minimizer for the constraint violation (in the $\ell_1$-norm), but is not feasible for the original problem.
    This indicates that the problem may be infeasible (or at least that the algorithm is stuck at a locally infeasible point).
    The returned point (the minimizer of the constraint violation) might help you to find which constraint is causing the problem.
    If you believe that the NLP is feasible, it might help to start the optimization from a different point.

\item[Search Direction is becoming Too Small.] ~

    This indicates that \IPOPT is calculating very small step sizes and making very little progress.
    This could happen if the problem has been solved to the best numerical accuracy possible given the current scaling.

\item[Iterates divering; problem might be unbounded.] ~

    This message is printed if the max-norm of the iterates becomes larger than the value of the option \texttt{diverging\_iterates\_tol}.
    This can happen if the problem is unbounded below and the iterates are diverging.

\item[Stopping optimization at current point as requested by user.] ~

    This message is printed if either the Ctrl+C was pressed or the domain violation limit is reached.

\item[Maximum Number of Iterations Exceeded.] ~

    This indicates that \IPOPT has exceeded the maximum number of iterations as specified by the \IPOPT option \texttt{max\_iter} or the GAMS option \texttt{iterlim}.

\item[Maximum CPU time exceeded.] ~

    This indicates that \IPOPT has exceeded the maximum number of seconds as specified by the \IPOPT option \texttt{max\_cpu\_time} or the GAMS option \texttt{reslim}.

\item[Restoration Failed!] ~

    This indicates that the restoration phase failed to find a feasible point that was acceptable to the filter line search for the original problem.
    This could happen if the problem is highly degenerate or does not satisfy the constraint qualification, or if an external function in \GAMS provides incorrect derivative information.

\item[Error in step computation (regularization becomes too large?)!] ~

    This messages is printed if \IPOPT is unable to compute a search direction, despite several attempts to modify the iteration matrix.
    Usually, the value of the regularization parameter then becomes too large.

\item[Problem has too few degrees of freedom.] ~

    This indicates that your problem, as specified, has too few degrees of freedom.
    This can happen if you have too many equality constraints, or if you fix too many variables (\IPOPT removes fixed variables).

\item[Not enough memory.] ~

    An error occurred while trying to allocate memory.
    The problem may be too large for your current memory and swap configuration.

\item[INTERNAL ERROR: Unknown SolverReturn value - Notify \IPOPT Authors.] ~

    An unknown internal error has occurred. Please notify the authors of the \GAMS/\IPOPT link or \IPOPT (refer to \url{https://projects.coin-or.org/GAMSlinks} or \url{https://projects.coin-or.org/Ipopt}).
\end{description}


\subsection{Diagnostic Tags for \IPOPT}

To print additional diagnostic tags for each iteration of \IPOPT, set
the options \texttt{print\_info\_string} to \texttt{yes}. With
this, a tag will appear at the end of an iteration line with the
following diagnostic meaning that are useful to flag difficulties for
a particular \IPOPT run.  The following is a list of possible strings:
\begin{list}{blub}{\itemsep0pt}
 \item[\texttt{!}] Tighten resto tolerance if only slightly infeasible \cite[Sec.~3.3]{WaBi2006}
 \item[\texttt{A}] Current iteration is acceptable (alternate termination)
 \item[\texttt{a}] Perturbation for PD Singularity can't be done, assume singular \cite[Sec.~3.1]{WaBi2006}
 \item[\texttt{C}] Second Order Correction taken \cite[Sec.~2.4]{WaBi2006}
 \item[\texttt{Dh}] Hessian degenerate based on multiple iterations \cite[Sec.~3.1]{WaBi2006}
 \item[\texttt{Dhj}] Hessian/Jacobian degenerate based on multiple iterations \cite[Sec.~3.1]{WaBi2006}
 \item[\texttt{Dj}] Jacobian degenerate based on multiple iterations \cite[Sec.~3.1]{WaBi2006}
 \item[\texttt{dx}] $\delta_x$ perturbation too large \cite[Sec.~3.1]{WaBi2006}
 \item[\texttt{e}] Cutting back $\alpha$ due to evaluation error (in backtracking line search)
 \item[\texttt{F-}] Filter should be reset, but maximal resets exceeded \cite[Sec.~2.3]{WaBi2006}
 \item[\texttt{F+}] Resetting filter due to last few rejections of filter \cite[Sec.~2.3]{WaBi2006}
 \item[\texttt{L}] Degenerate Jacobian, $\delta_c$ already perturbed \cite[Sec.~3.1]{WaBi2006}
 \item[\texttt{l}] Degenerate Jacobian, $\delta_c$ perturbed \cite[Sec.~3.1]{WaBi2006}
 \item[\texttt{M}] Magic step taken for slack variables (in backtracking line search)
 \item[\texttt{Nh}] Hessian not yet degenerate \cite[Sec.~3.1]{WaBi2006}
 \item[\texttt{Nhj}] Hessian/Jacobian not yet degenerate \cite[Sec.~3.1]{WaBi2006}
 \item[\texttt{Nj}] Jacobian not yet degenerate \cite[Sec.~3.1]{WaBi2006}
 \item[\texttt{NW}] Warm start initialization failed (in Warm Start Initialization)
 \item[\texttt{q}] PD system possibly singular, attempt to improve solution quality \cite[Sec.~3.1]{WaBi2006}
 \item[\texttt{R}] Solution of restoration phase \cite[Sec.~3.3]{WaBi2006}
 \item[\texttt{S}] PD system possibly singular, accept current solution \cite[Sec.~3.1]{WaBi2006}
 \item[\texttt{s}] PD system singular \cite[Sec.~3.1]{WaBi2006}
 \item[\texttt{s}] Square Problem. Set multipliers to zero (default initialization routine)
 \item[\texttt{Tmax}] Trial $\theta$ is larger than $\theta_{max}$ (filter parameter \cite[Eq.~(21)]{WaBi2006})
 \item[\texttt{W}] Watchdog line search procedure successful \cite[Sec.~3.2]{WaBi2006}
 \item[\texttt{w}] Watchdog line search procedure unsuccessful, stopped \cite[Sec.~3.2]{WaBi2006}
 \item[\texttt{Wb}] Undoing most recent SR1 update \cite[Sec.~5.4.1]{Biegler2010}
 \item[\texttt{We}] Skip Limited-Memory Update in restoration phase  \cite[Sec.~5.4.1]{Biegler2010}
 \item[\texttt{Wp}] Safeguard $B^0 = \sigma I$ for  Limited-Memory Update \cite[Sec.~5.4.1]{Biegler2010}
 \item[\texttt{Wr}] Resetting Limited-Memory Update \cite[Sec.~5.4.1]{Biegler2010}
 \item[\texttt{Ws}] Skip Limited-Memory Update since $s^Ty$ is not positive \cite[Sec.~5.4.1]{Biegler2010}
 \item[\texttt{WS}] Skip Limited-Memory Update since $\Delta x$ is too small \cite[Sec.~5.4.1]{Biegler2010}
 \item[\texttt{y}] Dual infeasibility, use least square multiplier update (during \IPOPT algorithm)
 \item[\texttt{z}] Apply correction to bound multiplier if too large (during \IPOPT algorithm)
\end{list}

\section{Detailed Options Description}
\label{sub:ipoptoptions}

% Note, that \GAMS/\IPOPT overwrites the \IPOPT default setting for the parameters \texttt{bound\_relax\_factor} (set to $10^{-10}$) and \texttt{mu\_strategy} (set to \texttt{adaptive}).
% You can change these values by specifying these options in your \IPOPT options file.

\subsubsection{Output}

\paragraph{print\_level:} Output verbosity level. $\;$ \\
 Sets the default verbosity level for console
output. The larger this value the more detailed
is the output. The valid range for this integer option is
$0 \le {\tt print\_level } \le 11$
and its default value is $4$.


\paragraph{print\_user\_options:} Print all options set by the user. $\;$ \\
 If selected, the algorithm will print the list of
all options set by the user including their
values and whether they have been used.
The default value for this string option is ``no''.
\\ 
Possible values:
\begin{itemize}
   \item no: don't print options
   \item yes: print options
\end{itemize}

\paragraph{print\_options\_documentation:} Switch to print all algorithmic options. $\;$ \\
 If selected, the algorithm will print the list of
all available algorithmic options with some
documentation before solving the optimization
problem.
The default value for this string option is ``no''.
\\ 
Possible values:
\begin{itemize}
   \item no: don't print list
   \item yes: print list
\end{itemize}

\paragraph{output\_file:} File name of desired output file (leave unset for no file output). $\;$ \\
An output file with this
name will be written (leave unset for no file
output).  The verbosity level is by default set
to ``print\_level'', but can be overridden with
``file\_print\_level''.  The file name is changed
to use only small letters.
With the default settings no output file is generated.
\\ 
Possible values:
\begin{itemize}
   \item *: Any acceptable standard file name
\end{itemize}

\paragraph{file\_print\_level:} Verbosity level for output file. $\;$ \\
 NOTE: This option only works when read from the
ipopt.opt options file! Determines the verbosity
level for the file specified by ``output\_file''.
By default it is the same as ``print\_level''. The valid range for this integer option is
$0 \le {\tt file\_print\_level } \le 11$
and its default value is $4$.

\subsubsection{Termination}

\paragraph{tol:} Desired convergence tolerance (relative). $\;$ \\
 Determines the convergence tolerance for the
algorithm.  The algorithm terminates
successfully, if the (scaled) NLP error becomes
smaller than this value, and if the (absolute)
criteria according to ``dual\_inf\_tol'',
``primal\_inf\_tol'', and ``cmpl\_inf\_tol'' are met.
 (This is $\varepsilon_\mathrm{tol}$ in Eqn. (6) in the
implementation paper).  See also
``acceptable\_tol'' as a second termination
criterion.  Note, some other algorithmic features
also use this quantity to determine thresholds
etc. The valid range for this real option is 
$0 <  {\tt tol } <  {\tt +inf}$
and its default value is $1 \cdot 10^{-08}$.

\paragraph{s\_max:} Scaling threshold for the NLP error. $\;$ \\
The valid range for this integer option is
$0 \le {\tt s\_max } <  {\tt +inf}$
and its default value is $100$.

\paragraph{max\_iter:} Maximum number of iterations. $\;$ \\
 The algorithm terminates with an error message if
the number of iterations exceeded this number. The valid range for this integer option is
$0 \le {\tt max\_iter } <  {\tt +inf}$
and its default value is the value of the GAMS parameter iterlim, which default value is $10000$.


\paragraph{compl\_inf\_tol:} Desired threshold for the complementarity conditions. $\;$ \\
 Absolute tolerance on the complementarity.
Successful termination requires that the max-norm
of the (unscaled) complementarity is less than
this threshold. The valid range for this real option is 
$0 <  {\tt compl\_inf\_tol } <  {\tt +inf}$
and its default value is $0.0001$.


\paragraph{constr\_viol\_tol:} Desired threshold for the constraint violation. $\;$ \\
 Absolute tolerance on the constraint violation.
Successful termination requires that the max-norm
of the (unscaled) constraint violation is less
than this threshold. The default value for this real option is $0.0001$ and its
valid range is $0<\texttt{constr\_viol\_tol}<{\tt +inf}$.


\paragraph{dual\_inf\_tol:} Desired threshold for the dual infeasibility. $\;$ \\
 Absolute tolerance on the dual infeasibility.
Successful termination requires that the max-norm
of the (unscaled) dual infeasibility is less than
this threshold. The valid range for this real option is 
$0 <  {\tt dual\_inf\_tol } <  {\tt +inf}$
and its default value is $0.0001$.


\paragraph{acceptable\_tol:} ``Acceptable'' convergence tolerance (relative). $\;$ \\
 Determines which (scaled) overall optimality
error is considered to be ``acceptable''. There are
two levels of termination criteria.  If the usual
``desired'' tolerances (see tol, dual\_inf\_tol
etc) are satisfied at an iteration, the algorithm
immediately terminates with a success message. 
On the other hand, if the algorithm encounters
``acceptable\_iter'' many iterations in a row that
are considered ``acceptable'', it will terminate
before the desired convergence tolerance is met.
This is useful in cases where the algorithm might
not be able to achieve the ``desired'' level of
accuracy. The valid range for this real option is 
$0 <  {\tt acceptable\_tol } <  {\tt +inf}$
and its default value is $1 \cdot 10^{-06}$.

\paragraph{acceptable\_iter:} Number of ``acceptable'' iterates before triggering termination. $\;$ \\
If the algorithm encounters this many successive ``acceptable'' iterates (see ``acceptable\_tol''), it terminates, assuming that the problem has been solved to best possible accuracy given round-off.
If it is set to zero, this heuristic is disabled.
The valid range for this integer option is
$0 \le {\tt acceptable\_iter } <  {\tt +inf}$
and its default value is $15$.

\paragraph{acceptable\_compl\_inf\_tol:} ``Acceptance'' threshold for the complementarity conditions. $\;$ \\
 Absolute tolerance on the complementarity.
``Acceptable'' termination requires that the
max-norm of the (unscaled) complementarity is
less than this threshold; see also
acceptable\_tol. The valid range for this real option is 
$0 <  {\tt acceptable\_compl\_inf\_tol } <  {\tt +inf}$
and its default value is $0.01$.


\paragraph{acceptable\_constr\_viol\_tol:} ``Acceptance'' threshold for the constraint violation. $\;$ \\
 Absolute tolerance on the constraint violation.
``Acceptable'' termination requires that the
max-norm of the (unscaled) constraint violation
is less than this threshold; see also
acceptable\_tol. The valid range for this real option is 
$0 <  {\tt acceptable\_constr\_viol\_tol } <  {\tt +inf}$
and its default value is $0.01$.


\paragraph{acceptable\_dual\_inf\_tol:} ``Acceptance'' threshold for the dual infeasibility. $\;$ \\
 Absolute tolerance on the dual infeasibility.
``Acceptable'' termination requires that the
(max-norm of the unscaled) dual infeasibility is
less than this threshold; see also
acceptable\_tol. The valid range for this real option is 
$0 <  {\tt acceptable\_dual\_inf\_tol } <  {\tt +inf}$
and its default value is $0.01$.


\paragraph{diverging\_iterates\_tol:} Threshold for maximal value of primal iterates. $\;$ \\
 If any component of the primal iterates exceeded
this value (in absolute terms), the optimization
is aborted with the exit message that the
iterates seem to be diverging. The valid range for this real option is 
$0 <  {\tt diverging\_iterates\_tol } <  {\tt +inf}$
and its default value is $1 \cdot 10^{+20}$.

\subsubsection{NLP Scaling}

\paragraph{obj\_scaling\_factor:} Scaling factor for the objective function. $\;$ \\
 This option sets a scaling factor for the
objective function. The scaling is seen
internally by Ipopt but the unscaled objective is
reported in the console output. If additional
scaling parameters are computed (e.g.
user-scaling or gradient-based), both factors are
multiplied. The valid range for this real option is 
${\tt -inf} <  {\tt obj\_scaling\_factor } <  {\tt +inf}$
and its default value is $1$.


\paragraph{nlp\_scaling\_method:} Select the technique used for scaling the NLP. $\;$ \\
 Selects the technique used for scaling the
problem internally before it is solved. For
user-scaling, the parameters come from the values of the .scale suffix in GAMS.
The default value for this string option is ``gradient-based'' if scaleopt is 0 (default).
If the user provides variable or equation scaling values in GAMS and sets $<$model$>$.scaleopt to 1, then the default for this parameter is ``user-scaling''.
\\ 
Possible values:
\begin{itemize}
   \item none: no problem scaling will be performed
   \item user-scaling: scaling parameters will come from the user
   \item gradient-based: scale the problem so the maximum gradient at
the starting point is scaling\_max\_gradient
   \item equilibration-based: scale the problem so that first derivatives are
of order 1 at random points (only available with MC19)
\end{itemize}

\paragraph{nlp\_scaling\_max\_gradient:} Maximum gradient after NLP scaling. $\;$ \\
 This is the gradient scaling cut-off. If the
maximum gradient is above this value, then
gradient based scaling will be performed. Scaling
parameters are calculated to scale the maximum
gradient back to this value. (This is $g_{\max}$ in
Section 3.8 of the implementation paper.) Note:
This option is only used if
``nlp\_scaling\_method'' is chosen as
``gradient-based''. The valid range for this real option is 
$0 <  {\tt nlp\_scaling\_max\_gradient } <  {\tt +inf}$
and its default value is $100$.

\paragraph{nlp\_scaling\_obj\_target\_gradient:} Target value for objective function gradient size. $\;$ \\
     If a positive number is chosen, the scaling factor the objective function
     is computed so that the gradient as the max norm of the given size at the
     starting point.  This overrides nlp\_scaling\_max\_gradient for the
     objective function.
The valid range for this real option is 
$0 <  {\tt nlp\_scaling\_obj\_target\_gradient } <  {\tt +inf}$
and its default value is $0$.

\subsubsection{NLP corrections}

\paragraph{dependency\_detector:} Indicates which linear solver should be used to detect linearly dependent equality constraints. $\;$ \\
The default value for this string option is ``none''.
\\ 
Possible values:
\begin{itemize}
\item none:                    don't check; no extra work at beginning
\item mumps:                   use MUMPS
\item ma28:                     use MA28
\end{itemize}

\paragraph{dependency\_detection\_with\_rhs:} Indicates if the right hand sides of the constraints should be considered during dependency detection. $\;$ \\
The default value for this string option is ``no''.\\
Possible values:
\begin{itemize}
\item no:                      only look at gradients
\item yes:                     also consider right hand side
\end{itemize}

\paragraph{point\_perturbation\_radius:} Maximal perturbation of an evaluation point. $\;$ \\
     If a random perturbation of a points is required, this number indicates
     the maximal perturbation.  Currently, this is only used when we perturb
     the initial point in order to get a random Jacobian for the linear
     dependency detection of equality constraints.
The valid range for this real option is 
$0 \le {\tt point\_perturbation\_radius } <  {\tt +inf}$
and its default value is $10$.

\paragraph{kappa\_d:} Weight for linear damping term (to handle one-sided bounds). $\;$ \\
The valid range for this real option is 
$0 \le {\tt kappa\_d } <  {\tt +inf}$
and its default value is $10^{-5}$.

\paragraph{bound\_relax\_factor:} Factor for initial relaxation of the bounds. $\;$ \\
 Before start of the optimization, the bounds
given by the user are relaxed.  This option sets
the factor for this relaxation.  If it is set to
zero, then then bounds relaxation is disabled.
(See Eqn.(35) in the implementation paper.)
The valid range for this real option is 
$0 \le {\tt bound\_relax\_factor } <  {\tt +inf}$
and its default value is $0$.


\paragraph{honor\_original\_bounds:} Indicates whether final points should be projected into original bounds. $\;$ \\
 Ipopt might relax the bounds during the
optimization (see, e.g., option
``bound\_relax\_factor'').  This option determines
whether the final point should be projected back
into the user-provide original bounds after the
optimization.
The default value for this string option is ``yes''.
\\ 
Possible values:
\begin{itemize}
   \item no: Leave final point unchanged
   \item yes: Project final point back into original bounds
\end{itemize}

% \paragraph{check\_derivatives\_for\_naninf:} Indicates whether it is desired to check for Nan/Inf in derivative matrices $\;$ \\
%  Activating this option will cause an error if an
% invalid number is detected in the constraint
% Jacobians or the Lagrangian Hessian.  If this is
% not activated, the test is skipped, and the
% algorithm might proceed with invalid numbers and
% fail.
% The default value for this string option is ``no''.
% \\ 
% Possible values:
% \begin{itemize}
%    \item no: Don't check (faster).
%    \item yes: Check Jacobians and Hessian for Nan and Inf.
% \end{itemize}

\paragraph{fixed\_variable\_treatment:} Determines how fixed variables should be handled. $\;$ \\
The main difference between those options is that the starting point in the ``make\_constraint'' case still has the fixed variables at their given values, whereas in the case ``make\_parameter''
the functions are always evaluated with the fixed values for those variables. 
Also, for ``relax\_bounds'', the fixing bound constraints are relaxed (according to ``bound\_relax\_factor'').
For both ``make\_constraints'' and ``relax\_bounds'', bound multipliers are computed for the fixed variables.
The default value for this string option is ``make\_parameter''.
\\ 
Possible values:
\begin{itemize}
\item make\_parameter:    Remove fixed variable from optimization variables.
\item make\_constraint:   Add equality constraints fixing variables.
\item relax\_bounds:      Relax fixing bound constraints.
\end{itemize}


\subsubsection{Initialization}

\paragraph{bound\_frac:} Desired minimum relative distance from the initial point to bound. $\;$ \\
 Determines how much the initial point might have
to be modified in order to be sufficiently inside
the bounds (together with ``bound\_push'').
(This is $\kappa_2$ in Section 3.6 of the implementation paper.)
The valid range for this real option is 
$0 <  {\tt bound\_frac } \le 0.5$
and its default value is $0.01$.


\paragraph{bound\_push:} Desired minimum absolute distance from the initial point to bound. $\;$ \\
 Determines how much the initial point might have
to be modified in order to be sufficiently inside
the bounds (together with ``bound\_frac'').
(This is $\kappa_1$ in Section 3.6 of the implementation paper.)
The valid range for this real option is 
$0 <  {\tt bound\_push } <  {\tt +inf}$
and its default value is $0.01$.


\paragraph{slack\_bound\_push:} Desired minimum absolute distance from the initial slack to bound. $\;$ \\       Determines how much the initial slack variables might have to be modified
in order to be sufficiently inside the inequality bounds (together with ``slack\_bound\_frac'').
(This is $\kappa_1$ in Section 3.6 of the implementation paper.)
The valid range for this real option is 
$0 <  {\tt slack\_bound\_push } <  {\tt +inf}$
and its default value is $0.01$.

\paragraph{slack\_bound\_frac:} Desired minimum relative distance from the initial slack to bound.
     Determines how much the initial slack variables might have to be modified
     in order to be sufficiently inside the inequality bounds (together with
     ``slack\_bound\_push'').
(This is $\kappa_2$ in Section 3.6 of the implementation paper.)
The valid range for this real option is 
$0 <  {\tt slack\_bound\_frac } \le  0.5$
and its default value is $0.01$.


\paragraph{bound\_mult\_init\_val:} Initial value for the bound multipliers. $\;$ \\
 All dual variables corresponding to bound
constraints are initialized to this value. The valid range for this real option is 
$0 <  {\tt bound\_mult\_init\_val } <  {\tt +inf}$
and its default value is $1$.


\paragraph{constr\_mult\_init\_max:} Maximum allowed least-square guess of constraint multipliers. $\;$ \\
 Determines how large the initial least-square
guesses of the constraint multipliers are allowed
to be (in max-norm). If the guess is larger than
this value, it is discarded and all constraint
multipliers are set to zero.  This options is
also used when initializing the restoration
phase. By default,
``resto.constr\_mult\_init\_max'' (the one used in
RestoIterateInitializer) is set to zero. The valid range for this real option is 
$0 \le {\tt constr\_mult\_init\_max } <  {\tt +inf}$
and its default value is $1000$.


\paragraph{bound\_mult\_init\_val:} Initial value for the bound multipliers. $\;$ \\
 All dual variables corresponding to bound
constraints are initialized to this value. The valid range for this real option is 
$0 <  {\tt bound\_mult\_init\_val } <  {\tt +inf}$
and its default value is $1$.

\paragraph{least\_square\_init\_primal:} Least square initialization of the primal variables. $\;$ \\
     If set to yes, Ipopt ignores the user provided point and solves a least
     square problem for the primal variables (x and s), to fit the linearize
     equality and inequality constraints.  This might be useful if the user
     doesn't know anything about the starting point, or for solving an LP or
     QP.
The default value for this string option is ``no''.\\
   Possible values:
\begin{itemize}
    \item no:                      take user-provided point
    \item yes:                     overwrite user-provided point with least-square estimates
\end{itemize}

\paragraph{least\_square\_init\_duals:} Least square initialization of all dual variables. $\;$ \\
     If set to yes, Ipopt tries to compute least-square multipliers
     (considering ALL dual variables).  If successful, the bound multipliers
     are possibly corrected to be at least bound\_mult\_init\_val. This might be
     useful if the user doesn't know anything about the starting point, or for
     solving an LP or QP.
The default value for this string option is ``no''.\\
   Possible values:
\begin{itemize}
    \item no:                      use bound\_mult\_init\_val and least-square equality constraint multipliers
    \item yes:                     overwrite user-provided point with least-square estimates
\end{itemize}

\subsubsection{Barrier parameter update}

\paragraph{mehrotra\_algorithm:} Indicates if we want to do Mehrotra's algorithm. $\;$ \\
 If set to yes, Ipopt runs as Mehrotra's
predictor-corrector algorithm. This works usually
very well for LPs and convex QPs.  This
automatically disables the line search, and
chooses the (unglobalized) adaptive mu strategy
with the ``probing'' oracle, and uses
``corrector\_type=affine'' without any safeguards;
you should not set any of those options
explicitly in addition.  Also, unlessotherwise
specified, the values of ``bound\_push'',
``bound\_frac'', and ``bound\_mult\_init\_val'' are
set more aggressive, and sets
``alpha\_for\_y=bound\_mult''.
The default value for this string option is ``no''.
\\ 
Possible values:
\begin{itemize}
   \item no: Do the usual Ipopt algorithm.
   \item yes: Do Mehrotra's predictor-corrector algorithm.
\end{itemize}

\paragraph{mu\_strategy:} Update strategy for barrier parameter. $\;$ \\
 Determines which barrier parameter update
strategy is to be used.
The default value for this string option is ``adaptive''.
\\ 
Possible values:
\begin{itemize}
   \item monotone: use the monotone (Fiacco-McCormick) strategy
   \item adaptive: use the adaptive update strategy
\end{itemize}

\paragraph{mu\_oracle:} Oracle for a new barrier parameter in the adaptive strategy. $\;$ \\
 Determines how a new barrier parameter is
computed in each ``free-mode'' iteration of the
adaptive barrier parameter strategy. (Only
considered if ``adaptive'' is selected for option
``mu\_strategy'').
The default value for this string option is ``quality-function''.
\\ 
Possible values:
\begin{itemize}
   \item probing: Mehrotra's probing heuristic
   \item loqo: LOQO's centrality rule
   \item quality-function: minimize a quality function
\end{itemize}

\paragraph{quality\_function\_max\_section\_steps:} Maximum number of search steps during direct search procedure determining the optimal centering parameter. $\;$ \\
 The golden section search is performed for the
quality function based mu oracle. The valid range for this integer option is
$0 \le {\tt quality\_function\_max\_section\_steps } <  {\tt +inf}$
and its default value is $8$.
This option is only used if the option ``mu\_oracle'' is set to ``quality-function''.


\paragraph{fixed\_mu\_oracle:} Oracle for the barrier parameter when switching to fixed mode. $\;$ \\
 Determines how the first value of the barrier
parameter should be computed when switching to
the ``monotone mode'' in the adaptive strategy.
(Only considered if ``adaptive'' is selected for
option ``mu\_strategy''.)
The default value for this string option is ``average\_compl''.
\\ 
Possible values:
\begin{itemize}
   \item probing: Mehrotra's probing heuristic
   \item loqo: LOQO's centrality rule
   \item quality-function: minimize a quality function
   \item average\_compl: base on current average complementarity
\end{itemize}

\paragraph{mu\_init:} Initial value for the barrier parameter. $\;$ \\
 This option determines the initial value for the
barrier parameter (mu).  It is only relevant in
the monotone, Fiacco-McCormick version of the
algorithm. (i.e., if ``mu\_strategy'' is chosen as
``monotone'') The valid range for this real option is 
$0 <  {\tt mu\_init } <  {\tt +inf}$
and its default value is $0.1$.

\paragraph{mu\_max\_fact:} Factor for initialization of maximum value for barrier parameter. $\;$ \\
 This option determines the upper bound on the
barrier parameter.  This upper bound is computed
as the average complementarity at the initial
point times the value of this option. (Only used
if option ``mu\_strategy'' is chosen as ``adaptive''.) The valid range for this real option is 
$0 <  {\tt mu\_max\_fact } <  {\tt +inf}$
and its default value is $1000$.


\paragraph{mu\_max:} Maximum value for barrier parameter. $\;$ \\
 This option specifies an upper bound on the
barrier parameter in the adaptive mu selection
mode.  If this option is set, it overwrites the
effect of mu\_max\_fact. (Only used if option
``mu\_strategy'' is chosen as ``adaptive''.) The valid range for this real option is 
$0 <  {\tt mu\_max } <  {\tt +inf}$
and its default value is $100000$.


\paragraph{mu\_min:} Minimum value for barrier parameter. $\;$ \\
 This option specifies the lower bound on the
barrier parameter in the adaptive mu selection
mode. By default, it is set to
min(``tol'', ``compl\_inf\_tol'')/(``barrier\_tol\_fact-
or''+1), which should be a reasonable value. (Only
used if option ``mu\_strategy'' is chosen as
``adaptive''.) The valid range for this real option is 
$0 <  {\tt mu\_min } <  {\tt +inf}$
and its default value is $1 \cdot 10^{-09}$.

\paragraph{barrier\_tol\_factor:} Factor for mu in barrier stop test. $\;$ \\
 The convergence tolerance for each barrier
problem in the monotone mode is the value of the
barrier parameter times ``barrier\_tol\_factor''.
This option is also used in the adaptive mu
strategy during the monotone mode. (This is
$\kappa_\varepsilon$ in the implementation paper). The valid range for this real option is 
$0 <  {\tt barrier\_tol\_factor } <  {\tt +inf}$
and its default value is $10$.

\paragraph{mu\_linear\_decrease\_factor:} Determines linear decrease rate of barrier parameter. $\;$ \\
 For the Fiacco-McCormick update procedure the new
barrier parameter mu is obtained by taking the
minimum of mu$\cdot$``mu\_linear\_decrease\_factor'' and
mu$^\textrm{``superlinear\_decrease\_power''}$.  (This is
$\kappa_\mu$ in the implementation paper.) This option
is also used in the adaptive mu strategy during
the monotone mode. The valid range for this real option is 
$0 <  {\tt mu\_linear\_decrease\_factor } <  1$
and its default value is $0.2$.


\paragraph{mu\_superlinear\_decrease\_power:} Determines superlinear decrease rate of barrier parameter. $\;$ \\
 For the Fiacco-McCormick update procedure the new
barrier parameter mu is obtained by taking the
minimum of mu$\cdot$``mu\_linear\_decrease\_factor'' and
mu$^\textrm{``superlinear\_decrease\_power''}$.  (This is
$\theta_\mu$ in the implementation paper.) This option
is also used in the adaptive mu strategy during
the monotone mode. The valid range for this real option is 
$1 <  {\tt mu\_superlinear\_decrease\_power } <  2$
and its default value is $1.5$.

\subsubsection{Barrier parameter update (expert options)}

\paragraph{mu\_allow\_fast\_monotone\_decrease:} Allow skipping of barrier problem if barrier test is already met. $\;$ \\
 If set to ``no'', the algorithm enforces at least
one iteration per barrier problem, even if the
barrier test is already met for the updated
barrier parameter.
The default value for this string option is ``yes''.
\\ 
Possible values:
\begin{itemize}
   \item no: Take at least one iteration per barrier problem
   \item yes: Allow fast decrease of mu if barrier test it met
\end{itemize}

\paragraph{adaptive\_mu\_globalization:} Globalization strategy for the adaptive mu selection mode. $\;$ \\
 To achieve global convergence of the adaptive
version, the algorithm has to switch to the
monotone mode (Fiacco-McCormick approach) when
convergence does not seem to appear.  This option
sets the criterion used to decide when to do this
switch. (Only used if option ``mu\_strategy'' is
chosen as ``adaptive''.)
The default value for this string option is ``obj-constr-filter''.
\\ 
Possible values:
\begin{itemize}
   \item kkt-error: nonmonotone decrease of kkt-error
   \item obj-constr-filter: 2-dim filter for objective and constraint
violation
   \item never-monotone-mode: disables globalization
\end{itemize}

\paragraph{adaptive\_mu\_kkterror\_red\_iters:} Maximum number of iterations requiring sufficient progress. $\;$ \\
 For the ``kkt-error'' based globalization strategy,
the progress made in at most ``adaptive\_mu\_kkterror\_red\_iters'' iterations must be sufficient.
If this number of iterations is exceeded, the
globalization strategy switches to the monotone
mode. The valid range for this integer option is
$0 \le {\tt adaptive\_mu\_kkterror\_red\_iters } <  {\tt +inf}$
and its default value is $4$.


\paragraph{adaptive\_mu\_kkterror\_red\_fact:} Sufficient decrease factor for ``kkt-error'' globalization strategy. $\;$ \\
 For the ``kkt-error'' based globalization strategy,
the error must decrease by this factor to be
deemed sufficient decrease. The valid range for this real option is 
$0 <  {\tt adaptive\_mu\_kkterror\_red\_fact } <  1$
and its default value is $0.9999$.


\paragraph{filter\_margin\_fact:} Factor determining width of margin for obj-constr-filter adaptive globalization strategy. $\;$ \\
 When using the adaptive globalization strategy,
``obj-constr-filter'', sufficient progress for a
filter entry is defined as follows: (new obj) <
(filter obj) - filter\_margin\_fact$\cdot$ (new
constr-viol) OR (new constr-viol) $<$ (filter
constr-viol) - filter\_margin\_fact$\cdot$ (new
constr-viol).  For the description of the
``kkt-error-filter'' option see
``filter\_max\_margin''. The valid range for this real option is 
$0 <  {\tt filter\_margin\_fact } <  1$
and its default value is $1 \cdot 10^{-05}$.


\paragraph{filter\_max\_margin:} Maximum width of margin in obj-constr-filter adaptive globalization strategy. $\;$ \\
 The valid range for this real option is 
$0 <  {\tt filter\_max\_margin } <  {\tt +inf}$
and its default value is $1$.


\paragraph{adaptive\_mu\_restore\_previous\_iterate:} Indicates if the previous iterate should be restored if the monotone mode is entered. $\;$ \\
 When the globalization strategy for the adaptive
barrier algorithm switches to the monotone mode,
it can either start from the most recent iterate
(no), or from the last iterate that was accepted
(yes).
The default value for this string option is ``no''.
\\ 
Possible values:
\begin{itemize}
   \item no: don't restore accepted iterate
   \item yes: restore accepted iterate
\end{itemize}

\paragraph{adaptive\_mu\_monotone\_init\_factor:} Determines the initial value of the barrier parameter when switching to the monotone mode. $\;$ \\
 When the globalization strategy for the adaptive
barrier algorithm switches to the monotone mode
and the option fixed\_mu\_oracle is chosen as
``average\_compl'', the barrier parameter is set to
the current average complementarity times the
value of ``adaptive\_mu\_monotone\_init\_factor''. The default value for this option is $0.8$ and its valid range is $0 <  {\tt adaptive\_mu\_monotone\_init\_factor } <  {\tt +inf}$.


\paragraph{adaptive\_mu\_kkt\_norm\_type:} Norm used for the KKT error in the adaptive mu globalization strategies. $\;$ \\
 When computing the KKT error for the
globalization strategies, the norm to be used is
specified with this option. Note, this options is
also used in the QualityFunctionMuOracle.
The default value for this string option is ``2-norm-squared''.
\\ 
Possible values:
\begin{itemize}
   \item 1-norm: use the 1-norm (abs sum)
   \item 2-norm-squared: use the 2-norm squared (sum of squares)
   \item max-norm: use the infinity norm (max)
   \item 2-norm: use 2-norm
\end{itemize}

\paragraph{tau\_min:} Lower bound on fraction-to-the-boundary parameter tau. $\;$ \\
 (This is $\tau_{\min}$ in the implementation paper.)  This
option is also used in the adaptive mu strategy
during the monotone mode. The valid range for this real option is 
$0 <  {\tt tau\_min } <  1$
and its default value is $0.99$.


\paragraph{sigma\_max:} Maximum value of the centering parameter. $\;$ \\
 This is the upper bound for the centering
parameter chosen by the quality function based
barrier parameter update. (Only used if option
``mu\_oracle'' is set to ``quality-function''.) The valid range for this real option is 
$0 <  {\tt sigma\_max } <  {\tt +inf}$
and its default value is $100$.


\paragraph{sigma\_min:} Minimum value of the centering parameter. $\;$ \\
 This is the lower bound for the centering
parameter chosen by the quality function based
barrier parameter update. (Only used if option
``mu\_oracle'' is set to ``quality-function''.) The valid range for this real option is 
$0 \le {\tt sigma\_min } <  {\tt +inf}$
and its default value is $1 \cdot 10^{-06}$.


\paragraph{quality\_function\_norm\_type:} Norm used for components of the quality function. $\;$ \\
 (Only used if option ``mu\_oracle'' is set to
``quality-function''.)
The default value for this string option is ``2-norm-squared''.
\\ 
Possible values:
\begin{itemize}
   \item 1-norm: use the 1-norm (abs sum)
   \item 2-norm-squared: use the 2-norm squared (sum of squares)
   \item max-norm: use the infinity norm (max)
   \item 2-norm: use 2-norm
\end{itemize}

\paragraph{quality\_function\_centrality:} The penalty term for centrality that is included in quality function. $\;$ \\
 This determines whether a term is added to the
quality function to penalize deviation from
centrality with respect to complementarity.  The
complementarity measure here is the xi in the
Loqo update rule. (Only used if option
``mu\_oracle'' is set to ``quality-function''.)
The default value for this string option is ``none''.
\\ 
Possible values:
\begin{itemize}
   \item none: no penalty term is added
   \item log: complementarity $\cdot$ the log of the centrality
measure
   \item reciprocal: complementarity $\cdot$ the reciprocal of the
centrality measure
   \item cubed-reciprocal: complementarity $\cdot$ the reciprocal of the
centrality measure cubed
\end{itemize}

\paragraph{quality\_function\_balancing\_term:} The balancing term included in the quality function for centrality. $\;$ \\
 This determines whether a term is added to the
quality function that penalizes situations where
the complementarity is much smaller than dual and
primal infeasibilities. (Only used if option
``mu\_oracle'' is set to ``quality-function''.)
The default value for this string option is ``none''.
\\ 
Possible values:
\begin{itemize}
   \item none: no balancing term is added
   \item cubic: $\max(0,\max(\textrm{dual\_inf},\textrm{primal\_inf})-\textrm{compl})^3$
\end{itemize}

\paragraph{quality\_function\_max\_section\_steps:} Maximum number of search steps during direct search procedure determining the optimal centering parameter. $\;$ \\
 The golden section search is performed for the
quality function based mu oracle. The valid range for this integer option is
$0 \le {\tt quality\_function\_max\_section\_steps } <  {\tt +inf}$
and its default value is $8$.
(Only used if
option ``mu\_oracle'' is set to ``quality-function''.)


\paragraph{quality\_function\_section\_sigma\_tol:} Tolerance for the section search procedure determining the optimal centering parameter (in sigma space). $\;$ \\
 The golden section search is performed for the
quality function based mu oracle. (Only used if
option ``mu\_oracle'' is set to ``quality-function''.) The valid range for this real option is 
$0 \le {\tt quality\_function\_section\_sigma\_tol } <  1$
and its default value is $0.01$.


\paragraph{quality\_function\_section\_qf\_tol:} Tolerance for the golden section search procedure determining the optimal centering parameter (in the function value space). $\;$ \\
 The golden section search is performed for the
quality function based mu oracle. (Only used if
option ``mu\_oracle'' is set to ``quality-function''.) The valid range for this real option is 
$0 \le {\tt quality\_function\_section\_qf\_tol } <  1$
and its default value is $0$.


\subsubsection{Warm start}

\paragraph{warm\_start\_init\_point:} Warm-start for initial point $\;$ \\
 Indicates whether this optimization should use a warm start initialization, where values of dual variables are given by GAMS (You can set marginal values for variables and equations in your GAMS model to set the starting point for the dual variables.)
For the primal values, Ipopt uses the starting point that is given by GAMS (You can set level values for variables (and equations) in your GAMS model to set the starting point for the primal variables.)
The default value for this string option is ``no''.
\\ 
Possible values:
\begin{itemize}
   \item no: do not use the warm start initialization
   \item yes: use the warm start initialization
\end{itemize}

\paragraph{warm\_start\_bound\_push:} same as bound\_push for the regular initializer. $\;$ \\
 The valid range for this real option is 
$0 <  {\tt warm\_start\_bound\_push } <  {\tt +inf}$
and its default value is $0.001$.


\paragraph{warm\_start\_bound\_frac:} same as bound\_frac for the regular initializer. $\;$ \\
 The valid range for this real option is 
$0 <  {\tt warm\_start\_bound\_frac } \le 0.5$
and its default value is $0.001$.


\paragraph{warm\_start\_slack\_bound\_frac:} same as slack\_bound\_frac for the regular initializer. $\;$ \\
 The valid range for this real option is 
$0 <  {\tt warm\_start\_slack\_bound\_frac } \le 0.5$
and its default value is $0.001$.


\paragraph{warm\_start\_slack\_bound\_push:} same as slack\_bound\_push for the regular initializer. $\;$ \\
 The valid range for this real option is 
$0 <  {\tt warm\_start\_slack\_bound\_push } <  {\tt +inf}$
and its default value is $0.001$.


\paragraph{warm\_start\_mult\_bound\_push:} same as mult\_bound\_push for the regular initializer. $\;$ \\
 The valid range for this real option is 
$0 <  {\tt warm\_start\_mult\_bound\_push } <  {\tt +inf}$
and its default value is $0.001$.


\paragraph{warm\_start\_mult\_init\_max:} Maximum initial value for the equality multipliers. $\;$ \\
 The valid range for this real option is 
${\tt -inf} <  {\tt warm\_start\_mult\_init\_max } <  {\tt +inf}$
and its default value is $1 \cdot 10^{+06}$.

\subsubsection{Multiplier updates}

\paragraph{alpha\_for\_y:} Method to determine the step size for constraint multipliers. $\;$ \\
 This option determines how the step size
(alpha\_y) will be calculated when updating the
constraint multipliers.
The default value for this string option is ``primal''.
\\ 
Possible values:
\begin{itemize}
   \item primal: use primal step size
   \item bound\_mult: use step size for the bound multipliers (good
for LPs)
   \item min: use the min of primal and bound multipliers
   \item max: use the max of primal and bound multipliers
   \item full: take a full step of size one
   \item min\_dual\_infeas: choose step size minimizing new dual
infeasibility
   \item safe\_min\_dual\_infeas: like ``min\_dual\_infeas'', but safeguarded by
``min'' and ``max''
\end{itemize}

\paragraph{alpha\_for\_y\_tol:} Tolerance for switching to full equality multiplier steps. $\;$ \\
 This is only relevant if ``alpha\_for\_y'' is
chosen ``primal-and-full'' or ``dual-and-full''.  The
step size for the equality constraint multipliers
is taken to be one if the max-norm of the primal
step is less than this tolerance. The valid range for this real option is 
$0 \le {\tt alpha\_for\_y\_tol } <  {\tt +inf}$
and its default value is $10$.

\paragraph{recalc\_y:} Tells the algorithm to recalculate the equality and inequality multipliers as least square estimates. $\;$ \\
 This asks Ipopt to recompute the
multipliers, whenever the current infeasibility
is less than recalc\_y\_feas\_tol. Choosing yes
might be helpful in the quasi-Newton option. 
However, each recalculation requires an extra
factorization of the linear system.  If a limited
memory quasi-Newton option is chosen, this is
used by default.
The default value for this string option is ``no''.
\\ 
Possible values:
\begin{itemize}
   \item no: use the Newton step to update the multipliers
   \item yes: use least-square multiplier estimates
\end{itemize}

\paragraph{recalc\_y\_feas\_tol:} Feasibility threshold for recomputation of multipliers. $\;$ \\
 If recalc\_y is chosen and the current
infeasibility is less than this value, then the
multipliers are recomputed. The valid range for this real option is 
$0 <  {\tt recalc\_y\_feas\_tol } <  {\tt +inf}$
and its default value is $1 \cdot 10^{-06}$.

\subsubsection{Line search}

\paragraph{max\_soc:} Maximum number of second order correction trial steps at each iteration. $\;$ \\
 Choosing 0 disables the second order corrections.
(This is $p^{\max}$ of Step A-5.9 of Algorithm A in
the implementation paper.) The valid range for this integer option is
$0 \le {\tt max\_soc } <  {\tt +inf}$
and its default value is $4$.


\paragraph{watchdog\_shortened\_iter\_trigger:} Number of shortened iterations that trigger the watchdog. $\;$ \\
 If the number of successive iterations in which
the backtracking line search did not accept the
first trial point exceeds this number, the
watchdog procedure is activated.  Choosing 0
here disables the watchdog procedure. The valid range for this integer option is
$0 \le {\tt watchdog\_shortened\_iter\_trigger } <  {\tt +inf}$
and its default value is $10$.


\paragraph{watchdog\_trial\_iter\_max:} Maximum number of watchdog iterations. $\;$ \\
 This option determines the number of trial
iterations allowed before the watchdog procedure
is aborted and the algorithm returns to the
stored point. The valid range for this integer option is
$1 \le {\tt watchdog\_trial\_iter\_max } <  {\tt +inf}$
and its default value is $3$.

\paragraph{corrector\_type:} The type of corrector steps that should be taken (experimental!). $\;$ \\
 If ``mu\_strategy'' is ``adaptive'', this option
determines what kind of corrector steps should be
tried.
The default value for this string option is ``none''.
\\ 
Possible values:
\begin{itemize}
   \item none: no corrector
   \item affine: corrector step towards mu=0
   \item primal-dual: corrector step towards current mu
\end{itemize}

\subsubsection{Line search (expert options)}

\paragraph{alpha\_red\_factor:} Fractional reduction of the trial step size in the backtracking line search. $\;$ \\
 At every step of the backtracking line search,
the trial step size is reduced by this factor. The valid range for this real option is 
$0 <  {\tt alpha\_red\_factor } <  1$
and its default value is $0.5$.


\paragraph{accept\_every\_trial\_step:} Always accept the first trial step. $\;$ \\
 Setting this option to ``yes'' essentially disables
the line search and makes the algorithm take
aggressive steps, without global convergence
guarantees.
The default value for this string option is ``no''.
\\ 
Possible values:
\begin{itemize}
   \item no: don't arbitrarily accept the full step
   \item yes: always accept the full step
\end{itemize}

\paragraph{tiny\_step\_tol:} Tolerance for detecting numerically insignificant steps. $\;$ \\
 If the search direction in the primal variables
(x and s) is, in relative terms for each
component, less than this value, the algorithm
accepts the full step without line search.  If
this happens repeatedly, the algorithm will
terminate with a corresponding exit message. The
default value is 10 times machine precision. The valid range for this real option is 
$0 \le {\tt tiny\_step\_tol } <  {\tt +inf}$
and its default value is $2.22045 \cdot 10^{-15}$.


\paragraph{tiny\_step\_y\_tol:} Tolerance for quitting because of numerically insignificant steps. $\;$ \\
 If the search direction in the primal variables
(x and s) is, in relative terms for each
component, repeatedly less than tiny\_step\_tol,
and the step in the y variables is smaller than
this threshold, the algorithm will terminate. The valid range for this real option is 
$0 \le {\tt tiny\_step\_y\_tol } <  {\tt +inf}$
and its default value is $0.01$.


\paragraph{theta\_max\_fact:} Determines upper bound for constraint violation in the filter. $\;$ \\
 The algorithmic parameter theta\_max is
determined as theta\_max\_fact times the maximum
of 1 and the constraint violation at initial
point.  Any point with a constraint violation
larger than theta\_max is unacceptable to the
filter (see Eqn. (21) in the implementation paper). The valid range for this real option is 
$0 <  {\tt theta\_max\_fact } <  {\tt +inf}$
and its default value is $10000$.


\paragraph{theta\_min\_fact:} Determines constraint violation threshold in the switching rule. $\;$ \\
 The algorithmic parameter theta\_min is
determined as theta\_min\_fact times the maximum
of 1 and the constraint violation at initial
point.  The switching rules treats an iteration
as an h-type iteration whenever the current
constraint violation is larger than theta\_min
(see paragraph before Eqn. (19) in the implementation
paper). The valid range for this real option is 
$0 <  {\tt theta\_min\_fact } <  {\tt +inf}$
and its default value is $0.0001$.


\paragraph{eta\_phi:} Relaxation factor in the Armijo condition. $\;$ \\
 (See Eqn. (20) in the implementation paper) The valid range for this real option is 
$0 <  {\tt eta\_phi } <  0.5$
and its default value is $1 \cdot 10^{-08}$.


\paragraph{delta:} Multiplier for constraint violation in the switching rule. $\;$ \\
 (See Eqn. (19) in the implementation paper.) The valid range for this real option is 
$0 <  {\tt delta } <  {\tt +inf}$
and its default value is $1$.


\paragraph{s\_phi:} Exponent for linear barrier function model in the switching rule. $\;$ \\
 (See Eqn. (19) in the implementation paper.) The valid range for this real option is 
$1 <  {\tt s\_phi } <  {\tt +inf}$
and its default value is $2.3$.


\paragraph{s\_theta:} Exponent for current constraint violation in the switching rule. $\;$ \\
 (See Eqn. (19) in the implementation paper.) The valid range for this real option is 
$1 <  {\tt s\_theta } <  {\tt +inf}$
and its default value is $1.1$.


\paragraph{gamma\_phi:} Relaxation factor in the filter margin for the barrier function. $\;$ \\
 (See Eqn. (18a) in the implementation paper.) The valid range for this real option is 
$0 <  {\tt gamma\_phi } <  1$
and its default value is $1 \cdot 10^{-08}$.


\paragraph{gamma\_theta:} Relaxation factor in the filter margin for the constraint violation. $\;$ \\
 (See Eqn. (18b) in the implementation paper.) The valid range for this real option is 
$0 <  {\tt gamma\_theta } <  1$
and its default value is $1 \cdot 10^{-05}$.


\paragraph{alpha\_min\_frac:} Safety factor for the minimal step size (before switching to restoration phase). $\;$ \\
 (This is $\gamma_\alpha$ in Eqn. (20) in the
implementation paper.) The default value of this real option is $0.05$ and its
valid range is $0 <  {\tt alpha\_min\_frac } <  1$.


\paragraph{kappa\_soc:} Factor in the sufficient reduction rule for second order correction. $\;$ \\
 This option determines how much a second order
correction step must reduce the constraint
violation so that further correction steps are
attempted.  (See Step A-5.9 of Algorithm A in
the implementation paper.) The valid range for this real option is 
$0 <  {\tt kappa\_soc } <  {\tt +inf}$
and its default value is $0.99$.


\paragraph{obj\_max\_inc:} Determines the upper bound on the acceptable increase of barrier objective function. $\;$ \\
 Trial points are rejected if they lead to an
increase in the barrier objective function by
more than obj\_max\_inc orders of magnitude. The valid range for this real option is 
$1 <  {\tt obj\_max\_inc } <  {\tt +inf}$
and its default value is $5$.


\paragraph{max\_filter\_resets:} Maximal allowed number of filter resets $\;$ \\
 A positive number enables a heuristic that resets
the filter, whenever in more than
``filter\_reset\_trigger'' successive iterations
the last rejected trial steps size was rejected
because of the filter.  This option determine the
maximal number of resets that are allowed to take
place. The valid range for this integer option is
$0 \le {\tt max\_filter\_resets } <  {\tt +inf}$
and its default value is $5$.


\paragraph{filter\_reset\_trigger:} Number of iterations that trigger the filter reset. $\;$ \\
 If the filter reset heuristic is active and the
number of successive iterations in which the last
rejected trial step size was rejected because of
the filter, the filter is reset. The valid range for this integer option is
$1 \le {\tt filter\_reset\_trigger } <  {\tt +inf}$
and its default value is $5$.


% \paragraph{skip\_corr\_if\_neg\_curv:} Skip the corrector step in negative curvature iteration (unsupported!). $\;$ \\
%  The corrector step is not tried if negative
% curvature has been encountered during the
% computation of the search direction in the
% current iteration. This option is only used if
% ``mu\_strategy'' is ``adaptive''.
% The default value for this string option is ``yes''.
% \\ 
% Possible values:
% \begin{itemize}
%    \item no: don't skip
%    \item yes: skip
% \end{itemize}

% \paragraph{skip\_corr\_in\_monotone\_mode:} Skip the corrector step during monotone barrier parameter mode (unsupported!). $\;$ \\
%  The corrector step is not tried if the algorithm
% is currently in the monotone mode (see also
% option ``barrier\_strategy'').This option is only
% used if ``mu\_strategy'' is ``adaptive''.
% The default value for this string option is ``yes''.
% \\ 
% Possible values:
% \begin{itemize}
%    \item no: don't skip
%    \item yes: skip
% \end{itemize}

% \paragraph{corrector\_compl\_avrg\_red\_fact:} Complementarity tolerance factor for accepting corrector step (unsupported!). $\;$ \\
%  This option determines the factor by which
% complementarity is allowed to increase for a
% corrector step to be accepted. The valid range for this real option is 
% $0 <  {\tt corrector\_compl\_avrg\_red\_fact } <  {\tt +inf}$
% and its default value is $1$.


\paragraph{kappa\_sigma:} Factor limiting the deviation of dual variables from primal estimates. $\;$ \\
 If the dual variables deviate from their primal
estimates, a correction is performed. (See Eqn.
(16) in the the implementation paper.) Setting the
value to less than 1 disables the correction. The valid range for this real option is 
$0 <  {\tt kappa\_sigma } <  {\tt +inf}$
and its default value is $1 \cdot 10^{+10}$.


\paragraph{slack\_move:} Correction size for very small slacks. $\;$ \\
 Due to numerical issues or the lack of an
interior, the slack variables might become very
small.  If a slack becomes very small compared to
machine precision, the corresponding bound is
moved slightly.  This parameter determines how
large the move should be.  Its default value is
mach\_eps$^{3/4}$.  (See also end of Section 3.5
in the implementation paper - but actual
the implementation might be somewhat different.) The valid range for this real option is 
$0 \le {\tt slack\_move } <  {\tt +inf}$
and its default value is $1.81899 \cdot 10^{-12}$.

\subsubsection{Restoration phase}

\paragraph{expect\_infeasible\_problem:} Enable heuristics to quickly detect an infeasible problem. $\;$ \\
 This options is meant to activate heuristics that
may speed up the infeasibility determination if
you expect that there is a good chance for the
problem to be infeasible.  In the filter line
search procedure, the restoration phase is called
more quickly than usually, and more reduction in
the constraint violation is enforced before the
restoration phase is left. If the problem is
square, this option is enabled automatically.
The default value for this string option is ``no''.
\\ 
Possible values:
\begin{itemize}
   \item no: the problem probably be feasible
   \item yes: the problem has a good chance to be infeasible
\end{itemize}

\paragraph{expect\_infeasible\_problem\_ctol:} Threshold for disabling ``expect\_infeasible\_problem'' option. $\;$ \\
 If the constraint violation becomes smaller than
this threshold, the ``expect\_infeasible\_problem''
heuristics in the filter line search are
disabled. If the problem is square, this options
is set to 0. The valid range for this real option is 
$0 \le {\tt expect\_infeasible\_problem\_ctol } <  {\tt +inf}$
and its default value is $0.001$.


\paragraph{start\_with\_resto:} Tells algorithm to switch to restoration phase in first iteration. $\;$ \\
 Setting this option to ``yes'' forces the algorithm
to switch to the feasibility restoration phase in
the first iteration. If the initial point is
feasible, the algorithm will abort with a failure.
The default value for this string option is ``no''.
\\ 
Possible values:
\begin{itemize}
   \item no: don't force start in restoration phase
   \item yes: force start in restoration phase
\end{itemize}

\paragraph{soft\_resto\_pderror\_reduction\_factor:} Required reduction in primal-dual error in the soft restoration phase. $\;$ \\
 The soft restoration phase attempts to reduce the
primal-dual error with regular steps. If the
damped primal-dual step (damped only to satisfy
the fraction-to-the-boundary rule) is not
decreasing the primal-dual error by at least this
factor, then the regular restoration phase is
called. Choosing 0 here disables the soft
restoration phase. The valid range for this real option is 
$0 \le {\tt soft\_resto\_pderror\_reduction\_factor } <  {\tt +inf}$
and its default value is $0.9999$.


\paragraph{required\_infeasibility\_reduction:} Required reduction of infeasibility before leaving restoration phase. $\;$ \\
 The restoration phase algorithm is performed,
until a point is found that is acceptable to the
filter and the infeasibility has been reduced by
at least the fraction given by this option. The valid range for this real option is 
$0 \le {\tt required\_infeasibility\_reduction } <  1$
and its default value is $0.9$.


\paragraph{max\_soft\_resto\_iters:} Maximum number of iterations performed successively in soft restoration phase. $\;$ \\
 If the soft restoration phase is performed for
more than so many iterations in a row, the regular
restoration phase is called. The valid range for this integer option is
$0 \le {\tt max\_soft\_resto\_iters } <  {\tt +inf}$
and its default value is $10$.


\paragraph{max\_resto\_iter:} Maximum number of successive iterations in restoration phase. $\;$ \\
 The algorithm terminates with an error message if
the number of iterations successively taken in
the restoration phase exceeds this number. The valid range for this integer option is
$0 \le {\tt max\_resto\_iter } <  {\tt +inf}$
and its default value is $3000000$.


\paragraph{bound\_mult\_reset\_threshold:} Threshold for resetting bound multipliers after the restoration phase. $\;$ \\
 After returning from the restoration phase, the
bound multipliers are updated with a Newton step
for complementarity.  Here, the change in the
primal variables during the entire restoration
phase is taken to be the corresponding primal
Newton step. However, if after the update the
largest bound multiplier exceeds the threshold
specified by this option, the multipliers are all
reset to 1. The valid range for this real option is 
$0 \le {\tt bound\_mult\_reset\_threshold } <  {\tt +inf}$
and its default value is $1000$.


\paragraph{constr\_mult\_reset\_threshold:} Threshold for resetting equality and inequality multipliers after restoration phase. $\;$ \\
 After returning from the restoration phase, the
constraint multipliers are recomputed by a least
square estimate.  This option triggers when those
least-square estimates should be ignored. The valid range for this real option is 
$0 \le {\tt constr\_mult\_reset\_threshold } <  {\tt +inf}$
and its default value is $0$.


\paragraph{evaluate\_orig\_obj\_at\_resto\_trial:} Determines if the original objective function should be evaluated at restoration phase trial points. $\;$ \\
 Setting this option to ``yes'' makes the
restoration phase algorithm evaluate the
objective function of the original problem at
every trial point encountered during the
restoration phase, even if this value is not
required.  In this way, it is guaranteed that the
original objective function can be evaluated
without error at all accepted iterates; otherwise
the algorithm might fail at a point where the
restoration phase accepts an iterate that is good
for the restoration phase problem, but not the
original problem.  On the other hand, if the
evaluation of the original objective is
expensive, this might be costly.
The default value for this string option is ``yes''.
\\ 
Possible values:
\begin{itemize}
   \item no: skip evaluation
   \item yes: evaluate at every trial point
\end{itemize}

\subsubsection{Linear solver}

\paragraph{linear\_solver:} Linear solver used for step computations. $\;$ \\
 Determines which linear algebra package is to be used for the solution of the augmented linear system (for obtaining the search directions).
Note, you need to provide an extra shared library to use MA27, MA57, or PARDISO, see \hyperlink{ipoptlinearsolver}{Section \ref{ipoptlinearsolver}}.
The default value for this string option is ``mumps''.
\\
Possible values:
\begin{itemize}
   \item ma27: use the Harwell routine MA27
   \item ma57: use the Harwell routine MA57
   \item pardiso: use the Pardiso package
%    \item wsmp: use WSMP package
%    \item taucs: use TAUCS package (not yet working)
   \item mumps: use MUMPS package
%    \item custom: use custom linear solver
\end{itemize}

\paragraph{hsl\_library:} Path and filename of HSL library for dynamic load. $\;$ \\
Specify the path to a library that contains HSL routines and can be load via dynamic linking, see also \hyperlink{ipoptlinearsolver}{Section \ref{ipoptlinearsolver}}.

\paragraph{pardiso\_library:} Path and filename of PARDISO library for dynamic load. $\;$ \\
Specify the path to a PARDISO library that and can be load via dynamic linking, see also \hyperlink{ipoptlinearsolver}{Section \ref{ipoptlinearsolver}}.

\paragraph{linear\_system\_scaling:} Method for scaling the linear system. $\;$ \\
 Determines the method used to compute symmetric scaling factors for the augmented system (see also the ``linear\_scaling\_on\_demand'' option).
This scaling is independent of the NLP problem scaling.
By default, MC19 is only used if MA27 or MA57 are selected as linear solvers.
% This option is only available if Ipopt has been compiled with MC19.
The default value for this string option is ``mc19''.
\\
Possible values:
\begin{itemize}
   \item none: no scaling will be performed
   \item mc19: use the Harwell routine MC19
\end{itemize}

\paragraph{linear\_scaling\_on\_demand:} Flag indicating that linear scaling is only done if it seems required. $\;$ \\
 This option is only important if a linear scaling method (e.g., mc19) is used.
If you choose ``no'', then the scaling factors are computed for every linear system from the start.
This can be quite expensive.
Choosing ``yes'' means that the algorithm will start the scaling method only when the solutions to the linear system seem not good, and then use it until the end.
The default value for this string option is ``yes''.
\\
Possible values:
\begin{itemize}
   \item no: Always scale the linear system.
   \item yes: Start using linear system scaling if solutions
seem not good.
\end{itemize}

\paragraph{fast\_step\_computation:} Indicates if the linear system should be solved quickly. $\;$ \\
 If set to yes, the algorithm assumes that the
linear system that is solved to obtain the search
direction, is solved sufficiently well. In that
case, no residuals are computed, and the
computation of the search direction is a little
faster.
The default value for this string option is ``no''.
\\ 
Possible values:
\begin{itemize}
   \item no: Verify solution of linear system by computing
residuals.
   \item yes: Trust that linear systems are solved well.
\end{itemize}

\paragraph{max\_refinement\_steps:} Maximum number of iterative refinement steps per linear system solve. $\;$ \\
 Iterative refinement (on the full unsymmetric
system) is performed for each right hand side. 
This option determines the maximum number of
iterative refinement steps. The valid range for this integer option is
$0 \le {\tt max\_refinement\_steps } <  {\tt +inf}$
and its default value is $10$.


\paragraph{min\_refinement\_steps:} Minimum number of iterative refinement steps per linear system solve. $\;$ \\
 Iterative refinement (on the full unsymmetric
system) is performed for each right hand side. 
This option determines the minimum number of
iterative refinements (i.e. at least
``min\_refinement\_steps'' iterative refinement
steps are enforced per right hand side.) The valid range for this integer option is
$0 \le {\tt min\_refinement\_steps } <  {\tt +inf}$
and its default value is $1$.

\paragraph{residual\_ratio\_max:} Iterative refinement tolerance $\;$ \\
 Iterative refinement is performed until the
residual test ratio is less than this tolerance
(or until the limit ``max\_refinement\_steps'' is hit). The valid range for this real option is 
$0 <  {\tt residual\_ratio\_max } <  {\tt +inf}$
and its default value is $1 \cdot 10^{-10}$.


\paragraph{residual\_ratio\_singular:} Threshold for declaring linear system singular after failed iterative refinement. $\;$ \\
 If the residual test ratio is larger than this
value after failed iterative refinement, the
algorithm pretends that the linear system is
singular. The valid range for this real option is 
$0 <  {\tt residual\_ratio\_singular } <  {\tt +inf}$
and its default value is $1 \cdot 10^{-05}$.


\paragraph{residual\_improvement\_factor:} Minimal required reduction of residual test ratio in iterative refinement. $\;$ \\
 If the improvement of the residual test ratio
made by one iterative refinement step is not
better than this factor, iterative refinement is
aborted. The valid range for this real option is 
$0 <  {\tt residual\_improvement\_factor } <  {\tt +inf}$
and its default value is $1$.


\subsubsection{MUMPS Linear Solver}

\paragraph{mumps\_pivtol:} Pivot tolerance for the linear solver MUMPS. \\
A smaller number pivots for sparsity, a larger number pivots for stability.
The valid range for this real option is
$0 \le {\tt mumps\_pivtol } < {\tt 1}$
and its default value is $1e-6$.

\paragraph{mumps\_pivtolmax:} Maximum pivot tolerance for the linear solver MUMPS. \\
Ipopt may increase pivtol as high as pivtolmax to get a more accurate solution to the linear system.
The valid range for this real option is
$0 \le {\tt mumps\_pivtolmax } < {\tt 1}$
and its default value is $0.1$.

\paragraph{mumps\_mem\_percent:} Percentage increase in the estimated working space for MUMPS. \\
In MUMPS when significant extra fill-in is caused by numerical pivoting, larger values of mumps\_mem\_percent may help use the workspace more efficiently.
The valid range for this integer option is
$0 \le {\tt mumps\_mem\_percent } < {\tt +inf}$
and its default value is $1000$.

\paragraph{mumps\_permuting\_scaling:} Controls permuting and scaling in MUMPS $\;$ \\
 This is ICTL(6) in MUMPS. The valid range for this integer option is
$0 \le {\tt mumps\_permuting\_scaling } \le 7$
and its default value is $7$.


\paragraph{mumps\_pivot\_order:} Controls pivot order in MUMPS $\;$ \\
 This is ICTL(7) in MUMPS. The valid range for this integer option is
$0 \le {\tt mumps\_pivot\_order } \le 7$
and its default value is $7$.


\paragraph{mumps\_scaling:} Controls scaling in MUMPS $\;$ \\
 This is ICTL(8) in MUMPS. The valid range for this integer option is
$-2 \le {\tt mumps\_scaling } \le 7$
and its default value is $7$.


\paragraph{mumps\_dep\_tol:} Pivot threshold for detection of linearly dependent constraints in MUMPS. $\;$ \\
 When MUMPS is used to determine linearly
dependent constraints, this is determines the
threshold for a pivot to be considered zero. 
This is CNTL(3) in MUMPS. The valid range for this real option is 
${\tt -inf} <  {\tt mumps\_dep\_tol } <  {\tt +inf}$
and its default value is $-1$.

\subsubsection{PARDISO Linear Solver}

\paragraph{pardiso\_matching\_strategy:} Matching strategy to be used by Pardiso $\;$ \\
This is IPAR(13) in Pardiso manual.
The default value for this string option is ``complete+2x2''.
\\
Possible values:
\begin{itemize}
   \item complete: Match complete (IPAR(13)=1)
   \item complete+2x2: Match complete+2x2 (IPAR(13)=2)
   \item constraints: Match constraints (IPAR(13)=3)
\end{itemize}

\paragraph{pardiso\_out\_of\_core\_power:} Enables out-of-core variant of Pardiso $\;$ \\
Setting this option to a positive integer $k$ makes Pardiso work in the out-of-core variant where the factor is split in $2^k$ subdomains.
This is IPARM(50) in the Pardiso manual.
The valid range for this integer option is $0 \le {\tt pardiso\_out\_of\_core\_power } <  {\tt +inf}$ and its default value is $0$.

\subsubsection{MA27 Linear Solver}

\paragraph{ma27\_pivtol:} Pivot tolerance for the linear solver MA27. $\;$ \\
A smaller number pivots for sparsity, a larger number pivots for stability.
The valid range for this real option is $0 <  {\tt ma27\_pivtol } <  1$ and its default value is $10^{-8}$.


\paragraph{ma27\_pivtolmax:} Maximum pivot tolerance for the linear solver MA27. $\;$ \\
Ipopt may increase pivtol as high as pivtolmax to get a more accurate solution to the linear
system.
The valid range for this real option is $0 <  {\tt ma27\_pivtolmax } <  1$ and its default value is $0.0001$.


\paragraph{ma27\_liw\_init\_factor:} Integer workspace memory for MA27. $\;$ \\
The initial integer workspace memory = liw\_init\_factor $*$ memory required by unfactored system.
Ipopt will increase the workspace size by meminc\_factor if required.
The default value for this real option is $5$ and its valid range is $1 \le {\tt ma27\_liw\_init\_factor } <  {\tt +inf}$.


\paragraph{ma27\_la\_init\_factor:} Real workspace memory for MA27. $\;$ \\
The initial real workspace memory = la\_init\_factor $*$ memory required by unfactored system. 
Ipopt will increase the workspace size by meminc\_factor if required.
The valid range for this real option is $1 \le {\tt ma27\_la\_init\_factor } <  {\tt +inf}$ and its default value is $5$.


\paragraph{ma27\_meminc\_factor:} Increment factor for workspace size for MA27. $\;$ \\
If the integer or real workspace is not large enough, Ipopt will increase its size by this factor.
The valid range for this real option is $1 \le {\tt ma27\_meminc\_factor } <  {\tt +inf}$ and its default value is $10$.

\subsubsection{MA57 Linear Solver}

\paragraph{ma57\_pivtol:} Pivot tolerance for the linear solver MA57. $\;$ \\
A smaller number pivots for sparsity, a larger number pivots for stability.
The valid range for this real option is $0 <  {\tt ma57\_pivtol } <  1$ and its default value is $10^{-8}$.


\paragraph{ma57\_pivtolmax:} Maximum pivot tolerance for the linear solver MA57. $\;$ \\
Ipopt may increase pivtol as high as ma57\_pivtolmax to get a more accurate solution to the linear system.
The valid range for this real option is $0 <  {\tt ma57\_pivtolmax } <  1$ and its default value is $0.0001$.


\paragraph{ma57\_pre\_alloc:} Safety factor for work space memory allocation for the linear solver MA57. $\;$ \\
If 1 is chosen, the suggested amount of work space is used.
However, choosing a larger number might avoid reallocation if the suggest values do not suffice.
The valid range for this real option is $1 \le {\tt ma57\_pre\_alloc } <  {\tt +inf}$ and its default value is $3$.



\subsubsection{Hessian perturbation}

\paragraph{max\_hessian\_perturbation:} Maximum value of regularization parameter for handling negative curvature. $\;$ \\
 In order to guarantee that the search directions
are indeed proper descent directions, Ipopt
requires that the inertia of the (augmented)
linear system for the step computation has the
correct number of negative and positive
eigenvalues. The idea is that this guides the
algorithm away from maximizers and makes Ipopt
more likely converge to first order optimal
points that are minimizers. If the inertia is not
correct, a multiple of the identity matrix is
added to the Hessian of the Lagrangian in the
augmented system. This parameter gives the
maximum value of the regularization parameter. If
a regularization of that size is not enough, the
algorithm skips this iteration and goes to the
restoration phase. (This is $\delta_w^{\max}$ in the
implementation paper.) The valid range for this real option is 
$0 <  {\tt max\_hessian\_perturbation } <  {\tt +inf}$
and its default value is $1 \cdot 10^{+20}$.


\paragraph{min\_hessian\_perturbation:} Smallest perturbation of the Hessian block. $\;$ \\
 The size of the perturbation of the Hessian block
is never selected smaller than this value, unless
no perturbation is necessary. (This is
$\delta_w^{\min}$ in the implementation paper.) The valid range for this real option is 
$0 \le {\tt min\_hessian\_perturbation } <  {\tt +inf}$
and its default value is $1 \cdot 10^{-20}$.


\paragraph{first\_hessian\_perturbation:} Size of first x-s perturbation tried. $\;$ \\
 The first value tried for the x-s perturbation in
the inertia correction scheme.(This is $\delta_0$
in the implementation paper.) The valid range for this real option is 
$0 <  {\tt first\_hessian\_perturbation } <  {\tt +inf}$
and its default value is $0.0001$.


\paragraph{perturb\_inc\_fact\_first:} Increase factor for x-s perturbation for very first perturbation. $\;$ \\
 The factor by which the perturbation is increased
when a trial value was not sufficient - this
value is used for the computation of the very
first perturbation and allows a different value
for for the first perturbation than that used for
the remaining perturbations. (This is
$\bar\kappa_w^+$ in the implementation paper.) The valid range for this real option is 
$1 <  {\tt perturb\_inc\_fact\_first } <  {\tt +inf}$
and its default value is $100$.


\paragraph{perturb\_inc\_fact:} Increase factor for x-s perturbation. $\;$ \\
 The factor by which the perturbation is increased
when a trial value was not sufficient - this
value is used for the computation of all
perturbations except for the first. (This is
$\kappa_w^+$ in the implementation paper.) The valid range for this real option is 
$1 <  {\tt perturb\_inc\_fact } <  {\tt +inf}$
and its default value is $8$.


\paragraph{perturb\_dec\_fact:} Decrease factor for x-s perturbation. $\;$ \\
 The factor by which the perturbation is decreased
when a trial value is deduced from the size of
the most recent successful perturbation. (This is
$\kappa_w^-$ in the implementation paper.) The valid range for this real option is 
$0 <  {\tt perturb\_dec\_fact } <  1$
and its default value is $0.333333$.


\paragraph{jacobian\_regularization\_value:} Size of the regularization for rank-deficient constraint Jacobians. $\;$ \\
 (This is $\bar\delta_c$ in the implementation
paper.) The valid range for this real option is\\ 
$0 \le {\tt jacobian\_regularization\_value } <  {\tt +inf}$
and its default value is $1 \cdot 10^{-08}$.


\paragraph{jacobian\_regularization\_exponent:} Exponent for mu in the regularization for rank-deficient constraint Jacobians. $\;$ \\
 (This is $\kappa_c$ in the implementation paper.) The default value for this real option is $0.25$
and its valid range is $0 \le {\tt jacobian\_regularization\_exponent } <  {\tt +inf}$.


\paragraph{perturb\_always\_cd:} Active permanent perturbation of constraint linearization. $\;$ \\
 This options makes the delta\_c and delta\_d
perturbation be used for the computation of every
search direction.  Usually, it is only used when
the iteration matrix is singular.
The default value for this string option is ``no''.
\\ 
Possible values:
\begin{itemize}
   \item no: perturbation only used when required
   \item yes: always use perturbation
\end{itemize}

\subsubsection{Hessian approximation}

\paragraph{hessian\_approximation:} Indicates what Hessian information is to be used. $\;$ \\
 This determines which kind of information for the
Hessian of the Lagrangian function is used by the
algorithm.
The default value for this string option is to use ``exact'' if the GAMS system is able to provide a hessian, and ``limited-memory'' otherwise (a warning is issued in this case).
\\ 
Possible values:
\begin{itemize}
   \item exact: Use second derivatives provided by the NLP.
   \item limited-memory: Perform a limited-memory quasi-Newton
approximation
\end{itemize}

\paragraph{hessian\_approximation\_space:} Indicates in which subspace the Hessian information is to be approximated. \\
The default value for this string option is ``nonlinear-variables''.
\\ 
Possible values:
\begin{itemize}
   \item nonlinear-variables: only in space of nonlinear variables.
   \item all-variables: in space of all variables (without slacks)
\end{itemize}

\paragraph{limited\_memory\_max\_history:} Maximum size of the history for the limited quasi-Newton Hessian approximation. $\;$ \\
 This option determines the number of most recent
iterations that are taken into account for the
limited-memory quasi-Newton approximation. The valid range for this integer option is
$0 \le {\tt limited\_memory\_max\_history } <  {\tt +inf}$
and its default value is $6$.


\paragraph{limited\_memory\_update\_type:} Quasi-Newton update formula for the limited memory approximation. $\;$ \\
 Determines which update formula is to be used for
the limited-memory quasi-Newton approximation.
The default value for this string option is ``bfgs''.
\\ 
Possible values:
\begin{itemize}
   \item bfgs: BFGS update (with skipping)
   \item sr1: SR1 (not working well)
\end{itemize}

\paragraph{limited\_memory\_initialization:} Initialization strategy for the limited memory quasi-Newton approximation. $\;$ \\
 Determines how the diagonal Matrix B\_0 as the
first term in the limited memory approximation
should be computed.
The default value for this string option is ``scalar1''.
\\ 
Possible values:
\begin{itemize}
   \item scalar1: sigma = $s^Ty/s^Ts$
   \item scalar2: sigma = $y^Ty/s^Ty$
   \item constant: sigma = limited\_memory\_init\_val
\end{itemize}

\paragraph{limited\_memory\_init\_val:} Value for B0 in low-rank update. $\;$ \\
 The starting matrix in the low rank update, B0,
is chosen to be this multiple of the identity in
the first iteration (when no updates have been
performed yet), and remains constant at this
value, if ``limited\_memory\_initialization'' is
``constant''. The valid range for this real option is 
$0 <  {\tt limited\_memory\_init\_val } <  {\tt +inf}$
and its default value is $1$.


\paragraph{limited\_memory\_max\_skipping:} Threshold for successive iterations where update is skipped. $\;$ \\
 If the update is skipped more than this number of
successive iterations, we quasi-Newton
approximation is reset. The valid range for this integer option is
$1 \le {\tt limited\_memory\_max\_skipping } <  {\tt +inf}$
and its default value is $2$.


% \paragraph{derivative\_test:} Enable derivative checker $\;$ \\
%  If this option is enabled, a (slow) derivative
% test will be performed before the optimization. 
% The test is performed at the user provided
% starting point and marks derivative values that
% seem suspicious.
% The default value for this string option is ``none''.
% \\ 
% Possible values:
% \begin{itemize}
%    \item none: do not perform derivative test
%    \item first-order: perform test of first derivatives at starting
% point
%    \item second-order: perform test of first and second derivatives at
% starting point
% \end{itemize}
% 
% \paragraph{derivative\_test\_perturbation:} Size of the finite difference perturbation in derivative test. $\;$ \\
%  This determines the relative perturbation of the
% variable entries. The valid range for this real option is 
% $0 <  {\tt derivative\_test\_perturbation } <  {\tt +inf}$
% and its default value is $1 \cdot 10^{-08}$.
% 
% 
% \paragraph{derivative\_test\_tol:} Threshold for indicating wrong derivative. $\;$ \\
%  If the relative deviation of the estimated
% derivative from the given one is larger than this
% value, the corresponding derivative is marked as
% wrong. The valid range for this real option is 
% $0 <  {\tt derivative\_test\_tol } <  {\tt +inf}$
% and its default value is $0.0001$.
% 
% 
% \paragraph{derivative\_test\_print\_all:} Indicates whether information for all estimated derivatives should be printed. $\;$ \\
%  Determines verbosity of derivative checker.
% The default value for this string option is ``no''.
% \\ 
% Possible values:
% \begin{itemize}
%    \item no: Print only suspect derivatives
%    \item yes: Print all derivatives
% \end{itemize}
% 


\bibliographystyle{plain}
%\bibliography{coinlibd}
%\renewcommand{\bibname}{Ipopt References}
\chapter{\IPOPT and \IPOPTH}
\label{cha:ipopt}

%\minitoc

COIN-OR \IPOPT (\textbf{I}nterior \textbf{P}oint \textbf{Opt}imizer) is an open-source solver for large-scale nonlinear programming.
The code has been written primarily by Andreas W\"achter, who is the COIN-OR project leader for \IPOPT.

\IPOPT implements an interior point line search filter method for nonlinear programming models which functions can be nonconvex, but should be twice continuously differentiable.
For more information on the algorithm we refer to~\cite{NoWaWa08,Waechter2002,WaBi05b,WaBi05a,WaBi2006} and the \IPOPT web site \url{https://projects.coin-or.org/Ipopt}.
Most of the \IPOPT documentation in the section was taken from the \IPOPT manual~\cite{IpoptManual}.



\section{The linear solver in \IPOPT}
\label{sec:ipoptlinearsolver}
\hypertarget{ipoptlinearsolver}{}

The performance and robustness of \IPOPT on larger models heavily relies on the used solver for sparse symmetric indefinite linear systems.

\GAMS/\IPOPT includes the sparse solver \textsc{MUMPS}~\cite{AmestoyDuffKosterLExcellent2001,AmestoyGuermoucheLExcellentPralet2006} (currently the default), cf.~\url{http://graal.ens-lyon.fr/MUMPS} and \textsc{MKL PARDISO}~\cite{SchGa04,SchGa06} (only Linux and Windows).
In the commerically licensed \GAMS/\IPOPTH version, also the Harwell Subroutine Library (HSL) solvers \textsc{MA27}, \textsc{MA57}, \textsc{HSL\_MA86}, and \textsc{HSL\_MA97} are available and MA27 is used by default.

\textsc{MUMPS}, \textsc{MA57}, \textsc{HSL\_MA86}, and \textsc{HSL\_MA97} use \textsc{METIS} for matrix ordering \cite{KaKu99}, cf.~\url{http://glaros.dtc.umn.edu/gkhome/views/metis/index.html} and \url{http://glaros.dtc.umn.edu/gkhome/fetch/sw/metis/manual.pdf}.
\textsc{METIS} is copyrighted by the regents of the University of Minnesota.

\IPOPT and \IPOPTH can exploit parallelization of the linear solver or the linear algebra routines (Blas and Lapack).
The following table summarizes which options are available on which platform.

\begin{tabular}{l|c|cc|cccc}
& \multicolumn{3}{c|}{\IPOPT and \IPOPTH} & \multicolumn{4}{c}{\IPOPTH only} \\
        & Linear Algebra & MUMPS & MKL PARDISO & MA27 & MA57 & HSL MA86 & HSL MA97 \\ \hline
Linux   & parallel & serial & parallel      & serial & serial & parallel & parallel \\
MacOS X & parallel & serial & not available & serial & serial & parallel & parallel  \\
Solaris & serial   & serial & not available & serial & serial & parallel & parallel  \\
Windows & parallel & serial & parallel      & serial & serial & parallel & parallel  \\
\end{tabular}

The linear solver is chosen by the \texttt{linear\_solver} option.
Benchmarks have shown that \textsc{MA57} and \textsc{HSL\_MA97} are often able to outperform \textsc{MA27} on larger instances. Further, \textsc{PARDISO} often allows for performance that is better than \textsc{MUMPS} and similar to the HSL solvers. If \IPOPT fails to solve an instance with \textsc{PARDISO}, it's worth to try changing the options \texttt{pardiso\_order} and \texttt{pardiso\_max\_iterative\_refinement\_steps}.

\section{Usage}

The following statement can be used inside your \GAMS program to specify using \IPOPT
\begin{verbatim}
  Option NLP = IPOPT;     { or LP, RMIP, DNLP, RMINLP, QCP, RMIQCP }
\end{verbatim}

The above statement should appear before the Solve statement.
If \IPOPT was specified as the default solver during \GAMS installation, the above statement is not necessary.

To use \IPOPTH, the statement should be
\begin{verbatim}
  Option NLP = IPOPTH;    { or LP, RMIP, DNLP, RMINLP, QCP, RMIQCP }
\end{verbatim}



\paragraph{Using Harwell Subroutine Library routines with \GAMS/\IPOPT.}

\GAMS/\IPOPT can use the HSL routines \texttt{MA27}, \texttt{MA28}, \texttt{MA57}, \textsc{HSL\_MA77}, \textsc{HSL\_MA86}, \textsc{HSL\_MA97}, \texttt{MC19}, and \textsc{HSL\_MC68} when provided as shared library.
By telling \IPOPT to use one of these routines (see options \texttt{linear\_solver}, \texttt{linear\_system\_scaling}, \texttt{nlp\_scaling\_method}, \texttt{dependency\_detector}), \GAMS/\IPOPT attempts to load the required routines from the library \texttt{libhsl.so} (Unix-Systems), \texttt{libhsl.dylib} (MacOS X), or \texttt{libhsl.dll} (Windows), respectively.

The HSL routines are available at \url{http://www.hsl.rl.ac.uk/ipopt}.
Note that it is your responsibility to ensure that you are entitled to download and use these routines!
% You can build a shared library using the ThirdParty/HSL project at COIN-OR.

\paragraph{Using PARDISO with \GAMS/\IPOPT or \GAMS/\IPOPTH.}
On Mac OS X and Solaris, setting the option \texttt{linear\_solver} to \texttt{pardiso} lets \GAMS/\IPOPT or \GAMS/\IPOPTH try to load the linear solver PARDISO from the library \texttt{libpardiso.so} (Unix) or \texttt{libpardiso.dylib} (MacOS X), respectively.

PARDISO is available as compiled shared library for several platforms at \texttt{http://www.pardiso-project.org}.
Note that it is your responsibility to ensure that you are entitled to download and use this package!

\subsection{Specification of Options}
\label{sub:ipoptoptionspec}

\IPOPT has many options that can be adjusted for the algorithm (see Section \ref{sub:ipoptoptions}).
Options are all identified by a string name, and their values can be of one of three types: Number (real), Integer, or String.
Number options are used for things like tolerances, integer options are used for things like maximum number of iterations, and string options are used for setting algorithm details, like the NLP scaling method.
Options can be set by creating a \texttt{ipopt.opt} file in the directory you are executing \IPOPT.

The \texttt{ipopt.opt} file is read line by line and each line should contain the option name, followed by whitespace, and then the value.
Comments can be included with the \# symbol. Don't forget to ensure you have a newline at the end of the file. For example,
\begin{verbatim}
# This is a comment

# Turn off the NLP scaling
nlp_scaling_method none

# Change the initial barrier parameter
mu_init 1e-2

# Set the max number of iterations
max_iter 500
\end{verbatim}
is a valid \texttt{ipopt.opt} file.

% You can print the documentation for all \IPOPT options by using the option
% \begin{verbatim}
% print_options_documentation yes
% \end{verbatim}
% and running \IPOPT.
% This will output all of the options documentation to the console.

\GAMS/\IPOPT understand currently the following \GAMS parameters: \texttt{reslim} (time limit), \texttt{iterlim} (iteration limit), \texttt{domlim} (domain violation limit).
You can set them either on the command line, e.g. \verb+iterlim=500+, or inside your \GAMS program, e.g. \verb+Option iterlim=500;+.
Further the option \texttt{threads} can be used to control the number of threads used in the linear algebra routines and the linear solver, see also Section~\ref{sec:ipoptlinearsolver}.

\subsection{Warmstarting Ipopt}

As an interior point solver, it is difficult to warm start \IPOPT.
By default, only the level values of the variables are passed as starting point to \IPOPT.
Setting the \IPOPT option \texttt{warm\_start\_init\_point} to \texttt{yes} enables that also dual values for variables and constraints are passed to \IPOPT.

However, the expected behavior that \IPOPT finishes within one iteration if optimal primal and dual values are passed is not reached this way, yet. This is, because \IPOPT by default moves any initial value that is close to a bound into the interior. The amount on how much the initial point is moved can be controlled by various \texttt{bound\_push} and \texttt{bound\_frac} options.
To make \IPOPT accept an optimal primal/dual solution within one iteration, it should be sufficient to set the following options:
\begin{verbatim}
  warm_start_init_point       yes
  warm_start_bound_push       1e-9
  warm_start_bound_frac       1e-9
  warm_start_slack_bound_frac 1e-9
  warm_start_slack_bound_push 1e-9
  warm_start_mult_bound_push  1e-9
\end{verbatim}

\section{Output}

This section describes the standard \IPOPT console output.
The output is designed to provide a quick summary of each iteration as \IPOPT solves the problem.

Before \IPOPT starts to solve the problem, it displays the problem statistics (number of nonzero-elements in the matrices, number of variables, etc.).
Note that if you have fixed variables (both upper and lower bounds are equal), \IPOPT may remove these variables from the problem internally and not include them in the problem statistics.

Following the problem statistics, \IPOPT will begin to solve the problem and you will see output resembling the following,
\begin{verbatim}
iter    objective    inf_pr   inf_du lg(mu)  ||d||  lg(rg) alpha_du alpha_pr  ls
   0  1.6109693e+01 1.12e+01 5.28e-01   0.0 0.00e+00    -  0.00e+00 0.00e+00   0
   1  1.8029749e+01 9.90e-01 6.62e+01   0.1 2.05e+00    -  2.14e-01 1.00e+00f  1
   2  1.8719906e+01 1.25e-02 9.04e+00  -2.2 5.94e-02   2.0 8.04e-01 1.00e+00h  1
\end{verbatim}
and the columns of output are defined as
\begin{description}
\item[iter]
The current iteration count.
This includes regular iterations and iterations while in restoration phase.
If the algorithm is in the restoration phase, the letter \texttt{r} will be appended to the iteration number.
\item[objective]
The unscaled objective value at the current point.
During the restoration phase, this value remains the unscaled objective value for the original problem.
\item[inf\_pr]
The unscaled constraint violation at the current point.
This quantity is the infinity-norm (max) of the (unscaled) constraint violation.
During the restoration phase, this value remains the constraint violation of the original problem at the current point.
The option ``\texttt{inf\_pr\_output}'' can be used to switch to the printing of a different quantity.
During the restoration phase, this value is the primal infeasibility of the original problem at the current point.
\item[inf\_du]
The scaled dual infeasibility at the current point.
This quantity measure the infinity-norm (max) of the internal dual infeasibility \cite[Eq.~(4a)]{WaBi2006}, including inequality constraints reformulated using slack variables and problem scaling.
During the restoration phase, this is the value of the dual infeasibility for the restoration phase problem.
\item[lg(mu)]
$\log_{10}$ of the value of the barrier parameter $\mu$.
\item[$\Vert d\Vert$]
The infinity norm (max) of the primal step (for the original variables $x$ and the internal slack variables $s$).
During the restoration phase, this value includes the values of additional variables, $p$ and $n$ \cite[Eq.~(10)]{WaBi2006}.
\item[lg(rg)]
$\log_{10}$ of the value of the regularization term for the Hessian of the Lagrangian in the augmented system ($\delta_w$ in \cite[Eq.~(26)]{WaBi2006}).
A dash (``\texttt{-}'') indicates that no regularization was done.
\item[alpha\_du]
The stepsize for the dual variables ($\alpha^z_k$ in \cite[Eq.~(14c)]{WaBi2006})..
\item[alpha\_pr]
The stepsize for the primal variables ($\alpha_k$ in \cite[Eq.~(14a)]{WaBi2006}).
The number is usually followed by a character for additional diagnostic information regarding the step acceptance criterion:
 \begin{list}{blub}{\itemsep0pt}
    \item[\texttt{f}] f-type iteration in the filter method w/o second order correction
    \item[\texttt{F}] f-type iteration in the filter method w/ second order correction
    \item[\texttt{h}] h-type iteration in the filter method w/o second order correction
    \item[\texttt{H}] h-type iteration in the filter method w/ second order correction
    \item[\texttt{k}] penalty value unchanged in merit function method w/o second order correction
    \item[\texttt{K}] penalty value unchanged in merit function method w/ second order correction
    \item[\texttt{n}] penalty value updated in merit function method w/o second order correction
    \item[\texttt{N}] penalty value updated in merit function method w/ second order correction
    \item[\texttt{R}] Restoration phase just started
    \item[\texttt{w}] in watchdog procedure
    \item[\texttt{s}] step accepted in soft restoration phase
    \item[\texttt{t}/\texttt{T}] tiny step accepted without line search
    \item[\texttt{r}] some previous iterate restored
 \end{list}
\item[ls]
The number of backtracking line search steps (does not include second-order correction steps).
\end{description}

Note that the step acceptance mechanisms in \IPOPT consider the
barrier objective function \cite[Eq.~(3a)]{WaBi2006} which is
usually different from the value reported in the \texttt{objective}
column.  Similarly, for the purposes of the step acceptance, the
constraint violation is measured for the internal problem formulation,
which includes slack variables for inequality constraints and
potentially scaling of the constraint functions.  This value, too, is
usually different from the value reported in \texttt{inf\_pr}.  As a
consequence, a new iterate might have worse values both for the
objective function and the constraint violation as reported in the
iteration output, seemingly contradicting globalization procedure.


When the algorithm terminates, \IPOPT will output a message to the screen.
The following is a list of the possible output messages and a brief description.

\begin{description}
\item[Optimal Solution Found.] ~

    This message indicates that \IPOPT found a (locally) optimal point within the desired tolerances.

\item[Solved To Acceptable Level.] ~

    This indicates that the algorithm did not converge to the ``desired'' tolerances, but that it was able to obtain a point satisfying the ``acceptable'' tolerance level as specified by \texttt{acceptable-*} options.
    This may happen if the desired tolerances are too small for the current problem.

\item[Feasible point for square problem found.] ~

    This message is printed if the problem is ``square'' (i.e., it has as many equality constraints as free variables) and \IPOPT found a feasible point.

\item[Converged to a point of local infeasibility. Problem may be infeasible.] ~

    The restoration phase converged to a point that is a minimizer for the constraint violation (in the $\ell_1$-norm), but is not feasible for the original problem.
    This indicates that the problem may be infeasible (or at least that the algorithm is stuck at a locally infeasible point).
    The returned point (the minimizer of the constraint violation) might help you to find which constraint is causing the problem.
    If you believe that the NLP is feasible, it might help to start the optimization from a different point.

\item[Search Direction is becoming Too Small.] ~

    This indicates that \IPOPT is calculating very small step sizes and making very little progress.
    This could happen if the problem has been solved to the best numerical accuracy possible given the current scaling.

\item[Iterates divering; problem might be unbounded.] ~

    This message is printed if the max-norm of the iterates becomes larger than the value of the option \texttt{diverging\_iterates\_tol}.
    This can happen if the problem is unbounded below and the iterates are diverging.

\item[Stopping optimization at current point as requested by user.] ~

    This message is printed if either the Ctrl+C was pressed or the domain violation limit is reached.

\item[Maximum Number of Iterations Exceeded.] ~

    This indicates that \IPOPT has exceeded the maximum number of iterations as specified by the \IPOPT option \texttt{max\_iter} or the GAMS option \texttt{iterlim}.

\item[Maximum CPU time exceeded.] ~

    This indicates that \IPOPT has exceeded the maximum number of seconds as specified by the \IPOPT option \texttt{max\_cpu\_time} or the GAMS option \texttt{reslim}.

\item[Restoration Failed!] ~

    This indicates that the restoration phase failed to find a feasible point that was acceptable to the filter line search for the original problem.
    This could happen if the problem is highly degenerate or does not satisfy the constraint qualification, or if an external function in \GAMS provides incorrect derivative information.

\item[Error in step computation (regularization becomes too large?)!] ~

    This messages is printed if \IPOPT is unable to compute a search direction, despite several attempts to modify the iteration matrix.
    Usually, the value of the regularization parameter then becomes too large.

\item[Problem has too few degrees of freedom.] ~

    This indicates that your problem, as specified, has too few degrees of freedom.
    This can happen if you have too many equality constraints, or if you fix too many variables (\IPOPT removes fixed variables).

\item[Not enough memory.] ~

    An error occurred while trying to allocate memory.
    The problem may be too large for your current memory and swap configuration.

\item[INTERNAL ERROR: Unknown SolverReturn value - Notify \IPOPT Authors.] ~

    An unknown internal error has occurred. Please notify the authors of the \GAMS/\IPOPT link or \IPOPT (refer to \url{https://projects.coin-or.org/GAMSlinks} or \url{https://projects.coin-or.org/Ipopt}).
\end{description}


\subsection{Diagnostic Tags for \IPOPT}

To print additional diagnostic tags for each iteration of \IPOPT, set
the options \texttt{print\_info\_string} to \texttt{yes}. With
this, a tag will appear at the end of an iteration line with the
following diagnostic meaning that are useful to flag difficulties for
a particular \IPOPT run.  The following is a list of possible strings:
\begin{list}{blub}{\itemsep0pt}
 \item[\texttt{!}] Tighten resto tolerance if only slightly infeasible \cite[Sec.~3.3]{WaBi2006}
 \item[\texttt{A}] Current iteration is acceptable (alternate termination)
 \item[\texttt{a}] Perturbation for PD Singularity can't be done, assume singular \cite[Sec.~3.1]{WaBi2006}
 \item[\texttt{C}] Second Order Correction taken \cite[Sec.~2.4]{WaBi2006}
 \item[\texttt{Dh}] Hessian degenerate based on multiple iterations \cite[Sec.~3.1]{WaBi2006}
 \item[\texttt{Dhj}] Hessian/Jacobian degenerate based on multiple iterations \cite[Sec.~3.1]{WaBi2006}
 \item[\texttt{Dj}] Jacobian degenerate based on multiple iterations \cite[Sec.~3.1]{WaBi2006}
 \item[\texttt{dx}] $\delta_x$ perturbation too large \cite[Sec.~3.1]{WaBi2006}
 \item[\texttt{e}] Cutting back $\alpha$ due to evaluation error (in backtracking line search)
 \item[\texttt{F-}] Filter should be reset, but maximal resets exceeded \cite[Sec.~2.3]{WaBi2006}
 \item[\texttt{F+}] Resetting filter due to last few rejections of filter \cite[Sec.~2.3]{WaBi2006}
 \item[\texttt{L}] Degenerate Jacobian, $\delta_c$ already perturbed \cite[Sec.~3.1]{WaBi2006}
 \item[\texttt{l}] Degenerate Jacobian, $\delta_c$ perturbed \cite[Sec.~3.1]{WaBi2006}
 \item[\texttt{M}] Magic step taken for slack variables (in backtracking line search)
 \item[\texttt{Nh}] Hessian not yet degenerate \cite[Sec.~3.1]{WaBi2006}
 \item[\texttt{Nhj}] Hessian/Jacobian not yet degenerate \cite[Sec.~3.1]{WaBi2006}
 \item[\texttt{Nj}] Jacobian not yet degenerate \cite[Sec.~3.1]{WaBi2006}
 \item[\texttt{NW}] Warm start initialization failed (in Warm Start Initialization)
 \item[\texttt{q}] PD system possibly singular, attempt to improve solution quality \cite[Sec.~3.1]{WaBi2006}
 \item[\texttt{R}] Solution of restoration phase \cite[Sec.~3.3]{WaBi2006}
 \item[\texttt{S}] PD system possibly singular, accept current solution \cite[Sec.~3.1]{WaBi2006}
 \item[\texttt{s}] PD system singular \cite[Sec.~3.1]{WaBi2006}
 \item[\texttt{s}] Square Problem. Set multipliers to zero (default initialization routine)
 \item[\texttt{Tmax}] Trial $\theta$ is larger than $\theta_{max}$ (filter parameter \cite[Eq.~(21)]{WaBi2006})
 \item[\texttt{W}] Watchdog line search procedure successful \cite[Sec.~3.2]{WaBi2006}
 \item[\texttt{w}] Watchdog line search procedure unsuccessful, stopped \cite[Sec.~3.2]{WaBi2006}
 \item[\texttt{Wb}] Undoing most recent SR1 update \cite[Sec.~5.4.1]{Biegler2010}
 \item[\texttt{We}] Skip Limited-Memory Update in restoration phase  \cite[Sec.~5.4.1]{Biegler2010}
 \item[\texttt{Wp}] Safeguard $B^0 = \sigma I$ for  Limited-Memory Update \cite[Sec.~5.4.1]{Biegler2010}
 \item[\texttt{Wr}] Resetting Limited-Memory Update \cite[Sec.~5.4.1]{Biegler2010}
 \item[\texttt{Ws}] Skip Limited-Memory Update since $s^Ty$ is not positive \cite[Sec.~5.4.1]{Biegler2010}
 \item[\texttt{WS}] Skip Limited-Memory Update since $\Delta x$ is too small \cite[Sec.~5.4.1]{Biegler2010}
 \item[\texttt{y}] Dual infeasibility, use least square multiplier update (during \IPOPT algorithm)
 \item[\texttt{z}] Apply correction to bound multiplier if too large (during \IPOPT algorithm)
\end{list}

\section{Detailed Options Description}
\label{sub:ipoptoptions}

% Note, that \GAMS/\IPOPT overwrites the \IPOPT default setting for the parameters \texttt{bound\_relax\_factor} (set to $10^{-10}$) and \texttt{mu\_strategy} (set to \texttt{adaptive}).
% You can change these values by specifying these options in your \IPOPT options file.

\subsubsection{Output}

\paragraph{print\_level:} Output verbosity level. $\;$ \\
 Sets the default verbosity level for console
output. The larger this value the more detailed
is the output. The valid range for this integer option is
$0 \le {\tt print\_level } \le 11$
and its default value is $4$.


\paragraph{print\_user\_options:} Print all options set by the user. $\;$ \\
 If selected, the algorithm will print the list of
all options set by the user including their
values and whether they have been used.
The default value for this string option is ``no''.
\\ 
Possible values:
\begin{itemize}
   \item no: don't print options
   \item yes: print options
\end{itemize}

\paragraph{print\_options\_documentation:} Switch to print all algorithmic options. $\;$ \\
 If selected, the algorithm will print the list of
all available algorithmic options with some
documentation before solving the optimization
problem.
The default value for this string option is ``no''.
\\ 
Possible values:
\begin{itemize}
   \item no: don't print list
   \item yes: print list
\end{itemize}

\paragraph{output\_file:} File name of desired output file (leave unset for no file output). $\;$ \\
An output file with this
name will be written (leave unset for no file
output).  The verbosity level is by default set
to ``print\_level'', but can be overridden with
``file\_print\_level''.  The file name is changed
to use only small letters.
With the default settings no output file is generated.
\\ 
Possible values:
\begin{itemize}
   \item *: Any acceptable standard file name
\end{itemize}

\paragraph{file\_print\_level:} Verbosity level for output file. $\;$ \\
 NOTE: This option only works when read from the
ipopt.opt options file! Determines the verbosity
level for the file specified by ``output\_file''.
By default it is the same as ``print\_level''. The valid range for this integer option is
$0 \le {\tt file\_print\_level } \le 11$
and its default value is $4$.

\subsubsection{Termination}

\paragraph{tol:} Desired convergence tolerance (relative). $\;$ \\
 Determines the convergence tolerance for the
algorithm.  The algorithm terminates
successfully, if the (scaled) NLP error becomes
smaller than this value, and if the (absolute)
criteria according to ``dual\_inf\_tol'',
``primal\_inf\_tol'', and ``cmpl\_inf\_tol'' are met.
 (This is $\varepsilon_\mathrm{tol}$ in Eqn. (6) in the
implementation paper).  See also
``acceptable\_tol'' as a second termination
criterion.  Note, some other algorithmic features
also use this quantity to determine thresholds
etc. The valid range for this real option is 
$0 <  {\tt tol } <  {\tt +inf}$
and its default value is $1 \cdot 10^{-08}$.

\paragraph{s\_max:} Scaling threshold for the NLP error. $\;$ \\
The valid range for this integer option is
$0 \le {\tt s\_max } <  {\tt +inf}$
and its default value is $100$.

\paragraph{max\_iter:} Maximum number of iterations. $\;$ \\
 The algorithm terminates with an error message if
the number of iterations exceeded this number. The valid range for this integer option is
$0 \le {\tt max\_iter } <  {\tt +inf}$
and its default value is the value of the GAMS parameter iterlim, which default value is $10000$.


\paragraph{compl\_inf\_tol:} Desired threshold for the complementarity conditions. $\;$ \\
 Absolute tolerance on the complementarity.
Successful termination requires that the max-norm
of the (unscaled) complementarity is less than
this threshold. The valid range for this real option is 
$0 <  {\tt compl\_inf\_tol } <  {\tt +inf}$
and its default value is $0.0001$.


\paragraph{constr\_viol\_tol:} Desired threshold for the constraint violation. $\;$ \\
 Absolute tolerance on the constraint violation.
Successful termination requires that the max-norm
of the (unscaled) constraint violation is less
than this threshold. The default value for this real option is $0.0001$ and its
valid range is $0<\texttt{constr\_viol\_tol}<{\tt +inf}$.


\paragraph{dual\_inf\_tol:} Desired threshold for the dual infeasibility. $\;$ \\
 Absolute tolerance on the dual infeasibility.
Successful termination requires that the max-norm
of the (unscaled) dual infeasibility is less than
this threshold. The valid range for this real option is 
$0 <  {\tt dual\_inf\_tol } <  {\tt +inf}$
and its default value is $0.0001$.


\paragraph{acceptable\_tol:} ``Acceptable'' convergence tolerance (relative). $\;$ \\
 Determines which (scaled) overall optimality
error is considered to be ``acceptable''. There are
two levels of termination criteria.  If the usual
``desired'' tolerances (see tol, dual\_inf\_tol
etc) are satisfied at an iteration, the algorithm
immediately terminates with a success message. 
On the other hand, if the algorithm encounters
``acceptable\_iter'' many iterations in a row that
are considered ``acceptable'', it will terminate
before the desired convergence tolerance is met.
This is useful in cases where the algorithm might
not be able to achieve the ``desired'' level of
accuracy. The valid range for this real option is 
$0 <  {\tt acceptable\_tol } <  {\tt +inf}$
and its default value is $1 \cdot 10^{-06}$.

\paragraph{acceptable\_iter:} Number of ``acceptable'' iterates before triggering termination. $\;$ \\
If the algorithm encounters this many successive ``acceptable'' iterates (see ``acceptable\_tol''), it terminates, assuming that the problem has been solved to best possible accuracy given round-off.
If it is set to zero, this heuristic is disabled.
The valid range for this integer option is
$0 \le {\tt acceptable\_iter } <  {\tt +inf}$
and its default value is $15$.

\paragraph{acceptable\_compl\_inf\_tol:} ``Acceptance'' threshold for the complementarity conditions. $\;$ \\
 Absolute tolerance on the complementarity.
``Acceptable'' termination requires that the
max-norm of the (unscaled) complementarity is
less than this threshold; see also
acceptable\_tol. The valid range for this real option is 
$0 <  {\tt acceptable\_compl\_inf\_tol } <  {\tt +inf}$
and its default value is $0.01$.


\paragraph{acceptable\_constr\_viol\_tol:} ``Acceptance'' threshold for the constraint violation. $\;$ \\
 Absolute tolerance on the constraint violation.
``Acceptable'' termination requires that the
max-norm of the (unscaled) constraint violation
is less than this threshold; see also
acceptable\_tol. The valid range for this real option is 
$0 <  {\tt acceptable\_constr\_viol\_tol } <  {\tt +inf}$
and its default value is $0.01$.


\paragraph{acceptable\_dual\_inf\_tol:} ``Acceptance'' threshold for the dual infeasibility. $\;$ \\
 Absolute tolerance on the dual infeasibility.
``Acceptable'' termination requires that the
(max-norm of the unscaled) dual infeasibility is
less than this threshold; see also
acceptable\_tol. The valid range for this real option is 
$0 <  {\tt acceptable\_dual\_inf\_tol } <  {\tt +inf}$
and its default value is $0.01$.


\paragraph{diverging\_iterates\_tol:} Threshold for maximal value of primal iterates. $\;$ \\
 If any component of the primal iterates exceeded
this value (in absolute terms), the optimization
is aborted with the exit message that the
iterates seem to be diverging. The valid range for this real option is 
$0 <  {\tt diverging\_iterates\_tol } <  {\tt +inf}$
and its default value is $1 \cdot 10^{+20}$.

\subsubsection{NLP Scaling}

\paragraph{obj\_scaling\_factor:} Scaling factor for the objective function. $\;$ \\
 This option sets a scaling factor for the
objective function. The scaling is seen
internally by Ipopt but the unscaled objective is
reported in the console output. If additional
scaling parameters are computed (e.g.
user-scaling or gradient-based), both factors are
multiplied. The valid range for this real option is 
${\tt -inf} <  {\tt obj\_scaling\_factor } <  {\tt +inf}$
and its default value is $1$.


\paragraph{nlp\_scaling\_method:} Select the technique used for scaling the NLP. $\;$ \\
 Selects the technique used for scaling the
problem internally before it is solved. For
user-scaling, the parameters come from the values of the .scale suffix in GAMS.
The default value for this string option is ``gradient-based'' if scaleopt is 0 (default).
If the user provides variable or equation scaling values in GAMS and sets $<$model$>$.scaleopt to 1, then the default for this parameter is ``user-scaling''.
\\ 
Possible values:
\begin{itemize}
   \item none: no problem scaling will be performed
   \item user-scaling: scaling parameters will come from the user
   \item gradient-based: scale the problem so the maximum gradient at
the starting point is scaling\_max\_gradient
   \item equilibration-based: scale the problem so that first derivatives are
of order 1 at random points (only available with MC19)
\end{itemize}

\paragraph{nlp\_scaling\_max\_gradient:} Maximum gradient after NLP scaling. $\;$ \\
 This is the gradient scaling cut-off. If the
maximum gradient is above this value, then
gradient based scaling will be performed. Scaling
parameters are calculated to scale the maximum
gradient back to this value. (This is $g_{\max}$ in
Section 3.8 of the implementation paper.) Note:
This option is only used if
``nlp\_scaling\_method'' is chosen as
``gradient-based''. The valid range for this real option is 
$0 <  {\tt nlp\_scaling\_max\_gradient } <  {\tt +inf}$
and its default value is $100$.

\paragraph{nlp\_scaling\_obj\_target\_gradient:} Target value for objective function gradient size. $\;$ \\
     If a positive number is chosen, the scaling factor the objective function
     is computed so that the gradient as the max norm of the given size at the
     starting point.  This overrides nlp\_scaling\_max\_gradient for the
     objective function.
The valid range for this real option is 
$0 <  {\tt nlp\_scaling\_obj\_target\_gradient } <  {\tt +inf}$
and its default value is $0$.

\subsubsection{NLP corrections}

\paragraph{dependency\_detector:} Indicates which linear solver should be used to detect linearly dependent equality constraints. $\;$ \\
The default value for this string option is ``none''.
\\ 
Possible values:
\begin{itemize}
\item none:                    don't check; no extra work at beginning
\item mumps:                   use MUMPS
\item ma28:                     use MA28
\end{itemize}

\paragraph{dependency\_detection\_with\_rhs:} Indicates if the right hand sides of the constraints should be considered during dependency detection. $\;$ \\
The default value for this string option is ``no''.\\
Possible values:
\begin{itemize}
\item no:                      only look at gradients
\item yes:                     also consider right hand side
\end{itemize}

\paragraph{point\_perturbation\_radius:} Maximal perturbation of an evaluation point. $\;$ \\
     If a random perturbation of a points is required, this number indicates
     the maximal perturbation.  Currently, this is only used when we perturb
     the initial point in order to get a random Jacobian for the linear
     dependency detection of equality constraints.
The valid range for this real option is 
$0 \le {\tt point\_perturbation\_radius } <  {\tt +inf}$
and its default value is $10$.

\paragraph{kappa\_d:} Weight for linear damping term (to handle one-sided bounds). $\;$ \\
The valid range for this real option is 
$0 \le {\tt kappa\_d } <  {\tt +inf}$
and its default value is $10^{-5}$.

\paragraph{bound\_relax\_factor:} Factor for initial relaxation of the bounds. $\;$ \\
 Before start of the optimization, the bounds
given by the user are relaxed.  This option sets
the factor for this relaxation.  If it is set to
zero, then then bounds relaxation is disabled.
(See Eqn.(35) in the implementation paper.)
The valid range for this real option is 
$0 \le {\tt bound\_relax\_factor } <  {\tt +inf}$
and its default value is $0$.


\paragraph{honor\_original\_bounds:} Indicates whether final points should be projected into original bounds. $\;$ \\
 Ipopt might relax the bounds during the
optimization (see, e.g., option
``bound\_relax\_factor'').  This option determines
whether the final point should be projected back
into the user-provide original bounds after the
optimization.
The default value for this string option is ``yes''.
\\ 
Possible values:
\begin{itemize}
   \item no: Leave final point unchanged
   \item yes: Project final point back into original bounds
\end{itemize}

% \paragraph{check\_derivatives\_for\_naninf:} Indicates whether it is desired to check for Nan/Inf in derivative matrices $\;$ \\
%  Activating this option will cause an error if an
% invalid number is detected in the constraint
% Jacobians or the Lagrangian Hessian.  If this is
% not activated, the test is skipped, and the
% algorithm might proceed with invalid numbers and
% fail.
% The default value for this string option is ``no''.
% \\ 
% Possible values:
% \begin{itemize}
%    \item no: Don't check (faster).
%    \item yes: Check Jacobians and Hessian for Nan and Inf.
% \end{itemize}

\paragraph{fixed\_variable\_treatment:} Determines how fixed variables should be handled. $\;$ \\
The main difference between those options is that the starting point in the ``make\_constraint'' case still has the fixed variables at their given values, whereas in the case ``make\_parameter''
the functions are always evaluated with the fixed values for those variables. 
Also, for ``relax\_bounds'', the fixing bound constraints are relaxed (according to ``bound\_relax\_factor'').
For both ``make\_constraints'' and ``relax\_bounds'', bound multipliers are computed for the fixed variables.
The default value for this string option is ``make\_parameter''.
\\ 
Possible values:
\begin{itemize}
\item make\_parameter:    Remove fixed variable from optimization variables.
\item make\_constraint:   Add equality constraints fixing variables.
\item relax\_bounds:      Relax fixing bound constraints.
\end{itemize}


\subsubsection{Initialization}

\paragraph{bound\_frac:} Desired minimum relative distance from the initial point to bound. $\;$ \\
 Determines how much the initial point might have
to be modified in order to be sufficiently inside
the bounds (together with ``bound\_push'').
(This is $\kappa_2$ in Section 3.6 of the implementation paper.)
The valid range for this real option is 
$0 <  {\tt bound\_frac } \le 0.5$
and its default value is $0.01$.


\paragraph{bound\_push:} Desired minimum absolute distance from the initial point to bound. $\;$ \\
 Determines how much the initial point might have
to be modified in order to be sufficiently inside
the bounds (together with ``bound\_frac'').
(This is $\kappa_1$ in Section 3.6 of the implementation paper.)
The valid range for this real option is 
$0 <  {\tt bound\_push } <  {\tt +inf}$
and its default value is $0.01$.


\paragraph{slack\_bound\_push:} Desired minimum absolute distance from the initial slack to bound. $\;$ \\       Determines how much the initial slack variables might have to be modified
in order to be sufficiently inside the inequality bounds (together with ``slack\_bound\_frac'').
(This is $\kappa_1$ in Section 3.6 of the implementation paper.)
The valid range for this real option is 
$0 <  {\tt slack\_bound\_push } <  {\tt +inf}$
and its default value is $0.01$.

\paragraph{slack\_bound\_frac:} Desired minimum relative distance from the initial slack to bound.
     Determines how much the initial slack variables might have to be modified
     in order to be sufficiently inside the inequality bounds (together with
     ``slack\_bound\_push'').
(This is $\kappa_2$ in Section 3.6 of the implementation paper.)
The valid range for this real option is 
$0 <  {\tt slack\_bound\_frac } \le  0.5$
and its default value is $0.01$.


\paragraph{bound\_mult\_init\_val:} Initial value for the bound multipliers. $\;$ \\
 All dual variables corresponding to bound
constraints are initialized to this value. The valid range for this real option is 
$0 <  {\tt bound\_mult\_init\_val } <  {\tt +inf}$
and its default value is $1$.


\paragraph{constr\_mult\_init\_max:} Maximum allowed least-square guess of constraint multipliers. $\;$ \\
 Determines how large the initial least-square
guesses of the constraint multipliers are allowed
to be (in max-norm). If the guess is larger than
this value, it is discarded and all constraint
multipliers are set to zero.  This options is
also used when initializing the restoration
phase. By default,
``resto.constr\_mult\_init\_max'' (the one used in
RestoIterateInitializer) is set to zero. The valid range for this real option is 
$0 \le {\tt constr\_mult\_init\_max } <  {\tt +inf}$
and its default value is $1000$.


\paragraph{bound\_mult\_init\_val:} Initial value for the bound multipliers. $\;$ \\
 All dual variables corresponding to bound
constraints are initialized to this value. The valid range for this real option is 
$0 <  {\tt bound\_mult\_init\_val } <  {\tt +inf}$
and its default value is $1$.

\paragraph{least\_square\_init\_primal:} Least square initialization of the primal variables. $\;$ \\
     If set to yes, Ipopt ignores the user provided point and solves a least
     square problem for the primal variables (x and s), to fit the linearize
     equality and inequality constraints.  This might be useful if the user
     doesn't know anything about the starting point, or for solving an LP or
     QP.
The default value for this string option is ``no''.\\
   Possible values:
\begin{itemize}
    \item no:                      take user-provided point
    \item yes:                     overwrite user-provided point with least-square estimates
\end{itemize}

\paragraph{least\_square\_init\_duals:} Least square initialization of all dual variables. $\;$ \\
     If set to yes, Ipopt tries to compute least-square multipliers
     (considering ALL dual variables).  If successful, the bound multipliers
     are possibly corrected to be at least bound\_mult\_init\_val. This might be
     useful if the user doesn't know anything about the starting point, or for
     solving an LP or QP.
The default value for this string option is ``no''.\\
   Possible values:
\begin{itemize}
    \item no:                      use bound\_mult\_init\_val and least-square equality constraint multipliers
    \item yes:                     overwrite user-provided point with least-square estimates
\end{itemize}

\subsubsection{Barrier parameter update}

\paragraph{mehrotra\_algorithm:} Indicates if we want to do Mehrotra's algorithm. $\;$ \\
 If set to yes, Ipopt runs as Mehrotra's
predictor-corrector algorithm. This works usually
very well for LPs and convex QPs.  This
automatically disables the line search, and
chooses the (unglobalized) adaptive mu strategy
with the ``probing'' oracle, and uses
``corrector\_type=affine'' without any safeguards;
you should not set any of those options
explicitly in addition.  Also, unlessotherwise
specified, the values of ``bound\_push'',
``bound\_frac'', and ``bound\_mult\_init\_val'' are
set more aggressive, and sets
``alpha\_for\_y=bound\_mult''.
The default value for this string option is ``no''.
\\ 
Possible values:
\begin{itemize}
   \item no: Do the usual Ipopt algorithm.
   \item yes: Do Mehrotra's predictor-corrector algorithm.
\end{itemize}

\paragraph{mu\_strategy:} Update strategy for barrier parameter. $\;$ \\
 Determines which barrier parameter update
strategy is to be used.
The default value for this string option is ``adaptive''.
\\ 
Possible values:
\begin{itemize}
   \item monotone: use the monotone (Fiacco-McCormick) strategy
   \item adaptive: use the adaptive update strategy
\end{itemize}

\paragraph{mu\_oracle:} Oracle for a new barrier parameter in the adaptive strategy. $\;$ \\
 Determines how a new barrier parameter is
computed in each ``free-mode'' iteration of the
adaptive barrier parameter strategy. (Only
considered if ``adaptive'' is selected for option
``mu\_strategy'').
The default value for this string option is ``quality-function''.
\\ 
Possible values:
\begin{itemize}
   \item probing: Mehrotra's probing heuristic
   \item loqo: LOQO's centrality rule
   \item quality-function: minimize a quality function
\end{itemize}

\paragraph{quality\_function\_max\_section\_steps:} Maximum number of search steps during direct search procedure determining the optimal centering parameter. $\;$ \\
 The golden section search is performed for the
quality function based mu oracle. The valid range for this integer option is
$0 \le {\tt quality\_function\_max\_section\_steps } <  {\tt +inf}$
and its default value is $8$.
This option is only used if the option ``mu\_oracle'' is set to ``quality-function''.


\paragraph{fixed\_mu\_oracle:} Oracle for the barrier parameter when switching to fixed mode. $\;$ \\
 Determines how the first value of the barrier
parameter should be computed when switching to
the ``monotone mode'' in the adaptive strategy.
(Only considered if ``adaptive'' is selected for
option ``mu\_strategy''.)
The default value for this string option is ``average\_compl''.
\\ 
Possible values:
\begin{itemize}
   \item probing: Mehrotra's probing heuristic
   \item loqo: LOQO's centrality rule
   \item quality-function: minimize a quality function
   \item average\_compl: base on current average complementarity
\end{itemize}

\paragraph{mu\_init:} Initial value for the barrier parameter. $\;$ \\
 This option determines the initial value for the
barrier parameter (mu).  It is only relevant in
the monotone, Fiacco-McCormick version of the
algorithm. (i.e., if ``mu\_strategy'' is chosen as
``monotone'') The valid range for this real option is 
$0 <  {\tt mu\_init } <  {\tt +inf}$
and its default value is $0.1$.

\paragraph{mu\_max\_fact:} Factor for initialization of maximum value for barrier parameter. $\;$ \\
 This option determines the upper bound on the
barrier parameter.  This upper bound is computed
as the average complementarity at the initial
point times the value of this option. (Only used
if option ``mu\_strategy'' is chosen as ``adaptive''.) The valid range for this real option is 
$0 <  {\tt mu\_max\_fact } <  {\tt +inf}$
and its default value is $1000$.


\paragraph{mu\_max:} Maximum value for barrier parameter. $\;$ \\
 This option specifies an upper bound on the
barrier parameter in the adaptive mu selection
mode.  If this option is set, it overwrites the
effect of mu\_max\_fact. (Only used if option
``mu\_strategy'' is chosen as ``adaptive''.) The valid range for this real option is 
$0 <  {\tt mu\_max } <  {\tt +inf}$
and its default value is $100000$.


\paragraph{mu\_min:} Minimum value for barrier parameter. $\;$ \\
 This option specifies the lower bound on the
barrier parameter in the adaptive mu selection
mode. By default, it is set to
min(``tol'', ``compl\_inf\_tol'')/(``barrier\_tol\_fact-
or''+1), which should be a reasonable value. (Only
used if option ``mu\_strategy'' is chosen as
``adaptive''.) The valid range for this real option is 
$0 <  {\tt mu\_min } <  {\tt +inf}$
and its default value is $1 \cdot 10^{-09}$.

\paragraph{barrier\_tol\_factor:} Factor for mu in barrier stop test. $\;$ \\
 The convergence tolerance for each barrier
problem in the monotone mode is the value of the
barrier parameter times ``barrier\_tol\_factor''.
This option is also used in the adaptive mu
strategy during the monotone mode. (This is
$\kappa_\varepsilon$ in the implementation paper). The valid range for this real option is 
$0 <  {\tt barrier\_tol\_factor } <  {\tt +inf}$
and its default value is $10$.

\paragraph{mu\_linear\_decrease\_factor:} Determines linear decrease rate of barrier parameter. $\;$ \\
 For the Fiacco-McCormick update procedure the new
barrier parameter mu is obtained by taking the
minimum of mu$\cdot$``mu\_linear\_decrease\_factor'' and
mu$^\textrm{``superlinear\_decrease\_power''}$.  (This is
$\kappa_\mu$ in the implementation paper.) This option
is also used in the adaptive mu strategy during
the monotone mode. The valid range for this real option is 
$0 <  {\tt mu\_linear\_decrease\_factor } <  1$
and its default value is $0.2$.


\paragraph{mu\_superlinear\_decrease\_power:} Determines superlinear decrease rate of barrier parameter. $\;$ \\
 For the Fiacco-McCormick update procedure the new
barrier parameter mu is obtained by taking the
minimum of mu$\cdot$``mu\_linear\_decrease\_factor'' and
mu$^\textrm{``superlinear\_decrease\_power''}$.  (This is
$\theta_\mu$ in the implementation paper.) This option
is also used in the adaptive mu strategy during
the monotone mode. The valid range for this real option is 
$1 <  {\tt mu\_superlinear\_decrease\_power } <  2$
and its default value is $1.5$.

\subsubsection{Barrier parameter update (expert options)}

\paragraph{mu\_allow\_fast\_monotone\_decrease:} Allow skipping of barrier problem if barrier test is already met. $\;$ \\
 If set to ``no'', the algorithm enforces at least
one iteration per barrier problem, even if the
barrier test is already met for the updated
barrier parameter.
The default value for this string option is ``yes''.
\\ 
Possible values:
\begin{itemize}
   \item no: Take at least one iteration per barrier problem
   \item yes: Allow fast decrease of mu if barrier test it met
\end{itemize}

\paragraph{adaptive\_mu\_globalization:} Globalization strategy for the adaptive mu selection mode. $\;$ \\
 To achieve global convergence of the adaptive
version, the algorithm has to switch to the
monotone mode (Fiacco-McCormick approach) when
convergence does not seem to appear.  This option
sets the criterion used to decide when to do this
switch. (Only used if option ``mu\_strategy'' is
chosen as ``adaptive''.)
The default value for this string option is ``obj-constr-filter''.
\\ 
Possible values:
\begin{itemize}
   \item kkt-error: nonmonotone decrease of kkt-error
   \item obj-constr-filter: 2-dim filter for objective and constraint
violation
   \item never-monotone-mode: disables globalization
\end{itemize}

\paragraph{adaptive\_mu\_kkterror\_red\_iters:} Maximum number of iterations requiring sufficient progress. $\;$ \\
 For the ``kkt-error'' based globalization strategy,
the progress made in at most ``adaptive\_mu\_kkterror\_red\_iters'' iterations must be sufficient.
If this number of iterations is exceeded, the
globalization strategy switches to the monotone
mode. The valid range for this integer option is
$0 \le {\tt adaptive\_mu\_kkterror\_red\_iters } <  {\tt +inf}$
and its default value is $4$.


\paragraph{adaptive\_mu\_kkterror\_red\_fact:} Sufficient decrease factor for ``kkt-error'' globalization strategy. $\;$ \\
 For the ``kkt-error'' based globalization strategy,
the error must decrease by this factor to be
deemed sufficient decrease. The valid range for this real option is 
$0 <  {\tt adaptive\_mu\_kkterror\_red\_fact } <  1$
and its default value is $0.9999$.


\paragraph{filter\_margin\_fact:} Factor determining width of margin for obj-constr-filter adaptive globalization strategy. $\;$ \\
 When using the adaptive globalization strategy,
``obj-constr-filter'', sufficient progress for a
filter entry is defined as follows: (new obj) <
(filter obj) - filter\_margin\_fact$\cdot$ (new
constr-viol) OR (new constr-viol) $<$ (filter
constr-viol) - filter\_margin\_fact$\cdot$ (new
constr-viol).  For the description of the
``kkt-error-filter'' option see
``filter\_max\_margin''. The valid range for this real option is 
$0 <  {\tt filter\_margin\_fact } <  1$
and its default value is $1 \cdot 10^{-05}$.


\paragraph{filter\_max\_margin:} Maximum width of margin in obj-constr-filter adaptive globalization strategy. $\;$ \\
 The valid range for this real option is 
$0 <  {\tt filter\_max\_margin } <  {\tt +inf}$
and its default value is $1$.


\paragraph{adaptive\_mu\_restore\_previous\_iterate:} Indicates if the previous iterate should be restored if the monotone mode is entered. $\;$ \\
 When the globalization strategy for the adaptive
barrier algorithm switches to the monotone mode,
it can either start from the most recent iterate
(no), or from the last iterate that was accepted
(yes).
The default value for this string option is ``no''.
\\ 
Possible values:
\begin{itemize}
   \item no: don't restore accepted iterate
   \item yes: restore accepted iterate
\end{itemize}

\paragraph{adaptive\_mu\_monotone\_init\_factor:} Determines the initial value of the barrier parameter when switching to the monotone mode. $\;$ \\
 When the globalization strategy for the adaptive
barrier algorithm switches to the monotone mode
and the option fixed\_mu\_oracle is chosen as
``average\_compl'', the barrier parameter is set to
the current average complementarity times the
value of ``adaptive\_mu\_monotone\_init\_factor''. The default value for this option is $0.8$ and its valid range is $0 <  {\tt adaptive\_mu\_monotone\_init\_factor } <  {\tt +inf}$.


\paragraph{adaptive\_mu\_kkt\_norm\_type:} Norm used for the KKT error in the adaptive mu globalization strategies. $\;$ \\
 When computing the KKT error for the
globalization strategies, the norm to be used is
specified with this option. Note, this options is
also used in the QualityFunctionMuOracle.
The default value for this string option is ``2-norm-squared''.
\\ 
Possible values:
\begin{itemize}
   \item 1-norm: use the 1-norm (abs sum)
   \item 2-norm-squared: use the 2-norm squared (sum of squares)
   \item max-norm: use the infinity norm (max)
   \item 2-norm: use 2-norm
\end{itemize}

\paragraph{tau\_min:} Lower bound on fraction-to-the-boundary parameter tau. $\;$ \\
 (This is $\tau_{\min}$ in the implementation paper.)  This
option is also used in the adaptive mu strategy
during the monotone mode. The valid range for this real option is 
$0 <  {\tt tau\_min } <  1$
and its default value is $0.99$.


\paragraph{sigma\_max:} Maximum value of the centering parameter. $\;$ \\
 This is the upper bound for the centering
parameter chosen by the quality function based
barrier parameter update. (Only used if option
``mu\_oracle'' is set to ``quality-function''.) The valid range for this real option is 
$0 <  {\tt sigma\_max } <  {\tt +inf}$
and its default value is $100$.


\paragraph{sigma\_min:} Minimum value of the centering parameter. $\;$ \\
 This is the lower bound for the centering
parameter chosen by the quality function based
barrier parameter update. (Only used if option
``mu\_oracle'' is set to ``quality-function''.) The valid range for this real option is 
$0 \le {\tt sigma\_min } <  {\tt +inf}$
and its default value is $1 \cdot 10^{-06}$.


\paragraph{quality\_function\_norm\_type:} Norm used for components of the quality function. $\;$ \\
 (Only used if option ``mu\_oracle'' is set to
``quality-function''.)
The default value for this string option is ``2-norm-squared''.
\\ 
Possible values:
\begin{itemize}
   \item 1-norm: use the 1-norm (abs sum)
   \item 2-norm-squared: use the 2-norm squared (sum of squares)
   \item max-norm: use the infinity norm (max)
   \item 2-norm: use 2-norm
\end{itemize}

\paragraph{quality\_function\_centrality:} The penalty term for centrality that is included in quality function. $\;$ \\
 This determines whether a term is added to the
quality function to penalize deviation from
centrality with respect to complementarity.  The
complementarity measure here is the xi in the
Loqo update rule. (Only used if option
``mu\_oracle'' is set to ``quality-function''.)
The default value for this string option is ``none''.
\\ 
Possible values:
\begin{itemize}
   \item none: no penalty term is added
   \item log: complementarity $\cdot$ the log of the centrality
measure
   \item reciprocal: complementarity $\cdot$ the reciprocal of the
centrality measure
   \item cubed-reciprocal: complementarity $\cdot$ the reciprocal of the
centrality measure cubed
\end{itemize}

\paragraph{quality\_function\_balancing\_term:} The balancing term included in the quality function for centrality. $\;$ \\
 This determines whether a term is added to the
quality function that penalizes situations where
the complementarity is much smaller than dual and
primal infeasibilities. (Only used if option
``mu\_oracle'' is set to ``quality-function''.)
The default value for this string option is ``none''.
\\ 
Possible values:
\begin{itemize}
   \item none: no balancing term is added
   \item cubic: $\max(0,\max(\textrm{dual\_inf},\textrm{primal\_inf})-\textrm{compl})^3$
\end{itemize}

\paragraph{quality\_function\_max\_section\_steps:} Maximum number of search steps during direct search procedure determining the optimal centering parameter. $\;$ \\
 The golden section search is performed for the
quality function based mu oracle. The valid range for this integer option is
$0 \le {\tt quality\_function\_max\_section\_steps } <  {\tt +inf}$
and its default value is $8$.
(Only used if
option ``mu\_oracle'' is set to ``quality-function''.)


\paragraph{quality\_function\_section\_sigma\_tol:} Tolerance for the section search procedure determining the optimal centering parameter (in sigma space). $\;$ \\
 The golden section search is performed for the
quality function based mu oracle. (Only used if
option ``mu\_oracle'' is set to ``quality-function''.) The valid range for this real option is 
$0 \le {\tt quality\_function\_section\_sigma\_tol } <  1$
and its default value is $0.01$.


\paragraph{quality\_function\_section\_qf\_tol:} Tolerance for the golden section search procedure determining the optimal centering parameter (in the function value space). $\;$ \\
 The golden section search is performed for the
quality function based mu oracle. (Only used if
option ``mu\_oracle'' is set to ``quality-function''.) The valid range for this real option is 
$0 \le {\tt quality\_function\_section\_qf\_tol } <  1$
and its default value is $0$.


\subsubsection{Warm start}

\paragraph{warm\_start\_init\_point:} Warm-start for initial point $\;$ \\
 Indicates whether this optimization should use a warm start initialization, where values of dual variables are given by GAMS (You can set marginal values for variables and equations in your GAMS model to set the starting point for the dual variables.)
For the primal values, Ipopt uses the starting point that is given by GAMS (You can set level values for variables (and equations) in your GAMS model to set the starting point for the primal variables.)
The default value for this string option is ``no''.
\\ 
Possible values:
\begin{itemize}
   \item no: do not use the warm start initialization
   \item yes: use the warm start initialization
\end{itemize}

\paragraph{warm\_start\_bound\_push:} same as bound\_push for the regular initializer. $\;$ \\
 The valid range for this real option is 
$0 <  {\tt warm\_start\_bound\_push } <  {\tt +inf}$
and its default value is $0.001$.


\paragraph{warm\_start\_bound\_frac:} same as bound\_frac for the regular initializer. $\;$ \\
 The valid range for this real option is 
$0 <  {\tt warm\_start\_bound\_frac } \le 0.5$
and its default value is $0.001$.


\paragraph{warm\_start\_slack\_bound\_frac:} same as slack\_bound\_frac for the regular initializer. $\;$ \\
 The valid range for this real option is 
$0 <  {\tt warm\_start\_slack\_bound\_frac } \le 0.5$
and its default value is $0.001$.


\paragraph{warm\_start\_slack\_bound\_push:} same as slack\_bound\_push for the regular initializer. $\;$ \\
 The valid range for this real option is 
$0 <  {\tt warm\_start\_slack\_bound\_push } <  {\tt +inf}$
and its default value is $0.001$.


\paragraph{warm\_start\_mult\_bound\_push:} same as mult\_bound\_push for the regular initializer. $\;$ \\
 The valid range for this real option is 
$0 <  {\tt warm\_start\_mult\_bound\_push } <  {\tt +inf}$
and its default value is $0.001$.


\paragraph{warm\_start\_mult\_init\_max:} Maximum initial value for the equality multipliers. $\;$ \\
 The valid range for this real option is 
${\tt -inf} <  {\tt warm\_start\_mult\_init\_max } <  {\tt +inf}$
and its default value is $1 \cdot 10^{+06}$.

\subsubsection{Multiplier updates}

\paragraph{alpha\_for\_y:} Method to determine the step size for constraint multipliers. $\;$ \\
 This option determines how the step size
(alpha\_y) will be calculated when updating the
constraint multipliers.
The default value for this string option is ``primal''.
\\ 
Possible values:
\begin{itemize}
   \item primal: use primal step size
   \item bound\_mult: use step size for the bound multipliers (good
for LPs)
   \item min: use the min of primal and bound multipliers
   \item max: use the max of primal and bound multipliers
   \item full: take a full step of size one
   \item min\_dual\_infeas: choose step size minimizing new dual
infeasibility
   \item safe\_min\_dual\_infeas: like ``min\_dual\_infeas'', but safeguarded by
``min'' and ``max''
\end{itemize}

\paragraph{alpha\_for\_y\_tol:} Tolerance for switching to full equality multiplier steps. $\;$ \\
 This is only relevant if ``alpha\_for\_y'' is
chosen ``primal-and-full'' or ``dual-and-full''.  The
step size for the equality constraint multipliers
is taken to be one if the max-norm of the primal
step is less than this tolerance. The valid range for this real option is 
$0 \le {\tt alpha\_for\_y\_tol } <  {\tt +inf}$
and its default value is $10$.

\paragraph{recalc\_y:} Tells the algorithm to recalculate the equality and inequality multipliers as least square estimates. $\;$ \\
 This asks Ipopt to recompute the
multipliers, whenever the current infeasibility
is less than recalc\_y\_feas\_tol. Choosing yes
might be helpful in the quasi-Newton option. 
However, each recalculation requires an extra
factorization of the linear system.  If a limited
memory quasi-Newton option is chosen, this is
used by default.
The default value for this string option is ``no''.
\\ 
Possible values:
\begin{itemize}
   \item no: use the Newton step to update the multipliers
   \item yes: use least-square multiplier estimates
\end{itemize}

\paragraph{recalc\_y\_feas\_tol:} Feasibility threshold for recomputation of multipliers. $\;$ \\
 If recalc\_y is chosen and the current
infeasibility is less than this value, then the
multipliers are recomputed. The valid range for this real option is 
$0 <  {\tt recalc\_y\_feas\_tol } <  {\tt +inf}$
and its default value is $1 \cdot 10^{-06}$.

\subsubsection{Line search}

\paragraph{max\_soc:} Maximum number of second order correction trial steps at each iteration. $\;$ \\
 Choosing 0 disables the second order corrections.
(This is $p^{\max}$ of Step A-5.9 of Algorithm A in
the implementation paper.) The valid range for this integer option is
$0 \le {\tt max\_soc } <  {\tt +inf}$
and its default value is $4$.


\paragraph{watchdog\_shortened\_iter\_trigger:} Number of shortened iterations that trigger the watchdog. $\;$ \\
 If the number of successive iterations in which
the backtracking line search did not accept the
first trial point exceeds this number, the
watchdog procedure is activated.  Choosing 0
here disables the watchdog procedure. The valid range for this integer option is
$0 \le {\tt watchdog\_shortened\_iter\_trigger } <  {\tt +inf}$
and its default value is $10$.


\paragraph{watchdog\_trial\_iter\_max:} Maximum number of watchdog iterations. $\;$ \\
 This option determines the number of trial
iterations allowed before the watchdog procedure
is aborted and the algorithm returns to the
stored point. The valid range for this integer option is
$1 \le {\tt watchdog\_trial\_iter\_max } <  {\tt +inf}$
and its default value is $3$.

\paragraph{corrector\_type:} The type of corrector steps that should be taken (experimental!). $\;$ \\
 If ``mu\_strategy'' is ``adaptive'', this option
determines what kind of corrector steps should be
tried.
The default value for this string option is ``none''.
\\ 
Possible values:
\begin{itemize}
   \item none: no corrector
   \item affine: corrector step towards mu=0
   \item primal-dual: corrector step towards current mu
\end{itemize}

\subsubsection{Line search (expert options)}

\paragraph{alpha\_red\_factor:} Fractional reduction of the trial step size in the backtracking line search. $\;$ \\
 At every step of the backtracking line search,
the trial step size is reduced by this factor. The valid range for this real option is 
$0 <  {\tt alpha\_red\_factor } <  1$
and its default value is $0.5$.


\paragraph{accept\_every\_trial\_step:} Always accept the first trial step. $\;$ \\
 Setting this option to ``yes'' essentially disables
the line search and makes the algorithm take
aggressive steps, without global convergence
guarantees.
The default value for this string option is ``no''.
\\ 
Possible values:
\begin{itemize}
   \item no: don't arbitrarily accept the full step
   \item yes: always accept the full step
\end{itemize}

\paragraph{tiny\_step\_tol:} Tolerance for detecting numerically insignificant steps. $\;$ \\
 If the search direction in the primal variables
(x and s) is, in relative terms for each
component, less than this value, the algorithm
accepts the full step without line search.  If
this happens repeatedly, the algorithm will
terminate with a corresponding exit message. The
default value is 10 times machine precision. The valid range for this real option is 
$0 \le {\tt tiny\_step\_tol } <  {\tt +inf}$
and its default value is $2.22045 \cdot 10^{-15}$.


\paragraph{tiny\_step\_y\_tol:} Tolerance for quitting because of numerically insignificant steps. $\;$ \\
 If the search direction in the primal variables
(x and s) is, in relative terms for each
component, repeatedly less than tiny\_step\_tol,
and the step in the y variables is smaller than
this threshold, the algorithm will terminate. The valid range for this real option is 
$0 \le {\tt tiny\_step\_y\_tol } <  {\tt +inf}$
and its default value is $0.01$.


\paragraph{theta\_max\_fact:} Determines upper bound for constraint violation in the filter. $\;$ \\
 The algorithmic parameter theta\_max is
determined as theta\_max\_fact times the maximum
of 1 and the constraint violation at initial
point.  Any point with a constraint violation
larger than theta\_max is unacceptable to the
filter (see Eqn. (21) in the implementation paper). The valid range for this real option is 
$0 <  {\tt theta\_max\_fact } <  {\tt +inf}$
and its default value is $10000$.


\paragraph{theta\_min\_fact:} Determines constraint violation threshold in the switching rule. $\;$ \\
 The algorithmic parameter theta\_min is
determined as theta\_min\_fact times the maximum
of 1 and the constraint violation at initial
point.  The switching rules treats an iteration
as an h-type iteration whenever the current
constraint violation is larger than theta\_min
(see paragraph before Eqn. (19) in the implementation
paper). The valid range for this real option is 
$0 <  {\tt theta\_min\_fact } <  {\tt +inf}$
and its default value is $0.0001$.


\paragraph{eta\_phi:} Relaxation factor in the Armijo condition. $\;$ \\
 (See Eqn. (20) in the implementation paper) The valid range for this real option is 
$0 <  {\tt eta\_phi } <  0.5$
and its default value is $1 \cdot 10^{-08}$.


\paragraph{delta:} Multiplier for constraint violation in the switching rule. $\;$ \\
 (See Eqn. (19) in the implementation paper.) The valid range for this real option is 
$0 <  {\tt delta } <  {\tt +inf}$
and its default value is $1$.


\paragraph{s\_phi:} Exponent for linear barrier function model in the switching rule. $\;$ \\
 (See Eqn. (19) in the implementation paper.) The valid range for this real option is 
$1 <  {\tt s\_phi } <  {\tt +inf}$
and its default value is $2.3$.


\paragraph{s\_theta:} Exponent for current constraint violation in the switching rule. $\;$ \\
 (See Eqn. (19) in the implementation paper.) The valid range for this real option is 
$1 <  {\tt s\_theta } <  {\tt +inf}$
and its default value is $1.1$.


\paragraph{gamma\_phi:} Relaxation factor in the filter margin for the barrier function. $\;$ \\
 (See Eqn. (18a) in the implementation paper.) The valid range for this real option is 
$0 <  {\tt gamma\_phi } <  1$
and its default value is $1 \cdot 10^{-08}$.


\paragraph{gamma\_theta:} Relaxation factor in the filter margin for the constraint violation. $\;$ \\
 (See Eqn. (18b) in the implementation paper.) The valid range for this real option is 
$0 <  {\tt gamma\_theta } <  1$
and its default value is $1 \cdot 10^{-05}$.


\paragraph{alpha\_min\_frac:} Safety factor for the minimal step size (before switching to restoration phase). $\;$ \\
 (This is $\gamma_\alpha$ in Eqn. (20) in the
implementation paper.) The default value of this real option is $0.05$ and its
valid range is $0 <  {\tt alpha\_min\_frac } <  1$.


\paragraph{kappa\_soc:} Factor in the sufficient reduction rule for second order correction. $\;$ \\
 This option determines how much a second order
correction step must reduce the constraint
violation so that further correction steps are
attempted.  (See Step A-5.9 of Algorithm A in
the implementation paper.) The valid range for this real option is 
$0 <  {\tt kappa\_soc } <  {\tt +inf}$
and its default value is $0.99$.


\paragraph{obj\_max\_inc:} Determines the upper bound on the acceptable increase of barrier objective function. $\;$ \\
 Trial points are rejected if they lead to an
increase in the barrier objective function by
more than obj\_max\_inc orders of magnitude. The valid range for this real option is 
$1 <  {\tt obj\_max\_inc } <  {\tt +inf}$
and its default value is $5$.


\paragraph{max\_filter\_resets:} Maximal allowed number of filter resets $\;$ \\
 A positive number enables a heuristic that resets
the filter, whenever in more than
``filter\_reset\_trigger'' successive iterations
the last rejected trial steps size was rejected
because of the filter.  This option determine the
maximal number of resets that are allowed to take
place. The valid range for this integer option is
$0 \le {\tt max\_filter\_resets } <  {\tt +inf}$
and its default value is $5$.


\paragraph{filter\_reset\_trigger:} Number of iterations that trigger the filter reset. $\;$ \\
 If the filter reset heuristic is active and the
number of successive iterations in which the last
rejected trial step size was rejected because of
the filter, the filter is reset. The valid range for this integer option is
$1 \le {\tt filter\_reset\_trigger } <  {\tt +inf}$
and its default value is $5$.


% \paragraph{skip\_corr\_if\_neg\_curv:} Skip the corrector step in negative curvature iteration (unsupported!). $\;$ \\
%  The corrector step is not tried if negative
% curvature has been encountered during the
% computation of the search direction in the
% current iteration. This option is only used if
% ``mu\_strategy'' is ``adaptive''.
% The default value for this string option is ``yes''.
% \\ 
% Possible values:
% \begin{itemize}
%    \item no: don't skip
%    \item yes: skip
% \end{itemize}

% \paragraph{skip\_corr\_in\_monotone\_mode:} Skip the corrector step during monotone barrier parameter mode (unsupported!). $\;$ \\
%  The corrector step is not tried if the algorithm
% is currently in the monotone mode (see also
% option ``barrier\_strategy'').This option is only
% used if ``mu\_strategy'' is ``adaptive''.
% The default value for this string option is ``yes''.
% \\ 
% Possible values:
% \begin{itemize}
%    \item no: don't skip
%    \item yes: skip
% \end{itemize}

% \paragraph{corrector\_compl\_avrg\_red\_fact:} Complementarity tolerance factor for accepting corrector step (unsupported!). $\;$ \\
%  This option determines the factor by which
% complementarity is allowed to increase for a
% corrector step to be accepted. The valid range for this real option is 
% $0 <  {\tt corrector\_compl\_avrg\_red\_fact } <  {\tt +inf}$
% and its default value is $1$.


\paragraph{kappa\_sigma:} Factor limiting the deviation of dual variables from primal estimates. $\;$ \\
 If the dual variables deviate from their primal
estimates, a correction is performed. (See Eqn.
(16) in the the implementation paper.) Setting the
value to less than 1 disables the correction. The valid range for this real option is 
$0 <  {\tt kappa\_sigma } <  {\tt +inf}$
and its default value is $1 \cdot 10^{+10}$.


\paragraph{slack\_move:} Correction size for very small slacks. $\;$ \\
 Due to numerical issues or the lack of an
interior, the slack variables might become very
small.  If a slack becomes very small compared to
machine precision, the corresponding bound is
moved slightly.  This parameter determines how
large the move should be.  Its default value is
mach\_eps$^{3/4}$.  (See also end of Section 3.5
in the implementation paper - but actual
the implementation might be somewhat different.) The valid range for this real option is 
$0 \le {\tt slack\_move } <  {\tt +inf}$
and its default value is $1.81899 \cdot 10^{-12}$.

\subsubsection{Restoration phase}

\paragraph{expect\_infeasible\_problem:} Enable heuristics to quickly detect an infeasible problem. $\;$ \\
 This options is meant to activate heuristics that
may speed up the infeasibility determination if
you expect that there is a good chance for the
problem to be infeasible.  In the filter line
search procedure, the restoration phase is called
more quickly than usually, and more reduction in
the constraint violation is enforced before the
restoration phase is left. If the problem is
square, this option is enabled automatically.
The default value for this string option is ``no''.
\\ 
Possible values:
\begin{itemize}
   \item no: the problem probably be feasible
   \item yes: the problem has a good chance to be infeasible
\end{itemize}

\paragraph{expect\_infeasible\_problem\_ctol:} Threshold for disabling ``expect\_infeasible\_problem'' option. $\;$ \\
 If the constraint violation becomes smaller than
this threshold, the ``expect\_infeasible\_problem''
heuristics in the filter line search are
disabled. If the problem is square, this options
is set to 0. The valid range for this real option is 
$0 \le {\tt expect\_infeasible\_problem\_ctol } <  {\tt +inf}$
and its default value is $0.001$.


\paragraph{start\_with\_resto:} Tells algorithm to switch to restoration phase in first iteration. $\;$ \\
 Setting this option to ``yes'' forces the algorithm
to switch to the feasibility restoration phase in
the first iteration. If the initial point is
feasible, the algorithm will abort with a failure.
The default value for this string option is ``no''.
\\ 
Possible values:
\begin{itemize}
   \item no: don't force start in restoration phase
   \item yes: force start in restoration phase
\end{itemize}

\paragraph{soft\_resto\_pderror\_reduction\_factor:} Required reduction in primal-dual error in the soft restoration phase. $\;$ \\
 The soft restoration phase attempts to reduce the
primal-dual error with regular steps. If the
damped primal-dual step (damped only to satisfy
the fraction-to-the-boundary rule) is not
decreasing the primal-dual error by at least this
factor, then the regular restoration phase is
called. Choosing 0 here disables the soft
restoration phase. The valid range for this real option is 
$0 \le {\tt soft\_resto\_pderror\_reduction\_factor } <  {\tt +inf}$
and its default value is $0.9999$.


\paragraph{required\_infeasibility\_reduction:} Required reduction of infeasibility before leaving restoration phase. $\;$ \\
 The restoration phase algorithm is performed,
until a point is found that is acceptable to the
filter and the infeasibility has been reduced by
at least the fraction given by this option. The valid range for this real option is 
$0 \le {\tt required\_infeasibility\_reduction } <  1$
and its default value is $0.9$.


\paragraph{max\_soft\_resto\_iters:} Maximum number of iterations performed successively in soft restoration phase. $\;$ \\
 If the soft restoration phase is performed for
more than so many iterations in a row, the regular
restoration phase is called. The valid range for this integer option is
$0 \le {\tt max\_soft\_resto\_iters } <  {\tt +inf}$
and its default value is $10$.


\paragraph{max\_resto\_iter:} Maximum number of successive iterations in restoration phase. $\;$ \\
 The algorithm terminates with an error message if
the number of iterations successively taken in
the restoration phase exceeds this number. The valid range for this integer option is
$0 \le {\tt max\_resto\_iter } <  {\tt +inf}$
and its default value is $3000000$.


\paragraph{bound\_mult\_reset\_threshold:} Threshold for resetting bound multipliers after the restoration phase. $\;$ \\
 After returning from the restoration phase, the
bound multipliers are updated with a Newton step
for complementarity.  Here, the change in the
primal variables during the entire restoration
phase is taken to be the corresponding primal
Newton step. However, if after the update the
largest bound multiplier exceeds the threshold
specified by this option, the multipliers are all
reset to 1. The valid range for this real option is 
$0 \le {\tt bound\_mult\_reset\_threshold } <  {\tt +inf}$
and its default value is $1000$.


\paragraph{constr\_mult\_reset\_threshold:} Threshold for resetting equality and inequality multipliers after restoration phase. $\;$ \\
 After returning from the restoration phase, the
constraint multipliers are recomputed by a least
square estimate.  This option triggers when those
least-square estimates should be ignored. The valid range for this real option is 
$0 \le {\tt constr\_mult\_reset\_threshold } <  {\tt +inf}$
and its default value is $0$.


\paragraph{evaluate\_orig\_obj\_at\_resto\_trial:} Determines if the original objective function should be evaluated at restoration phase trial points. $\;$ \\
 Setting this option to ``yes'' makes the
restoration phase algorithm evaluate the
objective function of the original problem at
every trial point encountered during the
restoration phase, even if this value is not
required.  In this way, it is guaranteed that the
original objective function can be evaluated
without error at all accepted iterates; otherwise
the algorithm might fail at a point where the
restoration phase accepts an iterate that is good
for the restoration phase problem, but not the
original problem.  On the other hand, if the
evaluation of the original objective is
expensive, this might be costly.
The default value for this string option is ``yes''.
\\ 
Possible values:
\begin{itemize}
   \item no: skip evaluation
   \item yes: evaluate at every trial point
\end{itemize}

\subsubsection{Linear solver}

\paragraph{linear\_solver:} Linear solver used for step computations. $\;$ \\
 Determines which linear algebra package is to be used for the solution of the augmented linear system (for obtaining the search directions).
Note, you need to provide an extra shared library to use MA27, MA57, or PARDISO, see \hyperlink{ipoptlinearsolver}{Section \ref{ipoptlinearsolver}}.
The default value for this string option is ``mumps''.
\\
Possible values:
\begin{itemize}
   \item ma27: use the Harwell routine MA27
   \item ma57: use the Harwell routine MA57
   \item pardiso: use the Pardiso package
%    \item wsmp: use WSMP package
%    \item taucs: use TAUCS package (not yet working)
   \item mumps: use MUMPS package
%    \item custom: use custom linear solver
\end{itemize}

\paragraph{hsl\_library:} Path and filename of HSL library for dynamic load. $\;$ \\
Specify the path to a library that contains HSL routines and can be load via dynamic linking, see also \hyperlink{ipoptlinearsolver}{Section \ref{ipoptlinearsolver}}.

\paragraph{pardiso\_library:} Path and filename of PARDISO library for dynamic load. $\;$ \\
Specify the path to a PARDISO library that and can be load via dynamic linking, see also \hyperlink{ipoptlinearsolver}{Section \ref{ipoptlinearsolver}}.

\paragraph{linear\_system\_scaling:} Method for scaling the linear system. $\;$ \\
 Determines the method used to compute symmetric scaling factors for the augmented system (see also the ``linear\_scaling\_on\_demand'' option).
This scaling is independent of the NLP problem scaling.
By default, MC19 is only used if MA27 or MA57 are selected as linear solvers.
% This option is only available if Ipopt has been compiled with MC19.
The default value for this string option is ``mc19''.
\\
Possible values:
\begin{itemize}
   \item none: no scaling will be performed
   \item mc19: use the Harwell routine MC19
\end{itemize}

\paragraph{linear\_scaling\_on\_demand:} Flag indicating that linear scaling is only done if it seems required. $\;$ \\
 This option is only important if a linear scaling method (e.g., mc19) is used.
If you choose ``no'', then the scaling factors are computed for every linear system from the start.
This can be quite expensive.
Choosing ``yes'' means that the algorithm will start the scaling method only when the solutions to the linear system seem not good, and then use it until the end.
The default value for this string option is ``yes''.
\\
Possible values:
\begin{itemize}
   \item no: Always scale the linear system.
   \item yes: Start using linear system scaling if solutions
seem not good.
\end{itemize}

\paragraph{fast\_step\_computation:} Indicates if the linear system should be solved quickly. $\;$ \\
 If set to yes, the algorithm assumes that the
linear system that is solved to obtain the search
direction, is solved sufficiently well. In that
case, no residuals are computed, and the
computation of the search direction is a little
faster.
The default value for this string option is ``no''.
\\ 
Possible values:
\begin{itemize}
   \item no: Verify solution of linear system by computing
residuals.
   \item yes: Trust that linear systems are solved well.
\end{itemize}

\paragraph{max\_refinement\_steps:} Maximum number of iterative refinement steps per linear system solve. $\;$ \\
 Iterative refinement (on the full unsymmetric
system) is performed for each right hand side. 
This option determines the maximum number of
iterative refinement steps. The valid range for this integer option is
$0 \le {\tt max\_refinement\_steps } <  {\tt +inf}$
and its default value is $10$.


\paragraph{min\_refinement\_steps:} Minimum number of iterative refinement steps per linear system solve. $\;$ \\
 Iterative refinement (on the full unsymmetric
system) is performed for each right hand side. 
This option determines the minimum number of
iterative refinements (i.e. at least
``min\_refinement\_steps'' iterative refinement
steps are enforced per right hand side.) The valid range for this integer option is
$0 \le {\tt min\_refinement\_steps } <  {\tt +inf}$
and its default value is $1$.

\paragraph{residual\_ratio\_max:} Iterative refinement tolerance $\;$ \\
 Iterative refinement is performed until the
residual test ratio is less than this tolerance
(or until the limit ``max\_refinement\_steps'' is hit). The valid range for this real option is 
$0 <  {\tt residual\_ratio\_max } <  {\tt +inf}$
and its default value is $1 \cdot 10^{-10}$.


\paragraph{residual\_ratio\_singular:} Threshold for declaring linear system singular after failed iterative refinement. $\;$ \\
 If the residual test ratio is larger than this
value after failed iterative refinement, the
algorithm pretends that the linear system is
singular. The valid range for this real option is 
$0 <  {\tt residual\_ratio\_singular } <  {\tt +inf}$
and its default value is $1 \cdot 10^{-05}$.


\paragraph{residual\_improvement\_factor:} Minimal required reduction of residual test ratio in iterative refinement. $\;$ \\
 If the improvement of the residual test ratio
made by one iterative refinement step is not
better than this factor, iterative refinement is
aborted. The valid range for this real option is 
$0 <  {\tt residual\_improvement\_factor } <  {\tt +inf}$
and its default value is $1$.


\subsubsection{MUMPS Linear Solver}

\paragraph{mumps\_pivtol:} Pivot tolerance for the linear solver MUMPS. \\
A smaller number pivots for sparsity, a larger number pivots for stability.
The valid range for this real option is
$0 \le {\tt mumps\_pivtol } < {\tt 1}$
and its default value is $1e-6$.

\paragraph{mumps\_pivtolmax:} Maximum pivot tolerance for the linear solver MUMPS. \\
Ipopt may increase pivtol as high as pivtolmax to get a more accurate solution to the linear system.
The valid range for this real option is
$0 \le {\tt mumps\_pivtolmax } < {\tt 1}$
and its default value is $0.1$.

\paragraph{mumps\_mem\_percent:} Percentage increase in the estimated working space for MUMPS. \\
In MUMPS when significant extra fill-in is caused by numerical pivoting, larger values of mumps\_mem\_percent may help use the workspace more efficiently.
The valid range for this integer option is
$0 \le {\tt mumps\_mem\_percent } < {\tt +inf}$
and its default value is $1000$.

\paragraph{mumps\_permuting\_scaling:} Controls permuting and scaling in MUMPS $\;$ \\
 This is ICTL(6) in MUMPS. The valid range for this integer option is
$0 \le {\tt mumps\_permuting\_scaling } \le 7$
and its default value is $7$.


\paragraph{mumps\_pivot\_order:} Controls pivot order in MUMPS $\;$ \\
 This is ICTL(7) in MUMPS. The valid range for this integer option is
$0 \le {\tt mumps\_pivot\_order } \le 7$
and its default value is $7$.


\paragraph{mumps\_scaling:} Controls scaling in MUMPS $\;$ \\
 This is ICTL(8) in MUMPS. The valid range for this integer option is
$-2 \le {\tt mumps\_scaling } \le 7$
and its default value is $7$.


\paragraph{mumps\_dep\_tol:} Pivot threshold for detection of linearly dependent constraints in MUMPS. $\;$ \\
 When MUMPS is used to determine linearly
dependent constraints, this is determines the
threshold for a pivot to be considered zero. 
This is CNTL(3) in MUMPS. The valid range for this real option is 
${\tt -inf} <  {\tt mumps\_dep\_tol } <  {\tt +inf}$
and its default value is $-1$.

\subsubsection{PARDISO Linear Solver}

\paragraph{pardiso\_matching\_strategy:} Matching strategy to be used by Pardiso $\;$ \\
This is IPAR(13) in Pardiso manual.
The default value for this string option is ``complete+2x2''.
\\
Possible values:
\begin{itemize}
   \item complete: Match complete (IPAR(13)=1)
   \item complete+2x2: Match complete+2x2 (IPAR(13)=2)
   \item constraints: Match constraints (IPAR(13)=3)
\end{itemize}

\paragraph{pardiso\_out\_of\_core\_power:} Enables out-of-core variant of Pardiso $\;$ \\
Setting this option to a positive integer $k$ makes Pardiso work in the out-of-core variant where the factor is split in $2^k$ subdomains.
This is IPARM(50) in the Pardiso manual.
The valid range for this integer option is $0 \le {\tt pardiso\_out\_of\_core\_power } <  {\tt +inf}$ and its default value is $0$.

\subsubsection{MA27 Linear Solver}

\paragraph{ma27\_pivtol:} Pivot tolerance for the linear solver MA27. $\;$ \\
A smaller number pivots for sparsity, a larger number pivots for stability.
The valid range for this real option is $0 <  {\tt ma27\_pivtol } <  1$ and its default value is $10^{-8}$.


\paragraph{ma27\_pivtolmax:} Maximum pivot tolerance for the linear solver MA27. $\;$ \\
Ipopt may increase pivtol as high as pivtolmax to get a more accurate solution to the linear
system.
The valid range for this real option is $0 <  {\tt ma27\_pivtolmax } <  1$ and its default value is $0.0001$.


\paragraph{ma27\_liw\_init\_factor:} Integer workspace memory for MA27. $\;$ \\
The initial integer workspace memory = liw\_init\_factor $*$ memory required by unfactored system.
Ipopt will increase the workspace size by meminc\_factor if required.
The default value for this real option is $5$ and its valid range is $1 \le {\tt ma27\_liw\_init\_factor } <  {\tt +inf}$.


\paragraph{ma27\_la\_init\_factor:} Real workspace memory for MA27. $\;$ \\
The initial real workspace memory = la\_init\_factor $*$ memory required by unfactored system. 
Ipopt will increase the workspace size by meminc\_factor if required.
The valid range for this real option is $1 \le {\tt ma27\_la\_init\_factor } <  {\tt +inf}$ and its default value is $5$.


\paragraph{ma27\_meminc\_factor:} Increment factor for workspace size for MA27. $\;$ \\
If the integer or real workspace is not large enough, Ipopt will increase its size by this factor.
The valid range for this real option is $1 \le {\tt ma27\_meminc\_factor } <  {\tt +inf}$ and its default value is $10$.

\subsubsection{MA57 Linear Solver}

\paragraph{ma57\_pivtol:} Pivot tolerance for the linear solver MA57. $\;$ \\
A smaller number pivots for sparsity, a larger number pivots for stability.
The valid range for this real option is $0 <  {\tt ma57\_pivtol } <  1$ and its default value is $10^{-8}$.


\paragraph{ma57\_pivtolmax:} Maximum pivot tolerance for the linear solver MA57. $\;$ \\
Ipopt may increase pivtol as high as ma57\_pivtolmax to get a more accurate solution to the linear system.
The valid range for this real option is $0 <  {\tt ma57\_pivtolmax } <  1$ and its default value is $0.0001$.


\paragraph{ma57\_pre\_alloc:} Safety factor for work space memory allocation for the linear solver MA57. $\;$ \\
If 1 is chosen, the suggested amount of work space is used.
However, choosing a larger number might avoid reallocation if the suggest values do not suffice.
The valid range for this real option is $1 \le {\tt ma57\_pre\_alloc } <  {\tt +inf}$ and its default value is $3$.



\subsubsection{Hessian perturbation}

\paragraph{max\_hessian\_perturbation:} Maximum value of regularization parameter for handling negative curvature. $\;$ \\
 In order to guarantee that the search directions
are indeed proper descent directions, Ipopt
requires that the inertia of the (augmented)
linear system for the step computation has the
correct number of negative and positive
eigenvalues. The idea is that this guides the
algorithm away from maximizers and makes Ipopt
more likely converge to first order optimal
points that are minimizers. If the inertia is not
correct, a multiple of the identity matrix is
added to the Hessian of the Lagrangian in the
augmented system. This parameter gives the
maximum value of the regularization parameter. If
a regularization of that size is not enough, the
algorithm skips this iteration and goes to the
restoration phase. (This is $\delta_w^{\max}$ in the
implementation paper.) The valid range for this real option is 
$0 <  {\tt max\_hessian\_perturbation } <  {\tt +inf}$
and its default value is $1 \cdot 10^{+20}$.


\paragraph{min\_hessian\_perturbation:} Smallest perturbation of the Hessian block. $\;$ \\
 The size of the perturbation of the Hessian block
is never selected smaller than this value, unless
no perturbation is necessary. (This is
$\delta_w^{\min}$ in the implementation paper.) The valid range for this real option is 
$0 \le {\tt min\_hessian\_perturbation } <  {\tt +inf}$
and its default value is $1 \cdot 10^{-20}$.


\paragraph{first\_hessian\_perturbation:} Size of first x-s perturbation tried. $\;$ \\
 The first value tried for the x-s perturbation in
the inertia correction scheme.(This is $\delta_0$
in the implementation paper.) The valid range for this real option is 
$0 <  {\tt first\_hessian\_perturbation } <  {\tt +inf}$
and its default value is $0.0001$.


\paragraph{perturb\_inc\_fact\_first:} Increase factor for x-s perturbation for very first perturbation. $\;$ \\
 The factor by which the perturbation is increased
when a trial value was not sufficient - this
value is used for the computation of the very
first perturbation and allows a different value
for for the first perturbation than that used for
the remaining perturbations. (This is
$\bar\kappa_w^+$ in the implementation paper.) The valid range for this real option is 
$1 <  {\tt perturb\_inc\_fact\_first } <  {\tt +inf}$
and its default value is $100$.


\paragraph{perturb\_inc\_fact:} Increase factor for x-s perturbation. $\;$ \\
 The factor by which the perturbation is increased
when a trial value was not sufficient - this
value is used for the computation of all
perturbations except for the first. (This is
$\kappa_w^+$ in the implementation paper.) The valid range for this real option is 
$1 <  {\tt perturb\_inc\_fact } <  {\tt +inf}$
and its default value is $8$.


\paragraph{perturb\_dec\_fact:} Decrease factor for x-s perturbation. $\;$ \\
 The factor by which the perturbation is decreased
when a trial value is deduced from the size of
the most recent successful perturbation. (This is
$\kappa_w^-$ in the implementation paper.) The valid range for this real option is 
$0 <  {\tt perturb\_dec\_fact } <  1$
and its default value is $0.333333$.


\paragraph{jacobian\_regularization\_value:} Size of the regularization for rank-deficient constraint Jacobians. $\;$ \\
 (This is $\bar\delta_c$ in the implementation
paper.) The valid range for this real option is\\ 
$0 \le {\tt jacobian\_regularization\_value } <  {\tt +inf}$
and its default value is $1 \cdot 10^{-08}$.


\paragraph{jacobian\_regularization\_exponent:} Exponent for mu in the regularization for rank-deficient constraint Jacobians. $\;$ \\
 (This is $\kappa_c$ in the implementation paper.) The default value for this real option is $0.25$
and its valid range is $0 \le {\tt jacobian\_regularization\_exponent } <  {\tt +inf}$.


\paragraph{perturb\_always\_cd:} Active permanent perturbation of constraint linearization. $\;$ \\
 This options makes the delta\_c and delta\_d
perturbation be used for the computation of every
search direction.  Usually, it is only used when
the iteration matrix is singular.
The default value for this string option is ``no''.
\\ 
Possible values:
\begin{itemize}
   \item no: perturbation only used when required
   \item yes: always use perturbation
\end{itemize}

\subsubsection{Hessian approximation}

\paragraph{hessian\_approximation:} Indicates what Hessian information is to be used. $\;$ \\
 This determines which kind of information for the
Hessian of the Lagrangian function is used by the
algorithm.
The default value for this string option is to use ``exact'' if the GAMS system is able to provide a hessian, and ``limited-memory'' otherwise (a warning is issued in this case).
\\ 
Possible values:
\begin{itemize}
   \item exact: Use second derivatives provided by the NLP.
   \item limited-memory: Perform a limited-memory quasi-Newton
approximation
\end{itemize}

\paragraph{hessian\_approximation\_space:} Indicates in which subspace the Hessian information is to be approximated. \\
The default value for this string option is ``nonlinear-variables''.
\\ 
Possible values:
\begin{itemize}
   \item nonlinear-variables: only in space of nonlinear variables.
   \item all-variables: in space of all variables (without slacks)
\end{itemize}

\paragraph{limited\_memory\_max\_history:} Maximum size of the history for the limited quasi-Newton Hessian approximation. $\;$ \\
 This option determines the number of most recent
iterations that are taken into account for the
limited-memory quasi-Newton approximation. The valid range for this integer option is
$0 \le {\tt limited\_memory\_max\_history } <  {\tt +inf}$
and its default value is $6$.


\paragraph{limited\_memory\_update\_type:} Quasi-Newton update formula for the limited memory approximation. $\;$ \\
 Determines which update formula is to be used for
the limited-memory quasi-Newton approximation.
The default value for this string option is ``bfgs''.
\\ 
Possible values:
\begin{itemize}
   \item bfgs: BFGS update (with skipping)
   \item sr1: SR1 (not working well)
\end{itemize}

\paragraph{limited\_memory\_initialization:} Initialization strategy for the limited memory quasi-Newton approximation. $\;$ \\
 Determines how the diagonal Matrix B\_0 as the
first term in the limited memory approximation
should be computed.
The default value for this string option is ``scalar1''.
\\ 
Possible values:
\begin{itemize}
   \item scalar1: sigma = $s^Ty/s^Ts$
   \item scalar2: sigma = $y^Ty/s^Ty$
   \item constant: sigma = limited\_memory\_init\_val
\end{itemize}

\paragraph{limited\_memory\_init\_val:} Value for B0 in low-rank update. $\;$ \\
 The starting matrix in the low rank update, B0,
is chosen to be this multiple of the identity in
the first iteration (when no updates have been
performed yet), and remains constant at this
value, if ``limited\_memory\_initialization'' is
``constant''. The valid range for this real option is 
$0 <  {\tt limited\_memory\_init\_val } <  {\tt +inf}$
and its default value is $1$.


\paragraph{limited\_memory\_max\_skipping:} Threshold for successive iterations where update is skipped. $\;$ \\
 If the update is skipped more than this number of
successive iterations, we quasi-Newton
approximation is reset. The valid range for this integer option is
$1 \le {\tt limited\_memory\_max\_skipping } <  {\tt +inf}$
and its default value is $2$.


% \paragraph{derivative\_test:} Enable derivative checker $\;$ \\
%  If this option is enabled, a (slow) derivative
% test will be performed before the optimization. 
% The test is performed at the user provided
% starting point and marks derivative values that
% seem suspicious.
% The default value for this string option is ``none''.
% \\ 
% Possible values:
% \begin{itemize}
%    \item none: do not perform derivative test
%    \item first-order: perform test of first derivatives at starting
% point
%    \item second-order: perform test of first and second derivatives at
% starting point
% \end{itemize}
% 
% \paragraph{derivative\_test\_perturbation:} Size of the finite difference perturbation in derivative test. $\;$ \\
%  This determines the relative perturbation of the
% variable entries. The valid range for this real option is 
% $0 <  {\tt derivative\_test\_perturbation } <  {\tt +inf}$
% and its default value is $1 \cdot 10^{-08}$.
% 
% 
% \paragraph{derivative\_test\_tol:} Threshold for indicating wrong derivative. $\;$ \\
%  If the relative deviation of the estimated
% derivative from the given one is larger than this
% value, the corresponding derivative is marked as
% wrong. The valid range for this real option is 
% $0 <  {\tt derivative\_test\_tol } <  {\tt +inf}$
% and its default value is $0.0001$.
% 
% 
% \paragraph{derivative\_test\_print\_all:} Indicates whether information for all estimated derivatives should be printed. $\;$ \\
%  Determines verbosity of derivative checker.
% The default value for this string option is ``no''.
% \\ 
% Possible values:
% \begin{itemize}
%    \item no: Print only suspect derivatives
%    \item yes: Print all derivatives
% \end{itemize}
% 


\bibliographystyle{plain}
%\bibliography{coinlibd}
%\renewcommand{\bibname}{Ipopt References}
\chapter{\IPOPT and \IPOPTH}
\label{cha:ipopt}

%\minitoc

COIN-OR \IPOPT (\textbf{I}nterior \textbf{P}oint \textbf{Opt}imizer) is an open-source solver for large-scale nonlinear programming.
The code has been written primarily by Andreas W\"achter, who is the COIN-OR project leader for \IPOPT.

\IPOPT implements an interior point line search filter method for nonlinear programming models which functions can be nonconvex, but should be twice continuously differentiable.
For more information on the algorithm we refer to~\cite{NoWaWa08,Waechter2002,WaBi05b,WaBi05a,WaBi2006} and the \IPOPT web site \url{https://projects.coin-or.org/Ipopt}.
Most of the \IPOPT documentation in the section was taken from the \IPOPT manual~\cite{IpoptManual}.



\section{The linear solver in \IPOPT}
\label{sec:ipoptlinearsolver}
\hypertarget{ipoptlinearsolver}{}

The performance and robustness of \IPOPT on larger models heavily relies on the used solver for sparse symmetric indefinite linear systems.

\GAMS/\IPOPT includes the sparse solver \textsc{MUMPS}~\cite{AmestoyDuffKosterLExcellent2001,AmestoyGuermoucheLExcellentPralet2006} (currently the default), cf.~\url{http://graal.ens-lyon.fr/MUMPS} and \textsc{MKL PARDISO}~\cite{SchGa04,SchGa06} (only Linux and Windows).
In the commerically licensed \GAMS/\IPOPTH version, also the Harwell Subroutine Library (HSL) solvers \textsc{MA27}, \textsc{MA57}, \textsc{HSL\_MA86}, and \textsc{HSL\_MA97} are available and MA27 is used by default.

\textsc{MUMPS}, \textsc{MA57}, \textsc{HSL\_MA86}, and \textsc{HSL\_MA97} use \textsc{METIS} for matrix ordering \cite{KaKu99}, cf.~\url{http://glaros.dtc.umn.edu/gkhome/views/metis/index.html} and \url{http://glaros.dtc.umn.edu/gkhome/fetch/sw/metis/manual.pdf}.
\textsc{METIS} is copyrighted by the regents of the University of Minnesota.

\IPOPT and \IPOPTH can exploit parallelization of the linear solver or the linear algebra routines (Blas and Lapack).
The following table summarizes which options are available on which platform.

\begin{tabular}{l|c|cc|cccc}
& \multicolumn{3}{c|}{\IPOPT and \IPOPTH} & \multicolumn{4}{c}{\IPOPTH only} \\
        & Linear Algebra & MUMPS & MKL PARDISO & MA27 & MA57 & HSL MA86 & HSL MA97 \\ \hline
Linux   & parallel & serial & parallel      & serial & serial & parallel & parallel \\
MacOS X & parallel & serial & not available & serial & serial & parallel & parallel  \\
Solaris & serial   & serial & not available & serial & serial & parallel & parallel  \\
Windows & parallel & serial & parallel      & serial & serial & parallel & parallel  \\
\end{tabular}

The linear solver is chosen by the \texttt{linear\_solver} option.
Benchmarks have shown that \textsc{MA57} and \textsc{HSL\_MA97} are often able to outperform \textsc{MA27} on larger instances. Further, \textsc{PARDISO} often allows for performance that is better than \textsc{MUMPS} and similar to the HSL solvers. If \IPOPT fails to solve an instance with \textsc{PARDISO}, it's worth to try changing the options \texttt{pardiso\_order} and \texttt{pardiso\_max\_iterative\_refinement\_steps}.

\section{Usage}

The following statement can be used inside your \GAMS program to specify using \IPOPT
\begin{verbatim}
  Option NLP = IPOPT;     { or LP, RMIP, DNLP, RMINLP, QCP, RMIQCP }
\end{verbatim}

The above statement should appear before the Solve statement.
If \IPOPT was specified as the default solver during \GAMS installation, the above statement is not necessary.

To use \IPOPTH, the statement should be
\begin{verbatim}
  Option NLP = IPOPTH;    { or LP, RMIP, DNLP, RMINLP, QCP, RMIQCP }
\end{verbatim}



\paragraph{Using Harwell Subroutine Library routines with \GAMS/\IPOPT.}

\GAMS/\IPOPT can use the HSL routines \texttt{MA27}, \texttt{MA28}, \texttt{MA57}, \textsc{HSL\_MA77}, \textsc{HSL\_MA86}, \textsc{HSL\_MA97}, \texttt{MC19}, and \textsc{HSL\_MC68} when provided as shared library.
By telling \IPOPT to use one of these routines (see options \texttt{linear\_solver}, \texttt{linear\_system\_scaling}, \texttt{nlp\_scaling\_method}, \texttt{dependency\_detector}), \GAMS/\IPOPT attempts to load the required routines from the library \texttt{libhsl.so} (Unix-Systems), \texttt{libhsl.dylib} (MacOS X), or \texttt{libhsl.dll} (Windows), respectively.

The HSL routines are available at \url{http://www.hsl.rl.ac.uk/ipopt}.
Note that it is your responsibility to ensure that you are entitled to download and use these routines!
% You can build a shared library using the ThirdParty/HSL project at COIN-OR.

\paragraph{Using PARDISO with \GAMS/\IPOPT or \GAMS/\IPOPTH.}
On Mac OS X and Solaris, setting the option \texttt{linear\_solver} to \texttt{pardiso} lets \GAMS/\IPOPT or \GAMS/\IPOPTH try to load the linear solver PARDISO from the library \texttt{libpardiso.so} (Unix) or \texttt{libpardiso.dylib} (MacOS X), respectively.

PARDISO is available as compiled shared library for several platforms at \texttt{http://www.pardiso-project.org}.
Note that it is your responsibility to ensure that you are entitled to download and use this package!

\subsection{Specification of Options}
\label{sub:ipoptoptionspec}

\IPOPT has many options that can be adjusted for the algorithm (see Section \ref{sub:ipoptoptions}).
Options are all identified by a string name, and their values can be of one of three types: Number (real), Integer, or String.
Number options are used for things like tolerances, integer options are used for things like maximum number of iterations, and string options are used for setting algorithm details, like the NLP scaling method.
Options can be set by creating a \texttt{ipopt.opt} file in the directory you are executing \IPOPT.

The \texttt{ipopt.opt} file is read line by line and each line should contain the option name, followed by whitespace, and then the value.
Comments can be included with the \# symbol. Don't forget to ensure you have a newline at the end of the file. For example,
\begin{verbatim}
# This is a comment

# Turn off the NLP scaling
nlp_scaling_method none

# Change the initial barrier parameter
mu_init 1e-2

# Set the max number of iterations
max_iter 500
\end{verbatim}
is a valid \texttt{ipopt.opt} file.

% You can print the documentation for all \IPOPT options by using the option
% \begin{verbatim}
% print_options_documentation yes
% \end{verbatim}
% and running \IPOPT.
% This will output all of the options documentation to the console.

\GAMS/\IPOPT understand currently the following \GAMS parameters: \texttt{reslim} (time limit), \texttt{iterlim} (iteration limit), \texttt{domlim} (domain violation limit).
You can set them either on the command line, e.g. \verb+iterlim=500+, or inside your \GAMS program, e.g. \verb+Option iterlim=500;+.
Further the option \texttt{threads} can be used to control the number of threads used in the linear algebra routines and the linear solver, see also Section~\ref{sec:ipoptlinearsolver}.

\subsection{Warmstarting Ipopt}

As an interior point solver, it is difficult to warm start \IPOPT.
By default, only the level values of the variables are passed as starting point to \IPOPT.
Setting the \IPOPT option \texttt{warm\_start\_init\_point} to \texttt{yes} enables that also dual values for variables and constraints are passed to \IPOPT.

However, the expected behavior that \IPOPT finishes within one iteration if optimal primal and dual values are passed is not reached this way, yet. This is, because \IPOPT by default moves any initial value that is close to a bound into the interior. The amount on how much the initial point is moved can be controlled by various \texttt{bound\_push} and \texttt{bound\_frac} options.
To make \IPOPT accept an optimal primal/dual solution within one iteration, it should be sufficient to set the following options:
\begin{verbatim}
  warm_start_init_point       yes
  warm_start_bound_push       1e-9
  warm_start_bound_frac       1e-9
  warm_start_slack_bound_frac 1e-9
  warm_start_slack_bound_push 1e-9
  warm_start_mult_bound_push  1e-9
\end{verbatim}

\section{Output}

This section describes the standard \IPOPT console output.
The output is designed to provide a quick summary of each iteration as \IPOPT solves the problem.

Before \IPOPT starts to solve the problem, it displays the problem statistics (number of nonzero-elements in the matrices, number of variables, etc.).
Note that if you have fixed variables (both upper and lower bounds are equal), \IPOPT may remove these variables from the problem internally and not include them in the problem statistics.

Following the problem statistics, \IPOPT will begin to solve the problem and you will see output resembling the following,
\begin{verbatim}
iter    objective    inf_pr   inf_du lg(mu)  ||d||  lg(rg) alpha_du alpha_pr  ls
   0  1.6109693e+01 1.12e+01 5.28e-01   0.0 0.00e+00    -  0.00e+00 0.00e+00   0
   1  1.8029749e+01 9.90e-01 6.62e+01   0.1 2.05e+00    -  2.14e-01 1.00e+00f  1
   2  1.8719906e+01 1.25e-02 9.04e+00  -2.2 5.94e-02   2.0 8.04e-01 1.00e+00h  1
\end{verbatim}
and the columns of output are defined as
\begin{description}
\item[iter]
The current iteration count.
This includes regular iterations and iterations while in restoration phase.
If the algorithm is in the restoration phase, the letter \texttt{r} will be appended to the iteration number.
\item[objective]
The unscaled objective value at the current point.
During the restoration phase, this value remains the unscaled objective value for the original problem.
\item[inf\_pr]
The unscaled constraint violation at the current point.
This quantity is the infinity-norm (max) of the (unscaled) constraint violation.
During the restoration phase, this value remains the constraint violation of the original problem at the current point.
The option ``\texttt{inf\_pr\_output}'' can be used to switch to the printing of a different quantity.
During the restoration phase, this value is the primal infeasibility of the original problem at the current point.
\item[inf\_du]
The scaled dual infeasibility at the current point.
This quantity measure the infinity-norm (max) of the internal dual infeasibility \cite[Eq.~(4a)]{WaBi2006}, including inequality constraints reformulated using slack variables and problem scaling.
During the restoration phase, this is the value of the dual infeasibility for the restoration phase problem.
\item[lg(mu)]
$\log_{10}$ of the value of the barrier parameter $\mu$.
\item[$\Vert d\Vert$]
The infinity norm (max) of the primal step (for the original variables $x$ and the internal slack variables $s$).
During the restoration phase, this value includes the values of additional variables, $p$ and $n$ \cite[Eq.~(10)]{WaBi2006}.
\item[lg(rg)]
$\log_{10}$ of the value of the regularization term for the Hessian of the Lagrangian in the augmented system ($\delta_w$ in \cite[Eq.~(26)]{WaBi2006}).
A dash (``\texttt{-}'') indicates that no regularization was done.
\item[alpha\_du]
The stepsize for the dual variables ($\alpha^z_k$ in \cite[Eq.~(14c)]{WaBi2006})..
\item[alpha\_pr]
The stepsize for the primal variables ($\alpha_k$ in \cite[Eq.~(14a)]{WaBi2006}).
The number is usually followed by a character for additional diagnostic information regarding the step acceptance criterion:
 \begin{list}{blub}{\itemsep0pt}
    \item[\texttt{f}] f-type iteration in the filter method w/o second order correction
    \item[\texttt{F}] f-type iteration in the filter method w/ second order correction
    \item[\texttt{h}] h-type iteration in the filter method w/o second order correction
    \item[\texttt{H}] h-type iteration in the filter method w/ second order correction
    \item[\texttt{k}] penalty value unchanged in merit function method w/o second order correction
    \item[\texttt{K}] penalty value unchanged in merit function method w/ second order correction
    \item[\texttt{n}] penalty value updated in merit function method w/o second order correction
    \item[\texttt{N}] penalty value updated in merit function method w/ second order correction
    \item[\texttt{R}] Restoration phase just started
    \item[\texttt{w}] in watchdog procedure
    \item[\texttt{s}] step accepted in soft restoration phase
    \item[\texttt{t}/\texttt{T}] tiny step accepted without line search
    \item[\texttt{r}] some previous iterate restored
 \end{list}
\item[ls]
The number of backtracking line search steps (does not include second-order correction steps).
\end{description}

Note that the step acceptance mechanisms in \IPOPT consider the
barrier objective function \cite[Eq.~(3a)]{WaBi2006} which is
usually different from the value reported in the \texttt{objective}
column.  Similarly, for the purposes of the step acceptance, the
constraint violation is measured for the internal problem formulation,
which includes slack variables for inequality constraints and
potentially scaling of the constraint functions.  This value, too, is
usually different from the value reported in \texttt{inf\_pr}.  As a
consequence, a new iterate might have worse values both for the
objective function and the constraint violation as reported in the
iteration output, seemingly contradicting globalization procedure.


When the algorithm terminates, \IPOPT will output a message to the screen.
The following is a list of the possible output messages and a brief description.

\begin{description}
\item[Optimal Solution Found.] ~

    This message indicates that \IPOPT found a (locally) optimal point within the desired tolerances.

\item[Solved To Acceptable Level.] ~

    This indicates that the algorithm did not converge to the ``desired'' tolerances, but that it was able to obtain a point satisfying the ``acceptable'' tolerance level as specified by \texttt{acceptable-*} options.
    This may happen if the desired tolerances are too small for the current problem.

\item[Feasible point for square problem found.] ~

    This message is printed if the problem is ``square'' (i.e., it has as many equality constraints as free variables) and \IPOPT found a feasible point.

\item[Converged to a point of local infeasibility. Problem may be infeasible.] ~

    The restoration phase converged to a point that is a minimizer for the constraint violation (in the $\ell_1$-norm), but is not feasible for the original problem.
    This indicates that the problem may be infeasible (or at least that the algorithm is stuck at a locally infeasible point).
    The returned point (the minimizer of the constraint violation) might help you to find which constraint is causing the problem.
    If you believe that the NLP is feasible, it might help to start the optimization from a different point.

\item[Search Direction is becoming Too Small.] ~

    This indicates that \IPOPT is calculating very small step sizes and making very little progress.
    This could happen if the problem has been solved to the best numerical accuracy possible given the current scaling.

\item[Iterates divering; problem might be unbounded.] ~

    This message is printed if the max-norm of the iterates becomes larger than the value of the option \texttt{diverging\_iterates\_tol}.
    This can happen if the problem is unbounded below and the iterates are diverging.

\item[Stopping optimization at current point as requested by user.] ~

    This message is printed if either the Ctrl+C was pressed or the domain violation limit is reached.

\item[Maximum Number of Iterations Exceeded.] ~

    This indicates that \IPOPT has exceeded the maximum number of iterations as specified by the \IPOPT option \texttt{max\_iter} or the GAMS option \texttt{iterlim}.

\item[Maximum CPU time exceeded.] ~

    This indicates that \IPOPT has exceeded the maximum number of seconds as specified by the \IPOPT option \texttt{max\_cpu\_time} or the GAMS option \texttt{reslim}.

\item[Restoration Failed!] ~

    This indicates that the restoration phase failed to find a feasible point that was acceptable to the filter line search for the original problem.
    This could happen if the problem is highly degenerate or does not satisfy the constraint qualification, or if an external function in \GAMS provides incorrect derivative information.

\item[Error in step computation (regularization becomes too large?)!] ~

    This messages is printed if \IPOPT is unable to compute a search direction, despite several attempts to modify the iteration matrix.
    Usually, the value of the regularization parameter then becomes too large.

\item[Problem has too few degrees of freedom.] ~

    This indicates that your problem, as specified, has too few degrees of freedom.
    This can happen if you have too many equality constraints, or if you fix too many variables (\IPOPT removes fixed variables).

\item[Not enough memory.] ~

    An error occurred while trying to allocate memory.
    The problem may be too large for your current memory and swap configuration.

\item[INTERNAL ERROR: Unknown SolverReturn value - Notify \IPOPT Authors.] ~

    An unknown internal error has occurred. Please notify the authors of the \GAMS/\IPOPT link or \IPOPT (refer to \url{https://projects.coin-or.org/GAMSlinks} or \url{https://projects.coin-or.org/Ipopt}).
\end{description}


\subsection{Diagnostic Tags for \IPOPT}

To print additional diagnostic tags for each iteration of \IPOPT, set
the options \texttt{print\_info\_string} to \texttt{yes}. With
this, a tag will appear at the end of an iteration line with the
following diagnostic meaning that are useful to flag difficulties for
a particular \IPOPT run.  The following is a list of possible strings:
\begin{list}{blub}{\itemsep0pt}
 \item[\texttt{!}] Tighten resto tolerance if only slightly infeasible \cite[Sec.~3.3]{WaBi2006}
 \item[\texttt{A}] Current iteration is acceptable (alternate termination)
 \item[\texttt{a}] Perturbation for PD Singularity can't be done, assume singular \cite[Sec.~3.1]{WaBi2006}
 \item[\texttt{C}] Second Order Correction taken \cite[Sec.~2.4]{WaBi2006}
 \item[\texttt{Dh}] Hessian degenerate based on multiple iterations \cite[Sec.~3.1]{WaBi2006}
 \item[\texttt{Dhj}] Hessian/Jacobian degenerate based on multiple iterations \cite[Sec.~3.1]{WaBi2006}
 \item[\texttt{Dj}] Jacobian degenerate based on multiple iterations \cite[Sec.~3.1]{WaBi2006}
 \item[\texttt{dx}] $\delta_x$ perturbation too large \cite[Sec.~3.1]{WaBi2006}
 \item[\texttt{e}] Cutting back $\alpha$ due to evaluation error (in backtracking line search)
 \item[\texttt{F-}] Filter should be reset, but maximal resets exceeded \cite[Sec.~2.3]{WaBi2006}
 \item[\texttt{F+}] Resetting filter due to last few rejections of filter \cite[Sec.~2.3]{WaBi2006}
 \item[\texttt{L}] Degenerate Jacobian, $\delta_c$ already perturbed \cite[Sec.~3.1]{WaBi2006}
 \item[\texttt{l}] Degenerate Jacobian, $\delta_c$ perturbed \cite[Sec.~3.1]{WaBi2006}
 \item[\texttt{M}] Magic step taken for slack variables (in backtracking line search)
 \item[\texttt{Nh}] Hessian not yet degenerate \cite[Sec.~3.1]{WaBi2006}
 \item[\texttt{Nhj}] Hessian/Jacobian not yet degenerate \cite[Sec.~3.1]{WaBi2006}
 \item[\texttt{Nj}] Jacobian not yet degenerate \cite[Sec.~3.1]{WaBi2006}
 \item[\texttt{NW}] Warm start initialization failed (in Warm Start Initialization)
 \item[\texttt{q}] PD system possibly singular, attempt to improve solution quality \cite[Sec.~3.1]{WaBi2006}
 \item[\texttt{R}] Solution of restoration phase \cite[Sec.~3.3]{WaBi2006}
 \item[\texttt{S}] PD system possibly singular, accept current solution \cite[Sec.~3.1]{WaBi2006}
 \item[\texttt{s}] PD system singular \cite[Sec.~3.1]{WaBi2006}
 \item[\texttt{s}] Square Problem. Set multipliers to zero (default initialization routine)
 \item[\texttt{Tmax}] Trial $\theta$ is larger than $\theta_{max}$ (filter parameter \cite[Eq.~(21)]{WaBi2006})
 \item[\texttt{W}] Watchdog line search procedure successful \cite[Sec.~3.2]{WaBi2006}
 \item[\texttt{w}] Watchdog line search procedure unsuccessful, stopped \cite[Sec.~3.2]{WaBi2006}
 \item[\texttt{Wb}] Undoing most recent SR1 update \cite[Sec.~5.4.1]{Biegler2010}
 \item[\texttt{We}] Skip Limited-Memory Update in restoration phase  \cite[Sec.~5.4.1]{Biegler2010}
 \item[\texttt{Wp}] Safeguard $B^0 = \sigma I$ for  Limited-Memory Update \cite[Sec.~5.4.1]{Biegler2010}
 \item[\texttt{Wr}] Resetting Limited-Memory Update \cite[Sec.~5.4.1]{Biegler2010}
 \item[\texttt{Ws}] Skip Limited-Memory Update since $s^Ty$ is not positive \cite[Sec.~5.4.1]{Biegler2010}
 \item[\texttt{WS}] Skip Limited-Memory Update since $\Delta x$ is too small \cite[Sec.~5.4.1]{Biegler2010}
 \item[\texttt{y}] Dual infeasibility, use least square multiplier update (during \IPOPT algorithm)
 \item[\texttt{z}] Apply correction to bound multiplier if too large (during \IPOPT algorithm)
\end{list}

\section{Detailed Options Description}
\label{sub:ipoptoptions}

% Note, that \GAMS/\IPOPT overwrites the \IPOPT default setting for the parameters \texttt{bound\_relax\_factor} (set to $10^{-10}$) and \texttt{mu\_strategy} (set to \texttt{adaptive}).
% You can change these values by specifying these options in your \IPOPT options file.

\subsubsection{Output}

\paragraph{print\_level:} Output verbosity level. $\;$ \\
 Sets the default verbosity level for console
output. The larger this value the more detailed
is the output. The valid range for this integer option is
$0 \le {\tt print\_level } \le 11$
and its default value is $4$.


\paragraph{print\_user\_options:} Print all options set by the user. $\;$ \\
 If selected, the algorithm will print the list of
all options set by the user including their
values and whether they have been used.
The default value for this string option is ``no''.
\\ 
Possible values:
\begin{itemize}
   \item no: don't print options
   \item yes: print options
\end{itemize}

\paragraph{print\_options\_documentation:} Switch to print all algorithmic options. $\;$ \\
 If selected, the algorithm will print the list of
all available algorithmic options with some
documentation before solving the optimization
problem.
The default value for this string option is ``no''.
\\ 
Possible values:
\begin{itemize}
   \item no: don't print list
   \item yes: print list
\end{itemize}

\paragraph{output\_file:} File name of desired output file (leave unset for no file output). $\;$ \\
An output file with this
name will be written (leave unset for no file
output).  The verbosity level is by default set
to ``print\_level'', but can be overridden with
``file\_print\_level''.  The file name is changed
to use only small letters.
With the default settings no output file is generated.
\\ 
Possible values:
\begin{itemize}
   \item *: Any acceptable standard file name
\end{itemize}

\paragraph{file\_print\_level:} Verbosity level for output file. $\;$ \\
 NOTE: This option only works when read from the
ipopt.opt options file! Determines the verbosity
level for the file specified by ``output\_file''.
By default it is the same as ``print\_level''. The valid range for this integer option is
$0 \le {\tt file\_print\_level } \le 11$
and its default value is $4$.

\subsubsection{Termination}

\paragraph{tol:} Desired convergence tolerance (relative). $\;$ \\
 Determines the convergence tolerance for the
algorithm.  The algorithm terminates
successfully, if the (scaled) NLP error becomes
smaller than this value, and if the (absolute)
criteria according to ``dual\_inf\_tol'',
``primal\_inf\_tol'', and ``cmpl\_inf\_tol'' are met.
 (This is $\varepsilon_\mathrm{tol}$ in Eqn. (6) in the
implementation paper).  See also
``acceptable\_tol'' as a second termination
criterion.  Note, some other algorithmic features
also use this quantity to determine thresholds
etc. The valid range for this real option is 
$0 <  {\tt tol } <  {\tt +inf}$
and its default value is $1 \cdot 10^{-08}$.

\paragraph{s\_max:} Scaling threshold for the NLP error. $\;$ \\
The valid range for this integer option is
$0 \le {\tt s\_max } <  {\tt +inf}$
and its default value is $100$.

\paragraph{max\_iter:} Maximum number of iterations. $\;$ \\
 The algorithm terminates with an error message if
the number of iterations exceeded this number. The valid range for this integer option is
$0 \le {\tt max\_iter } <  {\tt +inf}$
and its default value is the value of the GAMS parameter iterlim, which default value is $10000$.


\paragraph{compl\_inf\_tol:} Desired threshold for the complementarity conditions. $\;$ \\
 Absolute tolerance on the complementarity.
Successful termination requires that the max-norm
of the (unscaled) complementarity is less than
this threshold. The valid range for this real option is 
$0 <  {\tt compl\_inf\_tol } <  {\tt +inf}$
and its default value is $0.0001$.


\paragraph{constr\_viol\_tol:} Desired threshold for the constraint violation. $\;$ \\
 Absolute tolerance on the constraint violation.
Successful termination requires that the max-norm
of the (unscaled) constraint violation is less
than this threshold. The default value for this real option is $0.0001$ and its
valid range is $0<\texttt{constr\_viol\_tol}<{\tt +inf}$.


\paragraph{dual\_inf\_tol:} Desired threshold for the dual infeasibility. $\;$ \\
 Absolute tolerance on the dual infeasibility.
Successful termination requires that the max-norm
of the (unscaled) dual infeasibility is less than
this threshold. The valid range for this real option is 
$0 <  {\tt dual\_inf\_tol } <  {\tt +inf}$
and its default value is $0.0001$.


\paragraph{acceptable\_tol:} ``Acceptable'' convergence tolerance (relative). $\;$ \\
 Determines which (scaled) overall optimality
error is considered to be ``acceptable''. There are
two levels of termination criteria.  If the usual
``desired'' tolerances (see tol, dual\_inf\_tol
etc) are satisfied at an iteration, the algorithm
immediately terminates with a success message. 
On the other hand, if the algorithm encounters
``acceptable\_iter'' many iterations in a row that
are considered ``acceptable'', it will terminate
before the desired convergence tolerance is met.
This is useful in cases where the algorithm might
not be able to achieve the ``desired'' level of
accuracy. The valid range for this real option is 
$0 <  {\tt acceptable\_tol } <  {\tt +inf}$
and its default value is $1 \cdot 10^{-06}$.

\paragraph{acceptable\_iter:} Number of ``acceptable'' iterates before triggering termination. $\;$ \\
If the algorithm encounters this many successive ``acceptable'' iterates (see ``acceptable\_tol''), it terminates, assuming that the problem has been solved to best possible accuracy given round-off.
If it is set to zero, this heuristic is disabled.
The valid range for this integer option is
$0 \le {\tt acceptable\_iter } <  {\tt +inf}$
and its default value is $15$.

\paragraph{acceptable\_compl\_inf\_tol:} ``Acceptance'' threshold for the complementarity conditions. $\;$ \\
 Absolute tolerance on the complementarity.
``Acceptable'' termination requires that the
max-norm of the (unscaled) complementarity is
less than this threshold; see also
acceptable\_tol. The valid range for this real option is 
$0 <  {\tt acceptable\_compl\_inf\_tol } <  {\tt +inf}$
and its default value is $0.01$.


\paragraph{acceptable\_constr\_viol\_tol:} ``Acceptance'' threshold for the constraint violation. $\;$ \\
 Absolute tolerance on the constraint violation.
``Acceptable'' termination requires that the
max-norm of the (unscaled) constraint violation
is less than this threshold; see also
acceptable\_tol. The valid range for this real option is 
$0 <  {\tt acceptable\_constr\_viol\_tol } <  {\tt +inf}$
and its default value is $0.01$.


\paragraph{acceptable\_dual\_inf\_tol:} ``Acceptance'' threshold for the dual infeasibility. $\;$ \\
 Absolute tolerance on the dual infeasibility.
``Acceptable'' termination requires that the
(max-norm of the unscaled) dual infeasibility is
less than this threshold; see also
acceptable\_tol. The valid range for this real option is 
$0 <  {\tt acceptable\_dual\_inf\_tol } <  {\tt +inf}$
and its default value is $0.01$.


\paragraph{diverging\_iterates\_tol:} Threshold for maximal value of primal iterates. $\;$ \\
 If any component of the primal iterates exceeded
this value (in absolute terms), the optimization
is aborted with the exit message that the
iterates seem to be diverging. The valid range for this real option is 
$0 <  {\tt diverging\_iterates\_tol } <  {\tt +inf}$
and its default value is $1 \cdot 10^{+20}$.

\subsubsection{NLP Scaling}

\paragraph{obj\_scaling\_factor:} Scaling factor for the objective function. $\;$ \\
 This option sets a scaling factor for the
objective function. The scaling is seen
internally by Ipopt but the unscaled objective is
reported in the console output. If additional
scaling parameters are computed (e.g.
user-scaling or gradient-based), both factors are
multiplied. The valid range for this real option is 
${\tt -inf} <  {\tt obj\_scaling\_factor } <  {\tt +inf}$
and its default value is $1$.


\paragraph{nlp\_scaling\_method:} Select the technique used for scaling the NLP. $\;$ \\
 Selects the technique used for scaling the
problem internally before it is solved. For
user-scaling, the parameters come from the values of the .scale suffix in GAMS.
The default value for this string option is ``gradient-based'' if scaleopt is 0 (default).
If the user provides variable or equation scaling values in GAMS and sets $<$model$>$.scaleopt to 1, then the default for this parameter is ``user-scaling''.
\\ 
Possible values:
\begin{itemize}
   \item none: no problem scaling will be performed
   \item user-scaling: scaling parameters will come from the user
   \item gradient-based: scale the problem so the maximum gradient at
the starting point is scaling\_max\_gradient
   \item equilibration-based: scale the problem so that first derivatives are
of order 1 at random points (only available with MC19)
\end{itemize}

\paragraph{nlp\_scaling\_max\_gradient:} Maximum gradient after NLP scaling. $\;$ \\
 This is the gradient scaling cut-off. If the
maximum gradient is above this value, then
gradient based scaling will be performed. Scaling
parameters are calculated to scale the maximum
gradient back to this value. (This is $g_{\max}$ in
Section 3.8 of the implementation paper.) Note:
This option is only used if
``nlp\_scaling\_method'' is chosen as
``gradient-based''. The valid range for this real option is 
$0 <  {\tt nlp\_scaling\_max\_gradient } <  {\tt +inf}$
and its default value is $100$.

\paragraph{nlp\_scaling\_obj\_target\_gradient:} Target value for objective function gradient size. $\;$ \\
     If a positive number is chosen, the scaling factor the objective function
     is computed so that the gradient as the max norm of the given size at the
     starting point.  This overrides nlp\_scaling\_max\_gradient for the
     objective function.
The valid range for this real option is 
$0 <  {\tt nlp\_scaling\_obj\_target\_gradient } <  {\tt +inf}$
and its default value is $0$.

\subsubsection{NLP corrections}

\paragraph{dependency\_detector:} Indicates which linear solver should be used to detect linearly dependent equality constraints. $\;$ \\
The default value for this string option is ``none''.
\\ 
Possible values:
\begin{itemize}
\item none:                    don't check; no extra work at beginning
\item mumps:                   use MUMPS
\item ma28:                     use MA28
\end{itemize}

\paragraph{dependency\_detection\_with\_rhs:} Indicates if the right hand sides of the constraints should be considered during dependency detection. $\;$ \\
The default value for this string option is ``no''.\\
Possible values:
\begin{itemize}
\item no:                      only look at gradients
\item yes:                     also consider right hand side
\end{itemize}

\paragraph{point\_perturbation\_radius:} Maximal perturbation of an evaluation point. $\;$ \\
     If a random perturbation of a points is required, this number indicates
     the maximal perturbation.  Currently, this is only used when we perturb
     the initial point in order to get a random Jacobian for the linear
     dependency detection of equality constraints.
The valid range for this real option is 
$0 \le {\tt point\_perturbation\_radius } <  {\tt +inf}$
and its default value is $10$.

\paragraph{kappa\_d:} Weight for linear damping term (to handle one-sided bounds). $\;$ \\
The valid range for this real option is 
$0 \le {\tt kappa\_d } <  {\tt +inf}$
and its default value is $10^{-5}$.

\paragraph{bound\_relax\_factor:} Factor for initial relaxation of the bounds. $\;$ \\
 Before start of the optimization, the bounds
given by the user are relaxed.  This option sets
the factor for this relaxation.  If it is set to
zero, then then bounds relaxation is disabled.
(See Eqn.(35) in the implementation paper.)
The valid range for this real option is 
$0 \le {\tt bound\_relax\_factor } <  {\tt +inf}$
and its default value is $0$.


\paragraph{honor\_original\_bounds:} Indicates whether final points should be projected into original bounds. $\;$ \\
 Ipopt might relax the bounds during the
optimization (see, e.g., option
``bound\_relax\_factor'').  This option determines
whether the final point should be projected back
into the user-provide original bounds after the
optimization.
The default value for this string option is ``yes''.
\\ 
Possible values:
\begin{itemize}
   \item no: Leave final point unchanged
   \item yes: Project final point back into original bounds
\end{itemize}

% \paragraph{check\_derivatives\_for\_naninf:} Indicates whether it is desired to check for Nan/Inf in derivative matrices $\;$ \\
%  Activating this option will cause an error if an
% invalid number is detected in the constraint
% Jacobians or the Lagrangian Hessian.  If this is
% not activated, the test is skipped, and the
% algorithm might proceed with invalid numbers and
% fail.
% The default value for this string option is ``no''.
% \\ 
% Possible values:
% \begin{itemize}
%    \item no: Don't check (faster).
%    \item yes: Check Jacobians and Hessian for Nan and Inf.
% \end{itemize}

\paragraph{fixed\_variable\_treatment:} Determines how fixed variables should be handled. $\;$ \\
The main difference between those options is that the starting point in the ``make\_constraint'' case still has the fixed variables at their given values, whereas in the case ``make\_parameter''
the functions are always evaluated with the fixed values for those variables. 
Also, for ``relax\_bounds'', the fixing bound constraints are relaxed (according to ``bound\_relax\_factor'').
For both ``make\_constraints'' and ``relax\_bounds'', bound multipliers are computed for the fixed variables.
The default value for this string option is ``make\_parameter''.
\\ 
Possible values:
\begin{itemize}
\item make\_parameter:    Remove fixed variable from optimization variables.
\item make\_constraint:   Add equality constraints fixing variables.
\item relax\_bounds:      Relax fixing bound constraints.
\end{itemize}


\subsubsection{Initialization}

\paragraph{bound\_frac:} Desired minimum relative distance from the initial point to bound. $\;$ \\
 Determines how much the initial point might have
to be modified in order to be sufficiently inside
the bounds (together with ``bound\_push'').
(This is $\kappa_2$ in Section 3.6 of the implementation paper.)
The valid range for this real option is 
$0 <  {\tt bound\_frac } \le 0.5$
and its default value is $0.01$.


\paragraph{bound\_push:} Desired minimum absolute distance from the initial point to bound. $\;$ \\
 Determines how much the initial point might have
to be modified in order to be sufficiently inside
the bounds (together with ``bound\_frac'').
(This is $\kappa_1$ in Section 3.6 of the implementation paper.)
The valid range for this real option is 
$0 <  {\tt bound\_push } <  {\tt +inf}$
and its default value is $0.01$.


\paragraph{slack\_bound\_push:} Desired minimum absolute distance from the initial slack to bound. $\;$ \\       Determines how much the initial slack variables might have to be modified
in order to be sufficiently inside the inequality bounds (together with ``slack\_bound\_frac'').
(This is $\kappa_1$ in Section 3.6 of the implementation paper.)
The valid range for this real option is 
$0 <  {\tt slack\_bound\_push } <  {\tt +inf}$
and its default value is $0.01$.

\paragraph{slack\_bound\_frac:} Desired minimum relative distance from the initial slack to bound.
     Determines how much the initial slack variables might have to be modified
     in order to be sufficiently inside the inequality bounds (together with
     ``slack\_bound\_push'').
(This is $\kappa_2$ in Section 3.6 of the implementation paper.)
The valid range for this real option is 
$0 <  {\tt slack\_bound\_frac } \le  0.5$
and its default value is $0.01$.


\paragraph{bound\_mult\_init\_val:} Initial value for the bound multipliers. $\;$ \\
 All dual variables corresponding to bound
constraints are initialized to this value. The valid range for this real option is 
$0 <  {\tt bound\_mult\_init\_val } <  {\tt +inf}$
and its default value is $1$.


\paragraph{constr\_mult\_init\_max:} Maximum allowed least-square guess of constraint multipliers. $\;$ \\
 Determines how large the initial least-square
guesses of the constraint multipliers are allowed
to be (in max-norm). If the guess is larger than
this value, it is discarded and all constraint
multipliers are set to zero.  This options is
also used when initializing the restoration
phase. By default,
``resto.constr\_mult\_init\_max'' (the one used in
RestoIterateInitializer) is set to zero. The valid range for this real option is 
$0 \le {\tt constr\_mult\_init\_max } <  {\tt +inf}$
and its default value is $1000$.


\paragraph{bound\_mult\_init\_val:} Initial value for the bound multipliers. $\;$ \\
 All dual variables corresponding to bound
constraints are initialized to this value. The valid range for this real option is 
$0 <  {\tt bound\_mult\_init\_val } <  {\tt +inf}$
and its default value is $1$.

\paragraph{least\_square\_init\_primal:} Least square initialization of the primal variables. $\;$ \\
     If set to yes, Ipopt ignores the user provided point and solves a least
     square problem for the primal variables (x and s), to fit the linearize
     equality and inequality constraints.  This might be useful if the user
     doesn't know anything about the starting point, or for solving an LP or
     QP.
The default value for this string option is ``no''.\\
   Possible values:
\begin{itemize}
    \item no:                      take user-provided point
    \item yes:                     overwrite user-provided point with least-square estimates
\end{itemize}

\paragraph{least\_square\_init\_duals:} Least square initialization of all dual variables. $\;$ \\
     If set to yes, Ipopt tries to compute least-square multipliers
     (considering ALL dual variables).  If successful, the bound multipliers
     are possibly corrected to be at least bound\_mult\_init\_val. This might be
     useful if the user doesn't know anything about the starting point, or for
     solving an LP or QP.
The default value for this string option is ``no''.\\
   Possible values:
\begin{itemize}
    \item no:                      use bound\_mult\_init\_val and least-square equality constraint multipliers
    \item yes:                     overwrite user-provided point with least-square estimates
\end{itemize}

\subsubsection{Barrier parameter update}

\paragraph{mehrotra\_algorithm:} Indicates if we want to do Mehrotra's algorithm. $\;$ \\
 If set to yes, Ipopt runs as Mehrotra's
predictor-corrector algorithm. This works usually
very well for LPs and convex QPs.  This
automatically disables the line search, and
chooses the (unglobalized) adaptive mu strategy
with the ``probing'' oracle, and uses
``corrector\_type=affine'' without any safeguards;
you should not set any of those options
explicitly in addition.  Also, unlessotherwise
specified, the values of ``bound\_push'',
``bound\_frac'', and ``bound\_mult\_init\_val'' are
set more aggressive, and sets
``alpha\_for\_y=bound\_mult''.
The default value for this string option is ``no''.
\\ 
Possible values:
\begin{itemize}
   \item no: Do the usual Ipopt algorithm.
   \item yes: Do Mehrotra's predictor-corrector algorithm.
\end{itemize}

\paragraph{mu\_strategy:} Update strategy for barrier parameter. $\;$ \\
 Determines which barrier parameter update
strategy is to be used.
The default value for this string option is ``adaptive''.
\\ 
Possible values:
\begin{itemize}
   \item monotone: use the monotone (Fiacco-McCormick) strategy
   \item adaptive: use the adaptive update strategy
\end{itemize}

\paragraph{mu\_oracle:} Oracle for a new barrier parameter in the adaptive strategy. $\;$ \\
 Determines how a new barrier parameter is
computed in each ``free-mode'' iteration of the
adaptive barrier parameter strategy. (Only
considered if ``adaptive'' is selected for option
``mu\_strategy'').
The default value for this string option is ``quality-function''.
\\ 
Possible values:
\begin{itemize}
   \item probing: Mehrotra's probing heuristic
   \item loqo: LOQO's centrality rule
   \item quality-function: minimize a quality function
\end{itemize}

\paragraph{quality\_function\_max\_section\_steps:} Maximum number of search steps during direct search procedure determining the optimal centering parameter. $\;$ \\
 The golden section search is performed for the
quality function based mu oracle. The valid range for this integer option is
$0 \le {\tt quality\_function\_max\_section\_steps } <  {\tt +inf}$
and its default value is $8$.
This option is only used if the option ``mu\_oracle'' is set to ``quality-function''.


\paragraph{fixed\_mu\_oracle:} Oracle for the barrier parameter when switching to fixed mode. $\;$ \\
 Determines how the first value of the barrier
parameter should be computed when switching to
the ``monotone mode'' in the adaptive strategy.
(Only considered if ``adaptive'' is selected for
option ``mu\_strategy''.)
The default value for this string option is ``average\_compl''.
\\ 
Possible values:
\begin{itemize}
   \item probing: Mehrotra's probing heuristic
   \item loqo: LOQO's centrality rule
   \item quality-function: minimize a quality function
   \item average\_compl: base on current average complementarity
\end{itemize}

\paragraph{mu\_init:} Initial value for the barrier parameter. $\;$ \\
 This option determines the initial value for the
barrier parameter (mu).  It is only relevant in
the monotone, Fiacco-McCormick version of the
algorithm. (i.e., if ``mu\_strategy'' is chosen as
``monotone'') The valid range for this real option is 
$0 <  {\tt mu\_init } <  {\tt +inf}$
and its default value is $0.1$.

\paragraph{mu\_max\_fact:} Factor for initialization of maximum value for barrier parameter. $\;$ \\
 This option determines the upper bound on the
barrier parameter.  This upper bound is computed
as the average complementarity at the initial
point times the value of this option. (Only used
if option ``mu\_strategy'' is chosen as ``adaptive''.) The valid range for this real option is 
$0 <  {\tt mu\_max\_fact } <  {\tt +inf}$
and its default value is $1000$.


\paragraph{mu\_max:} Maximum value for barrier parameter. $\;$ \\
 This option specifies an upper bound on the
barrier parameter in the adaptive mu selection
mode.  If this option is set, it overwrites the
effect of mu\_max\_fact. (Only used if option
``mu\_strategy'' is chosen as ``adaptive''.) The valid range for this real option is 
$0 <  {\tt mu\_max } <  {\tt +inf}$
and its default value is $100000$.


\paragraph{mu\_min:} Minimum value for barrier parameter. $\;$ \\
 This option specifies the lower bound on the
barrier parameter in the adaptive mu selection
mode. By default, it is set to
min(``tol'', ``compl\_inf\_tol'')/(``barrier\_tol\_fact-
or''+1), which should be a reasonable value. (Only
used if option ``mu\_strategy'' is chosen as
``adaptive''.) The valid range for this real option is 
$0 <  {\tt mu\_min } <  {\tt +inf}$
and its default value is $1 \cdot 10^{-09}$.

\paragraph{barrier\_tol\_factor:} Factor for mu in barrier stop test. $\;$ \\
 The convergence tolerance for each barrier
problem in the monotone mode is the value of the
barrier parameter times ``barrier\_tol\_factor''.
This option is also used in the adaptive mu
strategy during the monotone mode. (This is
$\kappa_\varepsilon$ in the implementation paper). The valid range for this real option is 
$0 <  {\tt barrier\_tol\_factor } <  {\tt +inf}$
and its default value is $10$.

\paragraph{mu\_linear\_decrease\_factor:} Determines linear decrease rate of barrier parameter. $\;$ \\
 For the Fiacco-McCormick update procedure the new
barrier parameter mu is obtained by taking the
minimum of mu$\cdot$``mu\_linear\_decrease\_factor'' and
mu$^\textrm{``superlinear\_decrease\_power''}$.  (This is
$\kappa_\mu$ in the implementation paper.) This option
is also used in the adaptive mu strategy during
the monotone mode. The valid range for this real option is 
$0 <  {\tt mu\_linear\_decrease\_factor } <  1$
and its default value is $0.2$.


\paragraph{mu\_superlinear\_decrease\_power:} Determines superlinear decrease rate of barrier parameter. $\;$ \\
 For the Fiacco-McCormick update procedure the new
barrier parameter mu is obtained by taking the
minimum of mu$\cdot$``mu\_linear\_decrease\_factor'' and
mu$^\textrm{``superlinear\_decrease\_power''}$.  (This is
$\theta_\mu$ in the implementation paper.) This option
is also used in the adaptive mu strategy during
the monotone mode. The valid range for this real option is 
$1 <  {\tt mu\_superlinear\_decrease\_power } <  2$
and its default value is $1.5$.

\subsubsection{Barrier parameter update (expert options)}

\paragraph{mu\_allow\_fast\_monotone\_decrease:} Allow skipping of barrier problem if barrier test is already met. $\;$ \\
 If set to ``no'', the algorithm enforces at least
one iteration per barrier problem, even if the
barrier test is already met for the updated
barrier parameter.
The default value for this string option is ``yes''.
\\ 
Possible values:
\begin{itemize}
   \item no: Take at least one iteration per barrier problem
   \item yes: Allow fast decrease of mu if barrier test it met
\end{itemize}

\paragraph{adaptive\_mu\_globalization:} Globalization strategy for the adaptive mu selection mode. $\;$ \\
 To achieve global convergence of the adaptive
version, the algorithm has to switch to the
monotone mode (Fiacco-McCormick approach) when
convergence does not seem to appear.  This option
sets the criterion used to decide when to do this
switch. (Only used if option ``mu\_strategy'' is
chosen as ``adaptive''.)
The default value for this string option is ``obj-constr-filter''.
\\ 
Possible values:
\begin{itemize}
   \item kkt-error: nonmonotone decrease of kkt-error
   \item obj-constr-filter: 2-dim filter for objective and constraint
violation
   \item never-monotone-mode: disables globalization
\end{itemize}

\paragraph{adaptive\_mu\_kkterror\_red\_iters:} Maximum number of iterations requiring sufficient progress. $\;$ \\
 For the ``kkt-error'' based globalization strategy,
the progress made in at most ``adaptive\_mu\_kkterror\_red\_iters'' iterations must be sufficient.
If this number of iterations is exceeded, the
globalization strategy switches to the monotone
mode. The valid range for this integer option is
$0 \le {\tt adaptive\_mu\_kkterror\_red\_iters } <  {\tt +inf}$
and its default value is $4$.


\paragraph{adaptive\_mu\_kkterror\_red\_fact:} Sufficient decrease factor for ``kkt-error'' globalization strategy. $\;$ \\
 For the ``kkt-error'' based globalization strategy,
the error must decrease by this factor to be
deemed sufficient decrease. The valid range for this real option is 
$0 <  {\tt adaptive\_mu\_kkterror\_red\_fact } <  1$
and its default value is $0.9999$.


\paragraph{filter\_margin\_fact:} Factor determining width of margin for obj-constr-filter adaptive globalization strategy. $\;$ \\
 When using the adaptive globalization strategy,
``obj-constr-filter'', sufficient progress for a
filter entry is defined as follows: (new obj) <
(filter obj) - filter\_margin\_fact$\cdot$ (new
constr-viol) OR (new constr-viol) $<$ (filter
constr-viol) - filter\_margin\_fact$\cdot$ (new
constr-viol).  For the description of the
``kkt-error-filter'' option see
``filter\_max\_margin''. The valid range for this real option is 
$0 <  {\tt filter\_margin\_fact } <  1$
and its default value is $1 \cdot 10^{-05}$.


\paragraph{filter\_max\_margin:} Maximum width of margin in obj-constr-filter adaptive globalization strategy. $\;$ \\
 The valid range for this real option is 
$0 <  {\tt filter\_max\_margin } <  {\tt +inf}$
and its default value is $1$.


\paragraph{adaptive\_mu\_restore\_previous\_iterate:} Indicates if the previous iterate should be restored if the monotone mode is entered. $\;$ \\
 When the globalization strategy for the adaptive
barrier algorithm switches to the monotone mode,
it can either start from the most recent iterate
(no), or from the last iterate that was accepted
(yes).
The default value for this string option is ``no''.
\\ 
Possible values:
\begin{itemize}
   \item no: don't restore accepted iterate
   \item yes: restore accepted iterate
\end{itemize}

\paragraph{adaptive\_mu\_monotone\_init\_factor:} Determines the initial value of the barrier parameter when switching to the monotone mode. $\;$ \\
 When the globalization strategy for the adaptive
barrier algorithm switches to the monotone mode
and the option fixed\_mu\_oracle is chosen as
``average\_compl'', the barrier parameter is set to
the current average complementarity times the
value of ``adaptive\_mu\_monotone\_init\_factor''. The default value for this option is $0.8$ and its valid range is $0 <  {\tt adaptive\_mu\_monotone\_init\_factor } <  {\tt +inf}$.


\paragraph{adaptive\_mu\_kkt\_norm\_type:} Norm used for the KKT error in the adaptive mu globalization strategies. $\;$ \\
 When computing the KKT error for the
globalization strategies, the norm to be used is
specified with this option. Note, this options is
also used in the QualityFunctionMuOracle.
The default value for this string option is ``2-norm-squared''.
\\ 
Possible values:
\begin{itemize}
   \item 1-norm: use the 1-norm (abs sum)
   \item 2-norm-squared: use the 2-norm squared (sum of squares)
   \item max-norm: use the infinity norm (max)
   \item 2-norm: use 2-norm
\end{itemize}

\paragraph{tau\_min:} Lower bound on fraction-to-the-boundary parameter tau. $\;$ \\
 (This is $\tau_{\min}$ in the implementation paper.)  This
option is also used in the adaptive mu strategy
during the monotone mode. The valid range for this real option is 
$0 <  {\tt tau\_min } <  1$
and its default value is $0.99$.


\paragraph{sigma\_max:} Maximum value of the centering parameter. $\;$ \\
 This is the upper bound for the centering
parameter chosen by the quality function based
barrier parameter update. (Only used if option
``mu\_oracle'' is set to ``quality-function''.) The valid range for this real option is 
$0 <  {\tt sigma\_max } <  {\tt +inf}$
and its default value is $100$.


\paragraph{sigma\_min:} Minimum value of the centering parameter. $\;$ \\
 This is the lower bound for the centering
parameter chosen by the quality function based
barrier parameter update. (Only used if option
``mu\_oracle'' is set to ``quality-function''.) The valid range for this real option is 
$0 \le {\tt sigma\_min } <  {\tt +inf}$
and its default value is $1 \cdot 10^{-06}$.


\paragraph{quality\_function\_norm\_type:} Norm used for components of the quality function. $\;$ \\
 (Only used if option ``mu\_oracle'' is set to
``quality-function''.)
The default value for this string option is ``2-norm-squared''.
\\ 
Possible values:
\begin{itemize}
   \item 1-norm: use the 1-norm (abs sum)
   \item 2-norm-squared: use the 2-norm squared (sum of squares)
   \item max-norm: use the infinity norm (max)
   \item 2-norm: use 2-norm
\end{itemize}

\paragraph{quality\_function\_centrality:} The penalty term for centrality that is included in quality function. $\;$ \\
 This determines whether a term is added to the
quality function to penalize deviation from
centrality with respect to complementarity.  The
complementarity measure here is the xi in the
Loqo update rule. (Only used if option
``mu\_oracle'' is set to ``quality-function''.)
The default value for this string option is ``none''.
\\ 
Possible values:
\begin{itemize}
   \item none: no penalty term is added
   \item log: complementarity $\cdot$ the log of the centrality
measure
   \item reciprocal: complementarity $\cdot$ the reciprocal of the
centrality measure
   \item cubed-reciprocal: complementarity $\cdot$ the reciprocal of the
centrality measure cubed
\end{itemize}

\paragraph{quality\_function\_balancing\_term:} The balancing term included in the quality function for centrality. $\;$ \\
 This determines whether a term is added to the
quality function that penalizes situations where
the complementarity is much smaller than dual and
primal infeasibilities. (Only used if option
``mu\_oracle'' is set to ``quality-function''.)
The default value for this string option is ``none''.
\\ 
Possible values:
\begin{itemize}
   \item none: no balancing term is added
   \item cubic: $\max(0,\max(\textrm{dual\_inf},\textrm{primal\_inf})-\textrm{compl})^3$
\end{itemize}

\paragraph{quality\_function\_max\_section\_steps:} Maximum number of search steps during direct search procedure determining the optimal centering parameter. $\;$ \\
 The golden section search is performed for the
quality function based mu oracle. The valid range for this integer option is
$0 \le {\tt quality\_function\_max\_section\_steps } <  {\tt +inf}$
and its default value is $8$.
(Only used if
option ``mu\_oracle'' is set to ``quality-function''.)


\paragraph{quality\_function\_section\_sigma\_tol:} Tolerance for the section search procedure determining the optimal centering parameter (in sigma space). $\;$ \\
 The golden section search is performed for the
quality function based mu oracle. (Only used if
option ``mu\_oracle'' is set to ``quality-function''.) The valid range for this real option is 
$0 \le {\tt quality\_function\_section\_sigma\_tol } <  1$
and its default value is $0.01$.


\paragraph{quality\_function\_section\_qf\_tol:} Tolerance for the golden section search procedure determining the optimal centering parameter (in the function value space). $\;$ \\
 The golden section search is performed for the
quality function based mu oracle. (Only used if
option ``mu\_oracle'' is set to ``quality-function''.) The valid range for this real option is 
$0 \le {\tt quality\_function\_section\_qf\_tol } <  1$
and its default value is $0$.


\subsubsection{Warm start}

\paragraph{warm\_start\_init\_point:} Warm-start for initial point $\;$ \\
 Indicates whether this optimization should use a warm start initialization, where values of dual variables are given by GAMS (You can set marginal values for variables and equations in your GAMS model to set the starting point for the dual variables.)
For the primal values, Ipopt uses the starting point that is given by GAMS (You can set level values for variables (and equations) in your GAMS model to set the starting point for the primal variables.)
The default value for this string option is ``no''.
\\ 
Possible values:
\begin{itemize}
   \item no: do not use the warm start initialization
   \item yes: use the warm start initialization
\end{itemize}

\paragraph{warm\_start\_bound\_push:} same as bound\_push for the regular initializer. $\;$ \\
 The valid range for this real option is 
$0 <  {\tt warm\_start\_bound\_push } <  {\tt +inf}$
and its default value is $0.001$.


\paragraph{warm\_start\_bound\_frac:} same as bound\_frac for the regular initializer. $\;$ \\
 The valid range for this real option is 
$0 <  {\tt warm\_start\_bound\_frac } \le 0.5$
and its default value is $0.001$.


\paragraph{warm\_start\_slack\_bound\_frac:} same as slack\_bound\_frac for the regular initializer. $\;$ \\
 The valid range for this real option is 
$0 <  {\tt warm\_start\_slack\_bound\_frac } \le 0.5$
and its default value is $0.001$.


\paragraph{warm\_start\_slack\_bound\_push:} same as slack\_bound\_push for the regular initializer. $\;$ \\
 The valid range for this real option is 
$0 <  {\tt warm\_start\_slack\_bound\_push } <  {\tt +inf}$
and its default value is $0.001$.


\paragraph{warm\_start\_mult\_bound\_push:} same as mult\_bound\_push for the regular initializer. $\;$ \\
 The valid range for this real option is 
$0 <  {\tt warm\_start\_mult\_bound\_push } <  {\tt +inf}$
and its default value is $0.001$.


\paragraph{warm\_start\_mult\_init\_max:} Maximum initial value for the equality multipliers. $\;$ \\
 The valid range for this real option is 
${\tt -inf} <  {\tt warm\_start\_mult\_init\_max } <  {\tt +inf}$
and its default value is $1 \cdot 10^{+06}$.

\subsubsection{Multiplier updates}

\paragraph{alpha\_for\_y:} Method to determine the step size for constraint multipliers. $\;$ \\
 This option determines how the step size
(alpha\_y) will be calculated when updating the
constraint multipliers.
The default value for this string option is ``primal''.
\\ 
Possible values:
\begin{itemize}
   \item primal: use primal step size
   \item bound\_mult: use step size for the bound multipliers (good
for LPs)
   \item min: use the min of primal and bound multipliers
   \item max: use the max of primal and bound multipliers
   \item full: take a full step of size one
   \item min\_dual\_infeas: choose step size minimizing new dual
infeasibility
   \item safe\_min\_dual\_infeas: like ``min\_dual\_infeas'', but safeguarded by
``min'' and ``max''
\end{itemize}

\paragraph{alpha\_for\_y\_tol:} Tolerance for switching to full equality multiplier steps. $\;$ \\
 This is only relevant if ``alpha\_for\_y'' is
chosen ``primal-and-full'' or ``dual-and-full''.  The
step size for the equality constraint multipliers
is taken to be one if the max-norm of the primal
step is less than this tolerance. The valid range for this real option is 
$0 \le {\tt alpha\_for\_y\_tol } <  {\tt +inf}$
and its default value is $10$.

\paragraph{recalc\_y:} Tells the algorithm to recalculate the equality and inequality multipliers as least square estimates. $\;$ \\
 This asks Ipopt to recompute the
multipliers, whenever the current infeasibility
is less than recalc\_y\_feas\_tol. Choosing yes
might be helpful in the quasi-Newton option. 
However, each recalculation requires an extra
factorization of the linear system.  If a limited
memory quasi-Newton option is chosen, this is
used by default.
The default value for this string option is ``no''.
\\ 
Possible values:
\begin{itemize}
   \item no: use the Newton step to update the multipliers
   \item yes: use least-square multiplier estimates
\end{itemize}

\paragraph{recalc\_y\_feas\_tol:} Feasibility threshold for recomputation of multipliers. $\;$ \\
 If recalc\_y is chosen and the current
infeasibility is less than this value, then the
multipliers are recomputed. The valid range for this real option is 
$0 <  {\tt recalc\_y\_feas\_tol } <  {\tt +inf}$
and its default value is $1 \cdot 10^{-06}$.

\subsubsection{Line search}

\paragraph{max\_soc:} Maximum number of second order correction trial steps at each iteration. $\;$ \\
 Choosing 0 disables the second order corrections.
(This is $p^{\max}$ of Step A-5.9 of Algorithm A in
the implementation paper.) The valid range for this integer option is
$0 \le {\tt max\_soc } <  {\tt +inf}$
and its default value is $4$.


\paragraph{watchdog\_shortened\_iter\_trigger:} Number of shortened iterations that trigger the watchdog. $\;$ \\
 If the number of successive iterations in which
the backtracking line search did not accept the
first trial point exceeds this number, the
watchdog procedure is activated.  Choosing 0
here disables the watchdog procedure. The valid range for this integer option is
$0 \le {\tt watchdog\_shortened\_iter\_trigger } <  {\tt +inf}$
and its default value is $10$.


\paragraph{watchdog\_trial\_iter\_max:} Maximum number of watchdog iterations. $\;$ \\
 This option determines the number of trial
iterations allowed before the watchdog procedure
is aborted and the algorithm returns to the
stored point. The valid range for this integer option is
$1 \le {\tt watchdog\_trial\_iter\_max } <  {\tt +inf}$
and its default value is $3$.

\paragraph{corrector\_type:} The type of corrector steps that should be taken (experimental!). $\;$ \\
 If ``mu\_strategy'' is ``adaptive'', this option
determines what kind of corrector steps should be
tried.
The default value for this string option is ``none''.
\\ 
Possible values:
\begin{itemize}
   \item none: no corrector
   \item affine: corrector step towards mu=0
   \item primal-dual: corrector step towards current mu
\end{itemize}

\subsubsection{Line search (expert options)}

\paragraph{alpha\_red\_factor:} Fractional reduction of the trial step size in the backtracking line search. $\;$ \\
 At every step of the backtracking line search,
the trial step size is reduced by this factor. The valid range for this real option is 
$0 <  {\tt alpha\_red\_factor } <  1$
and its default value is $0.5$.


\paragraph{accept\_every\_trial\_step:} Always accept the first trial step. $\;$ \\
 Setting this option to ``yes'' essentially disables
the line search and makes the algorithm take
aggressive steps, without global convergence
guarantees.
The default value for this string option is ``no''.
\\ 
Possible values:
\begin{itemize}
   \item no: don't arbitrarily accept the full step
   \item yes: always accept the full step
\end{itemize}

\paragraph{tiny\_step\_tol:} Tolerance for detecting numerically insignificant steps. $\;$ \\
 If the search direction in the primal variables
(x and s) is, in relative terms for each
component, less than this value, the algorithm
accepts the full step without line search.  If
this happens repeatedly, the algorithm will
terminate with a corresponding exit message. The
default value is 10 times machine precision. The valid range for this real option is 
$0 \le {\tt tiny\_step\_tol } <  {\tt +inf}$
and its default value is $2.22045 \cdot 10^{-15}$.


\paragraph{tiny\_step\_y\_tol:} Tolerance for quitting because of numerically insignificant steps. $\;$ \\
 If the search direction in the primal variables
(x and s) is, in relative terms for each
component, repeatedly less than tiny\_step\_tol,
and the step in the y variables is smaller than
this threshold, the algorithm will terminate. The valid range for this real option is 
$0 \le {\tt tiny\_step\_y\_tol } <  {\tt +inf}$
and its default value is $0.01$.


\paragraph{theta\_max\_fact:} Determines upper bound for constraint violation in the filter. $\;$ \\
 The algorithmic parameter theta\_max is
determined as theta\_max\_fact times the maximum
of 1 and the constraint violation at initial
point.  Any point with a constraint violation
larger than theta\_max is unacceptable to the
filter (see Eqn. (21) in the implementation paper). The valid range for this real option is 
$0 <  {\tt theta\_max\_fact } <  {\tt +inf}$
and its default value is $10000$.


\paragraph{theta\_min\_fact:} Determines constraint violation threshold in the switching rule. $\;$ \\
 The algorithmic parameter theta\_min is
determined as theta\_min\_fact times the maximum
of 1 and the constraint violation at initial
point.  The switching rules treats an iteration
as an h-type iteration whenever the current
constraint violation is larger than theta\_min
(see paragraph before Eqn. (19) in the implementation
paper). The valid range for this real option is 
$0 <  {\tt theta\_min\_fact } <  {\tt +inf}$
and its default value is $0.0001$.


\paragraph{eta\_phi:} Relaxation factor in the Armijo condition. $\;$ \\
 (See Eqn. (20) in the implementation paper) The valid range for this real option is 
$0 <  {\tt eta\_phi } <  0.5$
and its default value is $1 \cdot 10^{-08}$.


\paragraph{delta:} Multiplier for constraint violation in the switching rule. $\;$ \\
 (See Eqn. (19) in the implementation paper.) The valid range for this real option is 
$0 <  {\tt delta } <  {\tt +inf}$
and its default value is $1$.


\paragraph{s\_phi:} Exponent for linear barrier function model in the switching rule. $\;$ \\
 (See Eqn. (19) in the implementation paper.) The valid range for this real option is 
$1 <  {\tt s\_phi } <  {\tt +inf}$
and its default value is $2.3$.


\paragraph{s\_theta:} Exponent for current constraint violation in the switching rule. $\;$ \\
 (See Eqn. (19) in the implementation paper.) The valid range for this real option is 
$1 <  {\tt s\_theta } <  {\tt +inf}$
and its default value is $1.1$.


\paragraph{gamma\_phi:} Relaxation factor in the filter margin for the barrier function. $\;$ \\
 (See Eqn. (18a) in the implementation paper.) The valid range for this real option is 
$0 <  {\tt gamma\_phi } <  1$
and its default value is $1 \cdot 10^{-08}$.


\paragraph{gamma\_theta:} Relaxation factor in the filter margin for the constraint violation. $\;$ \\
 (See Eqn. (18b) in the implementation paper.) The valid range for this real option is 
$0 <  {\tt gamma\_theta } <  1$
and its default value is $1 \cdot 10^{-05}$.


\paragraph{alpha\_min\_frac:} Safety factor for the minimal step size (before switching to restoration phase). $\;$ \\
 (This is $\gamma_\alpha$ in Eqn. (20) in the
implementation paper.) The default value of this real option is $0.05$ and its
valid range is $0 <  {\tt alpha\_min\_frac } <  1$.


\paragraph{kappa\_soc:} Factor in the sufficient reduction rule for second order correction. $\;$ \\
 This option determines how much a second order
correction step must reduce the constraint
violation so that further correction steps are
attempted.  (See Step A-5.9 of Algorithm A in
the implementation paper.) The valid range for this real option is 
$0 <  {\tt kappa\_soc } <  {\tt +inf}$
and its default value is $0.99$.


\paragraph{obj\_max\_inc:} Determines the upper bound on the acceptable increase of barrier objective function. $\;$ \\
 Trial points are rejected if they lead to an
increase in the barrier objective function by
more than obj\_max\_inc orders of magnitude. The valid range for this real option is 
$1 <  {\tt obj\_max\_inc } <  {\tt +inf}$
and its default value is $5$.


\paragraph{max\_filter\_resets:} Maximal allowed number of filter resets $\;$ \\
 A positive number enables a heuristic that resets
the filter, whenever in more than
``filter\_reset\_trigger'' successive iterations
the last rejected trial steps size was rejected
because of the filter.  This option determine the
maximal number of resets that are allowed to take
place. The valid range for this integer option is
$0 \le {\tt max\_filter\_resets } <  {\tt +inf}$
and its default value is $5$.


\paragraph{filter\_reset\_trigger:} Number of iterations that trigger the filter reset. $\;$ \\
 If the filter reset heuristic is active and the
number of successive iterations in which the last
rejected trial step size was rejected because of
the filter, the filter is reset. The valid range for this integer option is
$1 \le {\tt filter\_reset\_trigger } <  {\tt +inf}$
and its default value is $5$.


% \paragraph{skip\_corr\_if\_neg\_curv:} Skip the corrector step in negative curvature iteration (unsupported!). $\;$ \\
%  The corrector step is not tried if negative
% curvature has been encountered during the
% computation of the search direction in the
% current iteration. This option is only used if
% ``mu\_strategy'' is ``adaptive''.
% The default value for this string option is ``yes''.
% \\ 
% Possible values:
% \begin{itemize}
%    \item no: don't skip
%    \item yes: skip
% \end{itemize}

% \paragraph{skip\_corr\_in\_monotone\_mode:} Skip the corrector step during monotone barrier parameter mode (unsupported!). $\;$ \\
%  The corrector step is not tried if the algorithm
% is currently in the monotone mode (see also
% option ``barrier\_strategy'').This option is only
% used if ``mu\_strategy'' is ``adaptive''.
% The default value for this string option is ``yes''.
% \\ 
% Possible values:
% \begin{itemize}
%    \item no: don't skip
%    \item yes: skip
% \end{itemize}

% \paragraph{corrector\_compl\_avrg\_red\_fact:} Complementarity tolerance factor for accepting corrector step (unsupported!). $\;$ \\
%  This option determines the factor by which
% complementarity is allowed to increase for a
% corrector step to be accepted. The valid range for this real option is 
% $0 <  {\tt corrector\_compl\_avrg\_red\_fact } <  {\tt +inf}$
% and its default value is $1$.


\paragraph{kappa\_sigma:} Factor limiting the deviation of dual variables from primal estimates. $\;$ \\
 If the dual variables deviate from their primal
estimates, a correction is performed. (See Eqn.
(16) in the the implementation paper.) Setting the
value to less than 1 disables the correction. The valid range for this real option is 
$0 <  {\tt kappa\_sigma } <  {\tt +inf}$
and its default value is $1 \cdot 10^{+10}$.


\paragraph{slack\_move:} Correction size for very small slacks. $\;$ \\
 Due to numerical issues or the lack of an
interior, the slack variables might become very
small.  If a slack becomes very small compared to
machine precision, the corresponding bound is
moved slightly.  This parameter determines how
large the move should be.  Its default value is
mach\_eps$^{3/4}$.  (See also end of Section 3.5
in the implementation paper - but actual
the implementation might be somewhat different.) The valid range for this real option is 
$0 \le {\tt slack\_move } <  {\tt +inf}$
and its default value is $1.81899 \cdot 10^{-12}$.

\subsubsection{Restoration phase}

\paragraph{expect\_infeasible\_problem:} Enable heuristics to quickly detect an infeasible problem. $\;$ \\
 This options is meant to activate heuristics that
may speed up the infeasibility determination if
you expect that there is a good chance for the
problem to be infeasible.  In the filter line
search procedure, the restoration phase is called
more quickly than usually, and more reduction in
the constraint violation is enforced before the
restoration phase is left. If the problem is
square, this option is enabled automatically.
The default value for this string option is ``no''.
\\ 
Possible values:
\begin{itemize}
   \item no: the problem probably be feasible
   \item yes: the problem has a good chance to be infeasible
\end{itemize}

\paragraph{expect\_infeasible\_problem\_ctol:} Threshold for disabling ``expect\_infeasible\_problem'' option. $\;$ \\
 If the constraint violation becomes smaller than
this threshold, the ``expect\_infeasible\_problem''
heuristics in the filter line search are
disabled. If the problem is square, this options
is set to 0. The valid range for this real option is 
$0 \le {\tt expect\_infeasible\_problem\_ctol } <  {\tt +inf}$
and its default value is $0.001$.


\paragraph{start\_with\_resto:} Tells algorithm to switch to restoration phase in first iteration. $\;$ \\
 Setting this option to ``yes'' forces the algorithm
to switch to the feasibility restoration phase in
the first iteration. If the initial point is
feasible, the algorithm will abort with a failure.
The default value for this string option is ``no''.
\\ 
Possible values:
\begin{itemize}
   \item no: don't force start in restoration phase
   \item yes: force start in restoration phase
\end{itemize}

\paragraph{soft\_resto\_pderror\_reduction\_factor:} Required reduction in primal-dual error in the soft restoration phase. $\;$ \\
 The soft restoration phase attempts to reduce the
primal-dual error with regular steps. If the
damped primal-dual step (damped only to satisfy
the fraction-to-the-boundary rule) is not
decreasing the primal-dual error by at least this
factor, then the regular restoration phase is
called. Choosing 0 here disables the soft
restoration phase. The valid range for this real option is 
$0 \le {\tt soft\_resto\_pderror\_reduction\_factor } <  {\tt +inf}$
and its default value is $0.9999$.


\paragraph{required\_infeasibility\_reduction:} Required reduction of infeasibility before leaving restoration phase. $\;$ \\
 The restoration phase algorithm is performed,
until a point is found that is acceptable to the
filter and the infeasibility has been reduced by
at least the fraction given by this option. The valid range for this real option is 
$0 \le {\tt required\_infeasibility\_reduction } <  1$
and its default value is $0.9$.


\paragraph{max\_soft\_resto\_iters:} Maximum number of iterations performed successively in soft restoration phase. $\;$ \\
 If the soft restoration phase is performed for
more than so many iterations in a row, the regular
restoration phase is called. The valid range for this integer option is
$0 \le {\tt max\_soft\_resto\_iters } <  {\tt +inf}$
and its default value is $10$.


\paragraph{max\_resto\_iter:} Maximum number of successive iterations in restoration phase. $\;$ \\
 The algorithm terminates with an error message if
the number of iterations successively taken in
the restoration phase exceeds this number. The valid range for this integer option is
$0 \le {\tt max\_resto\_iter } <  {\tt +inf}$
and its default value is $3000000$.


\paragraph{bound\_mult\_reset\_threshold:} Threshold for resetting bound multipliers after the restoration phase. $\;$ \\
 After returning from the restoration phase, the
bound multipliers are updated with a Newton step
for complementarity.  Here, the change in the
primal variables during the entire restoration
phase is taken to be the corresponding primal
Newton step. However, if after the update the
largest bound multiplier exceeds the threshold
specified by this option, the multipliers are all
reset to 1. The valid range for this real option is 
$0 \le {\tt bound\_mult\_reset\_threshold } <  {\tt +inf}$
and its default value is $1000$.


\paragraph{constr\_mult\_reset\_threshold:} Threshold for resetting equality and inequality multipliers after restoration phase. $\;$ \\
 After returning from the restoration phase, the
constraint multipliers are recomputed by a least
square estimate.  This option triggers when those
least-square estimates should be ignored. The valid range for this real option is 
$0 \le {\tt constr\_mult\_reset\_threshold } <  {\tt +inf}$
and its default value is $0$.


\paragraph{evaluate\_orig\_obj\_at\_resto\_trial:} Determines if the original objective function should be evaluated at restoration phase trial points. $\;$ \\
 Setting this option to ``yes'' makes the
restoration phase algorithm evaluate the
objective function of the original problem at
every trial point encountered during the
restoration phase, even if this value is not
required.  In this way, it is guaranteed that the
original objective function can be evaluated
without error at all accepted iterates; otherwise
the algorithm might fail at a point where the
restoration phase accepts an iterate that is good
for the restoration phase problem, but not the
original problem.  On the other hand, if the
evaluation of the original objective is
expensive, this might be costly.
The default value for this string option is ``yes''.
\\ 
Possible values:
\begin{itemize}
   \item no: skip evaluation
   \item yes: evaluate at every trial point
\end{itemize}

\subsubsection{Linear solver}

\paragraph{linear\_solver:} Linear solver used for step computations. $\;$ \\
 Determines which linear algebra package is to be used for the solution of the augmented linear system (for obtaining the search directions).
Note, you need to provide an extra shared library to use MA27, MA57, or PARDISO, see \hyperlink{ipoptlinearsolver}{Section \ref{ipoptlinearsolver}}.
The default value for this string option is ``mumps''.
\\
Possible values:
\begin{itemize}
   \item ma27: use the Harwell routine MA27
   \item ma57: use the Harwell routine MA57
   \item pardiso: use the Pardiso package
%    \item wsmp: use WSMP package
%    \item taucs: use TAUCS package (not yet working)
   \item mumps: use MUMPS package
%    \item custom: use custom linear solver
\end{itemize}

\paragraph{hsl\_library:} Path and filename of HSL library for dynamic load. $\;$ \\
Specify the path to a library that contains HSL routines and can be load via dynamic linking, see also \hyperlink{ipoptlinearsolver}{Section \ref{ipoptlinearsolver}}.

\paragraph{pardiso\_library:} Path and filename of PARDISO library for dynamic load. $\;$ \\
Specify the path to a PARDISO library that and can be load via dynamic linking, see also \hyperlink{ipoptlinearsolver}{Section \ref{ipoptlinearsolver}}.

\paragraph{linear\_system\_scaling:} Method for scaling the linear system. $\;$ \\
 Determines the method used to compute symmetric scaling factors for the augmented system (see also the ``linear\_scaling\_on\_demand'' option).
This scaling is independent of the NLP problem scaling.
By default, MC19 is only used if MA27 or MA57 are selected as linear solvers.
% This option is only available if Ipopt has been compiled with MC19.
The default value for this string option is ``mc19''.
\\
Possible values:
\begin{itemize}
   \item none: no scaling will be performed
   \item mc19: use the Harwell routine MC19
\end{itemize}

\paragraph{linear\_scaling\_on\_demand:} Flag indicating that linear scaling is only done if it seems required. $\;$ \\
 This option is only important if a linear scaling method (e.g., mc19) is used.
If you choose ``no'', then the scaling factors are computed for every linear system from the start.
This can be quite expensive.
Choosing ``yes'' means that the algorithm will start the scaling method only when the solutions to the linear system seem not good, and then use it until the end.
The default value for this string option is ``yes''.
\\
Possible values:
\begin{itemize}
   \item no: Always scale the linear system.
   \item yes: Start using linear system scaling if solutions
seem not good.
\end{itemize}

\paragraph{fast\_step\_computation:} Indicates if the linear system should be solved quickly. $\;$ \\
 If set to yes, the algorithm assumes that the
linear system that is solved to obtain the search
direction, is solved sufficiently well. In that
case, no residuals are computed, and the
computation of the search direction is a little
faster.
The default value for this string option is ``no''.
\\ 
Possible values:
\begin{itemize}
   \item no: Verify solution of linear system by computing
residuals.
   \item yes: Trust that linear systems are solved well.
\end{itemize}

\paragraph{max\_refinement\_steps:} Maximum number of iterative refinement steps per linear system solve. $\;$ \\
 Iterative refinement (on the full unsymmetric
system) is performed for each right hand side. 
This option determines the maximum number of
iterative refinement steps. The valid range for this integer option is
$0 \le {\tt max\_refinement\_steps } <  {\tt +inf}$
and its default value is $10$.


\paragraph{min\_refinement\_steps:} Minimum number of iterative refinement steps per linear system solve. $\;$ \\
 Iterative refinement (on the full unsymmetric
system) is performed for each right hand side. 
This option determines the minimum number of
iterative refinements (i.e. at least
``min\_refinement\_steps'' iterative refinement
steps are enforced per right hand side.) The valid range for this integer option is
$0 \le {\tt min\_refinement\_steps } <  {\tt +inf}$
and its default value is $1$.

\paragraph{residual\_ratio\_max:} Iterative refinement tolerance $\;$ \\
 Iterative refinement is performed until the
residual test ratio is less than this tolerance
(or until the limit ``max\_refinement\_steps'' is hit). The valid range for this real option is 
$0 <  {\tt residual\_ratio\_max } <  {\tt +inf}$
and its default value is $1 \cdot 10^{-10}$.


\paragraph{residual\_ratio\_singular:} Threshold for declaring linear system singular after failed iterative refinement. $\;$ \\
 If the residual test ratio is larger than this
value after failed iterative refinement, the
algorithm pretends that the linear system is
singular. The valid range for this real option is 
$0 <  {\tt residual\_ratio\_singular } <  {\tt +inf}$
and its default value is $1 \cdot 10^{-05}$.


\paragraph{residual\_improvement\_factor:} Minimal required reduction of residual test ratio in iterative refinement. $\;$ \\
 If the improvement of the residual test ratio
made by one iterative refinement step is not
better than this factor, iterative refinement is
aborted. The valid range for this real option is 
$0 <  {\tt residual\_improvement\_factor } <  {\tt +inf}$
and its default value is $1$.


\subsubsection{MUMPS Linear Solver}

\paragraph{mumps\_pivtol:} Pivot tolerance for the linear solver MUMPS. \\
A smaller number pivots for sparsity, a larger number pivots for stability.
The valid range for this real option is
$0 \le {\tt mumps\_pivtol } < {\tt 1}$
and its default value is $1e-6$.

\paragraph{mumps\_pivtolmax:} Maximum pivot tolerance for the linear solver MUMPS. \\
Ipopt may increase pivtol as high as pivtolmax to get a more accurate solution to the linear system.
The valid range for this real option is
$0 \le {\tt mumps\_pivtolmax } < {\tt 1}$
and its default value is $0.1$.

\paragraph{mumps\_mem\_percent:} Percentage increase in the estimated working space for MUMPS. \\
In MUMPS when significant extra fill-in is caused by numerical pivoting, larger values of mumps\_mem\_percent may help use the workspace more efficiently.
The valid range for this integer option is
$0 \le {\tt mumps\_mem\_percent } < {\tt +inf}$
and its default value is $1000$.

\paragraph{mumps\_permuting\_scaling:} Controls permuting and scaling in MUMPS $\;$ \\
 This is ICTL(6) in MUMPS. The valid range for this integer option is
$0 \le {\tt mumps\_permuting\_scaling } \le 7$
and its default value is $7$.


\paragraph{mumps\_pivot\_order:} Controls pivot order in MUMPS $\;$ \\
 This is ICTL(7) in MUMPS. The valid range for this integer option is
$0 \le {\tt mumps\_pivot\_order } \le 7$
and its default value is $7$.


\paragraph{mumps\_scaling:} Controls scaling in MUMPS $\;$ \\
 This is ICTL(8) in MUMPS. The valid range for this integer option is
$-2 \le {\tt mumps\_scaling } \le 7$
and its default value is $7$.


\paragraph{mumps\_dep\_tol:} Pivot threshold for detection of linearly dependent constraints in MUMPS. $\;$ \\
 When MUMPS is used to determine linearly
dependent constraints, this is determines the
threshold for a pivot to be considered zero. 
This is CNTL(3) in MUMPS. The valid range for this real option is 
${\tt -inf} <  {\tt mumps\_dep\_tol } <  {\tt +inf}$
and its default value is $-1$.

\subsubsection{PARDISO Linear Solver}

\paragraph{pardiso\_matching\_strategy:} Matching strategy to be used by Pardiso $\;$ \\
This is IPAR(13) in Pardiso manual.
The default value for this string option is ``complete+2x2''.
\\
Possible values:
\begin{itemize}
   \item complete: Match complete (IPAR(13)=1)
   \item complete+2x2: Match complete+2x2 (IPAR(13)=2)
   \item constraints: Match constraints (IPAR(13)=3)
\end{itemize}

\paragraph{pardiso\_out\_of\_core\_power:} Enables out-of-core variant of Pardiso $\;$ \\
Setting this option to a positive integer $k$ makes Pardiso work in the out-of-core variant where the factor is split in $2^k$ subdomains.
This is IPARM(50) in the Pardiso manual.
The valid range for this integer option is $0 \le {\tt pardiso\_out\_of\_core\_power } <  {\tt +inf}$ and its default value is $0$.

\subsubsection{MA27 Linear Solver}

\paragraph{ma27\_pivtol:} Pivot tolerance for the linear solver MA27. $\;$ \\
A smaller number pivots for sparsity, a larger number pivots for stability.
The valid range for this real option is $0 <  {\tt ma27\_pivtol } <  1$ and its default value is $10^{-8}$.


\paragraph{ma27\_pivtolmax:} Maximum pivot tolerance for the linear solver MA27. $\;$ \\
Ipopt may increase pivtol as high as pivtolmax to get a more accurate solution to the linear
system.
The valid range for this real option is $0 <  {\tt ma27\_pivtolmax } <  1$ and its default value is $0.0001$.


\paragraph{ma27\_liw\_init\_factor:} Integer workspace memory for MA27. $\;$ \\
The initial integer workspace memory = liw\_init\_factor $*$ memory required by unfactored system.
Ipopt will increase the workspace size by meminc\_factor if required.
The default value for this real option is $5$ and its valid range is $1 \le {\tt ma27\_liw\_init\_factor } <  {\tt +inf}$.


\paragraph{ma27\_la\_init\_factor:} Real workspace memory for MA27. $\;$ \\
The initial real workspace memory = la\_init\_factor $*$ memory required by unfactored system. 
Ipopt will increase the workspace size by meminc\_factor if required.
The valid range for this real option is $1 \le {\tt ma27\_la\_init\_factor } <  {\tt +inf}$ and its default value is $5$.


\paragraph{ma27\_meminc\_factor:} Increment factor for workspace size for MA27. $\;$ \\
If the integer or real workspace is not large enough, Ipopt will increase its size by this factor.
The valid range for this real option is $1 \le {\tt ma27\_meminc\_factor } <  {\tt +inf}$ and its default value is $10$.

\subsubsection{MA57 Linear Solver}

\paragraph{ma57\_pivtol:} Pivot tolerance for the linear solver MA57. $\;$ \\
A smaller number pivots for sparsity, a larger number pivots for stability.
The valid range for this real option is $0 <  {\tt ma57\_pivtol } <  1$ and its default value is $10^{-8}$.


\paragraph{ma57\_pivtolmax:} Maximum pivot tolerance for the linear solver MA57. $\;$ \\
Ipopt may increase pivtol as high as ma57\_pivtolmax to get a more accurate solution to the linear system.
The valid range for this real option is $0 <  {\tt ma57\_pivtolmax } <  1$ and its default value is $0.0001$.


\paragraph{ma57\_pre\_alloc:} Safety factor for work space memory allocation for the linear solver MA57. $\;$ \\
If 1 is chosen, the suggested amount of work space is used.
However, choosing a larger number might avoid reallocation if the suggest values do not suffice.
The valid range for this real option is $1 \le {\tt ma57\_pre\_alloc } <  {\tt +inf}$ and its default value is $3$.



\subsubsection{Hessian perturbation}

\paragraph{max\_hessian\_perturbation:} Maximum value of regularization parameter for handling negative curvature. $\;$ \\
 In order to guarantee that the search directions
are indeed proper descent directions, Ipopt
requires that the inertia of the (augmented)
linear system for the step computation has the
correct number of negative and positive
eigenvalues. The idea is that this guides the
algorithm away from maximizers and makes Ipopt
more likely converge to first order optimal
points that are minimizers. If the inertia is not
correct, a multiple of the identity matrix is
added to the Hessian of the Lagrangian in the
augmented system. This parameter gives the
maximum value of the regularization parameter. If
a regularization of that size is not enough, the
algorithm skips this iteration and goes to the
restoration phase. (This is $\delta_w^{\max}$ in the
implementation paper.) The valid range for this real option is 
$0 <  {\tt max\_hessian\_perturbation } <  {\tt +inf}$
and its default value is $1 \cdot 10^{+20}$.


\paragraph{min\_hessian\_perturbation:} Smallest perturbation of the Hessian block. $\;$ \\
 The size of the perturbation of the Hessian block
is never selected smaller than this value, unless
no perturbation is necessary. (This is
$\delta_w^{\min}$ in the implementation paper.) The valid range for this real option is 
$0 \le {\tt min\_hessian\_perturbation } <  {\tt +inf}$
and its default value is $1 \cdot 10^{-20}$.


\paragraph{first\_hessian\_perturbation:} Size of first x-s perturbation tried. $\;$ \\
 The first value tried for the x-s perturbation in
the inertia correction scheme.(This is $\delta_0$
in the implementation paper.) The valid range for this real option is 
$0 <  {\tt first\_hessian\_perturbation } <  {\tt +inf}$
and its default value is $0.0001$.


\paragraph{perturb\_inc\_fact\_first:} Increase factor for x-s perturbation for very first perturbation. $\;$ \\
 The factor by which the perturbation is increased
when a trial value was not sufficient - this
value is used for the computation of the very
first perturbation and allows a different value
for for the first perturbation than that used for
the remaining perturbations. (This is
$\bar\kappa_w^+$ in the implementation paper.) The valid range for this real option is 
$1 <  {\tt perturb\_inc\_fact\_first } <  {\tt +inf}$
and its default value is $100$.


\paragraph{perturb\_inc\_fact:} Increase factor for x-s perturbation. $\;$ \\
 The factor by which the perturbation is increased
when a trial value was not sufficient - this
value is used for the computation of all
perturbations except for the first. (This is
$\kappa_w^+$ in the implementation paper.) The valid range for this real option is 
$1 <  {\tt perturb\_inc\_fact } <  {\tt +inf}$
and its default value is $8$.


\paragraph{perturb\_dec\_fact:} Decrease factor for x-s perturbation. $\;$ \\
 The factor by which the perturbation is decreased
when a trial value is deduced from the size of
the most recent successful perturbation. (This is
$\kappa_w^-$ in the implementation paper.) The valid range for this real option is 
$0 <  {\tt perturb\_dec\_fact } <  1$
and its default value is $0.333333$.


\paragraph{jacobian\_regularization\_value:} Size of the regularization for rank-deficient constraint Jacobians. $\;$ \\
 (This is $\bar\delta_c$ in the implementation
paper.) The valid range for this real option is\\ 
$0 \le {\tt jacobian\_regularization\_value } <  {\tt +inf}$
and its default value is $1 \cdot 10^{-08}$.


\paragraph{jacobian\_regularization\_exponent:} Exponent for mu in the regularization for rank-deficient constraint Jacobians. $\;$ \\
 (This is $\kappa_c$ in the implementation paper.) The default value for this real option is $0.25$
and its valid range is $0 \le {\tt jacobian\_regularization\_exponent } <  {\tt +inf}$.


\paragraph{perturb\_always\_cd:} Active permanent perturbation of constraint linearization. $\;$ \\
 This options makes the delta\_c and delta\_d
perturbation be used for the computation of every
search direction.  Usually, it is only used when
the iteration matrix is singular.
The default value for this string option is ``no''.
\\ 
Possible values:
\begin{itemize}
   \item no: perturbation only used when required
   \item yes: always use perturbation
\end{itemize}

\subsubsection{Hessian approximation}

\paragraph{hessian\_approximation:} Indicates what Hessian information is to be used. $\;$ \\
 This determines which kind of information for the
Hessian of the Lagrangian function is used by the
algorithm.
The default value for this string option is to use ``exact'' if the GAMS system is able to provide a hessian, and ``limited-memory'' otherwise (a warning is issued in this case).
\\ 
Possible values:
\begin{itemize}
   \item exact: Use second derivatives provided by the NLP.
   \item limited-memory: Perform a limited-memory quasi-Newton
approximation
\end{itemize}

\paragraph{hessian\_approximation\_space:} Indicates in which subspace the Hessian information is to be approximated. \\
The default value for this string option is ``nonlinear-variables''.
\\ 
Possible values:
\begin{itemize}
   \item nonlinear-variables: only in space of nonlinear variables.
   \item all-variables: in space of all variables (without slacks)
\end{itemize}

\paragraph{limited\_memory\_max\_history:} Maximum size of the history for the limited quasi-Newton Hessian approximation. $\;$ \\
 This option determines the number of most recent
iterations that are taken into account for the
limited-memory quasi-Newton approximation. The valid range for this integer option is
$0 \le {\tt limited\_memory\_max\_history } <  {\tt +inf}$
and its default value is $6$.


\paragraph{limited\_memory\_update\_type:} Quasi-Newton update formula for the limited memory approximation. $\;$ \\
 Determines which update formula is to be used for
the limited-memory quasi-Newton approximation.
The default value for this string option is ``bfgs''.
\\ 
Possible values:
\begin{itemize}
   \item bfgs: BFGS update (with skipping)
   \item sr1: SR1 (not working well)
\end{itemize}

\paragraph{limited\_memory\_initialization:} Initialization strategy for the limited memory quasi-Newton approximation. $\;$ \\
 Determines how the diagonal Matrix B\_0 as the
first term in the limited memory approximation
should be computed.
The default value for this string option is ``scalar1''.
\\ 
Possible values:
\begin{itemize}
   \item scalar1: sigma = $s^Ty/s^Ts$
   \item scalar2: sigma = $y^Ty/s^Ty$
   \item constant: sigma = limited\_memory\_init\_val
\end{itemize}

\paragraph{limited\_memory\_init\_val:} Value for B0 in low-rank update. $\;$ \\
 The starting matrix in the low rank update, B0,
is chosen to be this multiple of the identity in
the first iteration (when no updates have been
performed yet), and remains constant at this
value, if ``limited\_memory\_initialization'' is
``constant''. The valid range for this real option is 
$0 <  {\tt limited\_memory\_init\_val } <  {\tt +inf}$
and its default value is $1$.


\paragraph{limited\_memory\_max\_skipping:} Threshold for successive iterations where update is skipped. $\;$ \\
 If the update is skipped more than this number of
successive iterations, we quasi-Newton
approximation is reset. The valid range for this integer option is
$1 \le {\tt limited\_memory\_max\_skipping } <  {\tt +inf}$
and its default value is $2$.


% \paragraph{derivative\_test:} Enable derivative checker $\;$ \\
%  If this option is enabled, a (slow) derivative
% test will be performed before the optimization. 
% The test is performed at the user provided
% starting point and marks derivative values that
% seem suspicious.
% The default value for this string option is ``none''.
% \\ 
% Possible values:
% \begin{itemize}
%    \item none: do not perform derivative test
%    \item first-order: perform test of first derivatives at starting
% point
%    \item second-order: perform test of first and second derivatives at
% starting point
% \end{itemize}
% 
% \paragraph{derivative\_test\_perturbation:} Size of the finite difference perturbation in derivative test. $\;$ \\
%  This determines the relative perturbation of the
% variable entries. The valid range for this real option is 
% $0 <  {\tt derivative\_test\_perturbation } <  {\tt +inf}$
% and its default value is $1 \cdot 10^{-08}$.
% 
% 
% \paragraph{derivative\_test\_tol:} Threshold for indicating wrong derivative. $\;$ \\
%  If the relative deviation of the estimated
% derivative from the given one is larger than this
% value, the corresponding derivative is marked as
% wrong. The valid range for this real option is 
% $0 <  {\tt derivative\_test\_tol } <  {\tt +inf}$
% and its default value is $0.0001$.
% 
% 
% \paragraph{derivative\_test\_print\_all:} Indicates whether information for all estimated derivatives should be printed. $\;$ \\
%  Determines verbosity of derivative checker.
% The default value for this string option is ``no''.
% \\ 
% Possible values:
% \begin{itemize}
%    \item no: Print only suspect derivatives
%    \item yes: Print all derivatives
% \end{itemize}
% 


\bibliographystyle{plain}
%\bibliography{coinlibd}
%\renewcommand{\bibname}{Ipopt References}
\chapter{\IPOPT and \IPOPTH}
\label{cha:ipopt}

%\minitoc

COIN-OR \IPOPT (\textbf{I}nterior \textbf{P}oint \textbf{Opt}imizer) is an open-source solver for large-scale nonlinear programming.
The code has been written primarily by Andreas W\"achter, who is the COIN-OR project leader for \IPOPT.

\IPOPT implements an interior point line search filter method for nonlinear programming models which functions can be nonconvex, but should be twice continuously differentiable.
For more information on the algorithm we refer to~\cite{NoWaWa08,Waechter2002,WaBi05b,WaBi05a,WaBi2006} and the \IPOPT web site \url{https://projects.coin-or.org/Ipopt}.
Most of the \IPOPT documentation in the section was taken from the \IPOPT manual~\cite{IpoptManual}.



\section{The linear solver in \IPOPT}
\label{sec:ipoptlinearsolver}
\hypertarget{ipoptlinearsolver}{}

The performance and robustness of \IPOPT on larger models heavily relies on the used solver for sparse symmetric indefinite linear systems.

\GAMS/\IPOPT includes the sparse solver \textsc{MUMPS}~\cite{AmestoyDuffKosterLExcellent2001,AmestoyGuermoucheLExcellentPralet2006} (currently the default), cf.~\url{http://graal.ens-lyon.fr/MUMPS} and \textsc{MKL PARDISO}~\cite{SchGa04,SchGa06} (only Linux and Windows).
In the commerically licensed \GAMS/\IPOPTH version, also the Harwell Subroutine Library (HSL) solvers \textsc{MA27}, \textsc{MA57}, \textsc{HSL\_MA86}, and \textsc{HSL\_MA97} are available and MA27 is used by default.

\textsc{MUMPS}, \textsc{MA57}, \textsc{HSL\_MA86}, and \textsc{HSL\_MA97} use \textsc{METIS} for matrix ordering \cite{KaKu99}, cf.~\url{http://glaros.dtc.umn.edu/gkhome/views/metis/index.html} and \url{http://glaros.dtc.umn.edu/gkhome/fetch/sw/metis/manual.pdf}.
\textsc{METIS} is copyrighted by the regents of the University of Minnesota.

\IPOPT and \IPOPTH can exploit parallelization of the linear solver or the linear algebra routines (Blas and Lapack).
The following table summarizes which options are available on which platform.

\begin{tabular}{l|c|cc|cccc}
& \multicolumn{3}{c|}{\IPOPT and \IPOPTH} & \multicolumn{4}{c}{\IPOPTH only} \\
        & Linear Algebra & MUMPS & MKL PARDISO & MA27 & MA57 & HSL MA86 & HSL MA97 \\ \hline
Linux   & parallel & serial & parallel      & serial & serial & parallel & parallel \\
MacOS X & parallel & serial & not available & serial & serial & parallel & parallel  \\
Solaris & serial   & serial & not available & serial & serial & parallel & parallel  \\
Windows & parallel & serial & parallel      & serial & serial & parallel & parallel  \\
\end{tabular}

The linear solver is chosen by the \texttt{linear\_solver} option.
Benchmarks have shown that \textsc{MA57} and \textsc{HSL\_MA97} are often able to outperform \textsc{MA27} on larger instances. Further, \textsc{PARDISO} often allows for performance that is better than \textsc{MUMPS} and similar to the HSL solvers. If \IPOPT fails to solve an instance with \textsc{PARDISO}, it's worth to try changing the options \texttt{pardiso\_order} and \texttt{pardiso\_max\_iterative\_refinement\_steps}.

\section{Usage}

The following statement can be used inside your \GAMS program to specify using \IPOPT
\begin{verbatim}
  Option NLP = IPOPT;     { or LP, RMIP, DNLP, RMINLP, QCP, RMIQCP }
\end{verbatim}

The above statement should appear before the Solve statement.
If \IPOPT was specified as the default solver during \GAMS installation, the above statement is not necessary.

To use \IPOPTH, the statement should be
\begin{verbatim}
  Option NLP = IPOPTH;    { or LP, RMIP, DNLP, RMINLP, QCP, RMIQCP }
\end{verbatim}



\paragraph{Using Harwell Subroutine Library routines with \GAMS/\IPOPT.}

\GAMS/\IPOPT can use the HSL routines \texttt{MA27}, \texttt{MA28}, \texttt{MA57}, \textsc{HSL\_MA77}, \textsc{HSL\_MA86}, \textsc{HSL\_MA97}, \texttt{MC19}, and \textsc{HSL\_MC68} when provided as shared library.
By telling \IPOPT to use one of these routines (see options \texttt{linear\_solver}, \texttt{linear\_system\_scaling}, \texttt{nlp\_scaling\_method}, \texttt{dependency\_detector}), \GAMS/\IPOPT attempts to load the required routines from the library \texttt{libhsl.so} (Unix-Systems), \texttt{libhsl.dylib} (MacOS X), or \texttt{libhsl.dll} (Windows), respectively.

The HSL routines are available at \url{http://www.hsl.rl.ac.uk/ipopt}.
Note that it is your responsibility to ensure that you are entitled to download and use these routines!
% You can build a shared library using the ThirdParty/HSL project at COIN-OR.

\paragraph{Using PARDISO with \GAMS/\IPOPT or \GAMS/\IPOPTH.}
On Mac OS X and Solaris, setting the option \texttt{linear\_solver} to \texttt{pardiso} lets \GAMS/\IPOPT or \GAMS/\IPOPTH try to load the linear solver PARDISO from the library \texttt{libpardiso.so} (Unix) or \texttt{libpardiso.dylib} (MacOS X), respectively.

PARDISO is available as compiled shared library for several platforms at \texttt{http://www.pardiso-project.org}.
Note that it is your responsibility to ensure that you are entitled to download and use this package!

\subsection{Specification of Options}
\label{sub:ipoptoptionspec}

\IPOPT has many options that can be adjusted for the algorithm (see Section \ref{sub:ipoptoptions}).
Options are all identified by a string name, and their values can be of one of three types: Number (real), Integer, or String.
Number options are used for things like tolerances, integer options are used for things like maximum number of iterations, and string options are used for setting algorithm details, like the NLP scaling method.
Options can be set by creating a \texttt{ipopt.opt} file in the directory you are executing \IPOPT.

The \texttt{ipopt.opt} file is read line by line and each line should contain the option name, followed by whitespace, and then the value.
Comments can be included with the \# symbol. Don't forget to ensure you have a newline at the end of the file. For example,
\begin{verbatim}
# This is a comment

# Turn off the NLP scaling
nlp_scaling_method none

# Change the initial barrier parameter
mu_init 1e-2

# Set the max number of iterations
max_iter 500
\end{verbatim}
is a valid \texttt{ipopt.opt} file.

% You can print the documentation for all \IPOPT options by using the option
% \begin{verbatim}
% print_options_documentation yes
% \end{verbatim}
% and running \IPOPT.
% This will output all of the options documentation to the console.

\GAMS/\IPOPT understand currently the following \GAMS parameters: \texttt{reslim} (time limit), \texttt{iterlim} (iteration limit), \texttt{domlim} (domain violation limit).
You can set them either on the command line, e.g. \verb+iterlim=500+, or inside your \GAMS program, e.g. \verb+Option iterlim=500;+.
Further the option \texttt{threads} can be used to control the number of threads used in the linear algebra routines and the linear solver, see also Section~\ref{sec:ipoptlinearsolver}.

\subsection{Warmstarting Ipopt}

As an interior point solver, it is difficult to warm start \IPOPT.
By default, only the level values of the variables are passed as starting point to \IPOPT.
Setting the \IPOPT option \texttt{warm\_start\_init\_point} to \texttt{yes} enables that also dual values for variables and constraints are passed to \IPOPT.

However, the expected behavior that \IPOPT finishes within one iteration if optimal primal and dual values are passed is not reached this way, yet. This is, because \IPOPT by default moves any initial value that is close to a bound into the interior. The amount on how much the initial point is moved can be controlled by various \texttt{bound\_push} and \texttt{bound\_frac} options.
To make \IPOPT accept an optimal primal/dual solution within one iteration, it should be sufficient to set the following options:
\begin{verbatim}
  warm_start_init_point       yes
  warm_start_bound_push       1e-9
  warm_start_bound_frac       1e-9
  warm_start_slack_bound_frac 1e-9
  warm_start_slack_bound_push 1e-9
  warm_start_mult_bound_push  1e-9
\end{verbatim}

\section{Output}

This section describes the standard \IPOPT console output.
The output is designed to provide a quick summary of each iteration as \IPOPT solves the problem.

Before \IPOPT starts to solve the problem, it displays the problem statistics (number of nonzero-elements in the matrices, number of variables, etc.).
Note that if you have fixed variables (both upper and lower bounds are equal), \IPOPT may remove these variables from the problem internally and not include them in the problem statistics.

Following the problem statistics, \IPOPT will begin to solve the problem and you will see output resembling the following,
\begin{verbatim}
iter    objective    inf_pr   inf_du lg(mu)  ||d||  lg(rg) alpha_du alpha_pr  ls
   0  1.6109693e+01 1.12e+01 5.28e-01   0.0 0.00e+00    -  0.00e+00 0.00e+00   0
   1  1.8029749e+01 9.90e-01 6.62e+01   0.1 2.05e+00    -  2.14e-01 1.00e+00f  1
   2  1.8719906e+01 1.25e-02 9.04e+00  -2.2 5.94e-02   2.0 8.04e-01 1.00e+00h  1
\end{verbatim}
and the columns of output are defined as
\begin{description}
\item[iter]
The current iteration count.
This includes regular iterations and iterations while in restoration phase.
If the algorithm is in the restoration phase, the letter \texttt{r} will be appended to the iteration number.
\item[objective]
The unscaled objective value at the current point.
During the restoration phase, this value remains the unscaled objective value for the original problem.
\item[inf\_pr]
The unscaled constraint violation at the current point.
This quantity is the infinity-norm (max) of the (unscaled) constraint violation.
During the restoration phase, this value remains the constraint violation of the original problem at the current point.
The option ``\texttt{inf\_pr\_output}'' can be used to switch to the printing of a different quantity.
During the restoration phase, this value is the primal infeasibility of the original problem at the current point.
\item[inf\_du]
The scaled dual infeasibility at the current point.
This quantity measure the infinity-norm (max) of the internal dual infeasibility \cite[Eq.~(4a)]{WaBi2006}, including inequality constraints reformulated using slack variables and problem scaling.
During the restoration phase, this is the value of the dual infeasibility for the restoration phase problem.
\item[lg(mu)]
$\log_{10}$ of the value of the barrier parameter $\mu$.
\item[$\Vert d\Vert$]
The infinity norm (max) of the primal step (for the original variables $x$ and the internal slack variables $s$).
During the restoration phase, this value includes the values of additional variables, $p$ and $n$ \cite[Eq.~(10)]{WaBi2006}.
\item[lg(rg)]
$\log_{10}$ of the value of the regularization term for the Hessian of the Lagrangian in the augmented system ($\delta_w$ in \cite[Eq.~(26)]{WaBi2006}).
A dash (``\texttt{-}'') indicates that no regularization was done.
\item[alpha\_du]
The stepsize for the dual variables ($\alpha^z_k$ in \cite[Eq.~(14c)]{WaBi2006})..
\item[alpha\_pr]
The stepsize for the primal variables ($\alpha_k$ in \cite[Eq.~(14a)]{WaBi2006}).
The number is usually followed by a character for additional diagnostic information regarding the step acceptance criterion:
 \begin{list}{blub}{\itemsep0pt}
    \item[\texttt{f}] f-type iteration in the filter method w/o second order correction
    \item[\texttt{F}] f-type iteration in the filter method w/ second order correction
    \item[\texttt{h}] h-type iteration in the filter method w/o second order correction
    \item[\texttt{H}] h-type iteration in the filter method w/ second order correction
    \item[\texttt{k}] penalty value unchanged in merit function method w/o second order correction
    \item[\texttt{K}] penalty value unchanged in merit function method w/ second order correction
    \item[\texttt{n}] penalty value updated in merit function method w/o second order correction
    \item[\texttt{N}] penalty value updated in merit function method w/ second order correction
    \item[\texttt{R}] Restoration phase just started
    \item[\texttt{w}] in watchdog procedure
    \item[\texttt{s}] step accepted in soft restoration phase
    \item[\texttt{t}/\texttt{T}] tiny step accepted without line search
    \item[\texttt{r}] some previous iterate restored
 \end{list}
\item[ls]
The number of backtracking line search steps (does not include second-order correction steps).
\end{description}

Note that the step acceptance mechanisms in \IPOPT consider the
barrier objective function \cite[Eq.~(3a)]{WaBi2006} which is
usually different from the value reported in the \texttt{objective}
column.  Similarly, for the purposes of the step acceptance, the
constraint violation is measured for the internal problem formulation,
which includes slack variables for inequality constraints and
potentially scaling of the constraint functions.  This value, too, is
usually different from the value reported in \texttt{inf\_pr}.  As a
consequence, a new iterate might have worse values both for the
objective function and the constraint violation as reported in the
iteration output, seemingly contradicting globalization procedure.


When the algorithm terminates, \IPOPT will output a message to the screen.
The following is a list of the possible output messages and a brief description.

\begin{description}
\item[Optimal Solution Found.] ~

    This message indicates that \IPOPT found a (locally) optimal point within the desired tolerances.

\item[Solved To Acceptable Level.] ~

    This indicates that the algorithm did not converge to the ``desired'' tolerances, but that it was able to obtain a point satisfying the ``acceptable'' tolerance level as specified by \texttt{acceptable-*} options.
    This may happen if the desired tolerances are too small for the current problem.

\item[Feasible point for square problem found.] ~

    This message is printed if the problem is ``square'' (i.e., it has as many equality constraints as free variables) and \IPOPT found a feasible point.

\item[Converged to a point of local infeasibility. Problem may be infeasible.] ~

    The restoration phase converged to a point that is a minimizer for the constraint violation (in the $\ell_1$-norm), but is not feasible for the original problem.
    This indicates that the problem may be infeasible (or at least that the algorithm is stuck at a locally infeasible point).
    The returned point (the minimizer of the constraint violation) might help you to find which constraint is causing the problem.
    If you believe that the NLP is feasible, it might help to start the optimization from a different point.

\item[Search Direction is becoming Too Small.] ~

    This indicates that \IPOPT is calculating very small step sizes and making very little progress.
    This could happen if the problem has been solved to the best numerical accuracy possible given the current scaling.

\item[Iterates divering; problem might be unbounded.] ~

    This message is printed if the max-norm of the iterates becomes larger than the value of the option \texttt{diverging\_iterates\_tol}.
    This can happen if the problem is unbounded below and the iterates are diverging.

\item[Stopping optimization at current point as requested by user.] ~

    This message is printed if either the Ctrl+C was pressed or the domain violation limit is reached.

\item[Maximum Number of Iterations Exceeded.] ~

    This indicates that \IPOPT has exceeded the maximum number of iterations as specified by the \IPOPT option \texttt{max\_iter} or the GAMS option \texttt{iterlim}.

\item[Maximum CPU time exceeded.] ~

    This indicates that \IPOPT has exceeded the maximum number of seconds as specified by the \IPOPT option \texttt{max\_cpu\_time} or the GAMS option \texttt{reslim}.

\item[Restoration Failed!] ~

    This indicates that the restoration phase failed to find a feasible point that was acceptable to the filter line search for the original problem.
    This could happen if the problem is highly degenerate or does not satisfy the constraint qualification, or if an external function in \GAMS provides incorrect derivative information.

\item[Error in step computation (regularization becomes too large?)!] ~

    This messages is printed if \IPOPT is unable to compute a search direction, despite several attempts to modify the iteration matrix.
    Usually, the value of the regularization parameter then becomes too large.

\item[Problem has too few degrees of freedom.] ~

    This indicates that your problem, as specified, has too few degrees of freedom.
    This can happen if you have too many equality constraints, or if you fix too many variables (\IPOPT removes fixed variables).

\item[Not enough memory.] ~

    An error occurred while trying to allocate memory.
    The problem may be too large for your current memory and swap configuration.

\item[INTERNAL ERROR: Unknown SolverReturn value - Notify \IPOPT Authors.] ~

    An unknown internal error has occurred. Please notify the authors of the \GAMS/\IPOPT link or \IPOPT (refer to \url{https://projects.coin-or.org/GAMSlinks} or \url{https://projects.coin-or.org/Ipopt}).
\end{description}


\subsection{Diagnostic Tags for \IPOPT}

To print additional diagnostic tags for each iteration of \IPOPT, set
the options \texttt{print\_info\_string} to \texttt{yes}. With
this, a tag will appear at the end of an iteration line with the
following diagnostic meaning that are useful to flag difficulties for
a particular \IPOPT run.  The following is a list of possible strings:
\begin{list}{blub}{\itemsep0pt}
 \item[\texttt{!}] Tighten resto tolerance if only slightly infeasible \cite[Sec.~3.3]{WaBi2006}
 \item[\texttt{A}] Current iteration is acceptable (alternate termination)
 \item[\texttt{a}] Perturbation for PD Singularity can't be done, assume singular \cite[Sec.~3.1]{WaBi2006}
 \item[\texttt{C}] Second Order Correction taken \cite[Sec.~2.4]{WaBi2006}
 \item[\texttt{Dh}] Hessian degenerate based on multiple iterations \cite[Sec.~3.1]{WaBi2006}
 \item[\texttt{Dhj}] Hessian/Jacobian degenerate based on multiple iterations \cite[Sec.~3.1]{WaBi2006}
 \item[\texttt{Dj}] Jacobian degenerate based on multiple iterations \cite[Sec.~3.1]{WaBi2006}
 \item[\texttt{dx}] $\delta_x$ perturbation too large \cite[Sec.~3.1]{WaBi2006}
 \item[\texttt{e}] Cutting back $\alpha$ due to evaluation error (in backtracking line search)
 \item[\texttt{F-}] Filter should be reset, but maximal resets exceeded \cite[Sec.~2.3]{WaBi2006}
 \item[\texttt{F+}] Resetting filter due to last few rejections of filter \cite[Sec.~2.3]{WaBi2006}
 \item[\texttt{L}] Degenerate Jacobian, $\delta_c$ already perturbed \cite[Sec.~3.1]{WaBi2006}
 \item[\texttt{l}] Degenerate Jacobian, $\delta_c$ perturbed \cite[Sec.~3.1]{WaBi2006}
 \item[\texttt{M}] Magic step taken for slack variables (in backtracking line search)
 \item[\texttt{Nh}] Hessian not yet degenerate \cite[Sec.~3.1]{WaBi2006}
 \item[\texttt{Nhj}] Hessian/Jacobian not yet degenerate \cite[Sec.~3.1]{WaBi2006}
 \item[\texttt{Nj}] Jacobian not yet degenerate \cite[Sec.~3.1]{WaBi2006}
 \item[\texttt{NW}] Warm start initialization failed (in Warm Start Initialization)
 \item[\texttt{q}] PD system possibly singular, attempt to improve solution quality \cite[Sec.~3.1]{WaBi2006}
 \item[\texttt{R}] Solution of restoration phase \cite[Sec.~3.3]{WaBi2006}
 \item[\texttt{S}] PD system possibly singular, accept current solution \cite[Sec.~3.1]{WaBi2006}
 \item[\texttt{s}] PD system singular \cite[Sec.~3.1]{WaBi2006}
 \item[\texttt{s}] Square Problem. Set multipliers to zero (default initialization routine)
 \item[\texttt{Tmax}] Trial $\theta$ is larger than $\theta_{max}$ (filter parameter \cite[Eq.~(21)]{WaBi2006})
 \item[\texttt{W}] Watchdog line search procedure successful \cite[Sec.~3.2]{WaBi2006}
 \item[\texttt{w}] Watchdog line search procedure unsuccessful, stopped \cite[Sec.~3.2]{WaBi2006}
 \item[\texttt{Wb}] Undoing most recent SR1 update \cite[Sec.~5.4.1]{Biegler2010}
 \item[\texttt{We}] Skip Limited-Memory Update in restoration phase  \cite[Sec.~5.4.1]{Biegler2010}
 \item[\texttt{Wp}] Safeguard $B^0 = \sigma I$ for  Limited-Memory Update \cite[Sec.~5.4.1]{Biegler2010}
 \item[\texttt{Wr}] Resetting Limited-Memory Update \cite[Sec.~5.4.1]{Biegler2010}
 \item[\texttt{Ws}] Skip Limited-Memory Update since $s^Ty$ is not positive \cite[Sec.~5.4.1]{Biegler2010}
 \item[\texttt{WS}] Skip Limited-Memory Update since $\Delta x$ is too small \cite[Sec.~5.4.1]{Biegler2010}
 \item[\texttt{y}] Dual infeasibility, use least square multiplier update (during \IPOPT algorithm)
 \item[\texttt{z}] Apply correction to bound multiplier if too large (during \IPOPT algorithm)
\end{list}

\section{Detailed Options Description}
\label{sub:ipoptoptions}

% Note, that \GAMS/\IPOPT overwrites the \IPOPT default setting for the parameters \texttt{bound\_relax\_factor} (set to $10^{-10}$) and \texttt{mu\_strategy} (set to \texttt{adaptive}).
% You can change these values by specifying these options in your \IPOPT options file.

\input{optipopt_a}

\bibliographystyle{plain}
%\bibliography{coinlibd}
%\renewcommand{\bibname}{Ipopt References}
\input{ipopt.bbl}

\chapterend


\chapterend


\chapterend


\chapterend

\subsection{Summary of CoinCBC Options}

Among all CBC options, the following GAMS parameters are currently supported in CoinCBC:
\hyperlink{reslim}{reslim}, \hyperlink{iterlim}{iterlim}, \hyperlink{nodlim}{nodlim}, \hyperlink{optca}{optca}, \hyperlink{optcr}{optcr}, \hyperlink{increment}{cheat}, \hyperlink{cutoff}{cutoff}.

Currently CoinCBC understands the following options:

\subsubsection{General Options}
\begin{tabbing}
\hspace {1.3in} \= \\
\hyperlink{iterlim}
{iterlim} \> iteration limit \\
\hyperlink{reslim}
{reslim} \> resource limit \\
\hyperlink{special}
{special} \> options passed unseen to CBC \\
\hyperlink{writemps}
{writemps} \> create MPS file for problem \\
\end{tabbing}

\subsubsection{LP Options}
\begin{tabbing}
\hspace {1.3in} \= \\
\hyperlink{idiotcrash}
{idiotcrash} \> idiot crash \\
\hyperlink{sprintcrash}
{sprintcrash} \> sprint crash \\
\hyperlink{sifting}
{sifting} \> synonym for sprint crash \\
\hyperlink{crash}
{crash} \> use crash method to get dual feasible \\
\hyperlink{maxfactor}
{maxfactor} \> maximum number of iterations between refactorizations \\
\hyperlink{crossover}
{crossover} \> crossover to simplex algorithm after barrier \\
\hyperlink{dualpivot}
{dualpivot} \> dual pivot choice algorithm \\
\hyperlink{primalpivot}
{primalpivot} \> primal pivot choice algorithm \\
\hyperlink{perturbation}
{perturbation} \> perturbation of problem \\
\hyperlink{scaling}
{scaling} \> scaling method \\
\hyperlink{presolve}
{presolve} \> switch for initial presolve of LP \\
\hyperlink{tol_dual}
{tol\_dual} \> dual feasibility tolerance \\
\hyperlink{tol_primal}
{tol\_primal} \> primal feasibility tolerance \\
\hyperlink{tol_presolve}
{tol\_presolve} \> tolerance used in presolve \\
\hyperlink{startalg}
{startalg} \> LP solver for root node \\
\end{tabbing}


\subsubsection{MIP Options}
\begin{tabbing}
\hspace {1.3in} \= \\
\hyperlink{tol_integer}
{tol\_integer} \> tolerance for integrality \\
\hyperlink{sollim}
{sollim} \> limit on number of solutions \\
\hyperlink{strongbranching}
{strongbranching} \> strong branching \\
\hyperlink{trustpseudocosts}
{trustpseudocosts} \> after howmany nodes we trust the pseudo costs \\
\hyperlink{coststrategy}
{coststrategy} \> how to use costs as priorities \\
\hyperlink{nodestrategy}
{nodestrategy} \> how to select nodes \\
\hyperlink{preprocess}
{preprocess} \> integer presolve \\
\hyperlink{printfrequency}
{printfrequency} \> frequency of status prints \\
\hyperlink{increment}
{increment} \> increment of cutoff when new incumbent \\
\hyperlink{nodelim}
{nodelim} \> node limit \\
\hyperlink{nodlim}
{nodlim} \> node limit \\
\hyperlink{optca}
{optca} \> absolute stopping tolerance \\
\hyperlink{optcr}
{optcr} \> relative stopping tolerance \\
\hyperlink{cutoff}
{cutoff} \> cutoff for objective function value \\
\end{tabbing}


\subsubsection{MIP Options for Cutting Plane Generators}
\begin{tabbing}
\hspace {1.3in} \= \\
\hyperlink{cutdepth}
{cutdepth} \> depth in tree at which cuts are applied \\
\hyperlink{cut_passes_root}
{cut\_passes\_root} \> number of cut passes at root node \\
\hyperlink{cut_passes_tree}
{cut\_passes\_tree} \> number of cut passes at nodes in the tree \\
\hyperlink{cuts}
{cuts} \> global switch for cutgenerators \\
\hyperlink{cliquecuts}
{cliquecuts} \> Clique Cuts \\
\hyperlink{flowcovercuts}
{flowcovercuts} \> Flow Cover Cuts \\
\hyperlink{gomorycuts}
{gomorycuts} \> Gomory Cuts \\
\hyperlink{knapsackcuts}
{knapsackcuts} \> Knapsack Cover Cuts \\
\hyperlink{liftandprojectcuts}
{liftandprojectcuts} \> Lift and Project Cuts \\
\hyperlink{mircuts}
{mircuts} \> Mixed Integer Rounding Cuts \\
\hyperlink{twomircuts}
{twomircuts} \> Two Phase Mixed Integer Rounding Cuts \\
\hyperlink{probingcuts}
{probingcuts} \> Probing Cuts \\
\hyperlink{reduceandsplitcuts}
{reduceandsplitcuts} \> Reduce and Split Cuts \\
\hyperlink{residualcapacitycuts}
{residualcapacitycuts} \> Residual Capacity Cuts \\
\end{tabbing}


\subsubsection{MIP Options for Heuristics}
\begin{tabbing}
\hspace {1.3in} \= \\
\hyperlink{heuristics}
{heuristics} \> global switch for heuristics \\
\hyperlink{combinesolutions}
{combinesolutions} \> combine solutions heuristic \\
\hyperlink{feaspump}
{feaspump} \> feasibility pump \\
\hyperlink{feaspump_passes}
{feaspump\_passes} \> number of feasibility passes \\
\hyperlink{greedyheuristic}
{greedyheuristic} \> greedy heuristic \\
\hyperlink{localtreesearch}
{localtreesearch} \> local tree search heuristic \\
\hyperlink{rins}
{rins} \> relaxed induced neighborhood search \\
\hyperlink{roundingheuristic}
{roundingheuristic} \> rounding heuristic \\
\end{tabbing}


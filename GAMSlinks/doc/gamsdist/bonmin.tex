\chapter{\BONMIN and \BONMINH}

%\minitoc

COIN-OR \BONMIN (\textbf{B}asic \textbf{O}pen-source \textbf{N}onlinear \textbf{M}ixed \textbf{In}teger programming) is an open-source solver for mixed-integer nonlinear programming (MINLPs).
The code has been developed as part of a collaboration between Carnegie Mellon University and IBM Research.
The COIN-OR project leader for \BONMIN is Pierre Bonami.

\BONMIN can handle mixed-integer nonlinear programming models which functions should be twice continuously differentiable.
The \BONMIN link in \GAMS supports continuous, binary, and integer variables, special ordered sets, branching priorities, but no semi-continuous or semi-integer variables (see chapter 17.1 of the \GAMS User's Guide).


\BONMIN implements six different algorithms for solving MINLPs:
\begin{itemize}
\setlength{\partopsep}{0pt}
\setlength{\itemsep}{0pt}
\item {B-BB} (\textbf{default}): a simple branch-and-bound algorithm based on solving a continuous nonlinear program at each node of the search tree and branching on integer variables~\cite{GuptaRavindran85}; this algorithm is similar to the one implemented in the solver \textsc{SBB}
\item {B-OA}: an outer-approximation based decomposition algorithm based on iterating solving and improving of a MIP relaxation and solving NLP subproblems~\cite{DuGr86,FlLe94}; this algorithm is similar to the one implemented in the solver \textsc{DICOPT}
\item {B-QG}: an outer-approximation based branch-and-cut algorithm based on solving a continuous linear program at each node of the search tree, improving the linear program by outer approximation, and branching on integer variables~\cite{QeGr92}.
\item {B-Hyb}: a branch-and-bound algorithm which is a hybrid of B-BB and B-QG and is based on solving either a continuous nonlinear or a continuous linear program at each node of the search tree, improving the linear program by outer approximation, and branching on integer variables~\cite{BBCCGLLLMSW}
\item {B-ECP}: a Kelley's outer-approximation based branch-and-cut algorithm inspired by the settings used in the solver \textsc{FilMINT}~\cite{AbLeLi07}
\item {B-iFP}: an iterated feasibility pump algorithm~\cite{BoCoLoMa06}
\end{itemize}
The algorithms are exact when the problem is \textbf{convex}, otherwise they are heuristics.

For convex MINLPs, experiments on a reasonably large test set of problems have shown that B-Hyb is the algorithm of choice (it solved most of the problems in 3 hours of computing time).
Nevertheless, there are cases where B-OA (especially when used with CPLEX as MIP subproblem solver) is much faster than B-Hyb and others where B-BB is interesting.
B-QG and B-ECP corresponds mainly to a specific parameter setting of B-Hyb but they can be faster in some cases.
B-iFP is more tailored at finding quickly good solutions to very hard convex MINLP.
For \textbf{nonconvex} MINLPs, it is strongly recommended to use B-BB (the outer-approximation algorithms have not been tailored to treat nonconvex problems at this point).
Although even B-BB is only a heuristic for such problems, several options are available to try and improve the quality of the solutions it provides (see below).

NLPs are solved in \BONMIN by \IPOPT, which can use \textsc{MUMPS}~\cite{bonminAmestoyDuffKosterLExcellent2001,bonminAmestoyGuermoucheLExcellentPralet2006} (currently the default) or \textsc{MKL PARDISO}~\cite{bonminSchGa04,bonminSchGa06} (only Linux and Windows) as linear solver.
In the commerically licensed \GAMS/\BONMINH version, also the linear solvers \textsc{MA27}, \textsc{MA57}, \textsc{HSL\_MA86}, and \textsc{HSL\_MA97} from the Harwell Subroutines Library (HSL) are available in \IPOPT.
In this case, the default linear solver in \IPOPT is MA27.


For more information we refer to \cite{BoCoLoMa06,BoGo08,BoKiLi09,BBCCGLLLMSW} and the \BONMIN web site \texttt{https://projects.coin-or.org/Bonmin}.
Most of the \BONMIN documentation in this section is taken from the \BONMIN manual~\cite{BonminManual}.


%If \GAMS/\BONMIN is called for a model with only continuous variables, the interface switches over to \IPOPT.
%If \GAMS/\BONMIN is called for a model with only linear equations, the interface switches over to \CBC.

\section{Usage}

The following statement can be used inside your \GAMS program to specify using \BONMIN:
\begin{verbatim}
  Option MINLP = BONMIN;    { or Option MIQCP = BONMIN; }
\end{verbatim}
This statement should appear before the \texttt{Solve} statement.
If \BONMIN was specified as the default solver during \GAMS installation, the above statement is not necessary.

To use \BONMINH, one should use the statement
\begin{verbatim}
  Option MINLP = BONMINH;   { or Option MIQCP = BONMINH; }
\end{verbatim}

\GAMS/\BONMIN currently does not support the \GAMS Branch-and-Cut-and-Heuristic (BCH) Facility.
If you need to use \GAMS/\BONMIN with BCH, please consider to use a \GAMS system of version $\leq 23.3$, available at \url{http://www.gams.com/download/download_old.htm}.
% \GAMS/\BONMIN supports the \GAMS Branch-and-Cut-and-Heuristic (BCH) Facility.
% The \GAMS BCH facility automates all major steps necessary to define, execute, and control the use of user defined routines within the framework of general purpose MIP and MINLP codes.
% Currently supported are user defined cut generators and heuristics, where cut generator cannot be used in Bonmins pure B\&B algorithm (B-BB).
% Please see the BCH documentation at \texttt{http://www.gams.com/docs/bch.htm} for further information.

\subsection{Specification of Options}

A \BONMIN options file contains both \IPOPT and \BONMIN options, for clarity all \BONMIN options should be preceded with the prefix ``\texttt{bonmin.}''. %, except those corresponding to the BCH facility.
The scheme to name option files is the same as for all other \GAMS solvers.
Specifying \texttt{optfile=1} let \GAMS/\BONMIN read \texttt{bonmin.opt}, \texttt{optfile=2} corresponds to \texttt{bonmin.op2}, and so on.
The format of the option file is the same as for \IPOPT (see Section \ref{sub:ipoptoptionspec} in Chapter \ref{cha:ipopt}).

The most important option in \BONMIN is the choice of the solution algorithm.
This can be set by using the option named \texttt{bonmin.algorithm} which can be set to \texttt{B-BB}, \texttt{B-OA}, \texttt{B-QG}, \texttt{B-Hyb}, \texttt{B-ECP}, or \texttt{B-iFP} (its default value is \texttt{B-BB}).
Depending on the value of this option, certain other options may be available or not, cf.\ Section~\ref{sub:bonminalloptions}.

An example of a \texttt{bonmin.opt} file is the following:
\begin{verbatim}
   bonmin.algorithm       B-Hyb
   bonmin.oa_log_level    4
   print_level            6
\end{verbatim}
%    bonmin.milp_subsolver  Cbc_Par
%    milp_sub.cover_cuts    0
%    userheurcall           "bchheur.gms reslim 10"
This sets the algorithm to be used to the hybrid algorithm, the level of outer approximation related output to $4$,
% the MIP subsolver for outer approximation to a parameterized version of CBC, switches off cover cutting planes for the MIP subsolver,
and sets the print level for \IPOPT to $6$.
%  and let \BONMIN call a user defined heuristic specified in the model \texttt{bchheur.gms} with a timelimit of 10 seconds.

\GAMS/\BONMIN understands currently the following \GAMS parameters: \texttt{reslim} (time limit), \texttt{iterlim} (iteration limit), \texttt{nodlim} (node limit), \texttt{cutoff}, \texttt{optca} (absolute gap tolerance), and \texttt{optcr} (relative gap tolerance).
One can set them either on the command line, e.g. \verb+nodlim=1000+, or inside your \GAMS program, e.g. \verb+Option nodlim=1000;+.
Further, the option \texttt{threads} can be used to control the number of threads used in the linear algebra routines of \IPOPT, see Section~\ref{sec:ipoptlinearsolver} in Chapter~\ref{cha:ipopt} for details.

\subsection{Passing options to local search based heuristics and OA generators}
Several parts of the algorithms in \BONMIN are based on solving a simplified version of the problem with another instance of \BONMIN:
Outer Approximation Decomposition (called in {\tt B-Hyb} at the root node)
and Feasibility Pump for MINLP (called in {\tt B-Hyb} or {\tt B-BB} at the root node), RINS, RENS, Local Branching.

In all these cases, one can pass options to the sub-algorithm used through the option file.
The basic principle is that the ``\texttt{bonmin.}'' prefix  is replaced with a prefix that identifies the sub-algorithm used:
\begin{itemize}
\vspace{-2ex}
\setlength{\parskip}{.2ex}
\setlength{\itemsep}{0pt}
\setlength{\partopsep}{0pt}
\item to pass options to Outer Approximation Decomposition: {\tt oa\_decomposition.},
\item to pass options to Feasibility Pump for MINLP: {\tt pump\_for\_minlp.},
\item to pass options to RINS: {\tt rins.},
\item to pass options to RENS: {\tt rens.},
\item to pass options to Local Branching: {\tt local\_branch}.
\end{itemize}

\vspace{-2ex}
For example, to run a maximum of 60 seconds of feasibility pump (FP) for MINLP until 6 solutions are found at the beginning of the hybrid algorithm, one sets the following options:
\begin{verbatim}
bonmin.algorithm              B-Hyb
bonmin.pump_for_minlp         yes   # tells to run FP for MINLP
pump_for_minlp.time_limit     60    # set a time limit for the pump
pump_for_minlp.solution_limit 6     # set a solution limit
\end{verbatim}
Note that the actual solution and time limit will be the minimum of the global limits set for \BONMIN.

A slightly more complicated set of options may be used when using RINS.
Say for example that one wants to run RINS inside \texttt{B-BB}.
Each time RINS is called one wants to solve the small-size MINLP generated using B-QG (one may run any algorithm available in \BONMIN for solving an MINLP) and wants to stop as soon as \texttt{B-QG} found one solution.
To achieve this, one sets the following options
\begin{verbatim}
bonmin.algorithm      B-BB
bonmin.heuristic_rins yes
rins.algorithm        B-QG
rins.solution_limit   1
\end{verbatim}
This example shows that it is possible to set any option used in the sub-algorithm to be different than the one used for the main algorithm.

In the context of outer-approximation (OA) and feasibility pump for MINLP, a standard MILP solver is used.
Several options are available for configuring this MILP solver.
\BONMIN allows a choice of different MILP solvers through the option
\texttt{bonmin.milp\_sol\-ver}. Values for this option are: {\tt Cbc\_D} which uses \CBC with its
default settings, {\tt Cbc\_Par} which uses a version of \CBC that can be parameterized by the user, and \texttt{Cplex} which uses \CPLEX with its default settings.
The options that can be set in {\tt Cbc\_Par} are the number of strong-branching candidates,
the number of branches before pseudo costs are to be trusted, and the frequency of the various cut generators, c.f.\ Section~\ref{sub:bonminalloptions} for details.
To use the \texttt{Cplex} option, a valid \CPLEX licence (standalone or \GAMS/\CPLEX) is required.

\subsection{Getting good solutions to nonconvex problems}
To solve a problem with nonconvex constraints, one should only use the branch-and-bound algorithm {\tt B-BB}.

A few options have been designed in \BONMIN specifically to treat
problems that do not have a convex continuous relaxation.
In such problems, the solutions obtained from \IPOPT are
not necessarily globally optimal, but are only locally optimal.
Also the outer-approximation constraints are not necessarily valid inequalities for the problem.
No specific heuristic method for treating nonconvex problems is implemented
yet within the OA framework.
But for the pure branch-and-bound {\tt B-BB}, a few options have been implemented while having
in mind that lower bounds provided by \IPOPT should not be trusted and with the goal of
trying to get good solutions. Such options are at a very experimental stage.

First, in the context of nonconvex problems, \IPOPT may find different local optima when started
from different starting points. The two options {\tt num\_re\-solve\_at\_root} and {\tt num\_resolve\_at\_node}
allow for solving the root node or each node of the tree, respectively, with a user-specified
number of different randomly-chosen starting points, saving the best solution found.
Note that the function to generate a random starting point is very na\"{\i}ve:
it chooses a random point (uniformly) between the bounds provided for the variable.
In particular if there are some functions that can not be evaluated at some points of the domain, it may pick such points,
and so it is not robust in that respect.

Secondly, since the solution given by \IPOPT does not truly give a lower bound, the fathoming rule can be changed to continue branching even if the solution value to the current node is worse than the best-known solution.
This is achieved by setting {\tt allowable\_gap}
and {\tt allowable\_fraction\_gap} and {\tt cutoff\_decr} to negative values.

\subsection{\IPOPT options changed by \BONMIN}

\IPOPT has a very large number of options, see Section \ref{sub:ipoptoptions} to get a complete description.
To use \IPOPT more efficiently in the context of MINLP,
\BONMIN changes some \IPOPT options from their default values, which may help to improve \IPOPT's warm-starting capabilities and its ability to prove quickly that a subproblem is infeasible.
These are settings that \IPOPT does not use for ordinary NLP problems.
Note that options set by the user in an option file will override these settings.
\begin{itemize}
\vspace{-2ex}
\setlength{\partopsep}{0pt}
\setlength{\itemsep}{0pt}
\setlength{\parskip}{.5ex}
\item {\tt mu\_strategy} and {\tt mu\_oracle} are set, respectively, to
{\tt adaptive} and {\tt probing} by default. These are strategies in \IPOPT
for updating the barrier parameter. They were found to be more efficient in the context of MINLP.

\item {\tt gamma\_phi} and {\tt gamma\_theta} are set to $10^{-8}$ and $10^{-4}$ respectively.
This has the effect of reducing the size of the filter in the line search performed by \IPOPT.

\item {\tt required\_infeasibility\_reduction} is set to $0.1$.
This increases the required infeasibility reduction when \IPOPT enters the
restoration phase and should thus help to detect infeasible problems faster.

\item {\tt expect\_infeasible\_problem} is set to {\tt yes}, which enables some heuristics
to detect infeasible problems faster.

\item {\tt warm\_start\_init\_point} is set to {\tt yes} when a full primal/dual starting
point is available (generally for all the optimizations after the continuous relaxation has been solved).

\item {\tt print\_level} is set to $0$ by default to turn off \IPOPT output (except for the root node, which print level is controlled by the \BONMIN option \texttt{nlp\_log\_at\_root}).

\item \texttt{bound\_relax\_factor} is set to $10^{-10}$. All of the bounds
of the problem are relaxed by this factor. This may cause some trouble
when constraint functions can only be evaluated within their bounds.
In such cases, this option should be set to $0$.
\end{itemize}

\section{Detailed Options Description}
\label{sub:bonminalloptions}

The following tables give the list of options together with their types, default values, and availability in each of the main algorithms.
% The column labeled `type' indicates the type of the parameter.
%  (`F' stands for float, `I' for integer, and `S' for string).
% The column labeled `default' indicates the global default value.
% Then for each of the algorithms \texttt{B-BB}, \texttt{B-OA}, \texttt{B-QG}, \texttt{B-Hyb}, \texttt{B-Ecp}, and \texttt{B-iFP} `$+$' indicates that the option is available for that particular algorithm, while `$-$' indicates that it is not.
The column labeled `\texttt{Cbc\_Par}' indicates the options that can be used to parametrize the MLIP subsolver in the context of OA and FP.

% \newpage
\topcaption{\label{tab:bonminoptions} 
List of options and compatibility with the different algorithms.
}
\tablehead{\hline 
Option & type &  default & {\tt B-BB} & {\tt B-OA} & {\tt B-QG} & {\tt B-Hyb} & {\tt B-Ecp} & {\tt B-iFP} & {\tt Cbc\_Par} \\
\hline}
\tabletail{\hline \multicolumn{10}{|r|}{continued on next page}\\\hline}
{\small
\begin{xtabular}{@{}|l|@{\;}r@{\;}|@{\;}r@{\;}|@{\;}r@{\;}|@{\;}r@{\;}|@{\;}r@{\;}|@{\;}r@{\;}|@{\;}r@{\;}|@{\;}r@{\;}|@{\;}r@{\;}|@{}}
\multicolumn{1}{|c}{} & \multicolumn{9}{l|}{Algorithm choice}\\
\hline
algorithm & string & B-BB & $\checkmark$& $\checkmark$& $\checkmark$& $\checkmark$& $\checkmark$& $\checkmark$& $\checkmark$\\
\hline
\multicolumn{1}{|c}{} & \multicolumn{9}{l|}{Branch-and-bound options}\\
\hline
allowable\_fraction\_gap & $\mathbb{Q}$ & \GAMS \texttt{optcr} & $\checkmark$& $\checkmark$& $\checkmark$& $\checkmark$& $\checkmark$& $\checkmark$& $\checkmark$\\
allowable\_gap & $\mathbb{Q}$ & \GAMS \texttt{optca} & $\checkmark$& $\checkmark$& $\checkmark$& $\checkmark$& $\checkmark$& $\checkmark$& $\checkmark$\\
cutoff & $\mathbb{Q}$ & \GAMS \texttt{cutoff} & $\checkmark$& $\checkmark$& $\checkmark$& $\checkmark$& $\checkmark$& $\checkmark$& $\checkmark$\\
cutoff\_decr & $\mathbb{Q}$ & $10^{- 5}$ & $\checkmark$& $\checkmark$& $\checkmark$& $\checkmark$& $\checkmark$& $\checkmark$& $\checkmark$\\
enable\_dynamic\_nlp & string & no & $\checkmark$& --& --& --& --& --& --\\
integer\_tolerance & $\mathbb{Q}$ & $10^{- 6}$ & $\checkmark$& $\checkmark$& $\checkmark$& $\checkmark$& $\checkmark$& $\checkmark$& $\checkmark$\\
iteration\_limit & $\mathbb{Z}$ & \GAMS \texttt{iterlim} & $\checkmark$& $\checkmark$& $\checkmark$& $\checkmark$& $\checkmark$& $\checkmark$& $\checkmark$\\
nlp\_failure\_behavior & string & stop & $\checkmark$& --& --& --& --& --& --\\
node\_comparison & string & best-bound & $\checkmark$& $\checkmark$& $\checkmark$& $\checkmark$& $\checkmark$& $\checkmark$& --\\
node\_limit & $\mathbb{Z}$ & \GAMS \texttt{nodlim} & $\checkmark$& $\checkmark$& $\checkmark$& $\checkmark$& $\checkmark$& $\checkmark$& $\checkmark$\\
num\_cut\_passes & $\mathbb{Z}$ & $1$ & --& --& $\checkmark$& $\checkmark$& $\checkmark$& --& --\\
num\_cut\_passes\_at\_root & $\mathbb{Z}$ & $20$ & --& --& $\checkmark$& $\checkmark$& $\checkmark$& --& --\\
number\_before\_trust & $\mathbb{Z}$ & $8$ & $\checkmark$& $\checkmark$& $\checkmark$& $\checkmark$& $\checkmark$& $\checkmark$& $\checkmark$\\
number\_strong\_branch & $\mathbb{Z}$ & $20$ & $\checkmark$& $\checkmark$& $\checkmark$& $\checkmark$& $\checkmark$& $\checkmark$& $\checkmark$\\
random\_generator\_seed & $\mathbb{Z}$ & $0$ & $\checkmark$& $\checkmark$& $\checkmark$& $\checkmark$& $\checkmark$& $\checkmark$& $\checkmark$\\
read\_solution\_file & string & no & $\checkmark$& $\checkmark$& $\checkmark$& $\checkmark$& $\checkmark$& $\checkmark$& $\checkmark$\\
solution\_limit & $\mathbb{Z}$ & $\infty$ & $\checkmark$& $\checkmark$& $\checkmark$& $\checkmark$& $\checkmark$& $\checkmark$& $\checkmark$\\
time\_limit & $\mathbb{Q}$ & \GAMS \texttt{reslim} & $\checkmark$& $\checkmark$& $\checkmark$& $\checkmark$& $\checkmark$& $\checkmark$& $\checkmark$\\
tree\_search\_strategy & string & probed-dive & $\checkmark$& $\checkmark$& $\checkmark$& $\checkmark$& $\checkmark$& $\checkmark$& --\\
variable\_selection & string & strong-branching & $\checkmark$& --& $\checkmark$& $\checkmark$& $\checkmark$& --& --\\
\hline
\multicolumn{1}{|c}{} & \multicolumn{9}{l|}{ECP cuts generation}\\
\hline
ecp\_abs\_tol & $\mathbb{Q}$ & $10^{- 6}$ & --& --& $\checkmark$& $\checkmark$& --& --& --\\
ecp\_max\_rounds & $\mathbb{Z}$ & $5$ & --& --& $\checkmark$& $\checkmark$& --& --& --\\
ecp\_probability\_factor & $\mathbb{Q}$ & $10$ & --& --& $\checkmark$& $\checkmark$& --& --& --\\
ecp\_rel\_tol & $\mathbb{Q}$ & $0$ & --& --& $\checkmark$& $\checkmark$& --& --& --\\
filmint\_ecp\_cuts & $\mathbb{Z}$ & $0$ & --& --& $\checkmark$& $\checkmark$& --& --& --\\
\hline
\multicolumn{1}{|c}{} & \multicolumn{9}{l|}{Feasibility checker using OA cuts}\\
\hline
feas\_check\_cut\_types & string & outer-approx & --& --& $\checkmark$& $\checkmark$& $\checkmark$& --& --\\
feas\_check\_discard\_policy & string & detect-cycles & --& --& $\checkmark$& $\checkmark$& $\checkmark$& --& --\\
generate\_benders\_after\_so\_many\_oa & $\mathbb{Z}$ & $5000$ & --& --& $\checkmark$& $\checkmark$& $\checkmark$& --& --\\
\hline
\multicolumn{1}{|c}{} & \multicolumn{9}{l|}{MILP Solver}\\
\hline
cpx\_parallel\_strategy & $\mathbb{Z}$ & $0$ & --& --& --& --& --& --& $\checkmark$\\
milp\_solver & string & Cbc\_D & --& --& --& --& --& --& $\checkmark$\\
milp\_strategy & string & solve\_to\_optimality & --& --& --& --& --& --& $\checkmark$\\
number\_cpx\_threads & $\mathbb{Z}$ & $0$ & --& --& --& --& --& --& $\checkmark$\\
\hline
\multicolumn{1}{|c}{} & \multicolumn{9}{l|}{MILP cutting planes in hybrid algorithm}\\
\hline
2mir\_cuts & $\mathbb{Z}$ & $0$ & --& $\checkmark$& $\checkmark$& $\checkmark$& $\checkmark$& $\checkmark$& $\checkmark$\\
Gomory\_cuts & $\mathbb{Z}$ & $-5$ & --& $\checkmark$& $\checkmark$& $\checkmark$& $\checkmark$& $\checkmark$& $\checkmark$\\
clique\_cuts & $\mathbb{Z}$ & $-5$ & --& $\checkmark$& $\checkmark$& $\checkmark$& $\checkmark$& $\checkmark$& $\checkmark$\\
cover\_cuts & $\mathbb{Z}$ & $0$ & --& $\checkmark$& $\checkmark$& $\checkmark$& $\checkmark$& $\checkmark$& $\checkmark$\\
flow\_cover\_cuts & $\mathbb{Z}$ & $-5$ & --& $\checkmark$& $\checkmark$& $\checkmark$& $\checkmark$& $\checkmark$& $\checkmark$\\
lift\_and\_project\_cuts & $\mathbb{Z}$ & $0$ & --& $\checkmark$& $\checkmark$& $\checkmark$& $\checkmark$& $\checkmark$& $\checkmark$\\
mir\_cuts & $\mathbb{Z}$ & $-5$ & --& $\checkmark$& $\checkmark$& $\checkmark$& $\checkmark$& $\checkmark$& $\checkmark$\\
reduce\_and\_split\_cuts & $\mathbb{Z}$ & $0$ & --& $\checkmark$& $\checkmark$& $\checkmark$& $\checkmark$& $\checkmark$& $\checkmark$\\
\hline
\multicolumn{1}{|c}{} & \multicolumn{9}{l|}{NLP interface}\\
\hline
solvefinal & string & yes & $\checkmark$& $\checkmark$& $\checkmark$& $\checkmark$& $\checkmark$& $\checkmark$& $\checkmark$\\
warm\_start & string & none & $\checkmark$& --& --& --& --& --& --\\
\hline
\multicolumn{1}{|c}{} & \multicolumn{9}{l|}{NLP solution robustness}\\
\hline
max\_consecutive\_failures & $\mathbb{Z}$ & $10$ & $\checkmark$& --& --& --& --& --& --\\
max\_random\_point\_radius & $\mathbb{Q}$ & $100000$ & $\checkmark$& --& --& --& --& --& --\\
num\_iterations\_suspect & $\mathbb{Z}$ & $-1$ & $\checkmark$& $\checkmark$& $\checkmark$& $\checkmark$& $\checkmark$& $\checkmark$& $\checkmark$\\
num\_retry\_unsolved\_random\_point & $\mathbb{Z}$ & $0$ & $\checkmark$& $\checkmark$& $\checkmark$& $\checkmark$& $\checkmark$& $\checkmark$& $\checkmark$\\
random\_point\_perturbation\_interval & $\mathbb{Q}$ & $1$ & $\checkmark$& --& --& --& --& --& --\\
random\_point\_type & string & Jon & $\checkmark$& --& --& --& --& --& --\\
resolve\_on\_small\_infeasibility & $\mathbb{Q}$ & $0$ & $\checkmark$& --& --& --& --& --& --\\
\hline
\multicolumn{1}{|c}{} & \multicolumn{9}{l|}{NLP solves in hybrid algorithm (B-Hyb)}\\
\hline
nlp\_solve\_frequency & $\mathbb{Z}$ & $10$ & --& --& --& $\checkmark$& --& --& --\\
nlp\_solve\_max\_depth & $\mathbb{Z}$ & $10$ & --& --& --& $\checkmark$& --& --& --\\
nlp\_solves\_per\_depth & $\mathbb{Q}$ & $10^{ 100}$ & --& --& --& $\checkmark$& --& --& --\\
\hline
\multicolumn{1}{|c}{} & \multicolumn{9}{l|}{Nonconvex problems}\\
\hline
coeff\_var\_threshold & $\mathbb{Q}$ & $0.1$ & $\checkmark$& --& --& --& --& --& --\\
dynamic\_def\_cutoff\_decr & string & no & $\checkmark$& --& --& --& --& --& --\\
first\_perc\_for\_cutoff\_decr & $\mathbb{Q}$ & $-0.02$ & $\checkmark$& --& --& --& --& --& --\\
max\_consecutive\_infeasible & $\mathbb{Z}$ & $0$ & $\checkmark$& --& --& --& --& --& --\\
num\_resolve\_at\_infeasibles & $\mathbb{Z}$ & $0$ & $\checkmark$& --& --& --& --& --& --\\
num\_resolve\_at\_node & $\mathbb{Z}$ & $0$ & $\checkmark$& --& --& --& --& --& --\\
num\_resolve\_at\_root & $\mathbb{Z}$ & $0$ & $\checkmark$& --& --& --& --& --& --\\
second\_perc\_for\_cutoff\_decr & $\mathbb{Q}$ & $-0.05$ & $\checkmark$& --& --& --& --& --& --\\
\hline
\multicolumn{1}{|c}{} & \multicolumn{9}{l|}{Outer Approximation Decomposition (B-OA)}\\
\hline
oa\_decomposition & string & no & --& --& $\checkmark$& $\checkmark$& $\checkmark$& --& --\\
\hline
\multicolumn{1}{|c}{} & \multicolumn{9}{l|}{Outer Approximation cuts generation}\\
\hline
add\_only\_violated\_oa & string & no & --& $\checkmark$& $\checkmark$& $\checkmark$& $\checkmark$& $\checkmark$& $\checkmark$\\
oa\_cuts\_scope & string & global & --& $\checkmark$& $\checkmark$& $\checkmark$& $\checkmark$& $\checkmark$& $\checkmark$\\
oa\_rhs\_relax & $\mathbb{Q}$ & $10^{- 8}$ & --& $\checkmark$& $\checkmark$& $\checkmark$& $\checkmark$& $\checkmark$& $\checkmark$\\
tiny\_element & $\mathbb{Q}$ & $10^{- 8}$ & --& $\checkmark$& $\checkmark$& $\checkmark$& $\checkmark$& $\checkmark$& $\checkmark$\\
very\_tiny\_element & $\mathbb{Q}$ & $10^{-17}$ & --& $\checkmark$& $\checkmark$& $\checkmark$& $\checkmark$& $\checkmark$& $\checkmark$\\
\hline
\multicolumn{1}{|c}{} & \multicolumn{9}{l|}{Output and Loglevel}\\
\hline
bb\_log\_interval & $\mathbb{Z}$ & $100$ & $\checkmark$& $\checkmark$& $\checkmark$& $\checkmark$& $\checkmark$& $\checkmark$& $\checkmark$\\
bb\_log\_level & $\mathbb{Z}$ & $1$ & $\checkmark$& $\checkmark$& $\checkmark$& $\checkmark$& $\checkmark$& $\checkmark$& $\checkmark$\\
fp\_log\_frequency & $\mathbb{Q}$ & $100$ & --& --& $\checkmark$& $\checkmark$& --& --& --\\
fp\_log\_level & $\mathbb{Z}$ & $1$ & --& --& $\checkmark$& $\checkmark$& --& --& --\\
lp\_log\_level & $\mathbb{Z}$ & $0$ & --& $\checkmark$& $\checkmark$& $\checkmark$& $\checkmark$& $\checkmark$& $\checkmark$\\
milp\_log\_level & $\mathbb{Z}$ & $0$ & --& --& --& --& --& --& $\checkmark$\\
nlp\_log\_at\_root & $\mathbb{Z}$ & 5 & $\checkmark$& $\checkmark$& $\checkmark$& $\checkmark$& $\checkmark$& $\checkmark$& --\\
nlp\_log\_level & $\mathbb{Z}$ & $1$ & $\checkmark$& $\checkmark$& $\checkmark$& $\checkmark$& $\checkmark$& $\checkmark$& $\checkmark$\\
oa\_cuts\_log\_level & $\mathbb{Z}$ & $0$ & --& $\checkmark$& $\checkmark$& $\checkmark$& $\checkmark$& $\checkmark$& $\checkmark$\\
oa\_log\_frequency & $\mathbb{Q}$ & $100$ & $\checkmark$& --& --& $\checkmark$& $\checkmark$& --& --\\
oa\_log\_level & $\mathbb{Z}$ & $1$ & $\checkmark$& --& --& $\checkmark$& $\checkmark$& --& --\\
print\_funceval\_statistics & string & no & $\checkmark$& $\checkmark$& $\checkmark$& $\checkmark$& $\checkmark$& $\checkmark$& $\checkmark$\\
solvetrace & string &  & $\checkmark$& $\checkmark$& $\checkmark$& $\checkmark$& $\checkmark$& $\checkmark$& $\checkmark$\\
solvetracenodefreq & $\mathbb{Z}$ & $100$ & $\checkmark$& $\checkmark$& $\checkmark$& $\checkmark$& $\checkmark$& $\checkmark$& $\checkmark$\\
solvetracetimefreq & $\mathbb{Q}$ & $5$ & $\checkmark$& $\checkmark$& $\checkmark$& $\checkmark$& $\checkmark$& $\checkmark$& $\checkmark$\\
\hline
\multicolumn{1}{|c}{} & \multicolumn{9}{l|}{Primal Heuristics}\\
\hline
feasibility\_pump\_objective\_norm & $\mathbb{Z}$ & $1$ & $\checkmark$& $\checkmark$& $\checkmark$& $\checkmark$& $\checkmark$& $\checkmark$& --\\
fp\_pass\_infeasible & string & no & $\checkmark$& $\checkmark$& $\checkmark$& $\checkmark$& $\checkmark$& $\checkmark$& $\checkmark$\\
heuristic\_RINS & string & no & $\checkmark$& $\checkmark$& $\checkmark$& $\checkmark$& $\checkmark$& $\checkmark$& --\\
heuristic\_dive\_MIP\_fractional & string & no & $\checkmark$& $\checkmark$& $\checkmark$& $\checkmark$& $\checkmark$& $\checkmark$& --\\
heuristic\_dive\_MIP\_vectorLength & string & no & $\checkmark$& $\checkmark$& $\checkmark$& $\checkmark$& $\checkmark$& $\checkmark$& --\\
heuristic\_dive\_fractional & string & no & $\checkmark$& $\checkmark$& $\checkmark$& $\checkmark$& $\checkmark$& $\checkmark$& --\\
heuristic\_dive\_vectorLength & string & no & $\checkmark$& $\checkmark$& $\checkmark$& $\checkmark$& $\checkmark$& $\checkmark$& --\\
heuristic\_feasibility\_pump & string & no & $\checkmark$& $\checkmark$& $\checkmark$& $\checkmark$& $\checkmark$& $\checkmark$& --\\
pump\_for\_minlp & string & no & $\checkmark$& $\checkmark$& $\checkmark$& $\checkmark$& $\checkmark$& $\checkmark$& --\\
\hline
\multicolumn{1}{|c}{} & \multicolumn{9}{l|}{Strong branching setup}\\
\hline
candidate\_sort\_criterion & string & best-ps-cost & $\checkmark$& $\checkmark$& $\checkmark$& $\checkmark$& $\checkmark$& $\checkmark$& --\\
maxmin\_crit\_have\_sol & $\mathbb{Q}$ & $0.1$ & $\checkmark$& $\checkmark$& $\checkmark$& $\checkmark$& $\checkmark$& $\checkmark$& --\\
maxmin\_crit\_no\_sol & $\mathbb{Q}$ & $0.7$ & $\checkmark$& $\checkmark$& $\checkmark$& $\checkmark$& $\checkmark$& $\checkmark$& --\\
min\_number\_strong\_branch & $\mathbb{Z}$ & $0$ & $\checkmark$& $\checkmark$& $\checkmark$& $\checkmark$& $\checkmark$& $\checkmark$& --\\
number\_before\_trust\_list & $\mathbb{Z}$ & $0$ & $\checkmark$& $\checkmark$& $\checkmark$& $\checkmark$& $\checkmark$& $\checkmark$& --\\
number\_look\_ahead & $\mathbb{Z}$ & $0$ & $\checkmark$& $\checkmark$& $\checkmark$& $\checkmark$& $\checkmark$& --& --\\
number\_strong\_branch\_root & $\mathbb{Z}$ & $\infty$ & $\checkmark$& $\checkmark$& $\checkmark$& $\checkmark$& $\checkmark$& $\checkmark$& --\\
setup\_pseudo\_frac & $\mathbb{Q}$ & $0.5$ & $\checkmark$& $\checkmark$& $\checkmark$& $\checkmark$& $\checkmark$& $\checkmark$& --\\
trust\_strong\_branching\_for\_pseudo\_cost & string & yes & $\checkmark$& $\checkmark$& $\checkmark$& $\checkmark$& $\checkmark$& $\checkmark$& --\\
\hline
\end{xtabular}
}


In the following we give a detailed list of \BONMIN options.
The value on the right denotes the default value.
\printoptioncategory{Algorithm choice}
\printoption{algorithm}%
{\ttfamily B-BB, B-OA, B-QG, B-Hyb, B-Ecp, B-iFP}%
{B-BB}%
{Choice of the algorithm.\\
This will preset some of the options of bonmin depending on the algorithm choice.}%
{\begin{list}{}{
\setlength{\parsep}{0em}
\setlength{\leftmargin}{5ex}
\setlength{\labelwidth}{2ex}
\setlength{\itemindent}{0ex}
\setlength{\topsep}{0pt}}
\item[\texttt{B-BB}] simple branch-and-bound algorithm,
\item[\texttt{B-OA}] OA Decomposition algorithm,
\item[\texttt{B-QG}] Quesada and Grossmann branch-and-cut algorithm,
\item[\texttt{B-Hyb}] hybrid outer approximation based branch-and-cut,
\item[\texttt{B-Ecp}] ecp cuts based branch-and-cut a la FilMINT.
\item[\texttt{B-iFP}] Iterated Feasibility Pump for MINLP.
\end{list}
}

\printoptioncategory{Branch-and-bound options}
\printoption{allowable\_fraction\_gap}%
{$\textrm{real}$}%
{$0.1$}%
{Specify the value of relative gap under which the algorithm stops.\\
Stop the tree search when the gap between the objective value of the best known solution and the best bound on the objective of any solution is less than this fraction of the absolute value of the best known solution value.}%
{}

\printoption{allowable\_gap}%
{$\textrm{real}$}%
{$0$}%
{Specify the value of absolute gap under which the algorithm stops.\\
Stop the tree search when the gap between the objective value of the best known solution and the best bound on the objective of any solution is less than this.}%
{}

\printoption{cutoff}%
{$-10^{ 100}\leq\textrm{real}\leq10^{ 100}$}%
{$10^{ 100}$}%
{Specify cutoff value.\\
cutoff should be the value of a feasible solution known by the user (if any). The algorithm will only look for solutions better than cutoff.}%
{}

\printoption{cutoff\_decr}%
{$-10^{ 10}\leq\textrm{real}\leq10^{ 10}$}%
{$10^{- 5}$}%
{Specify cutoff decrement.\\
Specify the amount by which cutoff is decremented below a new best upper-bound (usually a small positive value but in non-convex problems it may be a negative value).}%
{}

\printoption{enable\_dynamic\_nlp}%
{\ttfamily no, yes}%
{no}%
{Enable dynamic linear and quadratic rows addition in nlp}%
{}

\printoption{integer\_tolerance}%
{$0<\textrm{real}$}%
{$10^{- 6}$}%
{Set integer tolerance.\\
Any number within that value of an integer is considered integer.}%
{}

\printoption{iteration\_limit}%
{$0\leq\textrm{integer}$}%
{$\infty$}%
{Set the cumulated maximum number of iteration in the algorithm used to process nodes continuous relaxations in the branch-and-bound.\\
value 0 deactivates option.}%
{}

\printoption{nlp\_failure\_behavior}%
{\ttfamily stop, fathom}%
{stop}%
{Set the behavior when an NLP or a series of NLP are unsolved by Ipopt (we call unsolved an NLP for which Ipopt is not able to guarantee optimality within the specified tolerances).\\
If set to "fathom", the algorithm will fathom the node when Ipopt fails to find a solution to the nlp at that node whithin the specified tolerances. The algorithm then becomes a heuristic, and the user will be warned that the solution might not be optimal.}%
{\begin{list}{}{
\setlength{\parsep}{0em}
\setlength{\leftmargin}{5ex}
\setlength{\labelwidth}{2ex}
\setlength{\itemindent}{0ex}
\setlength{\topsep}{0pt}}
\item[\texttt{stop}] Stop when failure happens.
\item[\texttt{fathom}] Continue when failure happens.
\end{list}
}

\printoption{node\_comparison}%
{\ttfamily best-bound, depth-first, breadth-first, dynamic, best-guess}%
{best-bound}%
{Choose the node selection strategy.\\
Choose the strategy for selecting the next node to be processed.}%
{\begin{list}{}{
\setlength{\parsep}{0em}
\setlength{\leftmargin}{5ex}
\setlength{\labelwidth}{2ex}
\setlength{\itemindent}{0ex}
\setlength{\topsep}{0pt}}
\item[\texttt{best-bound}] choose node with the smallest bound,
\item[\texttt{depth-first}] Perform depth first search,
\item[\texttt{breadth-first}] Perform breadth first search,
\item[\texttt{dynamic}] Cbc dynamic strategy (starts with a depth first search and turn to best bound after 3 integer feasible solutions have been found).
\item[\texttt{best-guess}] choose node with smallest guessed integer solution
\end{list}
}

\printoption{node\_limit}%
{$0\leq\textrm{integer}$}%
{$\infty$}%
{Set the maximum number of nodes explored in the branch-and-bound search.}%
{}

\printoption{num\_cut\_passes}%
{$0\leq\textrm{integer}$}%
{$1$}%
{Set the maximum number of cut passes at regular nodes of the branch-and-cut.}%
{}

\printoption{num\_cut\_passes\_at\_root}%
{$0\leq\textrm{integer}$}%
{$20$}%
{Set the maximum number of cut passes at regular nodes of the branch-and-cut.}%
{}

\printoption{number\_before\_trust}%
{$0\leq\textrm{integer}$}%
{$8$}%
{Set the number of branches on a variable before its pseudo costs are to be believed in dynamic strong branching.\\
A value of 0 disables pseudo costs.}%
{}

\printoption{number\_strong\_branch}%
{$0\leq\textrm{integer}$}%
{$20$}%
{Choose the maximum number of variables considered for strong branching.\\
Set the number of variables on which to do strong branching.}%
{}

\printoption{solution\_limit}%
{$0\leq\textrm{integer}$}%
{$\infty$}%
{Abort after that much integer feasible solution have been found by algorithm\\
value 0 deactivates option}%
{}

\printoption{time\_limit}%
{$0\leq\textrm{real}$}%
{$1000$}%
{Set the global maximum computation time (in secs) for the algorithm.}%
{}

\printoption{tree\_search\_strategy}%
{\ttfamily top-node, dive, probed-dive, dfs-dive, dfs-dive-dynamic}%
{probed-dive}%
{Pick a strategy for traversing the tree\\
All strategies can be used in conjunction with any of the node comparison functions. Options which affect dfs-dive are max-backtracks-in-dive and max-dive-depth. The dfs-dive won't work in a non-convex problem where objective does not decrease down branches.}%
{\begin{list}{}{
\setlength{\parsep}{0em}
\setlength{\leftmargin}{5ex}
\setlength{\labelwidth}{2ex}
\setlength{\itemindent}{0ex}
\setlength{\topsep}{0pt}}
\item[\texttt{top-node}]  Always pick the top node as sorted by the node comparison function
\item[\texttt{dive}] Dive in the tree if possible, otherwise pick top node as sorted by the tree comparison function.
\item[\texttt{probed-dive}] Dive in the tree exploring two childs before continuing the dive at each level.
\item[\texttt{dfs-dive}] Dive in the tree if possible doing a depth first search. Backtrack on leaves or when a prescribed depth is attained or when estimate of best possible integer feasible solution in subtree is worst than cutoff. Once a prescribed limit of backtracks is attained pick top node as sorted by the tree comparison function
\item[\texttt{dfs-dive-dynamic}] Same as dfs-dive but once enough solution are found switch to best-bound and if too many nodes switch to depth-first.
\end{list}
}

\printoption{variable\_selection}%
{\ttfamily most-fractional, strong-branching, reliability-branching, curvature-estimator, qp-strong-branching, lp-strong-branching, nlp-strong-branching, osi-simple, osi-strong, random}%
{strong-branching}%
{Chooses variable selection strategy}%
{\begin{list}{}{
\setlength{\parsep}{0em}
\setlength{\leftmargin}{5ex}
\setlength{\labelwidth}{2ex}
\setlength{\itemindent}{0ex}
\setlength{\topsep}{0pt}}
\item[\texttt{most-fractional}] Choose most fractional variable
\item[\texttt{strong-branching}] Perform strong branching
\item[\texttt{reliability-branching}] Use reliability branching
\item[\texttt{curvature-estimator}] Use curvature estimation to select branching variable
\item[\texttt{qp-strong-branching}] Perform strong branching with QP approximation
\item[\texttt{lp-strong-branching}] Perform strong branching with LP approximation
\item[\texttt{nlp-strong-branching}] Perform strong branching with NLP approximation
\item[\texttt{osi-simple}] Osi method to do simple branching
\item[\texttt{osi-strong}] Osi method to do strong branching
\item[\texttt{random}] Method to choose branching variable randomly
\end{list}
}

\printoptioncategory{ECP cuts generation}
\printoption{ecp\_abs\_tol}%
{$0\leq\textrm{real}$}%
{$10^{- 6}$}%
{Set the absolute termination tolerance for ECP rounds.}%
{}

\printoption{ecp\_max\_rounds}%
{$0\leq\textrm{integer}$}%
{$5$}%
{Set the maximal number of rounds of ECP cuts.}%
{}

\printoption{ecp\_probability\_factor}%
{$\textrm{real}$}%
{$10$}%
{Factor appearing in formula for skipping ECP cuts.\\
Choosing -1 disables the skipping.}%
{}

\printoption{ecp\_rel\_tol}%
{$0\leq\textrm{real}$}%
{$0$}%
{Set the relative termination tolerance for ECP rounds.}%
{}

\printoption{filmint\_ecp\_cuts}%
{$0\leq\textrm{integer}$}%
{$0$}%
{Specify the frequency (in terms of nodes) at which some a la filmint ecp cuts are generated.\\
A frequency of 0 amounts to to never solve the NLP relaxation.}%
{}

\printoptioncategory{Feasibility checker using OA cuts}
\printoption{feas\_check\_cut\_types}%
{\ttfamily outer-approx, Benders}%
{outer-approx}%
{Choose the type of cuts generated when an integer feasible solution is found\\
If it seems too much memory is used should try Benders to use less}%
{\begin{list}{}{
\setlength{\parsep}{0em}
\setlength{\leftmargin}{5ex}
\setlength{\labelwidth}{2ex}
\setlength{\itemindent}{0ex}
\setlength{\topsep}{0pt}}
\item[\texttt{outer-approx}] Generate a set of Outer Approximations cuts.
\item[\texttt{Benders}] Generate a single Benders cut.
\end{list}
}

\printoption{feas\_check\_discard\_policy}%
{\ttfamily detect-cycles, keep-all, treated-as-normal}%
{detect-cycles}%
{How cuts from feasibility checker are discarded\\
Normally to avoid cycle cuts from feasibility checker should not be discarded in the node where they are generated. However Cbc sometimes does it if no care is taken which can lead to an infinite loop in Bonmin (usualy on simple problems). To avoid this one can instruct Cbc to never discard a cut but if we do that for all cuts it can lead to memory problems. The default policy here is to detect cycles and only then impose to Cbc to keep the cut. The two other alternative are to instruct Cbc to keep all cuts or to just ignore the problem and hope for the best}%
{\begin{list}{}{
\setlength{\parsep}{0em}
\setlength{\leftmargin}{5ex}
\setlength{\labelwidth}{2ex}
\setlength{\itemindent}{0ex}
\setlength{\topsep}{0pt}}
\item[\texttt{detect-cycles}] Detect if a cycle occurs and only in this case force not to discard.
\item[\texttt{keep-all}] Force cuts from feasibility checker not to be discarded (memory hungry but sometimes better).
\item[\texttt{treated-as-normal}] Cuts from memory checker can be discarded as any other cuts (code may cycle then)
\end{list}
}

\printoption{generate\_benders\_after\_so\_many\_oa}%
{$0\leq\textrm{integer}$}%
{$5000$}%
{Specify that after so many oa cuts have been generated Benders cuts should be generated instead.\\
It seems that sometimes generating too many oa cuts slows down the optimization compared to Benders due to the size of the LP. With this option we specify that after so many OA cuts have been generated we should switch to Benders cuts.}%
{}

\printoptioncategory{MILP Solver}
\printoption{cpx\_parallel\_strategy}%
{$-1\leq\textrm{integer}\leq1$}%
{$0$}%
{Strategy of parallel search mode in CPLEX.\\
-1 = opportunistic, 0 = automatic, 1 = deterministic (refer to CPLEX documentation)}%
{}

\printoption{milp\_solver}%
{Cbc\_D, Cbc\_Par, Cplex}%
{Cbc\_D}%
{Choose the subsolver to solve MILP sub-problems in OA decompositions.\\
 To use Cplex, a valid license is required.}%
{\begin{list}{}{
\setlength{\parsep}{0em}
\setlength{\leftmargin}{5ex}
\setlength{\labelwidth}{2ex}
\setlength{\itemindent}{0ex}
\setlength{\topsep}{0pt}}
\item[\texttt{Cbc\_D}] Coin Branch and Cut with its default
\item[\texttt{Cbc\_Par}] Coin Branch and Cut with passed parameters
\item[\texttt{Cplex}] IBM CPLEX
\end{list}
}

\printoption{milp\_strategy}%
{\ttfamily find\_good\_sol, solve\_to\_optimality}%
{find\_good\_sol}%
{Choose a strategy for MILPs.}%
{\begin{list}{}{
\setlength{\parsep}{0em}
\setlength{\leftmargin}{5ex}
\setlength{\labelwidth}{2ex}
\setlength{\itemindent}{0ex}
\setlength{\topsep}{0pt}}
\item[\texttt{find\_good\_sol}] Stop sub milps when a solution improving the incumbent is found
\item[\texttt{solve\_to\_optimality}] Solve MILPs to optimality
\end{list}
}

\printoption{number\_cpx\_threads}%
{$0\leq\textrm{integer}$}%
{$0$}%
{Set number of threads to use with cplex.\\
(refer to CPLEX documentation)}%
{}

\printoptioncategory{MILP cutting planes in hybrid algorithm (B-Hyb)}
\printoption{2mir\_cuts}%
{$-100\leq\textrm{integer}$}%
{$0$}%
{Frequency (in terms of nodes) for generating 2-MIR cuts in branch-and-cut\\
If k $>$ 0, cuts are generated every k nodes, if -99 $<$ k $<$ 0 cuts are generated every -k nodes but Cbc may decide to stop generating cuts, if not enough are generated at the root node, if k=-99 generate cuts only at the root node, if k=0 or 100 do not generate cuts.}%
{}

\printoption{Gomory\_cuts}%
{$-100\leq\textrm{integer}$}%
{$-5$}%
{Frequency k (in terms of nodes) for generating Gomory cuts in branch-and-cut.\\
See option \texttt{2mir\_cuts} for the meaning of k.}%
{}

\printoption{clique\_cuts}%
{$-100\leq\textrm{integer}$}%
{$-5$}%
{Frequency (in terms of nodes) for generating clique cuts in branch-and-cut\\
See option \texttt{2mir\_cuts} for the meaning of k.}%
{}

\printoption{cover\_cuts}%
{$-100\leq\textrm{integer}$}%
{$0$}%
{Frequency (in terms of nodes) for generating cover cuts in branch-and-cut\\
See option \texttt{2mir\_cuts} for the meaning of k.}%
{}

\printoption{flow\_cover\_cuts}%
{$-100\leq\textrm{integer}$}%
{$-5$}%
{Frequency (in terms of nodes) for generating flow cover cuts in branch-and-cut\\
See option \texttt{2mir\_cuts} for the meaning of k.}%
{}

\printoption{lift\_and\_project\_cuts}%
{$-100\leq\textrm{integer}$}%
{$0$}%
{Frequency (in terms of nodes) for generating lift-and-project cuts in branch-and-cut\\
See option \texttt{2mir\_cuts} for the meaning of k.}%
{}

\printoption{mir\_cuts}%
{$-100\leq\textrm{integer}$}%
{$-5$}%
{Frequency (in terms of nodes) for generating MIR cuts in branch-and-cut\\
See option \texttt{2mir\_cuts} for the meaning of k.}%
{}

\printoption{reduce\_and\_split\_cuts}%
{$-100\leq\textrm{integer}$}%
{$0$}%
{Frequency (in terms of nodes) for generating reduce-and-split cuts in branch-and-cut\\
See option \texttt{2mir\_cuts} for the meaning of k.}%
{}

\printoptioncategory{MINLP Heuristics}
\printoption{feasibility\_pump\_objective\_norm}%
{$1\leq\textrm{integer}\leq2$}%
{$1$}%
{Norm of feasibility pump objective function}%
{}

\printoption{fp\_pass\_infeasible}%
{\ttfamily no, yes}%
{no}%
{Say whether feasibility pump should claim to converge or not}%
{\begin{list}{}{
\setlength{\parsep}{0em}
\setlength{\leftmargin}{5ex}
\setlength{\labelwidth}{2ex}
\setlength{\itemindent}{0ex}
\setlength{\topsep}{0pt}}
\item[\texttt{no}] When master MILP is infeasible just bail out (don't stop all algorithm). This is the option for using in B-Hyb.
\item[\texttt{yes}] Claim convergence, numerically dangerous.
\end{list}
}

\printoption{heuristic\_RINS}%
{\ttfamily no, yes}%
{no}%
{if yes runs the RINS heuristic}%
{
}

\printoption{heuristic\_dive\_MIP\_fractional}%
{\ttfamily no, yes}%
{no}%
{if yes runs the Dive MIP Fractional heuristic}%
{
}

\printoption{heuristic\_dive\_MIP\_vectorLength}%
{\ttfamily no, yes}%
{no}%
{if yes runs the Dive MIP VectorLength heuristic}%
{
}

\printoption{heuristic\_dive\_fractional}%
{\ttfamily no, yes}%
{no}%
{if yes runs the Dive Fractional heuristic}%
{
}

\printoption{heuristic\_dive\_vectorLength}%
{\ttfamily no, yes}%
{no}%
{if yes runs the Dive VectorLength heuristic}%
{
}

\printoption{heuristic\_feasibility\_pump}%
{\ttfamily no, yes}%
{no}%
{whether the heuristic feasibility pump should be used}%
{
}

\printoption{pump\_for\_minlp}%
{\ttfamily no, yes}%
{no}%
{if yes runs FP for MINLP}%
{
}

\printoptioncategory{NLP interface}
\printoption{warm\_start}%
{\ttfamily none, optimum, interior\_point}%
{none}%
{Select the warm start method\\
This will affect the function getWarmStart(), and as a consequence the warm starting in the various algorithms.}%
{\begin{list}{}{
\setlength{\parsep}{0em}
\setlength{\leftmargin}{5ex}
\setlength{\labelwidth}{2ex}
\setlength{\itemindent}{0ex}
\setlength{\topsep}{0pt}}
\item[\texttt{none}] No warm start
\item[\texttt{optimum}] Warm start with direct parent optimum
\item[\texttt{interior\_point}] Warm start with an interior point of direct parent
\end{list}
}

\printoptioncategory{NLP solution robustness}
\printoption{max\_consecutive\_failures}%
{$0\leq\textrm{integer}$}%
{$10$}%
{(temporarily removed) Number $n$ of consecutive unsolved problems before aborting a branch of the tree.\\
When $n > 0$, continue exploring a branch of the tree until $n$ consecutive problems in the branch are unsolved (we call unsolved a problem for which Ipopt can not guarantee optimality within the specified tolerances).}%
{}

\printoption{max\_random\_point\_radius}%
{$0<\textrm{real}$}%
{$100000$}%
{Set max value r for coordinate of a random point.\\
When picking a random point, coordinate i will be in the interval [min(max(l,-r),u-r), max(min(u,r),l+r)] (where l is the lower bound for the variable and u is its upper bound)}%
{}

\printoption{num\_iterations\_suspect}%
{$-1\leq\textrm{integer}$}%
{$-1$}%
{Number of iterations over which a node is considered "suspect" (for debugging purposes only, see detailed documentation).\\
When the number of iterations to solve a node is above this number, the subproblem at this node is considered to be suspect and it will be outputed in a file (set to -1 to deactivate this).}%
{}

\printoption{num\_retry\_unsolved\_random\_point}%
{$0\leq\textrm{integer}$}%
{$0$}%
{Number $k$ of times that the algorithm will try to resolve an unsolved NLP with a random starting point (we call unsolved an NLP for which Ipopt is not able to guarantee optimality within the specified tolerances).\\
When Ipopt fails to solve a continuous NLP sub-problem, if $k > 0$, the algorithm will try again to solve the failed NLP with $k$ new randomly chosen starting points  or until the problem is solved with success.}%
{}

\printoption{random\_point\_perturbation\_interval}%
{$0<\textrm{real}$}%
{$1$}%
{Amount by which starting point is perturbed when choosing to pick random point by perturbating starting point}%
{}

\printoption{random\_point\_type}%
{\ttfamily Jon, Andreas, Claudia}%
{Jon}%
{method to choose a random starting point}%
{\begin{list}{}{
\setlength{\parsep}{0em}
\setlength{\leftmargin}{5ex}
\setlength{\labelwidth}{2ex}
\setlength{\itemindent}{0ex}
\setlength{\topsep}{0pt}}
\item[\texttt{Jon}] Choose random point uniformly between the bounds
\item[\texttt{Andreas}] perturb the starting point of the problem within a prescribed interval
\item[\texttt{Claudia}] perturb the starting point using the perturbation radius suffix information
\end{list}
}

\printoptioncategory{NLP solves in hybrid algorithm (B-Hyb)}
\printoption{nlp\_solve\_frequency}%
{$0\leq\textrm{integer}$}%
{$10$}%
{Specify the frequency (in terms of nodes) at which NLP relaxations are solved in B-Hyb.\\
A frequency of 0 amounts to to never solve the NLP relaxation.}%
{}

\printoption{nlp\_solve\_max\_depth}%
{$0\leq\textrm{integer}$}%
{$10$}%
{Set maximum depth in the tree at which NLP relaxations are solved in B-Hyb.\\
A depth of 0 amounts to to never solve the NLP relaxation.}%
{}

\printoption{nlp\_solves\_per\_depth}%
{$0\leq\textrm{real}$}%
{$10^{ 100}$}%
{Set average number of nodes in the tree at which NLP relaxations are solved in B-Hyb for each depth.}%
{}

\printoptioncategory{Nonconvex problems}
\printoption{coeff\_var\_threshold}%
{$0\leq\textrm{real}$}%
{$0.1$}%
{Coefficient of variation threshold (for dynamic definition of cutoff\_decr).}%
{}

\printoption{dynamic\_def\_cutoff\_decr}%
{\ttfamily no, yes}%
{no}%
{Do you want to define the parameter cutoff\_decr dynamically?}%
{
}

\printoption{first\_perc\_for\_cutoff\_decr}%
{$\textrm{real}$}%
{$-0.02$}%
{The percentage used when, the coeff of variance is smaller than the threshold, to compute the cutoff\_decr dynamically.}%
{}

\printoption{max\_consecutive\_infeasible}%
{$0\leq\textrm{integer}$}%
{$0$}%
{Number of consecutive infeasible subproblems before aborting a branch.\\
Will continue exploring a branch of the tree until "max\_consecutive\_infeasible"consecutive problems are infeasibles by the NLP sub-solver.}%
{}

\printoption{num\_resolve\_at\_infeasibles}%
{$0\leq\textrm{integer}$}%
{$0$}%
{Number $k$ of tries to resolve an infeasible node (other than the root) of the tree with different starting point.\\
The algorithm will solve all the infeasible nodes with $k$ different random starting points and will keep the best local optimum found.}%
{}

\printoption{num\_resolve\_at\_node}%
{$0\leq\textrm{integer}$}%
{$0$}%
{Number $k$ of tries to resolve a node (other than the root) of the tree with different starting point.\\
The algorithm will solve all the nodes with $k$ different random starting points and will keep the best local optimum found.}%
{}

\printoption{num\_resolve\_at\_root}%
{$0\leq\textrm{integer}$}%
{$0$}%
{Number $k$ of tries to resolve the root node with different starting points.\\
The algorithm will solve the root node with $k$ random starting points and will keep the best local optimum found.}%
{}

\printoption{second\_perc\_for\_cutoff\_decr}%
{$\textrm{real}$}%
{$-0.05$}%
{The percentage used when, the coeff of variance is greater than the threshold, to compute the cutoff\_decr dynamically.}%
{}

\printoptioncategory{Outer Approximation Decomposition (B-OA)}
\printoption{oa\_decomposition}%
{\ttfamily no, yes}%
{no}%
{If yes do initial OA decomposition}%
{}

\printoptioncategory{Outer Approximation cuts generation}
\printoption{add\_only\_violated\_oa}%
{\ttfamily no, yes}%
{no}%
{Do we add all OA cuts or only the ones violated by current point?}%
{\begin{list}{}{
\setlength{\parsep}{0em}
\setlength{\leftmargin}{5ex}
\setlength{\labelwidth}{2ex}
\setlength{\itemindent}{0ex}
\setlength{\topsep}{0pt}}
\item[\texttt{no}] Add all cuts
\item[\texttt{yes}] Add only violated Cuts
\end{list}
}

\printoption{oa\_cuts\_scope}%
{\ttfamily local, global}%
{global}%
{Specify if OA cuts added are to be set globally or locally valid}%
{\begin{list}{}{
\setlength{\parsep}{0em}
\setlength{\leftmargin}{5ex}
\setlength{\labelwidth}{2ex}
\setlength{\itemindent}{0ex}
\setlength{\topsep}{0pt}}
\item[\texttt{local}] Cuts are treated as locally valid
\item[\texttt{global}] Cuts are treated as globally valid
\end{list}
}

\printoption{tiny\_element}%
{$-0\leq\textrm{real}$}%
{$10^{- 8}$}%
{Value for tiny element in OA cut\\
We will remove "cleanly" (by relaxing cut) an element lower than this.}%
{}

\printoption{very\_tiny\_element}%
{$-0\leq\textrm{real}$}%
{$10^{-17}$}%
{Value for very tiny element in OA cut\\
Algorithm will take the risk of neglecting an element lower than this.}%
{}

\printoptioncategory{Output}
\printoption{bb\_log\_interval}%
{$0\leq\textrm{integer}$}%
{$100$}%
{Interval at which node level output is printed.\\
Set the interval (in terms of number of nodes) at which a log on node resolutions (consisting of lower and upper bounds) is given.}%
{}

\printoption{bb\_log\_level}%
{$0\leq\textrm{integer}\leq5$}%
{$1$}%
{specify main branch-and-bound log level.\\
Set the level of output of the branch-and-bound : 0 - none, 1 - minimal, 2 - normal low, 3 - normal high}%
{}

\printoption{fp\_log\_frequency}%
{$0<\textrm{real}$}%
{$100$}%
{display an update on lower and upper bounds in FP every n seconds}%
{}

\printoption{fp\_log\_level}%
{$0\leq\textrm{integer}\leq2$}%
{$1$}%
{specify FP iterations log level.\\
Set the level of output of OA decomposition solver : 0 - none, 1 - normal, 2 - verbose}%
{}

\printoption{lp\_log\_level}%
{$0\leq\textrm{integer}\leq4$}%
{$0$}%
{specify LP log level.\\
Set the level of output of the linear programming sub-solver in B-Hyb or B-QG : 0 - none, 1 - minimal, 2 - normal low, 3 - normal high, 4 - verbose}%
{}

\printoption{milp\_log\_level}%
{$0\leq\textrm{integer}\leq4$}%
{$0$}%
{specify MILP solver log level.\\
Set the level of output of the MILP subsolver in OA : 0 - none, 1 - minimal, 2 - normal low, 3 - normal high}%
{}

\printoption{nlp\_log\_at\_root}%
{$0\leq\textrm{integer}\leq12$}%
{$5$}%
{ Specify a different log level for root relaxtion.}%
{}

\printoption{nlp\_log\_level}%
{$0\leq\textrm{integer}\leq2$}%
{$1$}%
{specify NLP solver interface log level (independent from ipopt print\_level).\\
Set the level of output of the OsiTMINLPInterface : 0 - none, 1 - normal, 2 - verbose}%
{}

\printoption{oa\_cuts\_log\_level}%
{$0\leq\textrm{integer}$}%
{$0$}%
{level of log when generating OA cuts.\\
0: outputs nothing,\\1: when a cut is generated, its violation and index of row from which it originates,\\2: always output violation of the cut.\\3: output generated cuts incidence vectors.}%
{}

\printoption{oa\_log\_frequency}%
{$0<\textrm{real}$}%
{$100$}%
{display an update on lower and upper bounds in OA every n seconds}%
{}

\printoption{oa\_log\_level}%
{$0\leq\textrm{integer}\leq2$}%
{$1$}%
{specify OA iterations log level.\\
Set the level of output of OA decomposition solver : 0 - none, 1 - normal, 2 - verbose}%
{}

\printoptioncategory{Strong branching setup}
\printoption{candidate\_sort\_criterion}%
{\ttfamily best-ps-cost, worst-ps-cost, most-fractional, least-fractional}%
{best-ps-cost}%
{Choice of the criterion to choose candidates in strong-branching}%
{\begin{list}{}{
\setlength{\parsep}{0em}
\setlength{\leftmargin}{5ex}
\setlength{\labelwidth}{2ex}
\setlength{\itemindent}{0ex}
\setlength{\topsep}{0pt}}
\item[\texttt{best-ps-cost}] Sort by decreasing pseudo-cost
\item[\texttt{worst-ps-cost}] Sort by increasing pseudo-cost
\item[\texttt{most-fractional}] Sort by decreasing integer infeasibility
\item[\texttt{least-fractional}] Sort by increasing integer infeasibility
\end{list}
}

\printoption{maxmin\_crit\_have\_sol}%
{$0\leq\textrm{real}\leq1$}%
{$0.1$}%
{Weight towards minimum in of lower and upper branching estimates when a solution has been found.}%
{}

\printoption{maxmin\_crit\_no\_sol}%
{$0\leq\textrm{real}\leq1$}%
{$0.7$}%
{Weight towards minimum in of lower and upper branching estimates when no solution has been found yet.}%
{}

\printoption{min\_number\_strong\_branch}%
{$0\leq\textrm{integer}$}%
{$0$}%
{Sets minimum number of variables for strong branching (overriding trust)}%
{}

\printoption{number\_before\_trust\_list}%
{$-1\leq\textrm{integer}$}%
{$0$}%
{Set the number of branches on a variable before its pseudo costs are to be believed during setup of strong branching candidate list.\\
The default value is that of "number\_before\_trust"}%
{}

\printoption{number\_look\_ahead}%
{$0\leq\textrm{integer}$}%
{$0$}%
{Sets limit of look-ahead strong-branching trials}%
{}

\printoption{number\_strong\_branch\_root}%
{$0\leq\textrm{integer}$}%
{$\infty$}%
{Maximum number of variables considered for strong branching in root node.}%
{}

\printoption{setup\_pseudo\_frac}%
{$0\leq\textrm{real}\leq1$}%
{$0.5$}%
{Proportion of strong branching list that has to be taken from most-integer-infeasible list.}%
{}

\printoption{trust\_strong\_branching\_for\_pseudo\_cost}%
{\ttfamily no, yes}%
{yes}%
{Whether or not to trust strong branching results for updating pseudo costs.}%
{}



\bibliographystyle{plain}
%\bibliography{coinlibd}
%\renewcommand{\bibname}{BONMIN References}
\chapter{\BONMIN and \BONMINH}

%\minitoc

COIN-OR \BONMIN (\textbf{B}asic \textbf{O}pen-source \textbf{N}onlinear \textbf{M}ixed \textbf{In}teger programming) is an open-source solver for mixed-integer nonlinear programming (MINLPs).
The code has been developed as part of a collaboration between Carnegie Mellon University and IBM Research.
The COIN-OR project leader for \BONMIN is Pierre Bonami.

\BONMIN can handle mixed-integer nonlinear programming models which functions should be twice continuously differentiable.
The \BONMIN link in \GAMS supports continuous, binary, and integer variables, special ordered sets, branching priorities, but no semi-continuous or semi-integer variables (see chapter 17.1 of the \GAMS User's Guide).


\BONMIN implements six different algorithms for solving MINLPs:
\begin{itemize}
\setlength{\partopsep}{0pt}
\setlength{\itemsep}{0pt}
\item {B-BB} (\textbf{default}): a simple branch-and-bound algorithm based on solving a continuous nonlinear program at each node of the search tree and branching on integer variables~\cite{GuptaRavindran85}; this algorithm is similar to the one implemented in the solver \textsc{SBB}
\item {B-OA}: an outer-approximation based decomposition algorithm based on iterating solving and improving of a MIP relaxation and solving NLP subproblems~\cite{DuGr86,FlLe94}; this algorithm is similar to the one implemented in the solver \textsc{DICOPT}
\item {B-QG}: an outer-approximation based branch-and-cut algorithm based on solving a continuous linear program at each node of the search tree, improving the linear program by outer approximation, and branching on integer variables~\cite{QeGr92}.
\item {B-Hyb}: a branch-and-bound algorithm which is a hybrid of B-BB and B-QG and is based on solving either a continuous nonlinear or a continuous linear program at each node of the search tree, improving the linear program by outer approximation, and branching on integer variables~\cite{BBCCGLLLMSW}
\item {B-ECP}: a Kelley's outer-approximation based branch-and-cut algorithm inspired by the settings used in the solver \textsc{FilMINT}~\cite{AbLeLi07}
\item {B-iFP}: an iterated feasibility pump algorithm~\cite{BoCoLoMa06}
\end{itemize}
The algorithms are exact when the problem is \textbf{convex}, otherwise they are heuristics.

For convex MINLPs, experiments on a reasonably large test set of problems have shown that B-Hyb is the algorithm of choice (it solved most of the problems in 3 hours of computing time).
Nevertheless, there are cases where B-OA (especially when used with CPLEX as MIP subproblem solver) is much faster than B-Hyb and others where B-BB is interesting.
B-QG and B-ECP corresponds mainly to a specific parameter setting of B-Hyb but they can be faster in some cases.
B-iFP is more tailored at finding quickly good solutions to very hard convex MINLP.
For \textbf{nonconvex} MINLPs, it is strongly recommended to use B-BB (the outer-approximation algorithms have not been tailored to treat nonconvex problems at this point).
Although even B-BB is only a heuristic for such problems, several options are available to try and improve the quality of the solutions it provides (see below).

NLPs are solved in \BONMIN by \IPOPT, which can use \textsc{MUMPS}~\cite{bonminAmestoyDuffKosterLExcellent2001,bonminAmestoyGuermoucheLExcellentPralet2006} (currently the default) or \textsc{MKL PARDISO}~\cite{bonminSchGa04,bonminSchGa06} (only Linux and Windows) as linear solver.
In the commerically licensed \GAMS/\BONMINH version, also the linear solvers \textsc{MA27}, \textsc{MA57}, \textsc{HSL\_MA86}, and \textsc{HSL\_MA97} from the Harwell Subroutines Library (HSL) are available in \IPOPT.
In this case, the default linear solver in \IPOPT is MA27.


For more information we refer to \cite{BoCoLoMa06,BoGo08,BoKiLi09,BBCCGLLLMSW} and the \BONMIN web site \texttt{https://projects.coin-or.org/Bonmin}.
Most of the \BONMIN documentation in this section is taken from the \BONMIN manual~\cite{BonminManual}.


%If \GAMS/\BONMIN is called for a model with only continuous variables, the interface switches over to \IPOPT.
%If \GAMS/\BONMIN is called for a model with only linear equations, the interface switches over to \CBC.

\section{Usage}

The following statement can be used inside your \GAMS program to specify using \BONMIN:
\begin{verbatim}
  Option MINLP = BONMIN;    { or Option MIQCP = BONMIN; }
\end{verbatim}
This statement should appear before the \texttt{Solve} statement.
If \BONMIN was specified as the default solver during \GAMS installation, the above statement is not necessary.

To use \BONMINH, one should use the statement
\begin{verbatim}
  Option MINLP = BONMINH;   { or Option MIQCP = BONMINH; }
\end{verbatim}

\GAMS/\BONMIN currently does not support the \GAMS Branch-and-Cut-and-Heuristic (BCH) Facility.
If you need to use \GAMS/\BONMIN with BCH, please consider to use a \GAMS system of version $\leq 23.3$, available at \url{http://www.gams.com/download/download_old.htm}.
% \GAMS/\BONMIN supports the \GAMS Branch-and-Cut-and-Heuristic (BCH) Facility.
% The \GAMS BCH facility automates all major steps necessary to define, execute, and control the use of user defined routines within the framework of general purpose MIP and MINLP codes.
% Currently supported are user defined cut generators and heuristics, where cut generator cannot be used in Bonmins pure B\&B algorithm (B-BB).
% Please see the BCH documentation at \texttt{http://www.gams.com/docs/bch.htm} for further information.

\subsection{Specification of Options}

A \BONMIN options file contains both \IPOPT and \BONMIN options, for clarity all \BONMIN options should be preceded with the prefix ``\texttt{bonmin.}''. %, except those corresponding to the BCH facility.
The scheme to name option files is the same as for all other \GAMS solvers.
Specifying \texttt{optfile=1} let \GAMS/\BONMIN read \texttt{bonmin.opt}, \texttt{optfile=2} corresponds to \texttt{bonmin.op2}, and so on.
The format of the option file is the same as for \IPOPT (see Section \ref{sub:ipoptoptionspec} in Chapter \ref{cha:ipopt}).

The most important option in \BONMIN is the choice of the solution algorithm.
This can be set by using the option named \texttt{bonmin.algorithm} which can be set to \texttt{B-BB}, \texttt{B-OA}, \texttt{B-QG}, \texttt{B-Hyb}, \texttt{B-ECP}, or \texttt{B-iFP} (its default value is \texttt{B-BB}).
Depending on the value of this option, certain other options may be available or not, cf.\ Section~\ref{sub:bonminalloptions}.

An example of a \texttt{bonmin.opt} file is the following:
\begin{verbatim}
   bonmin.algorithm       B-Hyb
   bonmin.oa_log_level    4
   print_level            6
\end{verbatim}
%    bonmin.milp_subsolver  Cbc_Par
%    milp_sub.cover_cuts    0
%    userheurcall           "bchheur.gms reslim 10"
This sets the algorithm to be used to the hybrid algorithm, the level of outer approximation related output to $4$,
% the MIP subsolver for outer approximation to a parameterized version of CBC, switches off cover cutting planes for the MIP subsolver,
and sets the print level for \IPOPT to $6$.
%  and let \BONMIN call a user defined heuristic specified in the model \texttt{bchheur.gms} with a timelimit of 10 seconds.

\GAMS/\BONMIN understands currently the following \GAMS parameters: \texttt{reslim} (time limit), \texttt{iterlim} (iteration limit), \texttt{nodlim} (node limit), \texttt{cutoff}, \texttt{optca} (absolute gap tolerance), and \texttt{optcr} (relative gap tolerance).
One can set them either on the command line, e.g. \verb+nodlim=1000+, or inside your \GAMS program, e.g. \verb+Option nodlim=1000;+.
Further, the option \texttt{threads} can be used to control the number of threads used in the linear algebra routines of \IPOPT, see Section~\ref{sec:ipoptlinearsolver} in Chapter~\ref{cha:ipopt} for details.

\subsection{Passing options to local search based heuristics and OA generators}
Several parts of the algorithms in \BONMIN are based on solving a simplified version of the problem with another instance of \BONMIN:
Outer Approximation Decomposition (called in {\tt B-Hyb} at the root node)
and Feasibility Pump for MINLP (called in {\tt B-Hyb} or {\tt B-BB} at the root node), RINS, RENS, Local Branching.

In all these cases, one can pass options to the sub-algorithm used through the option file.
The basic principle is that the ``\texttt{bonmin.}'' prefix  is replaced with a prefix that identifies the sub-algorithm used:
\begin{itemize}
\vspace{-2ex}
\setlength{\parskip}{.2ex}
\setlength{\itemsep}{0pt}
\setlength{\partopsep}{0pt}
\item to pass options to Outer Approximation Decomposition: {\tt oa\_decomposition.},
\item to pass options to Feasibility Pump for MINLP: {\tt pump\_for\_minlp.},
\item to pass options to RINS: {\tt rins.},
\item to pass options to RENS: {\tt rens.},
\item to pass options to Local Branching: {\tt local\_branch}.
\end{itemize}

\vspace{-2ex}
For example, to run a maximum of 60 seconds of feasibility pump (FP) for MINLP until 6 solutions are found at the beginning of the hybrid algorithm, one sets the following options:
\begin{verbatim}
bonmin.algorithm              B-Hyb
bonmin.pump_for_minlp         yes   # tells to run FP for MINLP
pump_for_minlp.time_limit     60    # set a time limit for the pump
pump_for_minlp.solution_limit 6     # set a solution limit
\end{verbatim}
Note that the actual solution and time limit will be the minimum of the global limits set for \BONMIN.

A slightly more complicated set of options may be used when using RINS.
Say for example that one wants to run RINS inside \texttt{B-BB}.
Each time RINS is called one wants to solve the small-size MINLP generated using B-QG (one may run any algorithm available in \BONMIN for solving an MINLP) and wants to stop as soon as \texttt{B-QG} found one solution.
To achieve this, one sets the following options
\begin{verbatim}
bonmin.algorithm      B-BB
bonmin.heuristic_rins yes
rins.algorithm        B-QG
rins.solution_limit   1
\end{verbatim}
This example shows that it is possible to set any option used in the sub-algorithm to be different than the one used for the main algorithm.

In the context of outer-approximation (OA) and feasibility pump for MINLP, a standard MILP solver is used.
Several options are available for configuring this MILP solver.
\BONMIN allows a choice of different MILP solvers through the option
\texttt{bonmin.milp\_sol\-ver}. Values for this option are: {\tt Cbc\_D} which uses \CBC with its
default settings, {\tt Cbc\_Par} which uses a version of \CBC that can be parameterized by the user, and \texttt{Cplex} which uses \CPLEX with its default settings.
The options that can be set in {\tt Cbc\_Par} are the number of strong-branching candidates,
the number of branches before pseudo costs are to be trusted, and the frequency of the various cut generators, c.f.\ Section~\ref{sub:bonminalloptions} for details.
To use the \texttt{Cplex} option, a valid \CPLEX licence (standalone or \GAMS/\CPLEX) is required.

\subsection{Getting good solutions to nonconvex problems}
To solve a problem with nonconvex constraints, one should only use the branch-and-bound algorithm {\tt B-BB}.

A few options have been designed in \BONMIN specifically to treat
problems that do not have a convex continuous relaxation.
In such problems, the solutions obtained from \IPOPT are
not necessarily globally optimal, but are only locally optimal.
Also the outer-approximation constraints are not necessarily valid inequalities for the problem.
No specific heuristic method for treating nonconvex problems is implemented
yet within the OA framework.
But for the pure branch-and-bound {\tt B-BB}, a few options have been implemented while having
in mind that lower bounds provided by \IPOPT should not be trusted and with the goal of
trying to get good solutions. Such options are at a very experimental stage.

First, in the context of nonconvex problems, \IPOPT may find different local optima when started
from different starting points. The two options {\tt num\_re\-solve\_at\_root} and {\tt num\_resolve\_at\_node}
allow for solving the root node or each node of the tree, respectively, with a user-specified
number of different randomly-chosen starting points, saving the best solution found.
Note that the function to generate a random starting point is very na\"{\i}ve:
it chooses a random point (uniformly) between the bounds provided for the variable.
In particular if there are some functions that can not be evaluated at some points of the domain, it may pick such points,
and so it is not robust in that respect.

Secondly, since the solution given by \IPOPT does not truly give a lower bound, the fathoming rule can be changed to continue branching even if the solution value to the current node is worse than the best-known solution.
This is achieved by setting {\tt allowable\_gap}
and {\tt allowable\_fraction\_gap} and {\tt cutoff\_decr} to negative values.

\subsection{\IPOPT options changed by \BONMIN}

\IPOPT has a very large number of options, see Section \ref{sub:ipoptoptions} to get a complete description.
To use \IPOPT more efficiently in the context of MINLP,
\BONMIN changes some \IPOPT options from their default values, which may help to improve \IPOPT's warm-starting capabilities and its ability to prove quickly that a subproblem is infeasible.
These are settings that \IPOPT does not use for ordinary NLP problems.
Note that options set by the user in an option file will override these settings.
\begin{itemize}
\vspace{-2ex}
\setlength{\partopsep}{0pt}
\setlength{\itemsep}{0pt}
\setlength{\parskip}{.5ex}
\item {\tt mu\_strategy} and {\tt mu\_oracle} are set, respectively, to
{\tt adaptive} and {\tt probing} by default. These are strategies in \IPOPT
for updating the barrier parameter. They were found to be more efficient in the context of MINLP.

\item {\tt gamma\_phi} and {\tt gamma\_theta} are set to $10^{-8}$ and $10^{-4}$ respectively.
This has the effect of reducing the size of the filter in the line search performed by \IPOPT.

\item {\tt required\_infeasibility\_reduction} is set to $0.1$.
This increases the required infeasibility reduction when \IPOPT enters the
restoration phase and should thus help to detect infeasible problems faster.

\item {\tt expect\_infeasible\_problem} is set to {\tt yes}, which enables some heuristics
to detect infeasible problems faster.

\item {\tt warm\_start\_init\_point} is set to {\tt yes} when a full primal/dual starting
point is available (generally for all the optimizations after the continuous relaxation has been solved).

\item {\tt print\_level} is set to $0$ by default to turn off \IPOPT output (except for the root node, which print level is controlled by the \BONMIN option \texttt{nlp\_log\_at\_root}).

\item \texttt{bound\_relax\_factor} is set to $10^{-10}$. All of the bounds
of the problem are relaxed by this factor. This may cause some trouble
when constraint functions can only be evaluated within their bounds.
In such cases, this option should be set to $0$.
\end{itemize}

\section{Detailed Options Description}
\label{sub:bonminalloptions}

The following tables give the list of options together with their types, default values, and availability in each of the main algorithms.
% The column labeled `type' indicates the type of the parameter.
%  (`F' stands for float, `I' for integer, and `S' for string).
% The column labeled `default' indicates the global default value.
% Then for each of the algorithms \texttt{B-BB}, \texttt{B-OA}, \texttt{B-QG}, \texttt{B-Hyb}, \texttt{B-Ecp}, and \texttt{B-iFP} `$+$' indicates that the option is available for that particular algorithm, while `$-$' indicates that it is not.
The column labeled `\texttt{Cbc\_Par}' indicates the options that can be used to parametrize the MLIP subsolver in the context of OA and FP.

% \newpage
\topcaption{\label{tab:bonminoptions} 
List of options and compatibility with the different algorithms.
}
\tablehead{\hline 
Option & type &  default & {\tt B-BB} & {\tt B-OA} & {\tt B-QG} & {\tt B-Hyb} & {\tt B-Ecp} & {\tt B-iFP} & {\tt Cbc\_Par} \\
\hline}
\tabletail{\hline \multicolumn{10}{|r|}{continued on next page}\\\hline}
{\small
\begin{xtabular}{@{}|l|@{\;}r@{\;}|@{\;}r@{\;}|@{\;}r@{\;}|@{\;}r@{\;}|@{\;}r@{\;}|@{\;}r@{\;}|@{\;}r@{\;}|@{\;}r@{\;}|@{\;}r@{\;}|@{}}
\multicolumn{1}{|c}{} & \multicolumn{9}{l|}{Algorithm choice}\\
\hline
algorithm & string & B-BB & $\checkmark$& $\checkmark$& $\checkmark$& $\checkmark$& $\checkmark$& $\checkmark$& $\checkmark$\\
\hline
\multicolumn{1}{|c}{} & \multicolumn{9}{l|}{Branch-and-bound options}\\
\hline
allowable\_fraction\_gap & $\mathbb{Q}$ & \GAMS \texttt{optcr} & $\checkmark$& $\checkmark$& $\checkmark$& $\checkmark$& $\checkmark$& $\checkmark$& $\checkmark$\\
allowable\_gap & $\mathbb{Q}$ & \GAMS \texttt{optca} & $\checkmark$& $\checkmark$& $\checkmark$& $\checkmark$& $\checkmark$& $\checkmark$& $\checkmark$\\
cutoff & $\mathbb{Q}$ & \GAMS \texttt{cutoff} & $\checkmark$& $\checkmark$& $\checkmark$& $\checkmark$& $\checkmark$& $\checkmark$& $\checkmark$\\
cutoff\_decr & $\mathbb{Q}$ & $10^{- 5}$ & $\checkmark$& $\checkmark$& $\checkmark$& $\checkmark$& $\checkmark$& $\checkmark$& $\checkmark$\\
enable\_dynamic\_nlp & string & no & $\checkmark$& --& --& --& --& --& --\\
integer\_tolerance & $\mathbb{Q}$ & $10^{- 6}$ & $\checkmark$& $\checkmark$& $\checkmark$& $\checkmark$& $\checkmark$& $\checkmark$& $\checkmark$\\
iteration\_limit & $\mathbb{Z}$ & \GAMS \texttt{iterlim} & $\checkmark$& $\checkmark$& $\checkmark$& $\checkmark$& $\checkmark$& $\checkmark$& $\checkmark$\\
nlp\_failure\_behavior & string & stop & $\checkmark$& --& --& --& --& --& --\\
node\_comparison & string & best-bound & $\checkmark$& $\checkmark$& $\checkmark$& $\checkmark$& $\checkmark$& $\checkmark$& --\\
node\_limit & $\mathbb{Z}$ & \GAMS \texttt{nodlim} & $\checkmark$& $\checkmark$& $\checkmark$& $\checkmark$& $\checkmark$& $\checkmark$& $\checkmark$\\
num\_cut\_passes & $\mathbb{Z}$ & $1$ & --& --& $\checkmark$& $\checkmark$& $\checkmark$& --& --\\
num\_cut\_passes\_at\_root & $\mathbb{Z}$ & $20$ & --& --& $\checkmark$& $\checkmark$& $\checkmark$& --& --\\
number\_before\_trust & $\mathbb{Z}$ & $8$ & $\checkmark$& $\checkmark$& $\checkmark$& $\checkmark$& $\checkmark$& $\checkmark$& $\checkmark$\\
number\_strong\_branch & $\mathbb{Z}$ & $20$ & $\checkmark$& $\checkmark$& $\checkmark$& $\checkmark$& $\checkmark$& $\checkmark$& $\checkmark$\\
random\_generator\_seed & $\mathbb{Z}$ & $0$ & $\checkmark$& $\checkmark$& $\checkmark$& $\checkmark$& $\checkmark$& $\checkmark$& $\checkmark$\\
read\_solution\_file & string & no & $\checkmark$& $\checkmark$& $\checkmark$& $\checkmark$& $\checkmark$& $\checkmark$& $\checkmark$\\
solution\_limit & $\mathbb{Z}$ & $\infty$ & $\checkmark$& $\checkmark$& $\checkmark$& $\checkmark$& $\checkmark$& $\checkmark$& $\checkmark$\\
time\_limit & $\mathbb{Q}$ & \GAMS \texttt{reslim} & $\checkmark$& $\checkmark$& $\checkmark$& $\checkmark$& $\checkmark$& $\checkmark$& $\checkmark$\\
tree\_search\_strategy & string & probed-dive & $\checkmark$& $\checkmark$& $\checkmark$& $\checkmark$& $\checkmark$& $\checkmark$& --\\
variable\_selection & string & strong-branching & $\checkmark$& --& $\checkmark$& $\checkmark$& $\checkmark$& --& --\\
\hline
\multicolumn{1}{|c}{} & \multicolumn{9}{l|}{ECP cuts generation}\\
\hline
ecp\_abs\_tol & $\mathbb{Q}$ & $10^{- 6}$ & --& --& $\checkmark$& $\checkmark$& --& --& --\\
ecp\_max\_rounds & $\mathbb{Z}$ & $5$ & --& --& $\checkmark$& $\checkmark$& --& --& --\\
ecp\_probability\_factor & $\mathbb{Q}$ & $10$ & --& --& $\checkmark$& $\checkmark$& --& --& --\\
ecp\_rel\_tol & $\mathbb{Q}$ & $0$ & --& --& $\checkmark$& $\checkmark$& --& --& --\\
filmint\_ecp\_cuts & $\mathbb{Z}$ & $0$ & --& --& $\checkmark$& $\checkmark$& --& --& --\\
\hline
\multicolumn{1}{|c}{} & \multicolumn{9}{l|}{Feasibility checker using OA cuts}\\
\hline
feas\_check\_cut\_types & string & outer-approx & --& --& $\checkmark$& $\checkmark$& $\checkmark$& --& --\\
feas\_check\_discard\_policy & string & detect-cycles & --& --& $\checkmark$& $\checkmark$& $\checkmark$& --& --\\
generate\_benders\_after\_so\_many\_oa & $\mathbb{Z}$ & $5000$ & --& --& $\checkmark$& $\checkmark$& $\checkmark$& --& --\\
\hline
\multicolumn{1}{|c}{} & \multicolumn{9}{l|}{MILP Solver}\\
\hline
cpx\_parallel\_strategy & $\mathbb{Z}$ & $0$ & --& --& --& --& --& --& $\checkmark$\\
milp\_solver & string & Cbc\_D & --& --& --& --& --& --& $\checkmark$\\
milp\_strategy & string & solve\_to\_optimality & --& --& --& --& --& --& $\checkmark$\\
number\_cpx\_threads & $\mathbb{Z}$ & $0$ & --& --& --& --& --& --& $\checkmark$\\
\hline
\multicolumn{1}{|c}{} & \multicolumn{9}{l|}{MILP cutting planes in hybrid algorithm}\\
\hline
2mir\_cuts & $\mathbb{Z}$ & $0$ & --& $\checkmark$& $\checkmark$& $\checkmark$& $\checkmark$& $\checkmark$& $\checkmark$\\
Gomory\_cuts & $\mathbb{Z}$ & $-5$ & --& $\checkmark$& $\checkmark$& $\checkmark$& $\checkmark$& $\checkmark$& $\checkmark$\\
clique\_cuts & $\mathbb{Z}$ & $-5$ & --& $\checkmark$& $\checkmark$& $\checkmark$& $\checkmark$& $\checkmark$& $\checkmark$\\
cover\_cuts & $\mathbb{Z}$ & $0$ & --& $\checkmark$& $\checkmark$& $\checkmark$& $\checkmark$& $\checkmark$& $\checkmark$\\
flow\_cover\_cuts & $\mathbb{Z}$ & $-5$ & --& $\checkmark$& $\checkmark$& $\checkmark$& $\checkmark$& $\checkmark$& $\checkmark$\\
lift\_and\_project\_cuts & $\mathbb{Z}$ & $0$ & --& $\checkmark$& $\checkmark$& $\checkmark$& $\checkmark$& $\checkmark$& $\checkmark$\\
mir\_cuts & $\mathbb{Z}$ & $-5$ & --& $\checkmark$& $\checkmark$& $\checkmark$& $\checkmark$& $\checkmark$& $\checkmark$\\
reduce\_and\_split\_cuts & $\mathbb{Z}$ & $0$ & --& $\checkmark$& $\checkmark$& $\checkmark$& $\checkmark$& $\checkmark$& $\checkmark$\\
\hline
\multicolumn{1}{|c}{} & \multicolumn{9}{l|}{NLP interface}\\
\hline
solvefinal & string & yes & $\checkmark$& $\checkmark$& $\checkmark$& $\checkmark$& $\checkmark$& $\checkmark$& $\checkmark$\\
warm\_start & string & none & $\checkmark$& --& --& --& --& --& --\\
\hline
\multicolumn{1}{|c}{} & \multicolumn{9}{l|}{NLP solution robustness}\\
\hline
max\_consecutive\_failures & $\mathbb{Z}$ & $10$ & $\checkmark$& --& --& --& --& --& --\\
max\_random\_point\_radius & $\mathbb{Q}$ & $100000$ & $\checkmark$& --& --& --& --& --& --\\
num\_iterations\_suspect & $\mathbb{Z}$ & $-1$ & $\checkmark$& $\checkmark$& $\checkmark$& $\checkmark$& $\checkmark$& $\checkmark$& $\checkmark$\\
num\_retry\_unsolved\_random\_point & $\mathbb{Z}$ & $0$ & $\checkmark$& $\checkmark$& $\checkmark$& $\checkmark$& $\checkmark$& $\checkmark$& $\checkmark$\\
random\_point\_perturbation\_interval & $\mathbb{Q}$ & $1$ & $\checkmark$& --& --& --& --& --& --\\
random\_point\_type & string & Jon & $\checkmark$& --& --& --& --& --& --\\
resolve\_on\_small\_infeasibility & $\mathbb{Q}$ & $0$ & $\checkmark$& --& --& --& --& --& --\\
\hline
\multicolumn{1}{|c}{} & \multicolumn{9}{l|}{NLP solves in hybrid algorithm (B-Hyb)}\\
\hline
nlp\_solve\_frequency & $\mathbb{Z}$ & $10$ & --& --& --& $\checkmark$& --& --& --\\
nlp\_solve\_max\_depth & $\mathbb{Z}$ & $10$ & --& --& --& $\checkmark$& --& --& --\\
nlp\_solves\_per\_depth & $\mathbb{Q}$ & $10^{ 100}$ & --& --& --& $\checkmark$& --& --& --\\
\hline
\multicolumn{1}{|c}{} & \multicolumn{9}{l|}{Nonconvex problems}\\
\hline
coeff\_var\_threshold & $\mathbb{Q}$ & $0.1$ & $\checkmark$& --& --& --& --& --& --\\
dynamic\_def\_cutoff\_decr & string & no & $\checkmark$& --& --& --& --& --& --\\
first\_perc\_for\_cutoff\_decr & $\mathbb{Q}$ & $-0.02$ & $\checkmark$& --& --& --& --& --& --\\
max\_consecutive\_infeasible & $\mathbb{Z}$ & $0$ & $\checkmark$& --& --& --& --& --& --\\
num\_resolve\_at\_infeasibles & $\mathbb{Z}$ & $0$ & $\checkmark$& --& --& --& --& --& --\\
num\_resolve\_at\_node & $\mathbb{Z}$ & $0$ & $\checkmark$& --& --& --& --& --& --\\
num\_resolve\_at\_root & $\mathbb{Z}$ & $0$ & $\checkmark$& --& --& --& --& --& --\\
second\_perc\_for\_cutoff\_decr & $\mathbb{Q}$ & $-0.05$ & $\checkmark$& --& --& --& --& --& --\\
\hline
\multicolumn{1}{|c}{} & \multicolumn{9}{l|}{Outer Approximation Decomposition (B-OA)}\\
\hline
oa\_decomposition & string & no & --& --& $\checkmark$& $\checkmark$& $\checkmark$& --& --\\
\hline
\multicolumn{1}{|c}{} & \multicolumn{9}{l|}{Outer Approximation cuts generation}\\
\hline
add\_only\_violated\_oa & string & no & --& $\checkmark$& $\checkmark$& $\checkmark$& $\checkmark$& $\checkmark$& $\checkmark$\\
oa\_cuts\_scope & string & global & --& $\checkmark$& $\checkmark$& $\checkmark$& $\checkmark$& $\checkmark$& $\checkmark$\\
oa\_rhs\_relax & $\mathbb{Q}$ & $10^{- 8}$ & --& $\checkmark$& $\checkmark$& $\checkmark$& $\checkmark$& $\checkmark$& $\checkmark$\\
tiny\_element & $\mathbb{Q}$ & $10^{- 8}$ & --& $\checkmark$& $\checkmark$& $\checkmark$& $\checkmark$& $\checkmark$& $\checkmark$\\
very\_tiny\_element & $\mathbb{Q}$ & $10^{-17}$ & --& $\checkmark$& $\checkmark$& $\checkmark$& $\checkmark$& $\checkmark$& $\checkmark$\\
\hline
\multicolumn{1}{|c}{} & \multicolumn{9}{l|}{Output and Loglevel}\\
\hline
bb\_log\_interval & $\mathbb{Z}$ & $100$ & $\checkmark$& $\checkmark$& $\checkmark$& $\checkmark$& $\checkmark$& $\checkmark$& $\checkmark$\\
bb\_log\_level & $\mathbb{Z}$ & $1$ & $\checkmark$& $\checkmark$& $\checkmark$& $\checkmark$& $\checkmark$& $\checkmark$& $\checkmark$\\
fp\_log\_frequency & $\mathbb{Q}$ & $100$ & --& --& $\checkmark$& $\checkmark$& --& --& --\\
fp\_log\_level & $\mathbb{Z}$ & $1$ & --& --& $\checkmark$& $\checkmark$& --& --& --\\
lp\_log\_level & $\mathbb{Z}$ & $0$ & --& $\checkmark$& $\checkmark$& $\checkmark$& $\checkmark$& $\checkmark$& $\checkmark$\\
milp\_log\_level & $\mathbb{Z}$ & $0$ & --& --& --& --& --& --& $\checkmark$\\
nlp\_log\_at\_root & $\mathbb{Z}$ & 5 & $\checkmark$& $\checkmark$& $\checkmark$& $\checkmark$& $\checkmark$& $\checkmark$& --\\
nlp\_log\_level & $\mathbb{Z}$ & $1$ & $\checkmark$& $\checkmark$& $\checkmark$& $\checkmark$& $\checkmark$& $\checkmark$& $\checkmark$\\
oa\_cuts\_log\_level & $\mathbb{Z}$ & $0$ & --& $\checkmark$& $\checkmark$& $\checkmark$& $\checkmark$& $\checkmark$& $\checkmark$\\
oa\_log\_frequency & $\mathbb{Q}$ & $100$ & $\checkmark$& --& --& $\checkmark$& $\checkmark$& --& --\\
oa\_log\_level & $\mathbb{Z}$ & $1$ & $\checkmark$& --& --& $\checkmark$& $\checkmark$& --& --\\
print\_funceval\_statistics & string & no & $\checkmark$& $\checkmark$& $\checkmark$& $\checkmark$& $\checkmark$& $\checkmark$& $\checkmark$\\
solvetrace & string &  & $\checkmark$& $\checkmark$& $\checkmark$& $\checkmark$& $\checkmark$& $\checkmark$& $\checkmark$\\
solvetracenodefreq & $\mathbb{Z}$ & $100$ & $\checkmark$& $\checkmark$& $\checkmark$& $\checkmark$& $\checkmark$& $\checkmark$& $\checkmark$\\
solvetracetimefreq & $\mathbb{Q}$ & $5$ & $\checkmark$& $\checkmark$& $\checkmark$& $\checkmark$& $\checkmark$& $\checkmark$& $\checkmark$\\
\hline
\multicolumn{1}{|c}{} & \multicolumn{9}{l|}{Primal Heuristics}\\
\hline
feasibility\_pump\_objective\_norm & $\mathbb{Z}$ & $1$ & $\checkmark$& $\checkmark$& $\checkmark$& $\checkmark$& $\checkmark$& $\checkmark$& --\\
fp\_pass\_infeasible & string & no & $\checkmark$& $\checkmark$& $\checkmark$& $\checkmark$& $\checkmark$& $\checkmark$& $\checkmark$\\
heuristic\_RINS & string & no & $\checkmark$& $\checkmark$& $\checkmark$& $\checkmark$& $\checkmark$& $\checkmark$& --\\
heuristic\_dive\_MIP\_fractional & string & no & $\checkmark$& $\checkmark$& $\checkmark$& $\checkmark$& $\checkmark$& $\checkmark$& --\\
heuristic\_dive\_MIP\_vectorLength & string & no & $\checkmark$& $\checkmark$& $\checkmark$& $\checkmark$& $\checkmark$& $\checkmark$& --\\
heuristic\_dive\_fractional & string & no & $\checkmark$& $\checkmark$& $\checkmark$& $\checkmark$& $\checkmark$& $\checkmark$& --\\
heuristic\_dive\_vectorLength & string & no & $\checkmark$& $\checkmark$& $\checkmark$& $\checkmark$& $\checkmark$& $\checkmark$& --\\
heuristic\_feasibility\_pump & string & no & $\checkmark$& $\checkmark$& $\checkmark$& $\checkmark$& $\checkmark$& $\checkmark$& --\\
pump\_for\_minlp & string & no & $\checkmark$& $\checkmark$& $\checkmark$& $\checkmark$& $\checkmark$& $\checkmark$& --\\
\hline
\multicolumn{1}{|c}{} & \multicolumn{9}{l|}{Strong branching setup}\\
\hline
candidate\_sort\_criterion & string & best-ps-cost & $\checkmark$& $\checkmark$& $\checkmark$& $\checkmark$& $\checkmark$& $\checkmark$& --\\
maxmin\_crit\_have\_sol & $\mathbb{Q}$ & $0.1$ & $\checkmark$& $\checkmark$& $\checkmark$& $\checkmark$& $\checkmark$& $\checkmark$& --\\
maxmin\_crit\_no\_sol & $\mathbb{Q}$ & $0.7$ & $\checkmark$& $\checkmark$& $\checkmark$& $\checkmark$& $\checkmark$& $\checkmark$& --\\
min\_number\_strong\_branch & $\mathbb{Z}$ & $0$ & $\checkmark$& $\checkmark$& $\checkmark$& $\checkmark$& $\checkmark$& $\checkmark$& --\\
number\_before\_trust\_list & $\mathbb{Z}$ & $0$ & $\checkmark$& $\checkmark$& $\checkmark$& $\checkmark$& $\checkmark$& $\checkmark$& --\\
number\_look\_ahead & $\mathbb{Z}$ & $0$ & $\checkmark$& $\checkmark$& $\checkmark$& $\checkmark$& $\checkmark$& --& --\\
number\_strong\_branch\_root & $\mathbb{Z}$ & $\infty$ & $\checkmark$& $\checkmark$& $\checkmark$& $\checkmark$& $\checkmark$& $\checkmark$& --\\
setup\_pseudo\_frac & $\mathbb{Q}$ & $0.5$ & $\checkmark$& $\checkmark$& $\checkmark$& $\checkmark$& $\checkmark$& $\checkmark$& --\\
trust\_strong\_branching\_for\_pseudo\_cost & string & yes & $\checkmark$& $\checkmark$& $\checkmark$& $\checkmark$& $\checkmark$& $\checkmark$& --\\
\hline
\end{xtabular}
}


In the following we give a detailed list of \BONMIN options.
The value on the right denotes the default value.
\printoptioncategory{Algorithm choice}
\printoption{algorithm}%
{\ttfamily B-BB, B-OA, B-QG, B-Hyb, B-Ecp, B-iFP}%
{B-BB}%
{Choice of the algorithm.\\
This will preset some of the options of bonmin depending on the algorithm choice.}%
{\begin{list}{}{
\setlength{\parsep}{0em}
\setlength{\leftmargin}{5ex}
\setlength{\labelwidth}{2ex}
\setlength{\itemindent}{0ex}
\setlength{\topsep}{0pt}}
\item[\texttt{B-BB}] simple branch-and-bound algorithm,
\item[\texttt{B-OA}] OA Decomposition algorithm,
\item[\texttt{B-QG}] Quesada and Grossmann branch-and-cut algorithm,
\item[\texttt{B-Hyb}] hybrid outer approximation based branch-and-cut,
\item[\texttt{B-Ecp}] ecp cuts based branch-and-cut a la FilMINT.
\item[\texttt{B-iFP}] Iterated Feasibility Pump for MINLP.
\end{list}
}

\printoptioncategory{Branch-and-bound options}
\printoption{allowable\_fraction\_gap}%
{$\textrm{real}$}%
{$0.1$}%
{Specify the value of relative gap under which the algorithm stops.\\
Stop the tree search when the gap between the objective value of the best known solution and the best bound on the objective of any solution is less than this fraction of the absolute value of the best known solution value.}%
{}

\printoption{allowable\_gap}%
{$\textrm{real}$}%
{$0$}%
{Specify the value of absolute gap under which the algorithm stops.\\
Stop the tree search when the gap between the objective value of the best known solution and the best bound on the objective of any solution is less than this.}%
{}

\printoption{cutoff}%
{$-10^{ 100}\leq\textrm{real}\leq10^{ 100}$}%
{$10^{ 100}$}%
{Specify cutoff value.\\
cutoff should be the value of a feasible solution known by the user (if any). The algorithm will only look for solutions better than cutoff.}%
{}

\printoption{cutoff\_decr}%
{$-10^{ 10}\leq\textrm{real}\leq10^{ 10}$}%
{$10^{- 5}$}%
{Specify cutoff decrement.\\
Specify the amount by which cutoff is decremented below a new best upper-bound (usually a small positive value but in non-convex problems it may be a negative value).}%
{}

\printoption{enable\_dynamic\_nlp}%
{\ttfamily no, yes}%
{no}%
{Enable dynamic linear and quadratic rows addition in nlp}%
{}

\printoption{integer\_tolerance}%
{$0<\textrm{real}$}%
{$10^{- 6}$}%
{Set integer tolerance.\\
Any number within that value of an integer is considered integer.}%
{}

\printoption{iteration\_limit}%
{$0\leq\textrm{integer}$}%
{$\infty$}%
{Set the cumulated maximum number of iteration in the algorithm used to process nodes continuous relaxations in the branch-and-bound.\\
value 0 deactivates option.}%
{}

\printoption{nlp\_failure\_behavior}%
{\ttfamily stop, fathom}%
{stop}%
{Set the behavior when an NLP or a series of NLP are unsolved by Ipopt (we call unsolved an NLP for which Ipopt is not able to guarantee optimality within the specified tolerances).\\
If set to "fathom", the algorithm will fathom the node when Ipopt fails to find a solution to the nlp at that node whithin the specified tolerances. The algorithm then becomes a heuristic, and the user will be warned that the solution might not be optimal.}%
{\begin{list}{}{
\setlength{\parsep}{0em}
\setlength{\leftmargin}{5ex}
\setlength{\labelwidth}{2ex}
\setlength{\itemindent}{0ex}
\setlength{\topsep}{0pt}}
\item[\texttt{stop}] Stop when failure happens.
\item[\texttt{fathom}] Continue when failure happens.
\end{list}
}

\printoption{node\_comparison}%
{\ttfamily best-bound, depth-first, breadth-first, dynamic, best-guess}%
{best-bound}%
{Choose the node selection strategy.\\
Choose the strategy for selecting the next node to be processed.}%
{\begin{list}{}{
\setlength{\parsep}{0em}
\setlength{\leftmargin}{5ex}
\setlength{\labelwidth}{2ex}
\setlength{\itemindent}{0ex}
\setlength{\topsep}{0pt}}
\item[\texttt{best-bound}] choose node with the smallest bound,
\item[\texttt{depth-first}] Perform depth first search,
\item[\texttt{breadth-first}] Perform breadth first search,
\item[\texttt{dynamic}] Cbc dynamic strategy (starts with a depth first search and turn to best bound after 3 integer feasible solutions have been found).
\item[\texttt{best-guess}] choose node with smallest guessed integer solution
\end{list}
}

\printoption{node\_limit}%
{$0\leq\textrm{integer}$}%
{$\infty$}%
{Set the maximum number of nodes explored in the branch-and-bound search.}%
{}

\printoption{num\_cut\_passes}%
{$0\leq\textrm{integer}$}%
{$1$}%
{Set the maximum number of cut passes at regular nodes of the branch-and-cut.}%
{}

\printoption{num\_cut\_passes\_at\_root}%
{$0\leq\textrm{integer}$}%
{$20$}%
{Set the maximum number of cut passes at regular nodes of the branch-and-cut.}%
{}

\printoption{number\_before\_trust}%
{$0\leq\textrm{integer}$}%
{$8$}%
{Set the number of branches on a variable before its pseudo costs are to be believed in dynamic strong branching.\\
A value of 0 disables pseudo costs.}%
{}

\printoption{number\_strong\_branch}%
{$0\leq\textrm{integer}$}%
{$20$}%
{Choose the maximum number of variables considered for strong branching.\\
Set the number of variables on which to do strong branching.}%
{}

\printoption{solution\_limit}%
{$0\leq\textrm{integer}$}%
{$\infty$}%
{Abort after that much integer feasible solution have been found by algorithm\\
value 0 deactivates option}%
{}

\printoption{time\_limit}%
{$0\leq\textrm{real}$}%
{$1000$}%
{Set the global maximum computation time (in secs) for the algorithm.}%
{}

\printoption{tree\_search\_strategy}%
{\ttfamily top-node, dive, probed-dive, dfs-dive, dfs-dive-dynamic}%
{probed-dive}%
{Pick a strategy for traversing the tree\\
All strategies can be used in conjunction with any of the node comparison functions. Options which affect dfs-dive are max-backtracks-in-dive and max-dive-depth. The dfs-dive won't work in a non-convex problem where objective does not decrease down branches.}%
{\begin{list}{}{
\setlength{\parsep}{0em}
\setlength{\leftmargin}{5ex}
\setlength{\labelwidth}{2ex}
\setlength{\itemindent}{0ex}
\setlength{\topsep}{0pt}}
\item[\texttt{top-node}]  Always pick the top node as sorted by the node comparison function
\item[\texttt{dive}] Dive in the tree if possible, otherwise pick top node as sorted by the tree comparison function.
\item[\texttt{probed-dive}] Dive in the tree exploring two childs before continuing the dive at each level.
\item[\texttt{dfs-dive}] Dive in the tree if possible doing a depth first search. Backtrack on leaves or when a prescribed depth is attained or when estimate of best possible integer feasible solution in subtree is worst than cutoff. Once a prescribed limit of backtracks is attained pick top node as sorted by the tree comparison function
\item[\texttt{dfs-dive-dynamic}] Same as dfs-dive but once enough solution are found switch to best-bound and if too many nodes switch to depth-first.
\end{list}
}

\printoption{variable\_selection}%
{\ttfamily most-fractional, strong-branching, reliability-branching, curvature-estimator, qp-strong-branching, lp-strong-branching, nlp-strong-branching, osi-simple, osi-strong, random}%
{strong-branching}%
{Chooses variable selection strategy}%
{\begin{list}{}{
\setlength{\parsep}{0em}
\setlength{\leftmargin}{5ex}
\setlength{\labelwidth}{2ex}
\setlength{\itemindent}{0ex}
\setlength{\topsep}{0pt}}
\item[\texttt{most-fractional}] Choose most fractional variable
\item[\texttt{strong-branching}] Perform strong branching
\item[\texttt{reliability-branching}] Use reliability branching
\item[\texttt{curvature-estimator}] Use curvature estimation to select branching variable
\item[\texttt{qp-strong-branching}] Perform strong branching with QP approximation
\item[\texttt{lp-strong-branching}] Perform strong branching with LP approximation
\item[\texttt{nlp-strong-branching}] Perform strong branching with NLP approximation
\item[\texttt{osi-simple}] Osi method to do simple branching
\item[\texttt{osi-strong}] Osi method to do strong branching
\item[\texttt{random}] Method to choose branching variable randomly
\end{list}
}

\printoptioncategory{ECP cuts generation}
\printoption{ecp\_abs\_tol}%
{$0\leq\textrm{real}$}%
{$10^{- 6}$}%
{Set the absolute termination tolerance for ECP rounds.}%
{}

\printoption{ecp\_max\_rounds}%
{$0\leq\textrm{integer}$}%
{$5$}%
{Set the maximal number of rounds of ECP cuts.}%
{}

\printoption{ecp\_probability\_factor}%
{$\textrm{real}$}%
{$10$}%
{Factor appearing in formula for skipping ECP cuts.\\
Choosing -1 disables the skipping.}%
{}

\printoption{ecp\_rel\_tol}%
{$0\leq\textrm{real}$}%
{$0$}%
{Set the relative termination tolerance for ECP rounds.}%
{}

\printoption{filmint\_ecp\_cuts}%
{$0\leq\textrm{integer}$}%
{$0$}%
{Specify the frequency (in terms of nodes) at which some a la filmint ecp cuts are generated.\\
A frequency of 0 amounts to to never solve the NLP relaxation.}%
{}

\printoptioncategory{Feasibility checker using OA cuts}
\printoption{feas\_check\_cut\_types}%
{\ttfamily outer-approx, Benders}%
{outer-approx}%
{Choose the type of cuts generated when an integer feasible solution is found\\
If it seems too much memory is used should try Benders to use less}%
{\begin{list}{}{
\setlength{\parsep}{0em}
\setlength{\leftmargin}{5ex}
\setlength{\labelwidth}{2ex}
\setlength{\itemindent}{0ex}
\setlength{\topsep}{0pt}}
\item[\texttt{outer-approx}] Generate a set of Outer Approximations cuts.
\item[\texttt{Benders}] Generate a single Benders cut.
\end{list}
}

\printoption{feas\_check\_discard\_policy}%
{\ttfamily detect-cycles, keep-all, treated-as-normal}%
{detect-cycles}%
{How cuts from feasibility checker are discarded\\
Normally to avoid cycle cuts from feasibility checker should not be discarded in the node where they are generated. However Cbc sometimes does it if no care is taken which can lead to an infinite loop in Bonmin (usualy on simple problems). To avoid this one can instruct Cbc to never discard a cut but if we do that for all cuts it can lead to memory problems. The default policy here is to detect cycles and only then impose to Cbc to keep the cut. The two other alternative are to instruct Cbc to keep all cuts or to just ignore the problem and hope for the best}%
{\begin{list}{}{
\setlength{\parsep}{0em}
\setlength{\leftmargin}{5ex}
\setlength{\labelwidth}{2ex}
\setlength{\itemindent}{0ex}
\setlength{\topsep}{0pt}}
\item[\texttt{detect-cycles}] Detect if a cycle occurs and only in this case force not to discard.
\item[\texttt{keep-all}] Force cuts from feasibility checker not to be discarded (memory hungry but sometimes better).
\item[\texttt{treated-as-normal}] Cuts from memory checker can be discarded as any other cuts (code may cycle then)
\end{list}
}

\printoption{generate\_benders\_after\_so\_many\_oa}%
{$0\leq\textrm{integer}$}%
{$5000$}%
{Specify that after so many oa cuts have been generated Benders cuts should be generated instead.\\
It seems that sometimes generating too many oa cuts slows down the optimization compared to Benders due to the size of the LP. With this option we specify that after so many OA cuts have been generated we should switch to Benders cuts.}%
{}

\printoptioncategory{MILP Solver}
\printoption{cpx\_parallel\_strategy}%
{$-1\leq\textrm{integer}\leq1$}%
{$0$}%
{Strategy of parallel search mode in CPLEX.\\
-1 = opportunistic, 0 = automatic, 1 = deterministic (refer to CPLEX documentation)}%
{}

\printoption{milp\_solver}%
{Cbc\_D, Cbc\_Par, Cplex}%
{Cbc\_D}%
{Choose the subsolver to solve MILP sub-problems in OA decompositions.\\
 To use Cplex, a valid license is required.}%
{\begin{list}{}{
\setlength{\parsep}{0em}
\setlength{\leftmargin}{5ex}
\setlength{\labelwidth}{2ex}
\setlength{\itemindent}{0ex}
\setlength{\topsep}{0pt}}
\item[\texttt{Cbc\_D}] Coin Branch and Cut with its default
\item[\texttt{Cbc\_Par}] Coin Branch and Cut with passed parameters
\item[\texttt{Cplex}] IBM CPLEX
\end{list}
}

\printoption{milp\_strategy}%
{\ttfamily find\_good\_sol, solve\_to\_optimality}%
{find\_good\_sol}%
{Choose a strategy for MILPs.}%
{\begin{list}{}{
\setlength{\parsep}{0em}
\setlength{\leftmargin}{5ex}
\setlength{\labelwidth}{2ex}
\setlength{\itemindent}{0ex}
\setlength{\topsep}{0pt}}
\item[\texttt{find\_good\_sol}] Stop sub milps when a solution improving the incumbent is found
\item[\texttt{solve\_to\_optimality}] Solve MILPs to optimality
\end{list}
}

\printoption{number\_cpx\_threads}%
{$0\leq\textrm{integer}$}%
{$0$}%
{Set number of threads to use with cplex.\\
(refer to CPLEX documentation)}%
{}

\printoptioncategory{MILP cutting planes in hybrid algorithm (B-Hyb)}
\printoption{2mir\_cuts}%
{$-100\leq\textrm{integer}$}%
{$0$}%
{Frequency (in terms of nodes) for generating 2-MIR cuts in branch-and-cut\\
If k $>$ 0, cuts are generated every k nodes, if -99 $<$ k $<$ 0 cuts are generated every -k nodes but Cbc may decide to stop generating cuts, if not enough are generated at the root node, if k=-99 generate cuts only at the root node, if k=0 or 100 do not generate cuts.}%
{}

\printoption{Gomory\_cuts}%
{$-100\leq\textrm{integer}$}%
{$-5$}%
{Frequency k (in terms of nodes) for generating Gomory cuts in branch-and-cut.\\
See option \texttt{2mir\_cuts} for the meaning of k.}%
{}

\printoption{clique\_cuts}%
{$-100\leq\textrm{integer}$}%
{$-5$}%
{Frequency (in terms of nodes) for generating clique cuts in branch-and-cut\\
See option \texttt{2mir\_cuts} for the meaning of k.}%
{}

\printoption{cover\_cuts}%
{$-100\leq\textrm{integer}$}%
{$0$}%
{Frequency (in terms of nodes) for generating cover cuts in branch-and-cut\\
See option \texttt{2mir\_cuts} for the meaning of k.}%
{}

\printoption{flow\_cover\_cuts}%
{$-100\leq\textrm{integer}$}%
{$-5$}%
{Frequency (in terms of nodes) for generating flow cover cuts in branch-and-cut\\
See option \texttt{2mir\_cuts} for the meaning of k.}%
{}

\printoption{lift\_and\_project\_cuts}%
{$-100\leq\textrm{integer}$}%
{$0$}%
{Frequency (in terms of nodes) for generating lift-and-project cuts in branch-and-cut\\
See option \texttt{2mir\_cuts} for the meaning of k.}%
{}

\printoption{mir\_cuts}%
{$-100\leq\textrm{integer}$}%
{$-5$}%
{Frequency (in terms of nodes) for generating MIR cuts in branch-and-cut\\
See option \texttt{2mir\_cuts} for the meaning of k.}%
{}

\printoption{reduce\_and\_split\_cuts}%
{$-100\leq\textrm{integer}$}%
{$0$}%
{Frequency (in terms of nodes) for generating reduce-and-split cuts in branch-and-cut\\
See option \texttt{2mir\_cuts} for the meaning of k.}%
{}

\printoptioncategory{MINLP Heuristics}
\printoption{feasibility\_pump\_objective\_norm}%
{$1\leq\textrm{integer}\leq2$}%
{$1$}%
{Norm of feasibility pump objective function}%
{}

\printoption{fp\_pass\_infeasible}%
{\ttfamily no, yes}%
{no}%
{Say whether feasibility pump should claim to converge or not}%
{\begin{list}{}{
\setlength{\parsep}{0em}
\setlength{\leftmargin}{5ex}
\setlength{\labelwidth}{2ex}
\setlength{\itemindent}{0ex}
\setlength{\topsep}{0pt}}
\item[\texttt{no}] When master MILP is infeasible just bail out (don't stop all algorithm). This is the option for using in B-Hyb.
\item[\texttt{yes}] Claim convergence, numerically dangerous.
\end{list}
}

\printoption{heuristic\_RINS}%
{\ttfamily no, yes}%
{no}%
{if yes runs the RINS heuristic}%
{
}

\printoption{heuristic\_dive\_MIP\_fractional}%
{\ttfamily no, yes}%
{no}%
{if yes runs the Dive MIP Fractional heuristic}%
{
}

\printoption{heuristic\_dive\_MIP\_vectorLength}%
{\ttfamily no, yes}%
{no}%
{if yes runs the Dive MIP VectorLength heuristic}%
{
}

\printoption{heuristic\_dive\_fractional}%
{\ttfamily no, yes}%
{no}%
{if yes runs the Dive Fractional heuristic}%
{
}

\printoption{heuristic\_dive\_vectorLength}%
{\ttfamily no, yes}%
{no}%
{if yes runs the Dive VectorLength heuristic}%
{
}

\printoption{heuristic\_feasibility\_pump}%
{\ttfamily no, yes}%
{no}%
{whether the heuristic feasibility pump should be used}%
{
}

\printoption{pump\_for\_minlp}%
{\ttfamily no, yes}%
{no}%
{if yes runs FP for MINLP}%
{
}

\printoptioncategory{NLP interface}
\printoption{warm\_start}%
{\ttfamily none, optimum, interior\_point}%
{none}%
{Select the warm start method\\
This will affect the function getWarmStart(), and as a consequence the warm starting in the various algorithms.}%
{\begin{list}{}{
\setlength{\parsep}{0em}
\setlength{\leftmargin}{5ex}
\setlength{\labelwidth}{2ex}
\setlength{\itemindent}{0ex}
\setlength{\topsep}{0pt}}
\item[\texttt{none}] No warm start
\item[\texttt{optimum}] Warm start with direct parent optimum
\item[\texttt{interior\_point}] Warm start with an interior point of direct parent
\end{list}
}

\printoptioncategory{NLP solution robustness}
\printoption{max\_consecutive\_failures}%
{$0\leq\textrm{integer}$}%
{$10$}%
{(temporarily removed) Number $n$ of consecutive unsolved problems before aborting a branch of the tree.\\
When $n > 0$, continue exploring a branch of the tree until $n$ consecutive problems in the branch are unsolved (we call unsolved a problem for which Ipopt can not guarantee optimality within the specified tolerances).}%
{}

\printoption{max\_random\_point\_radius}%
{$0<\textrm{real}$}%
{$100000$}%
{Set max value r for coordinate of a random point.\\
When picking a random point, coordinate i will be in the interval [min(max(l,-r),u-r), max(min(u,r),l+r)] (where l is the lower bound for the variable and u is its upper bound)}%
{}

\printoption{num\_iterations\_suspect}%
{$-1\leq\textrm{integer}$}%
{$-1$}%
{Number of iterations over which a node is considered "suspect" (for debugging purposes only, see detailed documentation).\\
When the number of iterations to solve a node is above this number, the subproblem at this node is considered to be suspect and it will be outputed in a file (set to -1 to deactivate this).}%
{}

\printoption{num\_retry\_unsolved\_random\_point}%
{$0\leq\textrm{integer}$}%
{$0$}%
{Number $k$ of times that the algorithm will try to resolve an unsolved NLP with a random starting point (we call unsolved an NLP for which Ipopt is not able to guarantee optimality within the specified tolerances).\\
When Ipopt fails to solve a continuous NLP sub-problem, if $k > 0$, the algorithm will try again to solve the failed NLP with $k$ new randomly chosen starting points  or until the problem is solved with success.}%
{}

\printoption{random\_point\_perturbation\_interval}%
{$0<\textrm{real}$}%
{$1$}%
{Amount by which starting point is perturbed when choosing to pick random point by perturbating starting point}%
{}

\printoption{random\_point\_type}%
{\ttfamily Jon, Andreas, Claudia}%
{Jon}%
{method to choose a random starting point}%
{\begin{list}{}{
\setlength{\parsep}{0em}
\setlength{\leftmargin}{5ex}
\setlength{\labelwidth}{2ex}
\setlength{\itemindent}{0ex}
\setlength{\topsep}{0pt}}
\item[\texttt{Jon}] Choose random point uniformly between the bounds
\item[\texttt{Andreas}] perturb the starting point of the problem within a prescribed interval
\item[\texttt{Claudia}] perturb the starting point using the perturbation radius suffix information
\end{list}
}

\printoptioncategory{NLP solves in hybrid algorithm (B-Hyb)}
\printoption{nlp\_solve\_frequency}%
{$0\leq\textrm{integer}$}%
{$10$}%
{Specify the frequency (in terms of nodes) at which NLP relaxations are solved in B-Hyb.\\
A frequency of 0 amounts to to never solve the NLP relaxation.}%
{}

\printoption{nlp\_solve\_max\_depth}%
{$0\leq\textrm{integer}$}%
{$10$}%
{Set maximum depth in the tree at which NLP relaxations are solved in B-Hyb.\\
A depth of 0 amounts to to never solve the NLP relaxation.}%
{}

\printoption{nlp\_solves\_per\_depth}%
{$0\leq\textrm{real}$}%
{$10^{ 100}$}%
{Set average number of nodes in the tree at which NLP relaxations are solved in B-Hyb for each depth.}%
{}

\printoptioncategory{Nonconvex problems}
\printoption{coeff\_var\_threshold}%
{$0\leq\textrm{real}$}%
{$0.1$}%
{Coefficient of variation threshold (for dynamic definition of cutoff\_decr).}%
{}

\printoption{dynamic\_def\_cutoff\_decr}%
{\ttfamily no, yes}%
{no}%
{Do you want to define the parameter cutoff\_decr dynamically?}%
{
}

\printoption{first\_perc\_for\_cutoff\_decr}%
{$\textrm{real}$}%
{$-0.02$}%
{The percentage used when, the coeff of variance is smaller than the threshold, to compute the cutoff\_decr dynamically.}%
{}

\printoption{max\_consecutive\_infeasible}%
{$0\leq\textrm{integer}$}%
{$0$}%
{Number of consecutive infeasible subproblems before aborting a branch.\\
Will continue exploring a branch of the tree until "max\_consecutive\_infeasible"consecutive problems are infeasibles by the NLP sub-solver.}%
{}

\printoption{num\_resolve\_at\_infeasibles}%
{$0\leq\textrm{integer}$}%
{$0$}%
{Number $k$ of tries to resolve an infeasible node (other than the root) of the tree with different starting point.\\
The algorithm will solve all the infeasible nodes with $k$ different random starting points and will keep the best local optimum found.}%
{}

\printoption{num\_resolve\_at\_node}%
{$0\leq\textrm{integer}$}%
{$0$}%
{Number $k$ of tries to resolve a node (other than the root) of the tree with different starting point.\\
The algorithm will solve all the nodes with $k$ different random starting points and will keep the best local optimum found.}%
{}

\printoption{num\_resolve\_at\_root}%
{$0\leq\textrm{integer}$}%
{$0$}%
{Number $k$ of tries to resolve the root node with different starting points.\\
The algorithm will solve the root node with $k$ random starting points and will keep the best local optimum found.}%
{}

\printoption{second\_perc\_for\_cutoff\_decr}%
{$\textrm{real}$}%
{$-0.05$}%
{The percentage used when, the coeff of variance is greater than the threshold, to compute the cutoff\_decr dynamically.}%
{}

\printoptioncategory{Outer Approximation Decomposition (B-OA)}
\printoption{oa\_decomposition}%
{\ttfamily no, yes}%
{no}%
{If yes do initial OA decomposition}%
{}

\printoptioncategory{Outer Approximation cuts generation}
\printoption{add\_only\_violated\_oa}%
{\ttfamily no, yes}%
{no}%
{Do we add all OA cuts or only the ones violated by current point?}%
{\begin{list}{}{
\setlength{\parsep}{0em}
\setlength{\leftmargin}{5ex}
\setlength{\labelwidth}{2ex}
\setlength{\itemindent}{0ex}
\setlength{\topsep}{0pt}}
\item[\texttt{no}] Add all cuts
\item[\texttt{yes}] Add only violated Cuts
\end{list}
}

\printoption{oa\_cuts\_scope}%
{\ttfamily local, global}%
{global}%
{Specify if OA cuts added are to be set globally or locally valid}%
{\begin{list}{}{
\setlength{\parsep}{0em}
\setlength{\leftmargin}{5ex}
\setlength{\labelwidth}{2ex}
\setlength{\itemindent}{0ex}
\setlength{\topsep}{0pt}}
\item[\texttt{local}] Cuts are treated as locally valid
\item[\texttt{global}] Cuts are treated as globally valid
\end{list}
}

\printoption{tiny\_element}%
{$-0\leq\textrm{real}$}%
{$10^{- 8}$}%
{Value for tiny element in OA cut\\
We will remove "cleanly" (by relaxing cut) an element lower than this.}%
{}

\printoption{very\_tiny\_element}%
{$-0\leq\textrm{real}$}%
{$10^{-17}$}%
{Value for very tiny element in OA cut\\
Algorithm will take the risk of neglecting an element lower than this.}%
{}

\printoptioncategory{Output}
\printoption{bb\_log\_interval}%
{$0\leq\textrm{integer}$}%
{$100$}%
{Interval at which node level output is printed.\\
Set the interval (in terms of number of nodes) at which a log on node resolutions (consisting of lower and upper bounds) is given.}%
{}

\printoption{bb\_log\_level}%
{$0\leq\textrm{integer}\leq5$}%
{$1$}%
{specify main branch-and-bound log level.\\
Set the level of output of the branch-and-bound : 0 - none, 1 - minimal, 2 - normal low, 3 - normal high}%
{}

\printoption{fp\_log\_frequency}%
{$0<\textrm{real}$}%
{$100$}%
{display an update on lower and upper bounds in FP every n seconds}%
{}

\printoption{fp\_log\_level}%
{$0\leq\textrm{integer}\leq2$}%
{$1$}%
{specify FP iterations log level.\\
Set the level of output of OA decomposition solver : 0 - none, 1 - normal, 2 - verbose}%
{}

\printoption{lp\_log\_level}%
{$0\leq\textrm{integer}\leq4$}%
{$0$}%
{specify LP log level.\\
Set the level of output of the linear programming sub-solver in B-Hyb or B-QG : 0 - none, 1 - minimal, 2 - normal low, 3 - normal high, 4 - verbose}%
{}

\printoption{milp\_log\_level}%
{$0\leq\textrm{integer}\leq4$}%
{$0$}%
{specify MILP solver log level.\\
Set the level of output of the MILP subsolver in OA : 0 - none, 1 - minimal, 2 - normal low, 3 - normal high}%
{}

\printoption{nlp\_log\_at\_root}%
{$0\leq\textrm{integer}\leq12$}%
{$5$}%
{ Specify a different log level for root relaxtion.}%
{}

\printoption{nlp\_log\_level}%
{$0\leq\textrm{integer}\leq2$}%
{$1$}%
{specify NLP solver interface log level (independent from ipopt print\_level).\\
Set the level of output of the OsiTMINLPInterface : 0 - none, 1 - normal, 2 - verbose}%
{}

\printoption{oa\_cuts\_log\_level}%
{$0\leq\textrm{integer}$}%
{$0$}%
{level of log when generating OA cuts.\\
0: outputs nothing,\\1: when a cut is generated, its violation and index of row from which it originates,\\2: always output violation of the cut.\\3: output generated cuts incidence vectors.}%
{}

\printoption{oa\_log\_frequency}%
{$0<\textrm{real}$}%
{$100$}%
{display an update on lower and upper bounds in OA every n seconds}%
{}

\printoption{oa\_log\_level}%
{$0\leq\textrm{integer}\leq2$}%
{$1$}%
{specify OA iterations log level.\\
Set the level of output of OA decomposition solver : 0 - none, 1 - normal, 2 - verbose}%
{}

\printoptioncategory{Strong branching setup}
\printoption{candidate\_sort\_criterion}%
{\ttfamily best-ps-cost, worst-ps-cost, most-fractional, least-fractional}%
{best-ps-cost}%
{Choice of the criterion to choose candidates in strong-branching}%
{\begin{list}{}{
\setlength{\parsep}{0em}
\setlength{\leftmargin}{5ex}
\setlength{\labelwidth}{2ex}
\setlength{\itemindent}{0ex}
\setlength{\topsep}{0pt}}
\item[\texttt{best-ps-cost}] Sort by decreasing pseudo-cost
\item[\texttt{worst-ps-cost}] Sort by increasing pseudo-cost
\item[\texttt{most-fractional}] Sort by decreasing integer infeasibility
\item[\texttt{least-fractional}] Sort by increasing integer infeasibility
\end{list}
}

\printoption{maxmin\_crit\_have\_sol}%
{$0\leq\textrm{real}\leq1$}%
{$0.1$}%
{Weight towards minimum in of lower and upper branching estimates when a solution has been found.}%
{}

\printoption{maxmin\_crit\_no\_sol}%
{$0\leq\textrm{real}\leq1$}%
{$0.7$}%
{Weight towards minimum in of lower and upper branching estimates when no solution has been found yet.}%
{}

\printoption{min\_number\_strong\_branch}%
{$0\leq\textrm{integer}$}%
{$0$}%
{Sets minimum number of variables for strong branching (overriding trust)}%
{}

\printoption{number\_before\_trust\_list}%
{$-1\leq\textrm{integer}$}%
{$0$}%
{Set the number of branches on a variable before its pseudo costs are to be believed during setup of strong branching candidate list.\\
The default value is that of "number\_before\_trust"}%
{}

\printoption{number\_look\_ahead}%
{$0\leq\textrm{integer}$}%
{$0$}%
{Sets limit of look-ahead strong-branching trials}%
{}

\printoption{number\_strong\_branch\_root}%
{$0\leq\textrm{integer}$}%
{$\infty$}%
{Maximum number of variables considered for strong branching in root node.}%
{}

\printoption{setup\_pseudo\_frac}%
{$0\leq\textrm{real}\leq1$}%
{$0.5$}%
{Proportion of strong branching list that has to be taken from most-integer-infeasible list.}%
{}

\printoption{trust\_strong\_branching\_for\_pseudo\_cost}%
{\ttfamily no, yes}%
{yes}%
{Whether or not to trust strong branching results for updating pseudo costs.}%
{}



\bibliographystyle{plain}
%\bibliography{coinlibd}
%\renewcommand{\bibname}{BONMIN References}
\chapter{\BONMIN and \BONMINH}

%\minitoc

COIN-OR \BONMIN (\textbf{B}asic \textbf{O}pen-source \textbf{N}onlinear \textbf{M}ixed \textbf{In}teger programming) is an open-source solver for mixed-integer nonlinear programming (MINLPs).
The code has been developed as part of a collaboration between Carnegie Mellon University and IBM Research.
The COIN-OR project leader for \BONMIN is Pierre Bonami.

\BONMIN can handle mixed-integer nonlinear programming models which functions should be twice continuously differentiable.
The \BONMIN link in \GAMS supports continuous, binary, and integer variables, special ordered sets, branching priorities, but no semi-continuous or semi-integer variables (see chapter 17.1 of the \GAMS User's Guide).


\BONMIN implements six different algorithms for solving MINLPs:
\begin{itemize}
\setlength{\partopsep}{0pt}
\setlength{\itemsep}{0pt}
\item {B-BB} (\textbf{default}): a simple branch-and-bound algorithm based on solving a continuous nonlinear program at each node of the search tree and branching on integer variables~\cite{GuptaRavindran85}; this algorithm is similar to the one implemented in the solver \textsc{SBB}
\item {B-OA}: an outer-approximation based decomposition algorithm based on iterating solving and improving of a MIP relaxation and solving NLP subproblems~\cite{DuGr86,FlLe94}; this algorithm is similar to the one implemented in the solver \textsc{DICOPT}
\item {B-QG}: an outer-approximation based branch-and-cut algorithm based on solving a continuous linear program at each node of the search tree, improving the linear program by outer approximation, and branching on integer variables~\cite{QeGr92}.
\item {B-Hyb}: a branch-and-bound algorithm which is a hybrid of B-BB and B-QG and is based on solving either a continuous nonlinear or a continuous linear program at each node of the search tree, improving the linear program by outer approximation, and branching on integer variables~\cite{BBCCGLLLMSW}
\item {B-ECP}: a Kelley's outer-approximation based branch-and-cut algorithm inspired by the settings used in the solver \textsc{FilMINT}~\cite{AbLeLi07}
\item {B-iFP}: an iterated feasibility pump algorithm~\cite{BoCoLoMa06}
\end{itemize}
The algorithms are exact when the problem is \textbf{convex}, otherwise they are heuristics.

For convex MINLPs, experiments on a reasonably large test set of problems have shown that B-Hyb is the algorithm of choice (it solved most of the problems in 3 hours of computing time).
Nevertheless, there are cases where B-OA (especially when used with CPLEX as MIP subproblem solver) is much faster than B-Hyb and others where B-BB is interesting.
B-QG and B-ECP corresponds mainly to a specific parameter setting of B-Hyb but they can be faster in some cases.
B-iFP is more tailored at finding quickly good solutions to very hard convex MINLP.
For \textbf{nonconvex} MINLPs, it is strongly recommended to use B-BB (the outer-approximation algorithms have not been tailored to treat nonconvex problems at this point).
Although even B-BB is only a heuristic for such problems, several options are available to try and improve the quality of the solutions it provides (see below).

NLPs are solved in \BONMIN by \IPOPT, which can use \textsc{MUMPS}~\cite{bonminAmestoyDuffKosterLExcellent2001,bonminAmestoyGuermoucheLExcellentPralet2006} (currently the default) or \textsc{MKL PARDISO}~\cite{bonminSchGa04,bonminSchGa06} (only Linux and Windows) as linear solver.
In the commerically licensed \GAMS/\BONMINH version, also the linear solvers \textsc{MA27}, \textsc{MA57}, \textsc{HSL\_MA86}, and \textsc{HSL\_MA97} from the Harwell Subroutines Library (HSL) are available in \IPOPT.
In this case, the default linear solver in \IPOPT is MA27.


For more information we refer to \cite{BoCoLoMa06,BoGo08,BoKiLi09,BBCCGLLLMSW} and the \BONMIN web site \texttt{https://projects.coin-or.org/Bonmin}.
Most of the \BONMIN documentation in this section is taken from the \BONMIN manual~\cite{BonminManual}.


%If \GAMS/\BONMIN is called for a model with only continuous variables, the interface switches over to \IPOPT.
%If \GAMS/\BONMIN is called for a model with only linear equations, the interface switches over to \CBC.

\section{Usage}

The following statement can be used inside your \GAMS program to specify using \BONMIN:
\begin{verbatim}
  Option MINLP = BONMIN;    { or Option MIQCP = BONMIN; }
\end{verbatim}
This statement should appear before the \texttt{Solve} statement.
If \BONMIN was specified as the default solver during \GAMS installation, the above statement is not necessary.

To use \BONMINH, one should use the statement
\begin{verbatim}
  Option MINLP = BONMINH;   { or Option MIQCP = BONMINH; }
\end{verbatim}

\GAMS/\BONMIN currently does not support the \GAMS Branch-and-Cut-and-Heuristic (BCH) Facility.
If you need to use \GAMS/\BONMIN with BCH, please consider to use a \GAMS system of version $\leq 23.3$, available at \url{http://www.gams.com/download/download_old.htm}.
% \GAMS/\BONMIN supports the \GAMS Branch-and-Cut-and-Heuristic (BCH) Facility.
% The \GAMS BCH facility automates all major steps necessary to define, execute, and control the use of user defined routines within the framework of general purpose MIP and MINLP codes.
% Currently supported are user defined cut generators and heuristics, where cut generator cannot be used in Bonmins pure B\&B algorithm (B-BB).
% Please see the BCH documentation at \texttt{http://www.gams.com/docs/bch.htm} for further information.

\subsection{Specification of Options}

A \BONMIN options file contains both \IPOPT and \BONMIN options, for clarity all \BONMIN options should be preceded with the prefix ``\texttt{bonmin.}''. %, except those corresponding to the BCH facility.
The scheme to name option files is the same as for all other \GAMS solvers.
Specifying \texttt{optfile=1} let \GAMS/\BONMIN read \texttt{bonmin.opt}, \texttt{optfile=2} corresponds to \texttt{bonmin.op2}, and so on.
The format of the option file is the same as for \IPOPT (see Section \ref{sub:ipoptoptionspec} in Chapter \ref{cha:ipopt}).

The most important option in \BONMIN is the choice of the solution algorithm.
This can be set by using the option named \texttt{bonmin.algorithm} which can be set to \texttt{B-BB}, \texttt{B-OA}, \texttt{B-QG}, \texttt{B-Hyb}, \texttt{B-ECP}, or \texttt{B-iFP} (its default value is \texttt{B-BB}).
Depending on the value of this option, certain other options may be available or not, cf.\ Section~\ref{sub:bonminalloptions}.

An example of a \texttt{bonmin.opt} file is the following:
\begin{verbatim}
   bonmin.algorithm       B-Hyb
   bonmin.oa_log_level    4
   print_level            6
\end{verbatim}
%    bonmin.milp_subsolver  Cbc_Par
%    milp_sub.cover_cuts    0
%    userheurcall           "bchheur.gms reslim 10"
This sets the algorithm to be used to the hybrid algorithm, the level of outer approximation related output to $4$,
% the MIP subsolver for outer approximation to a parameterized version of CBC, switches off cover cutting planes for the MIP subsolver,
and sets the print level for \IPOPT to $6$.
%  and let \BONMIN call a user defined heuristic specified in the model \texttt{bchheur.gms} with a timelimit of 10 seconds.

\GAMS/\BONMIN understands currently the following \GAMS parameters: \texttt{reslim} (time limit), \texttt{iterlim} (iteration limit), \texttt{nodlim} (node limit), \texttt{cutoff}, \texttt{optca} (absolute gap tolerance), and \texttt{optcr} (relative gap tolerance).
One can set them either on the command line, e.g. \verb+nodlim=1000+, or inside your \GAMS program, e.g. \verb+Option nodlim=1000;+.
Further, the option \texttt{threads} can be used to control the number of threads used in the linear algebra routines of \IPOPT, see Section~\ref{sec:ipoptlinearsolver} in Chapter~\ref{cha:ipopt} for details.

\subsection{Passing options to local search based heuristics and OA generators}
Several parts of the algorithms in \BONMIN are based on solving a simplified version of the problem with another instance of \BONMIN:
Outer Approximation Decomposition (called in {\tt B-Hyb} at the root node)
and Feasibility Pump for MINLP (called in {\tt B-Hyb} or {\tt B-BB} at the root node), RINS, RENS, Local Branching.

In all these cases, one can pass options to the sub-algorithm used through the option file.
The basic principle is that the ``\texttt{bonmin.}'' prefix  is replaced with a prefix that identifies the sub-algorithm used:
\begin{itemize}
\vspace{-2ex}
\setlength{\parskip}{.2ex}
\setlength{\itemsep}{0pt}
\setlength{\partopsep}{0pt}
\item to pass options to Outer Approximation Decomposition: {\tt oa\_decomposition.},
\item to pass options to Feasibility Pump for MINLP: {\tt pump\_for\_minlp.},
\item to pass options to RINS: {\tt rins.},
\item to pass options to RENS: {\tt rens.},
\item to pass options to Local Branching: {\tt local\_branch}.
\end{itemize}

\vspace{-2ex}
For example, to run a maximum of 60 seconds of feasibility pump (FP) for MINLP until 6 solutions are found at the beginning of the hybrid algorithm, one sets the following options:
\begin{verbatim}
bonmin.algorithm              B-Hyb
bonmin.pump_for_minlp         yes   # tells to run FP for MINLP
pump_for_minlp.time_limit     60    # set a time limit for the pump
pump_for_minlp.solution_limit 6     # set a solution limit
\end{verbatim}
Note that the actual solution and time limit will be the minimum of the global limits set for \BONMIN.

A slightly more complicated set of options may be used when using RINS.
Say for example that one wants to run RINS inside \texttt{B-BB}.
Each time RINS is called one wants to solve the small-size MINLP generated using B-QG (one may run any algorithm available in \BONMIN for solving an MINLP) and wants to stop as soon as \texttt{B-QG} found one solution.
To achieve this, one sets the following options
\begin{verbatim}
bonmin.algorithm      B-BB
bonmin.heuristic_rins yes
rins.algorithm        B-QG
rins.solution_limit   1
\end{verbatim}
This example shows that it is possible to set any option used in the sub-algorithm to be different than the one used for the main algorithm.

In the context of outer-approximation (OA) and feasibility pump for MINLP, a standard MILP solver is used.
Several options are available for configuring this MILP solver.
\BONMIN allows a choice of different MILP solvers through the option
\texttt{bonmin.milp\_sol\-ver}. Values for this option are: {\tt Cbc\_D} which uses \CBC with its
default settings, {\tt Cbc\_Par} which uses a version of \CBC that can be parameterized by the user, and \texttt{Cplex} which uses \CPLEX with its default settings.
The options that can be set in {\tt Cbc\_Par} are the number of strong-branching candidates,
the number of branches before pseudo costs are to be trusted, and the frequency of the various cut generators, c.f.\ Section~\ref{sub:bonminalloptions} for details.
To use the \texttt{Cplex} option, a valid \CPLEX licence (standalone or \GAMS/\CPLEX) is required.

\subsection{Getting good solutions to nonconvex problems}
To solve a problem with nonconvex constraints, one should only use the branch-and-bound algorithm {\tt B-BB}.

A few options have been designed in \BONMIN specifically to treat
problems that do not have a convex continuous relaxation.
In such problems, the solutions obtained from \IPOPT are
not necessarily globally optimal, but are only locally optimal.
Also the outer-approximation constraints are not necessarily valid inequalities for the problem.
No specific heuristic method for treating nonconvex problems is implemented
yet within the OA framework.
But for the pure branch-and-bound {\tt B-BB}, a few options have been implemented while having
in mind that lower bounds provided by \IPOPT should not be trusted and with the goal of
trying to get good solutions. Such options are at a very experimental stage.

First, in the context of nonconvex problems, \IPOPT may find different local optima when started
from different starting points. The two options {\tt num\_re\-solve\_at\_root} and {\tt num\_resolve\_at\_node}
allow for solving the root node or each node of the tree, respectively, with a user-specified
number of different randomly-chosen starting points, saving the best solution found.
Note that the function to generate a random starting point is very na\"{\i}ve:
it chooses a random point (uniformly) between the bounds provided for the variable.
In particular if there are some functions that can not be evaluated at some points of the domain, it may pick such points,
and so it is not robust in that respect.

Secondly, since the solution given by \IPOPT does not truly give a lower bound, the fathoming rule can be changed to continue branching even if the solution value to the current node is worse than the best-known solution.
This is achieved by setting {\tt allowable\_gap}
and {\tt allowable\_fraction\_gap} and {\tt cutoff\_decr} to negative values.

\subsection{\IPOPT options changed by \BONMIN}

\IPOPT has a very large number of options, see Section \ref{sub:ipoptoptions} to get a complete description.
To use \IPOPT more efficiently in the context of MINLP,
\BONMIN changes some \IPOPT options from their default values, which may help to improve \IPOPT's warm-starting capabilities and its ability to prove quickly that a subproblem is infeasible.
These are settings that \IPOPT does not use for ordinary NLP problems.
Note that options set by the user in an option file will override these settings.
\begin{itemize}
\vspace{-2ex}
\setlength{\partopsep}{0pt}
\setlength{\itemsep}{0pt}
\setlength{\parskip}{.5ex}
\item {\tt mu\_strategy} and {\tt mu\_oracle} are set, respectively, to
{\tt adaptive} and {\tt probing} by default. These are strategies in \IPOPT
for updating the barrier parameter. They were found to be more efficient in the context of MINLP.

\item {\tt gamma\_phi} and {\tt gamma\_theta} are set to $10^{-8}$ and $10^{-4}$ respectively.
This has the effect of reducing the size of the filter in the line search performed by \IPOPT.

\item {\tt required\_infeasibility\_reduction} is set to $0.1$.
This increases the required infeasibility reduction when \IPOPT enters the
restoration phase and should thus help to detect infeasible problems faster.

\item {\tt expect\_infeasible\_problem} is set to {\tt yes}, which enables some heuristics
to detect infeasible problems faster.

\item {\tt warm\_start\_init\_point} is set to {\tt yes} when a full primal/dual starting
point is available (generally for all the optimizations after the continuous relaxation has been solved).

\item {\tt print\_level} is set to $0$ by default to turn off \IPOPT output (except for the root node, which print level is controlled by the \BONMIN option \texttt{nlp\_log\_at\_root}).

\item \texttt{bound\_relax\_factor} is set to $10^{-10}$. All of the bounds
of the problem are relaxed by this factor. This may cause some trouble
when constraint functions can only be evaluated within their bounds.
In such cases, this option should be set to $0$.
\end{itemize}

\section{Detailed Options Description}
\label{sub:bonminalloptions}

The following tables give the list of options together with their types, default values, and availability in each of the main algorithms.
% The column labeled `type' indicates the type of the parameter.
%  (`F' stands for float, `I' for integer, and `S' for string).
% The column labeled `default' indicates the global default value.
% Then for each of the algorithms \texttt{B-BB}, \texttt{B-OA}, \texttt{B-QG}, \texttt{B-Hyb}, \texttt{B-Ecp}, and \texttt{B-iFP} `$+$' indicates that the option is available for that particular algorithm, while `$-$' indicates that it is not.
The column labeled `\texttt{Cbc\_Par}' indicates the options that can be used to parametrize the MLIP subsolver in the context of OA and FP.

% \newpage
\topcaption{\label{tab:bonminoptions} 
List of options and compatibility with the different algorithms.
}
\tablehead{\hline 
Option & type &  default & {\tt B-BB} & {\tt B-OA} & {\tt B-QG} & {\tt B-Hyb} & {\tt B-Ecp} & {\tt B-iFP} & {\tt Cbc\_Par} \\
\hline}
\tabletail{\hline \multicolumn{10}{|r|}{continued on next page}\\\hline}
{\small
\begin{xtabular}{@{}|l|@{\;}r@{\;}|@{\;}r@{\;}|@{\;}r@{\;}|@{\;}r@{\;}|@{\;}r@{\;}|@{\;}r@{\;}|@{\;}r@{\;}|@{\;}r@{\;}|@{\;}r@{\;}|@{}}
\multicolumn{1}{|c}{} & \multicolumn{9}{l|}{Algorithm choice}\\
\hline
algorithm & string & B-BB & $\checkmark$& $\checkmark$& $\checkmark$& $\checkmark$& $\checkmark$& $\checkmark$& $\checkmark$\\
\hline
\multicolumn{1}{|c}{} & \multicolumn{9}{l|}{Branch-and-bound options}\\
\hline
allowable\_fraction\_gap & $\mathbb{Q}$ & \GAMS \texttt{optcr} & $\checkmark$& $\checkmark$& $\checkmark$& $\checkmark$& $\checkmark$& $\checkmark$& $\checkmark$\\
allowable\_gap & $\mathbb{Q}$ & \GAMS \texttt{optca} & $\checkmark$& $\checkmark$& $\checkmark$& $\checkmark$& $\checkmark$& $\checkmark$& $\checkmark$\\
cutoff & $\mathbb{Q}$ & \GAMS \texttt{cutoff} & $\checkmark$& $\checkmark$& $\checkmark$& $\checkmark$& $\checkmark$& $\checkmark$& $\checkmark$\\
cutoff\_decr & $\mathbb{Q}$ & $10^{- 5}$ & $\checkmark$& $\checkmark$& $\checkmark$& $\checkmark$& $\checkmark$& $\checkmark$& $\checkmark$\\
enable\_dynamic\_nlp & string & no & $\checkmark$& --& --& --& --& --& --\\
integer\_tolerance & $\mathbb{Q}$ & $10^{- 6}$ & $\checkmark$& $\checkmark$& $\checkmark$& $\checkmark$& $\checkmark$& $\checkmark$& $\checkmark$\\
iteration\_limit & $\mathbb{Z}$ & \GAMS \texttt{iterlim} & $\checkmark$& $\checkmark$& $\checkmark$& $\checkmark$& $\checkmark$& $\checkmark$& $\checkmark$\\
nlp\_failure\_behavior & string & stop & $\checkmark$& --& --& --& --& --& --\\
node\_comparison & string & best-bound & $\checkmark$& $\checkmark$& $\checkmark$& $\checkmark$& $\checkmark$& $\checkmark$& --\\
node\_limit & $\mathbb{Z}$ & \GAMS \texttt{nodlim} & $\checkmark$& $\checkmark$& $\checkmark$& $\checkmark$& $\checkmark$& $\checkmark$& $\checkmark$\\
num\_cut\_passes & $\mathbb{Z}$ & $1$ & --& --& $\checkmark$& $\checkmark$& $\checkmark$& --& --\\
num\_cut\_passes\_at\_root & $\mathbb{Z}$ & $20$ & --& --& $\checkmark$& $\checkmark$& $\checkmark$& --& --\\
number\_before\_trust & $\mathbb{Z}$ & $8$ & $\checkmark$& $\checkmark$& $\checkmark$& $\checkmark$& $\checkmark$& $\checkmark$& $\checkmark$\\
number\_strong\_branch & $\mathbb{Z}$ & $20$ & $\checkmark$& $\checkmark$& $\checkmark$& $\checkmark$& $\checkmark$& $\checkmark$& $\checkmark$\\
random\_generator\_seed & $\mathbb{Z}$ & $0$ & $\checkmark$& $\checkmark$& $\checkmark$& $\checkmark$& $\checkmark$& $\checkmark$& $\checkmark$\\
read\_solution\_file & string & no & $\checkmark$& $\checkmark$& $\checkmark$& $\checkmark$& $\checkmark$& $\checkmark$& $\checkmark$\\
solution\_limit & $\mathbb{Z}$ & $\infty$ & $\checkmark$& $\checkmark$& $\checkmark$& $\checkmark$& $\checkmark$& $\checkmark$& $\checkmark$\\
time\_limit & $\mathbb{Q}$ & \GAMS \texttt{reslim} & $\checkmark$& $\checkmark$& $\checkmark$& $\checkmark$& $\checkmark$& $\checkmark$& $\checkmark$\\
tree\_search\_strategy & string & probed-dive & $\checkmark$& $\checkmark$& $\checkmark$& $\checkmark$& $\checkmark$& $\checkmark$& --\\
variable\_selection & string & strong-branching & $\checkmark$& --& $\checkmark$& $\checkmark$& $\checkmark$& --& --\\
\hline
\multicolumn{1}{|c}{} & \multicolumn{9}{l|}{ECP cuts generation}\\
\hline
ecp\_abs\_tol & $\mathbb{Q}$ & $10^{- 6}$ & --& --& $\checkmark$& $\checkmark$& --& --& --\\
ecp\_max\_rounds & $\mathbb{Z}$ & $5$ & --& --& $\checkmark$& $\checkmark$& --& --& --\\
ecp\_probability\_factor & $\mathbb{Q}$ & $10$ & --& --& $\checkmark$& $\checkmark$& --& --& --\\
ecp\_rel\_tol & $\mathbb{Q}$ & $0$ & --& --& $\checkmark$& $\checkmark$& --& --& --\\
filmint\_ecp\_cuts & $\mathbb{Z}$ & $0$ & --& --& $\checkmark$& $\checkmark$& --& --& --\\
\hline
\multicolumn{1}{|c}{} & \multicolumn{9}{l|}{Feasibility checker using OA cuts}\\
\hline
feas\_check\_cut\_types & string & outer-approx & --& --& $\checkmark$& $\checkmark$& $\checkmark$& --& --\\
feas\_check\_discard\_policy & string & detect-cycles & --& --& $\checkmark$& $\checkmark$& $\checkmark$& --& --\\
generate\_benders\_after\_so\_many\_oa & $\mathbb{Z}$ & $5000$ & --& --& $\checkmark$& $\checkmark$& $\checkmark$& --& --\\
\hline
\multicolumn{1}{|c}{} & \multicolumn{9}{l|}{MILP Solver}\\
\hline
cpx\_parallel\_strategy & $\mathbb{Z}$ & $0$ & --& --& --& --& --& --& $\checkmark$\\
milp\_solver & string & Cbc\_D & --& --& --& --& --& --& $\checkmark$\\
milp\_strategy & string & solve\_to\_optimality & --& --& --& --& --& --& $\checkmark$\\
number\_cpx\_threads & $\mathbb{Z}$ & $0$ & --& --& --& --& --& --& $\checkmark$\\
\hline
\multicolumn{1}{|c}{} & \multicolumn{9}{l|}{MILP cutting planes in hybrid algorithm}\\
\hline
2mir\_cuts & $\mathbb{Z}$ & $0$ & --& $\checkmark$& $\checkmark$& $\checkmark$& $\checkmark$& $\checkmark$& $\checkmark$\\
Gomory\_cuts & $\mathbb{Z}$ & $-5$ & --& $\checkmark$& $\checkmark$& $\checkmark$& $\checkmark$& $\checkmark$& $\checkmark$\\
clique\_cuts & $\mathbb{Z}$ & $-5$ & --& $\checkmark$& $\checkmark$& $\checkmark$& $\checkmark$& $\checkmark$& $\checkmark$\\
cover\_cuts & $\mathbb{Z}$ & $0$ & --& $\checkmark$& $\checkmark$& $\checkmark$& $\checkmark$& $\checkmark$& $\checkmark$\\
flow\_cover\_cuts & $\mathbb{Z}$ & $-5$ & --& $\checkmark$& $\checkmark$& $\checkmark$& $\checkmark$& $\checkmark$& $\checkmark$\\
lift\_and\_project\_cuts & $\mathbb{Z}$ & $0$ & --& $\checkmark$& $\checkmark$& $\checkmark$& $\checkmark$& $\checkmark$& $\checkmark$\\
mir\_cuts & $\mathbb{Z}$ & $-5$ & --& $\checkmark$& $\checkmark$& $\checkmark$& $\checkmark$& $\checkmark$& $\checkmark$\\
reduce\_and\_split\_cuts & $\mathbb{Z}$ & $0$ & --& $\checkmark$& $\checkmark$& $\checkmark$& $\checkmark$& $\checkmark$& $\checkmark$\\
\hline
\multicolumn{1}{|c}{} & \multicolumn{9}{l|}{NLP interface}\\
\hline
solvefinal & string & yes & $\checkmark$& $\checkmark$& $\checkmark$& $\checkmark$& $\checkmark$& $\checkmark$& $\checkmark$\\
warm\_start & string & none & $\checkmark$& --& --& --& --& --& --\\
\hline
\multicolumn{1}{|c}{} & \multicolumn{9}{l|}{NLP solution robustness}\\
\hline
max\_consecutive\_failures & $\mathbb{Z}$ & $10$ & $\checkmark$& --& --& --& --& --& --\\
max\_random\_point\_radius & $\mathbb{Q}$ & $100000$ & $\checkmark$& --& --& --& --& --& --\\
num\_iterations\_suspect & $\mathbb{Z}$ & $-1$ & $\checkmark$& $\checkmark$& $\checkmark$& $\checkmark$& $\checkmark$& $\checkmark$& $\checkmark$\\
num\_retry\_unsolved\_random\_point & $\mathbb{Z}$ & $0$ & $\checkmark$& $\checkmark$& $\checkmark$& $\checkmark$& $\checkmark$& $\checkmark$& $\checkmark$\\
random\_point\_perturbation\_interval & $\mathbb{Q}$ & $1$ & $\checkmark$& --& --& --& --& --& --\\
random\_point\_type & string & Jon & $\checkmark$& --& --& --& --& --& --\\
resolve\_on\_small\_infeasibility & $\mathbb{Q}$ & $0$ & $\checkmark$& --& --& --& --& --& --\\
\hline
\multicolumn{1}{|c}{} & \multicolumn{9}{l|}{NLP solves in hybrid algorithm (B-Hyb)}\\
\hline
nlp\_solve\_frequency & $\mathbb{Z}$ & $10$ & --& --& --& $\checkmark$& --& --& --\\
nlp\_solve\_max\_depth & $\mathbb{Z}$ & $10$ & --& --& --& $\checkmark$& --& --& --\\
nlp\_solves\_per\_depth & $\mathbb{Q}$ & $10^{ 100}$ & --& --& --& $\checkmark$& --& --& --\\
\hline
\multicolumn{1}{|c}{} & \multicolumn{9}{l|}{Nonconvex problems}\\
\hline
coeff\_var\_threshold & $\mathbb{Q}$ & $0.1$ & $\checkmark$& --& --& --& --& --& --\\
dynamic\_def\_cutoff\_decr & string & no & $\checkmark$& --& --& --& --& --& --\\
first\_perc\_for\_cutoff\_decr & $\mathbb{Q}$ & $-0.02$ & $\checkmark$& --& --& --& --& --& --\\
max\_consecutive\_infeasible & $\mathbb{Z}$ & $0$ & $\checkmark$& --& --& --& --& --& --\\
num\_resolve\_at\_infeasibles & $\mathbb{Z}$ & $0$ & $\checkmark$& --& --& --& --& --& --\\
num\_resolve\_at\_node & $\mathbb{Z}$ & $0$ & $\checkmark$& --& --& --& --& --& --\\
num\_resolve\_at\_root & $\mathbb{Z}$ & $0$ & $\checkmark$& --& --& --& --& --& --\\
second\_perc\_for\_cutoff\_decr & $\mathbb{Q}$ & $-0.05$ & $\checkmark$& --& --& --& --& --& --\\
\hline
\multicolumn{1}{|c}{} & \multicolumn{9}{l|}{Outer Approximation Decomposition (B-OA)}\\
\hline
oa\_decomposition & string & no & --& --& $\checkmark$& $\checkmark$& $\checkmark$& --& --\\
\hline
\multicolumn{1}{|c}{} & \multicolumn{9}{l|}{Outer Approximation cuts generation}\\
\hline
add\_only\_violated\_oa & string & no & --& $\checkmark$& $\checkmark$& $\checkmark$& $\checkmark$& $\checkmark$& $\checkmark$\\
oa\_cuts\_scope & string & global & --& $\checkmark$& $\checkmark$& $\checkmark$& $\checkmark$& $\checkmark$& $\checkmark$\\
oa\_rhs\_relax & $\mathbb{Q}$ & $10^{- 8}$ & --& $\checkmark$& $\checkmark$& $\checkmark$& $\checkmark$& $\checkmark$& $\checkmark$\\
tiny\_element & $\mathbb{Q}$ & $10^{- 8}$ & --& $\checkmark$& $\checkmark$& $\checkmark$& $\checkmark$& $\checkmark$& $\checkmark$\\
very\_tiny\_element & $\mathbb{Q}$ & $10^{-17}$ & --& $\checkmark$& $\checkmark$& $\checkmark$& $\checkmark$& $\checkmark$& $\checkmark$\\
\hline
\multicolumn{1}{|c}{} & \multicolumn{9}{l|}{Output and Loglevel}\\
\hline
bb\_log\_interval & $\mathbb{Z}$ & $100$ & $\checkmark$& $\checkmark$& $\checkmark$& $\checkmark$& $\checkmark$& $\checkmark$& $\checkmark$\\
bb\_log\_level & $\mathbb{Z}$ & $1$ & $\checkmark$& $\checkmark$& $\checkmark$& $\checkmark$& $\checkmark$& $\checkmark$& $\checkmark$\\
fp\_log\_frequency & $\mathbb{Q}$ & $100$ & --& --& $\checkmark$& $\checkmark$& --& --& --\\
fp\_log\_level & $\mathbb{Z}$ & $1$ & --& --& $\checkmark$& $\checkmark$& --& --& --\\
lp\_log\_level & $\mathbb{Z}$ & $0$ & --& $\checkmark$& $\checkmark$& $\checkmark$& $\checkmark$& $\checkmark$& $\checkmark$\\
milp\_log\_level & $\mathbb{Z}$ & $0$ & --& --& --& --& --& --& $\checkmark$\\
nlp\_log\_at\_root & $\mathbb{Z}$ & 5 & $\checkmark$& $\checkmark$& $\checkmark$& $\checkmark$& $\checkmark$& $\checkmark$& --\\
nlp\_log\_level & $\mathbb{Z}$ & $1$ & $\checkmark$& $\checkmark$& $\checkmark$& $\checkmark$& $\checkmark$& $\checkmark$& $\checkmark$\\
oa\_cuts\_log\_level & $\mathbb{Z}$ & $0$ & --& $\checkmark$& $\checkmark$& $\checkmark$& $\checkmark$& $\checkmark$& $\checkmark$\\
oa\_log\_frequency & $\mathbb{Q}$ & $100$ & $\checkmark$& --& --& $\checkmark$& $\checkmark$& --& --\\
oa\_log\_level & $\mathbb{Z}$ & $1$ & $\checkmark$& --& --& $\checkmark$& $\checkmark$& --& --\\
print\_funceval\_statistics & string & no & $\checkmark$& $\checkmark$& $\checkmark$& $\checkmark$& $\checkmark$& $\checkmark$& $\checkmark$\\
solvetrace & string &  & $\checkmark$& $\checkmark$& $\checkmark$& $\checkmark$& $\checkmark$& $\checkmark$& $\checkmark$\\
solvetracenodefreq & $\mathbb{Z}$ & $100$ & $\checkmark$& $\checkmark$& $\checkmark$& $\checkmark$& $\checkmark$& $\checkmark$& $\checkmark$\\
solvetracetimefreq & $\mathbb{Q}$ & $5$ & $\checkmark$& $\checkmark$& $\checkmark$& $\checkmark$& $\checkmark$& $\checkmark$& $\checkmark$\\
\hline
\multicolumn{1}{|c}{} & \multicolumn{9}{l|}{Primal Heuristics}\\
\hline
feasibility\_pump\_objective\_norm & $\mathbb{Z}$ & $1$ & $\checkmark$& $\checkmark$& $\checkmark$& $\checkmark$& $\checkmark$& $\checkmark$& --\\
fp\_pass\_infeasible & string & no & $\checkmark$& $\checkmark$& $\checkmark$& $\checkmark$& $\checkmark$& $\checkmark$& $\checkmark$\\
heuristic\_RINS & string & no & $\checkmark$& $\checkmark$& $\checkmark$& $\checkmark$& $\checkmark$& $\checkmark$& --\\
heuristic\_dive\_MIP\_fractional & string & no & $\checkmark$& $\checkmark$& $\checkmark$& $\checkmark$& $\checkmark$& $\checkmark$& --\\
heuristic\_dive\_MIP\_vectorLength & string & no & $\checkmark$& $\checkmark$& $\checkmark$& $\checkmark$& $\checkmark$& $\checkmark$& --\\
heuristic\_dive\_fractional & string & no & $\checkmark$& $\checkmark$& $\checkmark$& $\checkmark$& $\checkmark$& $\checkmark$& --\\
heuristic\_dive\_vectorLength & string & no & $\checkmark$& $\checkmark$& $\checkmark$& $\checkmark$& $\checkmark$& $\checkmark$& --\\
heuristic\_feasibility\_pump & string & no & $\checkmark$& $\checkmark$& $\checkmark$& $\checkmark$& $\checkmark$& $\checkmark$& --\\
pump\_for\_minlp & string & no & $\checkmark$& $\checkmark$& $\checkmark$& $\checkmark$& $\checkmark$& $\checkmark$& --\\
\hline
\multicolumn{1}{|c}{} & \multicolumn{9}{l|}{Strong branching setup}\\
\hline
candidate\_sort\_criterion & string & best-ps-cost & $\checkmark$& $\checkmark$& $\checkmark$& $\checkmark$& $\checkmark$& $\checkmark$& --\\
maxmin\_crit\_have\_sol & $\mathbb{Q}$ & $0.1$ & $\checkmark$& $\checkmark$& $\checkmark$& $\checkmark$& $\checkmark$& $\checkmark$& --\\
maxmin\_crit\_no\_sol & $\mathbb{Q}$ & $0.7$ & $\checkmark$& $\checkmark$& $\checkmark$& $\checkmark$& $\checkmark$& $\checkmark$& --\\
min\_number\_strong\_branch & $\mathbb{Z}$ & $0$ & $\checkmark$& $\checkmark$& $\checkmark$& $\checkmark$& $\checkmark$& $\checkmark$& --\\
number\_before\_trust\_list & $\mathbb{Z}$ & $0$ & $\checkmark$& $\checkmark$& $\checkmark$& $\checkmark$& $\checkmark$& $\checkmark$& --\\
number\_look\_ahead & $\mathbb{Z}$ & $0$ & $\checkmark$& $\checkmark$& $\checkmark$& $\checkmark$& $\checkmark$& --& --\\
number\_strong\_branch\_root & $\mathbb{Z}$ & $\infty$ & $\checkmark$& $\checkmark$& $\checkmark$& $\checkmark$& $\checkmark$& $\checkmark$& --\\
setup\_pseudo\_frac & $\mathbb{Q}$ & $0.5$ & $\checkmark$& $\checkmark$& $\checkmark$& $\checkmark$& $\checkmark$& $\checkmark$& --\\
trust\_strong\_branching\_for\_pseudo\_cost & string & yes & $\checkmark$& $\checkmark$& $\checkmark$& $\checkmark$& $\checkmark$& $\checkmark$& --\\
\hline
\end{xtabular}
}


In the following we give a detailed list of \BONMIN options.
The value on the right denotes the default value.
\printoptioncategory{Algorithm choice}
\printoption{algorithm}%
{\ttfamily B-BB, B-OA, B-QG, B-Hyb, B-Ecp, B-iFP}%
{B-BB}%
{Choice of the algorithm.\\
This will preset some of the options of bonmin depending on the algorithm choice.}%
{\begin{list}{}{
\setlength{\parsep}{0em}
\setlength{\leftmargin}{5ex}
\setlength{\labelwidth}{2ex}
\setlength{\itemindent}{0ex}
\setlength{\topsep}{0pt}}
\item[\texttt{B-BB}] simple branch-and-bound algorithm,
\item[\texttt{B-OA}] OA Decomposition algorithm,
\item[\texttt{B-QG}] Quesada and Grossmann branch-and-cut algorithm,
\item[\texttt{B-Hyb}] hybrid outer approximation based branch-and-cut,
\item[\texttt{B-Ecp}] ecp cuts based branch-and-cut a la FilMINT.
\item[\texttt{B-iFP}] Iterated Feasibility Pump for MINLP.
\end{list}
}

\printoptioncategory{Branch-and-bound options}
\printoption{allowable\_fraction\_gap}%
{$\textrm{real}$}%
{$0.1$}%
{Specify the value of relative gap under which the algorithm stops.\\
Stop the tree search when the gap between the objective value of the best known solution and the best bound on the objective of any solution is less than this fraction of the absolute value of the best known solution value.}%
{}

\printoption{allowable\_gap}%
{$\textrm{real}$}%
{$0$}%
{Specify the value of absolute gap under which the algorithm stops.\\
Stop the tree search when the gap between the objective value of the best known solution and the best bound on the objective of any solution is less than this.}%
{}

\printoption{cutoff}%
{$-10^{ 100}\leq\textrm{real}\leq10^{ 100}$}%
{$10^{ 100}$}%
{Specify cutoff value.\\
cutoff should be the value of a feasible solution known by the user (if any). The algorithm will only look for solutions better than cutoff.}%
{}

\printoption{cutoff\_decr}%
{$-10^{ 10}\leq\textrm{real}\leq10^{ 10}$}%
{$10^{- 5}$}%
{Specify cutoff decrement.\\
Specify the amount by which cutoff is decremented below a new best upper-bound (usually a small positive value but in non-convex problems it may be a negative value).}%
{}

\printoption{enable\_dynamic\_nlp}%
{\ttfamily no, yes}%
{no}%
{Enable dynamic linear and quadratic rows addition in nlp}%
{}

\printoption{integer\_tolerance}%
{$0<\textrm{real}$}%
{$10^{- 6}$}%
{Set integer tolerance.\\
Any number within that value of an integer is considered integer.}%
{}

\printoption{iteration\_limit}%
{$0\leq\textrm{integer}$}%
{$\infty$}%
{Set the cumulated maximum number of iteration in the algorithm used to process nodes continuous relaxations in the branch-and-bound.\\
value 0 deactivates option.}%
{}

\printoption{nlp\_failure\_behavior}%
{\ttfamily stop, fathom}%
{stop}%
{Set the behavior when an NLP or a series of NLP are unsolved by Ipopt (we call unsolved an NLP for which Ipopt is not able to guarantee optimality within the specified tolerances).\\
If set to "fathom", the algorithm will fathom the node when Ipopt fails to find a solution to the nlp at that node whithin the specified tolerances. The algorithm then becomes a heuristic, and the user will be warned that the solution might not be optimal.}%
{\begin{list}{}{
\setlength{\parsep}{0em}
\setlength{\leftmargin}{5ex}
\setlength{\labelwidth}{2ex}
\setlength{\itemindent}{0ex}
\setlength{\topsep}{0pt}}
\item[\texttt{stop}] Stop when failure happens.
\item[\texttt{fathom}] Continue when failure happens.
\end{list}
}

\printoption{node\_comparison}%
{\ttfamily best-bound, depth-first, breadth-first, dynamic, best-guess}%
{best-bound}%
{Choose the node selection strategy.\\
Choose the strategy for selecting the next node to be processed.}%
{\begin{list}{}{
\setlength{\parsep}{0em}
\setlength{\leftmargin}{5ex}
\setlength{\labelwidth}{2ex}
\setlength{\itemindent}{0ex}
\setlength{\topsep}{0pt}}
\item[\texttt{best-bound}] choose node with the smallest bound,
\item[\texttt{depth-first}] Perform depth first search,
\item[\texttt{breadth-first}] Perform breadth first search,
\item[\texttt{dynamic}] Cbc dynamic strategy (starts with a depth first search and turn to best bound after 3 integer feasible solutions have been found).
\item[\texttt{best-guess}] choose node with smallest guessed integer solution
\end{list}
}

\printoption{node\_limit}%
{$0\leq\textrm{integer}$}%
{$\infty$}%
{Set the maximum number of nodes explored in the branch-and-bound search.}%
{}

\printoption{num\_cut\_passes}%
{$0\leq\textrm{integer}$}%
{$1$}%
{Set the maximum number of cut passes at regular nodes of the branch-and-cut.}%
{}

\printoption{num\_cut\_passes\_at\_root}%
{$0\leq\textrm{integer}$}%
{$20$}%
{Set the maximum number of cut passes at regular nodes of the branch-and-cut.}%
{}

\printoption{number\_before\_trust}%
{$0\leq\textrm{integer}$}%
{$8$}%
{Set the number of branches on a variable before its pseudo costs are to be believed in dynamic strong branching.\\
A value of 0 disables pseudo costs.}%
{}

\printoption{number\_strong\_branch}%
{$0\leq\textrm{integer}$}%
{$20$}%
{Choose the maximum number of variables considered for strong branching.\\
Set the number of variables on which to do strong branching.}%
{}

\printoption{solution\_limit}%
{$0\leq\textrm{integer}$}%
{$\infty$}%
{Abort after that much integer feasible solution have been found by algorithm\\
value 0 deactivates option}%
{}

\printoption{time\_limit}%
{$0\leq\textrm{real}$}%
{$1000$}%
{Set the global maximum computation time (in secs) for the algorithm.}%
{}

\printoption{tree\_search\_strategy}%
{\ttfamily top-node, dive, probed-dive, dfs-dive, dfs-dive-dynamic}%
{probed-dive}%
{Pick a strategy for traversing the tree\\
All strategies can be used in conjunction with any of the node comparison functions. Options which affect dfs-dive are max-backtracks-in-dive and max-dive-depth. The dfs-dive won't work in a non-convex problem where objective does not decrease down branches.}%
{\begin{list}{}{
\setlength{\parsep}{0em}
\setlength{\leftmargin}{5ex}
\setlength{\labelwidth}{2ex}
\setlength{\itemindent}{0ex}
\setlength{\topsep}{0pt}}
\item[\texttt{top-node}]  Always pick the top node as sorted by the node comparison function
\item[\texttt{dive}] Dive in the tree if possible, otherwise pick top node as sorted by the tree comparison function.
\item[\texttt{probed-dive}] Dive in the tree exploring two childs before continuing the dive at each level.
\item[\texttt{dfs-dive}] Dive in the tree if possible doing a depth first search. Backtrack on leaves or when a prescribed depth is attained or when estimate of best possible integer feasible solution in subtree is worst than cutoff. Once a prescribed limit of backtracks is attained pick top node as sorted by the tree comparison function
\item[\texttt{dfs-dive-dynamic}] Same as dfs-dive but once enough solution are found switch to best-bound and if too many nodes switch to depth-first.
\end{list}
}

\printoption{variable\_selection}%
{\ttfamily most-fractional, strong-branching, reliability-branching, curvature-estimator, qp-strong-branching, lp-strong-branching, nlp-strong-branching, osi-simple, osi-strong, random}%
{strong-branching}%
{Chooses variable selection strategy}%
{\begin{list}{}{
\setlength{\parsep}{0em}
\setlength{\leftmargin}{5ex}
\setlength{\labelwidth}{2ex}
\setlength{\itemindent}{0ex}
\setlength{\topsep}{0pt}}
\item[\texttt{most-fractional}] Choose most fractional variable
\item[\texttt{strong-branching}] Perform strong branching
\item[\texttt{reliability-branching}] Use reliability branching
\item[\texttt{curvature-estimator}] Use curvature estimation to select branching variable
\item[\texttt{qp-strong-branching}] Perform strong branching with QP approximation
\item[\texttt{lp-strong-branching}] Perform strong branching with LP approximation
\item[\texttt{nlp-strong-branching}] Perform strong branching with NLP approximation
\item[\texttt{osi-simple}] Osi method to do simple branching
\item[\texttt{osi-strong}] Osi method to do strong branching
\item[\texttt{random}] Method to choose branching variable randomly
\end{list}
}

\printoptioncategory{ECP cuts generation}
\printoption{ecp\_abs\_tol}%
{$0\leq\textrm{real}$}%
{$10^{- 6}$}%
{Set the absolute termination tolerance for ECP rounds.}%
{}

\printoption{ecp\_max\_rounds}%
{$0\leq\textrm{integer}$}%
{$5$}%
{Set the maximal number of rounds of ECP cuts.}%
{}

\printoption{ecp\_probability\_factor}%
{$\textrm{real}$}%
{$10$}%
{Factor appearing in formula for skipping ECP cuts.\\
Choosing -1 disables the skipping.}%
{}

\printoption{ecp\_rel\_tol}%
{$0\leq\textrm{real}$}%
{$0$}%
{Set the relative termination tolerance for ECP rounds.}%
{}

\printoption{filmint\_ecp\_cuts}%
{$0\leq\textrm{integer}$}%
{$0$}%
{Specify the frequency (in terms of nodes) at which some a la filmint ecp cuts are generated.\\
A frequency of 0 amounts to to never solve the NLP relaxation.}%
{}

\printoptioncategory{Feasibility checker using OA cuts}
\printoption{feas\_check\_cut\_types}%
{\ttfamily outer-approx, Benders}%
{outer-approx}%
{Choose the type of cuts generated when an integer feasible solution is found\\
If it seems too much memory is used should try Benders to use less}%
{\begin{list}{}{
\setlength{\parsep}{0em}
\setlength{\leftmargin}{5ex}
\setlength{\labelwidth}{2ex}
\setlength{\itemindent}{0ex}
\setlength{\topsep}{0pt}}
\item[\texttt{outer-approx}] Generate a set of Outer Approximations cuts.
\item[\texttt{Benders}] Generate a single Benders cut.
\end{list}
}

\printoption{feas\_check\_discard\_policy}%
{\ttfamily detect-cycles, keep-all, treated-as-normal}%
{detect-cycles}%
{How cuts from feasibility checker are discarded\\
Normally to avoid cycle cuts from feasibility checker should not be discarded in the node where they are generated. However Cbc sometimes does it if no care is taken which can lead to an infinite loop in Bonmin (usualy on simple problems). To avoid this one can instruct Cbc to never discard a cut but if we do that for all cuts it can lead to memory problems. The default policy here is to detect cycles and only then impose to Cbc to keep the cut. The two other alternative are to instruct Cbc to keep all cuts or to just ignore the problem and hope for the best}%
{\begin{list}{}{
\setlength{\parsep}{0em}
\setlength{\leftmargin}{5ex}
\setlength{\labelwidth}{2ex}
\setlength{\itemindent}{0ex}
\setlength{\topsep}{0pt}}
\item[\texttt{detect-cycles}] Detect if a cycle occurs and only in this case force not to discard.
\item[\texttt{keep-all}] Force cuts from feasibility checker not to be discarded (memory hungry but sometimes better).
\item[\texttt{treated-as-normal}] Cuts from memory checker can be discarded as any other cuts (code may cycle then)
\end{list}
}

\printoption{generate\_benders\_after\_so\_many\_oa}%
{$0\leq\textrm{integer}$}%
{$5000$}%
{Specify that after so many oa cuts have been generated Benders cuts should be generated instead.\\
It seems that sometimes generating too many oa cuts slows down the optimization compared to Benders due to the size of the LP. With this option we specify that after so many OA cuts have been generated we should switch to Benders cuts.}%
{}

\printoptioncategory{MILP Solver}
\printoption{cpx\_parallel\_strategy}%
{$-1\leq\textrm{integer}\leq1$}%
{$0$}%
{Strategy of parallel search mode in CPLEX.\\
-1 = opportunistic, 0 = automatic, 1 = deterministic (refer to CPLEX documentation)}%
{}

\printoption{milp\_solver}%
{Cbc\_D, Cbc\_Par, Cplex}%
{Cbc\_D}%
{Choose the subsolver to solve MILP sub-problems in OA decompositions.\\
 To use Cplex, a valid license is required.}%
{\begin{list}{}{
\setlength{\parsep}{0em}
\setlength{\leftmargin}{5ex}
\setlength{\labelwidth}{2ex}
\setlength{\itemindent}{0ex}
\setlength{\topsep}{0pt}}
\item[\texttt{Cbc\_D}] Coin Branch and Cut with its default
\item[\texttt{Cbc\_Par}] Coin Branch and Cut with passed parameters
\item[\texttt{Cplex}] IBM CPLEX
\end{list}
}

\printoption{milp\_strategy}%
{\ttfamily find\_good\_sol, solve\_to\_optimality}%
{find\_good\_sol}%
{Choose a strategy for MILPs.}%
{\begin{list}{}{
\setlength{\parsep}{0em}
\setlength{\leftmargin}{5ex}
\setlength{\labelwidth}{2ex}
\setlength{\itemindent}{0ex}
\setlength{\topsep}{0pt}}
\item[\texttt{find\_good\_sol}] Stop sub milps when a solution improving the incumbent is found
\item[\texttt{solve\_to\_optimality}] Solve MILPs to optimality
\end{list}
}

\printoption{number\_cpx\_threads}%
{$0\leq\textrm{integer}$}%
{$0$}%
{Set number of threads to use with cplex.\\
(refer to CPLEX documentation)}%
{}

\printoptioncategory{MILP cutting planes in hybrid algorithm (B-Hyb)}
\printoption{2mir\_cuts}%
{$-100\leq\textrm{integer}$}%
{$0$}%
{Frequency (in terms of nodes) for generating 2-MIR cuts in branch-and-cut\\
If k $>$ 0, cuts are generated every k nodes, if -99 $<$ k $<$ 0 cuts are generated every -k nodes but Cbc may decide to stop generating cuts, if not enough are generated at the root node, if k=-99 generate cuts only at the root node, if k=0 or 100 do not generate cuts.}%
{}

\printoption{Gomory\_cuts}%
{$-100\leq\textrm{integer}$}%
{$-5$}%
{Frequency k (in terms of nodes) for generating Gomory cuts in branch-and-cut.\\
See option \texttt{2mir\_cuts} for the meaning of k.}%
{}

\printoption{clique\_cuts}%
{$-100\leq\textrm{integer}$}%
{$-5$}%
{Frequency (in terms of nodes) for generating clique cuts in branch-and-cut\\
See option \texttt{2mir\_cuts} for the meaning of k.}%
{}

\printoption{cover\_cuts}%
{$-100\leq\textrm{integer}$}%
{$0$}%
{Frequency (in terms of nodes) for generating cover cuts in branch-and-cut\\
See option \texttt{2mir\_cuts} for the meaning of k.}%
{}

\printoption{flow\_cover\_cuts}%
{$-100\leq\textrm{integer}$}%
{$-5$}%
{Frequency (in terms of nodes) for generating flow cover cuts in branch-and-cut\\
See option \texttt{2mir\_cuts} for the meaning of k.}%
{}

\printoption{lift\_and\_project\_cuts}%
{$-100\leq\textrm{integer}$}%
{$0$}%
{Frequency (in terms of nodes) for generating lift-and-project cuts in branch-and-cut\\
See option \texttt{2mir\_cuts} for the meaning of k.}%
{}

\printoption{mir\_cuts}%
{$-100\leq\textrm{integer}$}%
{$-5$}%
{Frequency (in terms of nodes) for generating MIR cuts in branch-and-cut\\
See option \texttt{2mir\_cuts} for the meaning of k.}%
{}

\printoption{reduce\_and\_split\_cuts}%
{$-100\leq\textrm{integer}$}%
{$0$}%
{Frequency (in terms of nodes) for generating reduce-and-split cuts in branch-and-cut\\
See option \texttt{2mir\_cuts} for the meaning of k.}%
{}

\printoptioncategory{MINLP Heuristics}
\printoption{feasibility\_pump\_objective\_norm}%
{$1\leq\textrm{integer}\leq2$}%
{$1$}%
{Norm of feasibility pump objective function}%
{}

\printoption{fp\_pass\_infeasible}%
{\ttfamily no, yes}%
{no}%
{Say whether feasibility pump should claim to converge or not}%
{\begin{list}{}{
\setlength{\parsep}{0em}
\setlength{\leftmargin}{5ex}
\setlength{\labelwidth}{2ex}
\setlength{\itemindent}{0ex}
\setlength{\topsep}{0pt}}
\item[\texttt{no}] When master MILP is infeasible just bail out (don't stop all algorithm). This is the option for using in B-Hyb.
\item[\texttt{yes}] Claim convergence, numerically dangerous.
\end{list}
}

\printoption{heuristic\_RINS}%
{\ttfamily no, yes}%
{no}%
{if yes runs the RINS heuristic}%
{
}

\printoption{heuristic\_dive\_MIP\_fractional}%
{\ttfamily no, yes}%
{no}%
{if yes runs the Dive MIP Fractional heuristic}%
{
}

\printoption{heuristic\_dive\_MIP\_vectorLength}%
{\ttfamily no, yes}%
{no}%
{if yes runs the Dive MIP VectorLength heuristic}%
{
}

\printoption{heuristic\_dive\_fractional}%
{\ttfamily no, yes}%
{no}%
{if yes runs the Dive Fractional heuristic}%
{
}

\printoption{heuristic\_dive\_vectorLength}%
{\ttfamily no, yes}%
{no}%
{if yes runs the Dive VectorLength heuristic}%
{
}

\printoption{heuristic\_feasibility\_pump}%
{\ttfamily no, yes}%
{no}%
{whether the heuristic feasibility pump should be used}%
{
}

\printoption{pump\_for\_minlp}%
{\ttfamily no, yes}%
{no}%
{if yes runs FP for MINLP}%
{
}

\printoptioncategory{NLP interface}
\printoption{warm\_start}%
{\ttfamily none, optimum, interior\_point}%
{none}%
{Select the warm start method\\
This will affect the function getWarmStart(), and as a consequence the warm starting in the various algorithms.}%
{\begin{list}{}{
\setlength{\parsep}{0em}
\setlength{\leftmargin}{5ex}
\setlength{\labelwidth}{2ex}
\setlength{\itemindent}{0ex}
\setlength{\topsep}{0pt}}
\item[\texttt{none}] No warm start
\item[\texttt{optimum}] Warm start with direct parent optimum
\item[\texttt{interior\_point}] Warm start with an interior point of direct parent
\end{list}
}

\printoptioncategory{NLP solution robustness}
\printoption{max\_consecutive\_failures}%
{$0\leq\textrm{integer}$}%
{$10$}%
{(temporarily removed) Number $n$ of consecutive unsolved problems before aborting a branch of the tree.\\
When $n > 0$, continue exploring a branch of the tree until $n$ consecutive problems in the branch are unsolved (we call unsolved a problem for which Ipopt can not guarantee optimality within the specified tolerances).}%
{}

\printoption{max\_random\_point\_radius}%
{$0<\textrm{real}$}%
{$100000$}%
{Set max value r for coordinate of a random point.\\
When picking a random point, coordinate i will be in the interval [min(max(l,-r),u-r), max(min(u,r),l+r)] (where l is the lower bound for the variable and u is its upper bound)}%
{}

\printoption{num\_iterations\_suspect}%
{$-1\leq\textrm{integer}$}%
{$-1$}%
{Number of iterations over which a node is considered "suspect" (for debugging purposes only, see detailed documentation).\\
When the number of iterations to solve a node is above this number, the subproblem at this node is considered to be suspect and it will be outputed in a file (set to -1 to deactivate this).}%
{}

\printoption{num\_retry\_unsolved\_random\_point}%
{$0\leq\textrm{integer}$}%
{$0$}%
{Number $k$ of times that the algorithm will try to resolve an unsolved NLP with a random starting point (we call unsolved an NLP for which Ipopt is not able to guarantee optimality within the specified tolerances).\\
When Ipopt fails to solve a continuous NLP sub-problem, if $k > 0$, the algorithm will try again to solve the failed NLP with $k$ new randomly chosen starting points  or until the problem is solved with success.}%
{}

\printoption{random\_point\_perturbation\_interval}%
{$0<\textrm{real}$}%
{$1$}%
{Amount by which starting point is perturbed when choosing to pick random point by perturbating starting point}%
{}

\printoption{random\_point\_type}%
{\ttfamily Jon, Andreas, Claudia}%
{Jon}%
{method to choose a random starting point}%
{\begin{list}{}{
\setlength{\parsep}{0em}
\setlength{\leftmargin}{5ex}
\setlength{\labelwidth}{2ex}
\setlength{\itemindent}{0ex}
\setlength{\topsep}{0pt}}
\item[\texttt{Jon}] Choose random point uniformly between the bounds
\item[\texttt{Andreas}] perturb the starting point of the problem within a prescribed interval
\item[\texttt{Claudia}] perturb the starting point using the perturbation radius suffix information
\end{list}
}

\printoptioncategory{NLP solves in hybrid algorithm (B-Hyb)}
\printoption{nlp\_solve\_frequency}%
{$0\leq\textrm{integer}$}%
{$10$}%
{Specify the frequency (in terms of nodes) at which NLP relaxations are solved in B-Hyb.\\
A frequency of 0 amounts to to never solve the NLP relaxation.}%
{}

\printoption{nlp\_solve\_max\_depth}%
{$0\leq\textrm{integer}$}%
{$10$}%
{Set maximum depth in the tree at which NLP relaxations are solved in B-Hyb.\\
A depth of 0 amounts to to never solve the NLP relaxation.}%
{}

\printoption{nlp\_solves\_per\_depth}%
{$0\leq\textrm{real}$}%
{$10^{ 100}$}%
{Set average number of nodes in the tree at which NLP relaxations are solved in B-Hyb for each depth.}%
{}

\printoptioncategory{Nonconvex problems}
\printoption{coeff\_var\_threshold}%
{$0\leq\textrm{real}$}%
{$0.1$}%
{Coefficient of variation threshold (for dynamic definition of cutoff\_decr).}%
{}

\printoption{dynamic\_def\_cutoff\_decr}%
{\ttfamily no, yes}%
{no}%
{Do you want to define the parameter cutoff\_decr dynamically?}%
{
}

\printoption{first\_perc\_for\_cutoff\_decr}%
{$\textrm{real}$}%
{$-0.02$}%
{The percentage used when, the coeff of variance is smaller than the threshold, to compute the cutoff\_decr dynamically.}%
{}

\printoption{max\_consecutive\_infeasible}%
{$0\leq\textrm{integer}$}%
{$0$}%
{Number of consecutive infeasible subproblems before aborting a branch.\\
Will continue exploring a branch of the tree until "max\_consecutive\_infeasible"consecutive problems are infeasibles by the NLP sub-solver.}%
{}

\printoption{num\_resolve\_at\_infeasibles}%
{$0\leq\textrm{integer}$}%
{$0$}%
{Number $k$ of tries to resolve an infeasible node (other than the root) of the tree with different starting point.\\
The algorithm will solve all the infeasible nodes with $k$ different random starting points and will keep the best local optimum found.}%
{}

\printoption{num\_resolve\_at\_node}%
{$0\leq\textrm{integer}$}%
{$0$}%
{Number $k$ of tries to resolve a node (other than the root) of the tree with different starting point.\\
The algorithm will solve all the nodes with $k$ different random starting points and will keep the best local optimum found.}%
{}

\printoption{num\_resolve\_at\_root}%
{$0\leq\textrm{integer}$}%
{$0$}%
{Number $k$ of tries to resolve the root node with different starting points.\\
The algorithm will solve the root node with $k$ random starting points and will keep the best local optimum found.}%
{}

\printoption{second\_perc\_for\_cutoff\_decr}%
{$\textrm{real}$}%
{$-0.05$}%
{The percentage used when, the coeff of variance is greater than the threshold, to compute the cutoff\_decr dynamically.}%
{}

\printoptioncategory{Outer Approximation Decomposition (B-OA)}
\printoption{oa\_decomposition}%
{\ttfamily no, yes}%
{no}%
{If yes do initial OA decomposition}%
{}

\printoptioncategory{Outer Approximation cuts generation}
\printoption{add\_only\_violated\_oa}%
{\ttfamily no, yes}%
{no}%
{Do we add all OA cuts or only the ones violated by current point?}%
{\begin{list}{}{
\setlength{\parsep}{0em}
\setlength{\leftmargin}{5ex}
\setlength{\labelwidth}{2ex}
\setlength{\itemindent}{0ex}
\setlength{\topsep}{0pt}}
\item[\texttt{no}] Add all cuts
\item[\texttt{yes}] Add only violated Cuts
\end{list}
}

\printoption{oa\_cuts\_scope}%
{\ttfamily local, global}%
{global}%
{Specify if OA cuts added are to be set globally or locally valid}%
{\begin{list}{}{
\setlength{\parsep}{0em}
\setlength{\leftmargin}{5ex}
\setlength{\labelwidth}{2ex}
\setlength{\itemindent}{0ex}
\setlength{\topsep}{0pt}}
\item[\texttt{local}] Cuts are treated as locally valid
\item[\texttt{global}] Cuts are treated as globally valid
\end{list}
}

\printoption{tiny\_element}%
{$-0\leq\textrm{real}$}%
{$10^{- 8}$}%
{Value for tiny element in OA cut\\
We will remove "cleanly" (by relaxing cut) an element lower than this.}%
{}

\printoption{very\_tiny\_element}%
{$-0\leq\textrm{real}$}%
{$10^{-17}$}%
{Value for very tiny element in OA cut\\
Algorithm will take the risk of neglecting an element lower than this.}%
{}

\printoptioncategory{Output}
\printoption{bb\_log\_interval}%
{$0\leq\textrm{integer}$}%
{$100$}%
{Interval at which node level output is printed.\\
Set the interval (in terms of number of nodes) at which a log on node resolutions (consisting of lower and upper bounds) is given.}%
{}

\printoption{bb\_log\_level}%
{$0\leq\textrm{integer}\leq5$}%
{$1$}%
{specify main branch-and-bound log level.\\
Set the level of output of the branch-and-bound : 0 - none, 1 - minimal, 2 - normal low, 3 - normal high}%
{}

\printoption{fp\_log\_frequency}%
{$0<\textrm{real}$}%
{$100$}%
{display an update on lower and upper bounds in FP every n seconds}%
{}

\printoption{fp\_log\_level}%
{$0\leq\textrm{integer}\leq2$}%
{$1$}%
{specify FP iterations log level.\\
Set the level of output of OA decomposition solver : 0 - none, 1 - normal, 2 - verbose}%
{}

\printoption{lp\_log\_level}%
{$0\leq\textrm{integer}\leq4$}%
{$0$}%
{specify LP log level.\\
Set the level of output of the linear programming sub-solver in B-Hyb or B-QG : 0 - none, 1 - minimal, 2 - normal low, 3 - normal high, 4 - verbose}%
{}

\printoption{milp\_log\_level}%
{$0\leq\textrm{integer}\leq4$}%
{$0$}%
{specify MILP solver log level.\\
Set the level of output of the MILP subsolver in OA : 0 - none, 1 - minimal, 2 - normal low, 3 - normal high}%
{}

\printoption{nlp\_log\_at\_root}%
{$0\leq\textrm{integer}\leq12$}%
{$5$}%
{ Specify a different log level for root relaxtion.}%
{}

\printoption{nlp\_log\_level}%
{$0\leq\textrm{integer}\leq2$}%
{$1$}%
{specify NLP solver interface log level (independent from ipopt print\_level).\\
Set the level of output of the OsiTMINLPInterface : 0 - none, 1 - normal, 2 - verbose}%
{}

\printoption{oa\_cuts\_log\_level}%
{$0\leq\textrm{integer}$}%
{$0$}%
{level of log when generating OA cuts.\\
0: outputs nothing,\\1: when a cut is generated, its violation and index of row from which it originates,\\2: always output violation of the cut.\\3: output generated cuts incidence vectors.}%
{}

\printoption{oa\_log\_frequency}%
{$0<\textrm{real}$}%
{$100$}%
{display an update on lower and upper bounds in OA every n seconds}%
{}

\printoption{oa\_log\_level}%
{$0\leq\textrm{integer}\leq2$}%
{$1$}%
{specify OA iterations log level.\\
Set the level of output of OA decomposition solver : 0 - none, 1 - normal, 2 - verbose}%
{}

\printoptioncategory{Strong branching setup}
\printoption{candidate\_sort\_criterion}%
{\ttfamily best-ps-cost, worst-ps-cost, most-fractional, least-fractional}%
{best-ps-cost}%
{Choice of the criterion to choose candidates in strong-branching}%
{\begin{list}{}{
\setlength{\parsep}{0em}
\setlength{\leftmargin}{5ex}
\setlength{\labelwidth}{2ex}
\setlength{\itemindent}{0ex}
\setlength{\topsep}{0pt}}
\item[\texttt{best-ps-cost}] Sort by decreasing pseudo-cost
\item[\texttt{worst-ps-cost}] Sort by increasing pseudo-cost
\item[\texttt{most-fractional}] Sort by decreasing integer infeasibility
\item[\texttt{least-fractional}] Sort by increasing integer infeasibility
\end{list}
}

\printoption{maxmin\_crit\_have\_sol}%
{$0\leq\textrm{real}\leq1$}%
{$0.1$}%
{Weight towards minimum in of lower and upper branching estimates when a solution has been found.}%
{}

\printoption{maxmin\_crit\_no\_sol}%
{$0\leq\textrm{real}\leq1$}%
{$0.7$}%
{Weight towards minimum in of lower and upper branching estimates when no solution has been found yet.}%
{}

\printoption{min\_number\_strong\_branch}%
{$0\leq\textrm{integer}$}%
{$0$}%
{Sets minimum number of variables for strong branching (overriding trust)}%
{}

\printoption{number\_before\_trust\_list}%
{$-1\leq\textrm{integer}$}%
{$0$}%
{Set the number of branches on a variable before its pseudo costs are to be believed during setup of strong branching candidate list.\\
The default value is that of "number\_before\_trust"}%
{}

\printoption{number\_look\_ahead}%
{$0\leq\textrm{integer}$}%
{$0$}%
{Sets limit of look-ahead strong-branching trials}%
{}

\printoption{number\_strong\_branch\_root}%
{$0\leq\textrm{integer}$}%
{$\infty$}%
{Maximum number of variables considered for strong branching in root node.}%
{}

\printoption{setup\_pseudo\_frac}%
{$0\leq\textrm{real}\leq1$}%
{$0.5$}%
{Proportion of strong branching list that has to be taken from most-integer-infeasible list.}%
{}

\printoption{trust\_strong\_branching\_for\_pseudo\_cost}%
{\ttfamily no, yes}%
{yes}%
{Whether or not to trust strong branching results for updating pseudo costs.}%
{}



\bibliographystyle{plain}
%\bibliography{coinlibd}
%\renewcommand{\bibname}{BONMIN References}
\chapter{\BONMIN and \BONMINH}

%\minitoc

COIN-OR \BONMIN (\textbf{B}asic \textbf{O}pen-source \textbf{N}onlinear \textbf{M}ixed \textbf{In}teger programming) is an open-source solver for mixed-integer nonlinear programming (MINLPs).
The code has been developed as part of a collaboration between Carnegie Mellon University and IBM Research.
The COIN-OR project leader for \BONMIN is Pierre Bonami.

\BONMIN can handle mixed-integer nonlinear programming models which functions should be twice continuously differentiable.
The \BONMIN link in \GAMS supports continuous, binary, and integer variables, special ordered sets, branching priorities, but no semi-continuous or semi-integer variables (see chapter 17.1 of the \GAMS User's Guide).


\BONMIN implements six different algorithms for solving MINLPs:
\begin{itemize}
\setlength{\partopsep}{0pt}
\setlength{\itemsep}{0pt}
\item {B-BB} (\textbf{default}): a simple branch-and-bound algorithm based on solving a continuous nonlinear program at each node of the search tree and branching on integer variables~\cite{GuptaRavindran85}; this algorithm is similar to the one implemented in the solver \textsc{SBB}
\item {B-OA}: an outer-approximation based decomposition algorithm based on iterating solving and improving of a MIP relaxation and solving NLP subproblems~\cite{DuGr86,FlLe94}; this algorithm is similar to the one implemented in the solver \textsc{DICOPT}
\item {B-QG}: an outer-approximation based branch-and-cut algorithm based on solving a continuous linear program at each node of the search tree, improving the linear program by outer approximation, and branching on integer variables~\cite{QeGr92}.
\item {B-Hyb}: a branch-and-bound algorithm which is a hybrid of B-BB and B-QG and is based on solving either a continuous nonlinear or a continuous linear program at each node of the search tree, improving the linear program by outer approximation, and branching on integer variables~\cite{BBCCGLLLMSW}
\item {B-ECP}: a Kelley's outer-approximation based branch-and-cut algorithm inspired by the settings used in the solver \textsc{FilMINT}~\cite{AbLeLi07}
\item {B-iFP}: an iterated feasibility pump algorithm~\cite{BoCoLoMa06}
\end{itemize}
The algorithms are exact when the problem is \textbf{convex}, otherwise they are heuristics.

For convex MINLPs, experiments on a reasonably large test set of problems have shown that B-Hyb is the algorithm of choice (it solved most of the problems in 3 hours of computing time).
Nevertheless, there are cases where B-OA (especially when used with CPLEX as MIP subproblem solver) is much faster than B-Hyb and others where B-BB is interesting.
B-QG and B-ECP corresponds mainly to a specific parameter setting of B-Hyb but they can be faster in some cases.
B-iFP is more tailored at finding quickly good solutions to very hard convex MINLP.
For \textbf{nonconvex} MINLPs, it is strongly recommended to use B-BB (the outer-approximation algorithms have not been tailored to treat nonconvex problems at this point).
Although even B-BB is only a heuristic for such problems, several options are available to try and improve the quality of the solutions it provides (see below).

NLPs are solved in \BONMIN by \IPOPT, which can use \textsc{MUMPS}~\cite{bonminAmestoyDuffKosterLExcellent2001,bonminAmestoyGuermoucheLExcellentPralet2006} (currently the default) or \textsc{MKL PARDISO}~\cite{bonminSchGa04,bonminSchGa06} (only Linux and Windows) as linear solver.
In the commerically licensed \GAMS/\BONMINH version, also the linear solvers \textsc{MA27}, \textsc{MA57}, \textsc{HSL\_MA86}, and \textsc{HSL\_MA97} from the Harwell Subroutines Library (HSL) are available in \IPOPT.
In this case, the default linear solver in \IPOPT is MA27.


For more information we refer to \cite{BoCoLoMa06,BoGo08,BoKiLi09,BBCCGLLLMSW} and the \BONMIN web site \texttt{https://projects.coin-or.org/Bonmin}.
Most of the \BONMIN documentation in this section is taken from the \BONMIN manual~\cite{BonminManual}.


%If \GAMS/\BONMIN is called for a model with only continuous variables, the interface switches over to \IPOPT.
%If \GAMS/\BONMIN is called for a model with only linear equations, the interface switches over to \CBC.

\section{Usage}

The following statement can be used inside your \GAMS program to specify using \BONMIN:
\begin{verbatim}
  Option MINLP = BONMIN;    { or Option MIQCP = BONMIN; }
\end{verbatim}
This statement should appear before the \texttt{Solve} statement.
If \BONMIN was specified as the default solver during \GAMS installation, the above statement is not necessary.

To use \BONMINH, one should use the statement
\begin{verbatim}
  Option MINLP = BONMINH;   { or Option MIQCP = BONMINH; }
\end{verbatim}

\GAMS/\BONMIN currently does not support the \GAMS Branch-and-Cut-and-Heuristic (BCH) Facility.
If you need to use \GAMS/\BONMIN with BCH, please consider to use a \GAMS system of version $\leq 23.3$, available at \url{http://www.gams.com/download/download_old.htm}.
% \GAMS/\BONMIN supports the \GAMS Branch-and-Cut-and-Heuristic (BCH) Facility.
% The \GAMS BCH facility automates all major steps necessary to define, execute, and control the use of user defined routines within the framework of general purpose MIP and MINLP codes.
% Currently supported are user defined cut generators and heuristics, where cut generator cannot be used in Bonmins pure B\&B algorithm (B-BB).
% Please see the BCH documentation at \texttt{http://www.gams.com/docs/bch.htm} for further information.

\subsection{Specification of Options}

A \BONMIN options file contains both \IPOPT and \BONMIN options, for clarity all \BONMIN options should be preceded with the prefix ``\texttt{bonmin.}''. %, except those corresponding to the BCH facility.
The scheme to name option files is the same as for all other \GAMS solvers.
Specifying \texttt{optfile=1} let \GAMS/\BONMIN read \texttt{bonmin.opt}, \texttt{optfile=2} corresponds to \texttt{bonmin.op2}, and so on.
The format of the option file is the same as for \IPOPT (see Section \ref{sub:ipoptoptionspec} in Chapter \ref{cha:ipopt}).

The most important option in \BONMIN is the choice of the solution algorithm.
This can be set by using the option named \texttt{bonmin.algorithm} which can be set to \texttt{B-BB}, \texttt{B-OA}, \texttt{B-QG}, \texttt{B-Hyb}, \texttt{B-ECP}, or \texttt{B-iFP} (its default value is \texttt{B-BB}).
Depending on the value of this option, certain other options may be available or not, cf.\ Section~\ref{sub:bonminalloptions}.

An example of a \texttt{bonmin.opt} file is the following:
\begin{verbatim}
   bonmin.algorithm       B-Hyb
   bonmin.oa_log_level    4
   print_level            6
\end{verbatim}
%    bonmin.milp_subsolver  Cbc_Par
%    milp_sub.cover_cuts    0
%    userheurcall           "bchheur.gms reslim 10"
This sets the algorithm to be used to the hybrid algorithm, the level of outer approximation related output to $4$,
% the MIP subsolver for outer approximation to a parameterized version of CBC, switches off cover cutting planes for the MIP subsolver,
and sets the print level for \IPOPT to $6$.
%  and let \BONMIN call a user defined heuristic specified in the model \texttt{bchheur.gms} with a timelimit of 10 seconds.

\GAMS/\BONMIN understands currently the following \GAMS parameters: \texttt{reslim} (time limit), \texttt{iterlim} (iteration limit), \texttt{nodlim} (node limit), \texttt{cutoff}, \texttt{optca} (absolute gap tolerance), and \texttt{optcr} (relative gap tolerance).
One can set them either on the command line, e.g. \verb+nodlim=1000+, or inside your \GAMS program, e.g. \verb+Option nodlim=1000;+.
Further, the option \texttt{threads} can be used to control the number of threads used in the linear algebra routines of \IPOPT, see Section~\ref{sec:ipoptlinearsolver} in Chapter~\ref{cha:ipopt} for details.

\subsection{Passing options to local search based heuristics and OA generators}
Several parts of the algorithms in \BONMIN are based on solving a simplified version of the problem with another instance of \BONMIN:
Outer Approximation Decomposition (called in {\tt B-Hyb} at the root node)
and Feasibility Pump for MINLP (called in {\tt B-Hyb} or {\tt B-BB} at the root node), RINS, RENS, Local Branching.

In all these cases, one can pass options to the sub-algorithm used through the option file.
The basic principle is that the ``\texttt{bonmin.}'' prefix  is replaced with a prefix that identifies the sub-algorithm used:
\begin{itemize}
\vspace{-2ex}
\setlength{\parskip}{.2ex}
\setlength{\itemsep}{0pt}
\setlength{\partopsep}{0pt}
\item to pass options to Outer Approximation Decomposition: {\tt oa\_decomposition.},
\item to pass options to Feasibility Pump for MINLP: {\tt pump\_for\_minlp.},
\item to pass options to RINS: {\tt rins.},
\item to pass options to RENS: {\tt rens.},
\item to pass options to Local Branching: {\tt local\_branch}.
\end{itemize}

\vspace{-2ex}
For example, to run a maximum of 60 seconds of feasibility pump (FP) for MINLP until 6 solutions are found at the beginning of the hybrid algorithm, one sets the following options:
\begin{verbatim}
bonmin.algorithm              B-Hyb
bonmin.pump_for_minlp         yes   # tells to run FP for MINLP
pump_for_minlp.time_limit     60    # set a time limit for the pump
pump_for_minlp.solution_limit 6     # set a solution limit
\end{verbatim}
Note that the actual solution and time limit will be the minimum of the global limits set for \BONMIN.

A slightly more complicated set of options may be used when using RINS.
Say for example that one wants to run RINS inside \texttt{B-BB}.
Each time RINS is called one wants to solve the small-size MINLP generated using B-QG (one may run any algorithm available in \BONMIN for solving an MINLP) and wants to stop as soon as \texttt{B-QG} found one solution.
To achieve this, one sets the following options
\begin{verbatim}
bonmin.algorithm      B-BB
bonmin.heuristic_rins yes
rins.algorithm        B-QG
rins.solution_limit   1
\end{verbatim}
This example shows that it is possible to set any option used in the sub-algorithm to be different than the one used for the main algorithm.

In the context of outer-approximation (OA) and feasibility pump for MINLP, a standard MILP solver is used.
Several options are available for configuring this MILP solver.
\BONMIN allows a choice of different MILP solvers through the option
\texttt{bonmin.milp\_sol\-ver}. Values for this option are: {\tt Cbc\_D} which uses \CBC with its
default settings, {\tt Cbc\_Par} which uses a version of \CBC that can be parameterized by the user, and \texttt{Cplex} which uses \CPLEX with its default settings.
The options that can be set in {\tt Cbc\_Par} are the number of strong-branching candidates,
the number of branches before pseudo costs are to be trusted, and the frequency of the various cut generators, c.f.\ Section~\ref{sub:bonminalloptions} for details.
To use the \texttt{Cplex} option, a valid \CPLEX licence (standalone or \GAMS/\CPLEX) is required.

\subsection{Getting good solutions to nonconvex problems}
To solve a problem with nonconvex constraints, one should only use the branch-and-bound algorithm {\tt B-BB}.

A few options have been designed in \BONMIN specifically to treat
problems that do not have a convex continuous relaxation.
In such problems, the solutions obtained from \IPOPT are
not necessarily globally optimal, but are only locally optimal.
Also the outer-approximation constraints are not necessarily valid inequalities for the problem.
No specific heuristic method for treating nonconvex problems is implemented
yet within the OA framework.
But for the pure branch-and-bound {\tt B-BB}, a few options have been implemented while having
in mind that lower bounds provided by \IPOPT should not be trusted and with the goal of
trying to get good solutions. Such options are at a very experimental stage.

First, in the context of nonconvex problems, \IPOPT may find different local optima when started
from different starting points. The two options {\tt num\_re\-solve\_at\_root} and {\tt num\_resolve\_at\_node}
allow for solving the root node or each node of the tree, respectively, with a user-specified
number of different randomly-chosen starting points, saving the best solution found.
Note that the function to generate a random starting point is very na\"{\i}ve:
it chooses a random point (uniformly) between the bounds provided for the variable.
In particular if there are some functions that can not be evaluated at some points of the domain, it may pick such points,
and so it is not robust in that respect.

Secondly, since the solution given by \IPOPT does not truly give a lower bound, the fathoming rule can be changed to continue branching even if the solution value to the current node is worse than the best-known solution.
This is achieved by setting {\tt allowable\_gap}
and {\tt allowable\_fraction\_gap} and {\tt cutoff\_decr} to negative values.

\subsection{\IPOPT options changed by \BONMIN}

\IPOPT has a very large number of options, see Section \ref{sub:ipoptoptions} to get a complete description.
To use \IPOPT more efficiently in the context of MINLP,
\BONMIN changes some \IPOPT options from their default values, which may help to improve \IPOPT's warm-starting capabilities and its ability to prove quickly that a subproblem is infeasible.
These are settings that \IPOPT does not use for ordinary NLP problems.
Note that options set by the user in an option file will override these settings.
\begin{itemize}
\vspace{-2ex}
\setlength{\partopsep}{0pt}
\setlength{\itemsep}{0pt}
\setlength{\parskip}{.5ex}
\item {\tt mu\_strategy} and {\tt mu\_oracle} are set, respectively, to
{\tt adaptive} and {\tt probing} by default. These are strategies in \IPOPT
for updating the barrier parameter. They were found to be more efficient in the context of MINLP.

\item {\tt gamma\_phi} and {\tt gamma\_theta} are set to $10^{-8}$ and $10^{-4}$ respectively.
This has the effect of reducing the size of the filter in the line search performed by \IPOPT.

\item {\tt required\_infeasibility\_reduction} is set to $0.1$.
This increases the required infeasibility reduction when \IPOPT enters the
restoration phase and should thus help to detect infeasible problems faster.

\item {\tt expect\_infeasible\_problem} is set to {\tt yes}, which enables some heuristics
to detect infeasible problems faster.

\item {\tt warm\_start\_init\_point} is set to {\tt yes} when a full primal/dual starting
point is available (generally for all the optimizations after the continuous relaxation has been solved).

\item {\tt print\_level} is set to $0$ by default to turn off \IPOPT output (except for the root node, which print level is controlled by the \BONMIN option \texttt{nlp\_log\_at\_root}).

\item \texttt{bound\_relax\_factor} is set to $10^{-10}$. All of the bounds
of the problem are relaxed by this factor. This may cause some trouble
when constraint functions can only be evaluated within their bounds.
In such cases, this option should be set to $0$.
\end{itemize}

\section{Detailed Options Description}
\label{sub:bonminalloptions}

The following tables give the list of options together with their types, default values, and availability in each of the main algorithms.
% The column labeled `type' indicates the type of the parameter.
%  (`F' stands for float, `I' for integer, and `S' for string).
% The column labeled `default' indicates the global default value.
% Then for each of the algorithms \texttt{B-BB}, \texttt{B-OA}, \texttt{B-QG}, \texttt{B-Hyb}, \texttt{B-Ecp}, and \texttt{B-iFP} `$+$' indicates that the option is available for that particular algorithm, while `$-$' indicates that it is not.
The column labeled `\texttt{Cbc\_Par}' indicates the options that can be used to parametrize the MLIP subsolver in the context of OA and FP.

% \newpage
\input{optbonmin_s}

In the following we give a detailed list of \BONMIN options.
The value on the right denotes the default value.
\input{optbonmin_a}

\bibliographystyle{plain}
%\bibliography{coinlibd}
%\renewcommand{\bibname}{BONMIN References}
\input{bonmin.bbl}

\chapterend


\chapterend


\chapterend


\chapterend

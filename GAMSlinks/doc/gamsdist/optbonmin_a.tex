\printoptioncategory{Algorithm choice}
\printoption{algorithm}%
{\ttfamily B-BB, B-OA, B-QG, B-Hyb, B-Ecp, B-iFP}%
{B-BB}%
{Choice of the algorithm.\\
This will preset some of the options of bonmin depending on the algorithm choice.}%
{\begin{list}{}{
\setlength{\parsep}{0em}
\setlength{\leftmargin}{5ex}
\setlength{\labelwidth}{2ex}
\setlength{\itemindent}{0ex}
\setlength{\topsep}{0pt}}
\item[\texttt{B-BB}] simple branch-and-bound algorithm,
\item[\texttt{B-OA}] OA Decomposition algorithm,
\item[\texttt{B-QG}] Quesada and Grossmann branch-and-cut algorithm,
\item[\texttt{B-Hyb}] hybrid outer approximation based branch-and-cut,
\item[\texttt{B-Ecp}] ecp cuts based branch-and-cut a la FilMINT.
\item[\texttt{B-iFP}] Iterated Feasibility Pump for MINLP.
\end{list}
}

\printoptioncategory{Branch-and-bound options}
\printoption{allowable\_fraction\_gap}%
{$\textrm{real}$}%
{$0.1$}%
{Specify the value of relative gap under which the algorithm stops.\\
Stop the tree search when the gap between the objective value of the best known solution and the best bound on the objective of any solution is less than this fraction of the absolute value of the best known solution value.}%
{}

\printoption{allowable\_gap}%
{$\textrm{real}$}%
{$0$}%
{Specify the value of absolute gap under which the algorithm stops.\\
Stop the tree search when the gap between the objective value of the best known solution and the best bound on the objective of any solution is less than this.}%
{}

\printoption{cutoff}%
{$-10^{ 100}\leq\textrm{real}\leq10^{ 100}$}%
{$10^{ 100}$}%
{Specify cutoff value.\\
cutoff should be the value of a feasible solution known by the user (if any). The algorithm will only look for solutions better than cutoff.}%
{}

\printoption{cutoff\_decr}%
{$-10^{ 10}\leq\textrm{real}\leq10^{ 10}$}%
{$10^{- 5}$}%
{Specify cutoff decrement.\\
Specify the amount by which cutoff is decremented below a new best upper-bound (usually a small positive value but in non-convex problems it may be a negative value).}%
{}

\printoption{enable\_dynamic\_nlp}%
{\ttfamily no, yes}%
{no}%
{Enable dynamic linear and quadratic rows addition in nlp}%
{}

\printoption{integer\_tolerance}%
{$0<\textrm{real}$}%
{$10^{- 6}$}%
{Set integer tolerance.\\
Any number within that value of an integer is considered integer.}%
{}

\printoption{iteration\_limit}%
{$0\leq\textrm{integer}$}%
{$\infty$}%
{Set the cumulated maximum number of iteration in the algorithm used to process nodes continuous relaxations in the branch-and-bound.\\
value 0 deactivates option.}%
{}

\printoption{nlp\_failure\_behavior}%
{\ttfamily stop, fathom}%
{stop}%
{Set the behavior when an NLP or a series of NLP are unsolved by Ipopt (we call unsolved an NLP for which Ipopt is not able to guarantee optimality within the specified tolerances).\\
If set to "fathom", the algorithm will fathom the node when Ipopt fails to find a solution to the nlp at that node whithin the specified tolerances. The algorithm then becomes a heuristic, and the user will be warned that the solution might not be optimal.}%
{\begin{list}{}{
\setlength{\parsep}{0em}
\setlength{\leftmargin}{5ex}
\setlength{\labelwidth}{2ex}
\setlength{\itemindent}{0ex}
\setlength{\topsep}{0pt}}
\item[\texttt{stop}] Stop when failure happens.
\item[\texttt{fathom}] Continue when failure happens.
\end{list}
}

\printoption{node\_comparison}%
{\ttfamily best-bound, depth-first, breadth-first, dynamic, best-guess}%
{best-bound}%
{Choose the node selection strategy.\\
Choose the strategy for selecting the next node to be processed.}%
{\begin{list}{}{
\setlength{\parsep}{0em}
\setlength{\leftmargin}{5ex}
\setlength{\labelwidth}{2ex}
\setlength{\itemindent}{0ex}
\setlength{\topsep}{0pt}}
\item[\texttt{best-bound}] choose node with the smallest bound,
\item[\texttt{depth-first}] Perform depth first search,
\item[\texttt{breadth-first}] Perform breadth first search,
\item[\texttt{dynamic}] Cbc dynamic strategy (starts with a depth first search and turn to best bound after 3 integer feasible solutions have been found).
\item[\texttt{best-guess}] choose node with smallest guessed integer solution
\end{list}
}

\printoption{node\_limit}%
{$0\leq\textrm{integer}$}%
{$\infty$}%
{Set the maximum number of nodes explored in the branch-and-bound search.}%
{}

\printoption{num\_cut\_passes}%
{$0\leq\textrm{integer}$}%
{$1$}%
{Set the maximum number of cut passes at regular nodes of the branch-and-cut.}%
{}

\printoption{num\_cut\_passes\_at\_root}%
{$0\leq\textrm{integer}$}%
{$20$}%
{Set the maximum number of cut passes at regular nodes of the branch-and-cut.}%
{}

\printoption{number\_before\_trust}%
{$0\leq\textrm{integer}$}%
{$8$}%
{Set the number of branches on a variable before its pseudo costs are to be believed in dynamic strong branching.\\
A value of 0 disables pseudo costs.}%
{}

\printoption{number\_strong\_branch}%
{$0\leq\textrm{integer}$}%
{$20$}%
{Choose the maximum number of variables considered for strong branching.\\
Set the number of variables on which to do strong branching.}%
{}

\printoption{read\_solution\_file}%
{\ttfamily no, yes}%
{no}%
{Read a file with the optimal solution to test if algorithms cuts it.\\
For Debugging purposes only.}%
{}

\printoption{solution\_limit}%
{$0\leq\textrm{integer}$}%
{$\infty$}%
{Abort after that much integer feasible solution have been found by algorithm\\
value 0 deactivates option}%
{}

\printoption{time\_limit}%
{$0\leq\textrm{real}$}%
{$1000$}%
{Set the global maximum computation time (in secs) for the algorithm.}%
{}

\printoption{tree\_search\_strategy}%
{\ttfamily top-node, dive, probed-dive, dfs-dive, dfs-dive-dynamic}%
{probed-dive}%
{Pick a strategy for traversing the tree\\
All strategies can be used in conjunction with any of the node comparison functions. Options which affect dfs-dive are max-backtracks-in-dive and max-dive-depth. The dfs-dive won't work in a non-convex problem where objective does not decrease down branches.}%
{\begin{list}{}{
\setlength{\parsep}{0em}
\setlength{\leftmargin}{5ex}
\setlength{\labelwidth}{2ex}
\setlength{\itemindent}{0ex}
\setlength{\topsep}{0pt}}
\item[\texttt{top-node}]  Always pick the top node as sorted by the node comparison function
\item[\texttt{dive}] Dive in the tree if possible, otherwise pick top node as sorted by the tree comparison function.
\item[\texttt{probed-dive}] Dive in the tree exploring two childs before continuing the dive at each level.
\item[\texttt{dfs-dive}] Dive in the tree if possible doing a depth first search. Backtrack on leaves or when a prescribed depth is attained or when estimate of best possible integer feasible solution in subtree is worst than cutoff. Once a prescribed limit of backtracks is attained pick top node as sorted by the tree comparison function
\item[\texttt{dfs-dive-dynamic}] Same as dfs-dive but once enough solution are found switch to best-bound and if too many nodes switch to depth-first.
\end{list}
}

\printoption{variable\_selection}%
{\ttfamily most-fractional, strong-branching, reliability-branching, qp-strong-branching, lp-strong-branching, nlp-strong-branching, osi-simple, osi-strong, random}%
{strong-branching}%
{Chooses variable selection strategy}%
{\begin{list}{}{
\setlength{\parsep}{0em}
\setlength{\leftmargin}{5ex}
\setlength{\labelwidth}{2ex}
\setlength{\itemindent}{0ex}
\setlength{\topsep}{0pt}}
\item[\texttt{most-fractional}] Choose most fractional variable
\item[\texttt{strong-branching}] Perform strong branching
\item[\texttt{reliability-branching}] Use reliability branching
\item[\texttt{qp-strong-branching}] Perform strong branching with QP approximation
\item[\texttt{lp-strong-branching}] Perform strong branching with LP approximation
\item[\texttt{nlp-strong-branching}] Perform strong branching with NLP approximation
\item[\texttt{osi-simple}] Osi method to do simple branching
\item[\texttt{osi-strong}] Osi method to do strong branching
\item[\texttt{random}] Method to choose branching variable randomly
\end{list}
}

\printoptioncategory{ECP cuts generation}
\printoption{ecp\_abs\_tol}%
{$0\leq\textrm{real}$}%
{$10^{- 6}$}%
{Set the absolute termination tolerance for ECP rounds.}%
{}

\printoption{ecp\_max\_rounds}%
{$0\leq\textrm{integer}$}%
{$5$}%
{Set the maximal number of rounds of ECP cuts.}%
{}

\printoption{ecp\_probability\_factor}%
{$\textrm{real}$}%
{$10$}%
{Factor appearing in formula for skipping ECP cuts.\\
Choosing -1 disables the skipping.}%
{}

\printoption{ecp\_rel\_tol}%
{$0\leq\textrm{real}$}%
{$0$}%
{Set the relative termination tolerance for ECP rounds.}%
{}

\printoption{filmint\_ecp\_cuts}%
{$0\leq\textrm{integer}$}%
{$0$}%
{Specify the frequency (in terms of nodes) at which some a la filmint ecp cuts are generated.\\
A frequency of 0 amounts to to never solve the NLP relaxation.}%
{}

\printoptioncategory{Feasibility checker using OA cuts}
\printoption{feas\_check\_cut\_types}%
{\ttfamily outer-approx, Benders}%
{outer-approx}%
{Choose the type of cuts generated when an integer feasible solution is found\\
If it seems too much memory is used should try Benders to use less}%
{\begin{list}{}{
\setlength{\parsep}{0em}
\setlength{\leftmargin}{5ex}
\setlength{\labelwidth}{2ex}
\setlength{\itemindent}{0ex}
\setlength{\topsep}{0pt}}
\item[\texttt{outer-approx}] Generate a set of Outer Approximations cuts.
\item[\texttt{Benders}] Generate a single Benders cut.
\end{list}
}

\printoption{feas\_check\_discard\_policy}%
{\ttfamily detect-cycles, keep-all, treated-as-normal}%
{detect-cycles}%
{How cuts from feasibility checker are discarded\\
Normally to avoid cycle cuts from feasibility checker should not be discarded in the node where they are generated. However Cbc sometimes does it if no care is taken which can lead to an infinite loop in Bonmin (usualy on simple problems). To avoid this one can instruct Cbc to never discard a cut but if we do that for all cuts it can lead to memory problems. The default policy here is to detect cycles and only then impose to Cbc to keep the cut. The two other alternative are to instruct Cbc to keep all cuts or to just ignore the problem and hope for the best}%
{\begin{list}{}{
\setlength{\parsep}{0em}
\setlength{\leftmargin}{5ex}
\setlength{\labelwidth}{2ex}
\setlength{\itemindent}{0ex}
\setlength{\topsep}{0pt}}
\item[\texttt{detect-cycles}] Detect if a cycle occurs and only in this case force not to discard.
\item[\texttt{keep-all}] Force cuts from feasibility checker not to be discarded (memory hungry but sometimes better).
\item[\texttt{treated-as-normal}] Cuts from memory checker can be discarded as any other cuts (code may cycle then)
\end{list}
}

\printoption{generate\_benders\_after\_so\_many\_oa}%
{$0\leq\textrm{integer}$}%
{$5000$}%
{Specify that after so many oa cuts have been generated Benders cuts should be generated instead.\\
It seems that sometimes generating too many oa cuts slows down the optimization compared to Benders due to the size of the LP. With this option we specify that after so many OA cuts have been generated we should switch to Benders cuts.}%
{}

\printoptioncategory{MILP Solver}
\printoption{cpx\_parallel\_strategy}%
{$-1\leq\textrm{integer}\leq1$}%
{$0$}%
{Strategy of parallel search mode in CPLEX.\\
-1 = opportunistic, 0 = automatic, 1 = deterministic (refer to CPLEX documentation)}%
{}

\printoption{milp\_solver}%
{Cbc\_D, Cbc\_Par, Cplex}%
{Cbc\_D}%
{Choose the subsolver to solve MILP sub-problems in OA decompositions.\\
 To use Cplex, a valid license is required.}%
{\begin{list}{}{
\setlength{\parsep}{0em}
\setlength{\leftmargin}{5ex}
\setlength{\labelwidth}{2ex}
\setlength{\itemindent}{0ex}
\setlength{\topsep}{0pt}}
\item[\texttt{Cbc\_D}] Coin Branch and Cut with its default
\item[\texttt{Cbc\_Par}] Coin Branch and Cut with passed parameters
\item[\texttt{Cplex}] IBM CPLEX
\end{list}
}

\printoption{milp\_strategy}%
{\ttfamily find\_good\_sol, solve\_to\_optimality}%
{find\_good\_sol}%
{Choose a strategy for MILPs.}%
{\begin{list}{}{
\setlength{\parsep}{0em}
\setlength{\leftmargin}{5ex}
\setlength{\labelwidth}{2ex}
\setlength{\itemindent}{0ex}
\setlength{\topsep}{0pt}}
\item[\texttt{find\_good\_sol}] Stop sub milps when a solution improving the incumbent is found
\item[\texttt{solve\_to\_optimality}] Solve MILPs to optimality
\end{list}
}

\printoption{number\_cpx\_threads}%
{$0\leq\textrm{integer}$}%
{$0$}%
{Set number of threads to use with cplex.\\
(refer to CPLEX documentation)}%
{}

\printoptioncategory{MILP cutting planes in hybrid algorithm (B-Hyb)}
\printoption{2mir\_cuts}%
{$-100\leq\textrm{integer}$}%
{$0$}%
{Frequency (in terms of nodes) for generating 2-MIR cuts in branch-and-cut\\
If $k > 0$, cuts are generated every k nodes, if $-99 < k < 0$ cuts are generated every $-k$ nodes but Cbc may decide to stop generating cuts, if not enough are generated at the root node, if $k=-99$ generate cuts only at the root node, if $k=0$ or $100$ do not generate cuts.}%
{}

\printoption{Gomory\_cuts}%
{$-100\leq\textrm{integer}$}%
{$-5$}%
{Frequency k (in terms of nodes) for generating Gomory cuts in branch-and-cut.\\
See option \texttt{2mir\_cuts} for the meaning of k.}%
{}

\printoption{clique\_cuts}%
{$-100\leq\textrm{integer}$}%
{$-5$}%
{Frequency (in terms of nodes) for generating clique cuts in branch-and-cut\\
See option \texttt{2mir\_cuts} for the meaning of k.}%
{}

\printoption{cover\_cuts}%
{$-100\leq\textrm{integer}$}%
{$0$}%
{Frequency (in terms of nodes) for generating cover cuts in branch-and-cut\\
See option \texttt{2mir\_cuts} for the meaning of k.}%
{}

\printoption{flow\_cover\_cuts}%
{$-100\leq\textrm{integer}$}%
{$-5$}%
{Frequency (in terms of nodes) for generating flow cover cuts in branch-and-cut\\
See option \texttt{2mir\_cuts} for the meaning of k.}%
{}

\printoption{lift\_and\_project\_cuts}%
{$-100\leq\textrm{integer}$}%
{$0$}%
{Frequency (in terms of nodes) for generating lift-and-project cuts in branch-and-cut\\
See option \texttt{2mir\_cuts} for the meaning of k.}%
{}

\printoption{mir\_cuts}%
{$-100\leq\textrm{integer}$}%
{$-5$}%
{Frequency (in terms of nodes) for generating MIR cuts in branch-and-cut\\
See option \texttt{2mir\_cuts} for the meaning of k.}%
{}

\printoption{reduce\_and\_split\_cuts}%
{$-100\leq\textrm{integer}$}%
{$0$}%
{Frequency (in terms of nodes) for generating reduce-and-split cuts in branch-and-cut\\
See option \texttt{2mir\_cuts} for the meaning of k.}%
{}

\printoptioncategory{MINLP Heuristics}
\printoption{feasibility\_pump\_objective\_norm}%
{$1\leq\textrm{integer}\leq2$}%
{$1$}%
{Norm of feasibility pump objective function}%
{}

\printoption{fp\_pass\_infeasible}%
{\ttfamily no, yes}%
{no}%
{Say whether feasibility pump should claim to converge or not}%
{\begin{list}{}{
\setlength{\parsep}{0em}
\setlength{\leftmargin}{5ex}
\setlength{\labelwidth}{2ex}
\setlength{\itemindent}{0ex}
\setlength{\topsep}{0pt}}
\item[\texttt{no}] When master MILP is infeasible just bail out (don't stop all algorithm). This is the option for using in B-Hyb.
\item[\texttt{yes}] Claim convergence, numerically dangerous.
\end{list}
}

\printoption{heuristic\_RINS}%
{\ttfamily no, yes}%
{no}%
{if yes runs the RINS heuristic}%
{
}

\printoption{heuristic\_dive\_MIP\_fractional}%
{\ttfamily no, yes}%
{no}%
{if yes runs the Dive MIP Fractional heuristic}%
{
}

\printoption{heuristic\_dive\_MIP\_vectorLength}%
{\ttfamily no, yes}%
{no}%
{if yes runs the Dive MIP VectorLength heuristic}%
{
}

\printoption{heuristic\_dive\_fractional}%
{\ttfamily no, yes}%
{no}%
{if yes runs the Dive Fractional heuristic}%
{
}

\printoption{heuristic\_dive\_vectorLength}%
{\ttfamily no, yes}%
{no}%
{if yes runs the Dive VectorLength heuristic}%
{
}

\printoption{heuristic\_feasibility\_pump}%
{\ttfamily no, yes}%
{no}%
{whether the heuristic feasibility pump should be used}%
{
}

\printoption{pump\_for\_minlp}%
{\ttfamily no, yes}%
{no}%
{if yes runs FP for MINLP}%
{
}

\printoptioncategory{NLP interface}
\printoption{warm\_start}%
{\ttfamily none, fake\_basis, optimum, interior\_point}%
{none}%
{Select the warm start method\\
This will affect the function getWarmStart(), and as a consequence the warm starting in the various algorithms.}%
{\begin{list}{}{
\setlength{\parsep}{0em}
\setlength{\leftmargin}{5ex}
\setlength{\labelwidth}{2ex}
\setlength{\itemindent}{0ex}
\setlength{\topsep}{0pt}}
\item[\texttt{none}] No warm start, just start NLPs from optimal solution of the root relaxation
\item[\texttt{fake\_basis}] builds fake basis, useful for cut management in Cbc (warm start is the same as in none)
\item[\texttt{optimum}] Warm start with direct parent optimum
\item[\texttt{interior\_point}] Warm start with an interior point of direct parent
\end{list}
}

\printoptioncategory{NLP solution robustness}
\printoption{max\_consecutive\_failures}%
{$0\leq\textrm{integer}$}%
{$10$}%
{(temporarily removed) Number $n$ of consecutive unsolved problems before aborting a branch of the tree.\\
When $n > 0$, continue exploring a branch of the tree until $n$ consecutive problems in the branch are unsolved (we call unsolved a problem for which Ipopt can not guarantee optimality within the specified tolerances).}%
{}

\printoption{max\_random\_point\_radius}%
{$0<\textrm{real}$}%
{$100000$}%
{Set max value r for coordinate of a random point.\\
When picking a random point, coordinate i will be in the interval [min(max(l,-r),u-r), max(min(u,r),l+r)] (where l is the lower bound for the variable and u is its upper bound)}%
{}

\printoption{num\_iterations\_suspect}%
{$-1\leq\textrm{integer}$}%
{$-1$}%
{Number of iterations over which a node is considered "suspect" (for debugging purposes only, see detailed documentation).\\
When the number of iterations to solve a node is above this number, the subproblem at this node is considered to be suspect and it will be outputed in a file (set to -1 to deactivate this).}%
{}

\printoption{num\_retry\_unsolved\_random\_point}%
{$0\leq\textrm{integer}$}%
{$0$}%
{Number $k$ of times that the algorithm will try to resolve an unsolved NLP with a random starting point (we call unsolved an NLP for which Ipopt is not able to guarantee optimality within the specified tolerances).\\
When Ipopt fails to solve a continuous NLP sub-problem, if $k > 0$, the algorithm will try again to solve the failed NLP with $k$ new randomly chosen starting points  or until the problem is solved with success.}%
{}

\printoption{random\_point\_perturbation\_interval}%
{$0<\textrm{real}$}%
{$1$}%
{Amount by which starting point is perturbed when choosing to pick random point by perturbating starting point}%
{}

\printoption{random\_point\_type}%
{\ttfamily Jon, Andreas, Claudia}%
{Jon}%
{method to choose a random starting point}%
{\begin{list}{}{
\setlength{\parsep}{0em}
\setlength{\leftmargin}{5ex}
\setlength{\labelwidth}{2ex}
\setlength{\itemindent}{0ex}
\setlength{\topsep}{0pt}}
\item[\texttt{Jon}] Choose random point uniformly between the bounds
\item[\texttt{Andreas}] perturb the starting point of the problem within a prescribed interval
\item[\texttt{Claudia}] perturb the starting point using the perturbation radius suffix information
\end{list}
}

\printoption{resolve\_on\_small\_infeasibility}%
{$0\leq\textrm{real}$}%
{$0$}%
{If a locally infeasible problem is infeasible by less than this, resolve it with initial starting point.}%
{}

\printoptioncategory{NLP solves in hybrid algorithm (B-Hyb)}
\printoption{nlp\_solve\_frequency}%
{$0\leq\textrm{integer}$}%
{$10$}%
{Specify the frequency (in terms of nodes) at which NLP relaxations are solved in B-Hyb.\\
A frequency of 0 amounts to to never solve the NLP relaxation.}%
{}

\printoption{nlp\_solve\_max\_depth}%
{$0\leq\textrm{integer}$}%
{$10$}%
{Set maximum depth in the tree at which NLP relaxations are solved in B-Hyb.\\
A depth of 0 amounts to to never solve the NLP relaxation.}%
{}

\printoption{nlp\_solves\_per\_depth}%
{$0\leq\textrm{real}$}%
{$10^{ 100}$}%
{Set average number of nodes in the tree at which NLP relaxations are solved in B-Hyb for each depth.}%
{}

\printoptioncategory{Nonconvex problems}
\printoption{coeff\_var\_threshold}%
{$0\leq\textrm{real}$}%
{$0.1$}%
{Coefficient of variation threshold (for dynamic definition of cutoff\_decr).}%
{}

\printoption{dynamic\_def\_cutoff\_decr}%
{\ttfamily no, yes}%
{no}%
{Do you want to define the parameter cutoff\_decr dynamically?}%
{
}

\printoption{first\_perc\_for\_cutoff\_decr}%
{$\textrm{real}$}%
{$-0.02$}%
{The percentage used when, the coeff of variance is smaller than the threshold, to compute the cutoff\_decr dynamically.}%
{}

\printoption{max\_consecutive\_infeasible}%
{$0\leq\textrm{integer}$}%
{$0$}%
{Number of consecutive infeasible subproblems before aborting a branch.\\
Will continue exploring a branch of the tree until "max\_consecutive\_infeasible"consecutive problems are infeasibles by the NLP sub-solver.}%
{}

\printoption{num\_resolve\_at\_infeasibles}%
{$0\leq\textrm{integer}$}%
{$0$}%
{Number $k$ of tries to resolve an infeasible node (other than the root) of the tree with different starting point.\\
The algorithm will solve all the infeasible nodes with $k$ different random starting points and will keep the best local optimum found.}%
{}

\printoption{num\_resolve\_at\_node}%
{$0\leq\textrm{integer}$}%
{$0$}%
{Number $k$ of tries to resolve a node (other than the root) of the tree with different starting point.\\
The algorithm will solve all the nodes with $k$ different random starting points and will keep the best local optimum found.}%
{}

\printoption{num\_resolve\_at\_root}%
{$0\leq\textrm{integer}$}%
{$0$}%
{Number $k$ of tries to resolve the root node with different starting points.\\
The algorithm will solve the root node with $k$ random starting points and will keep the best local optimum found.}%
{}

\printoption{second\_perc\_for\_cutoff\_decr}%
{$\textrm{real}$}%
{$-0.05$}%
{The percentage used when, the coeff of variance is greater than the threshold, to compute the cutoff\_decr dynamically.}%
{}

\printoptioncategory{Outer Approximation Decomposition (B-OA)}
\printoption{oa\_decomposition}%
{\ttfamily no, yes}%
{no}%
{If yes do initial OA decomposition}%
{}

\printoptioncategory{Outer Approximation cuts generation}
\printoption{add\_only\_violated\_oa}%
{\ttfamily no, yes}%
{no}%
{Do we add all OA cuts or only the ones violated by current point?}%
{\begin{list}{}{
\setlength{\parsep}{0em}
\setlength{\leftmargin}{5ex}
\setlength{\labelwidth}{2ex}
\setlength{\itemindent}{0ex}
\setlength{\topsep}{0pt}}
\item[\texttt{no}] Add all cuts
\item[\texttt{yes}] Add only violated Cuts
\end{list}
}

\printoption{oa\_cuts\_scope}%
{\ttfamily local, global}%
{global}%
{Specify if OA cuts added are to be set globally or locally valid}%
{\begin{list}{}{
\setlength{\parsep}{0em}
\setlength{\leftmargin}{5ex}
\setlength{\labelwidth}{2ex}
\setlength{\itemindent}{0ex}
\setlength{\topsep}{0pt}}
\item[\texttt{local}] Cuts are treated as locally valid
\item[\texttt{global}] Cuts are treated as globally valid
\end{list}
}

\printoption{tiny\_element}%
{$-0\leq\textrm{real}$}%
{$10^{- 8}$}%
{Value for tiny element in OA cut\\
We will remove "cleanly" (by relaxing cut) an element lower than this.}%
{}

\printoption{very\_tiny\_element}%
{$-0\leq\textrm{real}$}%
{$10^{-17}$}%
{Value for very tiny element in OA cut\\
Algorithm will take the risk of neglecting an element lower than this.}%
{}

\printoptioncategory{Output}
\printoption{bb\_log\_interval}%
{$0\leq\textrm{integer}$}%
{$100$}%
{Interval at which node level output is printed.\\
Set the interval (in terms of number of nodes) at which a log on node resolutions (consisting of lower and upper bounds) is given.}%
{}

\printoption{bb\_log\_level}%
{$0\leq\textrm{integer}\leq5$}%
{$1$}%
{specify main branch-and-bound log level.\\
Set the level of output of the branch-and-bound : 0 - none, 1 - minimal, 2 - normal low, 3 - normal high}%
{}

\printoption{fp\_log\_frequency}%
{$0<\textrm{real}$}%
{$100$}%
{display an update on lower and upper bounds in FP every n seconds}%
{}

\printoption{fp\_log\_level}%
{$0\leq\textrm{integer}\leq2$}%
{$1$}%
{specify FP iterations log level.\\
Set the level of output of OA decomposition solver : 0 - none, 1 - normal, 2 - verbose}%
{}

\printoption{lp\_log\_level}%
{$0\leq\textrm{integer}\leq4$}%
{$0$}%
{specify LP log level.\\
Set the level of output of the linear programming sub-solver in B-Hyb or B-QG : 0 - none, 1 - minimal, 2 - normal low, 3 - normal high, 4 - verbose}%
{}

\printoption{milp\_log\_level}%
{$0\leq\textrm{integer}\leq4$}%
{$0$}%
{specify MILP solver log level.\\
Set the level of output of the MILP subsolver in OA : 0 - none, 1 - minimal, 2 - normal low, 3 - normal high}%
{}

\printoption{nlp\_log\_at\_root}%
{$0\leq\textrm{integer}\leq12$}%
{$5$}%
{ Specify a different log level for root relaxtion.}%
{}

\printoption{nlp\_log\_level}%
{$0\leq\textrm{integer}\leq2$}%
{$1$}%
{specify NLP solver interface log level (independent from ipopt print\_level).\\
Set the level of output of the OsiTMINLPInterface : 0 - none, 1 - normal, 2 - verbose}%
{}

\printoption{oa\_cuts\_log\_level}%
{$0\leq\textrm{integer}$}%
{$0$}%
{level of log when generating OA cuts.\\
0: outputs nothing,\\1: when a cut is generated, its violation and index of row from which it originates,\\2: always output violation of the cut.\\3: output generated cuts incidence vectors.}%
{}

\printoption{oa\_log\_frequency}%
{$0<\textrm{real}$}%
{$100$}%
{display an update on lower and upper bounds in OA every n seconds}%
{}

\printoption{oa\_log\_level}%
{$0\leq\textrm{integer}\leq2$}%
{$1$}%
{specify OA iterations log level.\\
Set the level of output of OA decomposition solver : 0 - none, 1 - normal, 2 - verbose}%
{}

\printoptioncategory{Strong branching setup}
\printoption{candidate\_sort\_criterion}%
{\ttfamily best-ps-cost, worst-ps-cost, most-fractional, least-fractional}%
{best-ps-cost}%
{Choice of the criterion to choose candidates in strong-branching}%
{\begin{list}{}{
\setlength{\parsep}{0em}
\setlength{\leftmargin}{5ex}
\setlength{\labelwidth}{2ex}
\setlength{\itemindent}{0ex}
\setlength{\topsep}{0pt}}
\item[\texttt{best-ps-cost}] Sort by decreasing pseudo-cost
\item[\texttt{worst-ps-cost}] Sort by increasing pseudo-cost
\item[\texttt{most-fractional}] Sort by decreasing integer infeasibility
\item[\texttt{least-fractional}] Sort by increasing integer infeasibility
\end{list}
}

\printoption{maxmin\_crit\_have\_sol}%
{$0\leq\textrm{real}\leq1$}%
{$0.1$}%
{Weight towards minimum in of lower and upper branching estimates when a solution has been found.}%
{}

\printoption{maxmin\_crit\_no\_sol}%
{$0\leq\textrm{real}\leq1$}%
{$0.7$}%
{Weight towards minimum in of lower and upper branching estimates when no solution has been found yet.}%
{}

\printoption{min\_number\_strong\_branch}%
{$0\leq\textrm{integer}$}%
{$0$}%
{Sets minimum number of variables for strong branching (overriding trust)}%
{}

\printoption{number\_before\_trust\_list}%
{$-1\leq\textrm{integer}$}%
{$0$}%
{Set the number of branches on a variable before its pseudo costs are to be believed during setup of strong branching candidate list.\\
The default value is that of "number\_before\_trust"}%
{}

\printoption{number\_look\_ahead}%
{$0\leq\textrm{integer}$}%
{$0$}%
{Sets limit of look-ahead strong-branching trials}%
{}

\printoption{number\_strong\_branch\_root}%
{$0\leq\textrm{integer}$}%
{$\infty$}%
{Maximum number of variables considered for strong branching in root node.}%
{}

\printoption{setup\_pseudo\_frac}%
{$0\leq\textrm{real}\leq1$}%
{$0.5$}%
{Proportion of strong branching list that has to be taken from most-integer-infeasible list.}%
{}

\printoption{trust\_strong\_branching\_for\_pseudo\_cost}%
{\ttfamily no, yes}%
{yes}%
{Whether or not to trust strong branching results for updating pseudo costs.}%
{}

